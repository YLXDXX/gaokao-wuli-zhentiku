\gaokaoheader{2020}{全国\lmd{2}卷}



\gaokaoxz


\begin{enumerate}
\item
管道高频焊机可以对由钢板卷成的圆管的接缝实施焊接。焊机的原理如图所示,圆管通过一个接有高
频交流电源的线圈,线圈所产生的交变磁场使圆管中产生交变电流,电流产生的热量使接缝处的材料
熔化将其焊接。焊接过程中所利用的电磁学规律的发现者为 \xzanswer{D} 
\begin{figure}[h!]
\centering
\includesvg[width=0.33\linewidth]{picture/svg/GZ-3-tiyou-0619}
\end{figure}


\fourchoices
{库仑}
{霍尔}
{洛伦兹}
{法拉第}




\item
若一均匀球形星体的密度为$ \rho $,引力常量为 $ G $,则在该星体表面附近沿圆轨道绕其运动的卫星的周期是 \xzanswer{A} 

\fourchoices
{$\sqrt{\frac{3 \pi}{G \rho}}$}
{$ \sqrt{\frac{4 \pi}{G \rho}}$}
{$ \sqrt{\frac{1}{3 \pi G \rho}}$}
{$\sqrt{\frac{1}{4 \pi G \rho}}$}



\item
如图,在摩托车越野赛途中的水平路段前方有一个坑,该坑沿摩托车前进方向的水平宽度为 $ 3h $,其左
边缘 $ a $ 点比右边缘 $ b $ 点高 $ 0.5h $。若摩托车经过 $ a $ 点时的动能为 $ E_{1} $,它会落到坑内 $ c $ 点。$ c $ 与 $ a $ 的水平距
离和高度差均为 $ h $;若经过 $ a $ 点时的动能为 $ E_{2} $,该摩托车恰能越过坑到达 $ b $ 点。
$\frac{E_{2}}{E_{1}}$
等于 \xzanswer{B} 
\begin{figure}[h!]
\centering
\includesvg[width=0.43\linewidth]{picture/svg/GZ-3-tiyou-0620}
\end{figure}


\fourchoices
{$ 20 $}
{$ 18 $}
{$ 9.0 $}
{$ 3.0 $}




\item
$ CT $ 扫描是计算机 $ X $ 射线断层扫描技术的简称,$ CT $ 扫描机可用于对多种病情的探测。图 \subref{2020:全国2:4a} 是某种 $ C $
$ T $ 机主要部分的剖面图,其中 $ X $ 射线产生部分的示意图如图 \subref{2020:全国2:4b} 所示。图 \subref{2020:全国2:4b} 中 $ M $、$ N $ 之间有一电
子束的加速电场,虚线框内有匀强偏转磁场;经调节后电子束从静止开始沿带箭头的实线所示的方向
前进,打到靶上,产生 $ X $ 射线(如图中带箭头的虚线所示);将电子束打到靶上的点记为 $ P $ 点。则 \xzanswer{D} 
\begin{figure}[h!]
\centering
\begin{subfigure}{0.45\linewidth}
\centering
\includesvg[width=0.9\linewidth]{picture/svg/GZ-3-tiyou-0621} 
\caption{}\label{2020:全国2:4a}
\end{subfigure}
\hfil
\begin{subfigure}{0.45\linewidth}
\centering
\includesvg[width=0.9\linewidth]{picture/svg/GZ-3-tiyou-0622} 
\caption{}\label{2020:全国2:4b}
\end{subfigure}

\end{figure}


\fourchoices
{$ M $ 处的电势高于 $ N $ 处的电势}
{增大 $ M $、$ N $ 之间的加速电压可使 $ P $ 点左移}
{偏转磁场的方向垂直于纸面向外}
{增大偏转磁场磁感应强度的大小可使 $ P $ 点左移}



\item
氘核 \ce{^{2}_{1}H} 可通过一系列聚变反应释放能量,其总效果可用反应式
\[ \ce{6^{2}_{1}H \rightarrow {2}_{4}^{2}He + 2^{1}_{0}n + 43.15MeV} \]
表示。海水中富含氘,已知 $ 1 \ kg $ 海水中含有的氘核约为 $ 1.0 \times 10^{22} $ 个,若全都发生聚变反应,其释放的能
量与质量为 $ M $ 的标准煤燃烧时释放的热量相等;已知 $ 1 \ kg $ 标准煤燃烧释放的热量约为 $ 2.9 \times 10^{7} \ J $,$ 1 \ MeV=1.6 \times 10^{-13} \ J $,则 $ M $ 约为 \xzanswer{C} 

\fourchoices
{$ 40 \ kg $}
{$ 100 \ kg $}
{$ 400 \ kg $}
{$ 1000 \ kg $}


\item
特高压输电可使输送中的电能损耗和电压损失大幅降低。我国已成功掌握并实际应用了特高压输电技术。假设从$ A $处采用$ 550 \ kV $的超高压向$ B $处输电,输电线上损耗的电功率为$ \Delta P $,到达$ B $处时电压下降了
$ \Delta U $。在保持$ A $处输送的电功率和输电线电阻都不变的条件下,改用$ 1100 \ kV $特高压输电,输电线上损耗
的电功率变为$ \Delta P ^{\prime} $,到达$ B $处时电压下降了$ \Delta U ^{\prime} $。不考虑其他因素的影响,则 \xzanswer{AD} 

\fourchoices
{$\Delta P^{\prime}=\frac{1}{4} \Delta P$}
{$\Delta P^{\prime}=\frac{1}{2} \Delta P$}
{$\Delta U^{\prime}=\frac{1}{4} \Delta U$}
{$\Delta U^{\prime}=\frac{1}{2} \Delta U$}



\item
如图,竖直面内一绝缘细圆环的上、下半圆分别均匀分布着等量异种电荷。$ a $、$ b $为圆环水平直径上的
两个点,$ c $、$ d $为竖直直径上的两个点,它们与圆心的距离均相等。则 \xzanswer{ABC} 
\begin{figure}[h!]
\centering
\includesvg[width=0.23\linewidth]{picture/svg/GZ-3-tiyou-0623}
\end{figure}


\fourchoices
{$ a $、$ b $两点的场强相等}
{$ a $、$ b $两点的电势相等}
{$ c $、$ d $两点的场强相等}
{$ c $、$ d $两点的电势相等}


\item
水平冰面上有一固定的竖直挡板,一滑冰运动员面对挡板静止在冰面上,他把一质量为$ 4.0 \ kg $的静止物
块以大小为$ 5.0 \ m/s $的速度沿与挡板垂直的方向推向挡板,运动员获得退行速度;物块与挡板弹性碰撞,
速度反向,追上运动员时,运动员又把物块推向挡板,使其再一次以大小为$ 5.0 \ m/s $的速度与挡板弹性
碰撞。总共经过$ 8 $次这样推物块后,运动员退行速度的大小大于$ 5.0 \ m/s $,反弹的物块不能再追上运动员。
不计冰面的摩擦力,该运动员的质量可能为 \xzanswer{BC} 

\fourchoices
{$ 48 \ kg $}
{$ 53 \ kg $}
{$ 58 \ kg $}
{$ 63 \ kg $}



\gaokaosy


\item 
一细绳跨过悬挂的定滑轮,两端分别系有小球 $ A $ 和 $ B $,如图所示。一实验小组用此装置测量小球 $ B $ 运
动的加速度。
\begin{figure}[h!]
\centering
\includesvg[width=0.14\linewidth]{picture/svg/GZ-3-tiyou-0624}
\end{figure}

令两小球静止,细绳拉紧,然后释放小球,测得小球 $ B $ 释放时的高度 $ h_{0} =0.590 \ m $,下降一段距离后的
高度 $ h=0.100 \ m $;由 $ h_{0} $ 下降至 $ h $ 所用的时间 $ T=0.730 \ s $。由此求得小球 $ B $ 加速度的大小为 $ a=$ \underlinegap $m/s^{2} $(保
留 $ 3 $ 位有效数字)。

从实验室提供的数据得知,小球 $ A $、$ B $ 的质量分别为 $ 100.0 \ g $ 和 $ 150.0 \ g $,当地重力加速度大小为 $ g=9.80 \ m/s^{2} $。根据牛顿第二定律计算可得小球 $ B $ 加速度的大小为 $ a ^{\prime}= $ \underlinegap $m/s^{2} $(保留 $ 3 $ 位有效数字)。

可以看出,$ a ^{\prime} $与 $ a $ 有明显差异,除实验中的偶然误差外,写出一条可能产生这一结果的原因: \hfullline 

\hfullline 。

\tk{1.84 \quad 1.96 \quad 滑轮的轴不光滑(或滑轮有质量)} 



\newpage
\item
某同学要研究一小灯泡 $ L $($ 3.6 \ V $,$ 0.30 \ A $)的伏安特性。所用器材有:电流表 $ A_{1} $(量程 $ 200 \ mA $,内
阻 $ R_{g1} =10.0 \ \Omega $),电流表 $ A_{2} $(量程 $ 500 \ mA $,内阻 $ R_{g2} =1.0 \ \Omega $)、定值电阻 $ R_{0} $(阻值 $ R_{0} =10.0 \ \Omega $)、滑动变
阻器 $ R_{1} $(最大阻值 $ 10 \ \Omega $)、电源 $ E $(电动势 $ 4.5 \ V $,内阻很小)、开关 $ S $ 和若干导线。该同学设计的电路
如图 \subref{2020:全国2:10a} 所示。


\begin{enumerate}
\item
根据图 \subref{2020:全国2:10a} ,在图 \subref{2020:全国2:10b} 的实物图中画出连线。
\begin{figure}[h!]
\centering
\begin{subfigure}{0.4\linewidth}
\centering
\includesvg[width=0.8\linewidth]{picture/svg/GZ-3-tiyou-0625} 
\caption{}\label{2020:全国2:10a}
\end{subfigure}
\hfil
\begin{subfigure}{0.4\linewidth}
\centering
\includesvg[width=0.8\linewidth]{picture/svg/GZ-3-tiyou-0626} 
\caption{}\label{2020:全国2:10b}
\end{subfigure}
\end{figure}

\item 
若 $ I_{1} $、$ I_{2} $ 分别为流过电流表 $ A_{1} $ 和 $ A_{2} $ 的电流,利用 $ I_{1} $、$ I_{2} $、$ R_{g1} $ 和 $ R_{0} $ 写出:小灯泡两端的电压 $ U= $ \underlinegap 
,流过小灯泡的电流 $ I= $ \underlinegap 。为保证小灯泡的安全,$ I_{1} $ 不能超过 \underlinegap $ mA $。

\item 
实验时,调节滑动变阻器,使开关闭合后两电流表的示数为零。逐次改变滑动变阻器滑片位置并
读取相应的 $ I_{1} $ 和 $ I_{2} $。所得实验数据在下表中给出。
\begin{table}[h!]
\centering 
\begin{tabular}{|l|l|l|l|l|l|l|}
\hline$I_{1} / \mathrm{mA}$ & 32 & 55 & 85 & 125 & 144 & 173 \\
\hline$I_{2} / \mathrm{mA}$ & 171 & 229 & 299 & 379 & 424 & 470 \\
\hline
\end{tabular}
\end{table} 





根据实验数据可算得,当 $ I_{1} =173 \ mA $ 时,灯丝电阻 $ R=$ \underlinegap $ \Omega $(保留 $ 1 $ 位小数)。


\item 
如果用另一个电阻替代定值电阻 $ R_{0} $,其他不变,为了能够测量完整的伏安特性曲线,所用电阻的
阻值不能小于 \underlinegap $ \Omega $(保留 $ 1 $ 位小数)。

\end{enumerate}

\tk{
\begin{enumerate}
\item
如图所示:
\begin{center}
\includesvg[width=0.53\linewidth]{picture/svg/GZ-3-tiyou-0627} 
\end{center}	
\item 	
$I_{1}\left(R_{g1}+R_{0}\right) \quad I_{2}-I_{1} \quad 180$
\item 
$ 11.6 $
\item 
$ 8.0 $
\end{enumerate}
} 



\newpage

\gaokaojs

\item
如图,在 $ 0 \leq x \leq h $, $ - \infty <y<+ \infty$ 区域中存在方向垂直于纸面的匀强磁场,磁感应强度 $ B $ 的大小可调,方
向不变。一质量为 $ m $,电荷量为 $ q $($ q>0 $)的粒子以速度 $ v_{0} $ 从磁场区域左侧沿 $ x $ 轴进入磁场,不计重力。
\begin{enumerate}
\item
若粒子经磁场偏转后穿过 $ y $ 轴正半轴离开磁场,分析说明磁场的方向,并求在这种情况下磁感应
强度的最小值 $ B_{m} $;
\item 
如果磁感应强度大小为
$\frac{B_{m}}{2}$,粒子将通过虚线所示边界上的一点离开磁场。求粒子在该点的运动
方向与 $ x $ 轴正方向的夹角及该点到 $ x $ 轴的距离。

\end{enumerate}
\begin{figure}[h!]
\flushright
\includesvg[width=0.27\linewidth]{picture/svg/GZ-3-tiyou-0628}
\end{figure}

\banswer{
\begin{enumerate}
\item
$B_{\mathrm{m}}=\frac{m v_{0}}{q h}$	
\item 	
粒子会穿过图中 $P$ 点离开磁场, 运动轨迹如图所示。
\begin{center}
\includesvg[width=0.43\linewidth]{picture/svg/GZ-3-tiyou-0629} 
\end{center}
设粒子在 $P$ 点的运动方向与 $x$ 轴正方向的夹角为 $\alpha$,易得$\alpha=\frac{\pi}{6}$。	\\
$P$ 点与 $x$ 轴的距离为$y=(2-\sqrt{3}) h$。
\end{enumerate}
}




\newpage
\item
如图,一竖直圆管质量为 $ M $,下端距水平地面的高度为 $ H $,顶端塞有一质量为 $ m $ 的小球。圆管由
静止自由下落,与地面发生多次弹性碰撞,且每次碰撞时间均极短;在运动过程中,管始终保持竖直。
已知 $ M=4m $,球和管之间的滑动摩擦力大小为 $ 4mg,g $ 为重力加速度的大小,不计空气阻力。
\begin{enumerate}
\item
求管第一次与地面碰撞后的瞬间,管和球各自的加速度大小;
\item 
管第一次落地弹起后,在上升过程中球没有从管中滑出,求管上升的最大高度;
\item 
管第二次落地弹起的上升过程中,球仍没有从管中滑出,求圆管长度应满足的条件。

\end{enumerate}
\begin{figure}[h!]
\flushright
\includesvg[width=0.2\linewidth]{picture/svg/GZ-3-tiyou-0630}
\end{figure}

\banswer{
\begin{enumerate}
\item
设此时管的加速度大小为$ a_{1} $,方向向下;球的加速度大小为$ a_{2} $,方向向上;$a_{1}=2 g, \quad a_{2}=3 g$。		
\item 
$H_{max}=\frac{13}{25} H$		
\item 
$L \geq \frac{152}{125} H$		
\end{enumerate}
}



\newpage

\gaokaoxx{$ 3 - 3 $}


\item
\begin{enumerate}
\item
下列关于能量转换过程的叙述,违背热力学第一定律的有 \underlinegap ,不违背热力学第一定
律、但违背热力学第二定律的有 \underlinegap 。(填正确答案标号)

\fourchoices
{汽车通过燃烧汽油获得动力并向空气中散热}
{冷水倒入保温杯后,冷水和杯子的温度都变得更低}
{某新型热机工作时将从高温热源吸收的热量全部转化为功,而不产生其他影响}
{冰箱的制冷机工作时从箱内低温环境中提取热量散发到温度较高的室内}

\tk{B \quad C} 



\item 
潜水钟是一种水下救生设备,它是一个底部开口、上部封闭的容器,外形与钟相似。潜
水钟在水下时其内部上方空间里存有空气,以满足潜水员水下避险的需要。为计算方便,将潜水钟简化为
截面积为 $ S $、高度为 $ h $、开口向下的圆筒;工作母船将潜水钟由水面上方开口向下吊放至深度为 $ H $ 的水下,
如图所示。已知水的密度为$ \rho $,重力加速度大小为 $ g $,大气压强为 $ p_{0} $,$H \gg h$,忽略温度的变化和水密度随深度的变化。
\begin{enumerate}
\item
求进入圆筒内水的高度 $ l $;
\item 
保持 $ H $ 不变,压入空气使筒内的水全部排出,求压入的空气在其压强为 $ p_{0} $ 时的体积。
\end{enumerate}
\begin{figure}[h!]
\flushright
\includesvg[width=0.25\linewidth]{picture/svg/GZ-3-tiyou-0631}
\end{figure}

\banswer{
\begin{enumerate}
\item
考虑到$ H \gg h>l $,将$ l^{2} $当作二阶小量略去,解得$l=\frac{\rho g H}{p_{0}+\rho g (H+h)} h$,正式答案为$l=\frac{\rho g H}{p_{0}+\rho g H} h$。
\item 	
解得$V=\left( 1+\frac{\rho g h}{p_{0}} \right) S H$,正式答案为$V=\frac{\rho g S H h}{p_{0}}$
\end{enumerate}
}






\end{enumerate}





\newpage

\gaokaoxx{$ 3 - 4 $}


\item 
\begin{enumerate}
\item
用一个摆长为 $ 80.0 \ cm $ 的单摆做实验,要求摆动的最大角度小于 $ 5 ^{ \circ } $,则开始时将摆球拉离平衡位置的距离应不超过 \underlinegap $ cm $(保留 $ 1 $ 位小数)。(提示:单摆被拉开小角度的情况下,所求的距
离约等于摆球沿圆弧移动的路程。)

某同学想设计一个新单摆,要求新单摆摆动 $ 10 $ 个周期的时间与原单摆摆动 $ 11 $ 个周期的时间相等。新
单摆的摆长应该取为 \underlinegap $ cm $。

\tk{$ 6.9 \quad 96.8 $} 





\item 
直角棱镜的折射率 $ n=1.5 $,其横截面如图所示,图中$ \angle C=90 ^{ \circ } $,$ \angle A=30 ^{ \circ } $。截面内一细束
与 $ BC $ 边平行的光线,从棱镜 $ AB $ 边上的 $ D $ 点射入,经折射后射到 $ BC $ 边上。
\begin{enumerate}
\item
光线在 $ BC $ 边上是否会发生全反射?说明理由;
\item 
不考虑多次反射,求从 $ AC $ 边射出的光线与最初的入射光线夹角的正弦值。
\end{enumerate}
\begin{figure}[h!]
\flushright
\includesvg[width=0.22\linewidth]{picture/svg/GZ-3-tiyou-0632}
\end{figure}

\banswer{
\begin{enumerate}
\item
如图,设光线在$ D $点的入射角为$ i $,折射角为$ r $。折射光线射到$ BC $边上的$ E $点。设光线在$ E $点的入射角为$ \theta $,由几何关系,有	$\theta=90^{\circ}-\left(30^{\circ}-r\right)>60^{\circ}$。	
\begin{center}
\includesvg[width=0.45\linewidth]{picture/svg/GZ-3-tiyou-0633}
\end{center}
根据题给数据得	$\sin \theta>\sin 60^{\circ}>\frac{1}{n}$,即 $\theta$ 大于全反射临界角,因此光线在 $E$ 点发生全反射。	
\item 
$\sin r^{\prime}=\frac{2 \sqrt{2}-\sqrt{3}}{4}$			
\end{enumerate}
}





\end{enumerate}



\end{enumerate}

