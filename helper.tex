\documentclass{ctexbook}

%请使用laulatex编译该文档

\usepackage{graphicx}
\usepackage{amsmath}
\usepackage{enumerate}
\usepackage{tikz}
\usepackage{hyperref}%超链接
\hypersetup{
	hidelinks,
	colorlinks=true,
	linkcolor=red,
}
\usepackage{fontawesome}%图标(以字体的形式呈现)

\usetikzlibrary{mindmap,shadows}
\usepackage[top=2.5cm,bottom=2cm,left=2.5cm,right=2cm]{geometry}
\usepackage{listings}

\lstset{
	columns=fixed,       
	numbers=left,                                        % 在左侧显示行号
	numberstyle=\tiny\color{gray},                       % 设定行号格式
	frame=none,                                          % 不显示背景边框
	backgroundcolor=\color[RGB]{245,245,244},            % 设定背景颜色
	keywordstyle=\color[RGB]{40,40,255},                 % 设定关键字颜色
	numberstyle=\footnotesize\color{darkgray},           
	commentstyle=\it\color[RGB]{0,96,96},                % 设置代码注释的格式
	stringstyle=\rmfamily\slshape\color[RGB]{128,0,0},   % 设置字符串格式
	showstringspaces=false,                              % 不显示字符串中的空格
	language=Tex,                                        % 设置语言
}

\newcommand{\helpreffist}[1]{
\text{#1 \quad \hyperlink{#1}{\faForward}}
}

\newcommand{\helpreflast}[1]{
\hypertarget{#1}{#1}
}

\ctexset{
	section ={
		format = \raggedright \bfseries \zihao{-4},
	}
}

\ctexset{
	subsection ={
		format = \raggedright \bfseries,
	}
}

\begin{document}
	
%题目类型:
%题目区域:
%题目难度:
%思想方法:
%题目特征:
%题目备注:	

\chapter{辅助文档}

\section{题目类型}
\helpreflast{题目类型}	
	
	
	选择 \quad 实验 \quad 填空 \quad 作图 \quad 计算

\section{题目难度}	

\begin{table}[h!]
 \centering 
 \begin{tabular}{|c|c|c|c|c|c|c|c|c|c|c|c|c|c|c|c|c|c|c|c|c|}
 \hline 
0 & 0.5 &1 & 1.5 & 2 &  2.5 & 3 & 3.5 & 4 & 4.5 & 5 & 5.5 & 6 & 6.5 & 7 & 7.5  & 8 & 8.5 & 9 & 9.5 & 10\\ 
 \hline 
 \end{tabular}
 \end{table} 


\section{题目区域}
\hypertarget{题目区域}{题目区域}
\subsection{一级}
	
$  \text{必修1} 
\left\{
\begin{array}{l}
 \text{直线运动}  \\\\
 \text{相互作用} \\\\
 \helpreffist{运动定律} \\
\end{array}
\right.
 \quad \text{\hyperlink{思想方法}{\faCheckCircle}}
 $	
  \hfil 
 $  \text{必修2} 
 \left\{
\begin{array}{l}
  \helpreffist{曲线运动} \\\\
 \text{万有引力} \\\\
 \helpreffist{能量守恒}\\\\
  \text{相对论} \\
 \end{array}
 \right.
 \quad \text{\hyperlink{思想方法}{\faCheckCircle}}
 $	
  \hfil 
$  
\begin{array}{l}
\text{必修3} \\
\text{选修2} 
\end{array}
\left\{
\begin{array}{l}
\helpreffist{电场} \\\\
\text{电路} \\\\
 \helpreffist{磁场} \\\\
\helpreffist{电磁感应} \\\\
 \text{电磁波} \\
\end{array}
\right.
\quad \text{\hyperlink{思想方法}{\faCheckCircle}}
$

$  
 \text{选修1}
\left\{
\begin{array}{l}
 \helpreffist{动量} \\\\
  \helpreffist{机械波} \\\\
\helpreffist{光学} \\
\end{array}
\right.
\quad \text{\hyperlink{思想方法}{\faCheckCircle}}
$
\hfil
$  
\text{选修3}
\left\{
\begin{array}{l}
\text{分子动理论} \\\\
\helpreffist{热学} \\\\
\text{原子物理} \\\\
 \text{波粒二象性} \\
\end{array}
\right.
\quad \text{\hyperlink{思想方法}{\faCheckCircle}}
$


\newpage
\subsection{二级}
\begin{enumerate}
	%\renewcommand{\labelenumi}{\arabic{enumi}.}
	% A(\Alph) a(\alph) I(\Roman) i(\roman) 1(\arabic)
	%设定全局标号series=example	%引用全局变量resume=example
	%[topsep=-0.3em,parsep=-0.3em,itemsep=-0.3em,partopsep=-0.3em]
	%可使用leftmargin调整列表环境左边的空白长度 [leftmargin=0em]
\item [必修1  \quad ]

\helpreflast{运动定律}: \\\\
\hyperlink{题目区域}{\faBackward}  \quad  \quad 牛一律  \quad 牛二律  \quad 牛三律  \quad 超重失重 \quad \quad \hyperlink{思想方法}{\faForward}

\newpage

\item [必修2  \quad ]

\helpreflast{曲线运动}: \\\\
 \hyperlink{题目区域}{\faBackward}  \quad  \quad  平抛  \quad 斜抛 \quad 圆周 \quad \quad \hyperlink{思想方法}{\faForward}

\newpage

\helpreflast{能量守恒}: \\\\
  \hyperlink{题目区域}{\faBackward}   \quad  \quad 功与功率 \quad  \quad  动能定理 \quad 机械能守恒	 \quad \quad \hyperlink{思想方法}{\faForward}

\newpage

\item [必3  选2  \quad ]

\helpreflast{电场}: \\\\
  \hyperlink{题目区域}{\faBackward}  \quad  \quad  电场强度 \quad 电势能 \quad 电容器 \quad 电场中带电粒子的运动 \quad \quad \hyperlink{思想方法}{\faForward}

\newpage

\helpreflast{磁场}:\\\\
\hyperlink{题目区域}{\faBackward}  \quad  \quad 安培力  \quad 磁场中带电粒子的运动 \quad \quad \hyperlink{思想方法}{\faForward}



\newpage

\helpreflast{电磁感应}: \\\\
 \hyperlink{题目区域}{\faBackward}  \quad  \quad   楞次定律 \quad 电磁感应定律 \quad 变压器与高压输电 \quad \quad \hyperlink{思想方法}{\faForward}

\newpage

\item [选修1  \quad ]

\helpreflast{动量}:\\\\\
 \hyperlink{题目区域}{\faBackward}  \quad  \quad  动量定理 \quad 动量守恒 \quad \quad \hyperlink{思想方法}{\faForward}


\newpage
\helpreflast{光学}:\\\\\
\hyperlink{题目区域}{\faBackward}  \quad  \quad  光的折射 \quad 光的干涉 \quad 光的衍射 \quad \quad \hyperlink{思想方法}{\faForward}



\newpage

\helpreflast{机械波}: \\\\
  \hyperlink{题目区域}{\faBackward}  \quad  \quad  单摆 \quad 波的描述 \quad 多普勒效应 \quad 波的干涉 \quad \quad \hyperlink{思想方法}{\faForward}

\newpage

\item [选修3  \quad ]
\helpreflast{热学}: \\\\
 \hyperlink{题目区域}{\faBackward}  \quad  \quad   热力学第一定律 \quad 热力学第二定律 \quad 理想气体状态方程  \quad \quad \hyperlink{思想方法}{\faForward}


\newpage

\end{enumerate}



\hypertarget{思想方法}{思想方法}


\section{思想方法}


	等效 \quad 类比 \quad 相似 \quad 微元 \quad 对称 \quad 极限 \quad 假设 \quad 比例 
	
	 量纲 \quad 整体 \quad 隔离 \quad 函数 \quad 几何 \quad 参考系变换 \quad 小量近似


\section{题目特征}

图像选择 \quad 图像分析 \quad 材料分析 \quad 计算练习 \quad 物理学史


\section{题目备注} 

回首页  $ \rightarrow $ \helpreffist{题目类型}

\end{document}