\documentclass{ctexbook}
\usepackage{tikz}
\usetikzlibrary{mindmap,shadows}
\usepackage[top=2.5cm,bottom=2cm,left=2.5cm,right=2cm]{geometry}
\usepackage{listings}

\lstset{
	columns=fixed,       
	numbers=left,                                        % 在左侧显示行号
	numberstyle=\tiny\color{gray},                       % 设定行号格式
	frame=none,                                          % 不显示背景边框
	backgroundcolor=\color[RGB]{245,245,244},            % 设定背景颜色
	keywordstyle=\color[RGB]{40,40,255},                 % 设定关键字颜色
	numberstyle=\footnotesize\color{darkgray},           
	commentstyle=\it\color[RGB]{0,96,96},                % 设置代码注释的格式
	stringstyle=\rmfamily\slshape\color[RGB]{128,0,0},   % 设置字符串格式
	showstringspaces=false,                              % 不显示字符串中的空格
	language=Tex,                                        % 设置语言
}

\begin{document}
%\chapter{高考真题库建立细则}	

\chapter{高考真题库建立初步细则}	

高考历时多年,高考真题的数量已足够多,其质量自然是不言而谕,愚欲以历年的高考真题建立一套题库,供备课选题之用。为了后续的程序处理,摘录题的时候需要遵循相应的规则,现将规则阐述如下:
\section{必备项}
此选项为每一道题必须录入的信息。具体包含如下类型:
\begin{enumerate}
	\renewcommand{\labelenumi}{\arabic{enumi}.}
	% A(\Alph) a(\alph) I(\Roman) i(\roman) 1(\arabic)
	%设定全局标号series=example	%引用全局变量resume=example
	%[topsep=-0.3em,parsep=-0.3em,itemsep=-0.3em,partopsep=-0.3em]
	%可使用leftmargin调整列表环境左边的空白长度 [leftmargin=0em]
	\item
题目编号

该题在原真题试卷的题目顺序编号,不采用理综的真实编号,采用顺序编号,可选值依次为:01、02、03、04……

\item 
题目年份

可选值:…2018、2019、2020…

\item 
题目属地

可选值为:全国1、全国2、全国3、四川、浙江、北京……


\item 
题目类型

该题的题型,可选值为:选择、实验、填空、作图、计算。

\item 
题目区域(一级)

分为一级区域和二级区域,其中一级区域为必备选项,二级区域为可选选项。一级区域分类参见图 \ref{题目区域:一级区域},一级区域的关键词记为题目区域的可选值	。

\begin{figure}
	\centering
	\begin{tikzpicture}
	[mindmap,
	grow cyclic,
	every node/.style={concept,circular drop shadow,},
	concept color=teal!40,
	%当分支挤到一起时候作手动调节用
	level 1/.append style={level distance=5.5cm,sibling angle=360/5},
	%level 2/.append style={sibling angle=37.5},
	]
	\node [root concept] {区域\\关键词}
	%对child可以使用rotate=-10作手动调节
	%child [concept color=red!40, rotate=-10]
	%对第一阶child的伸展可使用counterclockwise from=-115来调节整体的旋转
	%node  {Cultural Factors}[counterclockwise from=-115]
	child [concept color=purple!40]{
		node    {必修1}
		child { node    {直线运动} }
		child { node    {相互作用} }
		child { node    {运动定律} }
	}
	child [concept color=pink!40]{
		node     {必修2}
		child { node    {曲线运动} 
			%child  { node {平抛} }	
			%child  { node {圆周} }		
		}
		child { node    {万有引力} }
		child { node    {能量守恒} 
			%child  { node {动能定理} }	
			%child  { node {机械能守恒} }		
		}
		child { node    {相对论} }
	}
	child [concept color=green!40,rotate=7]{
		node  {必修3\\选修2}
		child { node {电场} }
		child { node {电路} }
		child { node {磁场} }
		child { node {电磁感应} 
			%child  { node {楞次定律} }
			%child  { node {电磁感应定律} }		
			%child  { node {变压器} }		
		}
		child { node {电磁波} }
	}
	child [concept color=red!40]{
		node  {选修1}
		child  { node {动量} }
		child { node {机械波} 
			%child  { node {多普勒效应} }	
			%child  { node {干涉} }	
		}
		child { node {光学} }
	}
	child [concept color=violet!40] {
		node {选修3}
		child { node {分子动理论} }
		child { node {热学} }
		child { node {原子物理} }
		child { node {波粒二象性} }
	}
	;
	\end{tikzpicture}
	\caption{一级区域}\label{题目区域:一级区域}
\end{figure}

	
	
	
\end{enumerate}

\noindent
题目编号、题目年份、题目属地三者采用自动填充完成,只需要保证每张高考试卷的命名规范即可。

\newpage
\section{可选项}
	
此选项是为进一步增加题库的易用性,还待补充完善,可根据实际情况作相应添加。

\begin{enumerate}
	\renewcommand{\labelenumi}{\arabic{enumi}.}
	% A(\Alph) a(\alph) I(\Roman) i(\roman) 1(\arabic)
	%设定全局标号series=example	%引用全局变量resume=example
	%[topsep=-0.3em,parsep=-0.3em,itemsep=-0.3em,partopsep=-0.3em]
	%可使用leftmargin调整列表环境左边的空白长度 [leftmargin=0em]
\item 
题目难度

题目的难度系数为0 --- 9共10个等级,0表示极难,考试中只有千分之一或者万分之一的学生能将此题作出,9表示极易,百分之95及以上的学生都能将此题作出(其中的百分比可参考本省整体实力的正态分布)。
可选值为:0、1 …  8、9。

	\item
	思想方法
	
	按照物理思想和物理上常用方法进行分类,关键词有:等效、类比、相似、微元、对称、极限、假设、比例、量纲、整体、隔离、参考系变换、小量近似等。
	
	注:相对运动与参考系变换等效,不再单独区分。
	\item 
	题目区域(二级)
	
	例如:能量守恒下面可以再细分为动能定理、机械能守恒等。具体还待完善 。可参照图 \ref{题目区域:二级区域}。

\begin{figure}
	\centering
	\begin{tikzpicture}
[mindmap,
grow cyclic,
every node/.style={concept,circular drop shadow,},
concept color=teal!40,
%当分支挤到一起时候作手动调节用
level 1/.append style={level distance=6.5cm,sibling angle=360/5},
%level 2/.append style={sibling angle=37.5},
]
\node [root concept] {区域\\关键词}
%对child可以使用rotate=-10作手动调节
%child [concept color=red!40, rotate=-10]
%对第一阶child的伸展可使用counterclockwise from=-115来调节整体的旋转
%node  {Cultural Factors}[counterclockwise from=-115]
child [concept color=purple!40]{
	node    {必修1}
	child { node    {直线运动} }
	child { node    {相互作用} }
	child { node    {运动定律} }
}
child [concept color=pink!40]{
	node     {必修2}
	child { node    {曲线运动} 
		child  { node {平抛} }	
		child  { node {圆周} }		
	}
	child { node    {万有引力} }
	child { node    {能量守恒} 
		child  { node {动能定理} }	
		child  { node {机械能守恒} }		
	}
	child { node    {相对论} }
}
child [concept color=green!40,rotate=7]{
	node  {必修3\\选修2}
	child { node {电场} 
		child  { node {电场强度} }
		child  { node {电势能} }
		child  { node {电容器} }
	}
	child { node {电路} }
	child { node {磁场} }
	child { node {电磁感应} 
		child  { node {楞次定律} }
		child  { node {电磁感应定律} }		
		child  { node {变压器与高压输电} }		
	}
	child { node {电磁波} }
}
child [concept color=red!40]{
	node  {选修1}
	child  { node {动量} }
	child { node {机械波} 
		child  { node {单摆}}
		child  { node {多普勒效应} }	
		child  { node {干涉} }		
	}
	child { node {光学} }
}
	child [concept color=violet!40] {
		node {选修3}
		child { node {分子动理论} }
		child { node {热学} 
			child  { node {热力学第一定律}	}
			child  { node {热力学第二定律} }
			child  { node {理想气体状态方程} }
		}
		child { node {原子物理} }
		child { node {波粒二象性} }
	}
	;
	\end{tikzpicture}
	\caption{二级区域}\label{题目区域:二级区域}
\end{figure}


\item 
题目特征

针对选择题的图像问题来说存在两类问题:图像问题和图像分析问题,图像问题典型的是四个选项均为图像,选择恰当的图像;而图像分析一般是用根据所给的图表作为分析问题的切入点。(注意与一般所给的示意图作为题目说明不同)

关键词:图像选择、图像分析、材料分析、计算练习、物理学史。


注:其中的计算练习特征为题目难度不大,方法一般,但计算量大,易算错,可作为计算练习用。

\item 
题目备注

对一些题目给的提示、技巧解析等,可自由发挥。


\end{enumerate}


\noindent
以2020全国1卷的第1题举例,源码如下(分隔符为英文冒号):
\begin{lstlisting}
\item
行驶中的汽车如果发生剧烈碰撞,车内的安全气囊会被弹出并瞬间充满气体,若碰
撞后汽车的速度在很短时间内减小为零,关于安全气囊在此过程中的作用,下列说
法正确的是 \xzanswer{D} 

\fourchoices
{增加了司机单位面积的受力大小}
{减少了碰撞前后司机动量的变化量}
{将司机的动能全部转换成汽车的动能}
{延长了司机的受力时间并增大了司机的受力面积}

%题目类型:选择
%题目区域:动量:冲量
%题目难度:9
%思想方法:
%题目特征:

\end{lstlisting}

\section{录入格式说明}
\subsection{基本部分}
每张试卷共有头部、选择、实验、计算、选修几个部分组成,以前地方卷没有选修的不用加入,共有如下几个命令(去每个部分的英文缩写):
\begin{lstlisting}
\gaokaoheader{2020}{全国\lmd{2}卷}
\gaokaoxz
\gaokaosy
\gaokaojs
\gaokaoxx{$ 3 - 4 $}
\end{lstlisting}

其中的 header 部分带两个参数,一个是年份,一个是试卷所属地。其中选修部分带一个参数,为选修对应的书本名称编号。注意,其中的选修在原卷中一般是一个大题两个小问,在实际操作中需要在选修的大题好下加入一个选修的注释,具体如下:
\begin{lstlisting}
\item 
%选修 $ 3 - 4 $
\begin{enumerate}
\item
....
\end{enumerate}
\end{lstlisting}

\subsection{图片相关}
图片的编号与题目项关联,题目编号刷新,图片编号重新计数。采用图1、图2的方式显示,对于一些有子图的题目,对子图的编号按照$ (a) $、$ (b) $、$ (c) $的顺序进行。题中对于图片或者子图的引用最好使用原生引用的方法,而不是使用直接书写的方法。例如下面:
\begin{lstlisting}
\item 
($ 9 $ 分)
用图 \ref{2020:北京16:1} 所示的 \subref{2020:北京16:1a} 、
 \subref{2020:北京16:1b} 两种方法测量某电源的电动势和内...
\begin{figure}[h!]
\centering
\begin{subfigure}{0.4\linewidth}
\centering
\includesvg[width=0.7\linewidth]{picture/svg/GZ-3-tiyou-0797} 
\caption{}\label{2020:北京:16:1a}
\end{subfigure}
\hfil
\begin{subfigure}{0.4\linewidth}
\centering
\includesvg[width=0.7\linewidth]{picture/svg/GZ-3-tiyou-0798} 
\caption{}\label{2020:北京:16:1b}
\end{subfigure}
\caption{}\label{2020:北京:16:1}
\end{figure}	
\end{lstlisting}

另外标号(lable)的设定可参照上面代码的方式,可以根据编号直接定位题号,也可以根据题目直接写出其引用的lable。

\subsection{其它说明}

\begin{enumerate}
	%\renewcommand{\labelenumi}{\arabic{enumi}.}
	% A(\Alph) a(\alph) I(\Roman) i(\roman) 1(\arabic)
	%设定全局标号series=example	%引用全局变量resume=example
	%[topsep=-0.3em,parsep=-0.3em,itemsep=-0.3em,partopsep=-0.3em]
	%可使用leftmargin调整列表环境左边的空白长度 [leftmargin=0em]
	\item
题目中有关分值的信息不保留。
\item 
选修部分,一般一道题里面包含两个小题,在做答案的时候这两道小题分开做,各做各的,方便以后对选修相关题目的拆分。
另外选修部分题目的标注是对每个小题进行单独标注。


\item 
A4的排版与A3的没有差别,编译不同的版式不会导致格式的差异变化。但应当注意的是在排版的过程中以解答做题方便为主,不需要太过追求内容的连续,该分页的时候果断分页。	

\item 
在排版的过程中少用\lstinline|\vspace{-1em}|之类。


\item 
在标号的引用过程中直接引用即可\lstinline|\ref{2020:北京:16:1}|前后最好用空格与其它内容隔开,不需要加其它的格式(例如括号之类),后面如果需要该相应的格式或者标号统一在模板文件里面修改即可。

\end{enumerate}



\end{document}