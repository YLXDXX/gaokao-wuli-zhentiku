\bta{验证平行四边形法则}
\begin{enumerate}
\renewcommand{\labelenumi}{\arabic{enumi}.}
% A(\Alph) a(\alph) I(\Roman) i(\roman) 1(\arabic)
%设定全局标号series=example	%引用全局变量resume=example
%[topsep=-0.3em,parsep=-0.3em,itemsep=-0.3em,partopsep=-0.3em]
%可使用leftmargin调整列表环境左边的空白长度 [leftmargin=0em]
\item
\exwhere{$ 2011 $ 年物理江苏卷}
某同学用如图所示的实验装置来验证“力的平行四边形定则”。弹簧测力计 $ A $ 挂于固定点
$ P $,下端用细线挂一重物 $ M $。弹簧测力计 $ B $ 的一端用细线系于 $ O $ 点,手持另一端向左拉,使结点 $ O $
静止在某位置。分别读出弹簧测力计 $ A $ 和 $ B $ 的示数,并在贴于竖直木板的白纸上记录 $ O $ 点的位置
和拉线的方向。
\begin{enumerate}
\renewcommand{\labelenumi}{\arabic{enumi}.}
% A(\Alph) a(\alph) I(\Roman) i(\roman) 1(\arabic)
%设定全局标号series=example	%引用全局变量resume=example
%[topsep=-0.3em,parsep=-0.3em,itemsep=-0.3em,partopsep=-0.3em]
%可使用leftmargin调整列表环境左边的空白长度 [leftmargin=0em]
\item
本实验用的弹簧测力计示数的单位为 $ N $,图中 $ A $ 的示数为 \tk{$ 3.6 $} $ N $。
\begin{figure}[h!]
\centering
\includesvg[width=0.36\linewidth]{picture/svg/GZ-3-tiyou-0454}
\end{figure}



\item 
下列不必要的实验要求是 \tk{D} 。(请填写选项前对应的字母)

\fourchoices
{应测量重物 $ M $ 所受的重力}
{弹簧测力计应在使用前校零}
{拉线方向应与木板平面平行}
{改变拉力,进行多次实验,每次都要使 $ O $ 点静止在同一位置}

\item 
某次实验中,该同学发现弹簧测力计 $ A $ 的指针稍稍超出量程,请您提出两个解决办法。

\tk{①改变弹簧测力计 $ B $ 拉力的大小;
\\
②减小重物 $ M $ 的质量(或将 $ A $ 更换为较大的测力计,改变弹簧测力计 $ B $ 拉力的方向)} 


\end{enumerate}



\newpage
\item
\exwhere{$ 2015 $ 年理综安徽卷}
在“验证力的平行四边形定则”实验中,某同学用图钉把白纸固定在水平放置的木板上,将橡皮
条的一端固定在板上一点,两个细绳套系在橡皮条的另一端。用两个弹簧测力计分别拉住两个细绳
套,互成角度地施加拉力,使橡皮条伸长,结点到达纸面上某
一位置,如图所示。请将以下的实验操作和处理补充完整:
\begin{figure}[h!]
\centering
\includesvg[width=0.23\linewidth]{picture/svg/GZ-3-tiyou-0455}
\end{figure}

①用铅笔描下结点位置,记为 $ O $;

②记录两个弹簧测力计的示数 $ F_{1} $ 和 $ F_{2} $,沿每条细绳(套)的方
向用铅笔分别描出几个点,用刻度尺把相应的点连成线;

③只用一个弹簧测力计,通过细绳套把橡皮条的结点仍拉到位
置 $ O $,记录测力计的示数
$ F_{3} $,
\tk{沿此时细绳(套)的方向用铅笔描出几个点,用刻度尺把这些点连成直线};

④按照力的图示要求,作出拉力 $ F_{1} $、$ F_{2} $、$ F_{3} $;

⑤根据力的平行四边形定则作出 $ F_{1} $ 和 $ F_{2} $ 的合力 $ F $;


⑥比较
\tk{$ F $ 和 $ F_{3} $} 
的一致程度,若有较大差异,对其原因进行分析,并作出相应的改进后再次进
行实验。




\newpage
\item
\exwhere{$ 2018 $ 年天津卷}
某研究小组做“验证力的平行四边形定则”的实验,所有器材有:方木板一块,
白纸,量程为 $ 5 \ N $ 的弹簧测力计两个,橡皮条(带两个较长的细绳套),刻度尺,图钉(若干个)。


①具体操作前,同学们提出了如下关于实验操作的建议,其中正确的是 \tk{BC} 。
\fourchoices
{橡皮条应和两绳套夹角的角平分线在一条直线上}
{重复实验再次进行验证时,结点 $ O $ 的位置可以与前一次不同}
{使用测力计时,施力方向应沿测力计轴线;读数时视线应正对测力计刻度}
{用两个测力计互成角度拉橡皮条时的拉力必须都小于只用一个测力计时的拉力}


②该小组的同学用同一套器材做了四次实验,白纸上留下的标注信息有结点位置 $ O $,力的标度、分
力和合力的大小及表示力的作用线的点,如下图所示。其中对于提高实验精度最有利的是 \tk{B}。
\begin{figure}[h!]
\centering
\includesvg[width=0.83\linewidth]{picture/svg/GZ-3-tiyou-0456}
\end{figure}




\newpage
\item
\exwhere{$ 2012 $ 年理综浙江卷}
在“探究求合力的方法”实验中,现有木板、白纸、图钉、橡皮筋、细绳套和一把弹簧秤。
\begin{enumerate}
\renewcommand{\labelenumi}{\arabic{enumi}.}
% A(\Alph) a(\alph) I(\Roman) i(\roman) 1(\arabic)
%设定全局标号series=example	%引用全局变量resume=example
%[topsep=-0.3em,parsep=-0.3em,itemsep=-0.3em,partopsep=-0.3em]
%可使用leftmargin调整列表环境左边的空白长度 [leftmargin=0em]
\item
为完成实验,某同学另找来一根弹簧,先测量其劲度系数,得到的实验数据如下表:
\begin{table}[h!]
\centering 
\begin{tabular}{|c|c|c|c|c|c|c|c|}
\hline 
弹力$ F $($ N $) & $ 0.50 $ & $ 1.00 $ & $ 1.50 $ & $ 2.00 $ & $ 2.50 $ & $ 3.00 $ & $ 3.50 $
\\
\hline
伸长量$ x $($ 10^{-2} \ m $) & $ 0.74 $ & $ 1.80 $ & $ 2.80 $ & $ 3.72 $ & $ 4.60 $ & $ 5.58 $ & $ 6.42 $\\ 
\hline 
\end{tabular}
\end{table} 


用作图法求得该弹簧的劲度系数 $ k=$ \tk{54} $N/m $;
\begin{figure}[h!]
\centering
\includesvg[width=0.83\linewidth]{picture/svg/GZ-3-tiyou-0457}
\end{figure}



\item 
某次实验中,弹簧秤的指针位置如图所示,其读数为
\tk{$ 2.10 $} 
$ N $;
同时利用$ (1) $中结果获得弹簧上的弹力值为 $ 2.50 \ N $,请在答题纸上画出这两个共点力的合力 $ F_{ \text{合} } $ ;

\banswer{
 \includesvg[width=0.73\linewidth]{picture/svg/GZ-3-tiyou-0458} 
}


\item 
由图得到 $ F _{ \text{合} } = $
\tk{$ 3.3 $} 
$ N $。

\banswer{

}


\end{enumerate}


\newpage
\item
\exwhere{$ 2014 $ 年物理江苏卷}
小明通过实验验证力的平行四边形定则.
\begin{figure}[h!]
\centering
\includesvg[width=0.83\linewidth]{picture/svg/GZ-3-tiyou-0459}
\end{figure}

\begin{enumerate}
\renewcommand{\labelenumi}{\arabic{enumi}.}
% A(\Alph) a(\alph) I(\Roman) i(\roman) 1(\arabic)
%设定全局标号series=example	%引用全局变量resume=example
%[topsep=-0.3em,parsep=-0.3em,itemsep=-0.3em,partopsep=-0.3em]
%可使用leftmargin调整列表环境左边的空白长度 [leftmargin=0em]
\item
实验记录纸如题 $ 11-1 $ 图所示,$ O $ 点为橡皮筋被拉伸后伸长到的位置,两弹簧测力计共同作用
时,拉力 $ F_{1} $ 和 $ F_{2} $ 的方向分别过 $ P_{1} $ 和 $ P_{2} $ 点;一个弹簧测力计拉橡皮筋时,拉力 $ F_{3} $ 的方向过 $ P_{3} $
点。 三个力的大小分别为:$ F_{1} =3.30 \ N $、$ F_{2} =3.85 \ N $ 和 $ F_{3} =4.25 \ N $. 请根据图中给出的标度作图求
出 $ F_{1} $ 和 $ F_{2} $ 的合力.
\banswer{
 \includesvg[width=0.53\linewidth]{picture/svg/GZ-3-tiyou-0460} \\
 (4.6~4.9都算对);
}



\item 
仔细分析实验,小明怀疑实验中的橡皮筋被多次拉伸后弹性发生了变化,影响实验结果. 他用
弹簧测力计先后两次将橡皮筋拉伸到相同长度, 发现读数不相同,于是进一步探究了拉伸过程对
橡皮筋弹性的影响.

实验装置如题 $ 11-2 $ 图所示,将一张白纸固定在竖直放置的木板上,橡皮筋的上端固定于 $ O $ 点,
下端 $ N $ 挂一重物。用与白纸平行的水平力缓慢地移动 $ N $,在白纸上记录下 $ N $ 的轨迹. 重复上述过
程,再次记录下 $ N $ 的轨迹。

两次实验记录的轨迹如题 $ 11-3 $ 图所示. 过 $ O $ 点作一条直线与轨迹交于 $ a $、 $ b $ 两点, 则实验
中橡皮筋分别被拉伸到 $ a $ 和 $ b $ 时所受拉力 $ Fa $、$ Fb $ 的大小关系为
\tk{$ F_a=F_b $} 
.

\item 
根据($ 2) $ 中的实验,可以得出的实验结果有哪些? \tk{BD} ( 填写选项前的字母)
\fourchoices
{橡皮筋的长度与受到的拉力成正比}
{两次受到的拉力相同时,橡皮筋第 $ 2 $ 次的长度较长}
{两次被拉伸到相同长度时,橡皮筋第 $ 2 $ 次受到的拉力较大}
{两次受到的拉力相同时,拉力越大,橡皮筋两次的长度之差越大}



\item 
根据小明的上述实验探究,请对验证力的平行四边形定则实验提出两点注意事项.

\tk{橡皮筋不宜过长;选用新橡皮筋。
(或拉力不宜过大;选用弹性好的橡皮
筋;换用弹性好的弹簧。)} 

\end{enumerate}

\banswer{

}



\newpage
\item
\exwhere{$ 2015 $ 年理综山东卷}
某同学通过下述实验验证力的平行四边形定则。
\begin{figure}[h!]
\centering
\includesvg[width=0.43\linewidth]{picture/svg/GZ-3-tiyou-0461}
\end{figure}

实验步骤:

①
将弹簧秤固定在贴有白纸的竖直木板上,使其轴
线沿竖直方向。


②
如图甲所示,将环形橡皮筋一端挂在弹簧秤的秤
钩上,另一端用圆珠笔尖竖直向下拉,直到弹簧秤示
数为某一设定值时,将橡皮筋两端的位置标记为 $ O_{1} $、
$ O_{2} $,,记录弹簧秤的示数 $ F $,测量并记录 $ O_{1} $、$ O_{2} $,间的
距离(即橡皮筋的长度 $ l $)。每次将弹簧秤示数改变 $ 0.50 \ N $,测出所对应的 $ l $,部分数据如下表所示:
\begin{table}[h!]
\centering 
\begin{tabular}{|c|c|c|c|c|c|c|c|}
\hline 
$ F(N) $ & $ 0 $ & & $ 0.50 $ & $ 1.00 $ & $ 1.50 $ & $ 2.00 $ & $ 2.50 $
 \\
\hline
$ l $($ cm $) & $ l_{0} $ & & $ 10.97 $ & $ 12.02 $ & $ 13.00 $ & $ 13.98 $ & $ 15.05 $\\ 
\hline 
\end{tabular}
\end{table} 





③ 找出②中 $ F=2.50 \ N $ 时橡皮筋两端的位置,重新标记为 $ O $、$ O ^{\prime} $,橡皮筋的拉力计为 $ Foo ^{\prime} $。



④ 在秤钩上涂抹少许润滑油,将橡皮筋搭在秤钩上,如图乙所示。用两圆珠笔尖呈适当角度同时
拉橡皮筋的两端,使秤钩的下端达到 $ O $ 点,将两笔尖的位置标记为 $ A $、$ B $,橡皮筋 $ OA $ 段的拉力记
为 $ F_{OA} $,$ OB $ 段的拉力记为 $ F_{OB} $。


完成下列作图和填空:
\begin{enumerate}
\renewcommand{\labelenumi}{\arabic{enumi}.}
% A(\Alph) a(\alph) I(\Roman) i(\roman) 1(\arabic)
%设定全局标号series=example	%引用全局变量resume=example
%[topsep=-0.3em,parsep=-0.3em,itemsep=-0.3em,partopsep=-0.3em]
%可使用leftmargin调整列表环境左边的空白长度 [leftmargin=0em]
\item
利用表中数据在给出的坐标纸上(见答题卡)画出 $ F-l $ 图线,根据图线求得 $ l_{0}= $ \tk{$ 10.00 $} $ cm $ 。

\item 
测得 $ OA=6.00 \ cm $,$ OB=7.60 \ cm $,则 $ FOA $ 的大小为
\tk{$ 1.8 $} 
$ N $。

\item 
根据给出的标度,在答题卡上作出 $ F_{OA} $ 和 $ F_{OB} $ 的合力 $ F ^{\prime} $ 的图示。
\banswer{
 \includesvg[width=0.23\linewidth]{picture/svg/GZ-3-tiyou-0462} 
}


\item 
通过比较 $ F ^{\prime} $ 与
\tk{$ F_{OO ^{\prime} } $} 
的大小和方向,即可得出实验结论。




\end{enumerate}


\banswer{

}


\newpage
\item 
\exwhere{$ 2016 $ 年浙江卷}
某同学在“探究弹簧和弹簧伸长的关系”的实验中,测得图中弹簧 $ OC $
的劲度系数为 $ 500 \ N/m $。如图 $ 1 $ 所示,用弹簧 $ OC $ 和弹簧秤 $ a $、$ b $ 做“探究求合力的方法”实验。在保
持弹簧伸长 $ 1.00 \ cm $ 不变的条件下:
\begin{figure}[h!]
\centering
\includesvg[width=0.53\linewidth]{picture/svg/GZ-3-tiyou-0463}
\end{figure}
\begin{enumerate}
\renewcommand{\labelenumi}{\arabic{enumi}.}
% A(\Alph) a(\alph) I(\Roman) i(\roman) 1(\arabic)
%设定全局标号series=example	%引用全局变量resume=example
%[topsep=-0.3em,parsep=-0.3em,itemsep=-0.3em,partopsep=-0.3em]
%可使用leftmargin调整列表环境左边的空白长度 [leftmargin=0em]
\item
弹簧秤 $ a $、$ b $ 间夹角为 $ 90 ^{ \circ } $,弹簧秤 $ a $ 的读
数是 \tk{$ 3.00 \sim 3.02 $} $ N $(图 $ 2 $ 中所示),则弹簧秤 $ b $ 的读
数可能为 \tk{$ 3.9 \sim 4.1 $(有效数不作要求)} $ N $。

\item 
若弹簧秤 $ a $、$ b $ 间夹角大于 $ 90 ^{ \circ } $,保持弹簧
秤 $ a $ 与弹簧 $ OC $ 的夹角不变,减小弹簧秤 $ b $ 与弹
簧 $ OC $ 的夹角,则弹簧秤 $ a $ 的读数是 \tk{变大} 、弹簧秤 $ b $ 的读数 \tk{变大} (填“变大”、“变小”或“不变”)。

\end{enumerate}



\newpage
\item 
\exwhere{$ 2017 $ 年新课标$ \lmd{3} $卷}
某探究小组做“验证力的平行四边形定则”实验,将画有坐标轴(横轴为 $ x $ 轴,纵轴为 $ y $ 轴,最小刻
度表示 $ 1 \ mm $)的纸贴在桌面上,如图($ a $)所示。将橡皮筋的一端 $ Q $ 固定在 $ y $ 轴上的 $ B $ 点(位于图
示部分之外)
,另一端 $ P $ 位于 $ y $ 轴上的 $ A $ 点时,橡皮筋处于原长。
\begin{figure}[h!]
\centering
\includesvg[width=0.53\linewidth]{picture/svg/GZ-3-tiyou-0464}
\end{figure}

\begin{enumerate}
\renewcommand{\labelenumi}{\arabic{enumi}.}
% A(\Alph) a(\alph) I(\Roman) i(\roman) 1(\arabic)
%设定全局标号series=example	%引用全局变量resume=example
%[topsep=-0.3em,parsep=-0.3em,itemsep=-0.3em,partopsep=-0.3em]
%可使用leftmargin调整列表环境左边的空白长度 [leftmargin=0em]
\item
用一只测力计将橡皮筋的 $ P $ 端沿 $ y $ 轴从 $ A $ 点拉至坐标原点 $ O $,此时拉力 $ F $ 的大小可由测力计
读出。测力计的示数如图($ b $)所示,$ F $ 的大小为
\tk{$ 4.0 $} 
$ N $。



\item 
撤去($ 1 $)中的拉力,橡皮筋 $ P $ 端回到 $ A $
点;现使用两个测力计同时拉橡皮筋,再次将
$ P $ 端拉至 $ O $ 点,此时观察到两个拉力分别沿图
($ a $)中两条虚线所示的方向,由测力计的示数读出两个拉力的大小分别为 $ F_{1} =4.2 \ N $ 和 $ F_{2} =5.6 \ N $。
\begin{enumerate}
\renewcommand{\labelenumiii}{\roman{enumiii}.}
% A(\Alph) a(\alph) I(\Roman) i(\roman) 1(\arabic)
%设定全局标号series=example	%引用全局变量resume=example
%[topsep=-0.3em,parsep=-0.3em,itemsep=-0.3em,partopsep=-0.3em]
%可使用leftmargin调整列表环境左边的空白长度 [leftmargin=0em]
\item
用 $ 5 \ mm $ 长度的线段表示 $ 1 \ N $ 的力,以 $ O $ 点为作用点,在图($ a $)中画出力 $ F_{1} $、$ F_{2} $ 的图示,然
后按平形四边形定则画出它们的合力 $ F_{ \text{合} } $ ;

\banswer{
 \includesvg[width=0.23\linewidth]{picture/svg/GZ-3-tiyou-0465} 
}


\item 
$ F $ 合的大小为
\tk{$ 3.96 $} 
$ N $,$ F_{ \text{合} } $ 与拉力 $ F $ 的夹角的正切值
为 \tk{$ 0.1 $} 
。




\end{enumerate}

若 $ F_{ \text{合} } $ 与拉力 $ F $ 的大小及方向的偏差均在实验所允许的误差范围
之内,则该实验验证了力的平行四边形定则。



\end{enumerate}



\banswer{

}







\end{enumerate}

