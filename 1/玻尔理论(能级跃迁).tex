\bta{玻尔理论(能级跃迁)}


\begin{enumerate}
	%\renewcommand{\labelenumi}{\arabic{enumi}.}
	% A(\Alph) a(\alph) I(\Roman) i(\roman) 1(\arabic)
	%设定全局标号series=example	%引用全局变量resume=example
	%[topsep=-0.3em,parsep=-0.3em,itemsep=-0.3em,partopsep=-0.3em]
	%可使用leftmargin调整列表环境左边的空白长度 [leftmargin=0em]
	\item
\exwhere{$ 2019 $ 年 $ 4 $ 月浙江物理选考}
波长为$ \lambda _{1} $ 和$ \lambda _{2} $ 的两束可见光入射到双缝,在光屏上观察到
干涉条纹,其中波长为$ \lambda _{1} $ 的光的条纹间距大于波长为$ \lambda _{2} $ 的条纹间距。则(下列表述中,脚标“$ 1 $”和“$ 2 $”
分别代表波长为$ \lambda _{1} $ 和$ \lambda _{2} $ 的光所对应的物理量) \xzanswer{BD} 

\fourchoices
{这两束光的光子的动量 $ p_{1}>p_{2} $}
{这两束光从玻璃射向真空时,其临界角 $ C_{1} > C_{2} $}
{这两束光都能使某种金属发生光电效应,则遏止电压 $ U_{1} > U_{2} $}
{这两束光由氢原子从不同激发态跃迁到 $ n=2 $ 能级时产生,则相应激发态的电离能$ \triangle E_{1} > \triangle E_{2} $}



\item 
\exwhere{$ 2019 $年物理全国\lmd{1}卷}
氢原子能级示意图如图所示。光子能量在$ 1.63 \ eV \sim 3.10 \ eV $的光为可见光。
要使处于基态($ n=1 $)的氢原子被激发后可辐射出可见光光子,最少应给氢原子提供的能量为 \xzanswer{A} 
\begin{figure}[h!]
	\centering
	\includesvg[width=0.23\linewidth]{picture/svg/GZ-3-tiyou-1292}
\end{figure}


\fourchoices
{$ 12.09 \ eV $}
{$ 10.20 \ eV $}
{$ 1.89 \ eV $}
{$ 1.51 \ eV $}




\item 
\exwhere{$ 2016 $ 年北京卷}
处于 $ n=3 $ 能级的大量氢原子,向低能级跃迁时,辐射光的频率有 \xzanswer{C} 

\fourchoices
{$ 1 $ 种}
{$ 2 $ 种}
{$ 3 $ 种}
{$ 4 $ 种}


\item 
\exwhere{$ 2012 $ 年理综北京卷}
 一个氢原子从 $ n=3 $ 能级跃迁到 $ n=2 $ 能级。该氢原子 \xzanswer{B} 
 

\fourchoices
{放出光子,能量增加}
{放出光子,能量减少}
{吸收光子,能量增加}
{吸收光子,能量减少}


\item
\exwhere{$ 2012 $ 年理综四川卷}
如图为氢原子能级示意图的一部分,则氢原子 \xzanswer{A} 
\begin{figure}[h!]
	\centering
	\includesvg[width=0.23\linewidth]{picture/svg/GZ-3-tiyou-1292}
\end{figure}


\fourchoices
{从 $ n=4 $ 能级跃迁到 $ n=3 $ 能级比从 $ n=3 $ 能级跃迁到 $ n=2 $ 能级辐射出电磁波的波长长}
{从 $ n=5 $ 能级跃迁到 $ n=l $ 能级比从 $ n=5 $ 能级跃迁到 $ n=4 $ 能级辐射出电磁波的速度大}
{处于不同能级时,核外电子在各处出现的概率是一样的}
{从高能级向低能级跃迁时,氢原子核一定向外放出能量}

\item 
\exwhere{$ 2011 $ 年理综全国卷}
已知氢原子的基态能量为 $ E_{1} $,激发态能量 $ E_{n}= E_{1} / n^{2} $,其中 $ n=2 $,$ 3 $,$ \cdots $。用 $ h $ 表示普朗克常量,
$ c $ 表示真空中的光速。能使氢原子从第一激发态电离的光子的最大波长为 \xzanswer{C} 


\fourchoices
{$-\frac{4 h c}{3 E_{1}}$}
{$-\frac{2 h c}{E_{1}} $}
{$-\frac{4 h c}{E_{1}} $}
{$-\frac{9 h c}{E_{1}}$}

\item 
\exwhere{$ 2011 $ 年理综四川卷}
氢原子从能级 $ m $ 跃迁到能级 $ n $ 时辐射红光的频率为$ \nu _{1} $,从能级 $ n $ 跃迁到能级 $ k $ 时吸收紫光的频
率为$ \nu _{2} $。已知普朗克常量为 $ h $,若氢原子从能级 $ k $ 跃迁到能级 $ m $,则 \xzanswer{D} 

\fourchoices
{吸收光子的能量为 $ h \nu _{1} +h \nu _{2} $}
{辐射光子的能量为 $ h \nu _{1} +h \nu _{2} $}
{吸收光子的能量为 $ h \nu _{2} -h \nu _{1} $}
{辐射光子的能量为 $ h \nu _{2} -h \nu _{1} $}

\item 
\exwhere{$ 2017 $ 年浙江选考卷}
下列说法正确的是 \xzanswer{BC} 

\fourchoices
{$ \beta , \gamma $射线都是电磁波}
{原子核中所有核子单独存在时,质量总和大于该原子核的总质量,}
{在 $ LC $ 振荡电路中,电容器刚放电时电容器极板上电量最多,回路电流最小}
{处于 $ n=4 $ 激发态的氢原子,共能辐射出四种不同频率的光子}


\item 
\exwhere{$ 2018 $ 年天津卷}
氢原子光谱在可见光区域内有四条谱线 $ H_{ \alpha } $、 $ H_{\beta} $、 $ H_{\gamma} $、 $ H_{\delta} $,都是氢原子中
电子从量子数 $ n>2 $ 的能级跃迁到 $ n=2 $ 的能级发出的光,它们在真空中的波长由长到短,可以判定 \xzanswer{A} 

\fourchoices
{$ H_{ \alpha } $ 对应的前后能级之差最小}
{同一介质对 $ H_{ \alpha } $ 的折射率最大}
{同一介质中 $ H_{\delta} $ 的传播速度最大}
{用 $ H_{\gamma} $ 照射某一金属能发生光电效应,则 $ H_{\beta} $也一定能}


\item 
\exwhere{$ 2018 $ 年浙江卷($ 4 $ 月选考)}
氢原子的能级图如图所示,关于大量氢原子的能级跃
迁,下列说法正确的是(可见光的波长范围为 $ 4.0 \times 10^{-7} \ m - 7.6 \times 10^{-7} \ m $,
普朗克常量 $ h=6.6 \times 10^{-34} \ J \cdot s $,真空中的光速 $ c=3.0 \times 10^{8} \ m /s $) \xzanswer{BC} 
\begin{figure}[h!]
	\centering
	\includesvg[width=0.23\linewidth]{picture/svg/GZ-3-tiyou-1293}
\end{figure}


\fourchoices
{氢原子从高能级跃迁到基态时,会辐射$ \gamma $射线}
{氢原子处在 $ n=4 $ 能级,会辐射可见光}
{氢原子从高能级向 $ n=3 $ 能级跃迁时,辐射的光具有显著的热效应}
{氢原子从高能级向 $ n=2 $ 能级跃迁时,辐射的光在同一介质中传播速度最小的光子能量为 $ 1.89 \ eV $}




	
\end{enumerate}

