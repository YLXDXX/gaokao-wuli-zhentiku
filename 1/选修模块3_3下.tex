\bta{选修模块 $ 3 $—$ 3 $(下)}


\begin{enumerate}
	%\renewcommand{\labelenumi}{\arabic{enumi}.}
	% A(\Alph) a(\alph) I(\Roman) i(\roman) 1(\arabic)
	%设定全局标号series=example	%引用全局变量resume=example
	%[topsep=-0.3em,parsep=-0.3em,itemsep=-0.3em,partopsep=-0.3em]
	%可使用leftmargin调整列表环境左边的空白长度 [leftmargin=0em]
	\item
\exwhere{$ 2014 $ 年理综新课标 \lmd{1} 卷}
\begin{enumerate}
	%\renewcommand{\labelenumi}{\arabic{enumi}.}
	% A(\Alph) a(\alph) I(\Roman) i(\roman) 1(\arabic)
	%设定全局标号series=example	%引用全局变量resume=example
	%[topsep=-0.3em,parsep=-0.3em,itemsep=-0.3em,partopsep=-0.3em]
	%可使用leftmargin调整列表环境左边的空白长度 [leftmargin=0em]
	\item
一定量的理想气体从状态 $ a $ 开始,经历三个过程 $ ab $、$ bc $、$ ca $ 回到原状态.其 $ p - T $
图像如图所示.下列判断正确的是 \underlinegap .
\begin{figure}[h!]
	\centering
	\includesvg[width=0.23\linewidth]{picture/svg/GZ-3-tiyou-1483}
\end{figure}

\fivechoices
{过程 $ ab $ 中气体一定吸热}
{过程 $ bc $ 中气体既不吸热也不放热}
{过程 $ ca $ 中外界对气体所做的功等于气体所放的热}
{$ a $、$ b $ 和 $ c $ 三个状态中,状态 $ a $ 分子的平均动能最小}
{$ b $ 和 $ c $ 两个状态中,容器壁单位面积单位时间内受到气体分子撞击的次数不同}



 \tk{ADE} 


\item 
一定质量的理想气体被活塞封闭在竖直放置的圆柱形气缸内,气缸壁导热良好,活塞
可沿气缸壁无摩擦地滑动.开始时气体压强为 $ p $,活塞下表面相对于气缸底部的高度为 $ h $,外界的
温度为 $ T_{0} $.现取质量为 $ m $ 的沙子缓慢地倒在活塞的上表面,沙子倒完时,活塞下降了$ \frac{h}{4} $ 。若此后外
界的温度变为 $ T $,求重新达到平衡后气体的体积.已知外界大气的压强始终保持不变,重力加速度
大小为 $ g $.

\banswer{
	$\frac{9 m g h T}{4 p T_{0}}$
}

	
\end{enumerate}



\item 
\exwhere{$ 2014 $ 年理综新课标$ \lmd{2} $卷}
\begin{enumerate}
	%\renewcommand{\labelenumi}{\arabic{enumi}.}
	% A(\Alph) a(\alph) I(\Roman) i(\roman) 1(\arabic)
	%设定全局标号series=example	%引用全局变量resume=example
	%[topsep=-0.3em,parsep=-0.3em,itemsep=-0.3em,partopsep=-0.3em]
	%可使用leftmargin调整列表环境左边的空白长度 [leftmargin=0em]
	\item
下列说法正确的是 \underlinegap 。(填正确答案标号,选对 $ 1 $ 个给 $ 2 $ 分,选对 $ 2 $ 个得 $ 4 $ 分,选对 $ 3 $ 个得 $ 5 $
分,每选错 $ 1 $ 个扣 $ 3 $ 分,最低得分 $ 0 $ 分)
\fivechoices
{悬浮在水中的花粉的布朗运动反映了花粉分子的热运动}
{空气的小雨滴呈球形是水的表面张力作用的结果}
{彩色液晶显示器利用了液晶的光学性质具有各向异性的特点}
{高原地区水的沸点较低,这是高原地区温度较低的缘故}
{干湿泡温度计的湿泡显示的温度低于干泡显示的温度,这是湿泡外纱布中的水蒸发吸热的结果}


 \tk{BCE} 

\item 
如图,两气缸 $ AB $ 粗细均匀,等高且内壁光滑,其下部由体积可忽略的细管连通;$ A $ 的直径
为 $ B $ 的 $ 2 $ 倍,$ A $ 上端封闭,$ B $ 上端与大气连通;两气缸除 $ A $ 顶部导热外,其余部分均绝热。两气缸
中各有一厚度可忽略的绝热轻活塞 $ a $、$ b $,活塞下方充有氮气,活塞
$ a $ 上方充有氧气;当大气压为 $ P_{0} $,外界和气缸内气体温度均为 $ 7 \ \celsius $且
平衡时,活塞 $ a $ 离气缸顶的距离是气缸高度的$  \frac{ 1 }{ 4 }  $,活塞 $ b $ 在气缸的
正中央。
\begin{enumerate}
	%\renewcommand{\labelenumi}{\arabic{enumi}.}
	% A(\Alph) a(\alph) I(\Roman) i(\roman) 1(\arabic)
	%设定全局标号series=example	%引用全局变量resume=example
	%[topsep=-0.3em,parsep=-0.3em,itemsep=-0.3em,partopsep=-0.3em]
	%可使用leftmargin调整列表环境左边的空白长度 [leftmargin=0em]
	\item
现通过电阻丝缓慢加热氮气,当活塞 $ b $ 升至顶部时,求氮气的温度;
\item 
继续缓慢加热,使活塞 $ a $ 上升,当活塞 $ a $ 上升的距离是气缸高度的 $ 1/16 $ 时,求氧气的压
强。
\end{enumerate}
\begin{figure}[h!]
	\centering
	\includesvg[width=0.23\linewidth]{picture/svg/GZ-3-tiyou-1484}
\end{figure}

\banswer{
	\begin{enumerate}
		%\renewcommand{\labelenumi}{\arabic{enumi}.}
		% A(\Alph) a(\alph) I(\Roman) i(\roman) 1(\arabic)
		%设定全局标号series=example	%引用全局变量resume=example
		%[topsep=-0.3em,parsep=-0.3em,itemsep=-0.3em,partopsep=-0.3em]
		%可使用leftmargin调整列表环境左边的空白长度 [leftmargin=0em]
		\item
		$ 320 \ K $
		\item 
		$  \frac{ 4 }{ 3 } P_{0} $
	\end{enumerate}	
}


\end{enumerate}

\item 
\exwhere{$ 2014 $ 年理综重庆卷}
\begin{enumerate}
	%\renewcommand{\labelenumi}{\arabic{enumi}.}
	% A(\Alph) a(\alph) I(\Roman) i(\roman) 1(\arabic)
	%设定全局标号series=example	%引用全局变量resume=example
	%[topsep=-0.3em,parsep=-0.3em,itemsep=-0.3em,partopsep=-0.3em]
	%可使用leftmargin调整列表环境左边的空白长度 [leftmargin=0em]
	\item
重庆出租车常以天然气作为燃料,加气站储气罐中天然气的温度随气温升高的过程
中,若储气罐内气体体积及质量均不变,则罐内气体(可视为理想气体) \xzanswer{B} 

\fourchoices
{压强增大,内能减小}
{吸收热量,内能增大}
{压强减小,分子平均动能增大}
{对外做功,分子平均动能减小}



\item 
下图为一种减震垫,上面布满了圆柱状薄膜气泡,每个气泡内充满体积为 $ V_{0} $,压
强为 $ P_{0} $ 的气体。当平板状物品平放在气泡上时,气泡被压缩。若气泡内气体可视为理想气体,其
温度保持不变。当体积压缩到 $ V $ 时气泡与物品接触面的边界为 $ S $。求此时每个气泡内气体对接触面
处薄膜的压力。
\begin{figure}[h!]
	\centering
	\includesvg[width=0.23\linewidth]{picture/svg/GZ-3-tiyou-1486}
\end{figure}

\banswer{
	$\frac{V_{0}}{V} P_{0} S$
}

	
\end{enumerate}


\item 
\exwhere{$ 2014 $ 年物理江苏卷}
一种海浪发电机的气室如图所示. 工作时,活塞随海浪上升或下降,改变气室中空气的压强,从而
驱动进气阀门和出气阀门打开或关闭. 气室先后经历吸入、
压缩和排出空气的过程,推动出气口处的装置发电. 气室中
的空气可视为理想气体.
\begin{enumerate}
	%\renewcommand{\labelenumi}{\arabic{enumi}.}
	% A(\Alph) a(\alph) I(\Roman) i(\roman) 1(\arabic)
	%设定全局标号series=example	%引用全局变量resume=example
	%[topsep=-0.3em,parsep=-0.3em,itemsep=-0.3em,partopsep=-0.3em]
	%可使用leftmargin调整列表环境左边的空白长度 [leftmargin=0em]
	\item
 下列对理想气体的理解,正确的有 \underlinegap .
 \begin{figure}[h!]
 	\centering
 	\includesvg[width=0.23\linewidth]{picture/svg/GZ-3-tiyou-1487}
 \end{figure}
 
\fourchoices
{理想气体实际上并不存在,只是一种理想模型}
{只要气体压强不是很高就可视为理想气体}
{一定质量的某种理想气体的内能与温度、体积都有关}
{在任何温度、任何压强下,理想气体都遵循气体实验定律}



\item 
压缩过程中,两个阀门均关闭。若此过程中,气室中的气体与外界无热量交换,内能增加了
$ 3.4 \times 10^{4} \ J $,则该气体的分子平均动能
 \underlinegap 
(选填“增大”、“减小”或“不变”),活塞对该气体所做的功 \underlinegap 
(选填“大于”、“小于”或“等于”)$ 3.4 \times 10^{4} \ J $。



\item 
上述过程中, 气体刚被压缩时的温度为 $ 27 \ \celsius $, 体积为 $ 0.224 \ m^{3} $, 压强为 $ 1 $ 个标准大气压。
已知 $ 1 \ mol $ 气体在 $ 1 $ 个标准大气压、$ 0 \ \celsius $时的体积为 $ 22.4 \ L $, 阿伏加德罗常数 $ N_{A}=6.02 \times 10^{23}  \ mol ^{-1} $.
计算此时气室中气体的分子数.( 计算结果保留一位有效数字)

\end{enumerate}

\banswer{
\begin{enumerate}
	%\renewcommand{\labelenumi}{\arabic{enumi}.}
	% A(\Alph) a(\alph) I(\Roman) i(\roman) 1(\arabic)
	%设定全局标号series=example	%引用全局变量resume=example
	%[topsep=-0.3em,parsep=-0.3em,itemsep=-0.3em,partopsep=-0.3em]
	%可使用leftmargin调整列表环境左边的空白长度 [leftmargin=0em]
	\item
	 AD   
	\item 
	增大 \quad 等于
	\item 
	$ 5 \times 10^{24} $(或 $ 6 \times 10^{24} $)
\end{enumerate}
}


\item 
\exwhere{$ 2014 $ 年理综山东卷}
\begin{enumerate}
	%\renewcommand{\labelenumi}{\arabic{enumi}.}
	% A(\Alph) a(\alph) I(\Roman) i(\roman) 1(\arabic)
	%设定全局标号series=example	%引用全局变量resume=example
	%[topsep=-0.3em,parsep=-0.3em,itemsep=-0.3em,partopsep=-0.3em]
	%可使用leftmargin调整列表环境左边的空白长度 [leftmargin=0em]
	\item
如图,内壁光滑、导热良好的气缸中用活塞封闭有一定质量的理
想气体。当环境温度升高时,缸内气体 \underlinegap 。(双选,填正
确答案标号)
\begin{figure}[h!]
	\centering
	\includesvg[width=0.23\linewidth]{picture/svg/GZ-3-tiyou-1488}
\end{figure}

\fourchoices
{内能增加}
{对外做功}
{压强增大}
{分子间的引力和斥力都增大}


 \tk{AB} 

\item 
一种水下重物打捞方法的工作原理如图所示。将一质量 $ M=3 \times 10^{3} \ kg $、体积 $ V_{0}=0.5 \ m^{3} $ 的重物捆
绑在开口朝下的浮筒上。向浮筒内充入一定量的气体,开始时筒
内液面到水面的距离 $ h_{1} =40 \ m $,筒内气体体积 $ V_{1} =1 \ m^{3} $。在拉力作
用下浮筒缓慢上升,当筒内液面到水面的距离为 $ h_{2} $ 时,拉力减为
零,此时气体体积为 $ V_{2} $,随后浮筒和重物自动上浮。求 $ V_{2} $ 和$ h_{2} $。




已知大气压强 $ p_{0} =1 \times 10^{5} \ Pa $,水的密度 $ \rho=1 \times 10^{3} \ kg/m^{3} $,重力加速
度的大小 $ g=10 \ m/s^{2} $。不计水温度变化,筒内气体质量不变且可视为理想气体,浮筒质量和筒壁厚
度可忽略。
\begin{figure}[h!]
	\flushright
	\includesvg[width=0.25\linewidth]{picture/svg/GZ-3-tiyou-1489}
\end{figure}

	
\banswer{
\begin{enumerate}
	%\renewcommand{\labelenumi}{\arabic{enumi}.}
	% A(\Alph) a(\alph) I(\Roman) i(\roman) 1(\arabic)
	%设定全局标号series=example	%引用全局变量resume=example
	%[topsep=-0.3em,parsep=-0.3em,itemsep=-0.3em,partopsep=-0.3em]
	%可使用leftmargin调整列表环境左边的空白长度 [leftmargin=0em]
	\item
	$ V_{1}=2.5 \ m^{3} \quad h_{2}=10 \ m $
\end{enumerate}
}

	
\end{enumerate}


\item 
\exwhere{$ 2014 $ 年物理海南卷}
\begin{enumerate}
	%\renewcommand{\labelenumi}{\arabic{enumi}.}
	% A(\Alph) a(\alph) I(\Roman) i(\roman) 1(\arabic)
	%设定全局标号series=example	%引用全局变量resume=example
	%[topsep=-0.3em,parsep=-0.3em,itemsep=-0.3em,partopsep=-0.3em]
	%可使用leftmargin调整列表环境左边的空白长度 [leftmargin=0em]
	\item
下列说法正确的是 \underlinegap .

\fivechoices
{液体表面张力的方向与液面垂直并指向液体内部}
{单晶体有固定的熔点,多晶体没有固定的熔点}
{单晶体中原子(或分子、离子)的排列具有空间周期性}
{通常金属在各个方向的物理性质都相同,所以金属是非晶体}
{液晶具有液体的流动性,同时具有晶体的各向异性特征}


 \tk{CE} 

\item 
一竖直放置、缸壁光滑且导热的柱形气缸内盛有一定量的氮气,被活塞分隔成 \lmd{1} 、
\lmd{2} 两部分;达到平衡时,这两部分气体的体积相等,上部气体的压
强为 $ p_{10} $,如图($ a $)所示,若将气缸缓慢倒置,再次达到平衡时,
上下两部分气体的体积之比为 $ 3:1 $,如图($ b $)所示。设外界温度不
变,已知活塞面积为 $ S $,重力加速度大小为 $ g $,求活塞的质量。
\begin{figure}[h!]
	\centering
\begin{subfigure}{0.4\linewidth}
	\centering
	\includesvg[width=0.7\linewidth]{picture/svg/GZ-3-tiyou-1490} 
	\caption{}\label{}
\end{subfigure}
\begin{subfigure}{0.4\linewidth}
	\centering
	\includesvg[width=0.7\linewidth]{picture/svg/GZ-3-tiyou-1491} 
	\caption{}\label{}
\end{subfigure}
\end{figure}

\banswer{
	$m=\frac{4 p_{10} S}{5 g}$
}

	
\end{enumerate}


\item 
\exwhere{$ 2014 $ 年理综福建卷}
\begin{enumerate}
	%\renewcommand{\labelenumi}{\arabic{enumi}.}
	% A(\Alph) a(\alph) I(\Roman) i(\roman) 1(\arabic)
	%设定全局标号series=example	%引用全局变量resume=example
	%[topsep=-0.3em,parsep=-0.3em,itemsep=-0.3em,partopsep=-0.3em]
	%可使用leftmargin调整列表环境左边的空白长度 [leftmargin=0em]
	\item
如图,横坐标 $ v $ 表示分子速率,纵坐标 $ f(v) $表示各等
间隔速率区间的分子数占总分子数的百分比。图中曲线能
正确表示某一温度下气体分子麦克斯韦速率分布规律的
是
 \underlinegap 
。(填选项前的字母)
\begin{figure}[h!]
	\centering
	\includesvg[width=0.23\linewidth]{picture/svg/GZ-3-tiyou-1492}
\end{figure}

\fourchoices
{曲线①}
{曲线②}
{曲线③}
{曲线④}

 \tk{D} 


\item 
如图为一定质量理想气体的压强 $ p $ 与体积 $ V $ 关系图像,它由状态 $ A $ 经等容过程到状态 $ B $,再
经等压过程到状态 $ C $,设 $ A $、$ B $、$ C $ 状态对应的温度分别为 $ T_{A} $、
$ T_{B} $、$ T_{C} $,则下列关系式中正确的是
 \underlinegap 
。
(填选项前的字母)
\begin{figure}[h!]
	\centering
	\includesvg[width=0.23\linewidth]{picture/svg/GZ-3-tiyou-1493}
\end{figure}

\fourchoices
{$ T_{A} < T_{B} $,$ T_{B} < T_{C} $}
{$ T_{A} > T_{B} $,$ T_{B} = T_{C} $}
{$ T_{A} > T_{B} $,$ T_{B} < T_{C} $}
{$ T_{A} = T_{B} $,$ T_{B} > T_{C} $}

 \tk{C} 


\end{enumerate}



\item 
\exwhere{$ 2013 $ 年新课标 \lmd{1} 卷}
\begin{enumerate}
	%\renewcommand{\labelenumi}{\arabic{enumi}.}
	% A(\Alph) a(\alph) I(\Roman) i(\roman) 1(\arabic)
	%设定全局标号series=example	%引用全局变量resume=example
	%[topsep=-0.3em,parsep=-0.3em,itemsep=-0.3em,partopsep=-0.3em]
	%可使用leftmargin调整列表环境左边的空白长度 [leftmargin=0em]
	\item
两个相距较远的分子仅在分子力作用下由静止开始运动,直至不再靠近。在此过程中,
下列说法正确的是 \underlinegap 
(填正确答案标号。选对 $ 1 $ 个得 $ 3 $ 分,选对 $ 2 $ 个得 $ 4 $ 分,选对 $ 3 $ 个得 $ 6 $ 分。
每选错 $ 1 $ 个扣 $ 3 $ 分,最低得分为 $ 0 $ 分)

\fourchoices
{分子力先增大,后一直减小}
{分子力先做正功,后做负功}
{分子动能先增大,后减小}
{分子势能先增大,后减小}

 \tk{BCE} 

\item 
如图,两个侧壁绝热、顶部和底部都导热的相同气缸直立放置,气缸底部和顶部均有细
管连通,顶部的细管带有阀门 $ K $。两气缸的容积均为 $ V_{0} $,气缸中各有一个绝热活塞(质量不同,厚
度可忽略)。开始时 $ K $ 关闭,两活塞下方和右活塞上方充有气体(可视为理想气体),压强分别为 $ p_{0} $
和 $ p_{0} /3 $;左活塞在气缸正中间,其上方为真空; 右活塞上方气体体积为
$ V_{0}/4 $。现使气缸底与一恒温热源接触,平衡后左活塞升至气缸顶部,且与
顶部刚好没有接触;然后打开 $ K $,经过一段时间,重新达到平衡。已知外
界温度为 $ T_{0} $,不计活塞与气缸壁间的摩擦。求:
\begin{enumerate}
	%\renewcommand{\labelenumi}{\arabic{enumi}.}
	% A(\Alph) a(\alph) I(\Roman) i(\roman) 1(\arabic)
	%设定全局标号series=example	%引用全局变量resume=example
	%[topsep=-0.3em,parsep=-0.3em,itemsep=-0.3em,partopsep=-0.3em]
	%可使用leftmargin调整列表环境左边的空白长度 [leftmargin=0em]
	\item
恒温热源的温度 $ T $;
\item 
重新达到平衡后左气缸中活塞上方气体的体积 $ V_{x} $。

	
\end{enumerate}
\begin{figure}[h!]
	\flushright
	\includesvg[width=0.25\linewidth]{picture/svg/GZ-3-tiyou-1494}
\end{figure}

\banswer{
	\begin{enumerate}
		%\renewcommand{\labelenumi}{\arabic{enumi}.}
		% A(\Alph) a(\alph) I(\Roman) i(\roman) 1(\arabic)
		%设定全局标号series=example	%引用全局变量resume=example
		%[topsep=-0.3em,parsep=-0.3em,itemsep=-0.3em,partopsep=-0.3em]
		%可使用leftmargin调整列表环境左边的空白长度 [leftmargin=0em]
		\item
		$T=\frac{7}{5} T_{0}$
		\item 
		\begin{equation}\label{key}
			6 V_{x}^{2}-V_{0} V_{x}-V_{0}^{2}=0
		\end{equation}
		其解为 $V_{x}=\frac{1}{2} V_{0}, \quad$ 另一个解 $V_{x}=-\frac{1}{3} V_{0},$ 不符合题意,舍去。
	\end{enumerate}
}


	
\end{enumerate}


\item 
\exwhere{$ 2013 $ 年新课标 \lmd{2} 卷}
\begin{enumerate}
	%\renewcommand{\labelenumi}{\arabic{enumi}.}
	% A(\Alph) a(\alph) I(\Roman) i(\roman) 1(\arabic)
	%设定全局标号series=example	%引用全局变量resume=example
	%[topsep=-0.3em,parsep=-0.3em,itemsep=-0.3em,partopsep=-0.3em]
	%可使用leftmargin调整列表环境左边的空白长度 [leftmargin=0em]
	\item
关于一定量的气体,下列说法正确的是
 \underlinegap 
(填正确答案标号。选对 $ 1 $ 个得 $ 2 $ 分,选对 $ 2 $ 个
得 $ 4 $ 分.选对 $ 3 $ 个得 $ 5 $ 分;每选错 $ I $ 个扣 $ 3 $ 分,最低得分为 $ 0 $ 分).


\fivechoices
{气体的体积指的是该气体的分子所能到达的空间的体积,而不是该气体所有分子体积之和}
{只要能减弱气体分子热运动的剧烈程度,气体的温度就可以降低}
{在完全失重的情况下,气体对容器壁的压强为零}
{气体从外界吸收热量,其内能一定增加}
{气体在等压膨胀过程中温度一定升高}

 \tk{ABE} 

\item 
如图,一上端开口、下端封闭的细长玻璃管竖直放置。玻璃管的
下部封有长 $ l_{1} =25.0 \ cm $ 的空气柱,中间有一段长为 $ l_{2} =25.0 \ cm $ 的水银柱,上
部空气柱的长度 $ l_{3} =40.0 \ cm $。已知大气压强为 $ p_{0} =75.0 \ cm Hg $。现将一活塞
(图中未画出)从玻璃管开口处缓慢往下推,使管下部空气柱长度变为
$ l_{1} ^{\prime} =20.0 \ cm $。假设活塞下推过程中没有漏气,求活塞下推的距离。
\begin{figure}[h!]
	\flushright
	\includesvg[width=0.25\linewidth]{picture/svg/GZ-3-tiyou-1496}
\end{figure}


\banswer{
	$ \Delta l =15.0 \ cm $
}



\end{enumerate}


\item 
\exwhere{$ 2013 $ 年重庆卷}
 \begin{enumerate}
 	%\renewcommand{\labelenumi}{\arabic{enumi}.}
 	% A(\Alph) a(\alph) I(\Roman) i(\roman) 1(\arabic)
 	%设定全局标号series=example	%引用全局变量resume=example
 	%[topsep=-0.3em,parsep=-0.3em,itemsep=-0.3em,partopsep=-0.3em]
 	%可使用leftmargin调整列表环境左边的空白长度 [leftmargin=0em]
 	\item
某未密闭房间的空气温度与室外的相同,现对该室内空气缓慢加热,当室内空气温度高
于室外空气温度时 \xzanswer{B} 

\fourchoices
{室内空气的压强比室外的小}
{室内空气分子的平均动能比室外的大}
{室内空气的密度比室外大}
{室内空气对室外空气做了负功}

\item 
汽车未装载货物时,某个轮胎内气体的体积为 $ V_{0} $,压强为 $ P_{0} $;装载货物后,该轮胎内
气体的压强增加了$ \Delta P $。若轮胎内气体视为理想气体,其质量、温度在装载货物前后均不变,求装
载货物前后此轮胎内气体体积的变化量。

\banswer{
	$\Delta V=\frac{-\Delta p V_{0}}{p_{0}+\Delta p}$
}


\end{enumerate}


\item
\exwhere{$ 2013 $年江苏卷}
如图所示,一定质量的理想气体从状态$ A $ 依次经过状态$ B $、$ C $ 和$ D $ 后再回到状态A. 其中,$ A \rightarrow B $ 和$ C \rightarrow D $
为等温过程,$ B \rightarrow C $ 和$ D \rightarrow A $ 为绝热过程(气体与外界无热量交换). 这就是著名的“卡诺循环”.
\begin{enumerate}
	%\renewcommand{\labelenumi}{\arabic{enumi}.}
	% A(\Alph) a(\alph) I(\Roman) i(\roman) 1(\arabic)
	%设定全局标号series=example	%引用全局变量resume=example
	%[topsep=-0.3em,parsep=-0.3em,itemsep=-0.3em,partopsep=-0.3em]
	%可使用leftmargin调整列表环境左边的空白长度 [leftmargin=0em]
	\item
该循环过程中,下列说法正确的是 \underlinegap .
\begin{figure}[h!]
	\centering
	\includesvg[width=0.23\linewidth]{picture/svg/GZ-3-tiyou-1497}
\end{figure}

\fourchoices
{$A \rightarrow B $ 过程中,外界对气体做功}
{$B \rightarrow C $ 过程中,气体分子的平均动能增大}
{$C \rightarrow D $ 过程中,单位时间内碰撞单位面积器壁的分子数增多}
{$D \rightarrow A $ 过程中,气体分子的速率分布曲线不发生变化}

\item 
该循环过程中,内能减小的过程是 \underlinegap  (选填“$ A \rightarrow B $”、“$ B \rightarrow C $”、“$ C \rightarrow D $”或“$ D \rightarrow A $”). 若气体在
$ A \rightarrow B $ 过程中吸收$ 63 \ kJ $ 的热量,在$ C \rightarrow D $ 过程中放出$ 38 \ kJ $ 的热量,则气体完成一次循环对外做的功为 \underlinegap $ kJ $.

\item 
若该循环过程中的气体为$ 1 \ mol $,气体在$ A $ 状态时的体积为$ 10 \ L $,在$ B $ 状态时压强为$ A $状态时的$  \frac{ 2 }{ 3 } 	 $.
气体在$ B $状态时单位体积内的分子数.( 已知阿伏加德罗常数 $ NA=6.0 \times 10^{23} mol^{-1} $,计算结果保留一
位有效数字)





\end{enumerate}

\banswer{
\begin{enumerate}
	%\renewcommand{\labelenumi}{\arabic{enumi}.}
	% A(\Alph) a(\alph) I(\Roman) i(\roman) 1(\arabic)
	%设定全局标号series=example	%引用全局变量resume=example
	%[topsep=-0.3em,parsep=-0.3em,itemsep=-0.3em,partopsep=-0.3em]
	%可使用leftmargin调整列表环境左边的空白长度 [leftmargin=0em]
	\item
	C
	\item 
	$ B \rightarrow C $
	\item 
	$ n=4 \times 10^{25} \ m^{3} $
\end{enumerate}
}



\item 
\exwhere{$ 2013 $ 年山东卷}
\begin{enumerate}
	%\renewcommand{\labelenumi}{\arabic{enumi}.}
	% A(\Alph) a(\alph) I(\Roman) i(\roman) 1(\arabic)
	%设定全局标号series=example	%引用全局变量resume=example
	%[topsep=-0.3em,parsep=-0.3em,itemsep=-0.3em,partopsep=-0.3em]
	%可使用leftmargin调整列表环境左边的空白长度 [leftmargin=0em]
	\item
下列关于热现象的描述正确的一项是 \xzanswer{C} 


\fourchoices
{根据热力学定律,热机的效率可能达到 $ 100 \% $}
{做功和热传递都是通过能量转化的方式改变系统内能的}
{温度是描述热运动的物理量,一个系统与另一个系统达到热平衡时两系统温度相同}
{物体由大量分子组成,其单个分子的运动是无规则的,大量分子的运动也是无规律的}



\item 
我国“蛟龙”号深海探测船载人下潜超过七千米,再创载人深潜新纪录。
在某次深潜实验中,“蛟龙”号探测到 $ 990 \ m $ 深处的海水温度为 $ 280 \ K $。某同学
利用该数据来研究气体状态随海水深度的变化。如图所示,导热良好的气缸
内封闭一定质量的气体,不计活塞的质量和摩擦,气缸所处海平面的温度 $ T_{0}  =300 \ K $,压强 $ p_{0} =1 \ atm $,封闭气体的体积 $ V_{0}=3 \ m^{3} $。如果将该气缸下潜至
$ 990 \ m $ 深处,此过程中封闭气体可视为理想气体。
\begin{enumerate}
	%\renewcommand{\labelenumi}{\arabic{enumi}.}
	% A(\Alph) a(\alph) I(\Roman) i(\roman) 1(\arabic)
	%设定全局标号series=example	%引用全局变量resume=example
	%[topsep=-0.3em,parsep=-0.3em,itemsep=-0.3em,partopsep=-0.3em]
	%可使用leftmargin调整列表环境左边的空白长度 [leftmargin=0em]
	\item
求 $ 990 \ m $ 深处封闭气体的体积($ 1 \ atm $ 相当于 $ 10 \ m $ 深的海水产生的压强)。
\item 
下潜过程中封闭气体 \underlinegap (填“吸热”或“放热”),传递的热量 \underlinegap (填“大于”或“小于”)外界对气体所做的功。

\end{enumerate}
\begin{figure}[h!]
	\flushright
	\includesvg[width=0.25\linewidth]{picture/svg/GZ-3-tiyou-1498}
\end{figure}

\banswer{
	\begin{enumerate}
		%\renewcommand{\labelenumi}{\arabic{enumi}.}
		% A(\Alph) a(\alph) I(\Roman) i(\roman) 1(\arabic)
		%设定全局标号series=example	%引用全局变量resume=example
		%[topsep=-0.3em,parsep=-0.3em,itemsep=-0.3em,partopsep=-0.3em]
		%可使用leftmargin调整列表环境左边的空白长度 [leftmargin=0em]
		\item
		$ 2.8 \times 10^{-2} \ m^{3} $
		\item 
		放热 \quad 大于
	\end{enumerate}
}

	
\end{enumerate}





\item
\exwhere{$ 2013 $ 年海南卷}
\begin{enumerate}
	%\renewcommand{\labelenumi}{\arabic{enumi}.}
	% A(\Alph) a(\alph) I(\Roman) i(\roman) 1(\arabic)
	%设定全局标号series=example	%引用全局变量resume=example
	%[topsep=-0.3em,parsep=-0.3em,itemsep=-0.3em,partopsep=-0.3em]
	%可使用leftmargin调整列表环境左边的空白长度 [leftmargin=0em]
	\item
下列说法正确的是
 \underlinegap 
(填正确答案标号。选对 $ 1 $ 个得 $ 2 $ 分,选对 $ 2 $ 个得 $ 3 $ 分.选对 $ 3 $
个得 $ 4 $ 分;每选错 $ I $ 个扣 $ 2 $ 分,最低得分为 $ 0 $ 分)



\fivechoices
{把一枚针轻放在水面上,它会浮在水面,这是由于水表面存在表面张力的缘故}
{水在涂有油脂的玻璃板上能形成水珠,而在干净的玻璃板上却不能,这是因为油脂使水的表面张力增大的缘故}
{在围绕地球飞行的宇宙飞船中,自由飘浮的水滴呈球形,这是表面张力作用的结果}
{在毛细现象中,毛细管中的液面有的升高,有的降低,这与液体的种类和毛细管的材质有关}
{当两薄玻璃板间夹有一层水膜时,在垂直于玻璃板的方向很难将玻璃板拉开,这是由于水膜具有表面张力的缘故}

 \tk{ACD} 



\item 
如图,一带有活塞的气缸通过底部的水平细管与一个上端开
口的竖直管相连,气缸与竖直管的横截面面积之比为 $ 3 : 1 $,初始时,该装
置的底部盛有水银;活塞与水银面之间封有一定量的气体,气柱高度为 $ l $
(以 $ cm $ 为单位)
;竖直管内的水银面比气缸内的水银面高出 $ 3l/8 $。现使活
塞缓慢向上移动 $ 11l/32 $,这时气缸和竖直管内的水银面位于同一水平面
上,求初始时气缸内气体的压强(以 $ cmHg $ 为单位)。
\begin{figure}[h!]
	\flushright
	\includesvg[width=0.25\linewidth]{picture/svg/GZ-3-tiyou-1499}
\end{figure}


\banswer{
	$p=\frac{15}{8} l$
}


	
\end{enumerate}

\item 
\exwhere{$ 2013 $ 年福建卷}
\begin{enumerate}
	%\renewcommand{\labelenumi}{\arabic{enumi}.}
	% A(\Alph) a(\alph) I(\Roman) i(\roman) 1(\arabic)
	%设定全局标号series=example	%引用全局变量resume=example
	%[topsep=-0.3em,parsep=-0.3em,itemsep=-0.3em,partopsep=-0.3em]
	%可使用leftmargin调整列表环境左边的空白长度 [leftmargin=0em]
	\item
下列四幅图中,能正确反映分子间作用力 $ f $ 和分子势能 $ E_{P} $ 随分子间距离 $ r $ 变化关系的图线是 \underlinegap 。
\pfourchoices
{\includesvg[width=4.3cm]{picture/svg/GZ-3-tiyou-1500}}
{\includesvg[width=4.3cm]{picture/svg/GZ-3-tiyou-1501}}
{\includesvg[width=4.3cm]{picture/svg/GZ-3-tiyou-1502}}
{\includesvg[width=4.3cm]{picture/svg/GZ-3-tiyou-1505}}



\item 
某自行车轮胎的容积为 $ V $.里面已有压强为 $ p_{0} $ 的空气,现在要使轮胎内的气压增大到 $ p $,设充气
过程为等温过程,空气可看作理想气体,轮胎容积保持不变,则还要向轮胎充入温度相同,压强
也是 $ p_{0} $,体积为
 \underlinegap 
的空气。(填选项前的字母)

\fourchoices
{$\frac{p_{0}}{p} V$}
{$\frac{p}{p_{0}} V$}
{$\left(\frac{p}{p_{0}}-1\right) V$}
{$\left(\frac{p}{p_{0}}+1\right) V$}
	
 \tk{C} 
	
\end{enumerate}


\item 
\exwhere{$ 2012 $ 年理综新课标卷}
\begin{enumerate}
	%\renewcommand{\labelenumi}{\arabic{enumi}.}
	% A(\Alph) a(\alph) I(\Roman) i(\roman) 1(\arabic)
	%设定全局标号series=example	%引用全局变量resume=example
	%[topsep=-0.3em,parsep=-0.3em,itemsep=-0.3em,partopsep=-0.3em]
	%可使用leftmargin调整列表环境左边的空白长度 [leftmargin=0em]
	\item
关于热力学定律,下列说法正确的是 \underlinegap (填入正确选项前的字母,选对 $ 1 $ 个
给 $ 3 $ 分,选对 $ 2 $ 个给 $ 4 $ 分,选对 $ 3 $ 个给 $ 6 $ 分,每选错 $ 1 $ 个扣 $ 3 $ 分,最低得分为 $ 0 $ 分)。


\fivechoices
{为了增加物体的内能,必须对物体做功或向它传递热量}
{对某物体做功,必定会使该物体的内能增加}
{可以从单一热源吸收热量,使之完全变为功}
{不可能使热量从低温物体传向高温物体}
{功转变为热的实际宏观过程是不可逆过程}


 \tk{ACE} 

\item 
如图,由 $ U $ 形管和细管连接的玻璃泡 $ A $、 $ B $
和 $ C $ 浸泡在温度均为 $ 0 \ \celsius $ 的水槽中,$ B $ 的容积是 $ A $ 的 $ 3 $
倍。阀门 $ S $ 将 $ A $ 和 $ B $ 两部分隔开。$ A $ 内为真空,$ B $ 和 $ C $ 内
都充有气体。$ U $ 形管内左边水银柱比右边的低 $ 60 \ mm $。打
开阀门 $ S $,整个系统稳定后,$ U $ 形管内左右水银柱高度相
等。假设 $ U $ 形管和细管中的气体体积远小于玻璃泡的容积。
\begin{enumerate}
	%\renewcommand{\labelenumi}{\arabic{enumi}.}
	% A(\Alph) a(\alph) I(\Roman) i(\roman) 1(\arabic)
	%设定全局标号series=example	%引用全局变量resume=example
	%[topsep=-0.3em,parsep=-0.3em,itemsep=-0.3em,partopsep=-0.3em]
	%可使用leftmargin调整列表环境左边的空白长度 [leftmargin=0em]
	\item
求玻璃泡 $ C $ 中气体的压强(以 $ mmHg $ 为单位);
\item 
将右侧水槽的水从 $ 0 \ \celsius $ 加热到一定温度时,$ U $ 形管内左右水银柱高度差又为 $ 60 \ mm $,求加热
后右侧水槽的水温。
	
\end{enumerate}
\begin{figure}[h!]
	\flushright
	\includesvg[width=0.25\linewidth]{picture/svg/GZ-3-tiyou-1507}
\end{figure}

\banswer{
	\begin{enumerate}
		%\renewcommand{\labelenumi}{\arabic{enumi}.}
		% A(\Alph) a(\alph) I(\Roman) i(\roman) 1(\arabic)
		%设定全局标号series=example	%引用全局变量resume=example
		%[topsep=-0.3em,parsep=-0.3em,itemsep=-0.3em,partopsep=-0.3em]
		%可使用leftmargin调整列表环境左边的空白长度 [leftmargin=0em]
		\item
		$ 180 \ mm Hg $
		\item 
		$ 364 \ K $	
	\end{enumerate}
}



\end{enumerate}



\item 
\exwhere{$ 2012 $年理综山东卷}
\begin{enumerate}
	%\renewcommand{\labelenumi}{\arabic{enumi}.}
	% A(\Alph) a(\alph) I(\Roman) i(\roman) 1(\arabic)
	%设定全局标号series=example	%引用全局变量resume=example
	%[topsep=-0.3em,parsep=-0.3em,itemsep=-0.3em,partopsep=-0.3em]
	%可使用leftmargin调整列表环境左边的空白长度 [leftmargin=0em]
	\item
以下说法正确的是
 \underlinegap 。

\fourchoices
{水的饱和汽压随温度的升高而增大}
{扩散现象表明,分子在永不停息地运动}
{当分子间距离增大时,分子间引力增大,分子间斥力减小}
{一定质量的理想气体,在等压膨胀过程中,气体分子的平均动能减小}

 \tk{AB} 

\item 
如图所示,粗细均匀、导热良好、装有适量水银的$ U $型管竖直放置,右端与大
气相通,左端封闭气柱长$ l_{1} =20 \ cm $(可视为理想气体)
,两管中水银面等高。先将右
端与一低压舱(未画出)接通,稳定后右管水银面高出左管水银面$ h=10 \ cm $(环境温
度不变,大气压强$ P_{0} =75 \ cm Hg $)。
\begin{enumerate}
	%\renewcommand{\labelenumi}{\arabic{enumi}.}
	% A(\Alph) a(\alph) I(\Roman) i(\roman) 1(\arabic)
	%设定全局标号series=example	%引用全局变量resume=example
	%[topsep=-0.3em,parsep=-0.3em,itemsep=-0.3em,partopsep=-0.3em]
	%可使用leftmargin调整列表环境左边的空白长度 [leftmargin=0em]
	\item
求稳定后低压舱内的压强(用“$ cmHg $”做单位);
\item 
此过程中左管内的气体对外界 \underlinegap (填“做正功”“做负功”“不做功”)
,气体将 \underlinegap (填“吸热”或放热“)。
\end{enumerate}

\banswer{
	\begin{enumerate}
		%\renewcommand{\labelenumi}{\arabic{enumi}.}
		% A(\Alph) a(\alph) I(\Roman) i(\roman) 1(\arabic)
		%设定全局标号series=example	%引用全局变量resume=example
		%[topsep=-0.3em,parsep=-0.3em,itemsep=-0.3em,partopsep=-0.3em]
		%可使用leftmargin调整列表环境左边的空白长度 [leftmargin=0em]
		\item
		$ 50 \ cm Hg $
		\item 
		作正功 \quad 吸热
	\end{enumerate}
}


\end{enumerate}

\item 
\exwhere{$ 2012 $ 年物理江苏卷}
\begin{enumerate}
	%\renewcommand{\labelenumi}{\arabic{enumi}.}
	% A(\Alph) a(\alph) I(\Roman) i(\roman) 1(\arabic)
	%设定全局标号series=example	%引用全局变量resume=example
	%[topsep=-0.3em,parsep=-0.3em,itemsep=-0.3em,partopsep=-0.3em]
	%可使用leftmargin调整列表环境左边的空白长度 [leftmargin=0em]
	\item
下列现象中,能说明液体存在表面张力的有 \underlinegap .
\fourchoices
{水黾可以停在水面上}
{叶面上的露珠呈球形}
{滴入水中的红墨水很快散开}
{悬浮在水中的花粉做无规则运动}

 \tk{AB} 


\item 
密闭在钢瓶中的理想气体,温度升高时压强增大. 从分子动理论的角度分析,这是由于分子热运动
的 \underlinegap 增大了. 该气体在温度 $ T_{1} $、$ T_{2} $ 时的分子速率分布图象如图所示,则
$ T_{1} $ \underlinegap $ T_{2} $(选填“大于”或“小于”).
\begin{figure}[h!]
	\centering
	\includesvg[width=0.23\linewidth]{picture/svg/GZ-3-tiyou-1508}
\end{figure}

 \tk{平均动能 \quad 小于} 

\item 
如图所示,一定质量的理想气体从状态 $ A $ 经等压过程到状态 B. 此过程中,气体压强
$ p=1.0 \times 10^{5} \ Pa $,吸收的热量 $ Q=7.0 \times 10^{2} \ J $,求此过程中气体内能的增量.
\begin{figure}[h!]
	\flushright
	\includesvg[width=0.25\linewidth]{picture/svg/GZ-3-tiyou-1509}
\end{figure}

\banswer{
	等压变化 $\frac{V_{A}}{T_{A}}=\frac{V_{B}}{T_{B}},$ 对外做的功 $W=p\left(V_{B}-V_{A}\right)$,根据热力学第一定律 $\Delta U=Q-W$,解得
	$\Delta U=5.0 \times 10^{2} \ J$
}


\end{enumerate}


\item 
\exwhere{$ 2012 $ 年理综福建卷}
\begin{enumerate}
	%\renewcommand{\labelenumi}{\arabic{enumi}.}
	% A(\Alph) a(\alph) I(\Roman) i(\roman) 1(\arabic)
	%设定全局标号series=example	%引用全局变量resume=example
	%[topsep=-0.3em,parsep=-0.3em,itemsep=-0.3em,partopsep=-0.3em]
	%可使用leftmargin调整列表环境左边的空白长度 [leftmargin=0em]
	\item
关于热力学定律和分子动理论,下列说法正确的是 \underlinegap 。
(填选项前的字母)
\fourchoices
{一定量气体吸收热量,其内能一定增大}
{不可能使热量由低温物体传递到高温物体}
{若两分子间距离增大,分子势能一定增大}
{若两分子间距离减小,分子间引力和斥力都增大}

 \tk{D} 


\item 
空气压缩机的储气罐中储有 $ 1.0 \ atm $ 的空气 $ 6.0 \ L $,现再冲入 $ 1.0 \ atm $ 的空气 $ 9.0 \ L $。设充气过程
为等温过程,空气可看作理想气体,则充气后储气罐中气体压强为 \underlinegap 。(填选项前的字母)
\fourchoices
{$ 2.5 \ atm $}
{$ 2.0 \ atm $}
{$ 1.5 \ atm $}
{$ 1.0 \ atm $}

 \tk{A} 


\end{enumerate}



\item 
\exwhere{$ 2012 $ 年物理海南卷}
\begin{enumerate}
	%\renewcommand{\labelenumi}{\arabic{enumi}.}
	% A(\Alph) a(\alph) I(\Roman) i(\roman) 1(\arabic)
	%设定全局标号series=example	%引用全局变量resume=example
	%[topsep=-0.3em,parsep=-0.3em,itemsep=-0.3em,partopsep=-0.3em]
	%可使用leftmargin调整列表环境左边的空白长度 [leftmargin=0em]
	\item
两分子间的斥力和引力的合力 $ F $ 与分子间距离 $ r $ 的关系如图中曲线所示,曲线与 $ r $ 轴
交点的横坐标为 $ r_{0} $。相距很远的两分子在分子力作用下,由静止开始相互接近。若两分子相距无穷
远时分子势能为零,下列说法正确的是 \underlinegap (填入正确选项
前的字母。选对 $ 1 $ 个给 $ 2 $ 分,选对 $ 2 $ 个给 $ 3 $ 分,选对 $ 3 $ 个给 $ 4 $ 分;每
选错 $ 1 $ 个扣 $ 2 $ 分,最低得分为 $ 0 $ 分)。
\begin{figure}[h!]
	\centering
	\includesvg[width=0.23\linewidth]{picture/svg/GZ-3-tiyou-1510}
\end{figure}

\fourchoices
{在 $ r>r_0 $ 阶段,$ F $ 做正功,分子动能增加,势能减小}
{在 $ r<r_0 $ 阶段,$ F $ 做负功,分子动能减小,势能减小}
{在 $ r=r_0 $ 时,分子势能最小,动能最大}
{在 $ r=r_0 $ 时,分子势能为零}

 \tk{ACE} 

\item 
如图,一气缸水平固定在静止的小车上,一质量为 $ m $、面积为 $ S $ 的活塞将一定量的气
体封闭在气缸内,平衡时活塞与气缸底相距 $ L $。现让小车以一较小的
水平恒定加速度向右运动,稳定时发现活塞相对于气缸移动了距离
$ d $。已知大气压强为 $ p_{0} $,不计气缸和活塞间的摩擦;且小车运动时,大
气对活塞的压强仍可视为 $ p_{0} $;整个过程中温度保持不变。求小车加速度的大小。
\begin{figure}[h!]
	\flushright
	\includesvg[width=0.25\linewidth]{picture/svg/GZ-3-tiyou-1512}
\end{figure}


\banswer{
	$a=\frac{p_{0} S d}{m(L-d)}$
}



\end{enumerate}


\item 
\exwhere{$ 2011 $ 年新课标卷}
 \begin{enumerate}
 	%\renewcommand{\labelenumi}{\arabic{enumi}.}
 	% A(\Alph) a(\alph) I(\Roman) i(\roman) 1(\arabic)
 	%设定全局标号series=example	%引用全局变量resume=example
 	%[topsep=-0.3em,parsep=-0.3em,itemsep=-0.3em,partopsep=-0.3em]
 	%可使用leftmargin调整列表环境左边的空白长度 [leftmargin=0em]
 	\item
 对于一定量的理想气体,下列说法正确的是 \underlinegap 。
(选对一个给 $ 3 $ 分,选对两个给 $ 4 $
分,选对 $ 3 $ 个给 $ 6 $ 分。每选错一个扣 $ 3 $ 分,最低得分为 $ 0 $ 分)


\fivechoices
{若气体的压强和体积都不变,其内能也一定不变}
{若气体的内能不变,其状态也一定不变}
{若气体的温度随时间不断升高,其压强也一定不断增大}
{气体温度每升高 $ 1 \ K $ 所吸收的热量与气体经历的过程有关}
{当气体温度升高时,气体的内能一定增大}

 \tk{ADE} 


\item 
如图,一上端开口,下端封闭的细长玻璃管,下部有长 $ l_{1} =66 \ cm $ 的水银柱,中
间封有长 $ l_{2} =6.6 \ cm $ 的空气柱,上部有长 $ l_{3} =44 \ cm $ 的水银柱,此时水银面恰好与管口平齐。
已知大气压强为 $ p_{0} =76 \ cm Hg $。如果使玻璃管绕底端在竖直平面内缓慢地转动一周,求在开
口向下和转回到原来位置时管中空气柱的长度。封入的气体可视为理想气体,在转动过程中没有
发生漏气。
\begin{figure}[h!]
	\flushright
	\includesvg[width=0.05\linewidth]{picture/svg/GZ-3-tiyou-1513}
\end{figure}


\banswer{
	向下时,$ l=12 \ cm $;转回原来位置时,$ l=9.2 \ cm $
}


 \end{enumerate}


\item 
\exwhere{$ 2011 $ 年理综福建卷}
\begin{enumerate}
	%\renewcommand{\labelenumi}{\arabic{enumi}.}
	% A(\Alph) a(\alph) I(\Roman) i(\roman) 1(\arabic)
	%设定全局标号series=example	%引用全局变量resume=example
	%[topsep=-0.3em,parsep=-0.3em,itemsep=-0.3em,partopsep=-0.3em]
	%可使用leftmargin调整列表环境左边的空白长度 [leftmargin=0em]
	\item
如图所示,曲线 $ M $、$ N $ 分别表示晶体和非晶体在一定压强下的熔化过程,图
中横轴表示时间 $ t $,纵轴表示温度 $ T $。从图中可以确定的是 \underlinegap 。(填选项
前的字母)
\begin{figure}[h!]
	\centering
	\includesvg[width=0.23\linewidth]{picture/svg/GZ-3-tiyou-1514}
\end{figure}

\fourchoices
{晶体和非晶体均存在固定的熔点 $ T_{0} $}
{曲线 $ M $ 的 $ bc $ 段表示固液共存状态}
{曲线 $ M $ 的 $ ab $ 段、曲线 $ N $ 的 $ ef $ 段均表示固态}
{曲线 $ M $ 的 $ cd $ 段、曲线 $ N $ 的 $ fg $ 段均表示液态}

 \tk{B} 


\item 
一定量的理想气体在某一过程中,从外界吸收热量 $ 2.5 \times 10^{4} \ J $,气体对外界做功 $ 1.0 \times 10^{4} \ J $,则该理
想气体的 \underlinegap 。(填选项前的字母)
\fourchoices
{温度降低,密度增大}
{温度降低,密度减小}
{温度升高,密度增大}
{温度升高,密度减小}

 \tk{D} 

\end{enumerate}


\item 
\exwhere{$ 2011 $ 年海南卷}
\begin{enumerate}
	%\renewcommand{\labelenumi}{\arabic{enumi}.}
	% A(\Alph) a(\alph) I(\Roman) i(\roman) 1(\arabic)
	%设定全局标号series=example	%引用全局变量resume=example
	%[topsep=-0.3em,parsep=-0.3em,itemsep=-0.3em,partopsep=-0.3em]
	%可使用leftmargin调整列表环境左边的空白长度 [leftmargin=0em]
	\item
关于空气湿度,下列说法正确的是 \underlinegap 
(填入正确选项前的字母。选对 $ 1 $ 个给 $ 2 $
分,选对 $ 2 $ 个给 $ 4 $ 分;选错 $ 1 $ 个扣 $ 2 $ 分,最低得 $ 0 $ 分)。

\fourchoices
{当人们感到潮湿时,空气的绝对湿度一定较大}
{当人们感到干燥时,空气的相对湿度一定较小}
{空气的绝对湿度用空气中所含水蒸汽的压强表示}
{空气的相对湿度定义为水的饱和蒸汽压与相同温度时空气中所含水蒸气的压强之比}

 \tk{BC} 

\item 
如图,容积为 $ V_{1} $ 的容器内充有压缩空
气。容器与水银压强计相连,压强计左右两管下部
由软胶管相连。气阀关闭时,两管中水银面等高,
左管中水银面上方到气阀之间空气的体积为 $ V_{2} $。打开气阀,左管中水银下降;缓慢地向上提右
管,使左管中水银面回到原来高度,此时右管与左管中水银面的高度差为 $ h $。已知水银的密度为
$ \rho $,大气压强为 $ p_{0} $,重力加速度为 $ g $;空气可视为理想气体,其温度不变。求气阀打开前容器中压
缩空气的压强 $ p_{1} $。
\begin{figure}[h!]
	\flushright
	\includesvg[width=0.25\linewidth]{picture/svg/GZ-3-tiyou-1515}
\end{figure}

\banswer{
	$p_{1}=\frac{\rho g h\left(V_{1}+V_{2}\right)+p_{0} V_{1}}{V_{1}}$
}

	
\end{enumerate}

\item 
\exwhere{$ 2011 $ 年理综山东卷}
\begin{enumerate}
	%\renewcommand{\labelenumi}{\arabic{enumi}.}
	% A(\Alph) a(\alph) I(\Roman) i(\roman) 1(\arabic)
	%设定全局标号series=example	%引用全局变量resume=example
	%[topsep=-0.3em,parsep=-0.3em,itemsep=-0.3em,partopsep=-0.3em]
	%可使用leftmargin调整列表环境左边的空白长度 [leftmargin=0em]
	\item
人类对物质属性的认识是从宏观到微观不断深入的过程。以下说法正确的是 \underlinegap 。	

\fourchoices
{液体的分子势能与体积有关}
{晶体的物理性质都是各向异性的}
{温度升高,每个分子的动能都增大}
{露珠呈球状是由于液体表面张力的作用}

 \tk{AD} 

\item 
气体温度计结构如图所示。玻璃测温泡 $ A $ 内充有理想气体,通过细玻璃管 $ B $
和水银压强计相连。开始时 $ A $ 处于冰水混合物中,左管 $ C $ 中水银面在 $ O $ 点
处,右管 $ D $ 中水银面高出 $ O $ 点 $ h_{1} =14 \ cm $。后将 $ A $ 放入待测恒温槽中,上下移
动 $ D $,使 $ C $ 中水银面仍在 $ O $ 点处,测得 $ D $ 中水银面高出 $ O $ 点 $ h_{2} =44 \ cm $。(已知
外界大气压为 $ 1 $ 个标准大气压,$ 1 $ 标准大气压相当于 $ 76 \ cm Hg $)
\begin{enumerate}
	%\renewcommand{\labelenumi}{\arabic{enumi}.}
	% A(\Alph) a(\alph) I(\Roman) i(\roman) 1(\arabic)
	%设定全局标号series=example	%引用全局变量resume=example
	%[topsep=-0.3em,parsep=-0.3em,itemsep=-0.3em,partopsep=-0.3em]
	%可使用leftmargin调整列表环境左边的空白长度 [leftmargin=0em]
	\item
求恒温槽的温度。
\item 
此过程 $ A $ 内气体内能
 \underlinegap 
(填“增大”或“减小”),气体不对外做功,气体将
 \underlinegap 
(填“吸热”或“放热”)。
\end{enumerate}
\begin{figure}[h!]
	\flushright
	\includesvg[width=0.25\linewidth]{picture/svg/GZ-3-tiyou-1516}
\end{figure}

\banswer{
	\begin{enumerate}
		%\renewcommand{\labelenumi}{\arabic{enumi}.}
		% A(\Alph) a(\alph) I(\Roman) i(\roman) 1(\arabic)
		%设定全局标号series=example	%引用全局变量resume=example
		%[topsep=-0.3em,parsep=-0.3em,itemsep=-0.3em,partopsep=-0.3em]
		%可使用leftmargin调整列表环境左边的空白长度 [leftmargin=0em]
		\item
		$ 364 \ K $
		\item 
		增大 \quad 吸热
	\end{enumerate}	
}


\end{enumerate}

\item 
\exwhere{$ 2011 $ 年江苏卷}
\begin{enumerate}
	%\renewcommand{\labelenumi}{\arabic{enumi}.}
	% A(\Alph) a(\alph) I(\Roman) i(\roman) 1(\arabic)
	%设定全局标号series=example	%引用全局变量resume=example
	%[topsep=-0.3em,parsep=-0.3em,itemsep=-0.3em,partopsep=-0.3em]
	%可使用leftmargin调整列表环境左边的空白长度 [leftmargin=0em]
	\item
如图所示,一演示用的“永动机”转轮由 $ 5 $ 根轻杆和转轴构成,轻杆的末端装有形状记忆合金
制成的叶片。轻推转轮后,进入热水的叶片因伸展面“划水”,推动转轮转动。离开热水后,叶片形
状迅速恢复,转轮因此能较长时间转动。下列说法正确的是 \underlinegap 。
\begin{figure}[h!]
	\centering
	\includesvg[width=0.23\linewidth]{picture/svg/GZ-3-tiyou-1517}
\end{figure}

\fourchoices
{转轮依靠自身惯性转动,不需要消耗外界能量}
{转轮转动所需能量来自形状记忆合金自身}
{转动的叶片不断搅动热水,水温升高}
{叶片在热水中吸收的热量一定大于在空气中释放的热量}


 \tk{D} 


\item 
如图所示,内壁光滑的气缸水平放置。一定质量的理想气体被密封在气缸
内,外界大气压强为 $ p_{0} $。现对气缸缓慢加热,气体吸收热量 $ Q $ 后,体积由 $ V_{1} $ 增
大为 $ V_{2} $。则在此过程中,气体分子平均动能 \underlinegap (选增“增大”、“不变”或“减小”),气体内能变化了 \underlinegap 。
\begin{figure}[h!]
	\centering
	\includesvg[width=0.23\linewidth]{picture/svg/GZ-3-tiyou-1518}
\end{figure}

 \tk{增大; $ Q- p_{0} ( V_{2} - V_{1} ) $} 


\end{enumerate}





	
	
	
\end{enumerate}

