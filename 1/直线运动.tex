\bta{直线运动}


\begin{enumerate}[leftmargin=0em]
\renewcommand{\labelenumi}{\arabic{enumi}.}
% A(\Alph) a(\alph) I(\Roman) i(\roman) 1(\arabic)
%设定全局标号series=example	%引用全局变量resume=example
%[topsep=-0.3em,parsep=-0.3em,itemsep=-0.3em,partopsep=-0.3em]
%可使用leftmargin调整列表环境左边的空白长度 [leftmargin=0em]
\item
\exwhere{$ 2018 $年浙江卷($ 4 $月选考)}
某驾驶员使用定速巡航,在高速公路上以时速$ 110 $公里行驶了$ 200 $公里。其中“时速$ 110 $公里”、“行驶$ 200 $公里”分别是指 \xzanswer{D} 


\fourchoices
{速度、位移 }
{速度、路程}
{速率、位移}
{速率、路程}

\item
\exwhere{$ 2017 $年浙江选考卷}
$ 4 $月的江南,草长莺飞,桃红柳绿,雨水连绵。伴随温柔的雨势时常出现瓢泼大雨、雷电交加的景象,在某次闪电过后约$ 2 $秒小明听到雷声,则雷电生成处离小明的距离约为
\xzanswer{A} 
\fourchoices
{$ 6 \times 10 ^ { 2 } \mathrm { m } $}
{$ 6 \times 10 ^ { 4 } \mathrm { m } $}
{$ 6 \times 10 ^ { 6 } $}
{$ 6 \times 10 ^ { 8 } $}





\item 
\exwhere{$ 2012 $年物理上海卷}
质点做直线运动,其$ s-t $关系如图所示。质点在$ 0 \sim 20 \ s $内的平均速度大小为\tk{0.8}$ \ m/s $;质点在\tk{$ 10 \ s $和$ 14 \ s $}时的瞬时速度等于它在$ 6\sim 20 \ s $内的平均速度。
\begin{figure}[h!]
\centering
\includesvg[width=0.23\linewidth]{picture/svg/375}
\end{figure}


\item 
\exwhere{$ 2013 $年新课标 \lmd{1} 卷}
如图,直线$ a $和曲线$ b $分别是在平直公路上行驶的汽车$ a $和$ b $的位置一时间($ x-t $)图线,由图可知 \xzanswer{BC} 
\begin{figure}[h!]
\centering
\includesvg[width=0.23\linewidth]{picture/svg/376}
\end{figure}


\fourchoices
{在时刻$ t_{1} $,$ a $车追上$ b $车}
{在时刻$ t_{2} $,$ a $、$ b $两车运动方向相反}
{在$ t_{1} $到$ t_{2} $这段时间内,$ b $车的速率先减少后增加}
{在$ t_{1} $到$ t_{2} $这段时间内,$ b $车的速率一直比$ a $车的大}



\item 
\exwhere{$ 2015 $年理综浙江卷}
如图所示,气垫导轨上滑块经过光电门时,其上的遮光条将光遮住,电子计时器可自动记录遮光时间$ \Delta t $,测得遮光条的宽度为$ \Delta x $,用$\frac { \Delta x } { \Delta t }$近似代表滑块通过光电门时的瞬时速度,为使$\frac { \Delta x } { \Delta t }$更接近瞬时速度,正确的措施是 \xzanswer{A} 
\begin{figure}[h!]
\centering
\includesvg[width=0.23\linewidth]{picture/svg/377}
\end{figure}


\fourchoices
{换用宽度更窄的遮光条}
{提高测量遮光条宽度的精确度}
{使滑块的释放点更靠近光电门}
{增大气垫导轨与水平面的夹角}

\item 
\exwhere{$ 2015 $年广东卷}
甲乙两人同时同地出发骑自行车做直线运动,前$ 1 $小时内的位移$ - $时间图像如图所示。下列表述正确的是 \xzanswer{B} 
\begin{figure}[h!]
\centering
\includesvg[width=0.23\linewidth]{picture/svg/378}
\end{figure}


\fourchoices
{$ 0.2 \sim 0.5 $小时内,甲的加速度比乙的大}
{$ 0.2 \sim 0.5 $小时内,甲的速度比乙的大}
{$ 0.6 \sim 0.8 $小时内,甲的位移比乙的小}
{$ 0.8 $小时内,甲、乙骑行的路程相等}


\item 
\exwhere{$ 2016 $年浙江卷}
如图所示为一种常见的身高体重测量仪。测量仪顶部向下发射波速为$ v $的超声波,超声波经反射后返回,被测量仪接收,测量仪记录发射和接收的时间间隔。质量为$ M_{0} $的测重台置于压力传感器上,传感器输出电压与作用在其上的压力成正比。当测重台没有站人时,测量仪记录的时间间隔为$ t_{0} $,输出电压为$ U_{0} $,某同学站上测重台,测量仪记录的时间间隔为$ t $,输出电压为$ U $,则该同学的身高和质量分别为 \xzanswer{D} 
\begin{figure}[h!]
\centering
\includesvg[width=0.03\linewidth]{picture/svg/379}
\end{figure}


\fourchoices
{$ v \left( t _ { 0 } - t \right) , \quad \frac { M _ { 0 } } { U _ { 0 } } U $}
{$ \frac { 1 } { 2 } v \left( t _ { 0 } - t \right) , \quad \frac { M _ { 0 } } { U _ { 0 } } U $}
{$ v \left( t _ { 0 } - t \right) , \quad \frac { M _ { 0 } } { U _ { 0 } } \left( U - U _ { 0 } \right) $}
{$ \frac { 1 } { 2 } v \left( t _ { 0 } - t \right) , \quad \frac { M _ { 0 } } { U _ { 0 } } \left( U - U _ { 0 } \right) $}











\end{enumerate}




