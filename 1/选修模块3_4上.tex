\bta{选修模块 $ 3 $—$ 4 $(上)}


\begin{enumerate}
	%\renewcommand{\labelenumi}{\arabic{enumi}.}
	% A(\Alph) a(\alph) I(\Roman) i(\roman) 1(\arabic)
	%设定全局标号series=example	%引用全局变量resume=example
	%[topsep=-0.3em,parsep=-0.3em,itemsep=-0.3em,partopsep=-0.3em]
	%可使用leftmargin调整列表环境左边的空白长度 [leftmargin=0em]
	\item

\begin{enumerate}
	%\renewcommand{\labelenumi}{\arabic{enumi}.}
	% A(\Alph) a(\alph) I(\Roman) i(\roman) 1(\arabic)
	%设定全局标号series=example	%引用全局变量resume=example
	%[topsep=-0.3em,parsep=-0.3em,itemsep=-0.3em,partopsep=-0.3em]
	%可使用leftmargin调整列表环境左边的空白长度 [leftmargin=0em]
	\item
水槽中,与水面接触 的两根相同细杆固定在同一个振动片上。振动片做简
谐振动时,两根细杆周期性触动水面形成两个波源。两波源发出的波在水面上相遇。在重叠区域
发生干涉并形成了干涉图样。关于两列波重叠区域内水面上振动的质点,下列说法正确的是 \underlinegap 
。


\fivechoices
{不同质点的振幅都相同}
{不同质点振动的频率都相同}
{不同质点振动的相位都相同}
{不同质点振动的周期都与振动片的周期相同}
{同一质点处,两列波的相位差不随时间变化}

 \tk{BDE} 


\item 
如图,直角三角形 $ ABC $ 为一棱镜的横截面,$ \angle A=90 \degree $,$ \angle B=30 \degree $。一束光
线平行于底边 $ BC $ 射到 $ AB $ 边上并进入棱镜,然后垂直于 $ AC $ 边射出。
\begin{enumerate}
	%\renewcommand{\labelenumi}{\arabic{enumi}.}
	% A(\Alph) a(\alph) I(\Roman) i(\roman) 1(\arabic)
	%设定全局标号series=example	%引用全局变量resume=example
	%[topsep=-0.3em,parsep=-0.3em,itemsep=-0.3em,partopsep=-0.3em]
	%可使用leftmargin调整列表环境左边的空白长度 [leftmargin=0em]
	\item
求棱镜的折射率;

\item 
保持 $ AB $ 边上的入射点不变,逐渐减小入射角,直到 $ BC $ 边上恰好有光线射出。求此时 $ AB $ 边
上入射角的正弦。

\end{enumerate}
\begin{figure}[h!]
	\flushright
	\includesvg[width=0.25\linewidth]{picture/svg/GZ-3-tiyou-1519}
\end{figure}

	

\banswer{
	\begin{enumerate}
		%\renewcommand{\labelenumi}{\arabic{enumi}.}
		% A(\Alph) a(\alph) I(\Roman) i(\roman) 1(\arabic)
		%设定全局标号series=example	%引用全局变量resume=example
		%[topsep=-0.3em,parsep=-0.3em,itemsep=-0.3em,partopsep=-0.3em]
		%可使用leftmargin调整列表环境左边的空白长度 [leftmargin=0em]
		\item
		$ \sqrt{3} $
		\item 
		$\sin i^{\prime}=\frac{\sqrt{3}-\sqrt{2}}{2}$
	\end{enumerate}
}


	
\end{enumerate}


\item 
\exwhere{$ 2019 $ 年物理全国 \lmd{1} 卷}
\begin{enumerate}
	%\renewcommand{\labelenumi}{\arabic{enumi}.}
	% A(\Alph) a(\alph) I(\Roman) i(\roman) 1(\arabic)
	%设定全局标号series=example	%引用全局变量resume=example
	%[topsep=-0.3em,parsep=-0.3em,itemsep=-0.3em,partopsep=-0.3em]
	%可使用leftmargin调整列表环境左边的空白长度 [leftmargin=0em]
	\item
一简谐横波沿$ x $轴正方向传播,在$ t=5 $时刻,该波的波形图如图 \subref{2019全国134a} 所示,$ P $、$ Q $是介
质中的两个质点。图 \subref{2019全国134b} 表示介质中某质点的振动图像。下列说法正确的是 \underlinegap (填正确答案标号。
选对$ 1 $个得$ 2 $分,选对$ 2 $个得$ 4 $分,选对$ 3 $个得$ 5 $分。每选错$ 1 $个扣$ 3 $分,最低得分为$ 0 $分)
\begin{figure}[h!]
	\centering
\begin{subfigure}{0.4\linewidth}
	\centering
	\includesvg[width=0.7\linewidth]{picture/svg/GZ-3-tiyou-1520} 
	\caption{}\label{2019全国134a}
\end{subfigure}
\begin{subfigure}{0.4\linewidth}
	\centering
	\includesvg[width=0.7\linewidth]{picture/svg/GZ-3-tiyou-1521} 
	\caption{}\label{2019全国134b}
\end{subfigure}
\end{figure}

\fivechoices
{质点$ Q $的振动图像与图($ b $)相同}
{在$ t=0 $时刻,质点$ P $的速率比质点$ Q $的大}
{在$ t=0 $时刻,质点$ P $的加速度的大小比质点$ Q $的大}
{平衡位置在坐标原点的质点的振动图像如图($ b $)所示}
{在$ t=0 $时刻,质点$ P $与其平衡位置的距离比质点$ Q $的大}


 \tk{CDE} 


\item 
如图,一般帆船静止在湖面上,帆船的竖直桅杆顶端高出
水面$ 3 \ m $。距水面$ 4 \ m $的湖底$ P $点发出的激光束,从水面出射后恰好照射到桅杆顶端,该出射光束与
竖直方向的夹角为$ 53 \degree $(取$ \sin 53 \degree =0.8 $)。已知水的折射率为$  \frac{ 4 }{ 3 }  $。
\begin{enumerate}
	%\renewcommand{\labelenumi}{\arabic{enumi}.}
	% A(\Alph) a(\alph) I(\Roman) i(\roman) 1(\arabic)
	%设定全局标号series=example	%引用全局变量resume=example
	%[topsep=-0.3em,parsep=-0.3em,itemsep=-0.3em,partopsep=-0.3em]
	%可使用leftmargin调整列表环境左边的空白长度 [leftmargin=0em]
	\item
求桅杆到$ P $点的水平距离;
\item 
船向左行驶一段距离后停止,调整由$ P $点发出的激光束方向,当其与竖直方向夹角为$ 45 \degree $时,
从水面射出后仍然照射在桅杆顶端,求船行驶的距离。
\end{enumerate}
\begin{figure}[h!]
	\flushright
	\includesvg[width=0.25\linewidth]{picture/svg/GZ-3-tiyou-1522}
\end{figure}

\banswer{
	\begin{enumerate}
		%\renewcommand{\labelenumi}{\arabic{enumi}.}
		% A(\Alph) a(\alph) I(\Roman) i(\roman) 1(\arabic)
		%设定全局标号series=example	%引用全局变量resume=example
		%[topsep=-0.3em,parsep=-0.3em,itemsep=-0.3em,partopsep=-0.3em]
		%可使用leftmargin调整列表环境左边的空白长度 [leftmargin=0em]
		\item
		$ 7 \ m $
		\item 
		$ 5.5 \ m $
	\end{enumerate}
}


	
\end{enumerate}


\item 
\exwhere{$ 2019 $ 年物理全国\lmd{2}卷}
如图,长为 $ l $ 的细绳下方悬挂一小球 $ a $。绳的另一端固定在天花
板上 $ O $ 点处,在 $ O $ 点正下方$  \frac{ 3 }{ 4 } l $ 的 $ O ^{\prime}  $ 处有一固定细铁钉。将小球向右拉开,使细绳与竖直
方向成一小角度(约为 $ 2 \degree $)后由静止释放,并从释放时开始计时。当小球 $ a $ 摆至最低位置
时,细绳会受到铁钉的阻挡。设小球相对于其平衡位置的水平位移为 $ x $,向右为正。下列
图像中,能描述小球在开始一个周期内的 $ x-t $ 关系的是  \xzanswer{A} 

\begin{figure}[h!]
	\centering
	\includesvg[width=0.23\linewidth]{picture/svg/GZ-3-tiyou-1523}
\end{figure}

\pfourchoices
{\includesvg[width=4.3cm]{picture/svg/GZ-3-tiyou-1524}}
{\includesvg[width=4.3cm]{picture/svg/GZ-3-tiyou-1525}}
{\includesvg[width=4.3cm]{picture/svg/GZ-3-tiyou-1526}}
{\includesvg[width=4.3cm]{picture/svg/GZ-3-tiyou-1527}}



\item 
\exwhere{$ 2019 $ 年物理全国\lmd{2}卷}
某同学利用图示装置测量某种单色光的波长。实验时,接通电源
使光源正常发光:调整光路,使得从目镜中可以观察到干涉条纹。回答下列问题:
\begin{figure}[h!]
	\centering
	\includesvg[width=0.23\linewidth]{picture/svg/GZ-3-tiyou-1528}
\end{figure}

\begin{enumerate}
	%\renewcommand{\labelenumi}{\arabic{enumi}.}
	% A(\Alph) a(\alph) I(\Roman) i(\roman) 1(\arabic)
	%设定全局标号series=example	%引用全局变量resume=example
	%[topsep=-0.3em,parsep=-0.3em,itemsep=-0.3em,partopsep=-0.3em]
	%可使用leftmargin调整列表环境左边的空白长度 [leftmargin=0em]
	\item
若想增加从目镜中观察到的条纹个数,该同学可 \underlinegap ;
\fourchoices
{将单缝向双缝靠近}
{将屏向靠近双缝的方向移动}
{将屏向远离双缝的方向移动}
{使用间距更小的双缝}


\item 
若双缝的间距为 $ d $,屏与双缝间的距离为 $ l $,测得第 $ 1 $ 条暗条纹到第 $ n $ 条暗条纹之间
的距离为$ \Delta x $,则单色光的波长$ \lambda = $ \underlinegap ;

\item 
某次测量时,选用的双缝的间距为 $ 0.300 \ mm $,测得屏与双缝间的距离为 $ 1.20 \ m $,
第 $ 1 $ 条暗条纹到第 $ 4 $ 条暗条纹之间的距离为 $ 7.56 \ mm $。则所测单色光的波长为 \underlinegap 
$ nm $(结果保留 $ 3 $ 位有效数字)。

\end{enumerate}

 \tk{
\begin{enumerate}
	%\renewcommand{\labelenumi}{\arabic{enumi}.}
	% A(\Alph) a(\alph) I(\Roman) i(\roman) 1(\arabic)
	%设定全局标号series=example	%引用全局变量resume=example
	%[topsep=-0.3em,parsep=-0.3em,itemsep=-0.3em,partopsep=-0.3em]
	%可使用leftmargin调整列表环境左边的空白长度 [leftmargin=0em]
	\item
	B
\item 
$\frac{\Delta x \cdot d}{(n-1) l}$	
\item 	
	$ 630 $
\end{enumerate}
} 



\item 
\exwhere{$ 2018 $年江苏卷}
\begin{enumerate}
	%\renewcommand{\labelenumi}{\arabic{enumi}.}
	% A(\Alph) a(\alph) I(\Roman) i(\roman) 1(\arabic)
	%设定全局标号series=example	%引用全局变量resume=example
	%[topsep=-0.3em,parsep=-0.3em,itemsep=-0.3em,partopsep=-0.3em]
	%可使用leftmargin调整列表环境左边的空白长度 [leftmargin=0em]
	\item
梳子在梳头后带上电荷,摇动这把梳子在空中产生电磁波。该电磁波 \underlinegap 。

\fourchoices
{是横波}
{不能在真空中传播}
{只能沿着梳子摇动的方向传播}
{在空气中的传播速度约为$ 3 \times 10^{8} \ m /s $}

 \tk{AD} 


\item 
两束单色光$ A $、$ B $的波长分别为 $ \lambda_{A} $、 $ \lambda_{B} $,且 $\lambda_{A}>\lambda_{B}$,则 
 \underlinegap 
(选填“$ A $”或“$ B $”)在水中发生
全反射时的临界角较大。用同一装置进行杨氏双缝干涉实验时,可以观察到  \underlinegap  (选填“$ A $”或
“$ B $”)产生的条纹间距较大。

 \tk{A \quad A} 


\item 
一列简谐横波沿$ x $轴正方向传播,在$ x=0 $ 和$ x=0.6 \ m $处
的两个质点$ A $、$ B $的振动图象如图所示。已知该波的波长大
于$ 0.6 \ m $,求其波速和波长。
\begin{figure}[h!]
	\flushright
	\includesvg[width=0.25\linewidth]{picture/svg/GZ-3-tiyou-1529}
\end{figure}


\banswer{
	$ v=2 \ m/s  $ \quad $ \lambda=0.8 \ m $
}


\end{enumerate}




\item 
\exwhere{$ 2018 $ 年海南卷}
\begin{enumerate}
	%\renewcommand{\labelenumi}{\arabic{enumi}.}
	% A(\Alph) a(\alph) I(\Roman) i(\roman) 1(\arabic)
	%设定全局标号series=example	%引用全局变量resume=example
	%[topsep=-0.3em,parsep=-0.3em,itemsep=-0.3em,partopsep=-0.3em]
	%可使用leftmargin调整列表环境左边的空白长度 [leftmargin=0em]
	\item
警车向路上的车辆发射频率已知的超声波,同时探测反射波的频率。下列说法正确的
是 \underlinegap 。(填入正确答案标号。选对 $ 1 $ 个得 $ 2 $ 分,选对 $ 2 $ 个得 $ 4 $ 分;有选错的得 $ 0 $ 分)

\fourchoices
{车辆匀速驶向停在路边的警车,警车探测到的反射波频率增高}
{车辆匀速驶离停在路边的警车,警车探测到的反射波频率降低}
{警车匀速驶向停在路边的汽车,探测到的反射波频率降低}
{警车匀速驶离停在路边的汽车,探测到的反射波频率不变}


 \tk{AB} 

\item 
如图,由透明介质构成的半球壳的内外表面半径分别为 $ R $ 和 $ \sqrt{2}R $。一横截面半径为 $ R $
的平行光束入射到半球壳内表面,入射方向与半球壳的对称轴平
行,所有的入射光线都能从半球壳的外表面射出。已知透明介质
的折射率为 $ n=\sqrt{2} $。求半球壳外表面上有光线射出区域的圆形
边界的半径。不考虑多次反射。
\begin{figure}[h!]
	\flushright
	\includesvg[width=0.25\linewidth]{picture/svg/GZ-3-tiyou-1530}
\end{figure}


\banswer{
	$r=\frac{\sqrt{3}+1}{2} R$
}



\end{enumerate}



\item 
\exwhere{$ 2018 $ 年全国卷}
\begin{enumerate}
	%\renewcommand{\labelenumi}{\arabic{enumi}.}
	% A(\Alph) a(\alph) I(\Roman) i(\roman) 1(\arabic)
	%设定全局标号series=example	%引用全局变量resume=example
	%[topsep=-0.3em,parsep=-0.3em,itemsep=-0.3em,partopsep=-0.3em]
	%可使用leftmargin调整列表环境左边的空白长度 [leftmargin=0em]
	\item
如图,$ \triangle ABC $ 为一玻璃三棱镜的横截面,
$ \angle A=30 \degree $。一束红光垂直 $ AB $ 边射入,从 $ AC $ 边上的 $ D $ 点射出,
其折射角为 $ 60 \degree $,则玻璃对红光的折射率为 \underlinegap 。若改用蓝
光沿同一路径入射,则光线在 $ D $ 点射出时的折射角  \underlinegap 
(填“小于”“等于”或“大于”) $ 60 \degree $。
\begin{figure}[h!]
	\centering
	\includesvg[width=0.23\linewidth]{picture/svg/GZ-3-tiyou-1531}
\end{figure}


 \tk{$ \sqrt{3} $ \quad 大于} 


\item 
一列简谐横波在 $ t= \frac{ 1 }{ 3 } \ s $ 时的波形图如图  \subref{2018全国13402a} 所示,$ P $、$ Q $ 是介质中的两个质点。图 \subref{2018全国13402b} 是质点 $ Q $ 的振动图像。求:
\begin{enumerate}
	%\renewcommand{\labelenumi}{\arabic{enumi}.}
	% A(\Alph) a(\alph) I(\Roman) i(\roman) 1(\arabic)
	%设定全局标号series=example	%引用全局变量resume=example
	%[topsep=-0.3em,parsep=-0.3em,itemsep=-0.3em,partopsep=-0.3em]
	%可使用leftmargin调整列表环境左边的空白长度 [leftmargin=0em]
	\item
波速及波的传播方向;

\item 
质点 $ Q $ 的平衡位置的 $ x $ 坐标。


\end{enumerate}
\begin{figure}[h!]
	\flushright
\begin{subfigure}{0.4\linewidth}
	\centering
	\includesvg[width=0.7\linewidth]{picture/svg/GZ-3-tiyou-1532} 
	\caption{}\label{2018全国13402a}
\end{subfigure}
\begin{subfigure}{0.4\linewidth}
	\centering
	\includesvg[width=0.7\linewidth]{picture/svg/GZ-3-tiyou-1533} 
	\caption{}\label{2018全国13402b}
\end{subfigure}
\end{figure}


\banswer{
	\begin{enumerate}
		%\renewcommand{\labelenumi}{\arabic{enumi}.}
		% A(\Alph) a(\alph) I(\Roman) i(\roman) 1(\arabic)
		%设定全局标号series=example	%引用全局变量resume=example
		%[topsep=-0.3em,parsep=-0.3em,itemsep=-0.3em,partopsep=-0.3em]
		%可使用leftmargin调整列表环境左边的空白长度 [leftmargin=0em]
		\item
		$ v=18 \ cm /s $ ;波沿 $ x $ 轴负方向传播
		\item 
		$ x=9 \ cm $
	\end{enumerate}
}


\end{enumerate}



\item
\exwhere{$ 2018 $ 年全国\lmd{2}卷}
\begin{enumerate}
	%\renewcommand{\labelenumi}{\arabic{enumi}.}
	% A(\Alph) a(\alph) I(\Roman) i(\roman) 1(\arabic)
	%设定全局标号series=example	%引用全局变量resume=example
	%[topsep=-0.3em,parsep=-0.3em,itemsep=-0.3em,partopsep=-0.3em]
	%可使用leftmargin调整列表环境左边的空白长度 [leftmargin=0em]
	\item
声波在空气中的传播速度为 $ 340 \ m /s $,在钢铁中的传播速度为 $ 4900 \ m /s $。一平直桥由
钢铁制成,某同学用锤子敲击一下桥的一端发出声音,分别经空气和桥传到另一端的时间之差为$ 1.00 \ s $。桥的长度为  \underlinegap  $ m $。若该声波在空气中的波长为 $ \lambda $,则它在钢铁中的波长为 $ l $ 的
 \underlinegap 
倍。

 \tk{$ 365  \quad \frac{245}{17} $} 



\item 
如图, $ \triangle ABC $ 是一直角三棱镜的横截面,  $ \angle A=90 \degree $,  $ \angle B=60 \degree $。一细光束从 $ BC $ 边
的 $ D $ 点折射后,射到 $ AC $ 边的 $ E $ 点,发生全反射后经 $ AB $ 边的 $ F $ 点射出。 $ EG $ 垂直于 $ AC $ 交 $ BC $ 于
$ G $,$ D $ 恰好是 $ CG $ 的中点。不计多次反射。
\begin{enumerate}
	%\renewcommand{\labelenumi}{\arabic{enumi}.}
	% A(\Alph) a(\alph) I(\Roman) i(\roman) 1(\arabic)
	%设定全局标号series=example	%引用全局变量resume=example
	%[topsep=-0.3em,parsep=-0.3em,itemsep=-0.3em,partopsep=-0.3em]
	%可使用leftmargin调整列表环境左边的空白长度 [leftmargin=0em]
	\item
求出射光相对于 $ D $ 点的入射光的偏角;
\item 
为实现上述光路,棱镜折射率的取值应在什么范围?
\end{enumerate}
\begin{figure}[h!]
	\flushright
	\includesvg[width=0.25\linewidth]{picture/svg/GZ-3-tiyou-1534}
\end{figure}

\banswer{
	\begin{enumerate}
		%\renewcommand{\labelenumi}{\arabic{enumi}.}
		% A(\Alph) a(\alph) I(\Roman) i(\roman) 1(\arabic)
		%设定全局标号series=example	%引用全局变量resume=example
		%[topsep=-0.3em,parsep=-0.3em,itemsep=-0.3em,partopsep=-0.3em]
		%可使用leftmargin调整列表环境左边的空白长度 [leftmargin=0em]
		\item
		$\delta=60^{\circ}$
		\item 
		$\frac{2 \sqrt{3}}{3} \leq n<2$	
	\end{enumerate}
}


\end{enumerate}

\item 
\exwhere{$ 2018 $ 年全国\lmd{3}卷}
 \begin{enumerate}
 	%\renewcommand{\labelenumi}{\arabic{enumi}.}
 	% A(\Alph) a(\alph) I(\Roman) i(\roman) 1(\arabic)
 	%设定全局标号series=example	%引用全局变量resume=example
 	%[topsep=-0.3em,parsep=-0.3em,itemsep=-0.3em,partopsep=-0.3em]
 	%可使用leftmargin调整列表环境左边的空白长度 [leftmargin=0em]
 	\item
一列简谐横波沿 $ x $ 轴正方向传播,在 $ t=0 $ 和 $ t=0.20 \ s $ 时的波形分别如图中实线和虚线
所示。已知该波的周期 $ T>0.20 \ s $。下列说法正确的是  \underlinegap 。(填正确答案标号。选对 $ 1 $ 个得 $ 2 $
分,选对 $ 2 $ 个得 $ 4 $ 分,选对 $ 3 $ 个得 $ 5 $ 分。每选错 $ 1 $ 个扣 $ 3 $ 分,最低得分为 $ 0 $ 分)
\begin{figure}[h!]
	\centering
	\includesvg[width=0.23\linewidth]{picture/svg/GZ-3-tiyou-1535}
\end{figure}

\fivechoices
{波速为 $ 0.40 \ m /s $}
{波长为 $ 0.08 \ m $}
{$ x=0.08 \ m $ 的质点在 $ t=0.70 \ s $ 时位于波谷}
{$ x=0.08 \ m $ 的质点在 $ t=0.12 \ s $ 时位于波谷}
{若此波传入另一介质中其波速变为 $ 0.80 \ m /s $,则它在该介质中的波长为 $ 0.32 \ m $}




 \tk{ACE} 


\item 
如图,某同学在一张水平放置的白纸上画了一个小标记“ ”(图中 $ O $ 点)
,然后用
横截面为等边三角形 $ ABC $ 的三棱镜压在这个标记上,小标记位于
$ AC $ 边上。$ D $ 位于 $ AB $ 边上,过 $ D $ 点做 $ AC $ 边的垂线交 $ AC $ 于 $ F $。该
同学在 $ D $ 点正上方向下顺着直线 $ DF $ 的方向观察,恰好可以看到小
标记的像;过 $ O $ 点做 $ AB $ 边的垂线交直线 $ DF $ 于 $ E $; $ DE=2 \ cm $,
$ EF=1 \ cm $。求三棱镜的折射率。
(不考虑光线在三棱镜中的反射)
\begin{figure}[h!]
	\flushright
	\includesvg[width=0.25\linewidth]{picture/svg/GZ-3-tiyou-1536}
\end{figure}


\banswer{
	$ n=\sqrt{3} $
}



\end{enumerate}


\item 
\exwhere{$ 2017 $ 年新课标\lmd{1}卷}
\begin{enumerate}
	%\renewcommand{\labelenumi}{\arabic{enumi}.}
	% A(\Alph) a(\alph) I(\Roman) i(\roman) 1(\arabic)
	%设定全局标号series=example	%引用全局变量resume=example
	%[topsep=-0.3em,parsep=-0.3em,itemsep=-0.3em,partopsep=-0.3em]
	%可使用leftmargin调整列表环境左边的空白长度 [leftmargin=0em]
	\item
如图($ a $),在 $ xy $ 平面内有两个沿 $ z $ 方向做简谐振动的点波源 $ S_{1} (0 , 4) $和 $ S_{2} (0 , -2) $。
两波源的振动图线分别如图($ b $)和图($ c $)所示,两列波的波速均为 $ 1.00 \ m /s $。两列波从波源传播
到点 $ A $($ 8 , -2 $)的路程差为 \underlinegap $ m $,两列波引起的点 $ B $($ 4 $,$ 1 $)处质点的振动相互 \underlinegap 
(填“加强”或“减弱”),点 $ C $($ 0 $,$ 0.5 $)处质点的振动相互 \underlinegap (填“加强”或“减弱”)。
\begin{figure}[h!]
	\centering
\begin{subfigure}{0.4\linewidth}
	\centering
	\includesvg[width=0.7\linewidth]{picture/svg/GZ-3-tiyou-1537} 
	\caption{}\label{}
\end{subfigure}
\begin{subfigure}{0.4\linewidth}
	\centering
	\includesvg[width=0.7\linewidth]{picture/svg/GZ-3-tiyou-1538} 
	\caption{}\label{}
\end{subfigure}
\begin{subfigure}{0.4\linewidth}
	\centering
	\includesvg[width=0.7\linewidth]{picture/svg/GZ-3-tiyou-1539} 
	\caption{}\label{}
\end{subfigure}
\end{figure}


 \tk{$ 2 \ m  $ \quad 减弱 \quad 加强} 

\item 
如图,一玻璃工件的上半部是半径为 $ R $ 的半球体,$ O $ 点为球心;下半部是半径为
$ R $、高位 $ 2R $ 的圆柱体,圆柱体底面镀有反射膜。有一平行于中心轴 $ OC $
的光线从半球面射入,该光线与 $ OC $ 之间的距离为 $ 0.6R $。已知最后从半
球面射出的光线恰好与入射光线平行(不考虑多次反射)。求该玻璃的
折射率。
\begin{figure}[h!]
	\flushright
	\includesvg[width=0.25\linewidth]{picture/svg/GZ-3-tiyou-1540}
\end{figure}


\banswer{
	$n=\sqrt{2.05} \approx 1.43$
}


\end{enumerate}


\item 
\exwhere{$ 2017 $ 年新课标 \lmd{2} 卷}
\begin{enumerate}
	%\renewcommand{\labelenumi}{\arabic{enumi}.}
	% A(\Alph) a(\alph) I(\Roman) i(\roman) 1(\arabic)
	%设定全局标号series=example	%引用全局变量resume=example
	%[topsep=-0.3em,parsep=-0.3em,itemsep=-0.3em,partopsep=-0.3em]
	%可使用leftmargin调整列表环境左边的空白长度 [leftmargin=0em]
	\item
在双缝干涉实验中,用绿色激光照射在双缝上,在缝后的屏幕上显示出干涉图样。
若要增大干涉图样中两相邻亮条纹的间距,可选用的方法是 \underlinegap (选对 $ 1 $ 个得 $ 2 $ 分,选对 $ 2 $ 个
得 $ 4 $ 分,选对 $ 3 $ 个得 $ 5 $ 分;每选错 $ 1 $ 个扣 $ 3 $ 分,最低得分为 $ 0 $ 分)。
\fivechoices
{改用红色激光}
{改用蓝色激光}
{减小双缝间距}
{将屏幕向远离双缝的位置移动}
{将光源向远离双缝的位置移动}


 \tk{ACD} 

\item 
一直桶状容器的高为 $ 2l $,底面是边长为 $ l $ 的正方形;容器
内装满某种透明液体,过容器中心轴 $ DD ^{\prime} $、垂直于左右两侧面的剖面图如
图所示。容器右侧内壁涂有反光材料,其他内壁涂有吸光材料。在剖面
的左下角处有一点光源,已知由液体上表面的 $ D $ 点射出的两束光线相互
垂直,求该液体的折射率。
\begin{figure}[h!]
	\flushright
	\includesvg[width=0.25\linewidth]{picture/svg/GZ-3-tiyou-1541}
\end{figure}

\banswer{
	$ n=1.55 $
}



\end{enumerate}



\item 
\exwhere{$ 2017 $ 年新课标 \lmd{3} 卷}
\begin{enumerate}
	%\renewcommand{\labelenumi}{\arabic{enumi}.}
	% A(\Alph) a(\alph) I(\Roman) i(\roman) 1(\arabic)
	%设定全局标号series=example	%引用全局变量resume=example
	%[topsep=-0.3em,parsep=-0.3em,itemsep=-0.3em,partopsep=-0.3em]
	%可使用leftmargin调整列表环境左边的空白长度 [leftmargin=0em]
	\item
如图,一列简谐横波沿 $ x $ 轴正方向传播,实线为 $ t=0 $ 时的波形图,虚线为 $ t=0.5 \ s $ 时
的波形图。已知该简谐波的周期大于 $ 0.5 \ s $。关于该简谐波,下列说法正确的是 \underlinegap (填正确答
案标号。选对 $ 1 $ 个得 $ 2 $ 分,选对 $ 2 $ 个得 $ 4 $ 分,选对 $ 3 $ 个得 $ 5 $ 分。每选错 $ 1 $ 个扣 $ 3 $ 分,最低得分为 $ 0 $
分)。
\begin{figure}[h!]
	\centering
	\includesvg[width=0.23\linewidth]{picture/svg/GZ-3-tiyou-1542}
\end{figure}


\fivechoices
{波长为 $ 2 \ m $}
{波速为 $ 6 \ m /s $}
{频率为 $ 1.5 \ Hz $}
{$ t=1 \ s $ 时,$ x=1 \ m $ 处的质点处于波峰}
{$ t=2 \ s $ 时,$ x=2 \ m $ 处的质点经过平衡位置}

 \tk{BCE} 

\item 
如图,一半径为 $ R $ 的玻璃半球,$ O $ 点是半球的球心,虚线 $ OO ^{\prime} $表示光轴(过球心 $ O $
与半球底面垂直的直线)。已知玻璃的折射率为 $ 1.5 $。现有一束平行光
垂直入射到半球的底面上,有些光线能从球面射出(不考虑被半球的内
表面反射后的光线)。求:
\begin{enumerate}
	%\renewcommand{\labelenumi}{\arabic{enumi}.}
	% A(\Alph) a(\alph) I(\Roman) i(\roman) 1(\arabic)
	%设定全局标号series=example	%引用全局变量resume=example
	%[topsep=-0.3em,parsep=-0.3em,itemsep=-0.3em,partopsep=-0.3em]
	%可使用leftmargin调整列表环境左边的空白长度 [leftmargin=0em]
	\item
从球面射出的光线对应的入射光线到光轴距离的最大值;
\item 
距光轴$  \frac{ 1 }{ 3 } R $ 的入射光线经球面折射后与光轴的交点到 $ O $ 点的距离。
	
\end{enumerate}
\begin{figure}[h!]
	\flushright
	\includesvg[width=0.25\linewidth]{picture/svg/GZ-3-tiyou-1543}
\end{figure}



\banswer{
	\begin{enumerate}
		%\renewcommand{\labelenumi}{\arabic{enumi}.}
		% A(\Alph) a(\alph) I(\Roman) i(\roman) 1(\arabic)
		%设定全局标号series=example	%引用全局变量resume=example
		%[topsep=-0.3em,parsep=-0.3em,itemsep=-0.3em,partopsep=-0.3em]
		%可使用leftmargin调整列表环境左边的空白长度 [leftmargin=0em]
		\item
		$d_{M}=\frac{2}{3} R$
		\item 
		$d=\frac{3(2 \sqrt{2}+\sqrt{3})}{5} R$
	\end{enumerate}
}


	
\end{enumerate}


\item 
\exwhere{$ 2017 $ 年江苏卷}
\begin{enumerate}
	%\renewcommand{\labelenumi}{\arabic{enumi}.}
	% A(\Alph) a(\alph) I(\Roman) i(\roman) 1(\arabic)
	%设定全局标号series=example	%引用全局变量resume=example
	%[topsep=-0.3em,parsep=-0.3em,itemsep=-0.3em,partopsep=-0.3em]
	%可使用leftmargin调整列表环境左边的空白长度 [leftmargin=0em]
	\item
接近光速飞行的飞船和地球上各有一只相同的铯原子钟,飞船和地球上的人观测这两只钟的
快慢,下列说法正确的有 \underlinegap 。
\fourchoices
{飞船上的人观测到飞船上的钟较快}
{飞船上的人观测到飞船上的钟较慢}
{地球上的人观测到地球上的钟较快}
{地球上的人观测到地球上的钟较慢}

 \tk{AC} 

\item 
野生大象群也有自己的“语言”。研究人员录下象群“语言”交流时发出的声音,发现以 $ 2 $ 倍速
度快速播放时,能听到比正常播放时更多的声音。播放速度变为原来的 $ 2 $ 倍时,播出声波的 \underlinegap 
(选填“周期”或“频率”)也变为原来的 $ 2 $ 倍,声波的传播速度 \underlinegap (选填“变
大”、“变小”或“不变”)
。

 \tk{频率 \quad 不变} 


\item 
人的眼球可简化为如图所示的模型,折射率相同、半径不
同的两个球体共轴,平行光束宽度为 $ D $,对称地沿轴线方向射
入半径为 $ R $ 的小球,会聚在轴线上的 $ P $ 点。取球体的折射率为$ \sqrt{2} $,且 $ D=\sqrt{2}R $,求光线的会聚角$  \alpha  $。(示意图未按比例画出)
\begin{figure}[h!]
	\flushright
	\includesvg[width=0.25\linewidth]{picture/svg/GZ-3-tiyou-1544}
\end{figure}


\banswer{
	$\alpha=30^{\circ}$
}



\end{enumerate}


\item 
\exwhere{$ 2017 $ 年海南卷}
\begin{enumerate}
	%\renewcommand{\labelenumi}{\arabic{enumi}.}
	% A(\Alph) a(\alph) I(\Roman) i(\roman) 1(\arabic)
	%设定全局标号series=example	%引用全局变量resume=example
	%[topsep=-0.3em,parsep=-0.3em,itemsep=-0.3em,partopsep=-0.3em]
	%可使用leftmargin调整列表环境左边的空白长度 [leftmargin=0em]
	\item
如图,空气中有两块材质不同、上下表面平行的透明玻璃板平行放置;一细光束从空
气中以某一角度$ \theta $($ 0< \theta <90 \degree $)入射到第一块玻璃板的上表面。下列说法正确的是 \underlinegap 。(填入正确答案标号。选对 $ 1 $ 个得 $ 2 $ 分,选对 $ 2 $ 个得 $ 3 $ 分,选对 $ 3 $ 个得 $ 4 $ 分;有
选错的得 $ 0 $ 分)
\begin{figure}[h!]
	\centering
	\includesvg[width=0.23\linewidth]{picture/svg/GZ-3-tiyou-1545}
\end{figure}

\fivechoices
{在第一块玻璃板下表面一定有出射光}
{在第二块玻璃板下表面一定没有出射光}
{第二块玻璃板下表面的出射光方向一定与入射光方向平行}
{第二块玻璃板下表面的出射光一定在入射光延长线的左侧}
{第一块玻璃板下表面的出射光线一定在入射光延长线的右侧}

 \tk{ACD} 

\item 
从两个波源发出的两列振幅相同、频率均为 $ 5 \ Hz $ 的简谐横波,分别沿 $ x $ 轴正、负方向
传播,在某一时刻到达 $ A $、$ B $ 点,如图中实线、虚线所示。
两列波的波速均为 $ 10 \ m /s $。求:
\begin{enumerate}
	%\renewcommand{\labelenumi}{\arabic{enumi}.}
	% A(\Alph) a(\alph) I(\Roman) i(\roman) 1(\arabic)
	%设定全局标号series=example	%引用全局变量resume=example
	%[topsep=-0.3em,parsep=-0.3em,itemsep=-0.3em,partopsep=-0.3em]
	%可使用leftmargin调整列表环境左边的空白长度 [leftmargin=0em]
	\item
质点 $ P $、$ O $ 开始振动的时刻之差;
\item 
再经过半个周期后,两列波在 $ x=1 \ m $ 和 $ x=5 \ m $ 之间引
起的合振动振幅极大和极小的质点的 $ x $ 坐标。
	
\end{enumerate}
\begin{figure}[h!]
	\flushright
	\includesvg[width=0.25\linewidth]{picture/svg/GZ-3-tiyou-1546}
\end{figure}

\banswer{
	\begin{enumerate}
		%\renewcommand{\labelenumi}{\arabic{enumi}.}
		% A(\Alph) a(\alph) I(\Roman) i(\roman) 1(\arabic)
		%设定全局标号series=example	%引用全局变量resume=example
		%[topsep=-0.3em,parsep=-0.3em,itemsep=-0.3em,partopsep=-0.3em]
		%可使用leftmargin调整列表环境左边的空白长度 [leftmargin=0em]
		\item
		$ 0.05 \ s $
		\item 
		 极大的 $ x $ 坐标 $ x=2 \ m $、$ 3 \ m $、$ 4 \ m $; \\
		 极小的 $ x $ 坐标 $ x=1.5 \ m $、$ 2.5 \ m $、$ 3.5 \ m $、$ 4.5 \ m $
	\end{enumerate}
}


	
\end{enumerate}


\item 
\exwhere{$ 2016 $ 年新课标  \lmd{1}  卷}
\begin{enumerate}
	%\renewcommand{\labelenumi}{\arabic{enumi}.}
	% A(\Alph) a(\alph) I(\Roman) i(\roman) 1(\arabic)
	%设定全局标号series=example	%引用全局变量resume=example
	%[topsep=-0.3em,parsep=-0.3em,itemsep=-0.3em,partopsep=-0.3em]
	%可使用leftmargin调整列表环境左边的空白长度 [leftmargin=0em]
	\item
某同学漂浮在海面上,虽然水面波正平稳地以 $ 1.8 \ m /s $ 的速率向着海滩传播,但他并
不向海滩靠近。该同学发现从第 $ 1 $ 个波峰到第 $ 10 $ 个波峰通过身下的时间间隔为 $ 15 \ s $。下列说法正
确的是 \underlinegap 。(填正确答案标号,选对 $ 1 $ 个得 $ 2 $ 分,选对 $ 2 $ 个得 $ 4 $ 分,选对 $ 3 $ 个得 $ 5 $ 分。每
选错 $ 1 $ 个扣 $ 3 $ 分,最低得分为 $ 0 $ 分)
\fivechoices
{水面波是一种机械波}
{该水面波的频率为 $ 6 \ Hz $}
{该水面波的波长为 $ 3 \ m $}
{水面波没有将该同学推向岸边,是因为波传播时能量不会传递出去}
{水面波没有将该同学推向岸边,是因为波传播时振动的质点并不随波迁移}

 \tk{ACE} 


\item 
如图,在注满水的游泳池的池底有一点光源 $ A $,它到池边的水平距离为 $ 3.0 \ m $。从点
光源 $ A $ 射向池边的光线 $ AB $ 与竖直方向的夹角恰好等于全反射的
临界角,水的折射率为$  \frac{ 4 }{ 3 }  $ 。
\begin{enumerate}
	%\renewcommand{\labelenumi}{\arabic{enumi}.}
	% A(\Alph) a(\alph) I(\Roman) i(\roman) 1(\arabic)
	%设定全局标号series=example	%引用全局变量resume=example
	%[topsep=-0.3em,parsep=-0.3em,itemsep=-0.3em,partopsep=-0.3em]
	%可使用leftmargin调整列表环境左边的空白长度 [leftmargin=0em]
	\item
求池内的水深;

\item 
一救生员坐在离池边不远处的高凳上,他的眼睛到池面的高
度为 $ 2.0 \ m $。当他看到正前下方的点光源 $ A $ 时,他的眼睛所接受
的光线与竖直方向的夹角恰好为 $ 45 \degree $。求救生员的眼睛到池边的
水平距离(结果保留 $ 1 $ 位有效数字)。
	
\end{enumerate}
\begin{figure}[h!]
	\flushright
	\includesvg[width=0.25\linewidth]{picture/svg/GZ-3-tiyou-1547}
\end{figure}

\banswer{
	\begin{enumerate}
		%\renewcommand{\labelenumi}{\arabic{enumi}.}
		% A(\Alph) a(\alph) I(\Roman) i(\roman) 1(\arabic)
		%设定全局标号series=example	%引用全局变量resume=example
		%[topsep=-0.3em,parsep=-0.3em,itemsep=-0.3em,partopsep=-0.3em]
		%可使用leftmargin调整列表环境左边的空白长度 [leftmargin=0em]
		\item
		$ \sqrt{7} $
		\item 
		示意图
		\begin{center}
		 \includesvg[width=0.23\linewidth]{picture/svg/GZ-3-tiyou-1548} 
		\end{center}
	设 $|B E|=x \ m,$ 得
	\begin{equation}\label{key}
		\tan \alpha=\frac{|A Q|}{|Q E|}=\frac{3-x}{\sqrt{7}}
	\end{equation}
	代入数据得: $x=3-\frac{3}{23} \sqrt{161}$ ,
	由几何关系得,救生员到池边水平距离为 $(2-x) \ m \approx 0.7 \ m$
	\end{enumerate}
}

	
\end{enumerate}



\item 
\exwhere{$ 2016 $ 年新课标 \lmd{2} 卷}
 \begin{enumerate}
 	%\renewcommand{\labelenumi}{\arabic{enumi}.}
 	% A(\Alph) a(\alph) I(\Roman) i(\roman) 1(\arabic)
 	%设定全局标号series=example	%引用全局变量resume=example
 	%[topsep=-0.3em,parsep=-0.3em,itemsep=-0.3em,partopsep=-0.3em]
 	%可使用leftmargin调整列表环境左边的空白长度 [leftmargin=0em]
 	\item
关于电磁波,下列说法正确的是 \underlinegap 。(填正确答案标号。选对 $ 1 $ 个得 $ 2 $ 分,选对
$ 2 $个得 $ 4 $ 分,选对 $ 3 $ 个得 $ 5 $ 分。每选错 $ 1 $ 个扣 $ 3 $ 分,最低得分为 $ 0 $ 分)
\fivechoices
{电磁波在真空中的传播速度与电磁波的频率无关}
{周期性变化的电场和磁场可以相互激发,形成电磁波}
{电磁波在真空中自由传播时,其传播方向与电场强度、磁感应强度均垂直}
{利用电磁波传递信号可以实现无线通信,但电磁波不能通过电缆、光缆传输}
{电磁波可以由电磁振荡产生,若波源的电磁振荡停止,空间的电磁波随即消失}

 \tk{ABC} 


\item 
一列简谐横波在介质中沿 $ x $ 轴正向传播,波长不小于 $ 10 \ cm $。$ O $ 和 $ A $ 是介质中平衡位
置分别位于 $ x=0 $ 和 $ x=5 \ cm $ 处的两个质点。$ t=0 $ 时开始观测,此时质点 $ O $ 的位移为 $ y=4 \ cm $,质点 $ A $
处于波峰位置;$ t=  \frac{ 1 }{ 3 } \ s $ 时,质点 $ O $ 第一次回到平衡位置,$ t=1 \ s $ 时,质点 $ A $ 第一次回到平衡位置。求:
\begin{enumerate}
	%\renewcommand{\labelenumi}{\arabic{enumi}.}
	% A(\Alph) a(\alph) I(\Roman) i(\roman) 1(\arabic)
	%设定全局标号series=example	%引用全局变量resume=example
	%[topsep=-0.3em,parsep=-0.3em,itemsep=-0.3em,partopsep=-0.3em]
	%可使用leftmargin调整列表环境左边的空白长度 [leftmargin=0em]
	\item
简谐波的周期、波速和波长;
\item 
质点 $ O $ 的位移随时间变化的关系式。
\end{enumerate}

\banswer{
	\begin{enumerate}
		%\renewcommand{\labelenumi}{\arabic{enumi}.}
		% A(\Alph) a(\alph) I(\Roman) i(\roman) 1(\arabic)
		%设定全局标号series=example	%引用全局变量resume=example
		%[topsep=-0.3em,parsep=-0.3em,itemsep=-0.3em,partopsep=-0.3em]
		%可使用leftmargin调整列表环境左边的空白长度 [leftmargin=0em]
		\item
		$ T=4 \ s \quad v= 7.5 \ cm/s  \quad \lambda= 30 \ cm $
		\item 
		$y=0.08 \cos \left(\frac{\pi t}{2}+\frac{\pi}{3}\right)$	
	\end{enumerate}
}



 \end{enumerate}


\item 
\exwhere{$ 2016 $ 年新课标 \lmd{3} 卷}
\begin{enumerate}
	%\renewcommand{\labelenumi}{\arabic{enumi}.}
	% A(\Alph) a(\alph) I(\Roman) i(\roman) 1(\arabic)
	%设定全局标号series=example	%引用全局变量resume=example
	%[topsep=-0.3em,parsep=-0.3em,itemsep=-0.3em,partopsep=-0.3em]
	%可使用leftmargin调整列表环境左边的空白长度 [leftmargin=0em]
	\item
由波源 $ S $ 形成的简谐横波在均匀介质中向
左、右传播。波源振动的频率为 $ 20 \ Hz $,波速为 $ 16 \ m /s $。已知介质中 $ P $、$ Q $ 两质点位于波源 $ S $ 的两
侧,且 $ P $、$ Q $ 和 $ S $ 的平衡位置在一条直线上,$ P $、$ Q $ 的平衡位置到 $ S $ 的平衡位置之间的距离分别为
$ 15.8 \ m $、$ 14.6 \ m $,$ P $、$ Q $ 开始震动后,下列判断正确的是 \underlinegap 。(填正确答案标号。选对 $ 1 $ 个得 $ 2 $
分,选对 $ 2 $ 个得 $ 4 $ 分,选对 $ 3 $ 个得 $ 5 $ 分。每选错 $ 1 $ 个扣 $ 3 $ 分,最低得分为 $ 0 $ 分)
\fivechoices
{$ P $、$ Q $ 两质点运动的方向始终相同}
{$ P $、$ Q $ 两质点运动的方向始终相反}
{当 $ S $ 恰好通过平衡位置时,$ P $、$ Q $ 两点也正好通过平衡位置}
{当 $ S $ 恰好通过平衡位置向上运动时,$ P $ 在波峰}
{当 $ S $ 恰好通过平衡位置向下运动时,$ Q $ 在波峰}

 \tk{BDE} 

\item 
如图,玻璃球冠的折射率为 $ \sqrt{3} $,其底面镀银,底面的半径是球
半径的$ \frac{\sqrt{3}}{2} $倍;在过球心 $ O $ 且垂直于底面的平面(纸面)内,有一与
底面垂直的光线射到玻璃球冠上的 $ M $ 点,该光线的延长线恰好过底
面边缘上的 $ A $ 点。求该光线从球面射出的方向相对于其初始入射方向
的偏角。
\begin{figure}[h!]
	\flushright
	\includesvg[width=0.25\linewidth]{picture/svg/GZ-3-tiyou-1549}
\end{figure}

\banswer{
	$ 150 \degree  $
}


\end{enumerate}


\item 
\exwhere{$ 2016 $ 年江苏卷}
\begin{enumerate}
	%\renewcommand{\labelenumi}{\arabic{enumi}.}
	% A(\Alph) a(\alph) I(\Roman) i(\roman) 1(\arabic)
	%设定全局标号series=example	%引用全局变量resume=example
	%[topsep=-0.3em,parsep=-0.3em,itemsep=-0.3em,partopsep=-0.3em]
	%可使用leftmargin调整列表环境左边的空白长度 [leftmargin=0em]
	\item
一艘太空飞船静止时的长度为 $ 30 \ m $,它以 $ 0.6c(c $ 为光速)的速度沿长度方向飞行经过地球,下列
说法正确的是 \underlinegap 。
\fourchoices
{飞船上的观测者测得该飞船的长度小于 $ 30 \ m $}
{地球上的观测者测得该飞船的长度小于 $ 30 \ m $}
{飞船上的观测者测得地球上发来的光信号速度小于 $ c $}
{地球上的观测者测得飞船上发来的光信号速度小于 $ c $}

 \tk{B} 


\item 
杨氏干涉实验证明光的确是一种波,一束单色光投射在两条相距很近的狭缝上,两狭缝就成了
两个光源,它们发出的光波满足干涉的必要条件,则两列光的 \underlinegap 
相同。如图所示,在这两列光波相遇的区域中,实线表示波峰,虚线表
示波谷,如果放置光屏,在 \underlinegap (选填“$ A $”“$ B $”或“$ C $”)点会出现暗条
纹。
\begin{figure}[h!]
	\centering
	\includesvg[width=0.23\linewidth]{picture/svg/GZ-3-tiyou-1550}
\end{figure}

 \tk{频率 \quad $ C $} 


\item 
在上述杨氏干涉实验中,若单色光的波长$ \lambda =5.89 \times 10^{-7} \ m $,双缝间的距离 $ d=1 \ mm $,双缝到屏的距
离 $ l=2 \ m $。求第 $ 1 $ 个亮条纹到第 $ 11 $ 个亮条纹的中心间距。

\banswer{
	$1.178 \times 10^{-2} \ m$
}

	
\end{enumerate}


\item 
\exwhere{$ 2016 $ 年海南卷}
\begin{enumerate}
	%\renewcommand{\labelenumi}{\arabic{enumi}.}
	% A(\Alph) a(\alph) I(\Roman) i(\roman) 1(\arabic)
	%设定全局标号series=example	%引用全局变量resume=example
	%[topsep=-0.3em,parsep=-0.3em,itemsep=-0.3em,partopsep=-0.3em]
	%可使用leftmargin调整列表环境左边的空白长度 [leftmargin=0em]
	\item
下列说法正确的是 \underlinegap 。
\fivechoices
{在同一地点,单摆做简谐振动的周期的平方与其摆长成正比}
{弹簧振子做简谐振动时,振动系统的势能与动能之和保持不变}
{在同一地点,当摆长不变时,摆球质量越大,单摆做简谐振动的周期越小}
{系统做稳定的受迫振动时,系统振动的频率等于周期性驱动力的频率}
{已知弹簧振子初始时刻的位置及其振动周期,就可知振子在任意时刻运动速度的方向}



 \tk{ABD} 

\item 
如图,半径为 $ R $ 的半球形玻璃体置于水平桌面上,半球的上表面水平,球面与桌面相切
于 $ A $ 点。一细束单色光经球心 $ O $ 从空气中摄入玻璃体内(入射面即纸面),入射角为 $ 45 \degree $,出射光
线射在桌面上 $ B $ 点处。测得 $ AN $ 之间的距离为$ \frac{R}{2} $。现将入射光束在
纸面内向左平移,求射入玻璃体的光线在球面上恰好发生全反射
时,光束在上表面的入射点到 $ O $ 点的距离。不考虑光线在玻璃体
内的多次反射。
\begin{figure}[h!]
	\flushright
	\includesvg[width=0.25\linewidth]{picture/svg/GZ-3-tiyou-1551}
\end{figure}

\banswer{
	$O E=\frac{\sqrt{2}}{2} R$
}



\end{enumerate}


\item 
\exwhere{$ 2015 $ 年理综新课标  \lmd{1}  卷}
\begin{enumerate}
	%\renewcommand{\labelenumi}{\arabic{enumi}.}
	% A(\Alph) a(\alph) I(\Roman) i(\roman) 1(\arabic)
	%设定全局标号series=example	%引用全局变量resume=example
	%[topsep=-0.3em,parsep=-0.3em,itemsep=-0.3em,partopsep=-0.3em]
	%可使用leftmargin调整列表环境左边的空白长度 [leftmargin=0em]
	\item
在双缝干涉实验中,分别用红色和绿色的激光照射同一双缝,在双缝后的屏幕上,红光的干
涉条纹间距 $ \Delta x_{1} $ 与绿光的干涉条纹间距 $ \Delta x_{2} $ 相比 $ \Delta x_{1} $
 \underlinegap 
$ \Delta x_{2} $ (填“$ > $”“$ = $”或“$ < $”)。若实验中红光
的波长为 $ 630 \ nm $,双缝到屏幕的距离为 $ 1.00 \ m $,测得第 $ 1 $ 条到第 $ 6 $ 条亮条纹中心间的距离为
$ 10.5 \ mm $,则双缝之间的距离为
 \underlinegap 
$ mm $。

 \tk{$ > \quad 0.300 $} 



\item 
甲、乙两列简谐横波在同
一介质中分别沿 $ x $ 轴正向和负向传播,
波速均为 $ v=25 \ cm /s $,两列波在 $ t=0 $ 时的
波形曲线如图所示。求:
\begin{enumerate}
	%\renewcommand{\labelenumi}{\arabic{enumi}.}
	% A(\Alph) a(\alph) I(\Roman) i(\roman) 1(\arabic)
	%设定全局标号series=example	%引用全局变量resume=example
	%[topsep=-0.3em,parsep=-0.3em,itemsep=-0.3em,partopsep=-0.3em]
	%可使用leftmargin调整列表环境左边的空白长度 [leftmargin=0em]
	\item
$ t=0 $ 时,介质中偏离平衡位置位移为 $ 16 \ cm $ 的所有质点的 $ x $ 坐标;


\item 
从 $ t=0 $ 开始,介质中最早出现偏离平衡位置位移为$ -16 \ cm $ 的质点的时间。

	
\end{enumerate}
\begin{figure}[h!]
	\flushright
	\includesvg[width=0.25\linewidth]{picture/svg/GZ-3-tiyou-1552}
\end{figure}

\banswer{
	\begin{enumerate}
		%\renewcommand{\labelenumi}{\arabic{enumi}.}
		% A(\Alph) a(\alph) I(\Roman) i(\roman) 1(\arabic)
		%设定全局标号series=example	%引用全局变量resume=example
		%[topsep=-0.3em,parsep=-0.3em,itemsep=-0.3em,partopsep=-0.3em]
		%可使用leftmargin调整列表环境左边的空白长度 [leftmargin=0em]
		\item
		$x=(50+300 n) \ cm \quad(n=0,\pm 1,\pm 2, \pm 3 \ldots \ldots)$
		\item 
		$ 0.1 \ s $
	\end{enumerate}
}




\end{enumerate}


\item 
\exwhere{$ 2015 $ 年理综新课标 \lmd{2} 卷}
\begin{enumerate}
	%\renewcommand{\labelenumi}{\arabic{enumi}.}
	% A(\Alph) a(\alph) I(\Roman) i(\roman) 1(\arabic)
	%设定全局标号series=example	%引用全局变量resume=example
	%[topsep=-0.3em,parsep=-0.3em,itemsep=-0.3em,partopsep=-0.3em]
	%可使用leftmargin调整列表环境左边的空白长度 [leftmargin=0em]
	\item
如图,一束光沿半径方向射向一块半圆柱形玻璃砖,在玻璃砖底面上的入射角为$ \theta $,经
折射后射出 $ a $、$ b $ 两束光线。则 \underlinegap 。(填正确答案标号。选对 $ 1 $ 个得 $ 2 $ 分,选对 $ 2 $ 个得 $ 4 $ 分,选
对 $ 3 $ 个的 $ 5 $ 分。每选错 $ 1 $ 个扣 $ 3 $ 分,最低得分为 $ 0 $ 分。)
\begin{figure}[h!]
	\centering
	\includesvg[width=0.23\linewidth]{picture/svg/GZ-3-tiyou-1553}
\end{figure}

\fivechoices
{在玻璃中,$ a $ 光的传播速度小于 $ b $ 光的传播速度}
{在真空中,$ a $ 光的波长小于 $ b $ 光的波长}
{玻璃对 $ a $ 光的折射率小于对 $ b $ 光的折射率}
{若改变光束的入射方向使$ \theta $角逐渐变大,则折射光线 $ a $ 首先消失}
{分别用 $ a $、$ b $ 光在同一个双缝干涉实验装置上做实验,$ a $ 光的干涉条纹间距大于 $ b $ 光的干涉条纹间距}

 \tk{ABD} 

\item 
平衡位置位于原点 $ O $ 的波源发出的简谐横波在均匀介质中沿水平 $ x $ 轴传播,$ P $、$ Q $ 为 $ x $
轴上的两个点(均位于 $ x $ 轴正向),$ P $ 与 $ O $ 的距离为 $ 35 \ cm $,此距离介于一倍波长与二倍波长之间。
已知波源自 $ t=0 $ 时由平衡位置开始向上振动,周期 $ T=1 \ s $,振幅 $ A=5 \ cm $。当波传播到 $ P $ 点时,波源
恰好处于波峰位置;此后再经过 $ 5 \ s $,平衡位置在 $ Q $ 处的质点第一次处于波峰位置。求:
\begin{enumerate}
	%\renewcommand{\labelenumi}{\arabic{enumi}.}
	% A(\Alph) a(\alph) I(\Roman) i(\roman) 1(\arabic)
	%设定全局标号series=example	%引用全局变量resume=example
	%[topsep=-0.3em,parsep=-0.3em,itemsep=-0.3em,partopsep=-0.3em]
	%可使用leftmargin调整列表环境左边的空白长度 [leftmargin=0em]
	\item
$ P $、$ Q $ 间的距离;
\item 
从 $ t=0 $ 开始到平衡位置在 $ Q $ 处的质点第一次处于波峰位置时,波源在振动过程中通过的路
程。
\end{enumerate}

\banswer{
	\begin{enumerate}
		%\renewcommand{\labelenumi}{\arabic{enumi}.}
		% A(\Alph) a(\alph) I(\Roman) i(\roman) 1(\arabic)
		%设定全局标号series=example	%引用全局变量resume=example
		%[topsep=-0.3em,parsep=-0.3em,itemsep=-0.3em,partopsep=-0.3em]
		%可使用leftmargin调整列表环境左边的空白长度 [leftmargin=0em]
		\item
		$ PQ=133 \ cm $
		\item 
		$ s=125 \ cm $
	\end{enumerate}
}



\end{enumerate}

\item 
\exwhere{$ 2015 $ 年理综重庆卷}
\begin{enumerate}
	%\renewcommand{\labelenumi}{\arabic{enumi}.}
	% A(\Alph) a(\alph) I(\Roman) i(\roman) 1(\arabic)
	%设定全局标号series=example	%引用全局变量resume=example
	%[topsep=-0.3em,parsep=-0.3em,itemsep=-0.3em,partopsep=-0.3em]
	%可使用leftmargin调整列表环境左边的空白长度 [leftmargin=0em]
	\item
虹和霓是太阳光在水珠内分别经过一次和两次反射后出射形成的,可利用白光照射玻
璃球来说明。两束平行白光照射到透明玻璃球后,在水
平的白色桌面上会形成 $ MN $ 和 $ PQ $ 两条彩色光带,光路如图所示。$ M $、$ N $、$ P $、$ Q $ 点的颜色分别为 \xzanswer{A} 
\begin{figure}[h!]
	\centering
	\includesvg[width=0.23\linewidth]{picture/svg/GZ-3-tiyou-1554}
\end{figure}

\fourchoices
{紫、红、红、紫}
{红、紫、红、紫}
{红、紫、紫、红}
{紫、红、紫、红}



\item 
下图为一列沿 $ x $ 轴正方向传播的简谐机械横波某时刻的波形图,质点 $ P $ 的振动
周期为 $ 0.4 \ s $. 求该波的波速并判断 $ P $ 点此时的振动方向。
\begin{figure}[h!]
	\flushright
	\includesvg[width=0.25\linewidth]{picture/svg/GZ-3-tiyou-1556}
\end{figure}


\banswer{
	$ v=2.5 \ m/s  $ \quad ;$ P $ 点沿 $ y $ 轴正向振动
}


\end{enumerate}


\item 
\exwhere{$ 2015 $ 年江苏卷}
\begin{enumerate}
	%\renewcommand{\labelenumi}{\arabic{enumi}.}
	% A(\Alph) a(\alph) I(\Roman) i(\roman) 1(\arabic)
	%设定全局标号series=example	%引用全局变量resume=example
	%[topsep=-0.3em,parsep=-0.3em,itemsep=-0.3em,partopsep=-0.3em]
	%可使用leftmargin调整列表环境左边的空白长度 [leftmargin=0em]
	\item
一渔船向鱼群发出超声波,若鱼群正向渔船靠近,则被鱼群反射回来的超声波与发出的超声波
相比  \underlinegap 。
\fourchoices
{波速变大}
{波速不变}
{频率变高}
{频率不变}


 \tk{BC} 


\item 
用 $ 2 \times 10^{6} \ Hz $ 的超声波检查胆结石,该超声波在结石和胆汁中的波速分别为 $ 2250 \ m /s $ 和 $ 1500 \ m/s $,则该超声波在结石中的波长是胆汁中的 \underlinegap 倍。用超声波检查胆结石是因为超声波的波长
较短,遇到结石时  \underlinegap (选填“容易”或“不容易”)发生衍射。

 \tk{ $ 1.5 $  \quad 不容易} 


\item 
人造树脂是常用的眼镜镜片材料。 如图所示,光线射在一人造树脂立方体上,经折射后,射在
桌面上的 $ P $ 点。 已知光线的入射角为 $ 30 \degree $,$ OA=5 \ cm $,$ AB=20 \ cm $,
$ BP=12 \ cm $,求该人造树脂材料的折射率 $ n $。
\begin{figure}[h!]
	\flushright
	\includesvg[width=0.25\linewidth]{picture/svg/GZ-3-tiyou-1557}
\end{figure}

\banswer{
	$n=\frac{\sqrt{449}}{14} \quad$ (或 $\left.n \approx 1.5\right)$
}

	
\end{enumerate}


\item 
\exwhere{$ 2015 $ 年理综山东卷}
\begin{enumerate}
	%\renewcommand{\labelenumi}{\arabic{enumi}.}
	% A(\Alph) a(\alph) I(\Roman) i(\roman) 1(\arabic)
	%设定全局标号series=example	%引用全局变量resume=example
	%[topsep=-0.3em,parsep=-0.3em,itemsep=-0.3em,partopsep=-0.3em]
	%可使用leftmargin调整列表环境左边的空白长度 [leftmargin=0em]
	\item
如图,轻弹簧上端固定,下端连接一小物块,物块沿竖直方向做简谐运动。以竖直向上为正
方向,物块简谐运动的表达式为 $ y=0.1 \sin (2.5 \pi t) \ m $。$ t=0 $ 时刻,一小球从距物块 $ h $ 高处自由落下;
$ t=0.6 \ s $ 时,小球恰好与物块处于同一高度。取重力加速度的大小为 $ g=10 \ m/s^{2} $.以下判断正确的是 \underlinegap 
(双选,填正确答案标号)
\begin{figure}[h!]
	\centering
	\includesvg[width=0.23\linewidth]{picture/svg/GZ-3-tiyou-1558}
\end{figure}

\fourchoices
{$ h=1.7 \ m $}
{简谐运动的周期是 $ 0.8 \ s $}
{$ 0.6 \ s $ 内物块运动的路程是 $ 0.2 \ m $}
{$ t=0.4 \ s $ 时,物块与小球运动方向相反}

 \tk{AB} 

\item 
半径为 $ R $、介质折射率为 $ n $ 的透明圆柱体,过其轴线 $ OO ^{\prime} $ 的截面如图所示。位于截面所在平
面内的一细束光线,以角 $ i_{0} $ 由 $ O $ 点入射,折射光线由上边
界的 $ A $ 点射出。当光线在 $ O $ 点的入射角减小至某一值
时,折射光线在上边界的 $ B $ 点恰好发生全反射。求 $ A $、$ B $
两点间的距离。
\begin{figure}[h!]
	\flushright
	\includesvg[width=0.25\linewidth]{picture/svg/GZ-3-tiyou-1559}
\end{figure}

\banswer{
	$\Delta x=R\left(\frac{1}{\sqrt{n^{2}-1}}-\frac{\sqrt{n^{2}-\sin ^{2} i_{0}}}{\sin i_{0}}\right)$
}

	
\end{enumerate}

\item 
\exwhere{$ 2015 $ 年海南卷}
\begin{enumerate}
	%\renewcommand{\labelenumi}{\arabic{enumi}.}
	% A(\Alph) a(\alph) I(\Roman) i(\roman) 1(\arabic)
	%设定全局标号series=example	%引用全局变量resume=example
	%[topsep=-0.3em,parsep=-0.3em,itemsep=-0.3em,partopsep=-0.3em]
	%可使用leftmargin调整列表环境左边的空白长度 [leftmargin=0em]
	\item
一列沿 $ x $ 轴正方向传播的简谱横波在 $ t=0 $ 时刻的波形如图所示,质点 $ P $ 的 $ x $ 坐标为
$ 3 \ m $.。已知任意振动质点连续 $ 2 $ 次经过平衡位置的时间间隔为 $ 0.4 \ s $。下列说法正确的是
 \underlinegap 
(填正确
答案标号。选对 $ 1 $ 个得 $ 2 $ 分,选对 $ 2 $ 个得 $ 3 $ 分,选 $ 3 $ 个得 $ 4 $ 分;每选错 $ 1 $ 个扣 $ 2 $ 分,最低得分为 $ 0 $
分)。
\begin{figure}[h!]
	\centering
	\includesvg[width=0.23\linewidth]{picture/svg/GZ-3-tiyou-1560}
\end{figure}

\fivechoices
{波速为 $ 4 \ m /s $}
{波的频率为 $ 1.25 \ Hz $}
{$ x $ 坐标为 $ 15 \ m $ 的质点在 $ t=0.6 \ s $ 时恰好位于波谷}
{$ x $ 的坐标为 $ 22 \ m $ 的质点在 $ t=0.2 \ s $ 时恰好位于波峰}
{当质点 $ P $ 位于波峰时,$ x $ 坐标为 $ 17 \ m $ 的质点恰好位于波谷}

 \tk{BDE} 


\item 
一半径为 $ R $ 的半圆形玻璃砖,横截面如图所示。已
知玻璃的全反射临界角 $ r $($ r<\frac{\pi}{3} $)。与玻璃砖的底平面成
($ \frac{\pi}{2} -r $ )角度、且与玻璃砖横截面平行的平行光射到玻璃砖的
半圆柱面上。经柱面折射后,有部分光(包括与柱面相切的入
射光)能直接从玻璃砖底面射出。若忽略经半圆柱内表面反射后射出的光,求底面透光部分的宽
度。
\begin{figure}[h!]
	\flushright
	\includesvg[width=0.25\linewidth]{picture/svg/GZ-3-tiyou-1561}
\end{figure}

\banswer{
$ OE=R \sin r $	
}


	
\end{enumerate}



	
	
	
\end{enumerate}

