\bta{感应电流的方向 }


\begin{enumerate}
\renewcommand{\labelenumi}{\arabic{enumi}.}
% A(\Alph) a(\alph) I(\Roman) i(\roman) 1(\arabic)
%设定全局标号series=example	%引用全局变量resume=example
%[topsep=-0.3em,parsep=-0.3em,itemsep=-0.3em,partopsep=-0.3em]
%可使用leftmargin调整列表环境左边的空白长度 [leftmargin=0em]
\item
\exwhere{$ 2014 $年物理上海卷}
如图,匀强磁场垂直于软导线回路平面,由于磁场发生变化,回路变为圆形。则该磁场 \xzanswer{CD} 


\begin{minipage}[h!]{0.7\linewidth}
\vspace{0.3em}
\fourchoices
{逐渐增强,方向向外 }
{逐渐增强,方向向里}
{逐渐减弱,方向向外 }
{逐渐减弱,方向向里}
\vspace{0.3em}
\end{minipage}
\hfill
\begin{minipage}[h!]{0.3\linewidth}
\flushright
\vspace{0.3em}
\includesvg[width=0.38\linewidth]{picture/svg/329}
\vspace{0.3em}
\end{minipage}



\item 
\exwhere{$ 2012 $年海南卷}
如图,一质量为$ m $的条形磁铁用细线悬挂在天花板上,细线从一水平金属圆环中穿过。现将环从位置 \lmd{1} 释放,环经过磁铁到达位置 \lmd{2} 。设环经过磁铁上端和下端附近时细线的张力分别为$ T_{1} $和$ T_{2} $,重力加速度大小为$ g $,则 \xzanswer{A} 

\begin{minipage}[h!]{0.7\linewidth}
\vspace{0.3em}
\fourchoices
{$ T _ { 1 } > m g , T _ { 2 } > m g $}
{$ T _ { 1 } < m g , T _ { 2 } < m g $}
{$ T _ { 1 } > m g , T _ { 2 } < m g $}
{$ T _ { 1 } < m g , T _ { 2 } > m g $}

\vspace{0.3em}
\end{minipage}
\hfill
\begin{minipage}[h!]{0.3\linewidth}
\flushright
\vspace{0.3em}
\includesvg[width=0.27\linewidth]{picture/svg/330}
\vspace{0.3em}
\end{minipage}

\item 
\exwhere{$ 2014 $年物理海南卷}
如图,在一水平、固定的闭合导体圆环上方。有一条形磁铁$ (N $极朝上, $ S $极朝下)由静止开始下落,磁铁从圆环中穿过且不与圆环接触,关于圆环中感应电流的方向(从上向下看),下列说法正确的是 \xzanswer{C} 


\begin{minipage}[h!]{0.7\linewidth}
\vspace{0.3em}
\fourchoices
{总是顺时针}
{总是逆时针}
{先顺时针后逆时针}
{先逆时针后顺时针}

\vspace{0.3em}
\end{minipage}
\hfill
\begin{minipage}[h!]{0.3\linewidth}
\flushright
\vspace{0.3em}
\includesvg[width=0.38\linewidth]{picture/svg/331}
\vspace{0.3em}
\end{minipage}

\item 
\exwhere{$ 2013 $年上海卷}
如图,通电导线$ MN $与单匝矩形线圈$ abcd $共面,位置靠近$ ab $且相互绝缘。当$ MN $中电流突然减小时,线圈所受安培力的合力方向 \xzanswer{B} 


\begin{minipage}[h!]{0.7\linewidth}
\vspace{0.3em}
\fourchoices
{向左}
{向右}
{垂直纸面向外}
{垂直纸面向里}
\vspace{0.3em}
\end{minipage}
\hfill
\begin{minipage}[h!]{0.3\linewidth}
\flushright
\vspace{0.3em}
\includesvg[width=0.5\linewidth]{picture/svg/332}
\vspace{0.3em}
\end{minipage}

\item 
\exwhere{$ 2013 $年海南卷}
如图,在水平光滑桌面上,两相同的矩形刚性小线圈分别叠放在固定的绝缘矩形金属框的左右两边上,且每个小线圈都各有一半面积在金属框内,在金属框接通逆时针方向电流的瞬间 \xzanswer{BC} 
\begin{figure}[h!]
\centering
\includesvg[width=0.23\linewidth]{picture/svg/333}
\end{figure}


\fourchoices
{两小线圈会有相互靠拢的趋势}
{两小线圈会有相互远离的趋势}
{两小线圈中感应电流都沿顺时针方向}
{左边小线圈中感应电流沿顺时针方向,右边小线圈中感应电流沿逆时针方向}

\item 
\exwhere{$ 2016 $年北京卷}
如图所示,匀强磁场中有两个导体圆环$ a $、$ b $,磁场方向与圆环所在平面垂直。磁感应强度$ B $随时间均匀增大。两圆坏半径之比为$ 2:1 $,圆环中产生的感应电动势分别为$ E_a $和$ E_b $,不考虑两圆环间的相互影响。下列说法正确的是 \xzanswer{B} 
\begin{figure}[h!]
\centering
\includesvg[width=0.23\linewidth]{picture/svg/334}
\end{figure}


\fourchoices
{$ E_a:E_b=4:1 $,感应电流均沿逆时针方向}
{$ E_a:E_b=4:1 $,感应电流均沿顺时针方向}
{$ E_a:E_b=2:1 $,感应电流均沿逆时针方向}
{$ E_a:E_b=2:1 $,感应电流均沿顺时针方向}



\item 
\exwhere{$ 2016 $年上海卷}
磁铁在线圈中心上方开始运动时,线圈中产生如图方向的感应电流,则磁铁 \xzanswer{B} 

\begin{minipage}[h!]{0.7\linewidth}
\vspace{0.3em}
\fourchoices
{向上运动 }
{向下运动}
{向左运动}
{向右运动}
\vspace{0.3em}
\end{minipage}
\hfill
\begin{minipage}[h!]{0.3\linewidth}
\flushright
\vspace{0.3em}
\includesvg[width=0.5\linewidth]{picture/svg/335}
\vspace{0.3em}
\end{minipage}





\item 
\exwhere{$ 2016 $年浙江卷}
如图所示,$ a $、$ b $两个闭合正方形线圈用同样的导线制成,匝数均为$ 10 $匝,边长$ l_a=3l_b $,图示区域内有垂直纸面向里的均强磁场,且磁感应强度随时间均匀增大,不考虑线圈之间的相互影响,则 \xzanswer{B} 
\begin{figure}[h!]
\centering
\includesvg[width=0.23\linewidth]{picture/svg/336}
\end{figure}


\fourchoices
{两线圈内产生顺时针方向的感应电流}
{$ a $、$ b $线圈中感应电动势之比为$ 9:1 $}
{$ a $、$ b $线圈中感应电流之比为$ 3:4 $}
{$ a $、$ b $线圈中电功率之比为$ 3:1 $}




\item 
\exwhere{$ 2016 $年海南卷}
如图,一圆形金属环与两固定的平行长直导线在同一竖直平面内,环的圆心与两导线距离相等,环的直径小于两导线间距。两导线中通有大小相等、方向向下的恒定电流。若 \xzanswer{D} 


\begin{minipage}[h!]{0.7\linewidth}
\vspace{0.3em}
\fourchoices
{金属环向上运动,则环上的感应电流方向为顺时针方向}
{金属环向下运动,则环上的感应电流方向为顺时针方向}
{金属环向左侧直导线靠近,则环上的感应电流方向为逆时针}
{金属环向右侧直导线靠近,则环上的感应电流方向为逆时针}

\vspace{0.3em}
\end{minipage}
\hfill
\begin{minipage}[h!]{0.3\linewidth}
\flushright
\vspace{0.3em}
\includesvg[width=0.5\linewidth]{picture/svg/337}
\vspace{0.3em}
\end{minipage}


\item 
\exwhere{$ 2017 $年新课标 \lmd{2} 卷}
两条平行虚线间存在一匀强磁场,磁感应强度方向与纸面垂直。边长为$ 0.1 \ m $、总电阻为$ 0.005 $ $ \Omega $的正方形导线框$ abcd $位于纸面内,$ cd $边与磁场边界平行,如图($ a $)所示。已知导线框一直向右做匀速直线运动,$ cd $边于$ t=0 $时刻进入磁场。线框中感应电动势随时间变化的图线如图($ b $)所示(感应电流的方向为顺时针时,感应电动势取正)。下列说法正确的是 \xzanswer{BC} 
\begin{figure}[h!]
\centering
\includesvg[width=0.46\linewidth]{picture/svg/338}
\end{figure}


\fourchoices
{磁感应强度的大小为$ 0.5 $ $ T $}
{导线框运动的速度的大小为$ 0.5 \ m/s $}
{磁感应强度的方向垂直于纸面向外}
{在$ t=0.4 $ $ s $至$ t=0.6 $ $ s $这段时间内,导线框所受的安培力大小为$ 0.1 $ $ N $}





\item 
\exwhere{$ 2017 $年新课标 \lmd{3} 卷}
如图,在方向垂直于纸面向里的匀强磁场中有一$ U $形金属导轨,导轨平面与磁场垂直。金属杆$ PQ $置于导轨上并与导轨形成闭合回路$ PQRS $,一圆环形金属框$ T $位于回路围成的区域内,线框与导轨共面。现让金属杆$ PQ $突然向右运动,在运动开始的瞬间,关于感应电流的方向,下列说法正确的是 \xzanswer{D} 
\begin{figure}[h!]
\centering
\includesvg[width=0.23\linewidth]{picture/svg/339}
\end{figure}


\fourchoices
{$ PQRS $中沿顺时针方向,$ T $中沿逆时针方向}
{$ PQRS $中沿顺时针方向,$ T $中沿顺时针方向}
{$ PQRS $中沿逆时针方向,$ T $中沿逆时针方向}
{$ PQRS $中沿逆时针方向,$ T $中沿顺时针方向}


\item 
\exwhere{$ 2016 $年上海卷}
如图($ a $),螺线管内有平行于轴线的外加匀强磁场,以图中箭头所示方向为其正方向。螺线管与导线框$ abcd $相连,导线框内有一小金属圆环$ L $,圆环与导线框在同一平面内。当螺线管内的磁感应强度$ B $随时间按图($ b $)所示规律变化时 \xzanswer{AD} 
\begin{figure}[h!]
\centering
\includesvg[width=0.4\linewidth]{picture/svg/340}
\end{figure}



\fourchoices
{在$ t_1 \sim t_2 $时间内,$ L $有收缩趋势}
{在$ t_2 \sim t_3 $时间内,$ L $有扩张趋势}
{在$ t_2 \sim t_3 $时间内,$ L $内有逆时针方向的感应电流}
{在$ t_3 \sim t_4 $时间内,$ L $内有顺时针方向的感应电流}



\item 
\exwhere{$ 2018 $年全国\lmd{1}卷}
如图,两个线圈绕在同一根铁芯上,其中一线圈通过开关与电源连接,另一线圈与远处沿南北方向水平放置在纸面内的直导线连接成回路。将一小磁针悬挂在直导线正上方,开关未闭合时小磁针处于静止状态。下列说法正确的是 \xzanswer{AD} 
\begin{figure}[h!]
\centering
\includesvg[width=0.2\linewidth]{picture/svg/341}
\end{figure}


\fourchoices
{开关闭合后的瞬间,小磁针的$ N $极朝垂直纸面向里的方向转动}
{开关闭合并保持一段时间后,小磁针的$ N $极指向垂直纸面向里的方向}
{开关闭合并保持一段时间后,小磁针的$ N $极指向垂直纸面向外的方向}
{开关闭合并保持一段时间再断开后的瞬间,小磁针的$ N $极朝垂直纸面向外的方向转动}


\item 
\exwhere{$ 2016 $年江苏卷}
电吉他中电拾音器的基本结构如图所示,磁体附近的金属弦被磁化,因此弦振动时,在线圈中产生感应电流,电流经电路放大后传送到音箱发出声音,下列说法正确的有 \xzanswer{BCD} 


\begin{minipage}[h!]{0.7\linewidth}
\vspace{0.3em}
\fourchoices
{选用铜质弦,电吉他仍能正常工作}
{取走磁体,电吉他将不能正常工作}
{增加线圈匝数可以增大线圈中的感应电动势}
{磁振动过程中,线圈中的电流方向不断变化}

\vspace{0.3em}
\end{minipage}
\hfill
\begin{minipage}[h!]{0.3\linewidth}
\flushright
\vspace{0.3em}
\includesvg[width=0.8\linewidth]{picture/svg/342}
\vspace{0.3em}
\end{minipage}




\newpage
\item 
\exwhere{$ 2011 $年上海卷}
如图,均匀带正电的绝缘圆环$ a $与金属圆环$ b $同心共面放置。当$ a $绕$ O $点在其所在平面内旋转时,$ b $中产生顺时针方向的感应电流,且具有收缩趋势,由此可知,圆环$ a $ \xzanswer{B} 

\begin{figure}[h!]
\centering
\includesvg[width=0.23\linewidth]{picture/svg/343}
\end{figure}


\fourchoices
{顺时针加速旋转}
{顺时针减速旋转}
{逆时针加速旋转}
{逆时针减速旋转}

\item 
\exwhere{$ 2015 $年理综重庆卷}
下图为无线充电技术中使用的受电线圈示意图,线圈匝数为$ n $,面积为$ S $.若在$ t_{1} $到$ t_{2} $时间内,匀强磁场平行于线圈轴线向右穿过线圈,其磁感应强度大小由$ B_{1} $均匀增加到$ B_{2} $,则该段时间线圈两端$ a $和$ b $之间的电势差 $\varphi _ { a } - \varphi _ { b }$ \xzanswer{C} 
\begin{figure}[h!]
\centering
\includesvg[width=0.23\linewidth]{picture/svg/345}
\end{figure}


\fourchoices
{恒为$\frac { n S \left( B _ { 2 } - B _ { 1 } \right) } { t _ { 2 } - t _ { 1 } }$}
{从$ 0 $均匀变化到$\frac { n S \left( B _ { 2 } - B _ { 1 } \right) } { t _ { 2 } - t _ { 1 } }$}
{恒为$-\frac { n S \left( B _ { 2 } - B _ { 1 } \right) } { t _ { 2 } - t _ { 1 } }$}
{从$ 0 $均匀变化到$-\frac { n S \left( B _ { 2 } - B _ { 1 } \right) } { t _ { 2 } - t _ { 1 } }$}

\item 
\exwhere{$ 2015 $年理综山东卷}
如图,一均匀金属圆盘绕通过其圆心且与盘面垂直的轴逆时针匀速转动。现施加一垂直穿过圆盘的有界匀强磁场,圆盘开始减速。在圆盘减速过程中,以下说法正确的是 \xzanswer{ABD} 
\begin{figure}[h!]
\centering
\includesvg[width=0.23\linewidth]{picture/svg/346}
\end{figure}

\fourchoices
{处于磁场中的圆盘部分,靠近圆心处电势高}
{所加磁场越强越易使圆盘停止转动}
{若所加磁场反向,圆盘将加速转动}
{若所加磁场穿过整个圆盘,圆盘将匀速转动}

\item 
\exwhere{$ 2011 $年上海卷}
如图,磁场垂直于纸面,磁感应强度在竖直方向均匀分布,水平方向非均匀分布。一铜制圆环用丝线悬挂于$ O $点,将圆环拉至位置$ a $后无初速释放,在圆环从$ a $摆向$ b $的过程中 \xzanswer{AD} 
\begin{figure}[h!]
\centering
\includesvg[width=0.23\linewidth]{picture/svg/347}
\end{figure}


\fourchoices
{感应电流方向先逆时针后顺时针再逆时针}
{感应电流方向一直是逆时针}
{安培力方向始终与速度方向相反}
{安培力方向始终沿水平方向}


\item 
\exwhere{$ 2016 $年新课标 \lmd{3} 卷}
如图,$ M $为半圆形导线框,圆心为$ O_M $;$ N $是圆心角为直角的扇形导线框,圆心为$ O_N $;两导线框在同一竖直面(纸面)内;两圆弧半径相等;过直线$ O_MO_N $的水平面上方有一匀强磁场,磁场方向垂直于纸面。现使线框$ M $、$ N $在$ t=0 $时从图示位置开始,分别绕垂直于纸面、且过$ O_M $和$ O_N $的轴,以相同的周期$ T $逆时针匀速转动,则 \xzanswer{BC} 
\begin{figure}[h!]
\centering
\includesvg[width=0.23\linewidth]{picture/svg/348}
\end{figure}


\fourchoices
{两导线框中均会产生正弦交流电}
{两导线框中感应电流的周期都等于$ T $}
{在$ t=\frac{T}{8} $时,两导线框中产生的感应电动势相等}
{两导线框的电阻相等时,两导线框中感应电流的有效值也相等}



\item 
\exwhere{$ 2019 $年物理全国\lmd{3}卷}
楞次定律是下列哪个定律在电磁感应现象中的具体体现?

\fourchoices
{电阻定律}
{库仑定律}
{欧姆定律}
{能量守恒定律}




\newpage
\item 
\exwhere{$ 2016 $年江苏卷}
据报道,一法国摄影师拍到“天宫一号”空间站飞过太阳的瞬间.照片中,“天宫一号”的太阳帆板轮廓清晰可见。如图所示,假设“天宫一号”正以速度$ v=7.7 $ $ k \ m/s $绕地球做匀速圆周运动,运动方向与太阳帆板两端$ M $、$ N $的连线垂直,$ M $、$ N $间的距离$ L=20 $ $ m $,地磁场的磁感应强度垂直于$ v $,$ MN $所在平面的分量$ B=1.0 \times 10^{-5} $ $ T $,将太阳帆板视为导体。
\begin{enumerate}
\renewcommand{\labelenumii}{(\arabic{enumii})}

\item 
求$ M $、$ N $间感应电动势的大小$ E $;

\item 
在太阳帆板上将一只“$ 1.5 $ $ V $,$ 0.3 $ $ W $”的小灯泡与$ M $、$ N $相连构成闭合电路,不计太阳帆板和导线的电阻.试判断小灯泡能否发光,并说明理由;

\item 
取地球半径$ R=6.4 \times 10^3 $ $ km $,地球表面的重力加速度$ g=9.8 $ $ \ m/s ^{2} $,试估算“天宫一号”距离地球表面的高度$ h $(计算结果保留一位有效数字)。

\end{enumerate}
\begin{figure}[h!]
\flushright 
\includesvg[width=0.23\linewidth]{picture/svg/344}
\end{figure}

\banswer{
\begin{enumerate}
\renewcommand{\labelenumi}{\arabic{enumi}.}
% A(\Alph) a(\alph) I(\Roman) i(\roman) 1(\arabic)
%设定全局标号series=example	%引用全局变量resume=example
%[topsep=-0.3em,parsep=-0.3em,itemsep=-0.3em,partopsep=-0.3em]
%可使用leftmargin调整列表环境左边的空白长度 [leftmargin=0em]
\item
$ E=1.54\ V $
\item 
不能,因为穿过闭合回路的磁通量不变,不产生感应电流。
\item 
$h = \frac { g R ^ { 2 } } { v ^ { 2 } } - R$ \qquad $h \approx 4 \times 10 ^ { 5 } \mathrm { m }$(数量级正确都算对)


\end{enumerate}


}


\item 
\exwhere{$ 2013 $年天津卷}
超导现象是$ 20 $世纪人类重大发现之一,日前我国己研制出世界传输电流最大的高温超导电缆并成功示范运行。

\begin{enumerate}
\renewcommand{\labelenumi}{\arabic{enumi}.}
% A(\Alph) a(\alph) I(\Roman) i(\roman) 1(\arabic)
%设定全局标号series=example	%引用全局变量resume=example
%[topsep=-0.3em,parsep=-0.3em,itemsep=-0.3em,partopsep=-0.3em]
%可使用leftmargin调整列表环境左边的空白长度 [leftmargin=0em]
\item
超导体在温度特别低时电阻可以降到几乎为零,这种性质可以通过实验研究。将一个闭合超导金属圆环水平放置在匀强磁场中,磁感线垂直于圆环平面向上,逐渐降低温度使环发生由正常态到超导态的转变后突然撤去磁场,若此后环中的电流不随时间变化,则表明其电阻为零。请指出自上往下看环中电流方向,并说明理由。
\item 
为探究该圆环在超导状态的电阻率上限$ \rho $,研究人员测得撤去磁场后环中电流为$ I $,并经一年以上的时间$ t $未检测出电流变化。实际上仪器只能检测出大于 $ \Delta I $的电流变化,其中$ \Delta I << I $,当电流的变化小于 $\Delta I $时,仪器检测不出电流的变化,研究人员便认为电流没有变化。设环的横截面积为$ S $,环中定向移动电子的平均速率为$ v $,电子质量为$ m $、电荷量为$ e $。试用上述给出的各物理量,推导出$ \rho $的表达式。
\item 
若仍使用上述测量仪器,实验持续时间依旧为$ t $,为使实验获得的该圆环在超导状态的电阻率上限$ \rho $的准确程度更高,请提出你的建议,并简要说明实现方法。



\end{enumerate}

\banswer{
\begin{enumerate}
\renewcommand{\labelenumi}{\arabic{enumi}.}
% A(\Alph) a(\alph) I(\Roman) i(\roman) 1(\arabic)
%设定全局标号series=example	%引用全局变量resume=example
%[topsep=-0.3em,parsep=-0.3em,itemsep=-0.3em,partopsep=-0.3em]
%可使用leftmargin调整列表环境左边的空白长度 [leftmargin=0em]
\item
逆时针方向。撤去磁场瞬间,环所围面积的磁通量突变为零,由楞次定律可知,环中电流的磁场方向应与原磁场方向相同,即向上,由右手螺旋定则可知,环中电流的方向是沿逆时针方向。	
\item 
$\rho = \frac { m v S \Delta I } { e t I ^ { 2 } }$

\item 
由$\rho = \frac { m v S \Delta I } { e t I ^ { 2 } }$看出,在题设条件限制下,适当增大超导电流,可以使实验获得$ \rho $的准确程度更高,通过增大穿过该环的磁通量变化率可实现增大超导电流。

\end{enumerate}


}








\end{enumerate}



