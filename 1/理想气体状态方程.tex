\bta{理想气体状态方程}

\begin{enumerate}[leftmargin=0em]
\renewcommand{\labelenumi}{\arabic{enumi}.}
% A(\Alph) a(\alph) I(\Roman) i(\roman) 1(\arabic)
%设定全局标号series=example	%引用全局变量resume=example
%[topsep=-0.3em,parsep=-0.3em,itemsep=-0.3em,partopsep=-0.3em]
%可使用leftmargin调整列表环境左边的空白长度 [leftmargin=0em]
\item
\exwhere{$ 2012 $年理综重庆卷}
下图为伽利略设计的一种测温装置示意图,玻璃管的上端与导热良好的玻璃泡连通,下端插入水中,玻璃泡中封闭有一定量的空气。若玻璃管中水柱上升,则外界大气的变化可能是 \xzanswer{A} 


\begin{minipage}[h!]{0.7\linewidth}
\vspace{0.3em}
\fourchoices
{温度降低,压强增大}
{温度升高,压强不变}
{温度升高,压强减小}
{温度不变,压强减小}

\vspace{0.3em}
\end{minipage}
\hfill
\begin{minipage}[h!]{0.3\linewidth}
\flushright
\vspace{0.3em}
\includesvg[width=0.5\linewidth]{picture/svg/258}
\vspace{0.3em}
\end{minipage}

\item 
\exwhere{$ 2011 $年上海卷}
如图所示,一定量的理想气体从状态$ a $沿直线变化到状态$ b $,在此过程中,其压强 \xzanswer{A} 
\begin{figure}[h!]
\centering
\includesvg[width=0.23\linewidth]{picture/svg/259}
\end{figure}

\fourchoices
{逐渐增大}
{逐渐减小}
{始终不变}
{先增大后减小}


\item 
\exwhere{$ 2013 $年上海卷}
如图,柱形容器内用不漏气的轻质绝热活塞封闭一定量的理想气体,容器外包裹保温材料。开始时活塞至容器底部的高度为$ H_{1} $,容器内气体温度与外界温度相等。在活塞上逐步加上多个砝码后,活塞下降到距容器底部$ H_{2} $处,气体温度升高了$ \triangle T $;然后取走容器外的保温材料,活塞位置继续下降,最后静止于距容器底部$ H_{3} $处:已知大气压强为$ p_{0} $。求:气体最后的压强与温度。
\begin{figure}[h!]
\flushright
\includesvg[width=0.25\linewidth]{picture/svg/261}
\end{figure}

\banswer{
$P _ { 3 } = \frac { H _ { 1 } } { H _ { 3 } } P _ { 0 }$ \qquad $T _ { 0 } = \frac { H _ { 3 } } { H _ { 2 } - H _ { 3 } } \Delta T$.
}


\item 
\exwhere{$ 2014 $年理综大纲卷}
对于一定量的稀薄气体,下列说法正确的是 \xzanswer{BD} 

\fourchoices
{压强变大时,分子热运动必然变得剧烈}
{保持压强不变时,分子热运动可能变得剧烈}
{压强变大时,分子间的平均距离必然变小}
{压强变小时,分子间的平均距离可能变小}


\item 
\exwhere{$ 2011 $年上海卷}
如图,绝热气缸$ A $与导热气缸$ B $均固定于地面,由刚性杆连接的绝热活塞与两气缸间均无摩擦。两气缸内装有处于平衡状态的理想气体,开始时体积均为$ V_{0} $、温度均为$ T_{0} $。缓慢加热$ A $中气体,停止加热达到稳定后,$ A $中气体压强为原来的$ 1.2 $倍。设环境温度始终保持不变,求气缸$ A $中气体的体积$ V_A $和温度$ T_A $。
\begin{figure}[h!]
\flushright
\includesvg[width=0.25\linewidth]{picture/svg/262}
\end{figure}

\banswer{
$V _ { A } = \frac { 7 } { 6 } V _ { 0 }$, \qquad $T _ { A } = 1.4 T _ { 0 }$.
}


\item 
\exwhere{$ 2014 $年物理上海卷}
在“用$ DIS $研究在温度不变时,一定质量的气体压强与体积的关系”实验中,某同学将注射器活塞置于刻度为$ 10\ ml $处,然后将往射器连接压强传感器并开始实验,气体体积$ V $每增加$ 1\ ml $测一次压强$ p $,最后得到$ p $和$ V $的乘积逐渐增大。
\begin{enumerate}
\renewcommand{\labelenumii}{(\arabic{enumii})}

\item 
由此可推断,该同学的实验结果可能为图\tk{a}。
\begin{figure}[h!]
\centering
\includesvg[width=0.4\linewidth]{picture/svg/263}
\end{figure}

\item 
(单选题)图线弯曲的可能原因是在实验过程中 \xzanswer{C} 


\fourchoices
{注射器中有异物}
{连接软管中存在气体}
{注射器内气体温度升高}
{注射器内气体温度降低}


\end{enumerate}

\item 
\exwhere{$ 2014 $年物理上海卷}
如图,一端封闭、粗细均匀的$ U $形玻璃管开口向上竖直放置,管内用水银将一段气体封闭在管中。当温度为$ 280\ K $时,被封闭的气柱长$ L=22 \ cm $,两边水银柱高度差$ h=16 \ cm $,大气压强$ p_0=76 \ cmHg $。
\begin{enumerate}
\renewcommand{\labelenumii}{(\arabic{enumii})}

\item 
为使左端水银面下降$ 3 \ cm $,封闭气体温度应变为多少?

\item 
封闭气体的温度重新回到$ 280\ K $后,为使封闭气柱长度变为$ 20 \ cm $,需向开口端注入的水银柱长度为多少?


\end{enumerate}

\begin{figure}[h!]
\flushright
\includesvg[width=0.13\linewidth]{picture/svg/264}
\end{figure}

\banswer{
\begin{enumerate}
\renewcommand{\labelenumi}{\arabic{enumi}.}
% A(\Alph) a(\alph) I(\Roman) i(\roman) 1(\arabic)
%设定全局标号series=example	%引用全局变量resume=example
%[topsep=-0.3em,parsep=-0.3em,itemsep=-0.3em,partopsep=-0.3em]
%可使用leftmargin调整列表环境左边的空白长度 [leftmargin=0em]
\item
$T _ { 2 } = \frac { p _ { 2 } V _ { 2 } } { p _ { 1 } V } T _ { 1 } = \frac { 66 \times 25 } { 60 \times 22 } \times 280 \mathrm { K } = 350 \mathrm { K }$
\item 
$l = 10\ \mathrm { cm }$



\end{enumerate}
}


\item 
\exwhere{$ 2015 $年上海卷}
如图,气缸左右两侧气体由绝热活塞隔开,活塞与气缸光滑接触。初始时两侧气体均处于平衡态,体积之比$ V_{1} : V_2=1 : 2 $,温度之比$ T_{1} : T_2=2 : 5 $。先保持右侧气体温度不变,升高左侧气体温度,使两侧气体体积相同;然后使活塞导热,两侧气体最后达到平衡。求:
\begin{enumerate}
\renewcommand{\labelenumi}{\arabic{enumi}.}
% A(\Alph) a(\alph) I(\Roman) i(\roman) 1(\arabic)
%设定全局标号series=example	%引用全局变量resume=example
%[topsep=-0.3em,parsep=-0.3em,itemsep=-0.3em,partopsep=-0.3em]
%可使用leftmargin调整列表环境左边的空白长度 [leftmargin=0em]
\item
两侧气体体积相同时,左侧气体的温度与初始温度之比;
\item 
最后两侧气体的体积之比。



\end{enumerate}

\begin{figure}[h!]
\flushright
\includesvg[width=0.25\linewidth]{picture/svg/265}
\end{figure}


\banswer{
\begin{enumerate}
\renewcommand{\labelenumi}{\arabic{enumi}.}
% A(\Alph) a(\alph) I(\Roman) i(\roman) 1(\arabic)
%设定全局标号series=example	%引用全局变量resume=example
%[topsep=-0.3em,parsep=-0.3em,itemsep=-0.3em,partopsep=-0.3em]
%可使用leftmargin调整列表环境左边的空白长度 [leftmargin=0em]
\item
2
\item 
$ \dfrac{V_{1} ^{\prime} }{V_{2} ^{\prime} } = \dfrac{ 5 }{ 4 } $



\end{enumerate}
}



\item 
\exwhere{$ 2016 $年上海卷}
某同学制作了一个结构如图($ a $)所示的温度计。一端封闭的轻质细管可绕封闭端$ O $自由转动,管长$ 0.5\ m $。将一量程足够大的力传感器调零,细管的开口端通过细线挂于力传感器挂钩上,使细管保持水平、细线沿竖直方向。在气体温度为$ 270\ K $时,用一段水银将长度为$ 0.3\ m $的气柱封闭在管内。实验时改变气体温度,测得封闭气柱长度$ l $和力传感器读数$ F $之间的关系如图($ b $)所示(实验中大气压强不变)。
\begin{enumerate}
\renewcommand{\labelenumi}{\arabic{enumi}.}
% A(\Alph) a(\alph) I(\Roman) i(\roman) 1(\arabic)
%设定全局标号series=example	%引用全局变量resume=example
%[topsep=-0.3em,parsep=-0.3em,itemsep=-0.3em,partopsep=-0.3em]
%可使用leftmargin调整列表环境左边的空白长度 [leftmargin=0em]
\item
管内水银柱长度为 \tk{0.1} $ m $,为保证水银不溢出,该温度计能测得的最高温度为 \tk{360} $ K $。
\item 
若气柱初始长度大于$ 0.3m $,该温度计能测量的最高温度将\tk{减小}
(选填:“增大”,“不变”或“减小”)。
\item 
若实验中大气压强略有升高,则用该温度计测出的温度将 \tk{偏低} (选填:“偏高”,“不变”或“偏低”)。




\end{enumerate}
\begin{figure}[h!]
\centering
\includesvg[width=0.5\linewidth]{picture/svg/266}
\end{figure}



\item 
\exwhere{$ 2016 $年上海卷}
如图,两端封闭的直玻璃管竖直放置,一段水银将管内气体分隔为上下两部分$ A $和$ B $,上下两部分气体初始温度相等,且体积$ V_A > V_B $。
\begin{enumerate}
\renewcommand{\labelenumi}{\arabic{enumi}.}
% A(\Alph) a(\alph) I(\Roman) i(\roman) 1(\arabic)
%设定全局标号series=example	%引用全局变量resume=example
%[topsep=-0.3em,parsep=-0.3em,itemsep=-0.3em,partopsep=-0.3em]
%可使用leftmargin调整列表环境左边的空白长度 [leftmargin=0em]
\item
若$ A $、$ B $两部分气体同时升高相同的温度,水银柱将如何移动?
某同学解答如下:

设两部分气体压强不变,由$\frac { V _ { 1 } } { T _ { 1 } } = \frac { V _ { 2 } } { T _ { 2 } }$,$ \cdots $,$\Delta V = \frac { \Delta T } { T } V$,$ \cdots $,所以水银柱将向下移动。

上述解答是否正确?若正确,请写出完整的解答;若不正确,请说明理由并给出正确的解答。
\item 
在上下两部分气体升高相同温度的过程中,水银柱位置发生变化,最后稳定在新的平衡位置,$ A $、$ B $两部分气体始末状态压强的变化量分别为$ \Delta p_{A} $和$ \Delta p_{B} $,分析并比较二者的大小关系。


\end{enumerate}


\begin{minipage}[h!]{0.7\linewidth}
\vspace{0.3em}
\banswer{
\begin{enumerate}
\renewcommand{\labelenumi}{\arabic{enumi}.}
% A(\Alph) a(\alph) I(\Roman) i(\roman) 1(\arabic)
%设定全局标号series=example	%引用全局变量resume=example
%[topsep=-0.3em,parsep=-0.3em,itemsep=-0.3em,partopsep=-0.3em]
%可使用leftmargin调整列表环境左边的空白长度 [leftmargin=0em]
\item
不正确, 水银柱向上移动 
\item 
$\Delta p _ { A } = \Delta p _ { B }$



\end{enumerate}
}

\vspace{0.3em}
\end{minipage}
\hfill
\begin{minipage}[h!]{0.3\linewidth}
\flushright
\vspace{0.3em}
\includesvg[width=0.2\linewidth]{picture/svg/267}
\vspace{0.3em}
\end{minipage}







\end{enumerate}

