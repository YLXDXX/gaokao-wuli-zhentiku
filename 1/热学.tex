\bta{番外篇$ \quad $热学}

\btb{分子动理论、内能}

\btc{分子动理论}
\begin{enumerate}
\renewcommand{\labelenumi}{\arabic{enumi}.}
% A(\Alph) a(\alph) I(\Roman) i(\roman) 1(\arabic)
%设定全局标号series=example	%引用全局变量resume=example
%[topsep=-0.3em,parsep=-0.3em,itemsep=-0.3em,partopsep=-0.3em]
%可使用leftmargin调整列表环境左边的空白长度 [leftmargin=0em]
\item
物体是由大量分子组成的

①分子直径数量级是$ 10^{-10}\ m $,可以用油膜法测量分子的大小.(质量:$ 10^{-26}\ kg $)

②阿伏加德罗常数即$ 1\ mol $任何物质所含有的粒子数,$ N_{A}=6.02\times10^{23}\ mol^{-1} $

\item 
分子在做永不停息的无规则运动

①扩散现象

\begin{enumerate}
\renewcommand{\labelenumii}{\alph{enumii}.}
% A(\Alph) a(\alph) I(\Roman) i(\roman) 1(\arabic)
%设定全局标号series=example	%引用全局变量resume=example
%[topsep=-0.3em,parsep=-0.3em,itemsep=-0.3em,partopsep=-0.3em]
%可使用leftmargin调整列表环境左边的空白长度 [leftmargin=0em]
\item
定义:不同物质能够彼此进入对方的现象

\item 
实质:不是外界作用引起的,也不是化学反应的结果,而是
由自身物质分子的无规则运动产生的.

\item 
特点$ : $温度越高,扩散越快,气体、固体、液体都可以扩散,
气体和液体扩散较快,固体扩散较慢.	



\end{enumerate}

②布朗运动与分子热运动的比较
\begin{table}[h!]
\centering 
\begin{tabular}{|c|m{0.35\linewidth}|m{0.35\linewidth}|}
\hline 
& \multicolumn{1}{c|}{布朗运动} & \multicolumn{1}{c|}{分子热运动}
\\
\hline
活动主体 & \multicolumn{1}{c|}{固体小颗粒} & \multicolumn{1}{c|}{分子}
\\
\hline
区别 & 悬浮在液体(或气体)中的固体微粒做永不停息的无规则运动,微粒越小运动越明显. & 分子的运动,分子无论大小都做热运动.
\\
\hline
共同点 & \multicolumn{2}{c|}{都是无规则运动,温度越高运动越剧烈} 
\\
\hline
联系 & \multicolumn{2}{c|}{\begin{minipage}[h!]{0.7\linewidth}
\flushleft
\vspace{0.3em}
布朗运动是由于微粒受液体(或气体)分子无规则运动的撞击而产生的,布朗运动的无规则性间接反映了液体(或气体)分子运动的无规则性
\vspace{0.3em}
\end{minipage} } 
\\
\hline
\end{tabular}
\end{table} 

\item 
分子间的相互作用力

①分子间同时存在相互作用的引力和斥力.

②分子力:引力和斥力的合力.

③$ r_{0} $为分子间引力和斥力大小相等时的距
离,其数量级为$ 10^{-10}\ m $.

④如图所示,分子间的引力和斥力都随分子
间距离的增大而减小,随分子间距离的减小
而增大, 但引力没有斥力变化快.
\begin{figure}[h!]
\centering
\includesvg[width=0.23\linewidth]{picture/svg/177} \qquad \qquad 
\includesvg[width=0.23\linewidth]{picture/svg/119}
\end{figure}



\end{enumerate}











\btc{物体的内能}

\begin{enumerate}
\renewcommand{\labelenumii}{(\arabic{enumii})}

\item 
分子平均动能

物体内所有分子动能的平均值叫做分子的平均动能.温度是分
子的平均动能大小的标志,温度越高,分子的平均动能越大.

\item 
分子势能

①概念$ : $由分子间的相对位置和相互作用决定的能量.

②分子势能大小的相关因素

微观上$ : $分子势能的大小与分子间距离有关.

当$ r>r_{0} $时,分子势能随分子间距离的增大而增大;

当$ r<r_{0} $时,分子势能随分子间距离的减小而增大;

当$ r=r_{0} $时,分子势能最小.

宏观上:与物体的体积有关,大多数物体,体积变大时,分子
势能变大,也有少数物体(如$ 0 ^{ \circ } C $的水结成$ 0 ^{ \circ } C $冰、铸造铁
等),体积变大时,分子势能反而变小.


\item 
物体的内能

①定义物体内所有分子势能和分子动能的总和.

②决定内能的因素

微观上:分子动能、分子势能、分子个数.

宏观上$ : $温度、体积、物质的量.



\end{enumerate}


\btb{固体、液体和气体}

\btc{固体}
\begin{table}[h!]
\centering 
\begin{tabular}{|c|c|c|c|}
\hline 
\multirow{2}{*}{\diagbox[width=7.5em]{特征比较}{种类}} & \multicolumn{2}{c|}{晶体} & \multirow{2}{*}{非晶体}
\\
\cline{2-3}
& 单晶体 & 多晶体 & 
\\
\hline
外形 & 规则 & 不规则 & 不规则
\\
\hline
熔点 & 确定 & 确定 & 不确定
\\
\hline
物理性质 & 各向异性 & 各向同性 & 各向同性
\\
\hline
典型物质 & \multicolumn{2}{c|}{石英、云母、食盐、硫酸铜} & 玻璃、蜂蜡、松香
\\
\hline
形成与转化 & \multicolumn{3}{c|}{\tabincell{c}{
有些物质在不同条件下能够形成不同的形态.\\同一种物质可以以晶体和非晶体两种不同的形态出现,\\在一定条件下晶体和非晶体可以相互转化
}} 
\\
\hline
\end{tabular}
\end{table} 




\btc{液体}

\begin{enumerate}
\renewcommand{\labelenumii}{(\arabic{enumii})}

\item 
液体分子间距离比气体分子间距离小得多,液体分子间
的作用力比固体分子间的作用力要小.

\item 
液体的表面张力

液体表面层分子间距离较大,因此分子间的作用力表现为引
力;液体表面存在表面张力使液体表面绷紧,浸润与不浸润
也是分子力的表现.
\item 
液晶

液晶是一种特殊的物质,它既具有液体的流动性,又具有某
些晶体的各向异性,液晶在显示器方面具有广泛的应用.

\end{enumerate}


\btc{气体}
\begin{enumerate}
\renewcommand{\labelenumii}{(\arabic{enumii})}

\item 
气体分子之间的距离大约是分子直径的$ 10 $倍,气体分子
之间的作用力十分微弱,可以忽略不计.

\item 
饱和汽与未饱和汽

①饱和汽$ : $液体处于动态平衡的蒸汽.\\
②未饱和汽$ : $没有达到饱和状态的蒸汽.

\item 
饱和汽压\\
①定义$ : $饱和汽所具有的压强叫做这种液体的饱和汽压.\\
②特点$ : $饱和汽压随温度升高而增大,与液体种类有关,与饱和汽的体积无关.

\item 
湿度\\
①定义$ : $空气的潮湿程度.\\
②绝对湿度$ : $空气中所含水蒸气的压强.\\
③相对湿度$ : $在某一温度下,空气中的水蒸气的实际压强与同一温度下水的饱和汽压之比.\\
$ \text{相对湿度} = \dfrac{\text{水蒸气的实际压强}}{\text{同温下水的饱和汽压}}\times 100 \% $

\end{enumerate}


\btc{理想气体}
\begin{enumerate}
\renewcommand{\labelenumii}{(\arabic{enumii})}

\item 
气体分子运动的特点\\
①分子很小,间距很大,除碰撞外不受力.\\
②向各个方向运动的气体分子数目都相等.\\
③分子做无规则运动,大量分子的速率按“中间多、两头少”的规律分布.\\
④温度一定时,某种气体分子的速率分布是确定的,温度升
高时,速率小的分子数减少,速率大的分子数增多,分子的平
均速率增大,但不是每个分子的速率都增大.

\item 
气体的状态参量$ : $压强、温度、体积\\\
①气体压强的微观意义$ : $大量气体分子无规则运动碰撞器壁,
形成对器壁各处均匀而持续的压力.气体压强在数值上等于
气体分子作用在单位面积上的压力.\\
②气体温度的意义$ : $宏观上温度表示物体的冷热程度,微观
上温度是分子平均动能的标志\\
③气体的体积$ : $气体体积为气体分子所能达到的空间的体
积,即气体充满容器的容积. 国际单位制单位$ : $立方米.符号$ : $
$ m^{3} $.常用单位$ : $升$ (L) $、毫升$ (mL) $.

\item 
理想气体状态方程\\
①理想气体$ : $在任何温度、任何压强下都严格遵从气体实验
定律的气体.
\begin{enumerate}
\renewcommand{\labelenumii}{\alph{enumii}.}
% A(\Alph) a(\alph) I(\Roman) i(\roman) 1(\arabic)
%设定全局标号series=example	%引用全局变量resume=example
%[topsep=-0.3em,parsep=-0.3em,itemsep=-0.3em,partopsep=-0.3em]
%可使用leftmargin调整列表环境左边的空白长度 [leftmargin=0em]
\item
理想气体是一种经科学的抽象而建立的理想化模型,实际
上不存在.
\item 
理想气体不考虑分子间相互作用的分子力,不存在分子势
能, 内能等于分子的总动能,因此理想气体的内能只与气体
的温度和质量有关,与气体的体积无关.
\item 
实际气体特别是那些不易液化的气体在压强不太大,温度
不太低时都可当成理想气体来处理.




\end{enumerate}
②一定质量的理想气体状态方程$: \frac { p _ { 1 } V _ { 1 } } { T _ { 1 } } = \frac { p _ { 2 } V _ { 2 } } { T _ { 2 } }$或$\frac { p V } { T } = C$(常量)


\end{enumerate}

\btb{热力学定律与能量守恒定律}
\btc{热力学第一定律}

\begin{enumerate}
\renewcommand{\labelenumii}{(\arabic{enumii})}

\item 
内容$ : $外界对物体做的功$ W $与物体从外界吸收的热量$ Q $
之和等于物体内能的增量$ \triangle U $.

\item 
表达式$ : \triangle U=W+Q $.

\item 
符号规定
\begin{table}[h!]
\centering 
\begin{tabular}{|c|c|c|}
\hline 
\multirow{2}{*}{做功$ W $} & 外界对物体做功 &$ W>0 $
\\
\cline{2-3}
& 物体对外界做功 &$ W<0 $
\\
\hline
\multirow{2}{*}{吸、放热$ Q $ }& 物体从外界吸收热量 &$ Q>0 $
\\
\cline{2-3}
& 物体向外界放出热量 &$ Q<0 $
\\
\hline
\multirow{2}{*}{内能变化$ \triangle U $} & 物体内能增加 &$ \triangle U>0 $
\\
\cline{2-3}
& 物体内能减小 &$ \triangle U<0 $
\\
\hline
\end{tabular}
\end{table} 


\btc{热力学第二定律的两种表述}

\end{enumerate}

\begin{enumerate}
\renewcommand{\labelenumii}{(\arabic{enumii})}

\item 
热量不能自发地从低温物体传到高温物体.

\item 
不可能从单一热源吸收热量,使之完全变成功,而不产
生其他影响.

\end{enumerate}


\btc{能量守恒定律}

\begin{enumerate}
\renewcommand{\labelenumii}{(\arabic{enumii})}

\item 
内容$ : $能量既不会凭空产生,也不会凭空消失,它只能从
一种形式转化为另一种形式,或者是从一个物体转移到别的
物体,在转化或转移的过程中其总量保持不变.

\item 
条件性$ : $能量守恒定律是自然界的普遍规律,某一种形
式的能是否守恒是有条件的,例如,机械能守恒定律具有适
用条件,而能量守恒定律是无条件的,是一切自然现象都遵
守的基本规律.

\item 
两类永动机

①第一类永动机$ : $不消耗任何能量,却源源不断地对外做功
的机器. 违背能量守恒定律,因此不可能实现.\\
②第二类永动机$ : $从单一热库吸收热量并把它全部用来对外
做功,而不引起其他变化的机器. 违背热力学第二定律,不可
能实现.			

\end{enumerate}

\newpage
\bta{方法解读}

\btb{微观量的求解}
\begin{enumerate}
\renewcommand{\labelenumi}{\arabic{enumi}.}
% A(\Alph) a(\alph) I(\Roman) i(\roman) 1(\arabic)
%设定全局标号series=example	%引用全局变量resume=example
%[topsep=-0.3em,parsep=-0.3em,itemsep=-0.3em,partopsep=-0.3em]
%可使用leftmargin调整列表环境左边的空白长度 [leftmargin=0em]
\item
微观物理量$ : $分子的质量$ m_{0} $,分子体积$ V_{0} $,分子直径$ d $.


\item
宏观物理量$ : $物质的质量$ m $,体积$ V $,密度$ \rho $,摩尔质量$ M $,
摩尔体积$ V_{M} $.


\item
阿伏加德罗常数是联系宏观物理量与微观物理量的桥
梁,根据油膜法测出油酸分子的直径,再根据油酸的摩尔
体积,可算出阿伏加德罗常数;反过来,已知阿伏加德罗
常数,根据摩尔质量(或摩尔体积)就可以算出一个分子
的质量(或一个分子的体积).
\begin{enumerate}
\renewcommand{\labelenumii}{(\arabic{enumii})}

\item 
分子的质量:$m _ { 0 } = \frac { M } { N _ { \mathrm { A } } } = \frac { \rho V _ { M } } { N _ { \mathrm { A } } }$.


\item 
分子的体积:$V _ { 0 } = \frac { V _ { M } } { N _ { A } } = \frac { M } { \rho N _ { A } }$.

对于气体,由于分子间空隙很大,用上式估算出的是一个
分子所占据的空间体积.


\item 
分子的大小$ : $球模型直径$d = \sqrt [ 3 ] { \frac { 6 V _ { 0 } } { \pi } }$,立方体模型棱长
为$d = \sqrt [ 3 ] { V _ { 0 } }$.

\item 
物质所含的分子数:$N = n N _ { \mathrm { A } } = \frac { m } { M } N _ { \mathrm { A } } = \frac { V } { V _ { M } } N _ { \mathrm { A } } , N _ { \mathrm { A } } =$$\frac { V _ { M \rho } } { m _ { 0 } } = \frac { M } { \rho V _ { 0 } }$.


\end{enumerate}


\item
用油膜法估测分子大小

将油酸分子抽象成都是直立在水面上的,当油酸在足够
大水面上尽可能散开后将其看成单分子油膜,单分子油
膜的厚度等于油酸分子的直径.根据配制的油酸酒精溶
液的浓度,算出一滴溶液中纯油酸的体积$ V $,测出这滴溶
液中油酸形成薄膜的面积$ S $,即可算出油酸薄膜的厚度
$ d=\frac{V}{S} $,即油酸分子的直径.

注意:
\begin{enumerate}
\renewcommand{\labelenumii}{(\arabic{enumii})}
% A(\Alph) a(\alph) I(\Roman) i(\roman) 1(\arabic)
%设定全局标号series=example	%引用全局变量resume=example
%[topsep=-0.3em,parsep=-0.3em,itemsep=-0.3em,partopsep=-0.3em]
%可使用leftmargin调整列表环境左边的空白长度 [leftmargin=0em]
\item
油酸酒精溶液配制后长时间放置,溶液的
浓度容易改变,会给实验带来较大误差.


\item 
利用坐标纸小格子数计算轮廓面积时,轮廓的不规则
性容易带来计算误差.为减小误差,不足半个格子的舍
去,多于半个格子的算一个.方格边长越小,计算出的面
积越精确.



\end{enumerate}


\end{enumerate}	

\btb{物体内能的求解}
\begin{enumerate}
\renewcommand{\labelenumi}{\arabic{enumi}.}
% A(\Alph) a(\alph) I(\Roman) i(\roman) 1(\arabic)
%设定全局标号series=example	%引用全局变量resume=example
%[topsep=-0.3em,parsep=-0.3em,itemsep=-0.3em,partopsep=-0.3em]
%可使用leftmargin调整列表环境左边的空白长度 [leftmargin=0em]
\item
判断分子势能变化的两种方法
\begin{enumerate}
\renewcommand{\labelenumii}{(\arabic{enumii})}

\item 
根据分子力做功判断.分子力做正
功,分子势能减小;分子力做负功,分子
势能增加.

\item 
利用分子势能与分子间距的关系
图线判断,如图所示.但要注意此图线
和分子力与分子间距离的关系图线形
状虽然相似,但意义不同,不要混淆.

\end{enumerate}


\item
分子力、分子势能与分子间距离的关系
\begin{table}[h!]
\centering 
 \begin{tabular}{|c|c|m{0.4\linewidth}|m{0.4\linewidth}|}
\hline 
\multicolumn{2}{|c|}{\diagbox{项目}{名称}} & \multicolumn{1}{c|}{分子间的相互作用力$ F $} & \multicolumn{1}{c|}{分子势能$ E_p $}
\\
\hline
\multicolumn{2}{|c|}{\tabincell{c}{
\rule{0em}{3em}与分子间距离的\\关系图象
}} & \begin{minipage}[h!]{1\linewidth}
\centering
\vspace{0.3em}
\includesvg[width=0.44\linewidth]{picture/svg/118}
\vspace{0.3em}
\end{minipage} & \begin{minipage}[h!]{1\linewidth}
\centering
\vspace{0.3em}
\includesvg[width=0.5\linewidth]{picture/svg/119}
\vspace{0.3em}
\end{minipage} 
\\
\hline
\multirow{4}{0.05\linewidth}{随分子间距的变化情况} &$ r<r_{0} $ & $ F_{\text{引}} $和$ F_{\text{斥}} $都随距离的增大而减小,随距离的减小而增大,$ F_{\text{引}} < F_{\text{斥}} $,$ F $表现为斥力 &\tabincell{c}{
$ r $增大,分子力做正功,分子势能减少\\
$ r $减小,分子力做负功,分子势能增加
}
\\
\cline{2-4}
&$ r>r_{0} $ &$ F_{\text{引}} $和$ F_{\text{斥}} $都随距离的增大而减小,随距离的减小而增大,$ F_{\text{引}} > F_{\text{斥}} $,$ F $表现为引力&\tabincell{c}{
$ r $增大,分子力做负功,分子势能增加\\
$ r $减小,分子力做正功,分子势能减少
}
\\
\cline{2-4}
&$ r=r_{0} $ & $ F_{\text{引}} = F_{\text{斥}} $,$ F=0 $ & 分子势能最小,但不一定为零
\\
\cline{2-4}
&$ r>10r_0 $ & $ F_{\text{引}} $和$ F_{\text{斥}} $都已十分微弱,可以认为分子间没有相互作用力 & 分子势能为零
\\
\hline
\end{tabular}
\end{table} 

\item 
分析物体的内能问题时应当明确的几点
\begin{enumerate}
\renewcommand{\labelenumi}{\arabic{enumi}.}
% A(\Alph) a(\alph) I(\Roman) i(\roman) 1(\arabic)
%设定全局标号series=example	%引用全局变量resume=example
%[topsep=-0.3em,parsep=-0.3em,itemsep=-0.3em,partopsep=-0.3em]
%可使用leftmargin调整列表环境左边的空白长度 [leftmargin=0em]
\item
内能是对大量分子而言的,不存在某个分子的内能的
说法.
\item 
宏观上决定内能大小的因素有温度、体积和物质的
量,另外内能大小还与物态有关系.
\item 
通过做功或热传递可以改变物体的内能.
\item 
温度是分子平均动能的标志,温度相同的任何物体,
分子的平均动能相同.




\end{enumerate}


\end{enumerate}		


\btb{气体压强的计算}
\begin{enumerate}
\renewcommand{\labelenumi}{\arabic{enumi}.}
% A(\Alph) a(\alph) I(\Roman) i(\roman) 1(\arabic)
%设定全局标号series=example	%引用全局变量resume=example
%[topsep=-0.3em,parsep=-0.3em,itemsep=-0.3em,partopsep=-0.3em]
%可使用leftmargin调整列表环境左边的空白长度 [leftmargin=0em]
\item
气体压强的决定因素
\begin{enumerate}
\renewcommand{\labelenumii}{(\arabic{enumii})}

\item 
宏观上$ : $取决于气体的温度和体积.

\item 
微观上$ : $取决于分子的平均动能和分子数密度.


\end{enumerate}
\item 
气体压强的计算	
\begin{enumerate}
\renewcommand{\labelenumii}{(\arabic{enumii})}

\item 
在气体流通的区域,各处压强相等,如容器与外界相
通,容器内外压强相等;用细管相连的容器,平衡时两边
气体压强相等.

\item 
液体内深为$ h $处的总压强$ p=p_{0}+\rho gh $,式中的$ p_{0} $为液
面上方的压强.

\item 
连通器内静止的同种液体,在同一水平面上各处压强
相等.

\item 
当封闭气体所在的系统处于力学非平衡状态时,欲求
封闭气体的压强,首先选择恰当的对象(如与气体关联的液
柱、活塞等),并对研究对象进行正确的受力分析(特别注意
内、外气体的压力),然后根据牛顿第二定律列方程求解.


\end{enumerate}

\end{enumerate}



\btb{气变质量问题的求解}

分析气体变质量问题时,可以通过巧妙地选择合适的研究
对象,使变质量问题转化为气体质量一定的问题,然后利用
理想气体状态方程求解.

\begin{enumerate}
\renewcommand{\labelenumi}{\arabic{enumi}.}
% A(\Alph) a(\alph) I(\Roman) i(\roman) 1(\arabic)
%设定全局标号series=example	%引用全局变量resume=example
%[topsep=-0.3em,parsep=-0.3em,itemsep=-0.3em,partopsep=-0.3em]
%可使用leftmargin调整列表环境左边的空白长度 [leftmargin=0em]
\item
充气问题

设想将充进容器内的气体用一个无形的弹性口袋收集起
来,那么,当我们取容器和口袋内的全部气体为研究对象
时,这些气体的状态不管怎样变化,其质量总是不变的.
这样,我们就将变质量的问题转化成质量一定的问题了.


\item
抽气问题

在用抽气筒对容器抽气的过程中,对每一次抽气而言,气
体质量发生变化,解决该类变质量问题的方法与充气问题
类似$ : $假设把每次抽出的气体包含在气体变化的始末状态
中,即用等效法把变质量问题转化为质量一定的问题.


\item
灌气问题

将一个大容器里的气体分装到多个小容器中的问题也是
一种典型的变质量问题,分析这类问题时,可以把大容器
中的气体和多个小容器中的气体作为一个整体来进行研
究,即将变质量问题转化为质量一定的问题.



\item
漏气问题

容器漏气过程中容器内气体的质量不断发生变化,属于变
质量问题,不能直接用理想气体状态方程求解.如果选容
器内原有气体为研究对象,便可使问题变成质量一定的
气体状态变化问题,这时可用理想气体状态方程求解.


\end{enumerate}


\btb{气体实验定律及理想气体状态方程的应用方法}
\noindent
气体实验定律的比较
\begin{table}[h!]
\centering 
\zihao{-5}
\begin{tabular}{|c|c|c|c|}
\hline 
\diagbox{比较项目}{定律} & 玻意耳定律(等温变化) & 查理定律(等容变化) & 盖一吕萨克定律(等压变化)
\\
\hline
数学表达式 &$p _ { 1 } V _ { 1 } = p _ { 2 } V _ { 2 }$或$ pV=C $(常量) & $\frac { p _ { 1 } } { T _ { 1 } } = \frac { p _ { 2 } } { T _ { 2 } }$或$\frac { p } { T } = C$(常量) &$\frac { V _ { 1 } } { T _ { 1 } } = \frac { V _ { 2 } } { T _ { 2 } }$或$\frac { V } { T } = C$(常量)
\\
\hline
同一气体的两条图线 &
\begin{minipage}[h!]{0.26\linewidth}
\centering
\vspace{0.3em}
\includesvg[width=0.9\linewidth]{picture/svg/193}
\vspace{0.3em}
\end{minipage}
& \begin{minipage}[h!]{0.26\linewidth}
\centering
\vspace{0.3em}
\includesvg[width=0.9\linewidth]{picture/svg/194}
\vspace{0.3em}
\end{minipage} &\begin{minipage}[h!]{0.26\linewidth}
\centering
\vspace{0.3em}
\includesvg[width=0.9\linewidth]{picture/svg/195}
\vspace{0.3em}
\end{minipage}
\\
\hline
图线特点 & \begin{minipage}[h!]{0.26\linewidth}
\flushleft
\vspace{0.3em}
①等温变化在$ p-V $图象中是双曲线,由$\frac { p V } { T } =$常量知道,$ T $越大$ ,pV $值就越大故远离原点的等温线对应的温度高,即$ T_1<T_2 $

②等温变化的$p - \frac { 1 } { V }$图象是通过原点的直线,斜率越大则温度越高,所以$ T_2>T_1 $ 
\vspace{0.3em}
\end{minipage}& 
\begin{minipage}[h!]{0.26\linewidth}
\flushleft
\vspace{0.3em}
等容变化的$ p-T $图象是通过原点的直线,由$\frac { p V } { T } =$常量可知,斜率越小则体积越大,所以$ V_1>V_2 $.
\vspace{0.3em}
\end{minipage}
& \begin{minipage}[h!]{0.26\linewidth}
\flushleft
\vspace{0.3em}
等压变化的$ V-T $图象是通过原点的直线,由$\frac { p V } { T } =$常量可知,斜率越小则压强越大,所以$ p_1>P_2 $.
\vspace{0.3em}
\end{minipage} \\
\hline
\end{tabular}
\end{table} 



\btb{热力学定律与气体实验定律的综合应用}		

\begin{enumerate}
\renewcommand{\labelenumi}{\arabic{enumi}.}
% A(\Alph) a(\alph) I(\Roman) i(\roman) 1(\arabic)
%设定全局标号series=example	%引用全局变量resume=example
%[topsep=-0.3em,parsep=-0.3em,itemsep=-0.3em,partopsep=-0.3em]
%可使用leftmargin调整列表环境左边的空白长度 [leftmargin=0em]
\item
判定物体内能变化的方法
\begin{enumerate}
\renewcommand{\labelenumii}{(\arabic{enumii})}

\item 
内能的变化用热力学第一定律进行综合分析.

\item 
做功情况要看气体体积的变化情况$ : $体积增大,气体
对外做功,$ W $为负;体积减小,外界对气体做功,$ W $为正.

\item 
与外界绝热,则不发生热传递,此时$ Q=0 $.

\item 
如果研究对象是一定质量的理想气体,由于理想气体
没有分子势能,所以气体内能的变化体现在分子平均动
能的变化上,从宏观上看气体内能的变化体现在温度的
变化上.

\end{enumerate}


\item
对热力学第二定律的理解
\begin{enumerate}
\renewcommand{\labelenumii}{(\arabic{enumii})}

\item 
在热力学第二定律的表述中,“自发地”“不产生其他
影响”的含义$ : $

①“自发地”指明了热传递等热力学宏观现象的方向性,
不需要借助外界提供能量;

②“不产生其他影响”的含义是指发生的热力学宏观过程
只在本系统内完成,对周围环境不产生热力学方面的影
响,如吸热、放热、做功等.

\item 
热力学第二定律的实质

热力学第二定律的每一种表述,都揭示了一切与热现象
有关的宏观过程都具有方向性,即一切与热现象有关的
宏观自然过程都是不可逆的.

\end{enumerate}




\end{enumerate}










\iffalse
\btb{分子动理论 内能}

\btc{微观量的估算问题}
\begin{enumerate}
\renewcommand{\labelenumi}{\arabic{enumi}.}
% A(\Alph) a(\alph) I(\Roman) i(\roman) 1(\arabic)
%设定全局标号series=example	%引用全局变量resume=example
%[topsep=-0.3em,parsep=-0.3em,itemsep=-0.3em,partopsep=-0.3em]
%可使用leftmargin调整列表环境左边的空白长度 [leftmargin=0em]
\item
物体由大量分子组成$ \left\{
\begin{array}{l}
\text{分子的大小}
\left\{
\begin{array}{l}
\text{直径:$ 10^{-10}\ m $}\\
\text{质量:$ 10^{-26}\ kg $}
\end{array}
\right.
\vspace{1em}
\\
\text{阿伏加德罗常数$N _ { A } = 6.02 \times 10 ^ { 23 } \mathrm { mol } ^ { - 1 }$ }
\end{array}
\right. $
\item 
两种分子模型

物质有固态 、 液态和气态三种情况,不同物态下应将分子看成不同的模型.
\begin{enumerate}
\renewcommand{\labelenumi}{\arabic{enumi}.}
% A(\Alph) a(\alph) I(\Roman) i(\roman) 1(\arabic)
%设定全局标号series=example	%引用全局变量resume=example
%[topsep=-0.3em,parsep=-0.3em,itemsep=-0.3em,partopsep=-0.3em]
%可使用leftmargin调整列表环境左边的空白长度 [leftmargin=0em]
\item
固体、液体分子一个一个紧密排列,可将分子看成球形或立方体形,如图所示,分子间距等于小球的直径或立方体的棱长,所以$d=\sqrt [ 3 ] { \frac { 6 V } { \pi } }$
(球体模型)或$d = \sqrt [ 3 ] { V }$ (立方体模型).
\begin{figure}[h!]
\centering
\includesvg[width=0.23\linewidth]{picture/svg/117}
\end{figure}

\item 
气体分子不是一个一个紧密排列的,它们之间的距离很大,所以气体分子的大小不等于分子所占有的平均空间.如图所示,此时每个分
子占有的空间视为棱长为$ d $的立方体,所以$d = \sqrt [ 3 ] { V }$.



\end{enumerate}
\item 
宏观量与微观量的相互关系
\begin{enumerate}
\renewcommand{\labelenumi}{\arabic{enumi}.}
% A(\Alph) a(\alph) I(\Roman) i(\roman) 1(\arabic)
%设定全局标号series=example	%引用全局变量resume=example
%[topsep=-0.3em,parsep=-0.3em,itemsep=-0.3em,partopsep=-0.3em]
%可使用leftmargin调整列表环境左边的空白长度 [leftmargin=0em]
\item
微观量:分子体积$ V_0 $、分子直径$ d $、分子质量$ m_0 $.
\item 
宏观量:物体的体积$ V $、摩尔体积$ V_m $,物体的质量$ m $、摩尔质量$ M $、物体的密度 $ \rho $.
\item 
相互关系

①一个分子的质量:$m _ { 0 } = \frac { M } { N _ { \mathrm { A } } } = \frac { \rho V _ { \mathrm { m } } } { N _ { \mathrm { A } } }$.\\
②一个分子的体积:$V _ { 0 } = \frac { V _ { \mathrm { m } } } { N _ { \mathrm { A } } } = \frac { M } { \rho N _ { \mathrm { A } } }$.(注:对气体$ V_0 $为分子所占空间体积)\\
③物体所含的分子数:$n = \frac { V } { V _ { \mathrm { m } } } \cdot N _ { \mathrm { A } } = \frac { m } { \rho V _ { \mathrm { m } } } \cdot N _ { \mathrm { A } }$或$n = \frac { m } { M } \cdot N _ { \mathrm { A } } = \frac { \rho V } { M } \cdot N _ { \mathrm { A } }$.\\
④单位质量中所含的分子数:$n ^ { \prime } = \frac { N _ { \mathrm { A } } } { M }$.




\end{enumerate}


\end{enumerate}


\btc{布朗运动与分子热运动}

\begin{enumerate}
\renewcommand{\labelenumi}{\arabic{enumi}.}
% A(\Alph) a(\alph) I(\Roman) i(\roman) 1(\arabic)
%设定全局标号series=example	%引用全局变量resume=example
%[topsep=-0.3em,parsep=-0.3em,itemsep=-0.3em,partopsep=-0.3em]
%可使用leftmargin调整列表环境左边的空白长度 [leftmargin=0em]
\item
$ 
\begin{minipage}[h!]{0.15\linewidth}
\centering
分子永不停息做无规则的运动
\end{minipage}
\left\{
\begin{array}{l}
\text{扩散运动}
\left\{
\begin{array}{l}
\text{定义:不同物质能够彼此进入对方}\\
\text{实质:分子的无规则运动}
\end{array}
\right.
\vspace{1em}
\\
\text{布朗运动}
\left\{
\begin{array}{l}
\text{定义$ : $悬浮在液体中的小颗粒永不停息地做无规则运动}\\
\text{特点$ : $永不停息、无规则}\\
\text{实质$ : $固体小颗粒的运动}
\end{array}
\right.
\end{array}
\right.
$
\item 
布朗运动与分子热运动的比较如下
\begin{table}[h!]
\centering 
\begin{tabular}{|c|c|c|}
\hline 
& 布朗运动 & 分子热运动
\\
\hline
共同点 & \multicolumn{2}{c|}{都是无规则运动,都随温度的升高而变得更加剧烈} 
\\
\hline
不同点 & 小颗粒的运动 & 分子的运动
\\
\hline
& 使用光学显微镜观察 & 使用电子显微镜观察
\\
\hline
联系 & \multicolumn{2}{c|}{布朗运动是由于小颗粒受到周围分子热运动的撞击力而引起的,反映了分子做无规则运动} 
\\
\hline
\end{tabular}
\end{table} 




\end{enumerate}



\btc{分子力、分子势能与分子间距离的关系}
\begin{table}[h!]
\centering 
\zihao{5}
\begin{tabular}{|c|c|m{0.4\linewidth}|m{0.4\linewidth}|}
\hline 
\multicolumn{2}{|c|}{} & \multicolumn{1}{c|}{分子间的相互作用力$ F $} & \multicolumn{1}{c|}{分子势能$ E_p $}
\\
\hline
\multicolumn{2}{|c|}{\tabincell{c}{
\rule{0em}{3em}与分子间\\距离的\\关系图象
}} & \begin{minipage}[h!]{1\linewidth}
\centering
\vspace{0.3em}
\includesvg[width=0.44\linewidth]{picture/svg/118}
\vspace{0.3em}
\end{minipage} & \begin{minipage}[h!]{1\linewidth}
\centering
\vspace{0.3em}
\includesvg[width=0.5\linewidth]{picture/svg/119}
\vspace{0.3em}
\end{minipage} 
\\
\hline
\multirow{4}{0.05\linewidth}{随分子间距的变化情况} &$ r<r_{0} $ & $ F_{\text{引}} $和$ F_{\text{斥}} $都随距离的增大而减小,随距离的减小而增大,$ F_{\text{引}} < F_{\text{斥}} $,$ F $表现为斥力 &\tabincell{c}{
$ r $增大,分子力做正功,分子势能减少\\
$ r $减小,分子力做负功,分子势能增加
}
\\
\cline{2-4}
&$ r>r_{0} $ &$ F_{\text{引}} $和$ F_{\text{斥}} $都随距离的增大而减小,随距离的减小而增大,$ F_{\text{引}} > F_{\text{斥}} $,$ F $表现为引力&\tabincell{c}{
$ r $增大,分子力做负功,分子势能增加\\
$ r $减小,分子力做正功,分子势能减少
}
\\
\cline{2-4}
&$ r=r_{0} $ & $ F_{\text{引}} = F_{\text{斥}} $,$ F=0 $ & 分子势能最小,但不一定为零
\\
\cline{2-4}
&$ r>10r_0 $ & $ F_{\text{引}} $和$ F_{\text{斥}} $都已十分微弱,可以认为分子间没有相互作用力 & 分子势能为零
\\
\hline
\end{tabular}
\end{table} 


\btc{物体的内能}





\begin{enumerate}
\renewcommand{\labelenumi}{\arabic{enumi}.}
% A(\Alph) a(\alph) I(\Roman) i(\roman) 1(\arabic)
%设定全局标号series=example	%引用全局变量resume=example
%[topsep=-0.3em,parsep=-0.3em,itemsep=-0.3em,partopsep=-0.3em]
%可使用leftmargin调整列表环境左边的空白长度 [leftmargin=0em]
\item

$ \text{内能}
\left\{
\begin{array}{l}
\text{分子动能} \rightarrow \text{温度}
\left\{
\begin{array}{l}
\text{两种温标:$ T=t+273.15K $}\\
\text{温度是分子热运动的平均动能的标志}
\end{array}
\right.
\vspace{1em}
\\
\text{分子势能}\rightarrow\text{决定因素}
\left\{
\begin{array}{l}
\text{微观上$ : $分了间距离和分子排列情况}\\
\text{宏观上$ : $体积和状态}
\end{array}
\right.
\vspace{1em}\\
\text{内能}
\left\{
\begin{array}{l}
\text{大小$\rightarrow$所有分子热运动的动能与分子势能的总和}\\
\text{决定因素$ \rightarrow $温度和体积}
\end{array}
\right.
\vspace{1em}
\end{array}
\right.
$

\item 
内能和机械能的区别与联系
\begin{table}[h!]
\centering 
\zihao{5}
\begin{tabular}{|m{0.06\linewidth}|m{0.2\linewidth}|m{0.24\linewidth}|m{0.17\linewidth}|m{0.16\linewidth}|m{0.06\linewidth}|}
\hline 
能量 & \multicolumn{1}{c|}{定义} & \multicolumn{1}{c|}{决定} & \multicolumn{1}{c|}{量值} & \multicolumn{1}{c|}{测量} & 转化
\\
\hline
内能 & 物体内所有分子的动能和势能的总和 & 分子微观运由物体内部动状态决定,与物体整体运动情况无关 & 任何物体都具有内能,恒不为零 & 无法测量.其变化量可由做功和热传递来量度 & \multirow{2}{1\linewidth}{在一定条件下可相互转化
} \\
\cline{1-5}
机械能 & 物体的动能重力势能和弹性势能的总和 & 与物体宏观运动状态、参考系和零势能面选取有关,和物体内部分子运动情况无关 & 可以为零 & 可以测量 & 
\\
\hline
\end{tabular}
\end{table} 




\end{enumerate}

\newpage

\btb{固体、液体和气体}

\btc{固体、液体的性质}


$ 
\left\{
\begin{array}{l}
\text{固体}
\left\{
\begin{array}{l}
\text{晶体}
\left\{
\begin{array}{l}
\text{单晶体外形规则,各向异性,有确定熔点}\\
\text{多晶体$ : $外形不规则,各向同性,有确定熔点}
\end{array}
\right.
\vspace{1em}
\\
\text{非晶体:外形不规则、各向同性,无确定熔点}
\end{array}
\right.
\vspace{1em}\\
\text{液晶:各向异性,具有流动性}
\vspace{1em}
\\
\text{液体}
\left\{
\begin{array}{l}
\text{表面张力}
\left\{
\begin{array}{l}
\text{作用$ : $使液体的表面具有收缩到最小的趋势}\\
\text{方向$ : $与液面相切,与分界线垂直}
\end{array}
\right.
\vspace{1em}
\\
\text{饱和汽}
\left\{
\begin{array}{l}
\text{饱和汽压随温度而变,温度高,饱和汽压大}\\
\text{相对湿度}=\frac{\text{水蒸气的实际压强}}{\text{同温度水的饱和汽压}}
\end{array}
\right.
\end{array}
\right.
\end{array}
\right.
$


\btc{气体压强的产生与计算}
\begin{enumerate}
\renewcommand{\labelenumi}{\arabic{enumi}.}
% A(\Alph) a(\alph) I(\Roman) i(\roman) 1(\arabic)
%设定全局标号series=example	%引用全局变量resume=example
%[topsep=-0.3em,parsep=-0.3em,itemsep=-0.3em,partopsep=-0.3em]
%可使用leftmargin调整列表环境左边的空白长度 [leftmargin=0em]
\item
产生的原因

由于大量分子无规则运动而碰撞器壁,形成对器壁各处均匀、持续的压力 ,作用在器壁单位面积上的压力叫做气体的压强.
\item 
决定因素
\begin{enumerate}
\renewcommand{\labelenumi}{\arabic{enumi}.}
% A(\Alph) a(\alph) I(\Roman) i(\roman) 1(\arabic)
%设定全局标号series=example	%引用全局变量resume=example
%[topsep=-0.3em,parsep=-0.3em,itemsep=-0.3em,partopsep=-0.3em]
%可使用leftmargin调整列表环境左边的空白长度 [leftmargin=0em]
\item
宏观上:决定于气体的温度和体积.
\item 
微观上:决定于分子的平均动能和分子的密集程度.




\end{enumerate}

\item 
平衡状态下气体压强的求法
\begin{enumerate}
\renewcommand{\labelenumi}{\arabic{enumi}.}
% A(\Alph) a(\alph) I(\Roman) i(\roman) 1(\arabic)
%设定全局标号series=example	%引用全局变量resume=example
%[topsep=-0.3em,parsep=-0.3em,itemsep=-0.3em,partopsep=-0.3em]
%可使用leftmargin调整列表环境左边的空白长度 [leftmargin=0em]
\item
液片法:选取假想的液体薄片(自身重力不计)为研究对象,分析液片两侧受力情况,建立平衡方程,消去面积,得到液片两侧压强相等方程,求得气体的压强.
\item 
力平衡法:选取与气体接触的液柱(或活塞)为研究对象进行受力分析,得到液柱(或活塞)的受力平衡方程,求得气体的压强.
\item 
等压面法:在连通器中,同一种液体(中间不间断)同一深度处压强相等.液体内深$ h $处的总压强$P = P _ { 0 } + \rho g h$ ,$ P_{0} $ 为液面上方的压强.




\end{enumerate}
\item 
加速运动系统中封闭气体压强的求法

选取与气体接触的液柱(或活塞)为研究对象,进行受力分析,利用牛顿第二定律列方程求解.



\end{enumerate}


\btc{理想气体实验定律及应用}

\begin{enumerate}
\renewcommand{\labelenumi}{\arabic{enumi}.}
% A(\Alph) a(\alph) I(\Roman) i(\roman) 1(\arabic)
%设定全局标号series=example	%引用全局变量resume=example
%[topsep=-0.3em,parsep=-0.3em,itemsep=-0.3em,partopsep=-0.3em]
%可使用leftmargin调整列表环境左边的空白长度 [leftmargin=0em]
\item
气体实验定律
\begin{table}[h!]
\centering 
\begin{tabular}{|c|c|c|c|}
\hline 
& 玻意耳定律 & 查理定律 & 盖—吕萨克定律
\\
\hline
内容 & \begin{minipage}[h!]{0.2\linewidth}
\centering
\vspace{0.3em}
一定质量的某种气体,在温度不变的情况下,压强与体积成反比
\vspace{0.3em}
\end{minipage} & \begin{minipage}[h!]{0.2\linewidth}
\centering
\vspace{0.3em}
一定质量的某种气体,在体积不变的情况下,压强与热力学温度成正比
\vspace{0.3em}
\end{minipage} & 
\begin{minipage}[h!]{0.22\linewidth}
\centering
\vspace{0.3em}
一定质量的某种气体,在压强不变的情况下,其体积与热力学温度成正比
\vspace{0.3em}
\end{minipage} \\
\hline
表达式 & $p _ { 1 } V _ { 1 } = p _ { 2 } V _ { 2 }$ & $\frac { p _ { 1 } } { T _ { 1 } } = \frac { p _ { 2 } } { T _ { 2 } }$或 $\frac { p _ { 1 } } { p _ { 2 } } = \frac { T _ { 1 } } { T _ { 2 } }$& $\frac { V _ { 1 } } { T _ { 1 } } = \frac { V _ { 2 } } { T _ { 2 } }$或
$\frac { V _ { 1 } } { V _ { 2 } } = \frac { T _ { 1 } } { T _ { 2 } }$ \\
\hline
图象 &\begin{minipage}[h!]{0.2\linewidth}
\centering
\vspace{0.3em}
\includesvg[width=0.7\linewidth]{picture/svg/120}
\vspace{0.3em}
\end{minipage} &\begin{minipage}[h!]{0.2\linewidth}
\centering
\vspace{0.3em}
\includesvg[width=0.7\linewidth]{picture/svg/121}
\vspace{0.3em}
\end{minipage} & \begin{minipage}[h!]{0.2\linewidth}
\centering
\vspace{0.3em}
\includesvg[width=0.7\linewidth]{picture/svg/122}
\vspace{0.3em}
\end{minipage}
\\
\hline
\end{tabular}
\end{table} 

\item 
理想气体的状态方程
\begin{enumerate}
\renewcommand{\labelenumi}{\arabic{enumi}.}
% A(\Alph) a(\alph) I(\Roman) i(\roman) 1(\arabic)
%设定全局标号series=example	%引用全局变量resume=example
%[topsep=-0.3em,parsep=-0.3em,itemsep=-0.3em,partopsep=-0.3em]
%可使用leftmargin调整列表环境左边的空白长度 [leftmargin=0em]
\item
理想气体
①宏观上讲,理想气体是指在任何条件下始终遵守气体实验定律的气体,实际气体在压强不太大、温度不太低的条件下,可视为理想气体.

②微观上讲,理想气体的分子间除碰撞外无其他作用力,即分子间无分子势能.
\item 
理想气体的状态方程

一定质量的理想气体状态方程:$\frac { p _ { 1 } V _ { 1 } } { T _ { 1 } } = \frac { p _ { 2 } V _ { 2 } } { T _ { 2 } }$或$\frac { p V } { T } = C$.




\end{enumerate}	

气体实验定律可看做一定质量理想气体状态方程的特例.	
\end{enumerate}


\btc{气体状态变化中的图象问题}
\btd{一定质量的气体不同图象的比较}
\begin{table}[h!]
\centering 
\begin{tabular}{|c|c|m{0.43\linewidth}|c|}
\hline 
过程 & 图线类别 & 图象特点 & 图象示例
\\
\hline
\multirow{2}{*}{等温过程} &$ p-V $ &$ pV=CT $(其中$ C $为恒量),即$ pV $之积越大的等温线温度越高,线离原点越远 & \begin{minipage}[h!]{0.2\linewidth}
\centering
\vspace{0.3em}
\includesvg[width=0.7\linewidth]{picture/svg/123}
\vspace{0.3em}
\end{minipage} 
\\
\cline{2-4}
& &$ p=CT\frac{1}{V} $,斜率$ k=CT $,即斜率越大,温度越高 & \begin{minipage}[h!]{0.2\linewidth}
\centering
\vspace{0.3em}
\includesvg[width=0.7\linewidth]{picture/svg/124}
\vspace{0.3em}
\end{minipage}
\\
\hline
等容过程 &$ p-V $ & $p = \frac { C } { V } T$,斜率$ k=\frac{C}{V} $,即斜率越大,体积越小 & \begin{minipage}[h!]{0.2\linewidth}
\centering
\vspace{0.3em}
\includesvg[width=0.7\linewidth]{picture/svg/125}
\vspace{0.3em}
\end{minipage}
\\
\hline
等压过程 &$ V-T $ & $V = \frac { C } { p } T$,斜率$k = \frac { C } { p }$,即斜率越大,压强越小 & \begin{minipage}[h!]{0.2\linewidth}
\centering
\vspace{0.3em}
\includesvg[width=0.7\linewidth]{picture/svg/126}
\vspace{0.3em}
\end{minipage}
\\
\hline
\end{tabular}
\end{table} 

\btd{注:}
利用垂直于坐标轴的线作辅助线去分析同质量、不同温度的两条等温线,不同体积的两条等容线,不同压强的两条等压线的关系.
\begin{figure}[h!]
\centering
\includesvg[width=0.43\linewidth]{picture/svg/127}
\end{figure}

例如:在图甲中$ ,V_1 $对应虚线为等容线$ ,A $、$ B $分别是虚线与$ T_2 $、$ T_1 $ 两线的交点,可以认为从$ B $状态通过等容升压到$ A $状态,温度必然升
高,所以$ T_2>T_1 $.

又如图乙所示$ ,A $、$ B $两点的温度相等,从$ B $状态到$ A $状态压强增大,体积一定减小,所以$ V_2<V_1 $.


\btb{热力学定律与能量守恒定律}
\btc{热力学第一定律与能量守恒定律}
\begin{enumerate}
\renewcommand{\labelenumi}{\arabic{enumi}.}
% A(\Alph) a(\alph) I(\Roman) i(\roman) 1(\arabic)
%设定全局标号series=example	%引用全局变量resume=example
%[topsep=-0.3em,parsep=-0.3em,itemsep=-0.3em,partopsep=-0.3em]
%可使用leftmargin调整列表环境左边的空白长度 [leftmargin=0em]
\item
热力学第一定律
\begin{enumerate}
\renewcommand{\labelenumi}{\arabic{enumi}.}
% A(\Alph) a(\alph) I(\Roman) i(\roman) 1(\arabic)
%设定全局标号series=example	%引用全局变量resume=example
%[topsep=-0.3em,parsep=-0.3em,itemsep=-0.3em,partopsep=-0.3em]
%可使用leftmargin调整列表环境左边的空白长度 [leftmargin=0em]
\item
内容:一个热力学系统的内能增量等于外界向它传递的热量与外界对它所做的功的和.
\item 
表达式:$\Delta U = Q + W$.
\item 
$\Delta U = Q + W$中正、负号法则.
\begin{table}[h!]
\centering 
\begin{tabular}{|c|c|c|c|}
\hline 
物理量意义符号 & W & Q & $ \triangle U $
\\
\hline
$ + $& 外界对物体做功 & 物体吸收热量 & 内能增加
\\
\hline
$ - $& 物体对外界做功 & 物体放出热量 & 内能减少
\\
\hline
\end{tabular}
\end{table} 




\end{enumerate}

\item 
能量守恒定律

\begin{enumerate}
\renewcommand{\labelenumi}{\arabic{enumi}.}
% A(\Alph) a(\alph) I(\Roman) i(\roman) 1(\arabic)
%设定全局标号series=example	%引用全局变量resume=example
%[topsep=-0.3em,parsep=-0.3em,itemsep=-0.3em,partopsep=-0.3em]
%可使用leftmargin调整列表环境左边的空白长度 [leftmargin=0em]
\item
内容:能量既不会凭空产生,也不会凭空消失,它只能从一种形式转化为另一种形式,或者从一个物体转移到别的物体,在转化或转移的过程中,能量的总量保持不变.

\item 
能量守恒定律是一切自然现象都遵守的基本规律.



\end{enumerate}

\end{enumerate}

\btd{注:}
$\Delta U = Q + W$的三种特殊情况.
\begin{table}[h!]
\centering 
\begin{tabular}{|c|c|c|c|}
\hline 
过程名称 & 公式 & 内能变化 & 物理意义
\\
\hline
绝热 &$ Q=0 $ &$ \triangle U=W $ & 外界对物体做的功等于物体内能的增加
\\
\hline
等容 &$ W=0 $ &$ Q= \triangle U $ & 物体吸收的热量等于物体内能的增加
\\
\hline
等温 & $ \triangle U=0 $&$ W=-Q $& 外界对物体做的功等于物体放出的热量
\\
\hline
\end{tabular}
\end{table} 

\btc{热力学第二定律}
\begin{enumerate}
\renewcommand{\labelenumi}{\arabic{enumi}.}
% A(\Alph) a(\alph) I(\Roman) i(\roman) 1(\arabic)
%设定全局标号series=example	%引用全局变量resume=example
%[topsep=-0.3em,parsep=-0.3em,itemsep=-0.3em,partopsep=-0.3em]
%可使用leftmargin调整列表环境左边的空白长度 [leftmargin=0em]
\item
热力学第二定律的三种表述
\begin{enumerate}
\renewcommand{\labelenumi}{\arabic{enumi}.}
% A(\Alph) a(\alph) I(\Roman) i(\roman) 1(\arabic)
%设定全局标号series=example	%引用全局变量resume=example
%[topsep=-0.3em,parsep=-0.3em,itemsep=-0.3em,partopsep=-0.3em]
%可使用leftmargin调整列表环境左边的空白长度 [leftmargin=0em]
\item
克劳修斯表述:热量不能自发地从低温物体传到高温物体.
\item 
开尔文表述:不可能从单一热库吸收热量,使之完全变成功,而不产生其他影响.或表述为“第二类永动机不可能制成”.
\item 
用熵的概念进行表述:在任何自然过程中,一个孤立系统的总熵不会减小(热力学第二定律又叫做熵增加原理).




\end{enumerate}

\item 
热力学第二定律的微观意义

一切自发过程总是沿着分子热运动的无序性增大的方向进行.

\item 
热力学第二定律的实质

热力学第二定律的每一种表述,都揭示了大量分子参与宏观过程的方向性,进而使人们认识到自然界中进行的涉及热现象的宏观过程都具有方向性.

\item 
两类永动机的比较
\begin{table}[h!]
\centering 
\begin{tabular}{|c|c|}
\hline 
第一类永动机 & 第二类永动机
\\
\hline
\tabincell{c}{
不需要任何动力或燃料,\\却能不断地对外做功的机器
} & \tabincell{c}{
从单一热源吸收热量,\\使之完全变成功,而不产生其他影响的机器
}
\\
\hline
违背能量守恒定律,不可能制成 & 不违背能量守恒定律,但违背热力学第二定律,不可能制成
\\
\hline
\end{tabular}
\end{table} 




\end{enumerate}


\btd{注:}
\begin{enumerate}
\renewcommand{\labelenumi}{\arabic{enumi}.}
% A(\Alph) a(\alph) I(\Roman) i(\roman) 1(\arabic)
%设定全局标号series=example	%引用全局变量resume=example
%[topsep=-0.3em,parsep=-0.3em,itemsep=-0.3em,partopsep=-0.3em]
%可使用leftmargin调整列表环境左边的空白长度 [leftmargin=0em]
\item
“自发地”指明了热传递等热力学宏观现象的方向性,不需要借助外界提供能量的帮助.在借助外界提供能量的帮助下,热量可以由低
温物体传到高温物体,如冰箱、空调.
\item 
“不产生其他影响”的含义是发生的热力学宏观过程只在本系统内完成,对周围环境不产生热力学方面的影响.如吸热、放热、做功等.在
引起其他变化的条件下内能可以全部转化为机械能,如气体的等温膨胀过程.




\end{enumerate}







\iffalse

\btd{例题}
\begin{enumerate}
\renewcommand{\labelenumi}{\arabic{enumi}.}
% A(\Alph) a(\alph) I(\Roman) i(\roman) 1(\arabic)
%设定全局标号series=example	%引用全局变量resume=example
%[topsep=-0.3em,parsep=-0.3em,itemsep=-0.3em,partopsep=-0.3em]
%可使用leftmargin调整列表环境左边的空白长度 [leftmargin=0em]
\item
\exwhere{$ 2017 $年江苏卷}
科学家可以运用无规则运动的规律来研究生物蛋白分子.资料显示,某种蛋白的摩尔质量为$ 66\ kg/mol $,其分子
可视为半径为$ 3 \times 10^{-9}\ m $ 的球,已知阿伏加德罗常数为$ 6.0 \times 10^{23}\ mol^{-2} $.请估算该蛋白的密度.(计算结果保留一位有效数字)
\vspace{1cm}

\banswer{
$\rho \approx 1 \times 10 ^ { 3 } \mathrm { kg } / \mathrm { m } ^ { 3 }$\quad ($\rho = 5 \times 10 ^ { 2 } \mathrm { kg } / \mathrm { m } ^ { 3 } \cdot 5 \times 10 ^ { 2 } \sim 1 \times 10 ^ { 3 } \mathrm { kg } / \mathrm { m } ^ { 3 }$)
}
\item 
\exwhere{$ 2016 $上海卷}
某气体的摩尔质量为$ M $,分子质量为$ m $.若 1 摩尔该气体的体积为$ V_m $,密度为 $ \rho $,则该气体单位体积分子数为(阿伏伽
德罗常数为$ N_A $) \xzanswer{ABC} 


\fourchoices
{$ \frac { N _ { \mathrm { A } } } { V _ { \mathrm { m } } } $}
{$\frac { M } { m V _ { \mathrm { m } } }$}
{$ \frac { \rho N _ { \mathrm { A } } } { M } $}
{$\frac { \rho N _ { \mathrm { A } } } { m }$}




\end{enumerate}

\fi
\fi

