\bta{人造地球卫星}

\begin{enumerate}[leftmargin=0em]
\renewcommand{\labelenumi}{\arabic{enumi}.}
% A(\Alph) a(\alph) I(\Roman) i(\roman) 1(\arabic)
%设定全局标号series=example	%引用全局变量resume=example
%[topsep=-0.3em,parsep=-0.3em,itemsep=-0.3em,partopsep=-0.3em]
%可使用leftmargin调整列表环境左边的空白长度 [leftmargin=0em]
\item
\exwhere{$ 2019 $年$ 4 $月浙江物理选考}
某颗北斗导航卫星属于地球静止轨道卫星(即卫星相对于地面静止)。则此卫星的 \xzanswer{C} 

\fourchoices
{线速度大于第一宇宙速度}
{周期小于同步卫星的周期}
{角速度大于月球绕地球运行的角速度}
{向心加速度大于地面的重力加速度}




\item 
\exwhere{$ 2019 $年物理北京卷}
$ 2019 $年$ 5 $月$ 17 $日,我国成功发射第$ 45 $颗北斗导航卫星,该卫星属于地球静止轨道卫星(同步卫星)。该卫星 \xzanswer{D} 

\fourchoices
{入轨后可以位于北京正上方}
{入轨后的速度大于第一宇宙速度}
{发射速度大于第二宇宙速度}
{若发射到近地圆轨道所需能量较少}




\item 
\exwhere{$ 2019 $年物理全国\lmd{3}卷}
金星、地球和火星绕太阳的公转均可视为匀速圆周运动,它们的向心加速度大小分别为$ a_{ \text{金} } $、$ a_{ \text{地} } $、$ a_{ \text{火} } $,它们沿轨道运行的速率分别为$ v_{ \text{金} } $、$ v_{ \text{地} } $、$ v_{ \text{火} } $。已知它们的轨道半径$ R _{ \text{金} } <R _{ \text{地} } <R_{ \text{火} }$,由此可以判定 \xzanswer{A} 

\fourchoices
{$ a _{ \text{金} } >a _{ \text{地} } >a_{ \text{火} } $}
{$ a_{ \text{火} } >a _{ \text{地} } >a_{ \text{金} } $}
{$ v_{ \text{地} } >v_{ \text{火} } >v_{ \text{金} } $}
{$ v _{ \text{火} } >v _{ \text{地} } >v_{ \text{金} } $}




\item 
\exwhere{$ 2019 $年物理江苏卷}
$ 1970 $年成功发射的“东方红一号”是我国第一颗人造地球卫星,该卫星至今仍沿椭圆轨道绕地球运动.如图所示,设卫星在近地点、远地点的速度分别为$ v_{1} $、$ v_{2} $,近地点到地心的距离为$ r $,地球质量为$ M $,引力常量为$ G $.则 \xzanswer{B} 
\begin{figure}[h!]
\centering
\includesvg[width=0.3\linewidth]{picture/svg/592}
\end{figure}
\fourchoices
{$v _ { 1 } > v _ { 2 } \qquad v _ { 1 } = \sqrt { \frac { G M } { r } }$}
{$v _ { 1 } > v _ { 2 } \qquad v _ { 1 } > \sqrt { \frac { G M } { r } }$}
{$v _ { 1 } < v _ { 2 } \qquad v _ { 1 } = \sqrt { \frac { G M } { r } }$}
{$v _ { 1 } < v _ { 2 } \qquad v _ { 1 } > \sqrt { \frac { G M } { r } }$}




\item 
\exwhere{$ 2012 $年物理海南卷}
地球同步卫星到地心的距离$ r $ 可用质量$ M $、地球自转周期$ T $与 引力常量$ G $表示为$ r= $\tk{$\sqrt { \frac { G M T ^ { 2 } } { 4 \pi ^ { 2 } } }$}.


\item 
\exwhere{$ 2013 $年上海卷}
若两颗人造地球卫星的周期之比为$ T_{1} : T_{2} $=$ 2 : 1 $,则它们的轨道半径之比$ R_{1} : R_{2} $= \tk{$\sqrt [ 3 ] { 4 }: 1$} ,向心加速度之比$ a_{1} : a_{2} $= \tk{$1: 2 \sqrt [ 3 ] { 2 }$} 。

\item 
\exwhere{$ 2012 $年物理上海卷}
人造地球卫星做半径为$ r $,线速度大小为$ v $的匀速圆周运动。当其角速度变为原来的$ \frac{\sqrt{2}}{4} $倍后,运动半径为\tk{2r},线速度大小为\tk{$\frac { \sqrt { 2 } } { 2 } v$}。


\item 
\exwhere{$ 2011 $年上海卷}
人造地球卫星在运行过程中由于受到微小的阻力,轨道半径将缓慢减小。在此运动过程中,卫星所受万有引力大小将 \tk{增大} (填“减小”或“增大”);其动能将 \tk{增大} (填“减小”或“增大”)。

\item 
\exwhere{$ 2013 $年天津卷}
“嫦娥一号”和“嫦娥二号”卫星相继完成了对月球的环月飞行,标志着我国探月工程的第一阶段己经完成。设“嫦娥二号”卫星环绕月球的运动为匀速圆周运动,它距月球表面的高度为$ h $,己知月球的质量为$ M $、半径为$ R $,引力常量为$ G $,则卫星绕月球运动的向心加速度$ a $=\tk{$\frac { G M } { ( R + h ) ^ { 2 } }$}, 线速度 $ v= $\tk{$ \sqrt { \frac { G M } { R + h } }$}。 


\item 
\exwhere{$ 2018 $年全国\lmd{3}卷}
为了探测引力波,“天琴计划”预计发射地球卫星$ P $,其轨道半径约为地球半径的$ 16 $倍;另一地球卫星$ Q $的轨道半径约为地球半径的$ 4 $倍。$ P $与$ Q $的周期之比约为 \xzanswer{C} 
\fourchoices
{$ 2: 1 $}
{$ 4: 1 $}
{$ 8: 1 $}
{$ 16: 1 $}

\item 
\exwhere{$ 2011 $年海南卷}
$ 2011 $年$ 4 $月$ 10 $日,我国成功发射第$ 8 $颗北斗导航卫星,建成以后北斗导航卫星系统将包含多颗地球同步卫星,这有助于减少我国对$ GPS $导航系统的依赖,$ GPS $由运行周期为$ 12 $小时的卫星群组成,设北斗导航系统的同步卫星和$ GPS $导航卫星的轨道半径分别为$ R_{1} $和$ R_{2} $,向心加速度分别为$ a_{1} $和$ a_{2} $,则$ R_{1}:R_{2}= $\tk{$\sqrt[3]{4}$}。$ a_{1}:a_{2}= $ \tk{$\frac { \sqrt [ 3 ] { 2 } } { 4 }$} 。(可用根式表示)


\item 
\exwhere{$ 2016 $年上海卷}
两颗卫星绕地球运行的周期之比为$ 27:1 $,则它们的角速度之比为\tk{$ 27:1 $},轨道半径之比为\tk{$ 9:1 $}。



\item 
\exwhere{$ 2012 $年理综安徽卷}
我国发射的“天宫一号”和“神州八号”在对接前,“天宫一号”的运行轨道高度为$ 350 \ km $ ,“神州八号”的运行轨道高度为$ 343 \ km $。它们的运行轨道均视为圆周,则 \xzanswer{B} 

\fourchoices
{“天宫一号”比“神州八号”速度大}
{“天宫一号”比“神州八号”周期长}
{“天宫一号”比“神州八号”角速度大}
{“天宫一号”比“神州八号”加速度大}



\item 
\exwhere{$ 2014 $年理综福建卷}
若有一颗“宜居”行星,其质量为地球的$ p $倍,半径为地球的$ q $倍,则该行星卫星的环绕速度是地球卫星环绕速度的 \xzanswer{C} 

\fourchoices
{$\sqrt{pq}$倍}
{$\sqrt{\frac{q}{p}}$倍}
{$\sqrt{\frac{p}{q}}$倍}
{$\sqrt{pq^{3}}$倍}


\item 
\exwhere{$ 2012 $年理综四川卷}
今年$ 4 $月$ 30 $日,西昌卫星发射中心发射的中圆轨道卫星,其轨道半径为$ 2.8 \times l0^7\ m $。它与另一颗同质量的同步轨道卫星(轨道半径为$ 4.2 \times l0^7 \ m $)相比 \xzanswer{B} 

\fourchoices
{向心力较小 }
{动能较大}
{发射速度都是第一宇宙速度}
{角速度较小}



\item 
\exwhere{$ 2012 $年理综北京卷}
关于环绕地球运动的卫星,下列说法正确的是 \xzanswer{B} 
\fourchoices
{分别沿圆轨道和椭圆轨道运行的两颗卫星,不可能具有相同的周期}
{沿椭圆轨道运行的一颗卫星,在轨道不同位置可能具有相同的速率}
{在赤道上空运行的两颗地球同步卫星.它们的轨道半径有可能不同}
{沿不同轨道经过北京上空的两颗卫星,它们的轨道平面一定会重合}



\item 
\exwhere{$ 2018 $年江苏卷}
我国高分系列卫星的高分辨对地观察能力不断提高.今年$ 5 $月$ 9 $日发射的“高分五号”轨道高度约为$ 705 $ $ km $,之前已运行的“高分四号”轨道高度约为$ 36 $ $ 000 $ $ km $,它们都绕地球做圆周运动.与“高分四号”相比,下列物理量中“高分五号”较小的是 \xzanswer{A} 

\fourchoices
{周期 }
{角速度}
{线速度}
{向心加速度}


\item 
\exwhere{$ 2018 $年天津卷}
$ 2018 $年$ 2 $月$ 2 $日,我国成功将电磁监测试验卫星“张衡一号”发射升空,标志我国成为世界上少数拥有在轨运行高精度地球物理场探测卫星的国家之一。通过观测可以得到卫星绕地球运动的周期,并已知地球的半径和地球表面的重力加速度。若将卫星绕地球的运动看作是匀速圆周运动,且不考虑地球自转的影响,根据以上数据可以计算出卫星的 \xzanswer{CD} 

\fourchoices
{密度}
{向心力的大小}
{离地高度}
{线速度的大小}



\item 
\exwhere{$ 2012 $年理综天津卷}
一人造地球卫星绕地球做匀速圆周运动,假如该卫星变轨后做匀速圆周运动,动能减小为原来的$ 1/4 $,不考虑卫星质量的变化,则变轨前后卫星的 \xzanswer{C} 

\fourchoices
{向心加速度大小之比为$ 4:1 $ }
{角速度大小之比为$ 2:1 $}
{周期之比为$ 1:8 $ }
{轨道半径之比为$ 1:2 $}


\item 
\exwhere{$ 2011 $年理综北京卷}
由于通讯和广播等方面的需要,许多国家发射了地球同步轨道卫星,这些卫星的 \xzanswer{A} 

\fourchoices
{质量可以不同 }
{轨道半径可以不同}
{轨道平面可以不同 }
{速率可以不同}



\item 
\exwhere{$ 2011 $年理综全国卷}
我国“嫦娥一号”探月卫星发射后,先在“$ 24 $小时轨道”上绕地球运行(即绕地球一圈需要$ 24 $小时);然后,经过两次变轨依次到达“$ 48 $小时轨道”和“$ 72 $小时轨道”;最后奔向月球。如果按圆形轨道计算,并忽略卫星质量的变化,则在每次变轨完成后与变轨前相比 \xzanswer{D} 

\fourchoices
{卫星动能增大,引力势能减小}
{卫星动能增大,引力势能增大}
{卫星动能减小,引力势能减小}
{卫星动能减小,引力势能增大}



\item 
\exwhere{$ 2011 $年理综山东卷}
甲、乙为两颗地球卫星,其中甲为地球同步卫星,乙的运行高度低于甲的运行高度,两卫星轨道均可视为圆轨道。以下判断正确的是 \xzanswer{AC} 

\fourchoices
{甲的周期大于乙的周期 }
{乙的速度大于第一宇宙速度}
{甲的加速度小于乙的加速度 }
{甲在运行时能经过北极的正上方}



\item 
\exwhere{$ 2011 $年理综广东卷}
已知地球质量为$ M $,半径为$ R $,自转周期为$ T $,地球同步卫星质量为$ m $,引力常量为$ G $。有关同步卫星,下列表述正确的是 \xzanswer{BD} 

\fourchoices
{卫星距离地面的高度为$\sqrt [ 3 ] { \frac { G M T ^ { 2 } } { 4 \pi ^ { 2 } } }$}
{卫星的运行速度小于第一宇宙速度}
{卫星运行时受到的向心力大小为$G \frac { M m } { R ^ { 2 } }$}
{卫星运行的向心加速度小于地球表面的重力加速度}


\item 
\exwhere{$ 2013 $年新课标 \lmd{1} 卷}
$ 2012 $年$ 6 $曰$ 18 $日,神州九号飞船与天宫一号目标飞行器在离地面$ 343 \ km $的近圆形轨道上成功进行了我国首次载人空间交会对接。对接轨道所处的空间存在极其稀薄的大气,下列说法正确的是 \xzanswer{BC} 

\fourchoices
{为实现对接,两者运行速度的大小都应介于第一宇宙速度和第二宇宙速度之间}
{如不加干预,在运行一段时间后,天宫一号的动能可能会增加}
{如不加干预,天宫一号的轨道高度将缓慢降低}
{航天员在天宫一号中处于失重状态,说明航天员不受地球引力作用}




\item 
\exwhere{$ 2015 $年理综福建卷}
如图,若两颗人造卫星$ a $和$ b $均绕地球做匀速圆周运动,$ a $、$ b $到地心$ O $的距离分别为$ r_{1} $、$ r_{2} $,线速度大小分别为$ v_{1} $、$ v_{2} $,则 \xzanswer{A} 

\begin{figure}[h!]
\centering
\includesvg[width=0.19\linewidth]{picture/svg/593}
\end{figure}

\fourchoices
{$ \quad \frac { v _ { 1 } } { v _ { 2 } } = \sqrt { \frac { r _ { 2 } } { r _ { 1 } } } $}
{$ \quad \frac { v _ { 1 } } { v _ { 2 } } = \sqrt { \frac { r _ { 1 } } { r _ { 2 } } } $}
{$ \quad \frac { v _ { 1 } } { v _ { 2 } } = \left( \frac { r _ { 2 } } { r _ { 1 } } \right) ^ { 2 } $}
{$ \quad \frac { v _ { 1 } } { v _ { 2 } } = \left( \frac { r _ { 1 } } { r _ { 2 } } \right) ^ { 2 } $}


\item 
\exwhere{$ 2015 $年理综山东卷}
如图,拉格朗日点$ L_{1} $位于地球和月球连线上,处在该点的物体在地球和月球引力的共同作用下,可与月球一起以相同的周期绕地球运动。据此,科学家设想在拉格朗日点$ L_{1} $建立空间站,使其与月球同周期绕地球运动。以$ a_{1} $、$ a_{2} $分别表示该空间站和月球向心加速度的大小,$ a_{3} $表示地球同步卫星向心加速度的大小。以下判断正确的是 \xzanswer{D} 
\begin{figure}[h!]
\centering
\includesvg[width=0.3\linewidth]{picture/svg/594}
\end{figure}

\fourchoices
{$ a_{2} > $ $ a3> $ $ a_{1} $   }
{$ a_{2} > $ $ a_{1} > $ $ a_{3} $}
{$ a_3> $ $ a_{1} $ $ > $ $ a_{2} $  }
{$ a_3> $ $ a_{2} $ $ > $ $ a_{1} $}

\item 
\exwhere{$ 2017 $年新课标$ \lmd{3} $卷}
$ 2017 $年$ 4 $月,我国成功发射的天舟一号货运飞船与天宫二号空间实验室完成了首次交会对接,对接形成的组合体仍沿天宫二号原来的轨道(可视为圆轨道)运行。与天宫二号单独运行相比,组合体运行的 \xzanswer{C} 

\fourchoices
{周期变大 }
{速率变大}
{动能变大 }
{向心加速度变大}



\item 
\exwhere{$ 2017 $年江苏卷}
“天舟一号”货运飞船于$ 2017 $年$ 4 $月$ 20 $日在文昌航天发射中心成功发射升空,与“天宫二号”空间实验室对接前,“天舟一号”在距离地面约$ 380 $ $ km $的圆轨道上飞行,则其 \xzanswer{BCD} 


\fourchoices
{角速度小于地球自转角速度}
{线速度小于第一宇宙速度}
{周期小于地球自转周期}
{向心加速度小于地面的重力加速度}



\item 
\exwhere{$ 2015 $年理综新课标$ \lmd{2} $卷}
由于卫星的发射场不在赤道上,同步卫星发射后需要从转移轨道经过调整再进入地球同步轨道。当卫星在转移轨道上飞经赤道上空时,发动机点火,给卫星一附加速度,使卫星沿同步轨道运行。已知同步卫星的环绕速度约为$ 3.1 \times 10^3 $ $ m $ $ /s $,某次发射卫星飞经赤道上空时的速度为$ 1.55 \times 10^3 \ m/s $,此时卫星的高度与同步轨道的高度相同,转移轨道和同步轨道的夹角为$ 30 ^{ \circ } $,如图所示,发动机给卫星的附加速度的方向和大小约为 \xzanswer{B} 
\begin{figure}[h!]
\centering
\includesvg[width=0.34\linewidth]{picture/svg/595}
\end{figure}

\fourchoices
{西偏北方向,$ 1.9 \times 10^3 \ m/s $ }
{东偏南方向,$ 1.9 \times 10^3 \ m/s $}
{西偏北方向,$ 2.7 \times 10^3 \ m/s $ }
{东偏南方向,$ 2.7 \times 10^3 \ m/s $}


\item 
\exwhere{$ 2016 $年江苏卷}
如图所示,两质量相等的卫星$ A $、$ B $绕地球做匀速圆周运动,用$ R $、$ T $、$ E_{k} $、$ S $分别表示卫星的轨道半径、周期、动能、与地心连线在单位时间内扫过的面积。下列关系式正确的有 \xzanswer{AD} 
\begin{figure}[h!]
\centering
\includesvg[width=0.2\linewidth]{picture/svg/596}
\end{figure}

\fourchoices
{$ T_A>T_B $ }
{$ E_{kA} > E_{kB} $}
{$ S_A=S_B $ }
{$\frac { R _ { A } ^ { 3 } } { T _ { A } ^ { 2 } } = \frac { R _ { B } ^ { 3 } } { T _ { B } ^ { 2 } }$}




\item 
\exwhere{$ 2013 $年安徽卷}
质量为$ m $的人造地球卫星与地心的距离为$ r $时,引力势能可表示为$E _ { p } = - \frac { G M m } { r }$,其中$ G $为引力常量,$ M $为地球质量。该卫星原来的在半径为$ R_{1} $的轨道上绕地球做匀速圆周运动,由于受到极稀薄空气的摩擦作用,飞行一段时间后其圆周运动的半径变为$ R_{2} $,此过程中因摩擦而产生的热量为 \xzanswer{C} 

\fourchoices
{$G M m \left( \frac { 1 } { R _ { 2 } } - \frac { 1 } { R 1 } \right) \quad$}
{$G M m \left( \frac { 1 } { R _ { 1 } } - \frac { 1 } { R _ { 2 } } \right) \quad$}
{$\frac { G M m } { 2 } \left( \frac { 1 } { R _ { 2 } } - \frac { 1 } { R _ { 1 } } \right) \quad$}
{$\frac { G M m } { 2 } \left( \frac { 1 } { R _ { 1 } } - \frac { 1 } { R _ { 2 } } \right)$}



\item 
\exwhere{$ 2013 $年海南卷}
“北斗”卫星导航定位系统由地球静止轨道卫星(同步卫星)、中轨道卫星和倾斜同步卫星组成。地球静止轨道卫星和中轨道卫星都在圆轨道上运行,它们距地面的高度分别约为地球半径的$ 6 $倍和$ 3.4 $倍,下列说法中正确的是 \xzanswer{A} 

\fourchoices
{静止轨道卫星的周期约为中轨道卫星的$ 2 $倍}
{静止轨道卫星的线速度大小约为中轨道卫星的$ 2 $倍}
{静止轨道卫星的角速度大小约为中轨道卫星的$ 1/7 $}
{静止轨道卫星的向心加速度大小约为中轨道卫星的$ 1/7 $}



\item 
\exwhere{$ 2016 $年新课标 \lmd{1} 卷}
利用三颗位置适当的地球同步卫星,可使地球赤道上任意两点之间保持无线电通讯。目前,地球同步卫星的轨道半径约为地球半径的$ 6.6 $倍。假设地球的自转周期变小,若仍仅用三颗同步卫星来实现上述目的,则地球自转周期的最小值约为 \xzanswer{B} 
\fourchoices
{1 h}
{4 h}
{8 h}
{16 h}



\item 
\exwhere{$ 2014 $年理综山东卷}
$ 2013 $年我国相继完成“神十”与“天宫”对接、“嫦娥”携“玉兔”落月两大航天工程。某航天受好者提出“玉兔”回家的设想:如图,将携带“玉兔”的返回系统由月球表面发射到$ h $高度的轨道上,与在该轨道绕月球做圆周运动的飞船对接,然后由飞船送“玉兔”返回地球。设“玉兔”质量为$ m $,月球为$ R $,月面的重力加速度为$ g $月。以月面为零势能面。“玉兔”在$ h $高度的引力势能可表示为$E _ { p } = \frac { G M m h } { R ( R + h ) }$,其中$ G $为引力常量,$ M $为月球质量,若忽略月球的自转,从开始发射到对接完成需要对“玉兔”做的功为 \xzanswer{D} 
\begin{figure}[h!]
\centering
\includesvg[width=0.17\linewidth]{picture/svg/597}
\end{figure}



\fourchoices
{$ \frac { m g _ { \text{月} } R } { R + h } ( h + 2 R ) $}
{$ \frac { m g _ { \text{月} } R} { R + h } ( h + \sqrt{2} R ) $}
{$ \frac { m g _ { \text{月} } R} { R + h } ( h + \frac{\sqrt{2}}{2} R ) $}
{$ \frac { m g _ { \text{月} } R} { R + h } ( h + \frac{ 1 }{ 2 } R ) $}



\item 
\exwhere{$ 2014 $年理综大纲卷}
已知地球的自转周期和半径分别为$ T $和$ R $,地球同步卫星$ A $的圆轨道半径为$ h $。 卫星$ B $沿半径为$ r $ ($ r < h $)的圆轨道在地球赤道的正上方运行,其运行方向与地球自转方向相同。求:
\begin{enumerate}
\renewcommand{\labelenumii}{(\arabic{enumii})}
\item 
卫星$ B $做圆周运动的周期;


\item 
卫星$ A $和$ B $连续地不能直接通讯的最长时间间隔(信号传输时间可忽略)。

\end{enumerate}


\banswer{
\begin{enumerate}
\renewcommand{\labelenumi}{\arabic{enumi}.}
% A(\Alph) a(\alph) I(\Roman) i(\roman) 1(\arabic)
%设定全局标号series=example	%引用全局变量resume=example
%[topsep=-0.3em,parsep=-0.3em,itemsep=-0.3em,partopsep=-0.3em]
%可使用leftmargin调整列表环境左边的空白长度 [leftmargin=0em]
\item
$\left( \frac { r } { h } \right) ^ { \frac { 3 } { 2 } } T$
\item 
$\frac { r ^ { 3 / 2 } } { \pi \left( h ^ { 3 / 2 } - r ^ { 3 / 2 } \right) } \left( \arcsin \frac { R } { h } + \arcsin \frac { R } { r } \right)$


\end{enumerate}


}











\end{enumerate}

