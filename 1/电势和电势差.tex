\bta{第六讲$ \quad $电势和电势差}

\begin{enumerate}[leftmargin=0em]
\renewcommand{\labelenumi}{\arabic{enumi}.}
% A(\Alph) a(\alph) I(\Roman) i(\roman) 1(\arabic)
%设定全局标号series=example	%引用全局变量resume=example
%[topsep=-0.3em,parsep=-0.3em,itemsep=-0.3em,partopsep=-0.3em]
%可使用leftmargin调整列表环境左边的空白长度 [leftmargin=0em]
\item
\exwhere{$ 2015 $年上海卷}
静电场是\tk{静止电荷}周围空间存在的一种物质;通常用\tk{电势}来描述电场的能的性质。


\item
\exwhere{$ 2011 $年海南卷}
关于静电场,下列说法正确的是 \xzanswer{D} 


\fourchoices
{电势等于零的物体一定不带电}
{电场强度为零的点,电势一定为零}
{同一电场线上的各点,电势一定相等}
{负电荷沿电场线方向移动时,电势能一定增加}




\item
\exwhere{$ 2014 $年理综新课标\lmd{2}卷}
关于静电场的电场强度和电势,下列说法正确的是 \xzanswer{AD} 


\fourchoices
{电场强度的方向处处与等势面垂直 }
{电场强度为零的地方,电势也为零}
{随着电场强度的大小逐渐减小,电势也逐渐降低}
{任一点的电场强度总是指向该点电势降落最快的方向}






\item
\exwhere{$ 2014 $年理综北京卷}
如图所示,实线表示某静电场的电场线,虚线表示该电场的等势面。下列判断正确的是 \xzanswer{D} 
\begin{figure}[h!]
\centering
\includesvg[width=0.19\linewidth]{picture/svg/058}
\end{figure}

\fourchoices
{$ 1 $、$ 2 $两点的场强相等}
{$ 1 $、$ 3 $两点的场强相等}
{$ 1 $、$ 2 $两点的电势相等}
{$ 2 $、$ 3 $两点的电势相等}




\item
\exwhere{$ 2014 $年物理江苏卷}
如图所示,一圆环上均匀分布着正电荷, $ x $ 轴垂直于环面且过圆心 $ O $. 下列关于 $ x $ 轴上的电场强度和电势的说法中正确的是 \xzanswer{B} 
\begin{figure}[h!]
\centering
\includesvg[width=0.19\linewidth]{picture/svg/059}
\end{figure}


\fourchoices
{$ O $ 点的电场强度为零,电势最低}
{$ O $ 点的电场强度为零,电势最高}
{从 $ O $ 点沿 $ x $ 轴正方向,电场强度减小,电势升高}
{从 $ O $ 点沿 $ x $ 轴正方向,电场强度增大,电势降低}





\item
\exwhere{$ 2014 $年理综广东卷}
如图所示,光滑绝缘的水平桌面上,固定着一个带电量为$ +Q $的小球$ P $,带电量分别为$ -q $和$ +2q $的小球$ M $和$ N $,由绝缘细杆相连,静止在桌面上,$ P $与$ M $相距$ L $,$ P $、$ M $和$ N $视为点电荷,下列说法正确的是	 \xzanswer{BD} 
\begin{figure}[h!]
\centering
\includesvg[width=0.19\linewidth]{picture/svg/060}
\end{figure}


\fourchoices
{$ M $与$ N $的距离大于$ L $}
{$ P $、$ M $和$ N $在同一直线上}
{在$ P $产生的电场中,$ M $、$ N $处的电势相同}
{$ M $、$ N $及细杆组成的系统所受合外力为零}



\item
\exwhere{$ 2016 $年新课标\lmd{3}卷}
关于静电场的等势面,下列说法正确的是 \xzanswer{B} 


\fourchoices
{两个电势不同的等势面可能相交}
{电场线与等势面处处相互垂直}
{同一等势面上各点电场强度一定相等}
{将一负试探电荷从电势较高的等势面移至电势较低的等势面,电场力做正功}




\item
\exwhere{$ 2014 $年物理海南卷}
如图($ a $),直线$ MN $表示某电场中一条电场线,$ a $、$ b $是线上的两点。将一带负电荷的粒子从$ a $点处由静止释放,粒子从$ a $运动到$ b $过程中的$ v-t $图线如图($ b $)所示,设$ a $、$ b $两点的电势分别为$\varphi _ { a } , \varphi _ { b }$,场强大小分别为$E _ { a } , E _ { b }$,粒子在$ a $、$ b $两点的电势能分别为$ W_{a} $、$ W_{b} $,不计重力,则有 \xzanswer{BD} 
\begin{figure}[h!]
\centering
\includesvg[width=0.33\linewidth]{picture/svg/061}
\end{figure}


\fourchoices
{$\varphi _ { a } > \varphi _ { b }$}
{$E _ { a } > E _ { b }$}
{$E _ { a } < E _ { b }$}
{$W _ { a } > W _ { b }$}




\item
\exwhere{$ 2012 $年理综山东卷}
图中虚线为一组间距相等的同心圆,圆心处固定一带正电的点电荷。一带电粒子以一定初速度射入电场,实线为粒子仅在电场力作用下的运动轨迹,$ a $、$ b $、$ c $三点是实线与虚线的交点。则该粒子 \xzanswer{CD} 
\begin{figure}[h!]
\centering
\includesvg[width=0.19\linewidth]{picture/svg/062}
\end{figure}


\fourchoices
{带负电}
{在$ c $点受力最大}
{在$ b $点的电势能大于在$ c $点的电势能}
{由$ a $点到$ b $点的动能变化大于由$ b $点到$ c $点的动能变化}





\item
\exwhere{$ 2011 $年上海卷}
电场线分布如图昕示,电场中$ a $、$ b $两点的电场强度大小分别为已知$ E_a $和$ E_b $, 电势分别为$ \phi _ a $和$ \phi_b $,则 \xzanswer{C} 
\begin{figure}[h!]
\centering
\includesvg[width=0.19\linewidth]{picture/svg/063}
\end{figure}
\fourchoices
{$E _ { a } > E _ { b } , \quad \phi _ { a } > \phi _ { b }$}
{$E _ { a } > E _ { b } , \quad \phi _ { a } < \phi _ { b }$}
{$E _ { a } < E _ { b } , \quad \phi _ { a } > \phi _ { b }$}
{$E _ { a } < E _ { b } , \quad \phi _ { a } < \phi _ { b }$}





\item
\exwhere{$ 2011 $年理综山东卷}
如图所示,在两等量异种点电荷的电场中,$ MN $为两电荷连线的中垂线,$ a $、$ b $、$ c $三点所在直线平行于两电荷的连线,且$ a $与$ c $关于$ MN $对称,$ b $点位于$ MN $上,$ d $点位于两电荷的连线上。以下判断正确的是 \xzanswer{BC} 
\begin{figure}[h!]
\centering
\includesvg[width=0.19\linewidth]{picture/svg/064}
\end{figure}


\fourchoices
{$ b $点场强大于$ d $点场强 }
{$ b $点场强小于$ d $点场强}
{$ a $、$ b $两点间的电势差等于$ b $、$ c $两点间的电势差}
{试探电荷$ +q $在$ a $点的电势能小于在$ c $点的电势能}





\item
\exwhere{$ 2012 $年理综重庆卷}
空间中$ P $、$ Q $两点处各固定一个点电荷,其中$ P $点处为正点电荷,$ P $、$ Q $两点附近电场的等势面分布如题$ 20 $图所示,$ a $、$ b $、$ c $、$ d $为电场中的四个点。则 \xzanswer{D} 
\begin{figure}[h!]
\centering
\includesvg[width=0.19\linewidth]{picture/svg/065}
\end{figure}


\fourchoices
{$ P $、$ Q $两点处的电荷等量同种}
{$ a $点和$ b $点的电场强度相同}
{$ c $点的电势低于$ d $点的电势}
{负电荷从$ a $到$ c $,电势能减少}




\item
\exwhere{$ 2013 $年山东卷}
如图所示,在$ x $轴上相距为$ L $的两点固定两个等量异种电荷$ +Q $、$ -Q $,虚线是以$ +Q $所在点为圆心、$ L/2 $为半径的圆,$ a $、$ b $、$ c $、$ d $是圆上的四个点,其中$ a $、$ c $两点在$ x $轴上,$ b $、$ d $两点关于$ x $轴对称,下列说法正确的是 \xzanswer{ABD} 
\begin{figure}[h!]
\centering
\includesvg[width=0.19\linewidth]{picture/svg/066}
\end{figure}


\fourchoices
{$ b $、$ d $两点处的电势相同}
{四个点中$ c $点处的电势最低}
{$ b $、$ d $两点处的电场强度相同}
{将一试探电荷$ +q $沿圆周由$ a $点移至$ c $点,$ +q $的电势能减小}





\item
\exwhere{$ 2015 $年江苏卷}
两个相同的负电荷和一个正电荷附近的电场线分布如图所示。 $ c $ 是两负电荷连线的中点,$ d $ 点在正电荷的正上方,$ c $、$ d $ 到正电荷的距离相等,则 \xzanswer{ACD} 
\begin{figure}[h!]
\centering
\includesvg[width=0.24\linewidth]{picture/svg/067}
\end{figure}


\fourchoices
{$a $ 点的电场强度比 $ b $ 点的大}
{$a $ 点的电势比 $ b $ 点的高}
{$c $ 点的电场强度比 $ d $ 点的大}
{$c $ 点的电势比 $ d $ 点的低}




\item
\exwhere{$ 2014 $年理综新课标\lmd{1}卷}
如图,在正电荷$ Q $的电场中有$ M $、$ N $、$ P $和$ F $四点,$ M $、$ N $、$ P $为直角三角形的三个顶点,$ F $为$ MN $的中点,$\angle M = 30 ^ { \circ }$,$ M $、$ N $、$ P $、$ F $四点处的电势分别用$\varphi _ { M }$、$\varphi _ { N }$、$\varphi _ { N }$、$\varphi _ { F }$表示,已知$\varphi _ { M } = \varphi _ { N } , \varphi _ { P } = \varphi _ { F }$,点电荷$ Q $在$ M $、$ N $、$ P $三点所在平面内,则 \xzanswer{AD} 
\begin{figure}[h!]
\centering
\includesvg[width=0.19\linewidth]{picture/svg/068}
\end{figure}


\fourchoices
{点电荷$ Q $一定在$ MP $连线上}
{连线$ PF $一定在同一个等势面上}
{将正试探电荷从$ P $点搬运到$ N $点,电场力做负功}
{$\varphi _ { P }$大于$\varphi _ { M }$}




\item
\exwhere{$ 2012 $年理综安徽卷}
如图所示,在平面直角 中,有方向平行于坐标平面的匀强电场,其中坐标原点处的电势为$ 0\ V $,点$ A $处的电势为$ 6 $ $ V $, 点$ B $处的电势为$ 3 $ $ V $, 则电场强度的大小为 \xzanswer{A} 
\begin{figure}[h!]
\centering
\includesvg[width=0.29\linewidth]{picture/svg/069}
\end{figure}
\fourchoices
{$ 200\ V/m $}
{$ 200\sqrt{3}\ V/m $}
{$ 100\ V/m $}
{$ 100\sqrt{3}\ V/m $}





\item
\exwhere{$ 2011 $年上海卷}
两个等量异种点电荷位于$ x $轴上,相对原点对称分布,正确描述电势$ \phi $随位置$ x $变化规律的是图 \xzanswer{A} 
\begin{figure}[h!]
\centering
\includesvg[width=0.79\linewidth]{picture/svg/070}
\end{figure}



\newpage
\item
\exwhere{$ 2018 $年北京卷}
\begin{enumerate}
\renewcommand{\labelenumi}{\arabic{enumi}.}
% A(\Alph) a(\alph) I(\Roman) i(\roman) 1(\arabic)
%设定全局标号series=example	%引用全局变量resume=example
%[topsep=-0.3em,parsep=-0.3em,itemsep=-0.3em,partopsep=-0.3em]
%可使用leftmargin调整列表环境左边的空白长度 [leftmargin=0em]
\item
静电场可以用电场线和等势面形象描述。
\begin{figure}[h!]
\centering
\includesvg[width=0.24\linewidth]{picture/svg/071}
\end{figure}
\begin{enumerate}
\renewcommand{\labelenumiii}{\alph{enumiii}.}
% A(\Alph) a(\alph) I(\Roman) i(\roman) 1(\arabic)
%设定全局标号series=example	%引用全局变量resume=example
%[topsep=-0.3em,parsep=-0.3em,itemsep=-0.3em,partopsep=-0.3em]
%可使用leftmargin调整列表环境左边的空白长度 [leftmargin=0em]
\item
请根据电场强度的定义和库仑定律推导出点电荷$ Q $的场强表达式;
\item 
点电荷的电场线和等势面分布如图所示,等势面$ S_1 $、$ S_2 $到点电荷的距离分别为$ r_1 $、$ r_2 $。我们知道,电场线的疏密反映了空间区域电场强度的大小。请计算$ S_1 $、$ S_2 $上单位面积通过的电场线条数之比。




\end{enumerate}
\item 
观测宇宙中辐射电磁波的天体,距离越远单位面积接收的电磁波功率越小,观测越困难。为了收集足够强的来自天体的电磁波,增大望远镜口径是提高天文观测能力的一条重要途径。$ 2016 $年$ 9 $月$ 25 $日,世界上最大的单口径球面射电望远镜$ FAST $在我国贵州落成启用,被誉为“中国天眼”。$ FAST $直径为$ 500 $ $ m $,有效提高了人类观测宇宙的精度和范围。

\begin{enumerate}
\renewcommand{\labelenumiii}{\alph{enumiii}.}
% A(\Alph) a(\alph) I(\Roman) i(\roman) 1(\arabic)
%设定全局标号series=example	%引用全局变量resume=example
%[topsep=-0.3em,parsep=-0.3em,itemsep=-0.3em,partopsep=-0.3em]
%可使用leftmargin调整列表环境左边的空白长度 [leftmargin=0em]
\item
设直径为$ 100 $ $ m $的望远镜能够接收到的来自某天体的电磁波功率为$ P_{1} $,计算$ FAST $能够接收到的来自该天体的电磁波功率$ P_{2} $;
\item 
在宇宙大尺度上,天体的空间分布是均匀的。仅以辐射功率为$ P $的同类天体为观测对象,设直径为$ 100 $ $ m $望远镜能够观测到的此类天体数目是$ N_{0} $,计算$ FAST $能够观测到的此类天体数目$ N $。



\end{enumerate}



\end{enumerate}

\banswer{
\begin{enumerate}
\renewcommand{\labelenumi}{\arabic{enumi}.}
% A(\Alph) a(\alph) I(\Roman) i(\roman) 1(\arabic)
%设定全局标号series=example	%引用全局变量resume=example
%[topsep=-0.3em,parsep=-0.3em,itemsep=-0.3em,partopsep=-0.3em]
%可使用leftmargin调整列表环境左边的空白长度 [leftmargin=0em]
\item
\begin{enumerate}
\renewcommand{\labelenumiii}{\alph{enumiii}.}
% A(\Alph) a(\alph) I(\Roman) i(\roman) 1(\arabic)
%设定全局标号series=example	%引用全局变量resume=example
%[topsep=-0.3em,parsep=-0.3em,itemsep=-0.3em,partopsep=-0.3em]
%可使用leftmargin调整列表环境左边的空白长度 [leftmargin=0em]
\item
在距Q为r的位置放一电荷量为q的检验电荷。
根据库仑定律检验电荷受到的电场力:$F = k \frac { Q q } { r ^ { 2 } }$,
根据电场强度的定义:$E = \frac { F } { q }$, 得$E = k \frac { Q } { r ^ { 2 } }$.	
\item 
穿过两等势面单位面积上的电场线条数之比$\frac { N _ { 1 } } { N _ { 2 } } = \frac { E _ { 1 } } { E _ { 2 } } = \frac { r _ { 2 } ^ { 2 } } { r _ { 1 } ^ { 2 } }$


\end{enumerate}
\item 
\begin{enumerate}
\renewcommand{\labelenumiii}{\alph{enumiii}.}
% A(\Alph) a(\alph) I(\Roman) i(\roman) 1(\arabic)
%设定全局标号series=example	%引用全局变量resume=example
%[topsep=-0.3em,parsep=-0.3em,itemsep=-0.3em,partopsep=-0.3em]
%可使用leftmargin调整列表环境左边的空白长度 [leftmargin=0em]
\item
$P _ { 2 } = \frac { 500 ^ { 2 } } { 100 ^ { 2 } } P _ { 1 } = 25 P _ { 1 }$
\item 
$N = \frac { L ^ { 3 } } { L _ { 0 } ^ { 3 } } N _ { 0 } = 125 N _ { 0 }$


\end{enumerate}


\end{enumerate}
}


\end{enumerate}








