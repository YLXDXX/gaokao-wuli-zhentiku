\bta{简谐振动}

\begin{enumerate}
	%\renewcommand{\labelenumi}{\arabic{enumi}.}
	% A(\Alph) a(\alph) I(\Roman) i(\roman) 1(\arabic)
	%设定全局标号series=example	%引用全局变量resume=example
	%[topsep=-0.3em,parsep=-0.3em,itemsep=-0.3em,partopsep=-0.3em]
	%可使用leftmargin调整列表环境左边的空白长度 [leftmargin=0em]
	\item
\exwhere{$ 2019 $ 年物理江苏卷}
一单摆做简谐运动,在偏角增大的过程中,摆球的 \xzanswer{AC} 

\fourchoices
{位移增大}
{速度增大}
{回复力增大}
{机械能增大}



\item 
\exwhere{$ 2013 $ 年上海卷}
做简谐振动的物体,当它每次经过同一位置时,可能不同的物理量是 \xzanswer{} 

\fourchoices
{位移}
{速度}
{加速度}
{回复力}

\item 
\exwhere{$ 2015 $ 年上海卷}
质点运动的位移 $ x $ 与时间 $ t $ 的关系如图所示,其中做机械振动的是 \xzanswer{ABC} 
\pfourchoices
{\includesvg[width=4.3cm]{picture/svg/GZ-3-tiyou-1308}}
{\includesvg[width=4.3cm]{picture/svg/GZ-3-tiyou-1309}}
{\includesvg[width=4.3cm]{picture/svg/GZ-3-tiyou-1310}}
{\includesvg[width=4.3cm]{picture/svg/GZ-3-tiyou-1311}}



\item 
\exwhere{$ 2013 $ 年安徽卷}
如图所示,质量为 $ M $、倾角为$ \alpha $的斜面体(斜面光滑且足够长)放在粗糙的水平地面上,底部与地
面的动摩擦因数为$ \mu $,斜面顶端与劲度系数为 $ k $、自然长度为 $ l $
的轻质弹簧相连,弹簧的另一端连接着质量为 $ m $ 的物块。压缩
弹簧使其长度为$  \frac{ 3 }{ 4 } l $ 时将物块由静止开始释放,且物块在以后的
运动中,斜面体始终处于静止状态。重力加速度为 $ g $。
\begin{enumerate}
	%\renewcommand{\labelenumi}{\arabic{enumi}.}
	% A(\Alph) a(\alph) I(\Roman) i(\roman) 1(\arabic)
	%设定全局标号series=example	%引用全局变量resume=example
	%[topsep=-0.3em,parsep=-0.3em,itemsep=-0.3em,partopsep=-0.3em]
	%可使用leftmargin调整列表环境左边的空白长度 [leftmargin=0em]
	\item
求物块处于平衡位置时弹簧的长度;



\item 
选物块的平衡位置为坐标原点,沿斜面向下为正方向建立坐标轴,用 $ x $ 表示物块相对于平衡
位置的位移,证明物块做简谐运动;

\item 
求弹簧的最大伸长量;

\item 
为使斜面始终处于静止状态,动摩擦因数$ \mu $应满足什么条件(假设滑动摩擦力等于最大静摩擦
力)?

	
\end{enumerate}
\begin{figure}[h!]
	\flushright
	\includesvg[width=0.25\linewidth]{picture/svg/GZ-3-tiyou-1312}
\end{figure}




\banswer{
\begin{enumerate}
	%\renewcommand{\labelenumi}{\arabic{enumi}.}
	% A(\Alph) a(\alph) I(\Roman) i(\roman) 1(\arabic)
	%设定全局标号series=example	%引用全局变量resume=example
	%[topsep=-0.3em,parsep=-0.3em,itemsep=-0.3em,partopsep=-0.3em]
	%可使用leftmargin调整列表环境左边的空白长度 [leftmargin=0em]
	\item
	$L+\frac{m g \sin \alpha}{k}$
	\item 
	$F_{\text {合 }}=-k x,$ 可知物块作简谐振动。
	\item 
	$\frac{L}{4}+\frac{2 m g \sin \alpha}{k}$
	\item 
	$\mu \geq \frac{(k L+4 m g \sin \alpha) \cos \alpha}{4 M g+4 m g \cos ^{2} \alpha-k L \sin \alpha}$
\end{enumerate}
}






	
	
	
\end{enumerate}

