\bta{第十讲$ \quad $静电感应和静电屏蔽}


\begin{enumerate}[leftmargin=0em]
\renewcommand{\labelenumi}{\arabic{enumi}.}
% A(\Alph) a(\alph) I(\Roman) i(\roman) 1(\arabic)
%设定全局标号series=example	%引用全局变量resume=example
%[topsep=-0.3em,parsep=-0.3em,itemsep=-0.3em,partopsep=-0.3em]
%可使用leftmargin调整列表环境左边的空白长度 [leftmargin=0em]
\item
\exwhere{$ 2016 $年浙江卷}
如图所示,两个不带电的导体$ A $和$ B $,用一对绝缘柱支持使它们彼此接触。把一带正电荷的物体$ C $置于$ A $附近,贴在$ A $、$ B $下部的金属箔都张开,则 \xzanswer{C} 
\begin{figure}[h!]
\centering
\includesvg[width=0.23\linewidth]{picture/svg/128}
\end{figure}



\fourchoices
{此时$ A $带正电,$ B $带负电}
{此时$ A $电势低,$ B $电势高}
{移去$ C $,贴在$ A $、$ B $下部的金属箔都闭合}
{先把$ A $和$ B $分开,然后移去$ C $,贴在$ A $、$ B $下部的金属箔都闭合}





\item
\exwhere{$ 2012 $年理综广东卷}
图$ 5 $是某种静电矿料分选器的原理示意图,带电矿粉经漏斗落入水平匀强电场后,分落在收集板中央的两侧,对矿粉分离的过程,下列表述正确的有 \xzanswer{BD} 
\begin{figure}[h!]
\centering
\includesvg[width=0.23\linewidth]{picture/svg/129}
\end{figure}



\fourchoices
{带正电的矿粉落在右侧}
{电场力对矿粉做正功}
{带负电的矿粉电势能变大}
{带正电的矿粉电势能变小}





\item
\exwhere{$ 2012 $年理综浙江卷}
用金属箔做成一个不带电的圆环,放在干燥的绝缘桌面上。小明同学用绝缘材料做的笔套与头
发摩擦后,将笔套自上向下慢慢靠近圆环,当距离约为$ 0.5 \ cm $时圆环被吸引到笔套上。对上述现象的判断与分析,下列说法正确的是 \xzanswer{ABC} 


\fourchoices
{摩擦使笔套带电}
{笔套靠近圆环时,圆环上、下部感应出异号电荷}
{圆环被吸引到笔套的过程中,圆环所受静电力的合力大于圆环的重力}
{笔套碰到圆环后,笔套所带的电荷立刻被全部中和}




\item
\exwhere{$ 2011 $年理综广东卷}
图为静电除尘器除尘机理的示意图。尘埃在电场中通过某种
机制带电,在电场力的作用下向集尘极迁移并沉积,以达到除尘
的目的。下列表述正确的是 \xzanswer{BD} 
\begin{figure}[h!]
\centering
\includesvg[width=0.23\linewidth]{picture/svg/130}
\end{figure}



\fourchoices
{到达集尘极的尘埃带正电荷}
{电场方向由集尘极指向放电极}
{带电尘埃所受电场力的方向与电场方向相同}
{同一位置带电荷量越多的尘埃所受电场力越大}



\item
\exwhere{$ 2015 $年江苏卷}
静电现象在自然界中普遍存在,我国早在西汉末年已有对静电现象的记载,《春秋纬∙考异邮》中有“玳瑁吸(衣若)”之说,但下列不属于静电现象的是 \xzanswer{C} 


\fourchoices
{梳过头发的塑料梳子吸起纸屑}
{带电小球移至不带电金属球附近,两者相互吸引}
{小线圈接近通电线圈过程中,小线圈中产生电流}
{从干燥的地毯上走过,手碰到金属把手时有被电击的感觉}





\end{enumerate}





