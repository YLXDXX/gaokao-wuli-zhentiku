\bta{热学实验}


\begin{enumerate}
	%\renewcommand{\labelenumi}{\arabic{enumi}.}
	% A(\Alph) a(\alph) I(\Roman) i(\roman) 1(\arabic)
	%设定全局标号series=example	%引用全局变量resume=example
	%[topsep=-0.3em,parsep=-0.3em,itemsep=-0.3em,partopsep=-0.3em]
	%可使用leftmargin调整列表环境左边的空白长度 [leftmargin=0em]
	\item
\exwhere{$ 2011 $年上海卷}
在“用单分子油膜估测分子大小”实验中,
\begin{enumerate}
	%\renewcommand{\labelenumi}{\arabic{enumi}.}
	% A(\Alph) a(\alph) I(\Roman) i(\roman) 1(\arabic)
	%设定全局标号series=example	%引用全局变量resume=example
	%[topsep=-0.3em,parsep=-0.3em,itemsep=-0.3em,partopsep=-0.3em]
	%可使用leftmargin调整列表环境左边的空白长度 [leftmargin=0em]
	\item
某同学操作步骤如下:\\
①取一定量的无水酒精和油酸,制成一定浓度的油酸酒精溶液;\\
②在量筒中滴入一滴该溶液,测出它的体积;\\
③在蒸发皿内盛一定量的水,再滴入一滴油酸酒精溶液,待其散开稳定;\\
④在蒸发皿上覆盖透明玻璃,描出油膜形状,用透明方格纸测量油膜的面积。

改正其中的错误: \hfullline 


\item 
若油酸酒精溶液体积浓度为$ 0.10 \% $,一滴溶液的体积为$ 4.8 \times 10^{-3} \ mL $,其形成的油膜面积为$ 40 \ cm^{2} $,
则估测出油酸分子的直径为
 \underlinegap 
$ m $。

	
\end{enumerate}

 \tk{
\begin{enumerate}
	%\renewcommand{\labelenumi}{\arabic{enumi}.}
	% A(\Alph) a(\alph) I(\Roman) i(\roman) 1(\arabic)
	%设定全局标号series=example	%引用全局变量resume=example
	%[topsep=-0.3em,parsep=-0.3em,itemsep=-0.3em,partopsep=-0.3em]
	%可使用leftmargin调整列表环境左边的空白长度 [leftmargin=0em]
	\item
	②在量筒中滴入$ N $滴溶液,\\
	③在水面先撒上痱子粉。
	\item 
	$ 1.2 \times 10^{-9} $
\end{enumerate}
} 


\item 
\exwhere{$ 2011 $ 年理综全国卷}
在“油膜法估测油酸分子的大小”实验中,有下列实验步骤:

①往边长约为 $ 40 \ cm $ 的浅盘里倒入约 $ 2 \ cm $ 深的水.待水面稳定后将适量的痱子粉均匀地撒在水面上。

②用注射器将事先配好的油酸酒精溶液滴一滴在水面上,待薄膜形状稳定。

③将画有油膜形状的玻璃板平放在坐标纸上,计算出油膜的面积,根据油酸的体积和面积计算出油
酸分子直径的大小。

④用注射器将事先配好的油酸酒精溶液一滴一滴地滴入量筒中,记下量筒内每增加一定体积时的滴
数,由此计算出一滴油酸酒精溶液的体积。

⑤将玻璃板放在浅盘上,然后将油膜的形状用彩笔描绘在玻璃板上。


完成下列填空:
\begin{enumerate}
	%\renewcommand{\labelenumi}{\arabic{enumi}.}
	% A(\Alph) a(\alph) I(\Roman) i(\roman) 1(\arabic)
	%设定全局标号series=example	%引用全局变量resume=example
	%[topsep=-0.3em,parsep=-0.3em,itemsep=-0.3em,partopsep=-0.3em]
	%可使用leftmargin调整列表环境左边的空白长度 [leftmargin=0em]
	\item
上述步骤中,正确的顺序是 \underlinegap 。(填写步骤前面的数字)


\item 
将 $ 1 \ cm^{3} $ 的油酸溶于酒精,制成 $ 300 \ cm^{3} $ 的油酸酒精溶液;测得 $ 1 \ cm^{3} $ 的油酸酒精溶液有 $ 50 $ 滴。
现取一滴该油酸酒精溶液滴在水面上,测得所形成的油膜的面积是 $ 0.13 \ m^{2} $。由此估算出油酸分子
的直径为 \underlinegap $ m $。
(结果保留 $ 1 $ 位有效数字)

	
\end{enumerate}

 \tk{
 \begin{enumerate}
 	%\renewcommand{\labelenumi}{\arabic{enumi}.}
 	% A(\Alph) a(\alph) I(\Roman) i(\roman) 1(\arabic)
 	%设定全局标号series=example	%引用全局变量resume=example
 	%[topsep=-0.3em,parsep=-0.3em,itemsep=-0.3em,partopsep=-0.3em]
 	%可使用leftmargin调整列表环境左边的空白长度 [leftmargin=0em]
 	\item
 	④①②⑤③
 	\item 
 	$ 5 \times 10^{-10} \ m $
 \end{enumerate}
} 



\item 
\exwhere{$ 2013 $ 年上海卷}
利用如图装置可测量大气压强和容器的容积。步骤如下:

①将倒 $ U $ 形玻璃管 $ A $ 的一端通过橡胶软管与直玻璃管 $ B $ 连接,并注入适量的水,另一端插入橡皮
塞,然后塞住烧瓶口,并在 $ A $ 上标注此时水面的位置 $ K $;再将一活塞置于 $ 10 \ m l $ 位置的针筒插入烧
瓶,使活塞缓慢推移至 $ 0 $ 刻度位置;上下移动 $ B $,保持 $ A $ 中的水面位于 $ K $
处,测得此时水面的高度差为 $ 17.1 \ cm $。



②拔出橡皮塞,将针筒活塞置于 $ 0 \ m l $ 位置,使烧瓶与大气相通后再次塞住
瓶口;然后将活塞抽拔至 $ 10 \ m l $ 位置,上下移动 $ B $,使 $ A $ 中的水面仍位于 $ K $,
测得此时玻璃管中水面的高度差为 $ 16.8 \ cm $。
(玻璃管 $ A $ 内气体体积忽略不计,
$ \rho =1.0 \times 10^{3} \ kg/m^{3} $,取 $ g=10 \ m/s^{2} $)
\begin{figure}[h!]
	\centering
	\includesvg[width=0.23\linewidth]{picture/svg/GZ-3-tiyou-1283}
\end{figure}

\begin{enumerate}
	%\renewcommand{\labelenumi}{\arabic{enumi}.}
	% A(\Alph) a(\alph) I(\Roman) i(\roman) 1(\arabic)
	%设定全局标号series=example	%引用全局变量resume=example
	%[topsep=-0.3em,parsep=-0.3em,itemsep=-0.3em,partopsep=-0.3em]
	%可使用leftmargin调整列表环境左边的空白长度 [leftmargin=0em]
	\item
若用 $ V_{0} $ 表示烧瓶容积,$ p_{0} $ 表示大气压强,$ \Delta V $ 示针筒内气体的体积,
$ \Delta p_{1} $、$ \Delta p_{2} $ 表示上述步骤①、②中烧瓶内外气体压强差大小,则步骤①、②中,气体满足的方程分别
为 \underlinegap 、 \underlinegap 。

\item 
由实验数据得烧瓶容积 $ V_{0}=$ \underlinegap $ml $,大气压强 $ p_{0} =$ \underlinegap $Pa $。

\item 
(单选题)倒 $ U $ 形玻璃管 $ A $ 内气体的存在 \underlinegap 。
\fourchoices
{仅对容积的测量结果有影响}
{仅对压强的测量结果有影响}
{对二者的测量结果均有影响}
{对二者的测量结果均无影响}



	

\end{enumerate}

 \tk{
\begin{enumerate}
	%\renewcommand{\labelenumi}{\arabic{enumi}.}
	% A(\Alph) a(\alph) I(\Roman) i(\roman) 1(\arabic)
	%设定全局标号series=example	%引用全局变量resume=example
	%[topsep=-0.3em,parsep=-0.3em,itemsep=-0.3em,partopsep=-0.3em]
	%可使用leftmargin调整列表环境左边的空白长度 [leftmargin=0em]
	\item
	$p_{0}\left(V_{0}+\Delta V\right)=\left(p_{0}+\Delta p_{1}\right) V_{0}$\\
	$p_{0} V_{0}=\left(p_{0}-\Delta p_{2}\right)\left(V_{0}+\Delta V\right)$
	\item 
	$ 560 $ \quad $ 9.58 \times 10^{4} $
	\item 
	A
\end{enumerate}
} 


\item 
\exwhere{$ 2012 $ 年物理上海卷}
右图为“研究一定质量气体在压强不变的条件下,体积变化与温度变化关系”的实验装置
示意图。粗细均匀的弯曲玻璃管 $ A $ 臂插入烧瓶,$ B $ 臂与玻璃管 $ C $ 下部用
橡胶管连接,$ C $ 管开口向上,一定质量的气体被水银封闭于烧瓶内。开
始时,$ B $、$ C $ 内的水银面等高。
\begin{figure}[h!]
	\centering
	\includesvg[width=0.23\linewidth]{picture/svg/GZ-3-tiyou-1284}
\end{figure}


\begin{enumerate}
	%\renewcommand{\labelenumi}{\arabic{enumi}.}
	% A(\Alph) a(\alph) I(\Roman) i(\roman) 1(\arabic)
	%设定全局标号series=example	%引用全局变量resume=example
	%[topsep=-0.3em,parsep=-0.3em,itemsep=-0.3em,partopsep=-0.3em]
	%可使用leftmargin调整列表环境左边的空白长度 [leftmargin=0em]
	\item
若气体温度升高,为使瓶内气体的压强不变,应将 $ C $ 管 \underlinegap 
(填:“向上”或“向下”移动),直至 \underlinegap ;

\item 
(单选)实验中多次改变气体温度, 用$ \Delta t $ 表示气体升高的摄氏温度,用$ \Delta h $ 表示 $ B $ 管内水银面
高度的改变量。根据测量数据作出的图线是 \underlinegap .
\pfourchoices
{\includesvg[width=4.3cm]{picture/svg/GZ-3-tiyou-1285}}
{\includesvg[width=4.3cm]{picture/svg/GZ-3-tiyou-1286}}
{\includesvg[width=4.3cm]{picture/svg/GZ-3-tiyou-1287}}
{\includesvg[width=4.3cm]{picture/svg/GZ-3-tiyou-1288}}



	
\end{enumerate}


 \tk{
\begin{enumerate}
	%\renewcommand{\labelenumi}{\arabic{enumi}.}
	% A(\Alph) a(\alph) I(\Roman) i(\roman) 1(\arabic)
	%设定全局标号series=example	%引用全局变量resume=example
	%[topsep=-0.3em,parsep=-0.3em,itemsep=-0.3em,partopsep=-0.3em]
	%可使用leftmargin调整列表环境左边的空白长度 [leftmargin=0em]
	\item
	向下 \\
	$ B $、$ C $ 两管内水银面等高
	\item 
	A
\end{enumerate}
} 



\item 
\exwhere{$ 2015 $ 年上海卷}
简易温度计构造如图所示。两内径均匀的竖直玻璃管下端与软管连接,
在管中灌入液体后,将左管上端通过橡皮塞插入玻璃泡。在标准大气压下,
调节右管的高度,使左右两管的液面相平,在左管液面位置标上相应的温
度刻度。多次改变温度,重复上述操作。
\begin{figure}[h!]
	\centering
	\includesvg[width=0.23\linewidth]{picture/svg/GZ-3-tiyou-1289}
\end{figure}
\begin{enumerate}
	%\renewcommand{\labelenumi}{\arabic{enumi}.}
	% A(\Alph) a(\alph) I(\Roman) i(\roman) 1(\arabic)
	%设定全局标号series=example	%引用全局变量resume=example
	%[topsep=-0.3em,parsep=-0.3em,itemsep=-0.3em,partopsep=-0.3em]
	%可使用leftmargin调整列表环境左边的空白长度 [leftmargin=0em]
	\item
(单选题)此温度计的特点是 \underlinegap .
\fourchoices
{刻度均匀,刻度值上小下大}
{刻度均匀,刻度值上大下小}
{刻度不均匀,刻度值上小下大}
{刻度不均匀,刻度值上大下小}

\item 
(多选题)影响这个温度计灵敏度的因素有 \underlinegap .
\fourchoices
{液体密度}
{玻璃泡大小}
{左管内径粗细}
{右管内径粗细}


\item 
若管中液体是水银,当大气压变为 $ 75 \ cm Hg $ 时,用该温度计测得的温度值 \underlinegap (选填“偏
大”或“偏小”)。为测得准确的温度,在测量时需
 \hfullline 
。
	
\end{enumerate}


 \tk{
\begin{enumerate}
	%\renewcommand{\labelenumi}{\arabic{enumi}.}
	% A(\Alph) a(\alph) I(\Roman) i(\roman) 1(\arabic)
	%设定全局标号series=example	%引用全局变量resume=example
	%[topsep=-0.3em,parsep=-0.3em,itemsep=-0.3em,partopsep=-0.3em]
	%可使用leftmargin调整列表环境左边的空白长度 [leftmargin=0em]
	\item
	A
	\item 
	BC
	\item 
	偏大;调整两管液面高度差,使右管液面比左管液面高 $ 1 \ cm $,然后读数
\end{enumerate}
} 


	
	
	
\end{enumerate}

