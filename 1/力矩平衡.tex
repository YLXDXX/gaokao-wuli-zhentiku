\bta{力矩平衡}
\begin{enumerate}
\renewcommand{\labelenumi}{\arabic{enumi}.}
% A(\Alph) a(\alph) I(\Roman) i(\roman) 1(\arabic)
%设定全局标号series=example	%引用全局变量resume=example
%[topsep=-0.3em,parsep=-0.3em,itemsep=-0.3em,partopsep=-0.3em]
%可使用leftmargin调整列表环境左边的空白长度 [leftmargin=0em]
\item
\exwhere{$ 2016 $年上海卷}
如图,始终竖直向上的力$ F $作用在三角板$ A $端,使其绕$ B $点在竖直平面内缓慢地沿顺时针方向转动一小角度,力$ F $对$ B $点的力矩为$ M $,则转动过程中 \xzanswer{A} 
\begin{figure}[h!]
\centering
\includesvg[width=0.19\linewidth]{picture/svg/458}
\end{figure}

\fourchoices
{$ M $减小,$ F $增大}
{ $ M $减小,$ F $减小}
{$ M $增大,$ F $增大}
{$ M $增大,$ F $减小}





\item 
\exwhere{$ 2012 $年上海卷}	
如图,竖直轻质悬线上端固定,下端与均质硬棒$ AB $中点连接,棒长为线长二倍。棒的$ A $端用铰链固定在墙上,棒处于水平状态。改变悬线长度,使线与棒的连接点逐渐右移,并保持棒仍处于水平状态。则悬线拉力 \xzanswer{A} 
\begin{figure}[h!]
\centering
\includesvg[width=0.23\linewidth]{picture/svg/459}
\end{figure}


\fourchoices
{逐渐减小}
{逐渐增大}
{先减小后增大}
{先增大后减小}


\item 
\exwhere{$ 2013 $年上海卷}
如图,倾角为$ 37 ^{ \circ } $,质量不计的支架$ ABCD $的$ D $端有一大小与质量均可忽略的光滑定滑轮,$ A $点处有一固定转轴,$ CA \perp AB $,$ DC=CA=0.3\ m $。质量$ m=lkg $的物体置于支架的$ B $端,并与跨过定滑轮的轻绳相连,绳另一端作用一竖直向下的拉力$ F $,物体在拉力作用下沿$ BD $做匀速直线运动,己知物体与$ BD $间的动摩擦因数$ \mu =0.3 $。为保证支架不绕$ A $点转动,物体向上滑行的最大距离$ s= $ \tk{0.248} $ m $。若增大$ F $后,支架仍不绕$ A $点转动,物体能向上滑行的最大距离$ s ^{\prime} $ \tk{等于} $ s $(填:“大于”、“等于”或“小于”。)(取$ \sin 37 ^{ \circ } =0.6 $,$ \cos 37 ^{ \circ } =0.8 $)
\begin{figure}[h!]
\centering
\includesvg[width=0.3\linewidth]{picture/svg/460}
\end{figure}

\item 
\exwhere{$ 2015 $年上海卷}
如图,在场强大小为$ E $、水平向右的匀强电场中,一轻杆可绕固定转轴$ O $在竖直平面内自由转动。杆的两端分别固定两电荷量均为$ q $的小球$ A $、$ B $,$ A $带正电,$ B $带负电;$ A $、$ B $两球到转轴$ O $的距离分别为$ 2l $、$ l $,所受重力大小均为电场力大小的$ \sqrt{3} $倍。开始时杆与电场间夹角为$ \theta $($ 90 ^{ \circ } \leq \theta \leq 180 ^{ \circ } $)。将杆从初始位置由静止释放,以$ O $点为重力势能和电势能零点。求:
\begin{enumerate}
\renewcommand{\labelenumi}{\arabic{enumi}.}
% A(\Alph) a(\alph) I(\Roman) i(\roman) 1(\arabic)
%设定全局标号series=example	%引用全局变量resume=example
%[topsep=-0.3em,parsep=-0.3em,itemsep=-0.3em,partopsep=-0.3em]
%可使用leftmargin调整列表环境左边的空白长度 [leftmargin=0em]
\item
初始状态的电势能$ W_e $;
\item 
杆在平衡位置时与电场间的夹角;
\item 
杆在电势能为零处的角速度$ \omega $。



\end{enumerate}
\begin{figure}[h!]
\flushright
\includesvg[width=0.25\linewidth]{picture/svg/461}
\end{figure}


\banswer{
\begin{enumerate}
\renewcommand{\labelenumi}{\arabic{enumi}.}
% A(\Alph) a(\alph) I(\Roman) i(\roman) 1(\arabic)
%设定全局标号series=example	%引用全局变量resume=example
%[topsep=-0.3em,parsep=-0.3em,itemsep=-0.3em,partopsep=-0.3em]
%可使用leftmargin调整列表环境左边的空白长度 [leftmargin=0em]
\item
$ -3qEl \cos \theta $
\item 
$ 30 ^{\circ} $
\item 
当$ \theta < 150 ^{ \circ } $时,$\omega = \sqrt { \frac { 2 \sqrt { 3 } ( 1 - \sin \theta ) - 6 \cos \theta } { 5 \sqrt { 3 } l } g }$

\end{enumerate}


}









\end{enumerate}




