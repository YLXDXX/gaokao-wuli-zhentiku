\bta{波的叠加、干涉、衍射、偏振、 多普勒效应}

\begin{enumerate}
	%\renewcommand{\labelenumi}{\arabic{enumi}.}
	% A(\Alph) a(\alph) I(\Roman) i(\roman) 1(\arabic)
	%设定全局标号series=example	%引用全局变量resume=example
	%[topsep=-0.3em,parsep=-0.3em,itemsep=-0.3em,partopsep=-0.3em]
	%可使用leftmargin调整列表环境左边的空白长度 [leftmargin=0em]
	\item
\exwhere{$ 2019 $ 年物理江苏卷}
将两支铅笔并排放在一起,中间留一条狭缝,通过这条狭缝去看与其平行
的日光灯,能观察到彩色条纹,这是由于光的 \underlinegap (选填“折射”“干涉”或“衍射”).当缝的宽度 \underlinegap 
(选填“远大于”或“接近”)光波的波长时,这种现象十分明显.

 \tk{ 衍射 \quad 接近} 

\item 
\exwhere{$ 2013 $ 年全国卷大纲卷}
在学校运动场上 $ 50 \ m $ 直跑道的两端,分别安装了由同一信号发生器带动的两个相同的扬声器。
两个扬声器连续发出波长为 $ 5 \ m $ 的声波。一同学从该跑道的中点出发,向某一端点缓慢行进 $ 10 \ m $。
在此过程中,他听到的扬声器声音由强变弱的次数为 \xzanswer{B} 

\fourchoices
{$ 2 $}
{$ 4 $}
{$ 6 $}
{$ 8 $}

\item 
\exwhere{$ 2011 $ 年理综浙江卷}
关于波动,下列说法正确的是 \xzanswer{BD} 

\fourchoices
{各种波均会发生偏振现象}
{用白光做单缝衍射与双缝干涉实验,均可看到彩色条纹}
{声波传播过程中,介质中质点的运动速度等于声波的传播速度}
{已知地震波的纵波速度大于横波速度,此性质可用于横波的预警}

\item 
\exwhere{$ 2011 $年上海卷}
两波源$ S_{1} $、$ S_{2} $在水槽中形成的波形如图所示,其中实线表示波峰,虚线表示波谷,则 \xzanswer{B} 
\begin{figure}[h!]
	\centering
	\includesvg[width=0.23\linewidth]{picture/svg/GZ-3-tiyou-1404}
\end{figure}

\fourchoices
{在两波相遇的区域中会产生干涉}
{在两波相遇的区域中不会产生干涉}
{$ a $点的振动始终加强}
{$ a $点的振动始终减弱}


\item 
\exwhere{$ 2014 $ 年理综大纲卷}
两列振动方向相同、振幅分别为 $ A_{1} $ 和 $ A_{2} $ 的相干简谐横波相遇。下列说法正确的是 \xzanswer{AD} 
\fourchoices
{波峰与波谷相遇处质点的振幅为$ | A_{1} - A_{2} | $}
{波峰与波峰相遇处质点离开平衡位置的位移始终为 $ A_{1} + A_{2} $}
{波峰与波谷相遇处质点的位移总是小于波峰与波峰相遇处质点的位移}
{波峰与波峰相遇处质点的振幅一定大于波峰与波谷相遇处质点的振幅}


\item 
\exwhere{$ 2011 $年上海卷}
两列简谐波沿$ x $轴相向而行,波速均为$ v=0.4 \ m /s $,两波源分别位于$ A $、$ B $处,$ t=0 $时的波形如图所
示。当$ t=2.5 \ s $时,$ M $点的位移为 \underlinegap $ cm $,$ N $点的
位移为 \underlinegap $ cm $。
\begin{figure}[h!]
	\centering
	\includesvg[width=0.23\linewidth]{picture/svg/GZ-3-tiyou-1405}
\end{figure}


 \tk{$ 2 $ \quad $ 0 $} 



\item 
\exwhere{$ 2015 $ 年上海卷}
如图,$ P $ 为桥墩,$ A $ 为靠近桥墩浮在水面的叶片,波源 $ S $ 连续振动,形成水波,
此时叶片 $ A $ 静
止不动。为使水波能带动叶片振动,可用的方法是 \xzanswer{B} 
\begin{figure}[h!]
	\centering
	\includesvg[width=0.23\linewidth]{picture/svg/GZ-3-tiyou-1406}
\end{figure}

\fourchoices
{提高波源频率}
{降低波源频率}
{增加波源距桥墩的距离}
{减小波源距桥墩的距离}

\item 
\exwhere{$ 2017 $ 年北京卷}
 物理学原理在现代科技中有许多重要应用。例如,利用波的干涉,可将无线电
波的干涉信号用于飞机降落的导航。
如图所示,两个可发射无线电波的天线对称地固定于飞机跑道两侧,它们类似于杨氏干涉实验中的
双缝。两天线同时都发出波长为 $ \lambda_{1} $ 和 $ \lambda_{2} $ 的无线电波。飞机降落过程中,当接收到 $ \lambda_{1} $ 和 $ \lambda_{2} $ 的信号都
保持最强时,表明飞机已对准跑道。下列说法正确的是 \xzanswer{C} 
\begin{figure}[h!]
	\centering
	\includesvg[width=0.23\linewidth]{picture/svg/GZ-3-tiyou-1408}
\end{figure}

\fourchoices
{天线发出的两种无线电波必须一样强}
{导航利用了 $ \lambda_{1} $ 与 $ \lambda_{2} $ 两种无线电波之间的干涉}
{两种无线电波在空间的强弱分布稳定}
{两种无线电波各自在空间的强弱分布完全重合}


\item 
\exwhere{$ 2018 $ 年浙江卷($ 4 $ 月选考)}
两列频率相同、振幅均为 $ A $ 的简谐横波 $ P $、$ Q $ 分别沿$ +x $ 和$ -x $ 轴方
向在同一介质中传播,两列波的振动方向均沿 $ y $ 轴。某时刻两波的波面如图所示,实线表示 $ P $ 波的
波峰、$ Q $ 波的波谷;虚线表示 $ P $ 波的波谷、$ Q $ 波的波峰。$ a $、$ b $、$ c $
为三个等间距的质点,$ d $ 为 $ b $、$ c $ 中间的质点。下列判断正确的是 \xzanswer{CD} 
\begin{figure}[h!]
	\centering
	\includesvg[width=0.23\linewidth]{picture/svg/GZ-3-tiyou-1409}
\end{figure}


\fourchoices
{质点 $ a $ 的振幅为 $ 2 A $}
{质点 $ b $ 始终静止不动}
{图示时刻质点 $ c $ 的位移为 $ 0 $}
{图示时刻质点 $ d $ 的振动方向沿$ -y $ 轴}





	
	
	
\end{enumerate}

