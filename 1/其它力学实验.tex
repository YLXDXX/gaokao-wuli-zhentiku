\bta{其它力学实验}
\begin{enumerate}
\renewcommand{\labelenumi}{\arabic{enumi}.}
% A(\Alph) a(\alph) I(\Roman) i(\roman) 1(\arabic)
%设定全局标号series=example	%引用全局变量resume=example
%[topsep=-0.3em,parsep=-0.3em,itemsep=-0.3em,partopsep=-0.3em]
%可使用leftmargin调整列表环境左边的空白长度 [leftmargin=0em]
\item
\exwhere{$ 2015 $ 年理综重庆卷}
同学们利用如图 $ 1 $ 所示方法估测反应时间。首先,甲同学捏住直
尺上端,使直尺保持竖直状态,直尺零刻度线位于乙同学的两指之间。当乙
看见甲放开直尺时,立即用手指捏直尺,若捏住位置的刻度读数为 $ x $,则乙
同学的反应时间为 \tk{ $\sqrt{\frac{2 x}{g}}$ } 
(重力加速度为 $ g $)。
\begin{figure}[h!]
\centering
\includesvg[width=0.23\linewidth]{picture/svg/GZ-3-tiyou-0555}
\end{figure}



基于上述原理,某同学用直尺制作测量反应时间的工具,若测量范围为 $ 0 \sim 0.4 \ s $,则所用直尺的长度至少为
\tk{$ 80 $} 
$ cm $($ g $ 取 $ 10 \ m/s^{2} $);若以相等时间间隔
在该直尺的另一面标记出表示反应时间的刻度线,则每个时间间隔在直尺上
对应的长度是
\tk{不相等的} 
的(选填“相等”或“不相等”).




\newpage
\item 
\exwhere{$ 2015 $ 年理综新课标 \lmd{1} 卷}
某物理小组的同学设计了一个粗测玩具小车通过凹形桥最低点时的速度的实验。所用器材有:玩
具小车、压力式托盘秤、凹形桥模拟器(圆弧部分
的半径为 $ R=0.20 \ m $)。
\begin{figure}[h!]
\centering
\includesvg[width=0.43\linewidth]{picture/svg/GZ-3-tiyou-0556}
\end{figure}



完成下列填空:
\begin{enumerate}
\renewcommand{\labelenumi}{\arabic{enumi}.}
% A(\Alph) a(\alph) I(\Roman) i(\roman) 1(\arabic)
%设定全局标号series=example	%引用全局变量resume=example
%[topsep=-0.3em,parsep=-0.3em,itemsep=-0.3em,partopsep=-0.3em]
%可使用leftmargin调整列表环境左边的空白长度 [leftmargin=0em]
\item
将凹形桥模拟器静置于托盘秤上,如图($ a $)
所示,托盘秤的示数为 $ 1.00 \ kg $;

\item 
将玩具小车静置于凹形桥模拟器最低点时,
托盘秤的示数如图($ b $)所示,该示数为 \tk{1.40} $ kg $;

\item 
将小车从凹形桥模拟器某一位置释放,小车
经过最低点后滑向另一侧。此过程中托盘秤的最大示数为 $ m $;多次从同一位置释放小车,记录各次
的 $ m $ 值如下表所示:
\begin{table}[h!]
\centering 
\begin{tabular}{|c|c|c|c|c|c|}
\hline 
序号 & $ 1 $ & $ 2 $ & $ 3 $ & $ 4 $ & $ 5 $
\\
\hline
$ m $($ kg $) & $ 1.80 $ & $ 1.75 $ & $ 1.85 $ & $ 1.75 $ & $ 1.90 $\\ 
\hline 
\end{tabular}
\end{table} 


\item 
根据以上数据,可求出小车经过凹形桥最低点时对桥的压力为 \tk{$ 7.9 $} $ N $;小车通过最低点时的
速度大小为 \tk{$ 1.4 $} $ m/s $。(重力加速度大小取 $ 9.80 \ m/s^{2} $,计算结果保留 $ 2 $ 位有效数字)


\end{enumerate}





\item
\exwhere{$ 2013 $ 年重庆卷}
我国舰载飞机在“辽宁舰”上成功着舰后,某课外活动小组对舰载飞机利用
阻拦索着舰的力学问题很感兴趣。他们找来了木板、钢球、铁钉、橡皮条以及墨水,制作了如图所
示的装置,准备定量研究钢球在橡皮条阻拦下前进的距离与被阻拦前速率的关系。要达到实验目的,
需直接测量的物理量是钢球由静止释放时的
\tk{高度(距水平木板的高度)} 
和在橡皮条阻拦下前进的距离,还必须增加的一种实验器
材是
\tk{刻度尺} 
。忽略钢球所受的摩擦力和空气阻力,重
力加速度已知,根据
\tk{机械能守恒或动能定理} 
定律(定理),可得到钢
球被阻拦前的速率。
\begin{figure}[h!]
\centering
\includesvg[width=0.33\linewidth]{picture/svg/GZ-3-tiyou-0557}
\end{figure}


\newpage
\item 
\exwhere{$ 2016 $ 年新课标$ \lmd{2} $卷}
某物理小组对轻弹簧的弹性势能进行探究,实验装置如图($ a $)所
示:轻弹簧放置在光滑水平桌面上,弹簧左端固定,右端与
一物块接触而不连接,纸带穿过打点计时器并与物块连接。
向左推物块使弹簧压缩一段距离,由静止释放物快,通过测
量和计算,可求得弹簧被压缩后的弹性势能。
\begin{figure}[h!]
\centering
\includesvg[width=0.33\linewidth]{picture/svg/GZ-3-tiyou-0558}
\end{figure}

\begin{enumerate}
\renewcommand{\labelenumi}{\arabic{enumi}.}
% A(\Alph) a(\alph) I(\Roman) i(\roman) 1(\arabic)
%设定全局标号series=example	%引用全局变量resume=example
%[topsep=-0.3em,parsep=-0.3em,itemsep=-0.3em,partopsep=-0.3em]
%可使用leftmargin调整列表环境左边的空白长度 [leftmargin=0em]
\item
实验中涉及下列操作步骤:

①把纸带向左拉直

②松手释放物块

③接通打点计时器电源


④向左推物块使弹簧压缩,并测量弹簧压
缩量

上述步骤正确的操作顺序是 \tk{④①③②} 
(填入代表步骤的序号)。


\item 
图($ b $)中 $ M $ 和 $ L $ 纸带是分别把弹簧
压缩到不同位置后所得到的实际打点结果。打点计时器所用交流电的频率为 $ 50 \ Hz $。由 $ M $ 纸带所给
的数据,可求出在该纸带对应的实验中物块脱离弹簧时的速度为 \tk{$ 1.29 $} $ m/s $。比较两纸带可知,
\tk{$ M $} 
(填“$ M $”或“$ L $”)纸带对应的实验中弹簧被压缩后的弹性势能大。

\begin{figure}[h!]
\centering
\includesvg[width=0.63\linewidth]{picture/svg/GZ-3-tiyou-0559}
\end{figure}

\end{enumerate}





\newpage
\item 
\exwhere{$ 2016 $ 年四川卷}
用如图所示的装置测量弹簧的弹性势能。将弹簧放置在水平气垫导轨上,左端固定,右
端在 $ O $ 点;在 $ O $ 点右侧的 $ B $、$ C $ 位置各安装一个光电门,计时器(图中未画出)与两个光电门相连。
先用米尺测得 $ B $、$ C $ 两点间距离 $ x $,再用带有遮光片的滑块压缩弹簧到某位置 $ A $,静止释放,计时
器显示遮光片从 $ B $ 到 $ C $ 所用的时间 $ t $,用米尺测量 $ A $、$ O $ 之间的距离 $ x $。
\begin{figure}[h!]
\centering
\includesvg[width=0.43\linewidth]{picture/svg/GZ-3-tiyou-0560}
\end{figure}

\begin{enumerate}
\renewcommand{\labelenumi}{\arabic{enumi}.}
% A(\Alph) a(\alph) I(\Roman) i(\roman) 1(\arabic)
%设定全局标号series=example	%引用全局变量resume=example
%[topsep=-0.3em,parsep=-0.3em,itemsep=-0.3em,partopsep=-0.3em]
%可使用leftmargin调整列表环境左边的空白长度 [leftmargin=0em]
\item
计算滑块离开弹簧时速度大小的表达式是 \tk{$v=\frac{x}{t}$} 。



\item 
为求出弹簧的弹性势能,还需要测量 \tk{C} 。
\threechoices
{弹簧原长}
{当地重力加速度}
{滑块(含遮光片)的质量}

\item 
增大 $ A $、$ O $ 之间的距离 $ x $,计时器显示时间 $ t $ 将 \tk{B} 。
\threechoices
{增大}
{减小}
{不变}


\end{enumerate}






\item 
\exwhere{$ 2011 $ 年理综四川卷}
某研究性学习小组进行了如下实验:如图所示,在一端
封闭的光滑细玻璃管中注满清水,水中放一个红蜡做成的小圆柱体 $ R $。将
玻璃管的开口端用胶塞塞紧后竖直倒置且与 $ y $ 轴重合,在 $ R $ 从坐标原点以
速度 $ v_{0} =3 \ cm/s $ 匀速上浮的同时,玻璃管沿 $ x $ 轴正方向做初速为零的匀加速
直线运动。同学们测出某时刻 $ R $ 的坐标为$ (4,6) $
,此时 $ R $ 的速度大小为 \tk{5} 
$ cm/s $,$ R $ 在上升过程中运动轨迹的示意图是
\tk{D} 。($ R $ 视为质点)
\begin{figure}[h!]
\centering
\includesvg[width=0.23\linewidth]{picture/svg/GZ-3-tiyou-0561}\\
\vspace{1.6em}
 \includesvg[width=0.83\linewidth]{picture/svg/GZ-3-tiyou-0562} 
\end{figure}




\newpage
\item 
\exwhere{$ 2018 $ 年浙江卷($ 4 $ 月选考)}
\begin{enumerate}
\renewcommand{\labelenumi}{\arabic{enumi}.}
% A(\Alph) a(\alph) I(\Roman) i(\roman) 1(\arabic)
%设定全局标号series=example	%引用全局变量resume=example
%[topsep=-0.3em,parsep=-0.3em,itemsep=-0.3em,partopsep=-0.3em]
%可使用leftmargin调整列表环境左边的空白长度 [leftmargin=0em]
\item
用图 $ 1 $ 所示装置做“探究功与速度变化的关系”实验
时,除了图中已给出的实验器材外,还需要的测量工具有 \tk{C} (填字母);
\fourchoices
{秒表}
{天平}
{刻度尺}
{弹簧测力计}


\item 
用图$ 2 $所示装置做“验证机械能守恒定律”实验时,释放重物前有以下操作,其中正确的是 \tk{AC} 
(填字母);

\threechoices
{将打点计时器的两个限位孔调节到同一竖直线上}
{手提纸带任意位置}
{使重物靠近打点计时器}

\item 
图$ 3 $是小球做平抛运动的频闪照片,其上覆盖了一张透明的方格纸。已知方格纸每小格的边长
均为$ 0.80 \ cm $。由图可知小球的初速度大小为 \tk{$ 0.66 \sim 0.74 $} $ m/s $(结果保留两位有效数字)
\begin{figure}[h!]
\centering
\includesvg[width=0.83\linewidth]{picture/svg/GZ-3-tiyou-0564}
\end{figure}




\end{enumerate}



\newpage
\item 
\exwhere{$ 2018 $ 年北京卷}
用图 $ 1 $ 所示的实验装置研究小车速度随时间变化的规律。
\begin{figure}[h!]
\centering
\includesvg[width=0.43\linewidth]{picture/svg/GZ-3-tiyou-0565}
\end{figure}


主要实验步骤如下:
\begin{enumerate}
\renewcommand{\labelenumii}{$ \alph{enumii} $.}
% A(\Alph) a(\alph) I(\Roman) i(\roman) 1(\arabic)
%设定全局标号series=example	%引用全局变量resume=example
%[topsep=-0.3em,parsep=-0.3em,itemsep=-0.3em,partopsep=-0.3em]
%可使用leftmargin调整列表环境左边的空白长度 [leftmargin=0em]
\item






安装好实验器材。接通电源后,让拖着纸
带的小车沿长木板运动,重复几次。


\item 
选出一条点迹清晰的纸带,找一个合适的
点当作计时起点 $ O $($ t=0 $),然后每隔相同的
时间间隔 $ T $ 选取一个计数点,如图 $ 2 $ 中 $ A $、$ B $、
$ C $、$ D $、$ E $、$ F \cdots \cdots $所示。
\begin{figure}[h!]
\centering
\includesvg[width=0.83\linewidth]{picture/svg/GZ-3-tiyou-0566}
\end{figure}




\item 
通过测量、计算可以得到在打 $ A $、$ B $、$ C $、$ D $、$ E \cdots \cdots $点时小车的速度,分别记作 $ v_{1} $、$ v_{2} $、$ v_{3} $、$ v_{4} $、
$ v5 \cdots \cdots $


\item 
以速度 $ v $ 为纵轴、时间 $ t $ 为横轴建立直角坐标系,在坐标纸上描点,如图 $ 3 $ 所示。
\begin{figure}[h!]
\centering
\includesvg[width=0.63\linewidth]{picture/svg/GZ-3-tiyou-0567}
\end{figure}

\end{enumerate}




结合上述实验步骤,请你完成下列任务:
\begin{enumerate}
\renewcommand{\labelenumi}{\arabic{enumi}.}
% A(\Alph) a(\alph) I(\Roman) i(\roman) 1(\arabic)
%设定全局标号series=example	%引用全局变量resume=example
%[topsep=-0.3em,parsep=-0.3em,itemsep=-0.3em,partopsep=-0.3em]
%可使用leftmargin调整列表环境左边的空白长度 [leftmargin=0em]
\item
在下列仪器和器材中,还需要使用
的有
\tk{A} 
和
\tk{C} 
(填选项前的字母)。

\threechoices
{电压合适的 $ 50 \ Hz $ 交流电源}
{天平(含砝码)}
{刻度尺}

\item 
在图 $ 3 $ 中已标出计数点 $ A $、$ B $、$ D $、
$ E $ 对应的坐标点,请在该图中\CJKunderdot{标出}计数点
$ C $对应的坐标点,并画出 $ v-t $ 图像。


\item 
观察 $ v-t $ 图像,可以判断小车做匀
变速直线运动,其依据是 \tk{小车的速度随时间均匀变化} 。	$ v-t $
图像斜率的物理意义是 \tk{加速度} 。

\item 
描绘 $ v-t $ 图像前,还不知道小车是否做匀变速直线运动。用平均速度$\frac{\Delta x}{\Delta t}$
表示各计数点的瞬时
速度,从理论上讲,对$ \Delta t $ 的要求是
\tk{越小越好} 
(选填“越小越好”或“与大小无关”)
;从实验的角度看,
选取的$ \Delta x $ 大小与速度测量的误差 \tk{有关} (选填“有关”或“无关”)。





\item 
早在 $ 16 $ 世纪末,伽利略就猜想落体运动的速度应该是均匀变化的。当时只能靠滴水计时,为
此他设计了如图 $ 4 $ 所示的“斜面实验”,反复做了上
百次,验证了他的猜想。请你结合匀变速直线运动的知识,分析说明如何利用伽利略“斜面实验”检
验小球的速度是随时间均匀变化的。
\begin{figure}[h!]
\centering
\includesvg[width=0.43\linewidth]{picture/svg/GZ-3-tiyou-0568}
\end{figure}

\tk{如果小球的初速度为 $ 0 $,其速度 $ v \propto t $,
那么它通过的位移 $ x \propto t^{2} $。因此,只要测量
小球通过不同位移所用的时间,就可以检验
小球的速度是否随时间均匀变化。} 

\end{enumerate}




\newpage
\item 
\exwhere{$ 2013 $ 年浙江卷}
如图所示,装置甲中挂有小桶的细线绕过定滑轮,固定在小车上;装置乙中橡皮筋的
一端固定在导轨的左端,另一端系在小车上。一同学用装置甲和乙分别进行实验,经正确操作获得
两条纸带①和②,纸带上的 $ a $、$ b $、$ c \cdots \cdots $均为打点计时器打出的点。
\begin{figure}[h!]
\centering
\includesvg[width=0.83\linewidth]{picture/svg/GZ-3-tiyou-0570}
\end{figure}




\begin{enumerate}
\renewcommand{\labelenumi}{\arabic{enumi}.}
% A(\Alph) a(\alph) I(\Roman) i(\roman) 1(\arabic)
%设定全局标号series=example	%引用全局变量resume=example
%[topsep=-0.3em,parsep=-0.3em,itemsep=-0.3em,partopsep=-0.3em]
%可使用leftmargin调整列表环境左边的空白长度 [leftmargin=0em]
\item
任选一条纸带读出 $ b $、$ c $ 两点间的距离为
\tk{$ 2.10 \ cm $ 或 $ 2.40 \ cm $($ \pm 0.05 \ cm $,有效数字位数正确)} 
;

\item 
任选一条纸带求出 $ c $、$ e $ 两点间的平均速度大小为 \tk{$ 1.13 \ m/s $ 或 $ 1.25 \ m/s $($ \pm 0.05 \ m/s $,有效数字位数不作要求)} ,纸带①和②上 $ c $、$ e $ 两点间的平
均速度 $ \bar{v}_{ \text{①} } $ 
\tk{小于} 
$ \bar{v}_{ \text{②} } $ (填“大于”、”等于”或“小于”);

\item 
图中 \tk{C} (填选项)
\fourchoices
{两条纸带均为用装置甲实验所得}
{两条纸带均为用装置乙实验所得}
{纸带①为用装置甲实验所得,纸带②为用装置乙实验所得}
{纸带①为用装置乙实验所得,纸带②为用装置甲实验所得}


\end{enumerate}






\newpage
\item
\exwhere{$ 2014 $ 年理综重庆卷}
为了研究人们用绳索跨越山谷过程中绳索拉力的变化规律,同学们设计了如题 $ 6 $ 图 $ 3 $ 所示的实验装
置。他们将不可伸长轻绳的两端通过测力计(不计质量及长度)固定在相距为 $ D $ 的两立柱上,固定
点分别为 $ P $ 和 $ Q $,$ P $ 低于 $ Q $,绳长为 $ L $($ L>PQ $)。他们首先在绳上距离 $ P $ 点 $ 10 \ cm $ 处(标记为 $ C $)
系上质量为 $ m $ 的重物(不滑动),由测力计读出 $ PC $、$ QC $ 的拉力大
小 $ T_P $、$ T_Q $。随后,改变重物悬挂点 $ C $ 的位置,每次将 $ P $ 到 $ C $ 的距
离增加 $ 10 \ cm $,并读出测力计的示数,最后得到 $ T_P $、$ T_Q $ 与绳长 $ PC $
的关系曲线如题 $ 6 $ 图 $ 4 $ 所示。由实验可知:
\begin{figure}[h!]
\centering
\includesvg[width=0.23\linewidth]{picture/svg/GZ-3-tiyou-0571} \qquad \includesvg[width=0.63\linewidth]{picture/svg/GZ-3-tiyou-0572}
\end{figure}



①曲线$ \lmd{2} $中拉力最大时,$ C $ 与 $ P $ 点的距离为
为
\tk{$ 60 $($ 56-64 $ 均可)} 
$ cm $,该曲线 \tk{$ T_{P} $} 
(选填:$ T_P $ 或 $ T_Q $)的曲线。



②在重物从 $ P $ 移到 $ Q $ 的整个过程中,受到最大拉力的是 \tk{Q} 
(选填:$ P $ 或 $ Q $)点所在的立柱。


③曲线$ \lmd{1} $、$ \lmd{2} $相交处,可读出绳的拉力为 $ T_0= $ \tk{$ 4.30 $ ($ 4.25-4.35 $ 均可)} $ N $,它与 $ L $、$ D $、$ m $ 和重力加速度 $ g $ 的关系
为 $ T_0= $ \tk{$\frac{m g}{2 \cos \theta}=\frac{m g L}{2 \sqrt{L^{2}-D^{2}}}$} 。 






\newpage
\item
\exwhere{$ 2015 $ 年江苏卷}
某同学探究小磁铁在铜管中下落时受电
磁阻尼作用的运动规律. 实验装置如图所示,打点计时器的电
源为 $ 50 \ Hz $ 的交流电.
\begin{figure}[h!]
\centering
\includesvg[width=0.23\linewidth]{picture/svg/GZ-3-tiyou-0573}
\end{figure}

\begin{enumerate}
\renewcommand{\labelenumi}{\arabic{enumi}.}
% A(\Alph) a(\alph) I(\Roman) i(\roman) 1(\arabic)
%设定全局标号series=example	%引用全局变量resume=example
%[topsep=-0.3em,parsep=-0.3em,itemsep=-0.3em,partopsep=-0.3em]
%可使用leftmargin调整列表环境左边的空白长度 [leftmargin=0em]
\item
下列实验操作中,不正确的有 \tk{CD} .
\fourchoices
{将铜管竖直地固定在限位孔的正下方}
{纸带穿过限位孔,压在复写纸下面}
{用手捏紧磁铁保持静止,然后轻轻地松开让磁铁下落}
{在磁铁下落的同时接通打点计时器的电源}



\item 
该同学按正确的步骤进行实验(记为“实验①”),将磁铁从管口处释放,打出一条纸带,取开始
下落的一段,确定一合适的点为 $ O $ 点,每隔一个计时点取一个计数点,标为 $ 1 $,$ 2 $,$ \cdots $,$ 8 $. 用刻
度尺量出各计数点的相邻两计时点到 $ O $ 点的距离,记录在纸带上,如图$ 2 $所示.
计算相邻计时点间的平均速度 $ \bar{v} $,粗略地表示各计数点的速度,抄入下表。 请将表中的数据补充
完整.
\begin{figure}[h!]
\centering
\includesvg[width=0.73\linewidth]{picture/svg/GZ-3-tiyou-0574}
\end{figure}

\begin{table}[h!]
\centering 
\begin{tabular}{|c|c|c|c|c|c|c|c|c|}
\hline 
位置 & 1 & 2 & 3 & 4 & 5 & 6 & 7 & 8
 \\
\hline
$ \bar{v} (cm/s) $ & 24.5 & 33.8 & 37.8 & \tk{39.0} & 39.5 & 39.8 & 39.8 & 39.8\\ 
\hline 
\end{tabular}
\end{table} 






\item 
分析上表的实验数据可知:在这段纸带记录的时间内,磁铁运动速度的变化情况是 \tk{逐渐增大到 $ 39.8 \ cm/s $};磁铁受到阻尼作用的变化情况是
\tk{逐渐增大到等于重力} 。




\item 
该同学将装置中的铜管更换为相同尺寸的塑料管,重复上述实验操作 ( 记为 “ 实验②”),结果
表明磁铁下落的运动规律与自由落体运动规律几乎相同。 请问实验②是为了说明什么? 对比实验
①和②的结果可得到什么结论?

\tk{为了说明磁铁在塑料管中几乎不受阻尼作用
磁铁在铜管中受到的阻尼作用主要是电磁阻尼作用.} 

\end{enumerate}



\newpage
\item
\exwhere{$ 2015 $ 年上海卷}
改进后的“研究有固定转动轴物体平衡条件”的实验装置如图所示,力
传感器、定滑轮固定在横杆上,替代原装置中的弹簧秤。已知力矩
盘上各同心圆的间距均为 $ 5 \ cm $。
\begin{figure}[h!]
\centering
\includesvg[width=0.23\linewidth]{picture/svg/GZ-3-tiyou-0575}
\end{figure}

\begin{enumerate}
\renewcommand{\labelenumi}{\arabic{enumi}.}
% A(\Alph) a(\alph) I(\Roman) i(\roman) 1(\arabic)
%设定全局标号series=example	%引用全局变量resume=example
%[topsep=-0.3em,parsep=-0.3em,itemsep=-0.3em,partopsep=-0.3em]
%可使用leftmargin调整列表环境左边的空白长度 [leftmargin=0em]
\item
(多选题)做这样改进的优点是 \tk{BD} 
\fourchoices
{力传感器既可测拉力又可测压力}
{力传感器测力时不受主观判断影响,精度较高}
{能消除转轴摩擦引起的实验误差}
{保证力传感器所受拉力方向不变}



\item 
某同学用该装置做实验,检验时发现盘停止转动时 $ G $ 点始终
在最低处,他仍用该盘做实验。在对力传感器进行调零后,用力传感器将力矩盘的 $ G $ 点拉到图示位
置,此时力传感器读数为 $ 3 \ N $。再对力传感器进行调零,然后悬挂钩码进行实验。此方法
\tk{能} 
(选填“能”、“不能”)消除力矩盘偏心引起的实验误差。已知每个钩码所受重力为 $ 1 \ N $,力矩盘按图
示方式悬挂钩码后,力矩盘所受顺时针方向的合力矩为 \tk{$ 0.7 $} $ N \cdot m $。力传感器的读数为 \tk{$ -0.5 $} $ N $。

\end{enumerate}







\newpage
\item
\exwhere{$ 2017 $ 年新课标$ \lmd{2} $卷}
某同学研究在固定斜面上运动物体的平均速度、瞬时速度和加速
度的之间的关系。使用的器材有:斜面、滑块、长度不同的挡光片、光电计时器。
\begin{figure}[h!]
\centering
\includesvg[width=0.53\linewidth]{picture/svg/GZ-3-tiyou-0576}
\end{figure}


实验步骤如下:

①如图($ a $),将光电门固定在斜面下端附近:
将一挡光片安装在滑块上,记下挡光片前端相
对于斜面的位置,令滑块从斜面上方由静止开
始下滑;


②当滑块上的挡光片经过光电门时,用光电计时器测得光线被挡光片遮住的时间$ \Delta t $;



③用$ \Delta s $ 表示挡光片沿运动方向的长度(如图($ b $)所示), $ \bar{v} $ 表示滑块在挡光片遮住光线的$ \Delta t $ 时间
内的平均速度大小,求出 $ \bar{v} $;

④将另一挡光片换到滑块上,使滑块上的挡光片前端与①中的位置相同,令滑块由静止开始下滑,
重复步骤②、③;


⑤多次重复步骤④

⑥利用实验中得到的数据作出 $ v- \Delta t $ 图,如图($ c $)所示
\begin{figure}[h!]
\centering
\includesvg[width=0.23\linewidth]{picture/svg/GZ-3-tiyou-0577}
\end{figure}

完成下列填空:
\begin{enumerate}
\renewcommand{\labelenumi}{\arabic{enumi}.}
% A(\Alph) a(\alph) I(\Roman) i(\roman) 1(\arabic)
%设定全局标号series=example	%引用全局变量resume=example
%[topsep=-0.3em,parsep=-0.3em,itemsep=-0.3em,partopsep=-0.3em]
%可使用leftmargin调整列表环境左边的空白长度 [leftmargin=0em]
\item
用 $ a $ 表示滑块下滑的加速度大小,用 $ v_{A} $ 表示挡光片前端到
达光电门时滑块的瞬时速度大小,则 $ \bar{v} $ 与 $ v_{A} $、$ a $ 和$ \Delta t $ 的关系式为
$ \bar{v} $ \tk{$v_{A}+\frac{1}{2} a \Delta t$} 
。

\item 
由图($ c $)可求得 $ v_{A} = $
\tk{$ 52.1 $} 
$ cm/s $,$ a= $
\tk{$ 16.4(15.8 \sim 16.8) $} 
$ cm/s^{2} $.(结
果保留 $ 3 $ 位有效数字)

\end{enumerate}







\end{enumerate}

