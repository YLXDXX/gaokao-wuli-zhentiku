\bta{带电粒子在复合场中的运动}

\begin{enumerate}
\item
\exwhere{$ 2017 $ 年新课标  \lmd{1}  卷}
如图,空间某区域存在匀强电场和匀强磁场,电场方向竖直向上(与纸
面平行),磁场方向垂直于纸面向量,三个带正电的微粒 $ a $,$ b $,$ c $ 电荷量相等,质量分别为 $ m_{a} $,
$ m_{b} $,$ m_{c} $,已知在该区域内,$ a $ 在纸面内做匀速圆周运动,$ b $ 在纸面内向右做匀速直线运动,$ c $ 在纸
面内向左做匀速直线运动。下列选项正确的是 \xzanswer{B} 
\begin{figure}[h!]
	\centering
	\includesvg[width=0.23\linewidth]{picture/svg/GZ-3-tiyou-1244}
\end{figure}

\fourchoices
{$ m_{a} > m_{b} > m_{c} $}
{$ m_{b} > m_{a} > m_{c} $}
{$ m_{c} > m_{b} > m_{a} $}
{$ m_{a} > m_{c} > m_{b} $}




\item 
\exwhere{$ 2012 $ 年物理海南卷}
如图,在两水平极板间存在匀强电场和匀强磁场,电场方向竖直向下,磁场方向垂直于纸面向
里。一带电粒子以某一速度沿水平直线通过两极板。若不计重力,下列四个物理量中哪一个改变
时,粒子运动轨迹不会改变 \xzanswer{B} 
\begin{figure}[h!]
	\centering
	\includesvg[width=0.23\linewidth]{picture/svg/GZ-3-tiyou-1245}
\end{figure}

\fourchoices
{粒子速度的大小}
{粒子所带的电荷量}
{电场强度}
{磁感应强度}


\item 
\exwhere{$ 2013 $ 年浙江卷}
在半导体离子注入工艺中,初速度可忽略的离子  \ce{P^{+}}  和  \ce{P^{3+}} ,经电压为 $ U $ 的电场加速后,垂直进
入磁感应强度大小为 $ B $、方向垂直纸面向里,有一定的宽度的匀强
磁场区域,如图所示。已知离子 $ P+ $在磁场中转过$ \theta =30 \degree $后从磁场右
边界射出。在电场和磁场中运动时,离子  \ce{P^{+}}  和  \ce{P^{3+}}  \xzanswer{BCD} 
\begin{figure}[h!]
	\centering
	\includesvg[width=0.23\linewidth]{picture/svg/GZ-3-tiyou-1246}
\end{figure}

\fourchoices
{在电场中的加速度之比为 $ 1:1 $}
{在磁场中运动的半径之比为 $ \sqrt{3}:1 $}
{在磁场中转过的角度之比为 $ 1:2 $}
{离开电场区域时的动能之比为 $ 1:3 $}

\item 
\exwhere{$ 2018 $ 年北京卷}
某空间存在匀强磁场和匀强电场。一个带电粒子(不计重力)以一定初速
度射入该空间后,做匀速直线运动;若仅撤除电场,则该粒子做匀速圆周运动。下列因素与完成
上述两类运动无关的是  \xzanswer{C} 

\fourchoices
{磁场和电场的方向}
{磁场和电场的强弱}
{粒子的电性和电量}
{粒子入射时的速度}


\item 
\exwhere{$ 2016 $ 年天津卷}
如图所示,空间中存在着水平向右的匀强电场,电场强度大小为
$ E=5\sqrt{3} \ N/C $,同时存在着水平方向的匀强磁场,其方向与电场方向垂直,磁感应强度大小
$ B=0.5 \ T $。有一带正电的小球,质量 $ m=1.0 \times 10^{-6} \ kg $,电荷量 $ q=2 \times 10^{-6} \ C $,正以速度 $ v $ 在图示的
竖直面内做匀速直线运动,当经过 $ P $ 点时撤掉磁场(不考虑
磁场消失引起的电磁感应现象)取 $ g=10 \ m/s^{2} $,求:
\begin{enumerate}
	%\renewcommand{\labelenumi}{\arabic{enumi}.}
	% A(\Alph) a(\alph) I(\Roman) i(\roman) 1(\arabic)
	%设定全局标号series=example	%引用全局变量resume=example
	%[topsep=-0.3em,parsep=-0.3em,itemsep=-0.3em,partopsep=-0.3em]
	%可使用leftmargin调整列表环境左边的空白长度 [leftmargin=0em]
	\item
小球做匀速直线运动的速度 $ v $ 的大小和方向;


\item 
从撤掉磁场到小球再次穿过 $ P $ 点所在的这条电场线经
历的时间 $ t $。

	
\end{enumerate}
\begin{figure}[h!]
	\flushright
	\includesvg[width=0.25\linewidth]{picture/svg/GZ-3-tiyou-1247}
\end{figure}


\banswer{
	\begin{enumerate}
		%\renewcommand{\labelenumi}{\arabic{enumi}.}
		% A(\Alph) a(\alph) I(\Roman) i(\roman) 1(\arabic)
		%设定全局标号series=example	%引用全局变量resume=example
		%[topsep=-0.3em,parsep=-0.3em,itemsep=-0.3em,partopsep=-0.3em]
		%可使用leftmargin调整列表环境左边的空白长度 [leftmargin=0em]
		\item
		$ 20 \ m /s $;速度 $ v $ 的方向与电场 $ E $ 的方向之间的夹角 $ 60 \degree $
		\item 
		$ 3.5 \ s $	
	\end{enumerate}
}


\item 
\exwhere{$ 2012 $ 年理综浙江卷}
如图所示,两块水平放置、相距为 $ d $ 的长金属板接在电压可调的电源上。两板之间的
右侧区域存在方向垂直纸面向里的匀强磁场。将喷墨打印机的喷口靠近上板下表面,从喷口连续
不断喷出质量均为 $ m $、水平速度均为 $ v_{0} $,带相等电
荷量的墨滴。调节电源电压至 $ U $,墨滴在电场区域
恰能沿水平向右做匀速直线运动;进入电场、磁场
共存区域后,最终垂直打在下板的 $ M $ 点。
\begin{enumerate}
	%\renewcommand{\labelenumi}{\arabic{enumi}.}
	% A(\Alph) a(\alph) I(\Roman) i(\roman) 1(\arabic)
	%设定全局标号series=example	%引用全局变量resume=example
	%[topsep=-0.3em,parsep=-0.3em,itemsep=-0.3em,partopsep=-0.3em]
	%可使用leftmargin调整列表环境左边的空白长度 [leftmargin=0em]
	\item
判断墨滴所带电荷的种类,并求其电荷量;



\item 
求磁感应强度 $ B $ 的值;

\item 
现保持喷口方向不变,使其竖直下移到两板中间的位置。为了使墨滴仍能到达下板 $ M $ 点,应将
磁感应强度调至 $ B ^{\prime} $,则 $ B ^{\prime} $的大小为多少?

\end{enumerate}
\begin{figure}[h!]
	\flushright
	\includesvg[width=0.25\linewidth]{picture/svg/GZ-3-tiyou-1248}
\end{figure}


\banswer{
	\begin{enumerate}
		%\renewcommand{\labelenumi}{\arabic{enumi}.}
		% A(\Alph) a(\alph) I(\Roman) i(\roman) 1(\arabic)
		%设定全局标号series=example	%引用全局变量resume=example
		%[topsep=-0.3em,parsep=-0.3em,itemsep=-0.3em,partopsep=-0.3em]
		%可使用leftmargin调整列表环境左边的空白长度 [leftmargin=0em]
		\item
		$q=\frac{m g d}{U}$ \quad 墨滴带负电荷
		\item 
		$B=\frac{v_{0} U}{g d^{2}}$
		\item 
		$B^{\prime}=\frac{4 v_{0} U}{5 g d^{2}}$
	\end{enumerate}
}


\item 
\exwhere{$ 2015 $ 年理综福建卷}
如图,绝缘粗糙的竖直平面 $ MN $ 左侧同时存在相互垂直的匀强电
场和匀强磁场,电场方向水平向右,电场强度大小为 $ E $,磁场方向垂直纸面向外,磁感应强度大小
为 $ B $。一质量为 $ m $、电荷量为 $ q $ 的带正电的小滑块从 $ A $ 点由静止开始沿 $ MN $ 下滑,到达 $ C $ 点时离开
$ MN $ 做曲线运动。$ A $、$ C $ 两点间距离为 $ h $,重力加速度为 $ g $。
\begin{enumerate}
	%\renewcommand{\labelenumi}{\arabic{enumi}.}
	% A(\Alph) a(\alph) I(\Roman) i(\roman) 1(\arabic)
	%设定全局标号series=example	%引用全局变量resume=example
	%[topsep=-0.3em,parsep=-0.3em,itemsep=-0.3em,partopsep=-0.3em]
	%可使用leftmargin调整列表环境左边的空白长度 [leftmargin=0em]
	\item
求小滑块运动到 $ C $ 点时的速度大小 $ v_{C} $;



\item 
求小滑块从 $ A $ 点运动到 $ C $ 点过程中克服摩擦力做的功 $ W_f $;

\item 
若 $ D $ 点为小滑块在电场力、洛伦兹力及重力作用下运动过程中
速度最大的位置,当小滑块运动到 $ D $ 点时撤去磁场,此后小滑块继
续运动到水平地面上的 $ P $ 点。已知小滑块在 $ D $ 点时的速度大小为 $ v_{D} $,从 $ D $ 点运动到 $ P $ 点的时间为
$ t $,求小滑块运动到 $ P $ 点时速度的大小 $ v_{P} $。

	
\end{enumerate}
\begin{figure}[h!]
	\centering
	\includesvg[width=0.23\linewidth]{picture/svg/GZ-3-tiyou-1249}
\end{figure}



\banswer{
	\begin{enumerate}
		%\renewcommand{\labelenumi}{\arabic{enumi}.}
		% A(\Alph) a(\alph) I(\Roman) i(\roman) 1(\arabic)
		%设定全局标号series=example	%引用全局变量resume=example
		%[topsep=-0.3em,parsep=-0.3em,itemsep=-0.3em,partopsep=-0.3em]
		%可使用leftmargin调整列表环境左边的空白长度 [leftmargin=0em]
		\item
		$ E/B $
		\item 
		$W_{f}=m g h-\frac{1}{2} m \frac{E^{2}}{B^{2}}$
		\item 
		$v_{p}=\sqrt{\frac{(m g)^{2}+(q E)^{2}}{m^{2}} t^{2}+v_{D}^{2}}$
	\end{enumerate}
}


\item 
\exwhere{$ 2014 $ 年理综四川卷}
在如图所示的竖直平面内。水平轨道 $ CD $ 和倾斜轨道 $ GH $ 与半径 $ r= \frac{9}{44} \ m $ 的光滑圆弧轨道分别相
切于 $ D $ 点和 $ G $ 点,$ GH $ 与水平面的夹角$ \theta =37 \degree  $。过 $ G $ 点、垂直于纸面的竖直平面左侧有匀强磁
场,磁场方向垂直于纸面向里,磁感应强度 $ B=1.25 \ T $;过 $ D $ 点、垂直于纸面的竖直平面右侧有匀
强电场,电场方向水平向右,电场强度 $ E=1 \times 10^{4} \ N /C $。小物体 $ P_{1} $ 质量 $ m=2 \times 10^{-3} \ kg $、电荷量
$ q=+8 \times 10^{-6} \ C $,受到水平向右的推力 $ F=9.98 \times 10^{-3} \ N $ 的作用,沿 $ CD $ 向右做匀速直线运动,到达 $ D $
点后撤去推力。当 $ P_{1} $ 到达倾斜轨道底端 $ G $ 点时,不带电的小物体 $ P_{2} $ 在 $ GH $ 顶端静止释放,经过
时间 $ t=0.1 \ s $ 与 $ P_{1} $ 相遇。$ P_{1} $ 和 $ P_{2} $ 与轨道
$ CD $、$ GH $ 间的动摩擦因数均为 $ \mu=0.5 $,取
$ g=10 \ m/s^{2} , \sin 37 \degree =0.6 $,$ \cos 37 \degree =0.8 $,物体
电荷量保持不变,不计空气阻力。求:
\begin{enumerate}
	%\renewcommand{\labelenumi}{\arabic{enumi}.}
	% A(\Alph) a(\alph) I(\Roman) i(\roman) 1(\arabic)
	%设定全局标号series=example	%引用全局变量resume=example
	%[topsep=-0.3em,parsep=-0.3em,itemsep=-0.3em,partopsep=-0.3em]
	%可使用leftmargin调整列表环境左边的空白长度 [leftmargin=0em]
	\item
小物体 $ P_{1} $ 在水平轨道 $ CD $ 上运动速
度 $ v $ 的大小;

\item 
倾斜轨道 $ GH $ 的长度 $ s $。


\end{enumerate}
\begin{figure}[h!]
	\flushright
	\includesvg[width=0.25\linewidth]{picture/svg/GZ-3-tiyou-1250}
\end{figure}


\banswer{
	\begin{enumerate}
		%\renewcommand{\labelenumi}{\arabic{enumi}.}
		% A(\Alph) a(\alph) I(\Roman) i(\roman) 1(\arabic)
		%设定全局标号series=example	%引用全局变量resume=example
		%[topsep=-0.3em,parsep=-0.3em,itemsep=-0.3em,partopsep=-0.3em]
		%可使用leftmargin调整列表环境左边的空白长度 [leftmargin=0em]
		\item
		$v=4 \ m/s$
		\item 
		$ s=0.56 \ m $	
	\end{enumerate}
}



\item 
\exwhere{$ 2012 $ 年理综新课标卷}
如图,一半径为 $ R $ 的圆表示一柱形区域的横截面(纸面)。在柱形区域内加一方向垂
直于纸面的匀强磁场,一质量为 $ m $、电荷量为 $ q $ 的粒子沿图中直线在圆
上的 $ a $ 点射入柱形区域,在圆上的 $ b $ 点离开该区域,离开时速度方向与
直线垂直。圆心 $ O $ 到直线的距离为
$  \frac{ 3 }{ 5 } R $。现将磁场换为平行于纸面且垂
直于直线的匀强电场,同一粒子以同样速度沿直线在 $ a $ 点射入柱形区
域,也在 $ b $ 点离开该区域。若磁感应强度大小为 $ B $,不计重力,求电场强度的大小。
\begin{figure}[h!]
	\flushright
	\includesvg[width=0.25\linewidth]{picture/svg/GZ-3-tiyou-1251}
\end{figure}


\banswer{
	$E=\frac{14 q R B^{2}}{5 m}$
}


\item 
\exwhere{$ 2012 $ 年物理江苏卷}
如图所示,待测区域中存在匀强电场和匀强磁场,根据带电粒子射入时的受力情况可推测其
电场和磁场. 图中装置由加速器和平移器组成,平移器由两对水平放置、相距为 $ l $ 的相同平行金属
板构成,极板长度为 $ l $、间距为 $ d $,两对极板间偏转电压大小相等、电场方向相反. 质量为 $ m $、电荷量
为$ +q $ 的粒子经加速电压 $ U_{0} $ 加速后,水平射入偏转电压为 $ U_{1} $ 的平移器,最终从 $ A $ 点水平射入待测区域.
不考虑粒子受到的重力.
\begin{enumerate}
	%\renewcommand{\labelenumi}{\arabic{enumi}.}
	% A(\Alph) a(\alph) I(\Roman) i(\roman) 1(\arabic)
	%设定全局标号series=example	%引用全局变量resume=example
	%[topsep=-0.3em,parsep=-0.3em,itemsep=-0.3em,partopsep=-0.3em]
	%可使用leftmargin调整列表环境左边的空白长度 [leftmargin=0em]
	\item
求粒子射出平移器时的速度
大小 $ v_{1} $;

\item 
当加速电压变为 $ 4U_{0} $ 时,欲使
粒子仍从 $ A $ 点射入待测区域,求
此时的偏转电压 $ U $;


\item 
已知粒子以不同速度水平向右射入待测区域,刚进入时的受力大小均为 $ F $. 现取水平向右为 $ x $ 轴正
方向,建立如图所示的直角坐标系 $ Oxyz $. 保持加速电压为 $ U_{0} $ 不变,移动装置使粒子沿不同的坐标轴方
向射入待测区域,粒子刚射入时的受力大小如下表所示.
\begin{table}[h!]
 \centering 
 \begin{tabular}{|c|c|c|c|c|}
 	\hline 射入方向 & $y$ & $-y$ & $z$ & $-z$ \\
 	\hline 受力大小 & $\sqrt{5} F$ & $\sqrt{5} F$ & $\sqrt{7} F$ & $\sqrt{3} F$ \\
 	\hline
 \end{tabular}
 \end{table} 

请推测该区域中电场强度和磁感应强度的大小及可能的方向.

\end{enumerate}
\begin{figure}[h!]
	\flushright
	\includesvg[width=0.25\linewidth]{picture/svg/GZ-3-tiyou-1252}
\end{figure}


\banswer{
	\begin{enumerate}
		%\renewcommand{\labelenumi}{\arabic{enumi}.}
		% A(\Alph) a(\alph) I(\Roman) i(\roman) 1(\arabic)
		%设定全局标号series=example	%引用全局变量resume=example
		%[topsep=-0.3em,parsep=-0.3em,itemsep=-0.3em,partopsep=-0.3em]
		%可使用leftmargin调整列表环境左边的空白长度 [leftmargin=0em]
		\item
		$v_{1}=\sqrt{\frac{2 q U_{0}}{m}}$
		\item 
		$U=4 U_{1}$
		\item 
		若 $ B $ 沿 $ x $ 轴方向, $ E $ 与 $ Oxy $ 平面平行且与 $ x $ 轴方向的夹角为 $ 30 \degree  $ 或 $ 150 \degree  $\\
		同理若 $ B $ 沿$ -x $ 轴方向,$ E $ 与 $ Oxy $ 平面平行且与 $ x $ 轴方向的夹角为$ -30 \degree  $ 或$ -150 \degree  $。
		
	\end{enumerate}	
}

\item 
\exwhere{$ 2012 $ 年理综四川卷}
如图所示,水平虚线 $ X $ 下方区域分布着方向水平、垂直纸面向里、
磁感应强度为 $ B $ 的匀强磁场,整个空间存在匀强电场(图中未画
出)。质量为 $ m $,电荷量为$ +q $ 的小球 $ P $ 静止于虚线 $ X $ 上方 $ A $ 点,在
某一瞬间受到方向竖直向下、大小为 $ I $ 的冲量作用而做匀速直线运
动。在 $ A $ 点右下方的磁场中有定点 $ O $,长为 $ l $ 的绝缘轻绳一端固定
于 $ O $ 点,另一端连接不带电的质量同为 $ m $ 的小球 $ Q $,自然下垂。保
持轻绳伸直,向右拉起 $ Q $,直到绳与竖直方向有一小于 $ 5 \degree  $ 的夹角,
在 $ P $ 开始运动的同时自由释放 $ Q $,$ Q $ 到达 $ O $ 点正下方 $ W $ 点时速率为
$ v_{0} $。$ P $、$ Q $ 两小球在 $ W $ 点发生正碰,碰后电场、磁场消失,两小球粘
在一起运动。$ P $、$ Q $ 两小球均视为质点,$ P $ 小球的电荷量保持不变,绳不可伸长,不计空气阻力,
重力加速度为 $ g $。
\begin{enumerate}
	%\renewcommand{\labelenumi}{\arabic{enumi}.}
	% A(\Alph) a(\alph) I(\Roman) i(\roman) 1(\arabic)
	%设定全局标号series=example	%引用全局变量resume=example
	%[topsep=-0.3em,parsep=-0.3em,itemsep=-0.3em,partopsep=-0.3em]
	%可使用leftmargin调整列表环境左边的空白长度 [leftmargin=0em]
	\item
求匀强电场场强 $ E $ 的大小和 $ P $ 进入磁场时的速率 $ v $;

\item 
若绳能承受的最大拉力为 $ F $,要使绳不断,$ F $ 至少为多大?

\item 
求 $ A $ 点距虚线 $ X $ 的距离 $ s $。

\end{enumerate}
\begin{figure}[h!]
	\flushright
	\includesvg[width=0.25\linewidth]{picture/svg/GZ-3-tiyou-1253}
\end{figure}


\banswer{
	\begin{enumerate}
		%\renewcommand{\labelenumi}{\arabic{enumi}.}
		% A(\Alph) a(\alph) I(\Roman) i(\roman) 1(\arabic)
		%设定全局标号series=example	%引用全局变量resume=example
		%[topsep=-0.3em,parsep=-0.3em,itemsep=-0.3em,partopsep=-0.3em]
		%可使用leftmargin调整列表环境左边的空白长度 [leftmargin=0em]
		\item
		$ v=I/m $
		\item 
		$F=\frac{\left(I+m v_{0}\right)^{2}}{2 m l}+2 m g$
		\item 
		设 $P$ 在 $X$ 上方做匀速直线运动的时间为 $t_{P 1},$ 则 
		\begin{equation}\label{key}
		t_{P 1}=\frac{S}{v}
		\end{equation}
		设 $P$ 在 $X$ 下方做匀速圆周运动的时间为 $t_{P 2},$ 则
		\begin{equation}\label{key}
		t_{P 2}=\frac{\pi m}{2 B q}
		\end{equation}
		设小球 $Q$ 从开始运动到与 $P$ 球反向相碰的运动时间为 $t_{Q},$ 由单摆周期性,有
		\begin{equation}\label{key}
		t_{Q}=\left(n+\frac{1}{4}\right) 2 \pi \sqrt{\frac{l}{g}}
		\end{equation}
		由题意,有 $ t_{Q}=t_{P 1}+t_{P 2}$,联立相关方程,得
		\begin{equation}\label{key}
			s=\left(n+\frac{1}{4}\right) \frac{2 \pi l}{m} \sqrt{\frac{l}{g}}-\frac{\pi I}{2 B q} 
		\end{equation}
		$ n \text { 为大于 }\left(\frac{m}{4 B q} \sqrt{\frac{g}{l}}-\frac{1}{4}\right) \text { 的整数 } $。\\
		设小球 $Q$ 从开始运动到与 $P$ 球同向相碰的运动时间为 $t_{Q}^{\prime},$ 由单摆周期性,有
		\begin{equation}\label{key}
		t_{Q}^{\prime}=\left(n+\frac{3}{4}\right) 2 \pi \sqrt{\frac{l}{g}}
		\end{equation}
		同理可得
		\begin{equation}\label{key}
		s=\left(n+\frac{3}{4}\right) \frac{2 \pi I}{m} \sqrt{\frac{l}{g}}-\frac{\pi I}{2 B q} 
		\end{equation}
		$ n \text { 为大于 }\left(\frac{m}{4 B q} \sqrt{\frac{g}{l}}-\frac{3}{4}\right) \text { 的整数 } $。
	\end{enumerate}
}


\item 
\exwhere{$ 2012 $ 年理综重庆卷}
有人设计了一种带电颗粒的速率分选装置,其原理如图所示。两带电金属板间
有匀强电场,方向竖直向上,其中 $ PQNM $ 矩形区域内还有方向垂直纸面向外的匀强磁场。一束比
荷(电荷量与质量之比)均为 $ 1/k $ 的带正电颗
粒,以不同的速率沿着磁场区域的中心线 $ O ^{\prime} O $ 进
入两金属板之间,其中速率为 $ v_{0} $ 的颗粒刚好从 $ Q $
点处离开磁场,然后做匀速直线运动到达收集
板。重力加速度为 $ g $,$ PQ=3d $,$ NQ=2d $,收集板
与 $ NQ $ 的距离为 $ l $,不计颗粒间相互作用,求:
\begin{enumerate}
	%\renewcommand{\labelenumi}{\arabic{enumi}.}
	% A(\Alph) a(\alph) I(\Roman) i(\roman) 1(\arabic)
	%设定全局标号series=example	%引用全局变量resume=example
	%[topsep=-0.3em,parsep=-0.3em,itemsep=-0.3em,partopsep=-0.3em]
	%可使用leftmargin调整列表环境左边的空白长度 [leftmargin=0em]
	\item
电场强度 $ E $ 的大小;

\item 
磁感应强度 $ B $ 的大小;

\item 
速率为$ \lambda v_{0} $($ \lambda >1 $)的颗粒打在收集板上的位置到 $ O $ 点的距离。

	
\end{enumerate}
\begin{figure}[h!]
	\flushright
	\includesvg[width=0.25\linewidth]{picture/svg/GZ-3-tiyou-1254}
\end{figure}



\banswer{
	\begin{enumerate}
		%\renewcommand{\labelenumi}{\arabic{enumi}.}
		% A(\Alph) a(\alph) I(\Roman) i(\roman) 1(\arabic)
		%设定全局标号series=example	%引用全局变量resume=example
		%[topsep=-0.3em,parsep=-0.3em,itemsep=-0.3em,partopsep=-0.3em]
		%可使用leftmargin调整列表环境左边的空白长度 [leftmargin=0em]
		\item
		$E=k g$
		\item 
		如图,
		$B=k v_{0} / 5 d$
		\begin{center}
 \includesvg[width=0.63\linewidth]{picture/svg/GZ-3-tiyou-1255} 
		\end{center}
		\item 
		如图,
		$y=y_{1}+y_{2}=d\left(5 \lambda-\sqrt{25 \lambda^{2}-9}\right)+3 l / \sqrt{25 \lambda^{2}-9}$
		\begin{center}
 \includesvg[width=0.63\linewidth]{picture/svg/GZ-3-tiyou-1256} 
		\end{center}
	\end{enumerate}
}


\item 
\exwhere{$ 2011 $ 年理综安徽卷}
如图所示,在以坐标原点 $ O $ 为圆心、半径为 $ R $ 的半圆形区域内,有相互垂直的匀强
电场和匀强磁场,磁感应强度为 $ B $,磁场方向垂直于 $ xOy $ 平面向里。一带正电的粒子(不计重力)
从 $ O $ 点沿 $ y $ 轴正方向以某一速度射入,带电粒子恰好做匀
速直线运动,经 $ t_{0} $ 时间从 $ P $ 点射出。
\begin{enumerate}
	%\renewcommand{\labelenumi}{\arabic{enumi}.}
	% A(\Alph) a(\alph) I(\Roman) i(\roman) 1(\arabic)
	%设定全局标号series=example	%引用全局变量resume=example
	%[topsep=-0.3em,parsep=-0.3em,itemsep=-0.3em,partopsep=-0.3em]
	%可使用leftmargin调整列表环境左边的空白长度 [leftmargin=0em]
	\item
求电场强度的大小和方向。

\item 
若仅撤去磁场,带电粒子仍从 $ O $ 点以相同的速度射
入,经 $ t_{0} /2 $ 时间恰从半圆形区域的边界射出。求粒子运动
加速度的大小。



\item 
若仅撤去电场,带电粒子仍从 $ O $ 点射入,且速度为原来的 $ 4 $ 倍,求粒子在磁场中运动的时
间。

\end{enumerate}
\begin{figure}[h!]
	\flushright
	\includesvg[width=0.25\linewidth]{picture/svg/GZ-3-tiyou-1257}
\end{figure}

\banswer{
	\begin{enumerate}
		%\renewcommand{\labelenumi}{\arabic{enumi}.}
		% A(\Alph) a(\alph) I(\Roman) i(\roman) 1(\arabic)
		%设定全局标号series=example	%引用全局变量resume=example
		%[topsep=-0.3em,parsep=-0.3em,itemsep=-0.3em,partopsep=-0.3em]
		%可使用leftmargin调整列表环境左边的空白长度 [leftmargin=0em]
		\item
		电场强度沿 $ x $ 轴正方向,$E=\frac{B R}{t_{0}}$
		\item 
		$a=\frac{4 \sqrt{3}}{t_{0}^{2}} R$
		\item 
		$t_{R}=\frac{\sqrt{3} \pi}{18} t_{0}$
	\end{enumerate}
}



\item 
\exwhere{$ 2013 $ 年安徽卷}
如图所示的平面直角坐标系 $ xOy $,在第 \lmd{1} 象限内有平行于 $ y $ 轴的匀强电场,方向沿 $ y $ 正方向;在第
 \lmd{4} 象限的正三角形 $ abc $ 区域内有匀强电场,方向垂直于 $ xOy $ 平面向里,正三角形边长为 $ L $,且 $ ab $
边与 $ y $ 轴平行。一质量为 $ m $、电荷量为 $ q $ 的粒子,从 $ y $ 轴上的 $ P $($ 0 $,$ h $)点,以大小为 $ v_{0} $ 的速度沿
$ x $ 轴正方向射入电场,通过电场后从 $ x $ 轴上的 $ a ( 2h , 0) $点进入第Ⅳ象限,又经过磁场从 $ y $ 轴上的
某点进入第 \lmd{3} 象限,且速度与 $ y $ 轴负方向成 $ 45 \degree $角,不计粒子所受的重力。求:
\begin{enumerate}
	%\renewcommand{\labelenumi}{\arabic{enumi}.}
	% A(\Alph) a(\alph) I(\Roman) i(\roman) 1(\arabic)
	%设定全局标号series=example	%引用全局变量resume=example
	%[topsep=-0.3em,parsep=-0.3em,itemsep=-0.3em,partopsep=-0.3em]
	%可使用leftmargin调整列表环境左边的空白长度 [leftmargin=0em]
	\item
电场强度 $ E $ 的大小;



\item 
粒子到达 $ a $ 点时速度的大小和方向;


\item 
$ abc $ 区域内磁场的磁感应强度 $ B $ 的最小值。

	
\end{enumerate}
\begin{figure}[h!]
	\flushright
	\includesvg[width=0.25\linewidth]{picture/svg/GZ-3-tiyou-1258}
\end{figure}


\banswer{
	\begin{enumerate}
		%\renewcommand{\labelenumi}{\arabic{enumi}.}
		% A(\Alph) a(\alph) I(\Roman) i(\roman) 1(\arabic)
		%设定全局标号series=example	%引用全局变量resume=example
		%[topsep=-0.3em,parsep=-0.3em,itemsep=-0.3em,partopsep=-0.3em]
		%可使用leftmargin调整列表环境左边的空白长度 [leftmargin=0em]
		\item
		$E=\frac{m v_{0}^{2}}{2 q h}$
		\item 
		$v=\sqrt{v_{0}^{2}+v_{y}^{2}}=\sqrt{2} v_{0}$
		\item 
		$B=\frac{2 m v_{0}}{q L}$
	\end{enumerate}
}


\item 
在科学研究中,可以通过施加适当的电场和磁场来实现对带电粒子运动的控制. 如题$ 15-1 $
图所示的$ xOy $ 平面处于匀强电场和匀强磁场中,电场强度$ E $ 和磁感应强度$ B $ 随时间$ t $ 作周期性变化的
图象如题$ 15-2 $ 图所示.$ x $轴正方向为$ E $的正方向,垂直纸面向里为$ B $的正方向. 在坐标原点$ O $有一粒子$ P $,
其质量和电荷量分别为$ m $ 和$ +q $. 不计重力. 在 $ t=\frac{\tau}{2} $时刻释放$ P $,它恰能沿一定轨道做往复运动.
\begin{enumerate}
	%\renewcommand{\labelenumi}{\arabic{enumi}.}
	% A(\Alph) a(\alph) I(\Roman) i(\roman) 1(\arabic)
	%设定全局标号series=example	%引用全局变量resume=example
	%[topsep=-0.3em,parsep=-0.3em,itemsep=-0.3em,partopsep=-0.3em]
	%可使用leftmargin调整列表环境左边的空白长度 [leftmargin=0em]
	\item
求$ P $在磁场中运动时速度的大小$ v_{0} $;



\item 
求$ B_{0} $应满足的关系;



\item 
在 $ t_{0} (0< t_{0} <\frac{\tau}{2}) $ 时刻释放$ P $,求$ P $速度
为零时的坐标.

	
\end{enumerate}
\begin{figure}[h!]
	\centering
\begin{subfigure}{0.4\linewidth}
	\centering
	\includesvg[width=0.7\linewidth]{picture/svg/GZ-3-tiyou-1259} 
	\caption{}\label{}
\end{subfigure}
\begin{subfigure}{0.4\linewidth}
	\centering
	\includesvg[width=0.7\linewidth]{picture/svg/GZ-3-tiyou-1260} 
	\caption{}\label{}
\end{subfigure}
\begin{subfigure}{0.4\linewidth}
	\centering
	\includesvg[width=0.7\linewidth]{picture/svg/GZ-3-tiyou-1261} 
	\caption{}\label{}
\end{subfigure}
\end{figure}



\banswer{
	\begin{enumerate}
		%\renewcommand{\labelenumi}{\arabic{enumi}.}
		% A(\Alph) a(\alph) I(\Roman) i(\roman) 1(\arabic)
		%设定全局标号series=example	%引用全局变量resume=example
		%[topsep=-0.3em,parsep=-0.3em,itemsep=-0.3em,partopsep=-0.3em]
		%可使用leftmargin调整列表环境左边的空白长度 [leftmargin=0em]
		\item
		$v_{0}=\frac{q E_{0} \tau}{2 m}$
		\item 
		$B_{0}=\frac{2(n-1) \pi n}{q \tau},(n=1,2,3 \ldots \ldots)$
		\item 
		$ x=0 $\\\\
		$ 
		y=
		\left\{
		\begin{aligned}
		&\frac{2 E_{0}\left[k\left(\tau-2 t_{0}\right)+t_{0}\right]}{B_{0}}\\
		\\
		&\frac{2 k E_{0}\left(\tau-2 t_{0}\right)}{B_{0}}
		\end{aligned}
		\right.
		 $\\\\
		 $ 	(k=1,2,3 \ldots \ldots) $
	\end{enumerate}
}


\item 
\exwhere{$ 2013 $ 年山东卷}
如图所示,在坐标系 $ xOy $ 的第一、第三象限内存在相同的磁场,磁场方向垂直于 $ xOy $
平面向里;第四象限内有沿 $ y $ 轴正方向的匀强电场,电场强度大小为 $ E $。一带电量为$ +q $、质量为 $ m $
的粒子,自 $ y $ 轴上的 $ P $ 点沿 $ x $ 轴正方向射入第四象限,经 $ x $ 轴上
的 $ Q $ 点进入第一象限,随即撤去电场,以后仅保留磁场。已知
$ OP=d $,$ OQ=2d $。不计粒子重力。
\begin{enumerate}
	%\renewcommand{\labelenumi}{\arabic{enumi}.}
	% A(\Alph) a(\alph) I(\Roman) i(\roman) 1(\arabic)
	%设定全局标号series=example	%引用全局变量resume=example
	%[topsep=-0.3em,parsep=-0.3em,itemsep=-0.3em,partopsep=-0.3em]
	%可使用leftmargin调整列表环境左边的空白长度 [leftmargin=0em]
	\item
求粒子过 $ Q $ 点时速度的大小和方向。



\item 
若磁感应强度的大小为一确定值 $ B_{0} $,粒子将以垂直 $ y $ 轴的
方向进入第二象限,求 $ B_{0} $。

\item 
若磁感应强度的大小为另一确定值,经过一段时间后粒子将再次经过 $ Q $ 点,且速度与第一次
过 $ Q $ 点时相同,求该粒子相邻两次经过 $ Q $ 点所用的时间。

	
\end{enumerate}
\begin{figure}[h!]
	\flushright
	\includesvg[width=0.25\linewidth]{picture/svg/GZ-3-tiyou-1262}
\end{figure}


\banswer{
	\begin{enumerate}
		%\renewcommand{\labelenumi}{\arabic{enumi}.}
		% A(\Alph) a(\alph) I(\Roman) i(\roman) 1(\arabic)
		%设定全局标号series=example	%引用全局变量resume=example
		%[topsep=-0.3em,parsep=-0.3em,itemsep=-0.3em,partopsep=-0.3em]
		%可使用leftmargin调整列表环境左边的空白长度 [leftmargin=0em]
		\item
		$v=2 \sqrt{\frac{q E d}{m}}$,与$ x $轴正向夹角为$ 45 \degree  $。
		\item 
		$B_{0}=\sqrt{\frac{m E}{2 q d}}$
		\item 
		$t=(2+\pi) \sqrt{\frac{2 m d}{q E}}$
	\end{enumerate}
}


\item 
\exwhere{$ 2013 $ 年福建卷}
如图甲,空间存在—范围足够大的垂直于 $ xOy $ 平面向外的匀强磁场,磁感应强度大小
为 $ B $。让质量为 $ m $,电量为 $ q $($ q>0 $)的粒子从坐标原点 $ O $ 沿 $ xOy $ 平面以不同的初速度大小和方向
入射到该磁场中。不计重力和粒子间的影响。
\begin{enumerate}
	%\renewcommand{\labelenumi}{\arabic{enumi}.}
	% A(\Alph) a(\alph) I(\Roman) i(\roman) 1(\arabic)
	%设定全局标号series=example	%引用全局变量resume=example
	%[topsep=-0.3em,parsep=-0.3em,itemsep=-0.3em,partopsep=-0.3em]
	%可使用leftmargin调整列表环境左边的空白长度 [leftmargin=0em]
	\item
若粒子以初速度 $ v_{1} $ 沿 $ y $ 轴正向入
射,恰好能经过 $ x $ 轴上的 $ A(a , O) $
点,求 $ v_{1} $ 的大小;

\item 
已知一粒子的初建度大小为 $ v $
($ v> v_{1} $),为使该粒子能经过 $ A(a , O) $点,其入射角$ \theta $(粒子初速度与 $ x $ 轴正向的夹角)有几个?并
求出对应的 $ \sin \theta $值;

\item 
如图乙,若在此空间再加入沿 $ y $ 轴正向、大小为 $ E $ 的匀强电场,一粒子从 $ O $ 点以初速度 $ v_{0} $ 沿
$ x $ 轴正向发射。研究表明:粒子在 $ xOy $ 平面内做周期性运动,且在任一时刻,粒子速度的 $ x $ 分量 $ v_{x} $
与其所在位置的 $ y $ 坐标成正比,比例系数与场强大小 $ E $ 无关。求该粒子运动过程中的最大速度值
$ v_{m} $。

\end{enumerate}
\begin{figure}[h!]
	\centering
\begin{subfigure}{0.4\linewidth}
	\centering
	\includesvg[width=0.7\linewidth]{picture/svg/GZ-3-tiyou-1263} 
	\caption{}\label{}
\end{subfigure}
\begin{subfigure}{0.4\linewidth}
	\centering
	\includesvg[width=0.7\linewidth]{picture/svg/GZ-3-tiyou-1264} 
	\caption{}\label{}
\end{subfigure}
\end{figure}


\banswer{
	\begin{enumerate}
		%\renewcommand{\labelenumi}{\arabic{enumi}.}
		% A(\Alph) a(\alph) I(\Roman) i(\roman) 1(\arabic)
		%设定全局标号series=example	%引用全局变量resume=example
		%[topsep=-0.3em,parsep=-0.3em,itemsep=-0.3em,partopsep=-0.3em]
		%可使用leftmargin调整列表环境左边的空白长度 [leftmargin=0em]
		\item
		$v_{1}=\frac{B q a}{2 m}$
		\item 
		$\sin \theta=\frac{a q B}{2 m v}$
		\item 
		$v_{m}=\frac{E}{B}+\sqrt{\left(\frac{E}{B}\right)^{2}+v_{0}^{2}}$
	\end{enumerate}
}


\item 
\exwhere{$ 2014 $ 年理综四川卷}
如图所示,水平放置的不带电的平行金属板 $ p $ 和 $ b $ 相距 $ h $,与图示电路相连,金属板厚度不计,
忽略边缘效应。$ p $ 板上表面光滑,涂有绝缘层,其上 $ O $ 点右侧相距 $ h $ 处有小孔 $ K $;$ b $ 板上有小孔
$ T $,且 $ O $、$ T $ 在同一条竖直线上,图示平面为竖直平面。质量为 $ m $、电荷量为$ -q $($ q>0 $)的静止粒
子被发射装置(图中未画出)从 $ O $ 点发射,沿 $ P $ 板上表面运动时间 $ t $ 后到达 $ K $ 孔,不与板碰撞地
进入两板之间。粒子视为质点,在图示平面内运动,电荷量保持不变,不计空气阻力,重力加速
度大小为 $ g $。
\begin{enumerate}
	%\renewcommand{\labelenumi}{\arabic{enumi}.}
	% A(\Alph) a(\alph) I(\Roman) i(\roman) 1(\arabic)
	%设定全局标号series=example	%引用全局变量resume=example
	%[topsep=-0.3em,parsep=-0.3em,itemsep=-0.3em,partopsep=-0.3em]
	%可使用leftmargin调整列表环境左边的空白长度 [leftmargin=0em]
	\item
求发射装置对粒子做的功;

\item 
电路中的直流电源内阻为 $ r $,开关 $ S $ 接“$ 1 $”位置时,进入板间的粒子落在 $ h $ 板上的 $ A $ 点,$ A $ 点
与过 $ K $ 孔竖直线的距离为 $ l $。此后将开关 $ S $ 接“$ 2 $”位置,求阻值为 $ R $ 的电阻中的电流强度;

\item 
若选用恰当直流电源,电路中开关 $ S $ 接“$ l $”位置,使进入板间的粒子受力平衡,此时在板间
某区域加上方向垂直于图面的、磁感应强度大
小合适的匀强磁场(磁感应强度 $ B $ 只能在 $ 0 \sim B_{m}=\frac{(\sqrt{21}+5) m}{(\sqrt{21}-2) q t}$ 范围内选取),使粒子恰好从
$ b $ 板的 $ T $ 孔飞出,求粒子飞出时速度方向与 $ b $ 板板面夹角的所有可能值(可用反三角函数表
示)。

	
\end{enumerate}
\begin{figure}[h!]
	\flushright
	\includesvg[width=0.25\linewidth]{picture/svg/GZ-3-tiyou-1265}
\end{figure}

\banswer{
	\begin{enumerate}
		%\renewcommand{\labelenumi}{\arabic{enumi}.}
		% A(\Alph) a(\alph) I(\Roman) i(\roman) 1(\arabic)
		%设定全局标号series=example	%引用全局变量resume=example
		%[topsep=-0.3em,parsep=-0.3em,itemsep=-0.3em,partopsep=-0.3em]
		%可使用leftmargin调整列表环境左边的空白长度 [leftmargin=0em]
		\item
		$W=\frac{m h^{2}}{2 t^{2}}$
		\item 
		$I=\frac{m h}{q(R+r)}\left(g-\frac{2 h^{3}}{l^{2} t^{2}}\right)$
		\item 
		当 $ B $ 逐渐减小,粒子做匀速圆周运动的半径为 $ R $ 也随之变大,D点向 $ b $ 板靠近,$ DT $ 与 $ b $ 板上
		表面的夹角$ \theta $也越变越小,当 $ D $ 点无限接近于 $ b $ 板上表面时,粒子离开磁场后在板间几乎沿着 $ b $
		板上表面从 $ T $ 孔飞出板间区域,此时 $ B_m>B>0 $ 满足题目要求,夹角$ \theta $趋近$ \theta _{0} $,即
		$ \theta _{0}=0 $。
		则题目所求为$0<\theta \leq \arcsin \frac{2}{5}$。
	\end{enumerate}
}















\end{enumerate}

