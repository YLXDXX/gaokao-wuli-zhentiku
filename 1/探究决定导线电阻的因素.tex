\bta{探究决定导线电阻的因素}


\begin{enumerate}
%\renewcommand{\labelenumi}{\arabic{enumi}.}
% A(\Alph) a(\alph) I(\Roman) i(\roman) 1(\arabic)
%设定全局标号series=example	%引用全局变量resume=example
%[topsep=-0.3em,parsep=-0.3em,itemsep=-0.3em,partopsep=-0.3em]
%可使用leftmargin调整列表环境左边的空白长度 [leftmargin=0em]
\item
\exwhere{$ 2011 $ 年理综浙江卷}
在“探究导体电阻与其影响因素的定量关系”实验中,
为了探究 $ 3 $ 根材料未知,横截面积均为$ S=0.20 \ mm^{2} $ 的金属丝$ a $、$ b $、
$ c $ 的电阻率,采用如图所示的实验电路。$ M $ 为金属丝 $ c $ 的左端点,
$ O $ 为金属丝 $ a $ 的右端点,$ P $ 是金属丝上可移动的接触点。在实验
过程中,电流表读数始终为 $ I=1.25 \ A $,电压表读数 $ U $ 随 $ OP $ 间距离 $ x $ 的变化如下表:
\begin{table}[h!]
\centering 
\begin{tabular}{|c|c|c|c|c|c|c|c|c|c|c|c|c|c|c|}
\hline 
$ x/mm $ & $ 600 $ & $ 700 $ & $ 800 $ & $ 900 $ & $ 1000 $ & $ 1200 $ & $ 1400 $ & $ 1600 $ & $ 1800 $ & $ 2000 $ & $ 2100 $ & $ 2200 $ & $ 2300 $ & $ 2400 $
\\
\hline
$ U/V $ & $ 3.95 $ & $ 4.50 $ & $ 5.10 $ & $ 5.90 $ & $ 6.50 $ & $ 6.65 $ & $ 6.82 $ & $ 6.93 $ & $ 7.02 $ & $ 7.15 $ & $ 7.85 $ & $ 8.50 $ & $ 9.05 $ & $ 9.75 $\\ 
\hline 
\end{tabular}
\end{table} 
\begin{figure}[h!]
\centering
\includesvg[width=0.23\linewidth]{picture/svg/GZ-3-tiyou-0988}
\end{figure}

\begin{enumerate}
%\renewcommand{\labelenumi}{\arabic{enumi}.}
% A(\Alph) a(\alph) I(\Roman) i(\roman) 1(\arabic)
%设定全局标号series=example	%引用全局变量resume=example
%[topsep=-0.3em,parsep=-0.3em,itemsep=-0.3em,partopsep=-0.3em]
%可使用leftmargin调整列表环境左边的空白长度 [leftmargin=0em]
\item
绘出电压表读数 $ U $ 随 $ OP $ 间距离 $ x $ 变化的图线;
\item 
求出金属丝的电阻率$ \rho $,并进行比较。



\end{enumerate}

\banswer{
\begin{enumerate}
%\renewcommand{\labelenumi}{\arabic{enumi}.}
% A(\Alph) a(\alph) I(\Roman) i(\roman) 1(\arabic)
%设定全局标号series=example	%引用全局变量resume=example
%[topsep=-0.3em,parsep=-0.3em,itemsep=-0.3em,partopsep=-0.3em]
%可使用leftmargin调整列表环境左边的空白长度 [leftmargin=0em]
\item
如图所示;
\begin{center}
\includesvg[width=0.23\linewidth]{picture/svg/GZ-3-tiyou-0989} 
\end{center}
\item 	
电阻率的允许范围:\\	
$\rho_{a}: 0.96 \times 10^{-6} \Omega \cdot m \sim 1.10 \times 10^{-6} \Omega \cdot m$\\
$\rho_{b}: 8.5 \times 10^{-6} \Omega \cdot m \sim 1.10 \times 10^{-7} \Omega \cdot m$\\
$\rho_{c}: 0.96 \times 10^{-6} \Omega \cdot m \sim 1.10 \times 10^{-6} \Omega \cdot m$	\\
通过计算可知,金属丝$ a $与$ c $电阻率相同,远大于金属丝$ b $的电阻率。
\end{enumerate}
}








\end{enumerate}

