\gaokaoheader{2020}{上海卷}



%一、单项选择题(共80分,1至25题每小题2分,26至35题每小题3分。每小题只有一个正确选项)


\gaokaoxz


\begin{enumerate}
	%\renewcommand{\labelenumi}{\arabic{enumi}.}
	% A(\Alph) a(\alph) I(\Roman) i(\roman) 1(\arabic)
	%设定全局标号series=example	%引用全局变量resume=example
	%[topsep=-0.3em,parsep=-0.3em,itemsep=-0.3em,partopsep=-0.3em]
	%可使用leftmargin调整列表环境左边的空白长度 [leftmargin=0em]
	\item
原子核符号 $^{17}_{8} O $ 中,$ 17  $表示 \xzanswer{D} 
\fourchoices
{电子数}
{质子数}
{中子数}
{核子数}



\item
下列电磁波中穿透能力最强的是 \xzanswer{A} 
\fourchoices
{$ \gamma $射线}
{$ X $射线}
{紫外线}
{红外线}


\item 
天然放射性元素的发现揭示了 \xzanswer{D} 
\fourchoices
{质子拥有复杂结构}
{分子拥有复杂结构}
{原子拥有复杂结构}
{原子核拥有复杂结构}



\item
下列不是基本单位的是 \xzanswer{A} 
\fourchoices
{牛顿}
{千克}
{米}
{秒}


\item 
太阳辐射的能量主要来自于太阳内部的 \xzanswer{D}

\fourchoices
{化学反应}
{裂变反应}
{链式反应}
{热核反应}




\item
一个带电粒子的电量可能为 \xzanswer{A}
\fourchoices
{$2 e$}
{$1.6 e$}
{$1.9 \times 10^{-16} e$}
{$1.6 \times 10^{-19} e$}






 
\item 
机械波在一个周期内传播的距离等于 \xzanswer{A} 
\fourchoices
{一个波长}
{四个波长}
{一个振幅}
{四个振幅}



\item
下列物理量中是矢量的为 \xzanswer{B} 
\fourchoices
{磁通量}
{磁感应强度}
{电流强度}
{磁通量密度}


\item 
如图将小车沿光滑斜面释放瞬间,小车的 \xzanswer{A} 
\begin{figure}[h!]
	\centering
	\includesvg[width=0.19\linewidth]{picture/svg/GZ-3-tiyou-1653}
\end{figure}

\fourchoices
{速度为$ 0 $}
{动能不为$ 0 $}
{加速度为$ 0 $}
{合外力为$ 0 $}



\item
如图,质量为$ M $、内壁光滑的气缸开口向下悬挂于天花板。横截面积为$ S $、质量为$ m $的活塞将一定质量的气体封闭在气缸内。平衡后,封闭气体的压强为(大气压强为$ p_{0} $) \xzanswer{C} 
\begin{figure}[h!]
	\centering
	\includesvg[width=0.2\linewidth]{picture/svg/GZ-3-tiyou-1654}
\end{figure}


\fourchoices
{$p_{0}-\frac{M g}{S}$}
{$p_{0}+\frac{M g}{S}$}
{$p_{0}-\frac{m g}{S}$}
{$p_{0}+\frac{m g}{S}$}



\item
如图,陀螺在平铺于水平桌面的白纸上稳定转动,若在陀螺表面滴上几滴墨水,则由于陀螺转动甩出的墨水在纸上的痕迹最接近于 \xzanswer{B} 
\begin{figure}[h!]
	\centering
	\includesvg[width=0.19\linewidth]{picture/svg/GZ-3-tiyou-1655}
\end{figure}

\pfourchoices
{\includesvg[width=3cm]{picture/svg/GZ-3-tiyou-1657}}
{\includesvg[width=3cm]{picture/svg/GZ-3-tiyou-1658}}
{\includesvg[width=3cm]{picture/svg/GZ-3-tiyou-1659}}
{\includesvg[width=3cm]{picture/svg/GZ-3-tiyou-1660}}






\item
小车从一斜面下滑,受到恒定阻力,下列$ v-t $图中哪个能正确反应小车的运动情况? \xzanswer{C} 

\pfourchoices
{\includesvg[width=3cm]{picture/svg/GZ-3-tiyou-1661}}
{\includesvg[width=3cm]{picture/svg/GZ-3-tiyou-1662}}
{\includesvg[width=3cm]{picture/svg/GZ-3-tiyou-1663}}
{\includesvg[width=3cm]{picture/svg/GZ-3-tiyou-1664}}




\item
如图,在上端有活塞的玻璃管底部放置一小块硝化棉,用手快速向下压活塞,可观察到硝化棉被点燃,在此过程中 \xzanswer{B} 
\begin{figure}[h!]
	\centering
	\includesvg[width=0.036\linewidth]{picture/svg/GZ-3-tiyou-1665}
\end{figure}

\fourchoices
{气体对外界做功,气体内能增加}
{外界对气体做功,气体内能增加}
{气体对外界做功,气体内能减少}
{外界对气体做功,气体内能减少}




\item
波速$ v=4 \ m /s $,沿$ x $轴传播的横波,某时刻质点$ a $沿$ y $轴正方向运动,则波的传播方向与频率分别为 \xzanswer{D} 
\begin{figure}[h!]
	\centering
	\includesvg[width=0.33\linewidth]{picture/svg/GZ-3-tiyou-1666}
\end{figure}

\fourchoices
{$ x $轴正方向,$ 2 \ Hz $}
{$ x $轴正方向,$ 0.5 \ Hz $}
{$ x $轴负方向,$ 2 \ Hz $}
{$ x $轴负方向,$ 0.5 \ Hz $}





\item
如图,在负点电荷$ a $的电场中,$ M $、$ N $两点与$ a $所在处共线,两点的电场强度大小分别为$ E_M $和$ E_N $,则它们的电场强度 \xzanswer{B} 
\begin{figure}[h!]
	\centering
	\includesvg[width=0.5\linewidth]{picture/svg/GZ-3-tiyou-1667}
\end{figure}


\fourchoices
{方向相同, $E_{M}>E_{N}$}
{方向相反, $E_{M}>E_{N}$}
{方向相同, $E_{M}<E_{N}$}
{方向相反,$E_{M}<E_{N}$}





\item
质量为$ 2.0 \times 10^{5} \ kg $的火箭,发射时受到竖直向上、大小为$ 6.0 \times 10^{6} \ N $的推力,加速度大小为(不要忽略重力,$ g=10 \ m/s^{2} $) \xzanswer{C} 
\fourchoices
{$ 4.0 \ m/s^{2} $}
{$ 12 \ m/s^{2} $}
{$ 20 \ m/s^{2} $}
{$ 30 \ m/s^{2} $}




\item
阻值分别为$ 2 \ \Omega $、$ 4 \ \Omega $、$ 8 \ \Omega $的三个电阻$ R_{1} $,$ R_{2} $,$ R_{3} $如图所示连接,电路中$ ab $两点间电压恒定,当$ R_{1} $功率为$ 2 \ W $时,$ R_{3} $的功率为 \xzanswer{A} 
\begin{figure}[h!]
	\centering
	\includesvg[width=0.38\linewidth]{picture/svg/GZ-3-tiyou-1668}
\end{figure}

\fourchoices
{$ 0.5 \ W $}
{$ 1 \ W $}
{$ 4 \ W $}
{$ 8 \ W $}




\item
已知有三个力可以达成力的平衡,以下哪组是不可能的 \xzanswer{C} 

\fourchoices
{$ 4 \ N $,$ 7 \ N $,$ 8 \ N $}
{$ 1 \ N $,$ 8 \ N $,$ 8 \ N $}
{$ 1 \ N $,$ 4 \ N $,$ 6 \ N $}
{$ 1 \ N $,$ 4 \ N $,$ 5 \ N $}



\item 
列车沿平直轨道匀速行驶,车厢光滑地板上有一个相对列车静止的物体,当列车刹车过程中,物体相对轨道 \xzanswer{A} 

\fourchoices
{向前匀速运动}
{向后匀速运动}
{向前加速运动}
{向后加速运动}



\item 
如图,螺线管与电流表组成闭合回路,不能使电流表指针偏转的是(忽略地磁影响) \xzanswer{D} 
\begin{figure}[h!]
	\centering
	\includesvg[width=0.23\linewidth]{picture/svg/GZ-3-tiyou-1669}
\end{figure}

\fourchoices
{螺线管不动,磁铁向上运动}
{螺线管不动,磁铁向左运动}
{磁铁不动,螺线管向上运动}
{磁铁与螺线管以相同速度一起运动}





\item
洗衣机脱水桶上螺丝旋转半径为$ 0.2 \ m $,转速为$ 1200 \ r/min $,小螺丝转动周期和线速度大小分别为 \xzanswer{A} 

\fourchoices
{$ 0.05 \ s $,$ 8 \pi m/s $}
{$ 20 \ s $,$ 8 \pi m/s $}
{$ 0.05 \ s $,$ 16 \pi m/s $}
{$ 20 \ s $,$ 16 \pi m/s $}



\item
如图,质量为$ m $的小球,自井台上方$ H $高处,由静止释放。井深为$ h $,以井台为零势能面,小球落至井底时的的机械能为(不计阻力) \xzanswer{C} 
\begin{figure}[h!]
	\centering
	\includesvg[width=0.23\linewidth]{picture/svg/GZ-3-tiyou-1670}
\end{figure}

\fourchoices
{$ mgh $}
{$ mg ( H-h $) }
{$ mgH $}
{$ mg ( H+h) $}




\item
电动机以$ v $,竖直匀速提升质量为$ m $的物体时,测得电动机两端电压为$ U $,通过电动机的电流为$ I $,则电动机的效率为 \xzanswer{D} 


\fourchoices
{$\frac{U I}{m g v}$}
{$\frac{U I}{m g}$}
{$\frac{m g}{U I}$}
{$\frac{m g v}{U I}$}




\item
在磁场强度为$ B $的匀强磁场中,通过面积为$ S $的矩形面的磁通量大小不可能是 \xzanswer{D} 

\fourchoices
{$ 0 $}
{$ BS $}
{$ 0.5BS $}
{$ 2BS $}





\item
一物体在地面附近以小于$ g $的加速度沿竖直方向匀加速下降,运动过程中物体 \xzanswer{C} 

\fourchoices
{动能增大,机械能增大}
{动能减小,机械能增大}
{动能增大,机械能减小}
{动能减小,机械能减小}




\item
在匀强磁场中,长为$ 10 \ cm $的直导线与磁场方向垂直。当其通有$ 10 \ A $电流时,受到的磁场力大小为$ 0.2 \ N $,则该磁场的磁感应强度大小为 \xzanswer{C} 

\fourchoices
{$ 0.01 \ T $}
{$ 0.02 \ T $}
{$ 0.2 \ T $}
{$ 5 \ T $}




\item
地球在公转轨道的近日点和远日点的加速度 \xzanswer{D} 

\fourchoices
{大小相同,方向相同}
{大小不同,方向相同}
{大小相同,方向不同}
{大小不同,方向不同}




\item
质量为$ 0.1 \ kg $的小球做自由落体运动,下落前$ 2 \ s $内重力的平均功率和$ 2 \ s $末重力的瞬时功率分别为 \xzanswer{B} 

\fourchoices
{$ 10 \ W $,$ 10 \ W $}
{$ 10 \ W $,$ 20 \ W $}
{$ 20 \ W $,$ 10 \ W $}
{$ 20 \ W $,$ 20 \ W $}



\item
如图,$ O $为平衡位置,小球在$ B $、$ C $间做无摩擦往复运动。由$ B $向$ O $运动的过程中,振子的 \xzanswer{C} 
\begin{figure}[h!]
	\centering
	\includesvg[width=0.25\linewidth]{picture/svg/GZ-3-tiyou-1671}
\end{figure}

\fourchoices
{动能增大,势能增大}
{动能减小,势能增大}
{动能增大,势能减小}
{动能减小,势能减小}


\item 
下列选项中,能正确描述某种气体分子速率分布规律的是 \xzanswer{A} 

\pfourchoices
{\includesvg[width=4.3cm]{picture/svg/GZ-3-tiyou-1672}}
{\includesvg[width=4.3cm]{picture/svg/GZ-3-tiyou-1673}}
{\includesvg[width=4.3cm]{picture/svg/GZ-3-tiyou-1674}}
{\includesvg[width=4.3cm]{picture/svg/GZ-3-tiyou-1675}}




\item
如图,两通电直导线$ a $、$ b $平行,$ b $电流向上。两导线相互吸引,则$ a $电流在$ b $处产生的磁场方向 \xzanswer{B} 
\begin{figure}[h!]
	\centering
	\includesvg[width=0.13\linewidth]{picture/svg/GZ-3-tiyou-1676}
\end{figure}

\fourchoices
{向左}
{垂直纸面向里}
{向右}
{垂直纸面向外}






\item
棒$ ab $在匀强磁场中沿导轨运动时,棒中感应电流方向如图。则$ ab $棒的运动方向和螺线管内部磁场方向分别为 \xzanswer{C} 
\begin{figure}[h!]
	\centering
	\includesvg[width=0.23\linewidth]{picture/svg/GZ-3-tiyou-1677}
\end{figure}

\fourchoices
{向左,$ M $指向$ N $}
{向右,$ M $指向$ N $}
{向左,$ N $}
{向右,$ N $指向$ M $}



\item 
如图所示,电压$ U $恒定,灯泡$ A $、$ B $阻值不变,若滑动变阻器$ R $的滑片右移,则 \xzanswer{B} 
\begin{figure}[h!]
	\centering
	\includesvg[width=0.23\linewidth]{picture/svg/GZ-3-tiyou-1678}
\end{figure}

\fourchoices
{$ A $灯变亮,$ B $灯变亮}
{$ A $灯变暗,$ B $灯变亮}
{$ A $灯变亮,$ B $灯变暗}
{$ A $灯变暗,$ B $灯变暗}



\item
如图,质量为$ m $的物体静置于地面。用手缓慢提拉与物体相连的弹簧上端,使物体升高$ h $,手做功一定 \xzanswer{B} 
\begin{figure}[h!]
	\centering
	\includesvg[width=0.14\linewidth]{picture/svg/GZ-3-tiyou-1679}
\end{figure}

\fourchoices
{等于$ mgh $}
{大于$ mgh $}
{小于$ mgh $}
{大于$ 2 \ m gh $}




\item
固定三通管,$ AB $管竖直,$ CD $管水平,水银在管子的$ A $端封闭了一定量的气体。打开阀门,则$ A $端气体 \xzanswer{B} 
\begin{figure}[h!]
	\centering
	\includesvg[width=0.15\linewidth]{picture/svg/GZ-3-tiyou-1680}
\end{figure}

\fourchoices
{体积、压强均增大}
{体积减小,压强增大}
{体积、压强均减小}
{体积增大,压强减小}




%二、实验题(共12分,每小题4分)
\gaokaosy




\item
\begin{enumerate}
	%\renewcommand{\labelenumi}{\arabic{enumi}.}
	% A(\Alph) a(\alph) I(\Roman) i(\roman) 1(\arabic)
	%设定全局标号series=example	%引用全局变量resume=example
	%[topsep=-0.3em,parsep=-0.3em,itemsep=-0.3em,partopsep=-0.3em]
	%可使用leftmargin调整列表环境左边的空白长度 [leftmargin=0em]
	\item
在“用$ DIS $研究机械能守恒定律”的实验中,摆锤释放器的作用是保证每次释放摆锤
时,摆锤的位置 \underlinegap 和速度 \underlinegap 。


 \tk{相同 \quad 为零} 

\item 
在“用$ DIS $研究温度不变时,一定质量的气体与体积的关系”的实验中,
压强传感器 \underlinegap (选填“需要”或“不需要”)调零。能描述缓慢压缩气体过程中,气体压强$ p $与$ V $间关系的图线
是 \underlinegap 。(选填“ \subref{2020上海36a} ”或“ \subref{2020上海36b} ”)
\begin{figure}[h!]
	\centering
\begin{subfigure}{0.4\linewidth}
	\centering
	\includesvg[width=0.5\linewidth]{picture/svg/GZ-3-tiyou-1681} 
	\caption{}\label{2020上海36a}
\end{subfigure}
\begin{subfigure}{0.4\linewidth}
	\centering
	\includesvg[width=0.5\linewidth]{picture/svg/GZ-3-tiyou-1682} 
	\caption{}\label{2020上海36b}
\end{subfigure}
\end{figure}
\end{enumerate}



 \tk{不需要 \quad \subref{2020上海36a} } 



\item 
在“用$ DIS $研究通电螺线管的磁感应强度”的实验中,磁传感器 \underlinegap (选填“需要”或“不需要”)调零。能描述通电螺线管内磁感应强度大小$ B $与磁传感器插入螺线管的长度$ x $间关系的图线可能是 \underlinegap 。(选填“ \subref{2020上海37a} ”或者“ \subref{2020上海37b} ”
)
\begin{figure}[h!]
	\centering
\begin{subfigure}{0.4\linewidth}
	\centering
	\includesvg[width=0.7\linewidth]{picture/svg/GZ-3-tiyou-1683} 
	\caption{}\label{2020上海37a}
\end{subfigure}
\begin{subfigure}{0.4\linewidth}
	\centering
	\includesvg[width=0.7\linewidth]{picture/svg/GZ-3-tiyou-1684} 
	\caption{}\label{2020上海37b}
\end{subfigure}
\end{figure}


 \tk{需要 \quad \subref{2020上海37b} } 




\newpage


\gaokaojs



%三、简答题(共$ 8 $分)

\item 
如图,在点电荷电场中,从$ A $点由静止释放一带负电的微粒,仅受电场力的作用,微粒
\begin{enumerate}
	%\renewcommand{\labelenumi}{\arabic{enumi}.}
	% A(\Alph) a(\alph) I(\Roman) i(\roman) 1(\arabic)
	%设定全局标号series=example	%引用全局变量resume=example
	%[topsep=-0.3em,parsep=-0.3em,itemsep=-0.3em,partopsep=-0.3em]
	%可使用leftmargin调整列表环境左边的空白长度 [leftmargin=0em]
	\item
向何方向运动?
\item 
加速度大小如何变化?
\end{enumerate}
\begin{figure}[h!]
	\flushright 
	\includesvg[width=0.2\linewidth]{picture/svg/GZ-3-tiyou-1685}
\end{figure}

\banswer{
\begin{enumerate}
	%\renewcommand{\labelenumi}{\arabic{enumi}.}
	% A(\Alph) a(\alph) I(\Roman) i(\roman) 1(\arabic)
	%设定全局标号series=example	%引用全局变量resume=example
	%[topsep=-0.3em,parsep=-0.3em,itemsep=-0.3em,partopsep=-0.3em]
	%可使用leftmargin调整列表环境左边的空白长度 [leftmargin=0em]
	\item
	向左
	\item 
	变大
\end{enumerate}
}



\item 
如图,长为$ 10 \ m $的光滑斜面倾角为$ 30 \degree $。质量为$ m $的物体在一沿斜面向上、大小为$ mg $的拉力作用下,由斜面底端由静止开始沿斜面向上运动。($ g $取$ 10 \ m/s^{2} $)
\begin{enumerate}
	%\renewcommand{\labelenumi}{\arabic{enumi}.}
	% A(\Alph) a(\alph) I(\Roman) i(\roman) 1(\arabic)
	%设定全局标号series=example	%引用全局变量resume=example
	%[topsep=-0.3em,parsep=-0.3em,itemsep=-0.3em,partopsep=-0.3em]
	%可使用leftmargin调整列表环境左边的空白长度 [leftmargin=0em]
	\item
求出物体运动到斜面顶端时的速度大小;
\item 
取斜面底端为零势能面,通过分析说明物体沿斜面运动过程中动能与重力势能的大小
关系。

\end{enumerate}
\begin{figure}[h!]
	\flushright
	\includesvg[width=0.22\linewidth]{picture/svg/GZ-3-tiyou-1686}
\end{figure}


\banswer{
\begin{enumerate}
	%\renewcommand{\labelenumi}{\arabic{enumi}.}
	% A(\Alph) a(\alph) I(\Roman) i(\roman) 1(\arabic)
	%设定全局标号series=example	%引用全局变量resume=example
	%[topsep=-0.3em,parsep=-0.3em,itemsep=-0.3em,partopsep=-0.3em]
	%可使用leftmargin调整列表环境左边的空白长度 [leftmargin=0em]
	\item
	$ 10\ m/s  $
	\item 
$E_{k}=E_{p}$
\end{enumerate}
}


	
\end{enumerate}


