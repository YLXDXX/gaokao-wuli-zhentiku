\bta{第$ 11 $讲$ \quad $电场综合}

\begin{enumerate}[leftmargin=0em]
\renewcommand{\labelenumi}{\arabic{enumi}.}
% A(\Alph) a(\alph) I(\Roman) i(\roman) 1(\arabic)
%设定全局标号series=example	%引用全局变量resume=example
%[topsep=-0.3em,parsep=-0.3em,itemsep=-0.3em,partopsep=-0.3em]
%可使用leftmargin调整列表环境左边的空白长度 [leftmargin=0em]
\item
\exwhere{$ 2012 $年物理上海卷}
如图,质量分别为$ m_A $和$ m_B $的两小球带有同种电荷,电荷量分别为$ q_A $和$ q_B $,用绝缘细线悬挂在天花板上。平衡时,两小球恰处于同一水平位置,细线与竖直方向间夹角分别为$ \theta _1 $与$ \theta _2 $($ \theta _1 > \theta _2 $)。两小球突然失去各自所带电荷后开始摆动,最大速度分别$ v_A $和$ v_B $,最大动能分别为$ E_{kA} $和$ E_{kB} $。则 \xzanswer{ACD} 
\begin{figure}[h!]
\centering
\includesvg[width=0.23\linewidth]{picture/svg/131}
\end{figure}


\fourchoices
{$ m_A $一定小于$ m_B $ }
{$ q_A $一定大于$ q_B $}
{$ v_A $一定大于$ v_B $}
{$ E_{kA} $一定大于$ E_{kB} $}




\newpage
\item
\exwhere{$ 2013 $年上海卷}
半径为$ R $,均匀带正电荷的球体在空间产生球对称的电场;场强大小沿半径分布如图所示,图中$ E_{0} $已知,$ E-r $曲线下$ O-R $部分的面积等于$ R-2R $部分的面积。
\begin{enumerate}
\renewcommand{\labelenumi}{\arabic{enumi}.}
% A(\Alph) a(\alph) I(\Roman) i(\roman) 1(\arabic)
%设定全局标号series=example	%引用全局变量resume=example
%[topsep=-0.3em,parsep=-0.3em,itemsep=-0.3em,partopsep=-0.3em]
%可使用leftmargin调整列表环境左边的空白长度 [leftmargin=0em]
\item
写出$ E - r $曲线下面积的单位;
\item 
己知带电球在$ r \geq R $处的场强$ E=\frac{kQ}{r^{2}} $,式中$ k $为静电力常量,该均匀带电球所带的电荷量$ Q $为多大?
\item 
求球心与球表面间的电势差$ \Delta U $;
\item 
质量为$ m $,电荷量为$ q $的负电荷在球面处需具有多大的速度可以刚好运动到$ 2R $处?


\end{enumerate}
\begin{figure}[h!]
\flushright
\includesvg[width=0.23\linewidth]{picture/svg/132}
\end{figure}


\banswer{
\begin{enumerate}
\renewcommand{\labelenumi}{\arabic{enumi}.}
% A(\Alph) a(\alph) I(\Roman) i(\roman) 1(\arabic)
%设定全局标号series=example	%引用全局变量resume=example
%[topsep=-0.3em,parsep=-0.3em,itemsep=-0.3em,partopsep=-0.3em]
%可使用leftmargin调整列表环境左边的空白长度 [leftmargin=0em]
\item
$ V $(或$ N\cdot m/C $)
\item 
$ Q=\frac{E_{0}R^{2} }{k}$
\item 
$ \Delta U= \frac{E_{0}R }{2}$
\item 
$v = \sqrt { \frac { q E _ { 0 } R } { m } }$

\end{enumerate}
}

\newpage
\item
\exwhere{$ 2013 $年浙江卷}
“电子能量分析器”主要由处于真空中的电子偏转器和探测板组成。偏转器是由两个相互绝缘、半径分别为$ R_A $和$ R_B $的同心金属半球面$ A $和$ B $构成,$ A $、$ B $为电势值不等的等势面,其过球心的截面如图所示。一束电荷量为$ e $、质量为$ m $的电子以不同的动能从偏转器左端$ M $的正中间小孔垂直入射,进入偏转电场区域,最后到达偏转器右端的探测板$ N $,其中动能为$ E_{k0} $的电子沿等势面$ C $做匀速圆周运动到达$ N $板的正中间。忽略电场的边缘效应。
\begin{enumerate}
\renewcommand{\labelenumi}{\arabic{enumi}.}
% A(\Alph) a(\alph) I(\Roman) i(\roman) 1(\arabic)
%设定全局标号series=example	%引用全局变量resume=example
%[topsep=-0.3em,parsep=-0.3em,itemsep=-0.3em,partopsep=-0.3em]
%可使用leftmargin调整列表环境左边的空白长度 [leftmargin=0em]
\item
判断球面$ A $、$ B $的电势高低,并说明理由;
\item 
求等势面$ C $所在处电场强度$ E $的大小;
\item 
若半球面$ A $、$ B $和等势面$ C $的电势分别为$ \varphi _ A $、$ \varphi _B $和$ \varphi _C $,则到达$ N $板左、右边缘处的电子,经过偏转电场前、后的动能改变量$ \Delta E_{k\text{左}} $和$ \Delta E_{k\text{右}} $分别为多少?



\end{enumerate}
\begin{figure}[h!]
\flushright
\includesvg[width=0.35\linewidth]{picture/svg/133}
\end{figure}


\banswer{
\begin{enumerate}
\renewcommand{\labelenumi}{\arabic{enumi}.}
% A(\Alph) a(\alph) I(\Roman) i(\roman) 1(\arabic)
%设定全局标号series=example	%引用全局变量resume=example
%[topsep=-0.3em,parsep=-0.3em,itemsep=-0.3em,partopsep=-0.3em]
%可使用leftmargin调整列表环境左边的空白长度 [leftmargin=0em]
\item
$ B $板电势高于$ A $板;
\item 
$E = \frac { 4 E _ { k 0 } } { e \left( R _ { A } + R _ { B } \right) }$
\item 
$\Delta E _ { k \text{左} }= e \left( \varphi _ { B } - \varphi _ { C } \right) , \quad \Delta E _ { k \text{右} } = e \left( \varphi _ { A } - \varphi _ { C } \right)$
\item 
$ |\Delta E_{k\text{左}}|>|\Delta E_{k\text{右}}| $


\end{enumerate}
}


\newpage
\item
\exwhere{$ 2012 $年理综北京卷}
匀强电场的方向沿$ x $轴正向,电场强度$ E $随$ x $的分布如图所示,图中$ E_{0} $和$ d $均为已知量。将带正电的质点$ A $在$ O $点由静止释放。$ A $离开电场足够远后,再将另一带正电的质点$ B $放在$ O $点也由静止释放。当$ B $在电场中运动时,$ A $、$ B $间的相互作用力及相互作用能均为零;$ B $离开电场后,$ A $、$ B $间的相互作用视为静电作用。已知$ A $的电荷量为$ Q $,$ A $和$ B $的质量分别为$ m_{A} $和$ m_{B} $。不计重力。
\begin{enumerate}
\renewcommand{\labelenumi}{\arabic{enumi}.}
% A(\Alph) a(\alph) I(\Roman) i(\roman) 1(\arabic)
%设定全局标号series=example	%引用全局变量resume=example
%[topsep=-0.3em,parsep=-0.3em,itemsep=-0.3em,partopsep=-0.3em]
%可使用leftmargin调整列表环境左边的空白长度 [leftmargin=0em]
\item
求$ A $在电场中的运动时间$ t $;
\item 
若$ B $的电荷量$ q= \frac{ 4 }{ 9 } Q $,求两质点相互作用能的最大值$ E_{pm} $;
\item 
为使$ B $离开电场后不改变运动方向,求$ B $所带电荷量的最大值$ q_m $.



\end{enumerate}
\begin{figure}[h!]
\flushright
\includesvg[width=0.25\linewidth]{picture/svg/134}
\end{figure}

\banswer{
\begin{enumerate}
\renewcommand{\labelenumi}{\arabic{enumi}.}
% A(\Alph) a(\alph) I(\Roman) i(\roman) 1(\arabic)
%设定全局标号series=example	%引用全局变量resume=example
%[topsep=-0.3em,parsep=-0.3em,itemsep=-0.3em,partopsep=-0.3em]
%可使用leftmargin调整列表环境左边的空白长度 [leftmargin=0em]
\item
$t = \sqrt { \frac { 2 d } { a _ { A } } } = \sqrt { \frac { 2 d m } { E _ { 0 } Q } }$
\item 
$E _ { p m } = \frac { E _ { 0 } Q d } { 45 }$
\item 
$q _ { m } = \frac { 16 } { 9 } Q$



\end{enumerate}
}


\newpage
\item
\exwhere{$ 2011 $年理综北京卷}
静电场方向平行于$ x $轴,其电势$ \varphi $随$ x $的分布可简化为如图所示的折线,图中$ \varphi_{0} $和$ d $为已知量。一个带负电的粒子在电场中以$ x=0 $为中心,沿$ x $轴方向做周期性运动。已知该粒子质量为$ m $、电量为$ -q $,其动能与电势能之和为$ -A $($ 0<A<q $),忽略重力。求:
\begin{enumerate}
\renewcommand{\labelenumi}{\arabic{enumi}.}
% A(\Alph) a(\alph) I(\Roman) i(\roman) 1(\arabic)
%设定全局标号series=example	%引用全局变量resume=example
%[topsep=-0.3em,parsep=-0.3em,itemsep=-0.3em,partopsep=-0.3em]
%可使用leftmargin调整列表环境左边的空白长度 [leftmargin=0em]
\item
粒子所受电场力的大小;
\item 
粒子的运动区间;
\item 
粒子的运动周期.



\end{enumerate}
\begin{figure}[h!]
\flushright
\includesvg[width=0.4\linewidth]{picture/svg/135}
\end{figure}



\banswer{
\begin{enumerate}
\renewcommand{\labelenumi}{\arabic{enumi}.}
% A(\Alph) a(\alph) I(\Roman) i(\roman) 1(\arabic)
%设定全局标号series=example	%引用全局变量resume=example
%[topsep=-0.3em,parsep=-0.3em,itemsep=-0.3em,partopsep=-0.3em]
%可使用leftmargin调整列表环境左边的空白长度 [leftmargin=0em]
\item
$F = q E = \frac { q \varphi _ { 0 } } { d }$
\item 
$- d \left( 1 - \frac { A } { q \varphi _ { 0 } } \right) \leq d \leq d \left( 1 - \frac { A } { q \varphi _ { 0 } } \right)$
\item 
$T = 4 t = \frac { 4 d } { q \varphi _ { 0 } } \sqrt { 2 m \left( q \varphi _ { 0 } - A \right) }$



\end{enumerate}
}





\newpage
\item
\exwhere{$ 2013 $年全国卷大纲卷}
一电荷量为$ q $($ q>0 $)、质量为$ m $的带电粒子在匀强电场的作用下,在$ t=0 $时由静止开始运动,场强随时间变化的规律如图所示,不计重力。求在$ t=0 $到$ t=T $的时间间隔内:
\begin{enumerate}
\renewcommand{\labelenumii}{(\arabic{enumii})}

\item 
粒子位移的大小和方向;

\item 
粒子沿初始电场反方向运动的时间.


\end{enumerate}
\begin{figure}[h!]
\flushright
\includesvg[width=0.33\linewidth]{picture/svg/136}
\end{figure}


\banswer{
\begin{enumerate}
\renewcommand{\labelenumi}{\arabic{enumi}.}
% A(\Alph) a(\alph) I(\Roman) i(\roman) 1(\arabic)
%设定全局标号series=example	%引用全局变量resume=example
%[topsep=-0.3em,parsep=-0.3em,itemsep=-0.3em,partopsep=-0.3em]
%可使用leftmargin调整列表环境左边的空白长度 [leftmargin=0em]
\item
在$ t=0 $到$ t=T $时的位移为$s = \frac { q E _ { 0 } } { 16 m } T ^ { 2 }$,方向沿初始电场正方向.
\item 
粒子在$t = \frac { 3 } { 8 } T$到$t = \frac { 5 } { 8 } T$内沿初始电场的反方向运动,总的运动时间$ t $为$t = \frac { 5 } { 8 } T - \frac { 3 } { 8 } T = \frac { T } { 4 }$.



\end{enumerate}
}



\newpage
\item
\exwhere{$ 2011 $年理综浙江卷}
如图甲所示,静电除尘装置中有一长为$ L $、宽为$ b $、高为$ d $的矩形通道,其前、后面板使用绝缘材料,上、下面板使用金属材料。图乙是装置的截面图,上、下两板与电压恒定的高压直流电源相连。质量为$ m $、电荷量为$ -q $、分布均匀的尘埃以水平速度$ v_{0} $进入矩形通道,当带负电的尘埃碰到下板后其所带电荷被中和,同时被收集。通过调整两板间距$ d $可以改变收集效率$ \mu $。当$ d=d_{0} $时$ \mu $为$ 81 \% $(即离下板$ 0.81d_{0} $范围内的尘埃能够被收集)。不计尘埃的重力及尘埃之间的相互
作用。
\begin{enumerate}
\renewcommand{\labelenumi}{\arabic{enumi}.}
% A(\Alph) a(\alph) I(\Roman) i(\roman) 1(\arabic)
%设定全局标号series=example	%引用全局变量resume=example
%[topsep=-0.3em,parsep=-0.3em,itemsep=-0.3em,partopsep=-0.3em]
%可使用leftmargin调整列表环境左边的空白长度 [leftmargin=0em]
\item
求收集效率为$ 100 \% $时,两板间距的最大值$ d_m $;
\item 
求收集率$ \mu $与两板间距$ d $的函数关系;
\item 
若单位体积内的尘埃数为$ n $,求稳定工作时单位时间下板收集的尘埃质量$ \frac{\Delta M}{\Delta t} $与两板间距$ d $的函数关系,并绘出图线.


\end{enumerate}
\begin{figure}[h!]
\flushright
\includesvg[width=0.7\linewidth]{picture/svg/137}
\end{figure}



\banswer{
\begin{enumerate}
\renewcommand{\labelenumi}{\arabic{enumi}.}
% A(\Alph) a(\alph) I(\Roman) i(\roman) 1(\arabic)
%设定全局标号series=example	%引用全局变量resume=example
%[topsep=-0.3em,parsep=-0.3em,itemsep=-0.3em,partopsep=-0.3em]
%可使用leftmargin调整列表环境左边的空白长度 [leftmargin=0em]
\item
$d _ { m } = 0.9 d _ { 0 }$
\item 
$\eta = 0.81 \left( \frac { d _ { 0 } } { d } \right) ^ { 2 }$
\item 
稳定工作时单位时间下板收集的尘埃质量$\Delta M / \Delta t = \eta n m b d v _ { 0 }$

当$d \leq 0.9 d _ { 0 }$时,$ \mu =1 $,因此$\Delta M / \Delta t = n m b d v _ { 0 }$.

当$d > 0.9 d _ { 0 }$时,$\eta = 0.81 \left( \frac { d _ { 0 } } { d } \right) ^ { 2 }$,因此$\Delta M / \Delta t = \eta = 0.8 \ln m b v _ { 0 } \frac { d _ { 0 } ^ { 2 } } { d }$.

图略.

\end{enumerate}
}



\end{enumerate}





