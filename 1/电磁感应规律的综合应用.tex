\bta{电磁感应规律的综合应用}


\begin{enumerate}
%\renewcommand{\labelenumi}{\arabic{enumi}.}
% A(\Alph) a(\alph) I(\Roman) i(\roman) 1(\arabic)
%设定全局标号series=example	%引用全局变量resume=example
%[topsep=-0.3em,parsep=-0.3em,itemsep=-0.3em,partopsep=-0.3em]
%可使用leftmargin调整列表环境左边的空白长度 [leftmargin=0em]
\item
\exwhere{$ 2013 $ 年重庆卷}
如图所示,一段长方体形导电材料,左右两端面的边长都为 $ a $ 和 $ b $,内有带电量为 $ q $ 的某
种自由运动电荷。导电材料置于方向垂直于其前表面向里的匀强磁场中,内部磁感应强度大小为
$ B $。当通以从左到右的稳恒电流 $ I $ 时,测得导电材料上、下表面之间的电压为 $ U $,且上表面的电势
比下表面的低。由此可得该导电材料单位体积内自由运动电
荷数及自由运动电荷的正负分别为 \xzanswer{C} 
\begin{figure}[h!]
\centering
\includesvg[width=0.23\linewidth]{picture/svg/GZ-3-tiyou-0931}
\end{figure}
\fourchoices
{$\frac{I B}{|q| a U},$ 负}
{$\frac{I B}{|q| a U},$ 正}
{$\frac{I B}{|q| b U},$ 负}
{$\frac{I B}{|q| b U},$ 正}



\item
\exwhere{$ 2011 $ 年物理江苏卷}
如图所示,固定的水平长直导线中通有电流 $ I $,矩形线框与导线在同一竖直平面内,且一边与
导线平行。线框由静止释放,在下落过程中 \xzanswer{B} 
\begin{figure}[h!]
\centering
\includesvg[width=0.23\linewidth]{picture/svg/GZ-3-tiyou-0932}
\end{figure}

\fourchoices
{穿过线框的磁通量保持不变}
{线框中感应电流方向保持不变}
{线框所受安掊力的合力为零}
{线框的机械能不断增大}


\item 
\exwhere{$ 2015 $ 年理综新课标 \lmd{1} 卷}
$ 1824 $ 年,法国科学家阿拉果完成了著名的“圆盘实验”。实验中将一铜
圆盘水平放置,在其中心正上方用柔软细线悬挂一枚可以自由旋转的磁
针,如图所示。实验中发现,当圆盘在磁针的磁场中绕过圆盘中心的竖直
轴旋转时,磁针也随着一起转动起来,但略有滞后。下列说法正确的是 \xzanswer{AB} 
\begin{figure}[h!]
\centering
\includesvg[width=0.23\linewidth]{picture/svg/GZ-3-tiyou-0933}
\end{figure}


\fourchoices
{圆盘上产生了感应电动势}
{圆盘内的涡电流产生的磁场导致磁针转动}
{在圆盘转动过程中,磁针的磁场穿过整个圆盘的磁通量发生了变化}
{圆盘中的自由电子随圆盘一起运动形成电流,此电流产生的磁场导致磁针转动}


\item
\exwhere{$ 2015 $ 年理综安徽卷}
如图所示,$ abcd $ 为水平放置的平行“ \hlinepicture{\includesvg[width=1.5em]{picture/svg/GZ-3-tiyou-0934}} ”形光滑金属导轨,间距为 $ l $,
导轨间有垂直于导轨平面的匀强磁场,磁感应强度大小为 $ B $,导轨电阻不计。已知金属杆 $ MN $ 倾斜
放置,与导轨成$ \theta $角,单位长度的电阻为 $ r $,保持金属杆以速度 $ v $ 沿平行于 $ cd $ 的方向滑动(金属杆
滑动过程中与导轨接触良好)。则 \xzanswer{B} 
\begin{figure}[h!]
\centering
\includesvg[width=0.23\linewidth]{picture/svg/GZ-3-tiyou-0935}
\end{figure}

\fourchoices
{电路中感应电动势的大小为$\frac{B l v}{\sin \theta}$}
{电路中感应电流的大小为$\frac{B v \sin \theta}{r}$}
{金属杆所受安培力的大小为$\frac{B^{2} l v \sin \theta}{r}$}
{金属杆的热功率为$\frac{B^{2} l v^{2}}{r \sin \theta}$}

\item 
\exwhere{$ 2011 $ 年理综重庆卷}
有人设计了一种可测速的跑步机,测速原理如图所示,该机底面固定有间距为 $ L $、长
度为 $ d $ 的平行金属电极。电极间充满磁感应强度为 $ B $、方向垂直纸面向里的匀强磁场,且接有电压
表和电阻 $ R $。绝缘橡胶带上镀有间距为 $ d $ 的平行细金属条,磁场中始终仅有一根金属条,且与电极
接触良好,不计金属电阻,若橡胶带匀速运动时,电压表读数为 $ U $,求:
\begin{enumerate}
%\renewcommand{\labelenumi}{\arabic{enumi}.}
% A(\Alph) a(\alph) I(\Roman) i(\roman) 1(\arabic)
%设定全局标号series=example	%引用全局变量resume=example
%[topsep=-0.3em,parsep=-0.3em,itemsep=-0.3em,partopsep=-0.3em]
%可使用leftmargin调整列表环境左边的空白长度 [leftmargin=0em]
\item
橡胶带匀速运动的速率;



\item 
电阻 $ R $ 消耗的电功率;


\item 
一根金属条每次经过磁场区域克服安培力做的功。

\end{enumerate}
\begin{figure}[h!]
\flushright
\includesvg[width=0.25\linewidth]{picture/svg/GZ-3-tiyou-0936}
\end{figure}

\banswer{
\begin{enumerate}
%\renewcommand{\labelenumi}{\arabic{enumi}.}
% A(\Alph) a(\alph) I(\Roman) i(\roman) 1(\arabic)
%设定全局标号series=example	%引用全局变量resume=example
%[topsep=-0.3em,parsep=-0.3em,itemsep=-0.3em,partopsep=-0.3em]
%可使用leftmargin调整列表环境左边的空白长度 [leftmargin=0em]
\item
$v=\frac{U}{B L}$	
\item 	
$P=\frac{U^{2}}{R}$
\item 
$W=\frac{B L U d}{R}$
\end{enumerate}
}

\item 
\exwhere{$ 2017 $ 年天津卷}
如图所示,两根平行金属导轨置于水平面内,学$ | $科网导轨之间接有电阻 $ R $。金
属棒 $ ab $ 与两导轨垂直并保持良好接触,整个装置放在匀强磁场中,磁场方向垂直于导轨平面向
下。现使磁感应强度随时间均匀减小,$ ab $ 始终保持静止,下列说法
正确的是 \xzanswer{D} 
\begin{figure}[h!]
\centering
\includesvg[width=0.23\linewidth]{picture/svg/GZ-3-tiyou-0937}
\end{figure}


\fourchoices
{$ ab $ 中的感应电流方向由 $ b $ 到 $ a $}
{$ ab $ 中的感应电流逐渐减小}
{$ ab $ 所受的安培力保持不变}
{$ ab $ 所受的静摩擦力逐渐减小}


\item 
\exwhere{$ 2013 $ 年新课标 \lmd{1} 卷}
如图,两条平行导轨所在平面与水平地面的夹角为$ \theta $,间距为 $ L $。导轨上端接有一平行板电容器,
电容为 $ C $。导轨处于匀强磁场中,磁感应强度大小为 $ B $,方向垂直于导轨平面。在导轨上放置一质
量为 $ m $ 的金属棒,棒可沿导轨下滑,且在下滑过程中保持与导轨垂直并良好接触。已知金属棒与
导轨之间的动摩擦因数为$ \mu $,重力加速度大小为 $ g $。忽略
所有电阻。让金属棒从导轨上端由静止开始下滑,求:
\begin{enumerate}
%\renewcommand{\labelenumi}{\arabic{enumi}.}
% A(\Alph) a(\alph) I(\Roman) i(\roman) 1(\arabic)
%设定全局标号series=example	%引用全局变量resume=example
%[topsep=-0.3em,parsep=-0.3em,itemsep=-0.3em,partopsep=-0.3em]
%可使用leftmargin调整列表环境左边的空白长度 [leftmargin=0em]
\item
电容器极板上积累的电荷量与金属棒速度大小的关
系;


\item 
金属棒的速度大小随时间变化的关系。


\end{enumerate}
\begin{figure}[h!]
\flushright
\includesvg[width=0.25\linewidth]{picture/svg/GZ-3-tiyou-0938}
\end{figure}

\banswer{
\begin{enumerate}
%\renewcommand{\labelenumi}{\arabic{enumi}.}
% A(\Alph) a(\alph) I(\Roman) i(\roman) 1(\arabic)
%设定全局标号series=example	%引用全局变量resume=example
%[topsep=-0.3em,parsep=-0.3em,itemsep=-0.3em,partopsep=-0.3em]
%可使用leftmargin调整列表环境左边的空白长度 [leftmargin=0em]
\item
$Q=C B L v$	
\item 
$v=\frac{m(\sin \theta-\mu \cos \theta)}{m+B^{2} L^{2} C} g t$
\end{enumerate}
}


\item 
\exwhere{$ 2017 $ 年浙江选考卷}
间距为 $ l $ 的两
平行金属导轨由水平部分和倾斜部分
平滑连接而成,如图所示,倾角为$ \theta $
的导轨处于大小为 $ B_{1} $,方向垂直导轨平面向上的匀强磁场区间 \lmd{1} 中,水平导轨上的无磁场区间静
止放置一质量为 $ 3m $ 的“联动双杆”(由两根长为 $ l $ 的金属杆, $ cd $ 和 $ ef $,用长度为 $ L $ 的刚性绝缘杆
连接而成),在“联动双杆”右侧存在大小为 $ B_{2} $,方向垂直导轨平面向上的匀强磁场区间 \lmd{2} ,其长
度大于 $ L $,质量为 $ m $,长为 $ l $ 的金属杆 $ ab $,从倾斜导轨上端释放,达到匀速后进入水平导轨(无能
量损失),杆 $ cd $ 与“联动双杆”发生碰撞后杆 $ ab $ 和 $ cd $ 合在一起形成“联动三杆”,“联动三杆”继续沿
水平导轨进入磁场区间 \lmd{2} 并从中滑出,运动过程中,杆 $ ab $、 $ cd $ 和 $ ef $ 与导轨始终接触良好,且保
持与导轨垂直。已知杆 $ ab $、 $ cd $ 和 $ ef $ 电阻均为
$ R=0.02 \ \Omega $, $ m=0.1 \ kg $,$ l=0.5 \ m $,$ L=0.3 \ m $,$ \theta =30 \degree $,$ B_1=0.1 \ T $,$ B_2=0.2 \ T $。不计摩擦阻力和导轨电阻,忽
略磁场边界效应。求:
\begin{enumerate}
%\renewcommand{\labelenumi}{\arabic{enumi}.}
% A(\Alph) a(\alph) I(\Roman) i(\roman) 1(\arabic)
%设定全局标号series=example	%引用全局变量resume=example
%[topsep=-0.3em,parsep=-0.3em,itemsep=-0.3em,partopsep=-0.3em]
%可使用leftmargin调整列表环境左边的空白长度 [leftmargin=0em]
\item
杆 $ ab $ 在倾斜导轨上匀速运动时的速度大小 $ v_{0} $;
\item 
联动三杆进入磁场区间 \lmd{2} 前的速度大小 $ v $;
\item 
联动三杆滑过磁场区间 \lmd{2} 产生的焦耳热 $ Q $。


\end{enumerate}
\begin{figure}[h!]
\flushright
\includesvg[width=0.25\linewidth]{picture/svg/GZ-3-tiyou-0939}
\end{figure}


\banswer{
\begin{enumerate}
%\renewcommand{\labelenumi}{\arabic{enumi}.}
% A(\Alph) a(\alph) I(\Roman) i(\roman) 1(\arabic)
%设定全局标号series=example	%引用全局变量resume=example
%[topsep=-0.3em,parsep=-0.3em,itemsep=-0.3em,partopsep=-0.3em]
%可使用leftmargin调整列表环境左边的空白长度 [leftmargin=0em]
\item
$v_{0}=\frac{1.5 m g R \sin \theta}{B_{1}^{2} l^{2}}=6 \ m/ s$
\item 
$v=\frac{v_{0}}{4}=1.5 \ m/s $
\item 
$Q=\frac{1}{2} \times 4 m\left(v^{2}-v^{\prime 2}\right)=0.25 \ J$
\end{enumerate}
}


\item 
\exwhere{$ 2014 $ 年理综新课标\lmd{2}卷}
半径分别为 $ r $ 和 $ 2r $ 的同心圆形导轨固定在同一水平面上,一长为 $ r $,质量为 $ m $ 且质量分布均匀
的直导体棒 $ AB $ 置于圆导轨上面,$ BA $ 的延长线通过圆导轨的中心 $ O $,装置的俯视图如图所示;整
个装置位于一匀强磁场中,磁感应强度的大小为 $ B $,方向竖直向下;在内圆导轨的 $ C $ 点和外圆导轨
的 $ D $ 点之间接有一阻值为 $ R $ 的电阻(图中未画出)。直导体棒在水平
外力作用下以角速度$ \omega $绕 $ O $ 逆时针匀速转动,在转动过程中始终与导
轨保持良好接触。设导体棒与导轨之间的动摩擦因数为$ \mu $,导体棒和
导轨的电阻均可忽略,重力加速度大小为 $ g $,求:
\begin{enumerate}
%\renewcommand{\labelenumi}{\arabic{enumi}.}
% A(\Alph) a(\alph) I(\Roman) i(\roman) 1(\arabic)
%设定全局标号series=example	%引用全局变量resume=example
%[topsep=-0.3em,parsep=-0.3em,itemsep=-0.3em,partopsep=-0.3em]
%可使用leftmargin调整列表环境左边的空白长度 [leftmargin=0em]
\item
通过电阻 $ R $ 的感应电流的方向和大小;

\item 
外力的功率。

\end{enumerate}
\begin{figure}[h!]
\flushright
\includesvg[width=0.25\linewidth]{picture/svg/GZ-3-tiyou-0940}
\end{figure}

\banswer{
\begin{enumerate}
%\renewcommand{\labelenumi}{\arabic{enumi}.}
% A(\Alph) a(\alph) I(\Roman) i(\roman) 1(\arabic)
%设定全局标号series=example	%引用全局变量resume=example
%[topsep=-0.3em,parsep=-0.3em,itemsep=-0.3em,partopsep=-0.3em]
%可使用leftmargin调整列表环境左边的空白长度 [leftmargin=0em]
\item
通过电阻$ R $的感应电流的方向是从$ C $端流向$ D $端。$I=\frac{3 \omega B r^{2}}{2 R}$
\item 
$P=\frac{3}{2} \mu m g \omega r+\frac{9 \omega^{2} B^{2} r^{4}}{4 R}$
\end{enumerate}
}

\item 
\exwhere{$ 2014 $ 年理综安徽卷}
如图 $ 1 $ 所示,匀强磁场的磁感应强度 $ B $ 为 $ 0.5 \ T $,其方向垂直于倾角$ \theta $为 $ 30 \degree $ 的斜面向
上。绝缘斜面上固定有“$ \Lambda $”形状的光滑金属导轨 $ MPN $(电阻忽略不计),$ MP $ 和 $ NP $ 长度均为
$ 2.5 \ m $。$ MN $ 连线水平,长为 $ 3 \ m $。以 $ MN $ 的中点 $ O $ 为原点、$ OP $ 为 $ x $ 轴建立一坐标系 $ Ox $。一根粗细均
匀的金属杆 $ CD $,长度 $ d $ 为 $ 3 \ m $,质量 $ m $ 为 $ 1 \ kg $,电阻 $ R $ 为 $ 0.3 \ \Omega $,在拉力 $ F $ 的作用下,从 $ MN $ 处以
恒定的速度 $ v=1 \ m /s $ 在导轨上沿 $ x $ 轴正向运动(金属杆与导轨接触良好)。$ g $ 取 $ 10 \ m / s_{2} $。
\begin{enumerate}
%\renewcommand{\labelenumi}{\arabic{enumi}.}
% A(\Alph) a(\alph) I(\Roman) i(\roman) 1(\arabic)
%设定全局标号series=example	%引用全局变量resume=example
%[topsep=-0.3em,parsep=-0.3em,itemsep=-0.3em,partopsep=-0.3em]
%可使用leftmargin调整列表环境左边的空白长度 [leftmargin=0em]
\item
求金属杆 $ CD $ 运动过
程中产生的感应电动势 $ E $
及运动到 $ x=0.8 \ m $ 电势差
$ U_{CD} $;


\item 
推导金属杆 $ CD $ 从
$ MN $ 处运动到 $ P $ 点过程中
拉力 $ F $ 与位置坐标 $ x $ 的关系式,并在图 $ 2 $ 中画出 $ F-x $ 关系图象;
\item 
求金属杆 $ CD $ 从 $ MN $ 处运动到 $ P $ 点的全过程产生的焦耳热。


\end{enumerate}
\begin{figure}[h!]
\flushright 
\begin{subfigure}{0.4\linewidth}
\centering
\includesvg[width=0.7\linewidth]{picture/svg/GZ-3-tiyou-0941} 
\caption{}\label{}
\end{subfigure}
\begin{subfigure}{0.4\linewidth}
\centering
\includesvg[width=0.7\linewidth]{picture/svg/GZ-3-tiyou-0942} 
\caption{}\label{}
\end{subfigure}
\end{figure}

\banswer{
\begin{enumerate}
%\renewcommand{\labelenumi}{\arabic{enumi}.}
% A(\Alph) a(\alph) I(\Roman) i(\roman) 1(\arabic)
%设定全局标号series=example	%引用全局变量resume=example
%[topsep=-0.3em,parsep=-0.3em,itemsep=-0.3em,partopsep=-0.3em]
%可使用leftmargin调整列表环境左边的空白长度 [leftmargin=0em]
\item
$ E=1.5 \ V \quad U_{CD}=-0.6 \ V $ 
\item 	
$ F=12.5-3.75x $
\begin{center}
\includesvg[width=0.23\linewidth]{picture/svg/GZ-3-tiyou-0943} 
\end{center}
\item 
$ Q=7.5 \ J $
\end{enumerate}
}

\item 
\exwhere{$ 2011 $年上海卷}
电阻可忽略的光滑平行金属导轨长 $ s=1.15 \ m $,两导轨间距 $ L=0.75 \ m $,导轨倾角为
$ 30 \degree $,导轨上端 $ ab $ 接一阻值 $ R=1.5 \ \Omega $的电阻,磁感应强度 $ B=0.8 \ T $ 的匀强磁场垂直轨道平面向上。阻
值 $ r=0.5 \ \Omega $,质量 $ m=0.2 \ kg $ 的金属棒与轨道垂直且接触良好,从轨道上端 $ ab $ 处由静止开始下滑至底
端,在此过程中金属棒产生的焦耳热 $ Q_{r}=0.1 \ J $。(取 $ g=10 \ m / s_{2} $)求:
\begin{enumerate}
%\renewcommand{\labelenumi}{\arabic{enumi}.}
% A(\Alph) a(\alph) I(\Roman) i(\roman) 1(\arabic)
%设定全局标号series=example	%引用全局变量resume=example
%[topsep=-0.3em,parsep=-0.3em,itemsep=-0.3em,partopsep=-0.3em]
%可使用leftmargin调整列表环境左边的空白长度 [leftmargin=0em]
\item
金属棒在此过程中克服安培力的功$ W_{ \text{安} } $;


\item 
金属棒下滑速度$ v=2 \ m /s $时的加速度$ a $。

\item 
为求金属棒下滑的最大速度$ v_{m} $,有同学解答如下:由动能定理
$W_{\text {重 }}-W_{\text {安 }}=\frac{1}{2} m v_{m}^{2}$,$ \cdots $。由此所得结果是否正确?若正确,说明理
由并完成本小题;若不正确,给出正确的解答。



\end{enumerate}
\begin{figure}[h!]
\flushright
\includesvg[width=0.25\linewidth]{picture/svg/GZ-3-tiyou-0944}
\end{figure}

\banswer{
\begin{enumerate}
%\renewcommand{\labelenumi}{\arabic{enumi}.}
% A(\Alph) a(\alph) I(\Roman) i(\roman) 1(\arabic)
%设定全局标号series=example	%引用全局变量resume=example
%[topsep=-0.3em,parsep=-0.3em,itemsep=-0.3em,partopsep=-0.3em]
%可使用leftmargin调整列表环境左边的空白长度 [leftmargin=0em]
\item
$W_{\text {安 }}=Q=Q_{R}+Q_{r}=0.4 \ J$
\item 
$a=g \sin 30^{\circ}-\frac{B^{2} L^{2}}{m(R+r)} v=10 \times \frac{1}{2}-\frac{0.8^{2} \times 0.75^{2} \times 2}{0.2 \times(1.5+0.5)}=3.2 \ m/s^{2} $
\item 
此解法正确。属棒下滑时受重力和安培力作用,其运动满足
$m g \sin 30^{\circ}-\frac{B^{2} L^{2}}{R+r} v=m a$
上式表明,加速度随速度增加而减小,棒作加速度减小的加速运动。无论最终是否达到匀速,当棒到达斜面底端时速度一定为最大。由动能定理可以得到棒的末速度,因此上述解法正确。
\[ m g S \sin 30^{\circ}-Q=\frac{1}{2} m v_{m}^{2} \]
\[ \therefore v_{m}=\sqrt{2 g S \sin 30^{\circ}-\frac{2 Q}{m}}=\sqrt{2 \times 10 \times 1.15 \times \frac{1}{2}-\frac{2 \times 0.4}{0.2}}=2.74 \ m/s \]
\end{enumerate}
}


\item 
\exwhere{$ 2014 $ 年理综天津卷}
如图所示,两根足够长的平行金属导轨固定在倾角$ \theta =30 \degree $的斜面上,导轨电阻不计,
间距 $ L=0.4 \ m $.导轨所在空间被分成区域 \lmd{1} 和 \lmd{2} ,两区域的边界与斜面的交线为 $ MN $, \lmd{1} 中的匀强
磁场方向垂直斜面向下, \lmd{2} 中的匀强磁场方向垂直斜面向上,两磁场的磁感应强度大小均为
$ B=0.5 \ T $.在区域 \lmd{1} 中,将质量 $ m_{1} =0.1 \ kg $,电阻 $ R_{1} =0.1 \ \Omega $的金属条 $ ab $ 放在导轨上,$ ab $ 刚好不下
滑.然后,在区域 \lmd{2} 中将质量 $ m_{2} =0.4 \ kg $,电阻 $ R_{2} =0.1 \ \Omega $的光滑导体棒 $ cd $ 置于导轨上,由静止开始
下滑.$ cd $ 在滑动过程中始终处于区域 \lmd{2} 的磁场中,$ ab $、$ cd $ 始终与导轨垂直且两端与导轨保持良好
接触,取 $ g=10 \ m/s^{2} $.问
\begin{enumerate}
%\renewcommand{\labelenumi}{\arabic{enumi}.}
% A(\Alph) a(\alph) I(\Roman) i(\roman) 1(\arabic)
%设定全局标号series=example	%引用全局变量resume=example
%[topsep=-0.3em,parsep=-0.3em,itemsep=-0.3em,partopsep=-0.3em]
%可使用leftmargin调整列表环境左边的空白长度 [leftmargin=0em]
\item
$ cd $ 下滑的过程中,$ ab $ 中的电流方向;

\item 
$ ab $ 刚要向上滑动时,$ cd $ 的速度 $ v $ 多大;

\item 
从 $ cd $ 开始下滑到 $ ab $ 刚要向上滑动的过程
中,$ cd $ 滑动的距离 $ x=3.8 \ m $,此过程中 $ ab $ 上产
生的热量 $ Q $ 是多少.


\end{enumerate}
\begin{figure}[h!]
\flushright
\includesvg[width=0.25\linewidth]{picture/svg/GZ-3-tiyou-0945}
\end{figure}

\banswer{
\begin{enumerate}
%\renewcommand{\labelenumi}{\arabic{enumi}.}
% A(\Alph) a(\alph) I(\Roman) i(\roman) 1(\arabic)
%设定全局标号series=example	%引用全局变量resume=example
%[topsep=-0.3em,parsep=-0.3em,itemsep=-0.3em,partopsep=-0.3em]
%可使用leftmargin调整列表环境左边的空白长度 [leftmargin=0em]
\item
由 $ a $ 流向 $ b $
\item 
$ v=5 \ m /s $
\item 
$ Q=1.3 \ J $
\end{enumerate}
}

\item 
\exwhere{$ 2014 $ 年理综北京卷}
导体切割磁感线的运动可以从宏观和微观两个角度来认识。如图所示,固定于水平面的 $ U $ 型导线
框处于竖直向下的匀强磁场中,金属直导线 $ MN $ 在于其垂直的水平恒力 $ F $ 作用下,在导线框上以
速度 $ v $ 做匀速运动,速度 $ v $ 与恒力 $ F $ 的方向相同:导线 $ MN $ 始终与导线框形成闭合电路。已知导线
$ MN $ 电阻为 $ R $,其长度 $ l $ 恰好等于平行轨道间距,磁场的磁感应
强度为 $ B $。忽略摩擦阻力和导线框的电阻。
\begin{enumerate}
%\renewcommand{\labelenumi}{\arabic{enumi}.}
% A(\Alph) a(\alph) I(\Roman) i(\roman) 1(\arabic)
%设定全局标号series=example	%引用全局变量resume=example
%[topsep=-0.3em,parsep=-0.3em,itemsep=-0.3em,partopsep=-0.3em]
%可使用leftmargin调整列表环境左边的空白长度 [leftmargin=0em]
\item
通过公式推导验证:在$ \triangle t $ 时间内,$ F $ 对导线 $ MN $ 所做的功
$ W $ 等于电路获得的电能 $ W ^{\prime} $,也等于导线 $ MN $ 中产生的焦耳热 $ Q $;
\item 
若导线 $ MN $ 的质量 $ m=8.0 \ g $,长度 $ L=0.10 \ m $,感应电流
$ I=1.0 \ A $,假设一个原子贡献一个自由电子,计算导线 $ MN $ 中电子沿导线长度方向定向移动的平均速
率 $ v_{e} $;
\begin{table}[h!]
\centering
\begin{tabular}{|l|l|}
\hline 阿伏伽德罗常数 $N_{A}$ & $6.0 \times 10^{23} \ mol^{-1}$ \\
\hline 元电荷 $e$ & $1.6 \times 10^{-19} \ C$ \\
\hline 导线 $M N$ 的摩尔质量 $\mu$ & $6.0 \times 10^{-2} \ Kg / mol$ \\
\hline
\end{tabular}
\caption{可能会用到的数据}
\end{table}
\item 
经典物理学认为,金属的电阻源于定向运动的自由电子和金属离子(即金属原子失去电子后
的剩余部分)的碰撞。展开你想象的翅膀,给出一个合理的自由电子的运动模型;在此基础上,
求出导线 $ MN $ 中金属离子对一个自由电子沿导线长度方向的平均作用力 $ f $ 的表达式。


\end{enumerate}
\begin{figure}[h!]
\flushright
\includesvg[width=0.25\linewidth]{picture/svg/GZ-3-tiyou-0946}
\end{figure}



\banswer{
\begin{enumerate}
%\renewcommand{\labelenumi}{\arabic{enumi}.}
% A(\Alph) a(\alph) I(\Roman) i(\roman) 1(\arabic)
%设定全局标号series=example	%引用全局变量resume=example
%[topsep=-0.3em,parsep=-0.3em,itemsep=-0.3em,partopsep=-0.3em]
%可使用leftmargin调整列表环境左边的空白长度 [leftmargin=0em]
\item
推导略
\item 
$v_{e}=7.8 \times 10^{-6} \ m/s $
\item 
从微观角度看,导线中的自由电子与金属离子发生碰撞,可以看做非完全弹性碰撞,自由电子损失的动能转化为焦耳热。从整体角度看,可视为金属离子对自由电子整体运动的平均阻力导致自由电子动能的损失,即$W_{\text {损 }}=N \cdot \bar{F} l$。从宏观角度看,导线$ MN $速度不变,力$ F $做功使外界能量完全转化为焦耳热。有:
$\bar{F}=e v B$
\end{enumerate}
}


\item 
\exwhere{$ 2014 $ 年物理上海卷}
如图,水平面内有一光滑金属导轨,其 $ MN $、$ PQ $ 边的电阻不计,$ MP $ 边的电阻阻值
$ R=1.5 \ \Omega ,MN $ 与 $ MP $ 的夹角为 $ 135 \degree $, $ PQ $ 与 $ MP $ 垂直,$ MP $ 边长度小于 $ 1 \ m $。将质量 $ m=2 \ kg $,电阻不
计的足够长直导体棒搁在导轨上,并与 $ MP $ 平行。棒与 $ MN $、$ PQ $ 交点
$ G $、 $ H $ 间的距离 $ L=4 \ m $。空间存在垂直于导轨平面的匀强磁场,磁感
应强度 $ B=0.5 \ T $。在外力作用下,棒由 $ GH $ 处以一定的初速度向左做直
线运动,运动时回路中的电流强度始终与初始时的电流强度相等。
\begin{enumerate}
%\renewcommand{\labelenumi}{\arabic{enumi}.}
% A(\Alph) a(\alph) I(\Roman) i(\roman) 1(\arabic)
%设定全局标号series=example	%引用全局变量resume=example
%[topsep=-0.3em,parsep=-0.3em,itemsep=-0.3em,partopsep=-0.3em]
%可使用leftmargin调整列表环境左边的空白长度 [leftmargin=0em]
\item
若初速度 $ v_{1} =3 \ m /s $,求棒在 $ GH $ 处所受的安培力大小 $ F_{A} $.

\item 
若初速度 $ v_{2} =1.5 \ m /s $,求棒向左移动距离 $ 2 \ m $ 到达 $ EF $ 所需时间 $ \Delta t $。

\item 
在棒由 $ GH $ 处向左移动 $ 2 \ m $ 到达 $ EF $ 处的过程中,外力做功 $ W=7 \ J $,求初速度 $ v_{3} $。



\end{enumerate}
\begin{figure}[h!]
\flushright
\includesvg[width=0.25\linewidth]{picture/svg/GZ-3-tiyou-0947}
\end{figure}


\banswer{
\begin{enumerate}
%\renewcommand{\labelenumi}{\arabic{enumi}.}
% A(\Alph) a(\alph) I(\Roman) i(\roman) 1(\arabic)
%设定全局标号series=example	%引用全局变量resume=example
%[topsep=-0.3em,parsep=-0.3em,itemsep=-0.3em,partopsep=-0.3em]
%可使用leftmargin调整列表环境左边的空白长度 [leftmargin=0em]
\item
$ F_{A}=8 \ N $
\item 
$ \Delta t =1 \ s $
\item 
$ v_{3}=1 \ m /s $
\end{enumerate}
}


\item 
\exwhere{$ 2016 $ 年新课标 \lmd{3} 卷}
如图,两条相距 $ l $ 的光滑平行金属导轨位于同一水平面(纸
面)内,其左端接一阻值为 $ R $ 的电阻;一与导轨垂直的金属棒置于两导轨上;在电阻、导轨和金
属棒中间有一面积为 $ S $ 的区域,区域中存在垂直于纸面向里的
均匀磁场,磁感应强度 $ B_{1} $ 随时间 $ t $ 的变化关系为 $ B_1=kt $,式
中 $ k $ 为常量;在金属棒右侧还有一匀强磁场区域,区域左边界
$ MN $(虚线)与导轨垂直,磁场的磁感应强度大小为 $ B_{0} $,方向
也垂直于纸面向里。某时刻,金属棒在一外加水平恒力的作用
下从静止开始向右运动,在 $ t_{0} $ 时刻恰好以速度 $ v_{0} $ 越过 $ MN $,此
后向右做匀速运动。金属棒与导轨始终相互垂直并接触良好,它们的电阻均忽略不计。求:
\begin{enumerate}
%\renewcommand{\labelenumi}{\arabic{enumi}.}
% A(\Alph) a(\alph) I(\Roman) i(\roman) 1(\arabic)
%设定全局标号series=example	%引用全局变量resume=example
%[topsep=-0.3em,parsep=-0.3em,itemsep=-0.3em,partopsep=-0.3em]
%可使用leftmargin调整列表环境左边的空白长度 [leftmargin=0em]
\item
在 $ t=0 $ 到 $ t= t_{0} $ 时间间隔内,流过电阻的电荷量的绝对值;
\item 
在时刻 $ t $($ t> t_{0} $)穿过回路的总磁通量和金属棒所受外加水平恒力的大小。


\end{enumerate}
\begin{figure}[h!]
\flushright
\includesvg[width=0.25\linewidth]{picture/svg/GZ-3-tiyou-0948}
\end{figure}


\banswer{
\begin{enumerate}
%\renewcommand{\labelenumi}{\arabic{enumi}.}
% A(\Alph) a(\alph) I(\Roman) i(\roman) 1(\arabic)
%设定全局标号series=example	%引用全局变量resume=example
%[topsep=-0.3em,parsep=-0.3em,itemsep=-0.3em,partopsep=-0.3em]
%可使用leftmargin调整列表环境左边的空白长度 [leftmargin=0em]
\item
$|q|=\frac{k t_{0} S}{R}$
\item 
$f=\left(B_{0} l v_{0}+k S\right) \frac{B_{0} l}{R}$
\end{enumerate}
}


\item 
\exwhere{$ 2016 $ 年上海卷}
如图,一关于 $ y $ 轴对称的导体轨道位于水平面内,磁感应强度为 $ B $
的匀强磁场与平面垂直。一足够长,质量为 $ m $ 的直导体棒沿 $ x $ 轴方向置于轨道上,在外力 $ F $ 作用
下从原点由静止开始沿 $ y $ 轴正方向做加速度为 $ a $ 的匀加速直线运动,运动时棒与 $ x $ 轴始终平行。棒
单位长度的电阻为$ \rho $,与电阻不计的轨道接触良好,运动中产生的热功率随棒位置的变化规律为
$ P=k y^{3/2} $ ($ SI $)。求:
\begin{enumerate}
%\renewcommand{\labelenumi}{\arabic{enumi}.}
% A(\Alph) a(\alph) I(\Roman) i(\roman) 1(\arabic)
%设定全局标号series=example	%引用全局变量resume=example
%[topsep=-0.3em,parsep=-0.3em,itemsep=-0.3em,partopsep=-0.3em]
%可使用leftmargin调整列表环境左边的空白长度 [leftmargin=0em]
\item
导体轨道的轨道方程 $ y=f (x)$
;
\item 
棒在运动过程中受到的安培力 $ F_m $ 随 $ y $ 的变化关系;
\item 
棒从 $ y=0 $ 运动到 $ y=L $ 过程中外力 $ F $ 的功。



\end{enumerate}
\begin{figure}[h!]
\flushright
\includesvg[width=0.25\linewidth]{picture/svg/GZ-3-tiyou-0949}
\end{figure}

\banswer{
\begin{enumerate}
%\renewcommand{\labelenumi}{\arabic{enumi}.}
% A(\Alph) a(\alph) I(\Roman) i(\roman) 1(\arabic)
%设定全局标号series=example	%引用全局变量resume=example
%[topsep=-0.3em,parsep=-0.3em,itemsep=-0.3em,partopsep=-0.3em]
%可使用leftmargin调整列表环境左边的空白长度 [leftmargin=0em]
\item
$y=\left(\frac{4 a B^{2}}{k \rho}\right)^{2} x^{2}$
\item 
$F_{m}=\frac{k}{\sqrt{2 a}} y$
\item 
$W=\frac{k}{2 \sqrt{2 a}} L^{2}+m a L$		
\end{enumerate}
}



\item 
\exwhere{$ 2018 $ 年江苏卷}
如图所示,竖直放置的 \hlinepicture{\includesvg[width=2em]{picture/svg/GZ-3-tiyou-0950}} 
形光滑导轨宽为 $ L $,矩形匀强磁场 \lmd{1} 、 \lmd{2} 的高和间
距均为 $ d $,磁感应强度为 $ B $。质量为 $ m $ 的水平金属杆由静止释放,进入磁场 \lmd{1} 和 \lmd{2} 时的速度相等。
金属杆在导轨间的电阻为 $ R $,与导轨接触良好,其余电阻不计,重力加速度为 $ g $。金属杆 \xzanswer{BC} 
\begin{figure}[h!]
\centering
\includesvg[width=0.23\linewidth]{picture/svg/GZ-3-tiyou-0951}
\end{figure}

\fourchoices
{刚进入磁场 \lmd{1} 时加速度方向竖直向下}
{穿过磁场 \lmd{1} 的时间大于在两磁场之间 的运动时间}
{穿过两磁场产生的总热量为 $ 4 \ m gd $}
{释放时距磁场 \lmd{1} 上边界的高度 $ h $ 可能小于$\frac{m^{2} g R^{2}}{2 B^{4} L^{4}}$}

\item 
\exwhere{$ 2018 $年浙江卷($ 4 $月选考)}
如图所示,在竖直平面内建立$ xOy $坐标系,
在$ 0 \leq x \leq 0.65 \ m $、$ y \leq 0.40 \ m $范围内存在一具有理想边界,方
向垂直直面向内的匀强磁场区域。一边长$ l=0.10 \ m $、质量
$ m=0.02 \ kg $、电阻$ R=0.40 \ \Omega $ 的匀质正方形刚性导线框$ abcd $
处于图示位置,其中心的坐标为($ 0 $,$ 0.65 \ m $)。现将线
框以初速度$ v_{0}=2.0 \ m /s $水平向右抛出,线框在进入磁场过
程中速度保持不变,然后在磁场中运动,最后从磁场右
边界离开磁场区域,完成运动全过程。线框在全过程中
始终处于$ xOy $平面内,其$ ab $边与$ x $轴保持平行,空气阻力
不计。求:
\begin{enumerate}
%\renewcommand{\labelenumi}{\arabic{enumi}.}
% A(\Alph) a(\alph) I(\Roman) i(\roman) 1(\arabic)
%设定全局标号series=example	%引用全局变量resume=example
%[topsep=-0.3em,parsep=-0.3em,itemsep=-0.3em,partopsep=-0.3em]
%可使用leftmargin调整列表环境左边的空白长度 [leftmargin=0em]
\item
磁感应强度$ B $的大小;

\item 
线框在全过程中产生的焦耳热$ Q $;
\item 
在全过程中,$ cb $两端的电势差$ U_{cb} $与线框中心位置的$ x $坐标的函数关系。

\end{enumerate}
\begin{figure}[h!]
\flushright
\includesvg[width=0.25\linewidth]{picture/svg/GZ-3-tiyou-0952}
\end{figure}

\banswer{
\begin{enumerate}
%\renewcommand{\labelenumi}{\arabic{enumi}.}
% A(\Alph) a(\alph) I(\Roman) i(\roman) 1(\arabic)
%设定全局标号series=example	%引用全局变量resume=example
%[topsep=-0.3em,parsep=-0.3em,itemsep=-0.3em,partopsep=-0.3em]
%可使用leftmargin调整列表环境左边的空白长度 [leftmargin=0em]
\item
$ B=2 \ T $
\item 
$Q=m g l+\frac{1}{2} m v_{0}^{2}-\frac{1}{2} m v^{2}$ 得到 $Q=0.0375 \ J $	
\item 
进入磁场前:$ x \leq 0.4 \ m $,$U_{a b}=0$\\
进入磁场过程:$ 0.4 \ m <x \leq 0.5 \ m $, $U_{a b}=B v_{0} v_{y} t-I \frac{R}{4}=(4 x-1.7) V$\\
在磁场中$ 0.5 \ m <x \leq 0.6 \ m $,$U_{a b}=B v_{0} l=0.4 V$\\
出磁场过程 $ 0.6 \ m <x \leq 0.7 \ m $,$v_{x}=v_{0}-\frac{B l \Delta q^{\prime}}{m}=5(1-x) \ m/s $ \quad $U_{a b}=\frac{B v_{x} l}{R} \times \frac{R}{4}=\frac{1-x}{4} V$		
\end{enumerate}
}


\item 
\exwhere{$ 2016 $ 年天津卷}
电磁缓冲器是应用于车辆上以提高运行安全性的辅助制动装置,其工作原理
是利用电磁阻尼作用减缓车辆的速度。电磁阻尼作用可以借
助如下模型讨论:如图所示,将形状相同的两根平行且足够
长的铝条固定在光滑斜面上,斜面与水平方向夹角为$ \theta $。一
质量为 $ m $ 的条形磁铁滑入两铝条间,恰好匀速穿过,穿过时
磁铁两端面与两铝条的间距始终保持恒定,其引起电磁感应
的效果与磁铁不动,铝条相对磁铁运动相同。磁铁端面是边
长为 $ d $ 的正方形,由于磁铁距离铝条很近,磁铁端面正对两
铝条区域的磁场均可视为匀强磁场,磁感应强度为 $ B $,铝条
的高度大于 $ d $,电阻率为$ \rho $,为研究问题方便,铝条中只考虑
与磁铁正对部分的电阻和磁场,其他部分电阻和磁场可忽略
不计,假设磁铁进入铝条间以后,减少的机械能完全转化为铝条的内能,重力加速度为 $ g $。
\begin{enumerate}
%\renewcommand{\labelenumi}{\arabic{enumi}.}
% A(\Alph) a(\alph) I(\Roman) i(\roman) 1(\arabic)
%设定全局标号series=example	%引用全局变量resume=example
%[topsep=-0.3em,parsep=-0.3em,itemsep=-0.3em,partopsep=-0.3em]
%可使用leftmargin调整列表环境左边的空白长度 [leftmargin=0em]
\item
求铝条中与磁铁正对部分的电流 $ I $;
\item 
若两铝条的宽度均为 $ b $,推导磁铁匀速穿过铝条间时速度 $ v $ 的表达式;
\item 
在其他条件不变的情况下,仅将两铝条更换为宽度 $ b ^{\prime} >b $ 的铝条,磁铁仍以速度 $ v $ 进入铝条
间,试简要分析说明磁铁在铝条间运动时的加速度和速度如何变化。


\end{enumerate}
\begin{figure}[h!]
\flushright
\includesvg[width=0.25\linewidth]{picture/svg/GZ-3-tiyou-0953}
\end{figure}


\banswer{
\begin{enumerate}
%\renewcommand{\labelenumi}{\arabic{enumi}.}
% A(\Alph) a(\alph) I(\Roman) i(\roman) 1(\arabic)
%设定全局标号series=example	%引用全局变量resume=example
%[topsep=-0.3em,parsep=-0.3em,itemsep=-0.3em,partopsep=-0.3em]
%可使用leftmargin调整列表环境左边的空白长度 [leftmargin=0em]
\item
$I=\frac{m g \sin \theta}{2 B d}$
\item 
$v=\frac{\rho m g \sin \theta}{2 B^{2} d^{2} b}$
\item 
磁铁做加速度
逐渐减小的减速运动。直到 $ F ^{\prime} =mg \sin \theta $时,磁铁重新达到平衡状态,将再次以较小的速度匀速下
滑。
\end{enumerate}
}


\item 
\exwhere{$ 2012 $ 年理综天津卷}
如图所示,一对光滑的平行金属导轨固定在同一水平面内,导轨间距 $ L=0.5 \ m $,左端接
有阻值 $ R=0.3 \ \Omega $的电阻,一质量 $ m=0.1 \ kg $,电阻 $ r=0.1 \ \Omega $的金属棒 $ MN $ 放置在导轨上,整个装置置于
竖直向上的匀强磁场中,磁场的磁感应强度 $ B=0.4 \ T $。棒在水平向右的外力作用下,由静止开始
$ a=2 \ m /s^{2} $ 的加速度做匀加速运动,当棒的位移 $ x=9 \ m $ 时撤去外力,
棒继续运动一段距离后 停下来,已知撤去外力前后回路中产生的
焦耳热比 $ Q_{1} : Q_{2} =2:1 $.导轨足够长且电阻不计,棒在运动过程中始终
与导轨垂直且两端与导轨保持良好接触。求:
\begin{enumerate}
%\renewcommand{\labelenumi}{\arabic{enumi}.}
% A(\Alph) a(\alph) I(\Roman) i(\roman) 1(\arabic)
%设定全局标号series=example	%引用全局变量resume=example
%[topsep=-0.3em,parsep=-0.3em,itemsep=-0.3em,partopsep=-0.3em]
%可使用leftmargin调整列表环境左边的空白长度 [leftmargin=0em]
\item
棒在匀加速运动过程中,通过电阻 $ R $ 的电荷量 $ q $;

\item 
撤去外力后回路中产生的焦耳热 $ Q_{2} $;
\item 
外力做的功 $ W_{f} $。


\end{enumerate}
\begin{figure}[h!]
\flushright
\includesvg[width=0.25\linewidth]{picture/svg/GZ-3-tiyou-0954}
\end{figure}

\banswer{
\begin{enumerate}
%\renewcommand{\labelenumi}{\arabic{enumi}.}
% A(\Alph) a(\alph) I(\Roman) i(\roman) 1(\arabic)
%设定全局标号series=example	%引用全局变量resume=example
%[topsep=-0.3em,parsep=-0.3em,itemsep=-0.3em,partopsep=-0.3em]
%可使用leftmargin调整列表环境左边的空白长度 [leftmargin=0em]
\item
$q=\bar{I} t=1.5 \times 3=4.5 \ C$
\item 
$Q_{2}=\Delta E_{k}=\frac{1}{2} m v^{2}=\frac{1}{2} \times 0.1 \times 6^{2}=1.8 \ J$
\item 
$W_{F}=Q_{1}+\Delta E_{k}=3.6+1.8=5.4 \ J$
\end{enumerate}
}


\item 
\exwhere{$ 2012 $ 年理综福建卷}
如图甲,在圆柱形区域内存在一方向竖直向下、磁感应强度大小为 $ B $ 的匀强磁场,在此区域内,
沿水平面固定一半径为 $ r $ 的圆环形光滑细玻璃管,环心 $ O $ 在区域中心。一质量为 $ m $、带电量为 $ q $
($ q>0 $)的小球,在管内沿逆时针方向(从上向下看)做圆周运动。已知磁感应强度大小 $ B $ 随时间
$ t $ 的变化关系如图乙
所示,其中
$T_{0}=\frac{2 \pi m}{q B_{0}}$。设小球在运动过程中电量保持不变,对原磁场的影响可忽略。
\begin{enumerate}
%\renewcommand{\labelenumi}{\arabic{enumi}.}
% A(\Alph) a(\alph) I(\Roman) i(\roman) 1(\arabic)
%设定全局标号series=example	%引用全局变量resume=example
%[topsep=-0.3em,parsep=-0.3em,itemsep=-0.3em,partopsep=-0.3em]
%可使用leftmargin调整列表环境左边的空白长度 [leftmargin=0em]
\item
在 $ t=0 $ 到 $ t=T_{0} $ 这段时间内,小球不受细管侧壁的作用力,求小球的速度大小 $ v_{0} $;
\item 
在竖直向下的磁感应强度增大过程中,将产生涡旋电场,其电场线是在水平面内一系列沿逆
时针方向的同心圆,同一条电场线上各点的场强大小相等。试求 $ t=T_{0} $ 到 $ t=1.5 T_{0} $ 这段时间内:
\begin{enumerate}
%\renewcommand{\labelenumi}{\arabic{enumi}.}
% A(\Alph) a(\alph) I(\Roman) i(\roman) 1(\arabic)
%设定全局标号series=example	%引用全局变量resume=example
%[topsep=-0.3em,parsep=-0.3em,itemsep=-0.3em,partopsep=-0.3em]
%可使用leftmargin调整列表环境左边的空白长度 [leftmargin=0em]
\item
细管内涡旋电场的场强大小 $ E $;
\item 
电场力对小球做的功 $ W $。
\end{enumerate}

\end{enumerate}
\begin{figure}[h!]
\flushright
\begin{subfigure}{0.4\linewidth}
\centering
\includesvg[width=0.7\linewidth]{picture/svg/GZ-3-tiyou-0955} 
\caption{}\label{}
\end{subfigure}
\begin{subfigure}{0.4\linewidth}
\centering
\includesvg[width=0.7\linewidth]{picture/svg/GZ-3-tiyou-0956} 
\caption{}\label{}
\end{subfigure}
\end{figure}


\banswer{
\begin{enumerate}
%\renewcommand{\labelenumi}{\arabic{enumi}.}
% A(\Alph) a(\alph) I(\Roman) i(\roman) 1(\arabic)
%设定全局标号series=example	%引用全局变量resume=example
%[topsep=-0.3em,parsep=-0.3em,itemsep=-0.3em,partopsep=-0.3em]
%可使用leftmargin调整列表环境左边的空白长度 [leftmargin=0em]
\item
$v_{0}=\frac{q B_{0} r}{m}$
\item 
\begin{enumerate}
%\renewcommand{\labelenumi}{\arabic{enumi}.}
% A(\Alph) a(\alph) I(\Roman) i(\roman) 1(\arabic)
%设定全局标号series=example	%引用全局变量resume=example
%[topsep=-0.3em,parsep=-0.3em,itemsep=-0.3em,partopsep=-0.3em]
%可使用leftmargin调整列表环境左边的空白长度 [leftmargin=0em]
\item
$E=\frac{q B_{0}^{2} r}{2 \pi m}$
\item 
$W=\frac{5 q^{2} B_{0}^{2} r^{2}}{8 m}$	
\end{enumerate}
\end{enumerate}
}


\item 
\exwhere{$ 2015 $ 年理综天津卷}
如图所示,“凸”字形硬质金属线框质量为 $ m $,相邻各边互相
垂直,且处于同一竖直平面内,边长为 $ l $,$ cd $ 边长为 $ 2l $,$ ab $ 与 $ cd $ 平行,间距为 $ 2l $。匀强磁场区域
的上下边界均水平,磁场方向垂直于线框所在平面。开始时,$ cd $ 边
到磁场上边界的距离为 $ 2l $,线框由静止释放,从 $ cd $ 边进入磁场直到
$ ef $、$ pq $ 边进入磁场前,线框做匀速运动,在 $ ef $、$ pq $ 边离开磁场后,
$ ab $ 边离开磁场之前,线框又做匀速运动。线框完全穿过磁场过程中
产生的热量为 $ Q $。线框在下落过程中始终处于原竖直平面内,且 $ ab $、$ cd $ 边保持水平,重力加速度
为 $ g $;求:
\begin{enumerate}
%\renewcommand{\labelenumi}{\arabic{enumi}.}
% A(\Alph) a(\alph) I(\Roman) i(\roman) 1(\arabic)
%设定全局标号series=example	%引用全局变量resume=example
%[topsep=-0.3em,parsep=-0.3em,itemsep=-0.3em,partopsep=-0.3em]
%可使用leftmargin调整列表环境左边的空白长度 [leftmargin=0em]
\item
线框 $ ab $ 边将离开磁场时做匀速运动的速度大小是 $ cd $ 边刚进入磁场时的 几倍;
\item 
磁场上下边界间的距离 $ H $。

\end{enumerate}
\begin{figure}[h!]
\flushright
\includesvg[width=0.25\linewidth]{picture/svg/GZ-3-tiyou-0957}
\end{figure}


\banswer{
\begin{enumerate}
%\renewcommand{\labelenumi}{\arabic{enumi}.}
% A(\Alph) a(\alph) I(\Roman) i(\roman) 1(\arabic)
%设定全局标号series=example	%引用全局变量resume=example
%[topsep=-0.3em,parsep=-0.3em,itemsep=-0.3em,partopsep=-0.3em]
%可使用leftmargin调整列表环境左边的空白长度 [leftmargin=0em]
\item
$v_{2}=4 v_{1}$
\item 
$H=\frac{Q}{m g}+28 l$
\end{enumerate}
}




\item 
\exwhere{$ 2015 $ 年理综四川卷}
如图所示,金属导轨 $ MNC $ 和 $ P Q_{D} $,$ MN $ 与 $ PQ $ 平行且间距为
$ L $,所在平面与水平面夹角为$ \alpha $,$ N $、$ Q $ 连线与 $ MN $ 垂直,$ M $、$ P $ 间接有阻值为 $ R $ 的电阻;光滑直导
轨 $ NC $ 和 $ Q_{D} $ 在同一水平面内,与 $ NQ $ 的夹角都为锐角$ \theta $。均匀金属棒 $ ab $ 和 $ ef $ 质量均为 $ m $,长均为
$ L $,$ ab $ 棒初始位置在水平导轨上与 $ NQ $ 重合;$ ef $ 棒垂直放在倾斜导轨上,与导轨间的动摩擦因数为
$ \mu ( \mu $较小),由导轨上的小立柱 $ 1 $ 和 $ 2 $ 阻挡而静止。空间有方向竖直的匀强磁场(图中未画出)。两金
属棒与导轨保持良好接触。不计所有导轨和 $ ab $ 棒的电阻,$ ef $ 棒的阻值为 $ R $,最大静摩擦力与滑动
摩擦力大小相等,忽略感应电流产生的磁场,重力加速度为 $ g $。
\begin{enumerate}
%\renewcommand{\labelenumi}{\arabic{enumi}.}
% A(\Alph) a(\alph) I(\Roman) i(\roman) 1(\arabic)
%设定全局标号series=example	%引用全局变量resume=example
%[topsep=-0.3em,parsep=-0.3em,itemsep=-0.3em,partopsep=-0.3em]
%可使用leftmargin调整列表环境左边的空白长度 [leftmargin=0em]
\item
若磁感应强度大小为 $ B $,给 $ ab $ 棒一个垂直于 $ NQ $、水平向右的速度 $ v_{1} $,在水平导轨上沿运动方向
滑行一段距离后停止,$ ef $ 棒始终静止,求此过程 $ ef $ 棒
上产生的热量;

\item 
在(1)问过程中,$ ab $ 棒滑行距离为 $ d $,求通过 $ ab $ 棒某
横截面的电量;



\item 
若 $ ab $ 棒以垂直于 $ NQ $ 的速度 $ v_{2} $ 在水平导轨上向右匀
速运动,并在 $ NQ $ 位置时取走小立柱 $ 1 $ 和 $ 2 $,且运动过
程中 $ ef $ 棒始终静止。求此状态下最强磁场的磁感应强
度及此磁场下 $ ab $ 棒运动的最大距离。



\end{enumerate}
\begin{figure}[h!]
\flushright
\includesvg[width=0.25\linewidth]{picture/svg/GZ-3-tiyou-0958}
\end{figure}


\banswer{
\begin{enumerate}
%\renewcommand{\labelenumi}{\arabic{enumi}.}
% A(\Alph) a(\alph) I(\Roman) i(\roman) 1(\arabic)
%设定全局标号series=example	%引用全局变量resume=example
%[topsep=-0.3em,parsep=-0.3em,itemsep=-0.3em,partopsep=-0.3em]
%可使用leftmargin调整列表环境左边的空白长度 [leftmargin=0em]
\item
$Q_{e f}=\frac{1}{4} m v_{1}^{2}$
\item 
$q=\frac{2 B d(L-d \cot \theta)}{R}$
\item 
$B_{m}=\frac{1}{L} \sqrt{\frac{m g R(\sin \alpha+\mu \cos \alpha)}{(\cos \alpha-\mu \sin \alpha) v_{2}}}$,方向竖直向上或竖直向下均可,$x_{m}=\frac{\mu L \tan \theta}{\left(1+\mu^{2}\right) \sin \alpha \cos \alpha+\mu}$
\end{enumerate}
}










\end{enumerate}

