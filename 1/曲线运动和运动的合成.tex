\bta{曲线运动、运动的合成}


\begin{enumerate}[leftmargin=0em]
\renewcommand{\labelenumi}{\arabic{enumi}.}
% A(\Alph) a(\alph) I(\Roman) i(\roman) 1(\arabic)
%设定全局标号series=example	%引用全局变量resume=example
%[topsep=-0.3em,parsep=-0.3em,itemsep=-0.3em,partopsep=-0.3em]
%可使用leftmargin调整列表环境左边的空白长度 [leftmargin=0em]
\item
\exwhere{$ 2014 $年理综四川卷}
小文同学在探究物体做曲线运动的条件时,将一条形磁铁放在桌面的不同位置,让小钢珠在水平桌面上从同一位置以相同初速度$ v_{0} $运动,得到不同轨迹。图中$ a $、$ b $、$ c $、$ d $为其中四条运动轨迹,磁铁放在位置$ A $时,小钢珠的运动轨迹是 \tk{$ b $} (填轨迹字母代号),磁铁放在位置$ B $时,小钢珠的运动轨迹是 \tk{$ c $} (填轨迹字母代号)。实验表明,当物体所受合外力的方向跟它的速度方向 \tk{不在} (选填“在”或“不在”)同一直线上时,物体做曲线运动。
\begin{figure}[h!]
\centering
\includesvg[width=0.4\linewidth]{picture/svg/535}
\end{figure}


\item 
\exwhere{$ 2014 $年物理江苏卷}
为了验证平抛运动的小球在竖直方向上做自由落体运动,用如图所示的装置进行实验. 小锤打击弹性金属片,$ A $ 球水平抛出,同时 $ B $ 球被松开,自由下落。 关于该实验,下列说法中正确的有 \xzanswer{BC} 


\begin{minipage}[h!]{0.7\linewidth}
\vspace{0.3em}
\fourchoices
{两球的质量应相等}
{两球应同时落地}
{应改变装置的高度,多次实验}
{实验也能说明 $ A $ 球在水平方向上做匀速直线运动}


\vspace{0.3em}
\end{minipage}
\hfill
\begin{minipage}[h!]{0.3\linewidth}
\flushright
\vspace{0.3em}
\includesvg[width=0.5\linewidth]{picture/svg/536}
\vspace{0.3em}
\end{minipage}


\item 
\exwhere{$ 2014 $年理综四川卷}
有一条两岸平直、河水均匀流动、流速恒为$ v $的大河,小明驾着小船渡河,去程时船头指向始终与河岸垂直,回程时行驶路线与河岸垂直,去程与回程所用时间的比值为$ k $,船在静水中的速度大小相同,则小船在静水中的速度大小为 \xzanswer{B} 

\fourchoices
{$\frac { k v } { \sqrt { k ^ { 2 } - 1 } } \quad$}
{$\frac { v } { \sqrt { 1 - k ^ { 2 } } } \quad$}
{$\frac { k v } { \sqrt { 1 - k ^ { 2 } } } \quad$}
{$\frac { v } { \sqrt { k ^ { 2 } - 1 } }$}







\item 
\exwhere{$ 2012 $年物理上海卷}
图$ a $为测量分子速率分布的装置示意图。圆筒绕其中心匀速转动,侧面开有狭缝$ N $,内侧贴有记录薄膜,$ M $为正对狭缝的位置。从原子炉$ R $中射出的银原子蒸汽穿过屏上$ S $缝后进入狭缝$ N $,在圆筒转动半个周期的时间内相继到达并沉积在薄膜上。展开的薄膜如图$ b $所示,$ NP $,$ PQ $间距相等。则 \xzanswer{AC} 
\begin{figure}[h!]
\centering
\includesvg[width=0.43\linewidth]{picture/svg/537}
\end{figure}

\fourchoices
{到达$ M $附近的银原子速率较大}
{到达$ Q $附近的银原子速率较大}
{位于$ PQ $区间的分子百分率大于位于$ NP $区间的分子百分率}
{位于$ PQ $区间的分子百分率小于位于$ NP $区间的分子百分率}



\item 
\exwhere{$ 2011 $年理综安徽卷}
一般的曲线运动可以分成很多小段,每小段都可以看成圆周运动的一部分,即把整条曲线用一系列不同半径的小圆弧来代替。如图$ (a) $所示,曲线上$ A $点的曲率圆定义为:通过$ A $点和曲线上紧邻$ A $点两侧的两点作一圆,在极限情况下,这个圆就叫做$ A $点的曲率圆,其半径$ \rho $叫做$ A $点的曲率半径。现将一物体沿与水平面成$ \alpha $角的方向以速度$ v_{0} $抛出,如图$ (b) $所示。则在其轨迹最高点$ P $处的曲率半径是 \xzanswer{C} 
\begin{figure}[h!]
\centering
\includesvg[width=0.38\linewidth]{picture/svg/538}
\end{figure}

\fourchoices
{$ \frac { v _ { 0 } ^ { 2 } } { g } $}
{$ \frac { v _ { 0 } ^ { 2 } \sin ^ { 2 } \alpha } { g } $}
{$ \frac { v _ { 0 } ^ { 2 } \cos ^ { 2 } \alpha } { g } $}
{$ \frac { v _ { 0 } ^ { 2 } \cos ^ { 2 } \alpha } { g \sin \alpha } $}




\item 
\exwhere{$ 2011 $年物理江苏卷}
如图所示,甲、乙两同学从河中$ O $点出发,分别沿直线游到$ A $点和$ B $点后,立即沿原路线返回到$ O $点,$ OA $、$ OB $分别与水流方向平行和垂直,且$ OA=OB $。若水流速度不变,两人在靜水中游速相等,则他们所用时间$ t_{ \text{甲} } $、$ t_{ \text{乙} } $的大小关系为
\begin{figure}[h!]
\centering
\includesvg[width=0.23\linewidth]{picture/svg/539}
\end{figure}

\fourchoices
{$ t_{ \text{甲} } < t_{ \text{乙} } $}
{$ t_{ \text{甲} } = t_{ \text{乙} } $}
{$ t_{ \text{甲} } > t_{ \text{乙} } $}
{无法确定}





\item 
\exwhere{$ 2011 $年上海卷}
如图所示,人沿平直的河岸以速度$ v $行走,且通过不可伸长的绳拖船,船沿绳的方向行进,此过程中绳始终与水面平行。当绳与河岸的夹角为$ \alpha $ 时,船的速率为 \xzanswer{C} 
\begin{figure}[h!]
\centering
\includesvg[width=0.3\linewidth]{picture/svg/540}
\end{figure}

\fourchoices
{$ v \sin \alpha $}
{$ \frac { v } { \sin \alpha } $}
{$ v \cos \alpha $}
{$ \frac { v } { \cos \alpha } $}


\item 
\exwhere{$ 2013 $年上海卷}
右图为在平静海面上,两艘拖船$ A $、$ B $拖着驳船$ C $运动的示意图。$ A $、$ B $的速度分别沿着缆绳$ CA $、$ CB $方向,$ A $、$ B $、$ C $不在一条直线上。由于缆绳不可伸长,因此$ C $的速度在$ CA $、$ CB $方向的投影分别与$ A $、$ B $的速度相等,由此可知$ C $的 \xzanswer{BD} 


\begin{minipage}[h!]{0.7\linewidth}
\vspace{0.3em}
\fourchoices
{速度大小可以介于$ A $、$ B $的速度大小之间}
{速度大小一定不小于$ A $、$ B $的速度大小}
{速度方向可能在$ CA $和$ CB $的夹角范围外}
{速度方向一定在$ CA $和$ CB $的夹角范围内}

\vspace{0.3em}
\end{minipage}
\hfill
\begin{minipage}[h!]{0.3\linewidth}
\flushright
\vspace{0.3em}
\includesvg[width=0.5\linewidth]{picture/svg/541}
\vspace{0.3em}
\end{minipage}


\item 
\exwhere{$ 2014 $年理综新课标\lmd{2}卷}
取水平地面为重力势能零点。一物块从某一高度水平抛出,在抛出点其动能与重力势能恰好相等。不计空气阻力,该物块落地时的速度方向与水平方向的夹角为 \xzanswer{B} 

\fourchoices
{$ \frac { \pi } { 6 } $}
{$ \frac { \pi } { 4 } $}
{$ \frac { \pi } { 3 } $}
{$ \frac { 5 \pi } { 12 } $}




\item 
\exwhere{$ 2013 $年海南卷}
关于物体所受合外力的方向,下列说法正确的是 \xzanswer{AD} 

\fourchoices
{物体做速率逐渐增加的直线运动时,其所受合外力的方向一定与速度方向相同}
{物体做变速率曲线运动时,其所受合外力的方向一定改变}
{物体做变速率圆周运动时,其所受合外力的方向一定指向圆心}
{物体做匀速率曲线运动时,其所受合外力的方向总是与速度方向垂直}


\item 
\exwhere{$ 2015 $年广东卷}
如图所示,帆板在海面上以速度$ v $朝正西方向运动,帆船以速度$ v $朝正北方向航行,以帆板为参照物 \xzanswer{D} 
\begin{figure}[h!]
\centering
\includesvg[width=0.23\linewidth]{picture/svg/542}
\end{figure}

\fourchoices
{帆船朝正东方向航行,速度大小为$ v $}
{帆船朝正西方向航行,速度大小为$ v $}
{帆船朝南偏东$ 45 ^{ \circ } $方向航行,速度大小为$ \sqrt{2}v $}
{帆船朝北偏东$ 45 ^{ \circ } $方向航行,速度大小为$ \sqrt{2}v $}


\item 
\exwhere{$ 2018 $年北京卷}
根据高中所学知识可知,做自由落体运动的小球,将落在正下方位置。但实际上,赤道上方$ 200 $ $ m $处无初速下落的小球将落在正下方位置偏东约$ 6 $ $ cm $处。这一现象可解释为,除重力外,由于地球自转,下落过程小球还受到一个水平向东的“力”,该“力”与竖直方向的速度大小成正比。现将小球从赤道地面竖直上抛,考虑对称性,上升过程该“力”水平向西,则小球 \xzanswer{D} 

\fourchoices
{到最高点时,水平方向的加速度和速度均为零}
{到最高点时,水平方向的加速度和速度均不为零}
{落地点在抛出点东侧}
{落地点在抛出点西侧}







\end{enumerate}


