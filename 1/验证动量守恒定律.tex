\bta{验证动量守恒定律}
\begin{enumerate}
\renewcommand{\labelenumi}{\arabic{enumi}.}
% A(\Alph) a(\alph) I(\Roman) i(\roman) 1(\arabic)
%设定全局标号series=example	%引用全局变量resume=example
%[topsep=-0.3em,parsep=-0.3em,itemsep=-0.3em,partopsep=-0.3em]
%可使用leftmargin调整列表环境左边的空白长度 [leftmargin=0em]
\item
\exwhere{$ 2011 $ 年理综北京卷}
如图 $ 2 $,用“碰撞实验器”可以验证动量守恒定律,即研究两个小球在轨道水平部分碰撞前后的动
量关系。
\begin{figure}[h!]
\centering
\includesvg[width=0.33\linewidth]{picture/svg/GZ-3-tiyou-0481}
\end{figure}

①试验中,直接测定小球碰撞前后的速度是不容易的。但是,可以
通过仅测量
\tk{C} 
(填选项前的序号),间接地解决这个问题。
\threechoices
{小球开始释放高度 $ h $}
{小球抛出点距地面得高度 $ H $}
{小球做平抛运动的射程}



②图 $ 2 $ 中 $ O $ 点是小球抛出点在地面上的垂直投影。实验时,先让入射球 $ m_{1} $ 多次从斜轨上 $ S $ 位置静
止释放,找到其平均落地点的位置 $ P $,测量平抛射程 $ OP $。然后,把被碰小球 $ m_{2} $ 静置于轨道的水平
部分,再将入射球 $ m_{1} $ 从斜轨上 $ S $ 位置静止释放,与小球 $ m_{2} $ 相碰,并多次重复。

接下来要完成的必要步骤是
\tk{$ ADE $或$ DEA $或$ DAE $} 
。
(填选项前的符号)

\fivechoices
{用天平测量两个小球的质量 $ m_{1} $、$ m_{2} $}
{测量小球 $ m_{1} $ 开始释放高度 $ h $}
{测量抛出点距地面的高度 $ H $}
{分别找到 $ m_{1} $、$ m_{2} $ 相碰后平均落地点的位置 $ M $、$ N $}
{测量平抛射程 $ OM $、$ ON $}


③若两球相碰前后的动量守恒,其表达式可表示为 \tk{$m_{1} \cdot O M+m_{2} \cdot O N=m_{1} \cdot O P$} (用②中测量的量表示);若碰
撞是弹性碰撞。那么还应满足的表达式为
\tk{$m_{1} \cdot O M^{2}+m_{2} \cdot O N^{2}=m_{1} \cdot O P^{2}$} 
(用②中测量的量表示)。

④经测定,$ m_{1} =45.0 \ g $,$ m_{2} =7.5 \ g $,小球落地点的平均位置距 $ O $ 点的距离如图 $ 3 $ 所示。碰撞前、后 $ m_{1} $ 的动
量分别为 $ p_{1} $ 与 $ p_1 ^{\prime} $,则 $ p_1:p_1 ^{\prime} = $
\tk{14} 
$ :11 $;若碰撞结束时 $ m_{2} $ 的动量为 $ p_2 ^{\prime} $,则 $ p_1 ^{\prime} :p_2 ^{\prime} =11: $ \tk{2.9} 。

实验结果说明,碰撞前、后总动量的比值$\frac{p_{1}}{p_{1}^{\prime}+p_{2}^{\prime}}$
为 \tk{$1 \sim 1.01$} 。
\begin{figure}[h!]
\centering
\includesvg[width=0.43\linewidth]{picture/svg/GZ-3-tiyou-0482}
\end{figure}


⑤有同学认为,在上述实验中仅更换两个小球的材质,其它条件不变,可以使被撞小球做平抛运动的射程
增大。请你用④中已知的数据,分析和计算出被撞小球 $ m_{2} $ 平抛运动射程 $ ON $ 的最大值为 \tk{$ 76.8 $} $ cm $。









\end{enumerate}

