
\btb{选修模块 $ 3 $—$ 5 $(上)}



\begin{enumerate}
	%\renewcommand{\labelenumi}{\arabic{enumi}.}
	% A(\Alph) a(\alph) I(\Roman) i(\roman) 1(\arabic)
	%设定全局标号series=example	%引用全局变量resume=example
	%[topsep=-0.3em,parsep=-0.3em,itemsep=-0.3em,partopsep=-0.3em]
	%可使用leftmargin调整列表环境左边的空白长度 [leftmargin=0em]
	\item
\exwhere{$ 2018 $年江苏卷}
\begin{enumerate}
	%\renewcommand{\labelenumi}{\arabic{enumi}.}
	% A(\Alph) a(\alph) I(\Roman) i(\roman) 1(\arabic)
	%设定全局标号series=example	%引用全局变量resume=example
	%[topsep=-0.3em,parsep=-0.3em,itemsep=-0.3em,partopsep=-0.3em]
	%可使用leftmargin调整列表环境左边的空白长度 [leftmargin=0em]
	\item
已知$ A $和$ B $两种放射性元素的半衰期分别为$ T $和$ 2 T $,则相同质量的$ A $和$ B $经过$ 2 \ T $后,剩有的$ A $和$ B $
质量之比为  \underlinegap 。
\fourchoices
{$ 1:4 $}
{$ 1:2 $}
{$ 2:1 $}
{$ 4:1 $}


 \tk{B} 

\item 
光电效应实验中,用波长为$ \lambda _{0} $的单色光$ A $照射某金属板时,刚好有光电子从金属表面逸出。当
波长为$ \frac{\lambda_{0}}{2} $的单色光$ B $照射该金属板时,光电子的最大初动能为 \underlinegap ,$ A $、$ B $两种光子的动量之比
为  \underlinegap 。 (已知普朗克常量为$ h $、光速为$ c $)

	
 \tk{$ \frac{hc}{\lambda_{0}} $ \quad $ 1:2 $} 



\item 
如图所示,悬挂于竖直弹簧下端的小球质量为$ m $,运动速度的大小为$ v $,方向向下。经过时间
$ t $,小球的速度大小为$ v $,方向变为向上。忽略空气阻力,重力加速度为$ g $,求该
运动过程中,小球所受弹簧弹力冲量的大小。
\begin{figure}[h!]
	\flushright
	\includesvg[width=0.25\linewidth]{picture/svg/GZ-3-tiyou-1601}
\end{figure}

\banswer{
	$I_{F}=\bar{F} t=2 m v+m g t$
}



\end{enumerate}






\item
\exwhere{$ 2017 $ 年江苏卷}
\begin{enumerate}
	%\renewcommand{\labelenumi}{\arabic{enumi}.}
	% A(\Alph) a(\alph) I(\Roman) i(\roman) 1(\arabic)
	%设定全局标号series=example	%引用全局变量resume=example
	%[topsep=-0.3em,parsep=-0.3em,itemsep=-0.3em,partopsep=-0.3em]
	%可使用leftmargin调整列表环境左边的空白长度 [leftmargin=0em]
	\item
原子核的比结合能曲线如图所示,根据该曲线,
下列判断中正的有 \underlinegap .
\begin{figure}[h!]
	\centering
	\includesvg[width=0.23\linewidth]{picture/svg/GZ-3-tiyou-1602}
\end{figure}



\fourchoices
{$ ^{4}_{2}He $ 核的结合能约为 $ 14 \ MeV $}
{$ ^{4}_{2}He $ 核比 $ ^{6}_{3}Li $ 核更稳定}
{两个 $ ^{2}_{1}H $ 核结合成 $ ^{4}_{2}He $ 核时释放能量}
{$ ^{235}_{92}U $ 核中核子的平均结合能比 $ ^{89}_{36}K r $ 核中的大}


 \tk{BC} 

\item 
质子( $ ^{1}_{1}H $ )和$ \alpha $粒子( $ ^{4}_{2}He $ )被加速到相同动能时,质子的动量
 \underlinegap 
(选填“大于”、“小于”或“等于”)$ \alpha $粒子的动量,质子和$ \alpha $粒子的德布罗意波波长之比为 \underlinegap .

 \tk{小于 \quad $ 2:1 $} 


\item 
甲、乙两运动员在做花样滑冰表演,沿同一直线相向运动,速度大小都是 $ 1 \ m /s $,甲、乙相遇
时用力推对方,此后都沿各自原方向的反方向运动,速度大小分别为 $ 1 \ m /s $ 和 $ 2 \ m /s $.求甲、乙两
运动员的质量之比.

\banswer{
	$\frac{m_{1}}{m_{2}}=\frac{3}{2_{1}}$
}



\end{enumerate}


\item 
\exwhere{$ 2016 $ 年新课标  \lmd{1}  卷}
\begin{enumerate}
	%\renewcommand{\labelenumi}{\arabic{enumi}.}
	% A(\Alph) a(\alph) I(\Roman) i(\roman) 1(\arabic)
	%设定全局标号series=example	%引用全局变量resume=example
	%[topsep=-0.3em,parsep=-0.3em,itemsep=-0.3em,partopsep=-0.3em]
	%可使用leftmargin调整列表环境左边的空白长度 [leftmargin=0em]
	\item
现用某一光电管进行光电效应实验,当用某一频率的光入射时,有光电流产生。下
列说法正确的是 \underlinegap 。(填正确答案标号。选对 $ 1 $ 个得 $ 2 $ 分,选对 $ 2 $ 个得 $ 4 $ 分,选对 $ 3 $ 个得
$ 5 $分。每选错 $ 1 $ 个扣 $ 3 $ 分,最低得分为 $ 0 $ 分)


\fivechoices
{保持入射光的频率不变,入射光的光强变大,饱和光电流变大}
{入射光的频率变高,饱和光电流变大}
{入射光的频率变高,光电子的最大初动能变大}
{保持入射光的光强不变,不断减小入射光的频率,始终有光电流产生}
{遏止电压的大小与入射光的频率有关,与入射光的光强无关}


 \tk{ACE} 

\item 
某游乐园入口旁有一喷泉,喷出的水柱将一质量为 $ M $ 的卡通玩具稳定地悬停在空
中。为计算方便起见,假设水柱从横截面积为 $ S $ 的喷口持续以速度 $ v_{0} $ 竖直向上喷出;玩具底部为
平板(面积略大于 $ S $);水柱冲击到玩具底板后,在竖直方向水的速度变为零,在水平方向朝四周
均匀散开。忽略空气阻力。已知水的密度为$ \rho $,重力加速度大小为 $ g $,求:
\begin{enumerate}
	%\renewcommand{\labelenumi}{\arabic{enumi}.}
	% A(\Alph) a(\alph) I(\Roman) i(\roman) 1(\arabic)
	%设定全局标号series=example	%引用全局变量resume=example
	%[topsep=-0.3em,parsep=-0.3em,itemsep=-0.3em,partopsep=-0.3em]
	%可使用leftmargin调整列表环境左边的空白长度 [leftmargin=0em]
	\item
喷泉单位时间内喷出的水的质量;
\item 
玩具在空中悬停时,其底面相对于喷口的高度。
	
\end{enumerate}

\banswer{
	\begin{enumerate}
		%\renewcommand{\labelenumi}{\arabic{enumi}.}
		% A(\Alph) a(\alph) I(\Roman) i(\roman) 1(\arabic)
		%设定全局标号series=example	%引用全局变量resume=example
		%[topsep=-0.3em,parsep=-0.3em,itemsep=-0.3em,partopsep=-0.3em]
		%可使用leftmargin调整列表环境左边的空白长度 [leftmargin=0em]
		\item
		喷泉单位时间内喷出的水的质量为$\frac{\Delta m}{\Delta t}=\rho \cdot v_{0} \cdot S$
		\item 
		$h=\frac{v_{0}^{2}}{2 g}-\frac{M^{2} g}{2 \rho^{2} v_{0}^{2} S^{2}}$
	\end{enumerate}	
}

	
\end{enumerate}




\item 
\exwhere{$ 2016 $ 年新课标 \lmd{2} 卷}
\begin{enumerate}
	%\renewcommand{\labelenumi}{\arabic{enumi}.}
	% A(\Alph) a(\alph) I(\Roman) i(\roman) 1(\arabic)
	%设定全局标号series=example	%引用全局变量resume=example
	%[topsep=-0.3em,parsep=-0.3em,itemsep=-0.3em,partopsep=-0.3em]
	%可使用leftmargin调整列表环境左边的空白长度 [leftmargin=0em]
	\item
在下列描述核过程的方程中,属于$ \alpha $衰变的是 \underlinegap ,属于$ \beta $衰变的是 \underlinegap ,属于
裂变的是 \underlinegap ,属于聚变的是 \underlinegap 。(填正确答案标号)


\begin{multicols}{2}       
	\begin{enumerate}
		\renewcommand{\labelenumiii}{\Alph{enumiii}.}
		% A(\Alph) a(\alph) I(\Roman) i(\roman) 1(\arabic)
		%设定全局标号series=example	%引用全局变量resume=example
		%[topsep=-0.3em,parsep=-0.3em,itemsep=-0.3em,partopsep=-0.3em]
		%可使用leftmargin调整列表环境左边的空白长度 [leftmargin=0em]
		\item
		${ }_{6}^{14} C \rightarrow{ }_{7}^{14} N+{ }_{-1}^{0} e$
		\item 
		${ }_{15}^{32} P \rightarrow{ }_{16}^{32} S+{ }_{-1}^{0} e$
		\item 
		${ }_{92}^{238} U \rightarrow{ }_{90}^{234} T h+{ }_{2}^{4} H e$
		\item 
		${ }_{7}^{14} N+{ }_{2}^{4} He \rightarrow{ }_{8}^{17} O+{ }_{1}^{1} H$
		\item 
		${ }_{92}^{235} U+{ }_{0}^{1} n \rightarrow{ }_{54}^{140} Xe+{ }_{38}^{94} Sr+2 \frac{1}{0} n$
		\item 
		${ }_{1}^{3} H+{ }_{1}^{2} H \rightarrow{ }_{2}^{4} H e+\frac{1}{0} n$
	\end{enumerate}
\end{multicols}



 \tk{$ C \quad AB \quad E \quad F $} 

\item 
如图,光滑冰面上静止放置一表面光滑的斜面体,斜面体右侧一蹲在滑板上的小孩
和其面前的冰块均静止于冰面上。某时刻小孩将冰块以相对冰面 $ 3 \ m /s $ 的速度向斜面体推出,冰块
平滑地滑上斜面体,在斜面体上上升的最大高度为 $ h=0.3 \ m $($ h $ 小于斜面体的高度)。已知小孩与滑
板的总质量为 $ m_{1} =30 \ kg $,冰块的质量为 $ m_{2} =10 \ kg $,
小孩与滑板始终无相对运动。取重力加速度的大小 $ g=10 \ m/s^{2} $。
\begin{enumerate}
	%\renewcommand{\labelenumi}{\arabic{enumi}.}
	% A(\Alph) a(\alph) I(\Roman) i(\roman) 1(\arabic)
	%设定全局标号series=example	%引用全局变量resume=example
	%[topsep=-0.3em,parsep=-0.3em,itemsep=-0.3em,partopsep=-0.3em]
	%可使用leftmargin调整列表环境左边的空白长度 [leftmargin=0em]
	\item
求斜面体的质量;
\item 
通过计算判断,冰块与斜面体分离后能否追上小孩?
	
\end{enumerate}
\begin{figure}[h!]
	\flushright
	\includesvg[width=0.25\linewidth]{picture/svg/GZ-3-tiyou-1603}
\end{figure}


\banswer{
	\begin{enumerate}
		%\renewcommand{\labelenumi}{\arabic{enumi}.}
		% A(\Alph) a(\alph) I(\Roman) i(\roman) 1(\arabic)
		%设定全局标号series=example	%引用全局变量resume=example
		%[topsep=-0.3em,parsep=-0.3em,itemsep=-0.3em,partopsep=-0.3em]
		%可使用leftmargin调整列表环境左边的空白长度 [leftmargin=0em]
		\item
		$ 20 \ kg $
		\item 
		不能
	\end{enumerate}	
}


	
\end{enumerate}


\item 
\exwhere{$ 2016 $ 年新课标 \lmd{3} 卷}
\begin{enumerate}
	%\renewcommand{\labelenumi}{\arabic{enumi}.}
	% A(\Alph) a(\alph) I(\Roman) i(\roman) 1(\arabic)
	%设定全局标号series=example	%引用全局变量resume=example
	%[topsep=-0.3em,parsep=-0.3em,itemsep=-0.3em,partopsep=-0.3em]
	%可使用leftmargin调整列表环境左边的空白长度 [leftmargin=0em]
	\item
一静止的铝原子核 $ ^{27}_{13}A l $ 俘获一速度为 $ 1.0 \times10^{7} \ m /s $ 的质子 $ p $ 后,变为处于激发态的硅原子核
$ ^{28}_{14}Si $,下列说法正确的是 \underlinegap (填正确的答案标号,选对一个得 $ 2 $ 分,选对 $ 2 $ 个得 $ 4 $ 分,选

对 $ 3 $ 个得 $ 5 $ 分,没错选 $ 1 $ 个扣 $ 3 $ 分,最低得分为零分)


\fivechoices
{核反应方程为 $p+_{13}^{27} Al \rightarrow_{14}^{28} Si$}
{核反应方程过程中系统动量守恒}
{核反应过程中系统能量不守恒}
{核反应前后核子数相等,所以生成物的质量等于反应物的质量之和}
{硅原子核速度的数量级为 $ 10^{5} \ m /s $,方向与质子初速度方向一致}


 \tk{ABE} 


\item 
如图所示,水平地面上有两个静止的小物块 $ a $ 和 $ b $,其连线与墙垂直,$ a $ 和 $ b $ 相距
$ l $;$ b $ 与墙之间也相距 $ l $:$ a $ 的质量为 $ m $,$ b $ 的质量为
$  \frac{ 3 }{ 4 }  m $,两
物块与地面间的动摩擦因数均相同,现使 $ a $ 以初速度 $ v_{0} $ 向
右滑动,此后 $ a $ 与 $ b $ 发生弹性碰撞,但 $ b $ 没有与墙发生碰撞,重力加速度大小为 $ g $,求物块与地面
间的动摩擦力因数满足的条件。
\begin{figure}[h!]
	\flushright
	\includesvg[width=0.25\linewidth]{picture/svg/GZ-3-tiyou-1604}
\end{figure}


\banswer{
	$\frac{v_{0}^{2}}{2 g l} \geq \mu \geq \frac{32 v_{0}^{2}}{113 g l}$
}


	
\end{enumerate}


\item 
\exwhere{$ 2016 $ 年江苏卷}
\begin{enumerate}
	%\renewcommand{\labelenumi}{\arabic{enumi}.}
	% A(\Alph) a(\alph) I(\Roman) i(\roman) 1(\arabic)
	%设定全局标号series=example	%引用全局变量resume=example
	%[topsep=-0.3em,parsep=-0.3em,itemsep=-0.3em,partopsep=-0.3em]
	%可使用leftmargin调整列表环境左边的空白长度 [leftmargin=0em]
	\item
贝克勒尔在 $ 120 $ 年前首先发现了天然放射现象,如今原子核的放射性在众多领域中有着广泛应
用.下列属于放射性衰变的是 \underlinegap 。


\fourchoices
{${ }_{6}^{14} C \rightarrow{ }_{7}^{14} N+{ }_{-1}^{0} e$}
{${ }_{92}^{235} U+{ }_{0}^{1} n \rightarrow{ }^{139}{ }_{53}^{139} I+{ }_{39}^{95} Y+2{ }_{0}^{1} n$}
{${ }_{1}^{2} H+{ }_{1}^{3} H \rightarrow{ }_{2}^{4} He+{ }_{0}^{1} n$}
{$\frac{4}{2} He+{ }_{13}^{27} Al \rightarrow{ }_{15}^{30} P+{ }_{0}^{1} n$}


 \tk{A} 



\item 
已知光速为 $ c $,普朗克常数为 $ h $,则频率为$ \nu $的光子的动量为 \underlinegap .用该频率的光垂直照射
平面镜,光被镜面全部垂直反射回去,则光子在反射前后动量改变量的大小为 \underlinegap 。

 \tk{$ \frac{h\nu}{c} $ \quad $ 2\frac{h\nu}{c} $} 

\item 
几种金属的逸出功 $ W_{0} $ 见下表:

\begin{table}[h!]
 \centering 
 \begin{tabular}{|c|c|c|c|c|c|}
 \hline 
金属 & 钨 & 钙 & 钠 & 钾 & 铷
 \\
\hline
$ W_{0}/(10^{-19} \ J) $ & 7.26 & 5.12 & 3.66 & 3.60 & 3.41\\ 
 \hline 
 \end{tabular}
 \end{table} 

由一束可见光照射上述金属的表面,请通过计算说明哪些能发生光电效应.已知该可见光的波长
的范围为 $ 4.0 \times 10^{-7} \sim 7.6 \times 10^{-6} \ m $,普朗克常数 $ h=6.63 \times 10^{-34} \ J \cdot s $。

	
 \tk{钠、钾、铷能发生光电效应} 
	
\end{enumerate}


\item 
\exwhere{$ 2016 $ 年海南卷}
\begin{enumerate}
	%\renewcommand{\labelenumi}{\arabic{enumi}.}
	% A(\Alph) a(\alph) I(\Roman) i(\roman) 1(\arabic)
	%设定全局标号series=example	%引用全局变量resume=example
	%[topsep=-0.3em,parsep=-0.3em,itemsep=-0.3em,partopsep=-0.3em]
	%可使用leftmargin调整列表环境左边的空白长度 [leftmargin=0em]
	\item
下列说法正确的是 \underlinegap 。

\fivechoices
{爱因斯坦在光的粒子性的基础上,建立了光电效应方程}
{康普顿效应表明光子只具有能量,不具有动量}
{成功地解释了氢原子光谱的实验规律}
{卢瑟福根据$ \alpha $粒子散射实验提出了原子的核式结构模型}
{德布罗意指出微观粒子的动量越大,其对应的波长就越长}


 \tk{ACD} 


\item 
如图,物块 $ A $ 通过一不可伸长的轻绳悬挂在天花板下,初始时静止;从发射器(图中
未画出)射出的物块 $ B $ 沿水平方向与 $ A $ 相撞,碰撞后两者粘连在一起运
动,碰撞前 $ B $ 的速度的大小 $ v $ 及碰撞后 $ A $ 和 $ B $ 一起上升的高度 $ h $ 均可由传
感器(图中未画出)测得。某同学以 $ h $ 为纵坐标,$ v^{2} $ 为横坐标,利用实验数
据作直线拟合,求得该直线的斜率为 $ k=1.92  \times 10^{-3} \ s^{2} /m $。已知物块$ A $和$ B $的
质量分别为 $ m_{A} =0.400 \ kg $ 和 $ m_{B} =0.100 \ kg $,重力加速度大小 $ g=9.8 \ m/s^{2} $。
\begin{enumerate}
	%\renewcommand{\labelenumi}{\arabic{enumi}.}
	% A(\Alph) a(\alph) I(\Roman) i(\roman) 1(\arabic)
	%设定全局标号series=example	%引用全局变量resume=example
	%[topsep=-0.3em,parsep=-0.3em,itemsep=-0.3em,partopsep=-0.3em]
	%可使用leftmargin调整列表环境左边的空白长度 [leftmargin=0em]
	\item
若碰撞时间极短且忽略空气阻力,求 $ h- v^{2} $ 直线斜率的理论值 $ k_{0} $。

\item 
求 $ k $ 值的相对误差$\delta\left(\delta=\frac{\left|k-k_{0}\right|}{k_{0}}\right) \times 100 \%$,结果保留 $ 1 $ 位有效数字。
\end{enumerate}
\begin{figure}[h!]
	\flushright
	\includesvg[width=0.25\linewidth]{picture/svg/GZ-3-tiyou-1605}
\end{figure}



\banswer{
	\begin{enumerate}
		%\renewcommand{\labelenumi}{\arabic{enumi}.}
		% A(\Alph) a(\alph) I(\Roman) i(\roman) 1(\arabic)
		%设定全局标号series=example	%引用全局变量resume=example
		%[topsep=-0.3em,parsep=-0.3em,itemsep=-0.3em,partopsep=-0.3em]
		%可使用leftmargin调整列表环境左边的空白长度 [leftmargin=0em]
		\item
		$k_{0}=2.04 \times 10^{-3} \ s^{2} / m$
		\item 
		$\delta=6 \%$
	\end{enumerate}
}


	
\end{enumerate}



\item 
\exwhere{$ 2015 $ 年理综新课标  \lmd{1}  卷}
\begin{enumerate}
	%\renewcommand{\labelenumi}{\arabic{enumi}.}
	% A(\Alph) a(\alph) I(\Roman) i(\roman) 1(\arabic)
	%设定全局标号series=example	%引用全局变量resume=example
	%[topsep=-0.3em,parsep=-0.3em,itemsep=-0.3em,partopsep=-0.3em]
	%可使用leftmargin调整列表环境左边的空白长度 [leftmargin=0em]
	\item
在某次光电效应实验中,得到的遏止电压 $ U_{C} $ 与入射光的
频率$ \nu $的关系如图所示,若该直线的斜率和截距分别为 $ k $ 和 $ b $,电子电
荷量的绝对值为 $ e $,则普朗克常量可表示为 \underlinegap ,所用材料的逸出
功可表示为 \underlinegap 。
\begin{figure}[h!]
	\centering
	\includesvg[width=0.23\linewidth]{picture/svg/GZ-3-tiyou-1606}
\end{figure}


 \tk{$ ek $ \quad $ -eb $} 




\item 
如图,在足够长的光滑水平面上,物体 $ A $、$ B $、$ C $ 位于同一直线上,$ A $ 位于 $ B $、$ C $ 之
间。$ A $ 的质量为 $ m $,$ B $、$ C $ 的质量都为 $ M $,三者均处于静止状
态,现使 $ A $ 以某一速度向右运动,求 $ m $ 和 $ M $ 之间满足什么条
件才能使 $ A $ 只与 $ B $、$ C $ 各发生一次碰撞。设物体间的碰撞都是
弹性的。
\begin{figure}[h!]
	\flushright
	\includesvg[width=0.25\linewidth]{picture/svg/GZ-3-tiyou-1607}
\end{figure}



\banswer{
	$(\sqrt{5}-2)  M \leq m<M$
}




\end{enumerate}



\item
\exwhere{$ 2015 $ 年理综新课标 \lmd{2} 卷}
\begin{enumerate}
	%\renewcommand{\labelenumi}{\arabic{enumi}.}
	% A(\Alph) a(\alph) I(\Roman) i(\roman) 1(\arabic)
	%设定全局标号series=example	%引用全局变量resume=example
	%[topsep=-0.3em,parsep=-0.3em,itemsep=-0.3em,partopsep=-0.3em]
	%可使用leftmargin调整列表环境左边的空白长度 [leftmargin=0em]
	\item
实物粒子和光都具有波粒二象性。下列事实中突出体现波动性的是 \underlinegap 。(填正确答
案标号。选对 $ 1 $ 个得 $ 2 $ 分,选对 $ 2 $ 个得 $ 4 $ 分,选对 $ 3 $ 个的 $ 5 $ 分。每选错 $ 1 $ 个扣 $ 3 $ 分,最低得分为 $ 0 $
分。)


\fivechoices
{电子束通过双缝实验装置后可以形成干涉图样}
{$ \beta $射线在云室中穿过会留下清晰的径迹}
{人们利用慢中子衍射来研究晶体的结构}
{人们利用电子显微镜观测物质的微观结构}
{光电效应实验中,光电子的最大初动能与入射光的频率有关,与入射光的强度无关}


 \tk{ACD}
 
  
\item 
两滑块 $ a $、$ b $ 沿水平面上同一条直线运动,并发生发生碰撞;碰撞后两者粘在一起运
动;经过一段时间后,从光滑路段进入粗糙路段。
两者的位置 $ x $ 随时间 $ t $ 变化的图像如图所示。求:
\begin{enumerate}
	%\renewcommand{\labelenumi}{\arabic{enumi}.}
	% A(\Alph) a(\alph) I(\Roman) i(\roman) 1(\arabic)
	%设定全局标号series=example	%引用全局变量resume=example
	%[topsep=-0.3em,parsep=-0.3em,itemsep=-0.3em,partopsep=-0.3em]
	%可使用leftmargin调整列表环境左边的空白长度 [leftmargin=0em]
	\item
滑块 $ a $、$ b $ 的质量之比;
\item 
整个运动过程中,两滑块克服摩擦力做的功与
因碰撞而损失的机械能之比。
	
\end{enumerate}
\begin{figure}[h!]
	\flushright
	\includesvg[width=0.25\linewidth]{picture/svg/GZ-3-tiyou-1608}
\end{figure}

\banswer{
	\begin{enumerate}
		%\renewcommand{\labelenumi}{\arabic{enumi}.}
		% A(\Alph) a(\alph) I(\Roman) i(\roman) 1(\arabic)
		%设定全局标号series=example	%引用全局变量resume=example
		%[topsep=-0.3em,parsep=-0.3em,itemsep=-0.3em,partopsep=-0.3em]
		%可使用leftmargin调整列表环境左边的空白长度 [leftmargin=0em]
		\item
		$\frac{m_{1}}{m_{2}}=\frac{1}{8}$
		\item 
		$\frac{W}{\Delta E}=\frac{1}{2}$	
	\end{enumerate}
}



\end{enumerate}


\item 
\exwhere{$ 2015 $ 年理综福建卷}
\begin{enumerate}
	%\renewcommand{\labelenumi}{\arabic{enumi}.}
	% A(\Alph) a(\alph) I(\Roman) i(\roman) 1(\arabic)
	%设定全局标号series=example	%引用全局变量resume=example
	%[topsep=-0.3em,parsep=-0.3em,itemsep=-0.3em,partopsep=-0.3em]
	%可使用leftmargin调整列表环境左边的空白长度 [leftmargin=0em]
	\item
下列有关原子结构和原子核的认识,其中正确的是
 \underlinegap 
。(填选项前的字母)

\fourchoices
{$ \gamma $射线是高速运动的电子流}
{氢原子辐射光子后,其绕核运动的电子动能增大}
{太阳辐射能量的主要来源是太阳中发生的重核裂变}
{$ ^{210}_{83}Bi $ 的半衰期是 $ 5 $ 天,$ 100 $ 克 $ ^{210}_{83}Bi $ 经过 $ 10 $ 天后还剩下 $ 50 $ 克}


 \tk{B} 

\item 
如图,两滑块 $ A $、$ B $ 在光滑水平面上沿同一直线相向运动,滑块 $ A $ 的质量为 $ m $,速度大小为
$ 2 v_{0} $,方向向右,滑块 $ B $ 的质量为 $ 2 \ m $,速度大小为 $ v_{0} $,方向向左,两滑块发生弹性碰撞后的运动状
态是 \underlinegap 。(填选项前的字母)
\begin{figure}[h!]
	\centering
	\includesvg[width=0.23\linewidth]{picture/svg/GZ-3-tiyou-1609}
\end{figure}

\fourchoices
{$ A $ 和 $ B $ 都向左运动}
{$ A $ 和 $ B $ 都向右运动}
{$ A $ 静止,$ B $ 向右运动}
{$ A $ 向左运动,$ B $ 向右运动}

 \tk{D} 
	
\end{enumerate}


\item 
\exwhere{$ 2015 $ 年江苏卷}
\begin{enumerate}
	%\renewcommand{\labelenumi}{\arabic{enumi}.}
	% A(\Alph) a(\alph) I(\Roman) i(\roman) 1(\arabic)
	%设定全局标号series=example	%引用全局变量resume=example
	%[topsep=-0.3em,parsep=-0.3em,itemsep=-0.3em,partopsep=-0.3em]
	%可使用leftmargin调整列表环境左边的空白长度 [leftmargin=0em]
	\item
波粒二象性是微观世界的基本特征,以下说法正确的有 \underlinegap  .
\fourchoices
{光电效应现象揭示了光的粒子性}
{热中子束射到晶体上产生衍射图样说明中子具有波动性}
{黑体辐射的实验规律可用光的波动性解释}
{动能相等的质子和电子,它们的德布罗意波长也相等}


 \tk{AB} 

\item 
核电站利用原子核链式反应放出的巨大能量进行发电, $ ^{235}_{92}U $ 是核电站常用的核燃料. $ ^{235}_{92}U $ 受一
个中子轰击后裂变成$ ^{144}_{56}Ba $ 和 $ ^{89}_{36} K r $ 两部分,并产生 \underlinegap 个中子。 要使链式反应发生,裂变物
质的体积要 \underlinegap  (选填 “大于”或“小于”)它的临界体积。

 \tk{$ 3 $ \quad 大于} 

	
\end{enumerate}


\item 
\exwhere{$ 2015 $ 年理综山东卷}
\begin{enumerate}
	%\renewcommand{\labelenumi}{\arabic{enumi}.}
	% A(\Alph) a(\alph) I(\Roman) i(\roman) 1(\arabic)
	%设定全局标号series=example	%引用全局变量resume=example
	%[topsep=-0.3em,parsep=-0.3em,itemsep=-0.3em,partopsep=-0.3em]
	%可使用leftmargin调整列表环境左边的空白长度 [leftmargin=0em]
	\item
$ ^{14}C $ 发生放射性衰变为 $ ^{14} N $,半衰期约为 $ 5700 $ 年。已知植物存活期间,其体内 $ ^{14}C $ 与 $ ^{12}C $ 的
比例不变;生命活动结束后, $ ^{14}C $ 的比例持续减少。现通过测量得知,某古木样品中 $ ^{14}C $ 的比例正
好是现代植物所制样品的二分之一。下列说法正确的是 \underlinegap 。
(双选,填正确答案标号)
\fourchoices
{该古木的年代距今约为 $ 5700 $ 年}
{$ ^{12}C $、$ ^{13}C $、 $ ^{14}C $ 具有相同的中子数}
{$ ^{14}C $衰变为 $ ^{14}N $ 的过程中放出$ \beta $射线}
{增加样品测量环境的压强将加速 $ ^{14}C $ 的衰变}

 \tk{AC} 


\item 
如图,三个质量相同的滑块 $ A $、$ B $、$ C $,间隔相等地静置于同一水平直轨道上。现给滑块 $ A $ 向
右的初速度 $ v_{0} $,一段时间后 $ A $ 与 $ B $ 发生碰撞,碰
后 $ AB $ 分别以 $  \frac{ 1 }{ 8 } v_{0} $、
$  \frac{ 3 }{ 4 } v_{0} $ 的速度向右运动,$ B $ 再与
$ C $发生碰撞,碰后 $ B $、$ C $ 粘在一起向右运动。滑块 $ A $、$ B $ 与轨道间的动摩擦因数为同一恒定值。两
次碰撞时间极短。求 $ B $、$ C $ 碰后瞬间共同速度的大小。
\begin{figure}[h!]
	\flushright
	\includesvg[width=0.25\linewidth]{picture/svg/GZ-3-tiyou-1610}
\end{figure}

\banswer{
	$v_{ \text{共} }=\frac{\sqrt{21}}{16} v_{0}$
}


	
\end{enumerate}


\item 
\exwhere{$ 2015 $ 年海南卷}
\begin{enumerate}
	%\renewcommand{\labelenumi}{\arabic{enumi}.}
	% A(\Alph) a(\alph) I(\Roman) i(\roman) 1(\arabic)
	%设定全局标号series=example	%引用全局变量resume=example
	%[topsep=-0.3em,parsep=-0.3em,itemsep=-0.3em,partopsep=-0.3em]
	%可使用leftmargin调整列表环境左边的空白长度 [leftmargin=0em]
	\item
氢原子基态的能量为 $ E_{1} =-13.6 \ eV $。大量氢原子处于某一激发态。由这些氢原子可能发
出的所有光子中,频率最大的光子能量为 $ -0.96 E_{1} $,频率最小的光子的能量为
 \underlinegap 
$ eV $(保留 $ 2 $ 位
有效数字),这些光子可具有
 \underlinegap 
种不同的频率。

 \tk{$ 10 \ eV $ \quad  $ 10 $} 



\item 
运动的原子核 $ ^{A}_{Z}X $ 放出$ \alpha $粒子后变成静止的原子核 $ Y $。已知 $ X $、$ Y $ 和$ \alpha $粒子的质量分别
是 $ M $、$ m_{1} $ 和 $ m_{2} $,真空中的光速为 $ c $,$ \alpha $粒子的速度远小于光速。求反应后与反应前的总动能之差以
及$ \alpha $粒子的动能。

\banswer{
	$\Delta E_{k}=\left(M-m_{1}-m_{2}\right) c^{2}, \quad \frac{M}{M-m_{2}}\left(M-m_{1}-m_{2}\right) c^{2}$
}


	
\end{enumerate}


\item 
\exwhere{$ 2014 $ 年理综新课标\lmd{1}卷}
\begin{enumerate}
	%\renewcommand{\labelenumi}{\arabic{enumi}.}
	% A(\Alph) a(\alph) I(\Roman) i(\roman) 1(\arabic)
	%设定全局标号series=example	%引用全局变量resume=example
	%[topsep=-0.3em,parsep=-0.3em,itemsep=-0.3em,partopsep=-0.3em]
	%可使用leftmargin调整列表环境左边的空白长度 [leftmargin=0em]
	\item
关于天然放射性,下列说法正确的是 \underlinegap .
(填写正确答案标号。选对 $ 1 $ 个得 $ 3 $ 分,
选对 $ 2 $ 个得 $ 4 $ 分,选对 $ 3 $ 个得 $ 6 $ 分。每选错 $ 1 $ 个扣 $ 3 $ 分,最低得分为 $ 0 $ 分)
\fivechoices
{所有元素都可能发生衰变}
{放射性元素的半衰期与外界的温度无关}
{放射性元素与别的元素与别的元素形成化合物时仍具有放射性}
{$ \alpha $、$ \beta $和$ \gamma $三种射线中,$ \gamma $射线的穿透能力最强}
{一个原子核在一次衰变中可同时放出$ \alpha $、$ \beta $和$ \gamma $三种射线}


 \tk{BCD} 


\item 
如图,质量分别为 $ m_{A} $、 $ m_{B} $ 的两个小球 $ A $、$ B $ 静止在地面上
方,$ B $ 球距地面的高度 $ h=0.8 \ m $,$ A $ 球在 $ B $ 球的正上方。 先将 $ B $ 球释放,
经过一段时间后再将 $ A $ 球释放。 当 $ A $ 球下落 $ t=0.3 \ s $ 时,刚好与 $ B $ 球在地面上方的 $ P $ 点处相碰,碰
撞时间极短,碰后瞬间 $ A $ 球的速度恰为零。已知 $ m_{B}  =3 \ mA $,重力加速度大小为 $ g=10 \ m/s^{2} $,忽
略空气阻力及碰撞中的动能损失。
\begin{enumerate}
	%\renewcommand{\labelenumi}{\arabic{enumi}.}
	% A(\Alph) a(\alph) I(\Roman) i(\roman) 1(\arabic)
	%设定全局标号series=example	%引用全局变量resume=example
	%[topsep=-0.3em,parsep=-0.3em,itemsep=-0.3em,partopsep=-0.3em]
	%可使用leftmargin调整列表环境左边的空白长度 [leftmargin=0em]
	\item
$ B $ 球第一次到达地面时的速度;
\item 
$ P $ 点距离地面的高度。
	
\end{enumerate}
\begin{figure}[h!]
	\flushright
	\includesvg[width=0.25\linewidth]{picture/svg/GZ-3-tiyou-1611}
\end{figure}

\banswer{
	\begin{enumerate}
		%\renewcommand{\labelenumi}{\arabic{enumi}.}
		% A(\Alph) a(\alph) I(\Roman) i(\roman) 1(\arabic)
		%设定全局标号series=example	%引用全局变量resume=example
		%[topsep=-0.3em,parsep=-0.3em,itemsep=-0.3em,partopsep=-0.3em]
		%可使用leftmargin调整列表环境左边的空白长度 [leftmargin=0em]
		\item
		$ v_{1}=4 \ m/s  $
		\item 
		$ h ^{\prime} =0.75 \ m $	
	\end{enumerate}
}

	
\end{enumerate}



\item 
\exwhere{$ 2014 $ 年理综新课标 \lmd{2} 卷}
\begin{enumerate}
	%\renewcommand{\labelenumi}{\arabic{enumi}.}
	% A(\Alph) a(\alph) I(\Roman) i(\roman) 1(\arabic)
	%设定全局标号series=example	%引用全局变量resume=example
	%[topsep=-0.3em,parsep=-0.3em,itemsep=-0.3em,partopsep=-0.3em]
	%可使用leftmargin调整列表环境左边的空白长度 [leftmargin=0em]
	\item
在人类对微观世界进行探索的过程中,科学实验起到了非常重要的作用。下列说法符合历史
事实的是
 \underlinegap 
(填正确答案标号,选对 $ 1 $ 个给 $ 2 $ 分,选对 $ 2 $ 个得 $ 4 $ 分,选对 $ 3 $ 个得 $ 5 $ 分,每选错 $ 1 $
个扣 $ 3 $ 分,最低得分 $ 0 $ 分)

\fivechoices
{密立根通过油滴实验测得了基本电荷的数值}
{贝克勒尔通过对天然放射性现象的研究,发现了原子中存在原子核}
{居里夫妇从沥青铀矿中分离出了钋($ P_{0} $)和镭$ (Ra) $两种新元素}
{卢瑟福通过а粒子散射实验,证实了在原子核内存在质子}
{汤姆孙通过阴极射线在电场和在磁场中的偏转实验,发现了阴极射线是由带负电的粒子组成,并测出了该粒子的比荷}

 \tk{ACE} 

\item 
利用图 \subref{2014新课标23602a} 所示的装置验证动量守恒定律。在图 \subref{2014新课标23602a} 中,气垫导轨上有 $ A $、$ B $ 两个滑块,
滑块 $ A $ 右侧带有一弹簧片,左侧与打点计时器(图中未画出)的纸带相连;滑块 $ B $ 左侧也带有一
弹簧片,上面固定一遮光片,光电计时器(未
完全画出)可以记录遮光片通过光电门的时
间。 实验测得滑块 $ A $ 质量 $ m_{1} =0.310 \ kg $,滑块 $ B $
的质量 $ m_{2} =0.108 \ kg $,遮光片的宽度 $ d=1.00 \ cm $;
打点计时器所用的交流电的频率为 $ f=50 \ HZ $。将光电门固定在滑块 $ B $ 的右侧,启动打点计时器,给
滑块 $ A $ 一向右的初速度,使它与 $ B $ 相碰;碰后光电计时器显示的时间为 $ \Delta t_{B} =3.500 \ m s $,碰撞前
后打出的纸带如图 \subref{2014新课标23602b} 所示。
\begin{figure}[h!]
	\centering
	\begin{subfigure}{0.4\linewidth}
		\centering
		\includesvg[width=0.7\linewidth]{picture/svg/GZ-3-tiyou-1612} 
		\caption{}\label{2014新课标23602a}
	\end{subfigure}
	\begin{subfigure}{0.4\linewidth}
		\centering
		\includesvg[width=0.7\linewidth]{picture/svg/GZ-3-tiyou-1613} 
		\caption{}\label{2014新课标23602b}
	\end{subfigure}
\end{figure}



若实验允许的相对误差绝对值($ \left|  \frac{ \text{碰撞前后总动量之差} }{ \text{碰前总动量} } \right| \times 100 \% $)
最大为 $ 5 $℅,本实验是否在误差
范围内验证了动量守恒定律?写出运算过程。

\banswer{
	$\delta_{1}=1.7 \%<5 \%$,本实验在允许的误差范围内验证了动量守恒定律。
}

	
\end{enumerate}





\item 
\exwhere{$ 2014 $ 年物理江苏卷}
\begin{enumerate}
	%\renewcommand{\labelenumi}{\arabic{enumi}.}
	% A(\Alph) a(\alph) I(\Roman) i(\roman) 1(\arabic)
	%设定全局标号series=example	%引用全局变量resume=example
	%[topsep=-0.3em,parsep=-0.3em,itemsep=-0.3em,partopsep=-0.3em]
	%可使用leftmargin调整列表环境左边的空白长度 [leftmargin=0em]
	\item
已知钙和钾的截止频率分别为 $ 7.73 \times 10^{14} \ Hz $ 和 $ 5.44 \times 10^{14} \ Hz $,在某种单色光的照射下两种金
属均发生光电效应,比较它们表面逸出的具有最大初动能的光电子,钙逸出的光电子具有较大的 \underlinegap .


\fourchoices
{波长}
{频率}
{能量}
{动量}

 \tk{A} 



\item 
氡 $ 222 $ 是一种天然放射性气体,被吸入后,会对人的呼吸系统造成辐射损伤. 它是世界卫生
组织公布的主要环境致癌物质之一 . 其衰变方程是 $ ^{222}_{86}Rn \rightarrow ^{218}_{84}Po + $ \underlinegap .已知 $ ^{222}_{86}Rn $ 的半衰期约为
$ 3.8 $ 天,则约经过
 \underlinegap 
天,$ 16 \ g $ 的$ ^{222}_{86}Rn $ 衰变后还剩 $ 1 \ g $.


 \tk{$ ^{4}_{2}He $ ($ \alpha $ ) \qquad  $ 15.2 $} 

\item 
 牛顿的《 自然哲学的数学原理》 中记载, $ A $、 $ B $ 两个玻璃球相碰,碰撞后的分离速度和它
们碰撞前的接近速度之比总是约为 $ 15 : 16 $. 分离速度是指碰撞后 $ B $ 对 $ A $ 的速度,接近速度是指碰
撞前 $ A $ 对 $ B $ 的速度. 若上述过程是质量为 $ 2 \ m $ 的玻璃球 $ A $ 以速度 $ v_{0} $ 碰撞质量为 $ m $ 的静止玻璃球$ B $,且为对心碰撞,求碰撞后 $ A $、$ B $ 的速度大小.

\banswer{
	$\frac{17}{48} v_{0}, \quad \frac{31}{24} v_{0}$
}


	
\end{enumerate}

\item 
\exwhere{$ 2014 $ 年理综山东卷}
\begin{enumerate}
	%\renewcommand{\labelenumi}{\arabic{enumi}.}
	% A(\Alph) a(\alph) I(\Roman) i(\roman) 1(\arabic)
	%设定全局标号series=example	%引用全局变量resume=example
	%[topsep=-0.3em,parsep=-0.3em,itemsep=-0.3em,partopsep=-0.3em]
	%可使用leftmargin调整列表环境左边的空白长度 [leftmargin=0em]
	\item
氢原子能级如图,当氢原子从 $ n=3 $ 跃迁到 $ n=2 $ 的能级时,辐射光的波长为 $ 656 \ nm $。以下判
断正确的是 \underlinegap .(双选,填正确答案标号)
\begin{figure}[h!]
	\centering
	\includesvg[width=0.23\linewidth]{picture/svg/GZ-3-tiyou-1614}
\end{figure}

\fourchoices
{氢原子从 $ n=2 $ 跃迁到 $ n=1 $ 的能级时,辐射光的波长大于 $ 656 \ nm $}
{用波长为 $ 325 \ nm $ 的光照射,可使氢原子从 $ n=1 $ 跃迁到 $ n=2 $ 的能级}
{一群处于 $ n=3 $ 能级上的氢原子向低能级跃迁时最多产生 $ 3 $ 种谱线}
{用波长为 $ 633 \ nm $ 的光照射,不能使氢原子从 $ n=2 $ 跃迁到 $ n=3 $ 的能级}


 \tk{CD} 

\item 
如图,光滑水平直轨道上两滑块 $ A $、$ B $ 用橡皮筋连接,$ A $ 的质量为 $ m $。开始时橡皮筋松驰,$ B $
静止,给 $ A $ 向左的初速度 $ v_{0} $。一段时间后,$ B $ 与 $ A $ 同向
运动发生碰撞并粘在一起。碰撞后的共同速度是碰撞前
瞬间 $ A $ 的速度的两倍,也是碰撞前瞬间 $ B $ 的速度的一半。求:
\begin{enumerate}
	%\renewcommand{\labelenumi}{\arabic{enumi}.}
	% A(\Alph) a(\alph) I(\Roman) i(\roman) 1(\arabic)
	%设定全局标号series=example	%引用全局变量resume=example
	%[topsep=-0.3em,parsep=-0.3em,itemsep=-0.3em,partopsep=-0.3em]
	%可使用leftmargin调整列表环境左边的空白长度 [leftmargin=0em]
	\item
$ B $ 的质量;
\item 
碰撞过程中 $ A $、$ B $ 系统机械能的损失。
	
\end{enumerate}
\begin{figure}[h!]
	\flushright
	\includesvg[width=0.25\linewidth]{picture/svg/GZ-3-tiyou-1615}
\end{figure}

\banswer{
	\begin{enumerate}
		%\renewcommand{\labelenumi}{\arabic{enumi}.}
		% A(\Alph) a(\alph) I(\Roman) i(\roman) 1(\arabic)
		%设定全局标号series=example	%引用全局变量resume=example
		%[topsep=-0.3em,parsep=-0.3em,itemsep=-0.3em,partopsep=-0.3em]
		%可使用leftmargin调整列表环境左边的空白长度 [leftmargin=0em]
		\item
		$ m/2 $
		\item 
		$  \frac{ 1 }{ 6 } mv_{0}^{2} $
	\end{enumerate}
}


\end{enumerate}


	
	
	
\end{enumerate}

