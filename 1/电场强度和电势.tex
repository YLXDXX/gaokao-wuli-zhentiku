\bta{第五讲$ \quad $电场强度和电势}

\begin{enumerate} [leftmargin=0em]
\renewcommand{\labelenumi}{\arabic{enumi}.}
% A(\Alph) a(\alph) I(\Roman) i(\roman) 1(\arabic)
%设定全局标号series=example	%引用全局变量resume=example
%[topsep=-0.3em,parsep=-0.3em,itemsep=-0.3em,partopsep=-0.3em]
%可使用leftmargin调整列表环境左边的空白长度 [leftmargin=0em]
\item
\exwhere{$ 2019 $年物理全国\lmd{3}卷}
如图,电荷量分别为$ q $和$ -q $($ q>0 $)的点电荷固定在正方体的两个顶点上,$ a $、$ b $是正方体的另外两个顶点。则 \xzanswer{BC} 
\begin{figure}[h!]
\centering
\includesvg[width=0.19\linewidth]{picture/svg/051}
\end{figure}

\fourchoices
{$ a $点和$ b $点的电势相等}
{$ a $点和$ b $点的电场强度大小相等}
{$ a $点和$ b $点的电场强度方向相同}
{将负电荷从$ a $点移到$ b $点,电势能增加}





\item
\exwhere{$ 2019 $年物理北京卷}
如图所示,$ a $、$ b $两点位于以负点电荷$ - Q $($ Q>0 $)为球心的球面上,$ c $点在球面外,则 \xzanswer{D} 
\begin{figure}[h!]
\centering
\includesvg[width=0.19\linewidth]{picture/svg/052}
\end{figure}





\fourchoices
{$ a $点场强的大小比$ b $点大}
{$ b $点场强大小比$ c $点小}
{$ a $点电势比$ b $点高}
{$ b $点电势比$ c $点低}






\item
\exwhere{$ 2017 $年新课标\lmd{3}卷}
一匀强电场的方向平行于$ xOy $平面,平面内$ a $、$ b $、$ c $三点的位置如图所示,三点的电势分别为$ 10 $ $ V $、$ 17 $ $ V $、$ 26 $ $ V $。下列说法正确的是 \xzanswer{ABD} 


\begin{minipage}[h!]{0.7\linewidth}
\vspace{0.3em}
\fourchoices
{电场强度的大小为$ 2.5 $ $ V/cm $}
{坐标原点处的电势为$ 1 $ $ V $}
{电子在$ a $点的电势能比在$ b $点的低$ 7 $ $ eV $}
{电子从$ b $点运动到$ c $点,电场力做功为$ 9 $ $ eV $}

\vspace{0.3em}
\end{minipage}
\hfill
\begin{minipage}[h!]{0.3\linewidth}
\flushright
\vspace{0.3em}
\includesvg[width=0.9\linewidth]{picture/svg/053}
\vspace{0.3em}
\end{minipage}




\item
\exwhere{$ 2017 $年江苏卷}
在$ x $轴上有两个点电荷$ q_{1} $、$ q_{2} $,其静电场的电势$ \varphi $在$ x $轴上分布如图所示。下列说法正确的有 \xzanswer{AC} 
\begin{figure}[h!]
\centering
\includesvg[width=0.19\linewidth]{picture/svg/054}
\end{figure}


\fourchoices
{$ q_{1} $和$ q_{2} $带有异种电荷}
{$ x_{1} $处的电场强度为零}
{负电荷从$ x_{1} $移到$ x_{2} $,电势能减小}
{负电荷从$ x_{1} $移到$ x_{2} $,受到的电场力增大}





\item
\exwhere{$ 2017 $年新课标\lmd{1}卷}
在一静止点电荷的电场中,任一点的电势$ \varphi $与该点到点电荷的距离$ r $的关系如图所示。电场中四个点$ a $、$ b $、$ c $和$ d $的电场强度大小分别$ E_a $、$ E_b $、$ E_c $和$ E_d $ 。点$ a $到点电荷的距离$ r_a $与点$ a $的电势$ \varphi _a $已在图中用坐标($ r_a $,$ \varphi _a $)标出,其余类推。现将一带正电的试探电荷由$ a $点依次经$ b $、$ c $点移动到$ d $点,在相邻两点间移动的过程中,电场力所做的功分别为$ W_{ab} $、$ W_{bc} $和$ W_{cd} $。下列选项正确的是 \xzanswer{AC} 
\begin{figure}[h!]
\centering
\includesvg[width=0.25\linewidth]{picture/svg/055}
\end{figure}
\fourchoices
{$E _ { a }: E _ { b } = 4: 1$}
{$E _ { c }: E _ { d } = 2: 1$}
{$W _ { a b }: W _ { b c } = 3: 1$}
{$W _ { b c }: W _ { c d } = 1: 3$}





\item
\exwhere{$ 2018 $年海南卷}
如图,$ a $、$ b $、$ c $、$ d $为一边长为$ l $的正方形的顶点。电荷量均为$ q $($ q>0 $)的两个点电荷分别固定在$ a $、$ c $两点,静电力常量为$ k $,不计重力。下列说法正确的是 \xzanswer{AD} 


\begin{minipage}[h!]{0.7\linewidth}
\vspace{0.3em}
\fourchoices
{$ b $点的电场强度大小为$\frac { \sqrt { 2 } k q } { l ^ { 2 } }$}
{过$ b $、$ d $点的直线位于同一等势面上}
{在两点电荷产生的电场中,$ ac $中点的电势最低}
{在$ b $点从静止释放的电子,到达$ d $点时速度为零}

\vspace{0.3em}
\end{minipage}
\hfill
\begin{minipage}[h!]{0.3\linewidth}
\flushright
\vspace{0.3em}
\includesvg[width=0.6\linewidth]{picture/svg/056}
\vspace{0.3em}
\end{minipage}



\item
\exwhere{$ 2018 $年浙江卷($ 4 $月选考)}
一带电粒子仅在电场力作用下从$ A $点开始以$ -v_{0} $做直线运动,其$ v-t $图像如图所示。粒子在$ t_{0} $时刻运动到$ B $点,$ 3 $ $ t_{0} $时刻运动到$ C $点,以下判断正确的是 \xzanswer{C} 
\begin{figure}[h!]
\centering
\includesvg[width=0.23\linewidth]{picture/svg/057}
\end{figure}


\fourchoices
{$ A $、$ B $、$ C $三点的电势关系为$\varphi _ { B } > \varphi _ { A } > \varphi _ { C }$}
{$ A $、$ B $、$ C $三点的场强大小关系为$E _ { C } > E _ { B } > E _ { A }$}
{粒子从$ A $点经$ B $点运动到$ C $点,电势能先增加后减少}
{粒子从$ A $点经$ B $点运动到$ C $点,电场力先做正功后做负功}





\end{enumerate}








