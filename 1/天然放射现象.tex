\bta{天然放射现象}


\begin{enumerate}
	%\renewcommand{\labelenumi}{\arabic{enumi}.}
	% A(\Alph) a(\alph) I(\Roman) i(\roman) 1(\arabic)
	%设定全局标号series=example	%引用全局变量resume=example
	%[topsep=-0.3em,parsep=-0.3em,itemsep=-0.3em,partopsep=-0.3em]
	%可使用leftmargin调整列表环境左边的空白长度 [leftmargin=0em]
	\item
\exwhere{$ 2013 $ 年上海卷}
在一个 \ce{^{238}_{92}U} 原子核衰变为一个 \ce{^{206}_{82}Pb} 原子核的过程中,发生$ \beta $衰变的次数为 \xzanswer{A} 


\fourchoices
{$ 6 $ 次}
{$ 10 $ 次}
{$ 22 $ 次}
{$ 32 $ 次}



\item 
\exwhere{$ 2011 $ 年理综北京卷}
表示放射性元素碘 $ 131 $( \ce{^{131}_{53}I} )$ \beta $衰变的方程是 \xzanswer{B} 

 \fourchoices
 {\ce{_{53}^{131}I \rightarrow  _{51}^{127}Sb + _{2}^{4}He} }
 { \ce{_{53}^{131}I \rightarrow   ^{131}_{54}Xe + _{-1}^{0}e} }
 { \ce{_{53}^{131}I \rightarrow  _{53}^{130}I + _{0}^{1}n} }
 { \ce{_{53}^{131}I \rightarrow  ^{130}_{52}Te + _{1}^{1}H} }
 
 
 \item 
 \exwhere{$ 2012 $ 年理综全国卷}
 \ce{^{235}_{92}U} 
经过 $ m $ 次$ \alpha $衰变和 $ n $ 次$ \beta $衰变 \ce{^{207}_{82}Pb} ,则 \xzanswer{B} 


\fourchoices
{$ m=7 $,$ n=3 $}
{$ m=7 $, $ n=4 $}
{$ m=14 $, $ n=9 $}
{$ m=14 $, $ n=18 $}




\item 
\exwhere{$ 2015 $ 年上海卷}
 \ce{^{232}_{90}Th} ,经过一系列$ \alpha $衰变和$ \beta $衰变后变成 \ce{^{208}_{82}Pb},则 \ce{^{208}_{82}Pb} 比  \ce{^{232}_{90}Th}  少 \xzanswer{A} 


\fourchoices
{$ 16 $ 个中子,$ 8 $ 个质子}
{$ 8 $ 个中子,$ 16 $ 个质子}
{$ 24 $ 个中子,$ 8 $ 个质子}
{$ 8 $ 个中子,$ 24 $ 个质子}

\item 
\exwhere{$ 2015 $ 年理综北京卷}
下列核反应方程中,属于$ \alpha $衰变的是 \xzanswer{B} 


\fourchoices
{ \ce{^{14}N + _{2}^{4}He \rightarrow{ }_{8}^{17}O + _{1}^{1}H} }
{ \ce{_{92}^{238}U \rightarrow{ }_{90}^{234}Th + _{2}^{4}He} }
{ \ce{_{1}^{2}H + _{1}^{3}H \rightarrow{ }_{2}^{4}He + _{0}^{1}n} }
{ \ce{_{90}^{234}Th \rightarrow{ }_{91}^{234}Pa + _{-1}^{0}e} }


\item 
\exwhere{$ 2012 $ 年物理上海卷}
与原子核内部变化有关的现象是 \xzanswer{C} 


\fourchoices
{电离现象}
{光电效应现象}
{天然放射现象}
{$ \alpha $粒子散射现象}



\item 
\exwhere{$ 2011 $ 年理综浙江卷}
关于天然放射现象,下列说法正确的是 \xzanswer{D} 

\fourchoices
{$ \alpha $射线是由氦原子核衰变产生}
{$ \beta $射线是由原子核外电子电离产生}
{$ \gamma $射线是由原子核外的内层电子跃迁产生}
{通过化学反应不能改变物质的放射性}



\item 
\exwhere{$ 2015 $ 年理综重庆卷}
图中曲线 $ a $、$ b $、$ c $、$ d $ 为气泡室中某放射物质发生衰变放出的部分粒子
的径迹,气泡室中磁感应强度方向垂直纸面向里。以下判断可能正确
的是 \xzanswer{D} 
\begin{figure}[h!]
	\centering
	\includesvg[width=0.23\linewidth]{picture/svg/GZ-3-tiyou-1294}
\end{figure}


\fourchoices
{$ a $、$ b $ 为$ \beta $粒子的径迹}
{$ a $、$ b $ 为$ \beta $粒子的径迹}
{$ c $、$ d $ 为$ \alpha $粒子的径迹}
{$ c $、$ d $ 为$ \beta $粒子的径迹}




\item 
\exwhere{$ 2011 $年上海卷}
在存放放射性元素时,若把放射性元素①置于大量水中;②密封于铅盒中;③与轻核元素结合
成化合物。则 \xzanswer{D} 


\fourchoices
{措施①可减缓放射性元素衰变}
{措施②可减缓放射性元素衰变}
{措施③可减缓放射性元素衰变}
{上述措施均无法减缓放射性元素衰变}



\item
\exwhere{$ 2014 $ 年理综重庆卷}
碘 $ 131 $ 的半衰期约为 $ 8 $ 天,若某药物含有质量为 $ m $ 的碘 $ 131 $,经过 $ 32 $ 天后,该药物中碘 $ 131 $ 的
含量大约还有 \xzanswer{C} 

\fourchoices
{$ m/4 $}
{$ m/8 $}
{$ m/16 $}
{$ m/32 $}



\item 
\exwhere{$ 2016 $ 年上海卷}
放射性元素 $ A $ 经过 $ 2 $ 次$ \alpha $衰变和 $ 1 $ 次$ \beta $ 衰变后生成一新元素 $ B $,则元素 $ B $ 在元素
周期表中的位置较元素 $ A $ 的位置向前移动了 \xzanswer{C} 

\fourchoices
{$ 1 $ 位}
{$ 2 $ 位}
{$ 3 $ 位}
{$ 4 $ 位}


\item 
\exwhere{$ 2018 $ 年海南卷}
已知 \ce{^{234}_{90}Th} 的半衰期为 $ 24 $ 天。 $ 4 \ g $ \ce{^{234}_{90}Th} 经过 $ 72 $ 天还剩下  \xzanswer{B} 

\fourchoices
{$ 0 $ }
{$ 0.5 \ g $}
{$ 1 \ g $ }
{$ 1.5 \ g $}


\item 
\exwhere{$ 2016 $ 年上海卷}
研究放射性元素射线性质的实验装置如图所示。两块平行放置的金属板 $ A $、$ B $
分别与电源的两极 $ a $、$ b $ 连接,放射源发出的射线从其上方小孔向外射
出。则 \xzanswer{B} 
\begin{figure}[h!]
	\centering
	\includesvg[width=0.23\linewidth]{picture/svg/GZ-3-tiyou-1295}
\end{figure}


\fourchoices
{$ a $ 为电源正极,到达 $ A $ 板的为$ \alpha $射线}
{$ a $ 为电源正极,到达 $ A $ 板的为$ \beta $射线}
{$ a $ 为电源负极,到达 $ A $ 板的为$ \alpha $射线}
{$ a $ 为电源负极,到达 $ A $ 板的为$ \beta $射线}



\item 
\exwhere{$ 2013 $ 年全国卷大纲卷}
放射性元素氡( \ce{^{222}_{86}Rn} )经$ \alpha $衰变变成钋( \ce{^{218}_{84}Po} ),半衰期约为 $ 3.8 $ 天;但勘测表明,经过漫
长的地质年代后,目前地壳中仍存在天然的含有放射性元素 \ce{^{222}_{86}Rn} 的矿石,其原因是 \xzanswer{A} 


\fourchoices
{目前地壳中的 \ce{^{222}_{86}Rn} 主要来自于其它放射性元素的衰变}
{在地球形成初期,地壳中的元素 \ce{^{222}_{86}Rn} 的含量足够高}
{当衰变产物 \ce{^{218}_{84}Po} 积累到一定量以后,\ce{^{218}_{84}Po} 的增加会减慢 \ce{^{222}_{86}Rn} 的衰变进程}
{\ce{^{222}_{86}Rn} 主要存在于地球深处的矿石中,温度和压力改变了它的半衰期}


\item 
\exwhere{$ 2012 $ 年物理上海卷}
在轧制钢板时需要动态地监测钢板厚度,其检测装置由放射源、探测器等构成,如图所示。该
装置中探测器接收到的是 \xzanswer{D} 
\begin{figure}[h!]
	\centering
	\includesvg[width=0.23\linewidth]{picture/svg/GZ-3-tiyou-1296}
\end{figure}


\fourchoices
{$ X $ 射线}
{$ \alpha $射线}
{$ \beta $ 射线}
{$ \gamma $射线}


\item 
\exwhere{$ 2012 $ 年物理上海卷}
 \ce{^{60}_{27}Co} 发生一次$ \beta $衰变后变为 $ Ni $ 核,其衰变方程为 \hfullline ;\\
 在该衰变过程
中还发出频率为$ \nu _{1} $、$ \nu _{2} $ 的两个光子,其总能量为 \underlinegap 。


 \tk{
 \ce{_{27}^{60}Co \rightarrow{ }_{28}^{60}Ni + _{-1}^{60}e}   \quad $h\left(\nu_{1}+\nu_{2}\right)$
} 

\item 
\exwhere{$ 2011 $年上海卷}
天然放射性元素放出的三种射线的穿透能力实验结果如图所示,由此可推知 \xzanswer{D} 
\begin{figure}[h!]
	\centering
	\includesvg[width=0.23\linewidth]{picture/svg/GZ-3-tiyou-1297}
\end{figure}


\fourchoices
{②来自于原子核外的电子}
{①的电离作用最强,是一种电磁波}
{③的电离作用较强,是一种电磁波}
{③的电离作用最弱,属于原子核内释放的光子}




\item 
\exwhere{$ 2015 $ 年理综北京卷}
实验观察到,静止在匀强磁场中 $ A $ 点的原子核发生$ \beta $衰变,衰变产生的
新核与电子恰在纸面内做匀速圆周运动,运动方向和轨迹示意如图。
则 \xzanswer{D} 
\begin{figure}[h!]
	\centering
	\includesvg[width=0.23\linewidth]{picture/svg/GZ-3-tiyou-1298}
\end{figure}


\fourchoices
{轨迹 $ 1 $ 是电子的,磁场方向垂直纸面向外}
{轨迹 $ 2 $ 是电子的,磁场方向垂直纸面向外}
{轨迹 $ 1 $ 是新核的,磁场方向垂直纸面向里}
{轨迹 $ 2 $ 是新核的,磁场方向垂直纸面向里}



\item 
\exwhere{$ 2017 $ 年新课标 \lmd{2} 卷}
一静止的铀核放出一个$ \alpha $粒子衰变成钍核,衰变方程为
${ }_{92}^{238} \mathrm{U} \rightarrow{ }_{90}^{234} \mathrm{Th}+{ }_{2}^{4} \mathrm{He}$,下列说法正确的是 \xzanswer{B} 


\fourchoices
{衰变后钍核的动能等于 $ \alpha $ 粒子的动能}
{衰变后钍核的动量大小等于$ \alpha $粒子的动量大小}
{铀核的半衰期等于其放出一个$ \alpha $粒子所经历的时间}
{衰变后粒子与钍核的质量之和等于衰变前铀核的质量}



\item 
\exwhere{$ 2017 $ 年北京卷}
在磁感应强度为 $ B $ 的匀强磁场中,一个静止的放射性原子核发生了一次$ \alpha $衰变。放射出$ \alpha $粒子(  \ce{^{4}_{2}H}  )
在与磁场垂直的平面内做圆周运动,其轨道半径为 $ R $。以 $ m $、$ q $ 分别表示$ \alpha $粒子的质量和电荷量。
\begin{enumerate}
	%\renewcommand{\labelenumi}{\arabic{enumi}.}
	% A(\Alph) a(\alph) I(\Roman) i(\roman) 1(\arabic)
	%设定全局标号series=example	%引用全局变量resume=example
	%[topsep=-0.3em,parsep=-0.3em,itemsep=-0.3em,partopsep=-0.3em]
	%可使用leftmargin调整列表环境左边的空白长度 [leftmargin=0em]
	\item
放射性原子核用 \ce{^{A}_{Z}X} 表示,新核的元素符号用 \ce{Y} 表示,写出该$ \alpha $衰变的核反应方程。

\item 
$ \alpha $粒子的圆周运动可以等效成一个环形电流,求圆周运动的周期和环形电流大小。

\item 
设该衰变过程释放的核能都转为为$ \alpha $粒子和新核的动能,新核的质量为 $ M $,求衰变过程的质量
亏损$ \Delta m $。

	
\end{enumerate}


 \tk{
\begin{enumerate}
	%\renewcommand{\labelenumi}{\arabic{enumi}.}
	% A(\Alph) a(\alph) I(\Roman) i(\roman) 1(\arabic)
	%设定全局标号series=example	%引用全局变量resume=example
	%[topsep=-0.3em,parsep=-0.3em,itemsep=-0.3em,partopsep=-0.3em]
	%可使用leftmargin调整列表环境左边的空白长度 [leftmargin=0em]
	\item
 \ce{_{Z}^{A}X \rightarrow ^{A-4}_{Z-2}Y + _{2}^{4}He} 
	\item 
	$I=\frac{q^{2} B}{2 \pi m} \quad T=\frac{2 \pi m}{q B}$
	\item 
	$\Delta m=\frac{(M+m) q^{2} B^{2} R^{2}}{2 m M c^{2}}$
\end{enumerate}
} 





	
	
	
\end{enumerate}

