\bta{安培力}


\begin{enumerate}
\item 
\exwhere{$ 2019 $年$ 4 $月浙江物理选考}
在磁场中的同一位置放置一条直导线,导线的方向与磁场方向垂直,则下列描述导线受到的安培力$ F $的大小与通过导线的电流的关系图象正确的是 \xzanswer{A} 
\pfourchoices
{\includesvg[width=3cm]{picture/svg/GZ-3-tiyou-1213}}
{\includesvg[width=3cm]{picture/svg/GZ-3-tiyou-1214}}
{\includesvg[width=3cm]{picture/svg/GZ-3-tiyou-1215}}
{\includesvg[width=3cm]{picture/svg/GZ-3-tiyou-1216}}





\item
\exwhere{$ 2019 $年物理全国\lmd{1}卷}
如图,等边三角形线框$ LMN $由三根相同的导体棒连接而成,固定于匀强磁场中,线框平面与磁感应强度方向垂直,线框顶点$ M $、$ N $与直流电源两端相接,已如导体棒$ MN $受到的安培力大小为$ F $,则线框$ LMN $受到的安培力的大小为 \xzanswer{B} 
\begin{figure}[h!]
\centering
\includesvg[width=0.23\linewidth]{picture/svg/153}
\end{figure}

\fourchoices
{$ 2F $}
{$ 1.5F $}
{$ 0.5F $}
{$ 0 $}








\item
\exwhere{$ 2019 $年物理江苏卷}
如图所示,在光滑的水平桌面上,$ a $和$ b $是两条固定的平行长直导线,通过的电流强度相等.矩形线框位于两条导线的正中间,通有顺时针方向的电流,在$ a $、$ b $产生的磁场作用下静止.则$ a $、$ b $的电流方向可能是 \xzanswer{CD} 
\begin{figure}[h!]
	\centering
\includesvg[width=0.27\linewidth]{picture/svg/154}
\end{figure}

\fourchoices
{均向左}
{均向右}
{$ a $的向左,$ b $的向右}
{$ a $的向右,$ b $的向左}



\item
\exwhere{$ 2015 $年上海卷}
如图,两根通电长直导线$ a $、$ b $平行放置,$ a $、$ b $中的电流强度分别为$ I $和$ 2I $,此时$ a $受到的磁场力为$ F $,若以该磁场力的方向为正,则$ b $受到的磁场力为 \underlinegap 。当在$ a $、$ b $的正中间再放置一根与$ a $、$ b $平行共面的通电长直导线$ c $后,$ a $受到的磁场力大小变为$ 2F $,则此时$ b $受到的磁场力为 \underlinegap 。
\begin{figure}[h!]
\centering
\includesvg[width=0.23\linewidth]{picture/svg/158}
\end{figure}

\tk{$ -F $ \quad $ -3F $或$ 5F $}

\item
\exwhere{$ 2018 $年浙江卷($ 4 $月选考)}
处于磁场$ B $中的矩形金属线框可绕轴$ OO ^{\prime} $ 转动,当线框中通以电流$ I $时,如图所示,此时线框左右两边受安培力$ F $的方向正确的是 \xzanswer{D} 
\begin{figure}[h!]
\centering
\includesvg[width=0.23\linewidth]{picture/svg/GZ-3-tiyou-1217}
\end{figure}

\pfourchoices
{\includesvg[width=3.5cm]{picture/svg/GZ-3-tiyou-1218}}
{\includesvg[width=3cm]{picture/svg/GZ-3-tiyou-1219}}
{\includesvg[width=3cm]{picture/svg/GZ-3-tiyou-1220}}
{\includesvg[width=3cm]{picture/svg/GZ-3-tiyou-1221}}






\item
\exwhere{$ 2015 $年江苏卷}
如图所示,用天平测量匀强磁场的磁感应强度。 下列各选项所示的载流线圈匝数相同,边长 $ MN $ 相等,将它们分别挂在天平的右臂下方。 线圈中通有大小相同的电流,天平处于平衡状态。 若磁场发生微小变化,天平最容易失去平衡的是 \xzanswer{A} 
\begin{figure}[h!]
\centering
\includesvg[width=0.18\linewidth]{picture/svg/159}
\end{figure}

\pfourchoices
{\includesvg[width=3cm]{picture/svg/GZ-3-tiyou-1222}}
{\includesvg[width=3cm]{picture/svg/GZ-3-tiyou-1223}}
{\includesvg[width=3cm]{picture/svg/GZ-3-tiyou-1224}}
{\includesvg[width=3cm]{picture/svg/GZ-3-tiyou-1225}}



\item
\exwhere{$ 2011 $年新课标版}
电磁轨道炮工作原理如图所示。待发射弹体可在两平行轨道之间自由移动,并与轨道保持良好接触。电流$ I $从一条轨道流入,通过导电弹体后从另一条轨道流回。轨道电流可形成在弹体处垂直于轨道面的磁场(可视为匀强磁场),磁感应强度的大小与$ I $成正比。通电的弹体在轨道上受到安培力的作用而高速射出。现欲使弹体的出射速度增加至原来的$ 2 $倍,理论上可采用的办法是 \xzanswer{BD} 
\begin{figure}[h!]
\centering
\includesvg[width=0.23\linewidth]{picture/svg/161}
\end{figure}


\fourchoices
{只将轨道长度$ L $变为原来的$ 2 $倍}
{只将电流$ I $增加至原来的$ 2 $倍}
{只将弹体质量减至原来的一半}
{将弹体质量减至原来的一半,轨道长度$ L $变为原来的$ 2 $倍,其它量不变}





\item
\exwhere{$ 2012 $年物理海南卷}
图中装置可演示磁场对通电导线的作用。电磁铁上下两磁极之间某一水平面内固定两条平行金属导轨,$ L $是置于导轨上并与导轨垂直的金属杆。当电磁铁线圈两端$ a $、$ b $,导轨两端$ e $、$ f $,分别接到两个不同的直流电源上时,$ L $便在导轨上滑动。下列说法正确的是 \xzanswer{BD} 
\begin{figure}[h!]
\centering
\includesvg[width=0.23\linewidth]{picture/svg/162}
\end{figure}


\fourchoices
{若$ a $接正极,$ b $接负极,$ e $接正极,$ f $接负极,则$ L $向右滑动}
{若$ a $接正极,$ b $接负极,$ e $接负极,$ f $接正极,则$ L $向右滑动}
{若$ a $接负极,$ b $接正极,$ e $接正极,$ f $接负极,则$ L $向左滑动}
{若$ a $接负极,$ b $接正极,$ e $接负极,$ f $接正极,则$ L $向左滑动}




\item
\exwhere{$ 2012 $年理综天津卷}
如图所示,金属棒$ MN $两端由等长的轻质细线水平悬挂,处于竖直向上的匀强磁场中,棒中通以由$ M $向$ N $的电流,平衡时两悬线与竖直方向夹角均为$ \theta $,如果仅改变下列某一个条件,$ \theta $角的相应变化情况是
\xzanswer{A} 
\begin{figure}[h!]
\centering
\includesvg[width=0.23\linewidth]{picture/svg/163}
\end{figure}

\fourchoices
{棒中的电流变大,$	\theta$角变大}
{两悬线等长变短,$	\theta$角变小}
{金属棒质量变大,$	\theta$角变大}
{磁感应强度变大,$	\theta$角变小}





\item
\exwhere{$ 2017 $年新课标\lmd{1}卷}
如图,三根相互平行的固定长直导线$ L_{1} $、$ L_{2} $和$ L_{3} $两两等距,均通有电流$ I $,$ L_{1} $中电流方向与$ L_{2} $中的相同,与$ L_{3} $中的相反,下列说法正确的是 \xzanswer{BC} 
\begin{figure}[h!]
\centering
\includesvg[width=0.23\linewidth]{picture/svg/164}
\end{figure}



\fourchoices
{$ L_{1} $所受磁场作用力的方向与$ L_{2} $、$ L_{3} $所在平面垂直}
{$ L_{3} $所受磁场作用力的方向与$ L_{1} $、$ L_{2} $所在平面垂直}
{$ L_{1} $、$ L_{2} $和$ L_{3} $单位长度所受的磁场作用力大小之比为$1: 1: \sqrt { 3 }$}
{$ L_{1} $、$ L_{2} $和$ L_{3} $单位长度所受的磁场作用力大小之比为$\sqrt { 3 }: \sqrt { 3 }: 1$}





\item
\exwhere{$ 2017 $年浙江选考卷}
如图所示,两平行直导线$ cd $和$ ef $竖直放置,通以方向相反大小相等的电流,$ a $、$ b $两点位于两导线所在的平面内。则 \xzanswer{D} 
\begin{figure}[h!]
	\centering
\includesvg[width=0.15\linewidth]{picture/svg/165}
\end{figure}


\fourchoices
{$ b $点的磁感应强度为零}
{$ ef $导线在$ a $点产生的磁场方向垂直纸面向里}
{$ cd $导线受到的安培力方向向右}
{同时改变了导线的电流方向,$ cd $导线受到的安培力方向不变}







\item
\exwhere{$ 2014 $年物理上海卷}
如图,在磁感应强度为$ B $的匀强磁场中,面积为$ s $的矩形刚性导线框$ abcd $可绕过$ ad $边的固定轴$ OO ^{\prime} $ 转动,磁场方向与线框平面垂直。在线框中通以电流强度为$ I $的稳恒电流,并使线框与竖直平面成$ \theta $角,此时$ bc $边受到相对$ OO ^{\prime} $ 轴的安培力力矩大小为 \xzanswer{A} 
\begin{figure}[h!]
\centering
\includesvg[width=0.23\linewidth]{picture/svg/166}
\end{figure}

\fourchoices
{$I S B \sin \theta$}
{$I S B \cos \theta$}
{$\frac { I S B } { \sin \theta }$}
{$\frac { I S B } { \cos \theta }$}






\item
\exwhere{$ 2011 $年物理江苏卷}
如图所示,水平面内有一平行金属导轨,导轨光滑且电阻不计。匀强磁场与导轨平面垂直。阻值为$ R $的导体棒垂直于导轨静止放置,且与导轨接触良好。$ t=0 $时,将开关$ S $由$ 1 $掷到$ 2 $。$ q $、$ i $、$ v $和$ a $分别表示电容器所带的电荷量、棒中的电流、棒的速度和加速度。下列图象正确的是 \xzanswer{D} 
\begin{figure}[h!]
\centering
\includesvg[width=0.27\linewidth]{picture/svg/167}
\end{figure}

\pfourchoices
{\includesvg[width=3cm]{picture/svg/GZ-3-tiyou-1226}}
{\includesvg[width=3cm]{picture/svg/GZ-3-tiyou-1227}}
{\includesvg[width=3cm]{picture/svg/GZ-3-tiyou-1228}}
{\includesvg[width=3cm]{picture/svg/GZ-3-tiyou-1229}}






\item
\exwhere{$ 2014 $年理综浙江卷}
如图$ 1 $所示,两根光滑平行导轨水平放置,间距为$ L $,其间有竖直向下的匀强磁场,磁感应强度为$ B $。垂直于导轨水平对称放置一根均匀金属棒。从$ t=0 $时刻起,棒上有如图$ 2 $所示的持续交流电流$ I $,周期为$ T $,最大值为$ I_m $,图$ 1 $中$ I $所示方向为电流正方向。则金属棒
\xzanswer{ABC} 
\begin{figure}[h!]
\centering
\begin{subfigure}{0.4\linewidth}
	\centering
	\includesvg[width=0.7\linewidth]{picture/svg/GZ-3-tiyou-1230} 
	\caption{}\label{}
\end{subfigure}
\begin{subfigure}{0.4\linewidth}
	\centering
	\includesvg[width=0.7\linewidth]{picture/svg/GZ-3-tiyou-1231} 
	\caption{}\label{}
\end{subfigure}
\end{figure}



\fourchoices
{一直向右移动}
{速度随时间周期性变化}
{受到的安培力随时间周期性变化}
{受到的安培力在一个周期内做正功}





\item
\exwhere{$ 2011 $年上海卷}
如图,质量为$ m $、长为$ L $的直导线用两绝缘细线悬挂于$ O $、$ O ^{\prime} $,并处于匀强磁场中。当导线中通以沿$ x $正方向的电流$ I $,且导线保持静止时,悬线与竖直方向夹角为$ \theta $。则磁感应强度方向和大小可能为 \xzanswer{BC} 
\begin{figure}[h!]
	\centering
\includesvg[width=0.22\linewidth]{picture/svg/170}
\end{figure}



\fourchoices
{$ z $ 正向,$\frac { m g } { I L } \tan \theta$}
{$ y $ 正向,$\frac { m g } { I L }$}
{$ z $ 负向,$\frac { m g } { I L } \tan \theta$}
{沿悬线向上,$\frac { m g } { I L } \sin \theta$}







\item
\exwhere{$ 2014 $年理综新课标\lmd{1}卷}
关于通电直导线在匀强磁场中所受的安培力,下列说法正确的是 \xzanswer{B} 


\fourchoices
{安培力的方向可以不垂直于直导线}
{安培力的方向总是垂直于磁场的方向}
{安培力的大小与通电直导线和磁场方向的夹角无关}
{将直导线从中点折成直角,安培力的大小一定变为原来的一半}




\item
\exwhere{$ 2016 $年海南卷}
如图($ a $)所示,扬声器中有一线圈处于磁场中,当音频电流信号通过线圈时,线圈带动纸盆振动,发出声音。俯视图($ b $)表示处于辐射状磁场中的线圈(线圈平面即纸面)磁场方向如图中箭头所示,在图($ b $)中 \xzanswer{BC} 
\begin{figure}[h!]
\centering
\begin{subfigure}{0.4\linewidth}
	\centering
	\includesvg[width=0.7\linewidth]{picture/svg/GZ-3-tiyou-1234} 
	\caption{}\label{}
\end{subfigure}
\begin{subfigure}{0.4\linewidth}
	\centering
	\includesvg[width=0.7\linewidth]{picture/svg/GZ-3-tiyou-1235} 
	\caption{}\label{}
\end{subfigure}
\end{figure}


\fourchoices
{当电流沿顺时针方向时,线圈所受安培力的方向垂直于纸面向里}
{当电流沿顺时针方向时,线圈所受安培力的方向垂直于纸面向外}
{当电流沿逆时针方向时,线圈所受安培力的方向垂直于纸面向里}
{当电流沿逆时针方向时,线圈所受安培力的方向垂直于纸面向外}





\item
\exwhere{$ 2018 $年海南卷}
如图,一绝缘光滑固定斜面处于匀强磁场中,磁场的磁感应强度大小为$ B $,方向垂直于斜面向上,通有电流$ I $的金属细杆水平静止在斜面上。若电流变为$ 0.5I $,磁感应强度大小变为$ 3B $,电流和磁场的方向均不变,则金属细杆将 \xzanswer{A} 
\begin{figure}[h!]
	\centering
\includesvg[width=0.15\linewidth]{picture/svg/172}
\end{figure}


\fourchoices
{沿斜面加速上滑}
{沿斜面加速下滑}
{沿斜面匀速上滑}
{仍静止在斜面上 }




\item
\exwhere{$ 2017 $年新课标\lmd{2}卷}
某同学自制的简易电动机示意图如图所示。矩形线圈由一根漆包线绕制而成,漆包线的两端分别从线圈的一组对边的中间位置引出,并作为线圈的转轴。将线圈架在两个金属支架之间,线圈平面位于竖直面内,永磁铁置于线圈下方。为了使电池与两金属支架连接后线圈能连续转动起来,该同学应将 \xzanswer{AD} 
\begin{figure}[h!]
\centering
\includesvg[width=0.27\linewidth]{picture/svg/GZ-3-tiyou-1237}
\end{figure}


\fourchoices
{左、右转轴下侧的绝缘漆都刮掉}
{左、右转轴上下两侧的绝缘漆都刮掉}
{左转轴上侧的绝缘漆刮掉,右转轴下侧的绝缘漆刮掉}
{左转轴上下两侧的绝缘漆都刮掉,右转轴下侧的绝缘漆刮掉}




\newpage
\item
\exwhere{$ 2018 $年江苏卷}
如图所示,两条平行的光滑金属导轨所在平面与水平面的夹角为$ \theta $,间距为$ d $。导轨处于匀强磁场中,磁感应强度大小为$ B $,方向与导轨平面垂直。质量为$ m $的金属棒被固定在导轨上,距底端的距离为$ s $,导轨与外接电源相连,使金属棒通有电流。金属棒被松开后,以加速度$ a $沿导轨匀加速下滑,金属棒中的电流始终保持恒定,重力加速度为$ g $。求下滑到底端的过程中,金属棒:
\begin{enumerate}
\renewcommand{\labelenumi}{\arabic{enumi}.}
% A(\Alph) a(\alph) I(\Roman) i(\roman) 1(\arabic)
%设定全局标号series=example	%引用全局变量resume=example
%[topsep=-0.3em,parsep=-0.3em,itemsep=-0.3em,partopsep=-0.3em]
%可使用leftmargin调整列表环境左边的空白长度 [leftmargin=0em]
\item
末速度的大小$ v $;
\item 
通过的电流大小$ I $;
\item 
通过的电荷量$ Q $。

\end{enumerate}
\begin{figure}[h!]
\flushright
\includesvg[width=0.35\linewidth]{picture/svg/173}
\end{figure}

\banswer{
\begin{enumerate}
\renewcommand{\labelenumi}{\arabic{enumi}.}
% A(\Alph) a(\alph) I(\Roman) i(\roman) 1(\arabic)
%设定全局标号series=example	%引用全局变量resume=example
%[topsep=-0.3em,parsep=-0.3em,itemsep=-0.3em,partopsep=-0.3em]
%可使用leftmargin调整列表环境左边的空白长度 [leftmargin=0em]
\item
$v = \sqrt { 2 a s }$
\item 
$I = \frac { m ( \operatorname { gsin } \theta - a ) } { d B }$
\item 
$Q = \frac { \sqrt { 2 a s m ( g \sin \theta - a ) } } { d B a }$


\end{enumerate}
}



\newpage
\item
\exwhere{$ 2013 $年重庆卷}
小明在研究性学习中设计了一种可测量磁感应强度的实验,其装置如图所示。在该实验中,磁铁固定在水平放置的电子测力计上,此时电子测力计的计数为$ G_{1} $,磁铁两极之间的磁场可视为水平匀强磁场,其余区域磁场不计。直铜条$ AB $的两端通过导线与一电阻连接成闭合回路,总阻值为$ R $。若让铜条水平且垂直于磁场,以恒定的速率$ v $在磁场中竖直向下运动,这时电子测力计的读数为$ G_{2} $,铜条在磁场中的长度$ L $。
\begin{enumerate}
\renewcommand{\labelenumi}{\arabic{enumi}.}
% A(\Alph) a(\alph) I(\Roman) i(\roman) 1(\arabic)
%设定全局标号series=example	%引用全局变量resume=example
%[topsep=-0.3em,parsep=-0.3em,itemsep=-0.3em,partopsep=-0.3em]
%可使用leftmargin调整列表环境左边的空白长度 [leftmargin=0em]
\item
判断铜条所受安培力的方向,$ G_{1} $和$ G_{2} $哪个大?
\item 
求铜条匀速运动时所受安培力的大小和磁感应强度的大小。



\end{enumerate}
\begin{figure}[h!]
\flushright
\includesvg[width=0.35\linewidth]{picture/svg/GZ-3-tiyou-1236}
\end{figure}



\banswer{
\begin{enumerate}
\renewcommand{\labelenumi}{\arabic{enumi}.}
% A(\Alph) a(\alph) I(\Roman) i(\roman) 1(\arabic)
%设定全局标号series=example	%引用全局变量resume=example
%[topsep=-0.3em,parsep=-0.3em,itemsep=-0.3em,partopsep=-0.3em]
%可使用leftmargin调整列表环境左边的空白长度 [leftmargin=0em]
\item
由右手定则和左手定则知安培力方向向上,且$ G_2>G_1 $。
\item 
安培力的大小$F = G _ { 2 } - G _ { 1 }$,
$B = \frac { 1 } { L } \sqrt { \frac { \left( G _ { 2 } - G _ { 1 } \right) R } { v } }$

\end{enumerate}
}





\newpage
\item
\exwhere{$ 2012 $年物理上海卷}
载流长直导线周围磁场的磁感应强度大小为$ B=kI/r $,式中常量$ k>0 $,$ I $为电流强度,$ r $为距导线的距离。在水平长直导线$ MN $正下方,矩形线圈$ abcd $通以逆时针方向的恒定电流,被两根等长的轻质绝缘细线静止地悬挂,如图所示。开始时$ MN $内不通电流,此时两细线内的张力均为$ T_{0} $。当$ MN $通以强度为$ I_{1} $的电流时,两细线内的张力均减小为$ T_{1} $:当$ MN $内的电流强度变为$ I_{2} $时,两细线的张力均大于$ T_{0} $。
\begin{enumerate}
\renewcommand{\labelenumi}{\arabic{enumi}.}
% A(\Alph) a(\alph) I(\Roman) i(\roman) 1(\arabic)
%设定全局标号series=example	%引用全局变量resume=example
%[topsep=-0.3em,parsep=-0.3em,itemsep=-0.3em,partopsep=-0.3em]
%可使用leftmargin调整列表环境左边的空白长度 [leftmargin=0em]
\item
分别指出强度为$ I_{1} $、$ I_{2} $的电流的方向;
\item 
求$ MN $分别通以强度为$ I_{1} $和$ I_{2} $电流时,线框受到的安培力$ F_{1} $与$ F_{2} $大小之比;
\item 
当$ MN $内的电流强度为$ I_{3} $时两细线恰好断裂,在此瞬间线圈的加速度大小为$ a $,求$ I_{3} $。


\end{enumerate}
\begin{figure}[h!]
\flushright
\includesvg[width=0.25\linewidth]{picture/svg/176}
\end{figure}


\banswer{
\begin{enumerate}
\renewcommand{\labelenumi}{\arabic{enumi}.}
% A(\Alph) a(\alph) I(\Roman) i(\roman) 1(\arabic)
%设定全局标号series=example	%引用全局变量resume=example
%[topsep=-0.3em,parsep=-0.3em,itemsep=-0.3em,partopsep=-0.3em]
%可使用leftmargin调整列表环境左边的空白长度 [leftmargin=0em]
\item
$ I_{1} $方向向左,$ I_{2} $方向向右;
\item 
$F _ { 1 }: F _ { 2 } = I _ { 1 }: I _ { 2 }$;
\item 
$I _ { 3 } = \frac { T _ { 0 } ( a - g ) } { \left( T _ { 0 } - T _ { 1 } \right) g } I _ { 1 }$.



\end{enumerate}
}




\newpage
\item
\exwhere{$ 2015 $年理综新课标\lmd{1}卷}
$ (12 $分) 如图,一长为$ 10 \ cm $的金属棒$ ab $用两个完全相同的弹簧水平地悬挂在匀强磁场中;磁场的磁感应强度大小为$ 0.1 \ T $,方向垂直于纸面向里;弹簧上端固定,下端与金属棒绝缘,金属棒通过开关与一电动势为$ 12 \ V $的电池相连,电路总电阻为$ 2 \Omega $。已知开关断开时两弹簧的伸长量均为$ 0.5 \ cm $;闭合开关,系统重新平衡后,两弹簧的伸长量与开关断开时相比均改变了$ 0.3 \ cm $,重力加速度大小取$ 10 \ m/s ^{2} $。判断开关闭合后金属棒所受安培力的方向,并求出金属棒的质量。
\begin{figure}[h!]
\flushright
\includesvg[width=0.25\linewidth]{picture/svg/178}
\end{figure}

\banswer{
依题意,开关闭合后,电流方向从$ b $到$ a $,由左手定则可知,金属棒所受的安培力方向竖直向下。$ m=0.01 \ kg $ 
}








\newpage
\item
\exwhere{$ 2015 $年理综重庆卷}
$ (15 $分)音圈电机是一种应用于硬盘、光驱等系统的特殊电动机。下图是某音圈电机的原理示意图,它由一对正对的磁极和一个正方形刚性线圈构成,线圈边长为$ L $,匝数为$ n $,磁极正对区域内的磁感应强度方向垂直于线圈平面竖直向下,大小为$ B $,区域外的磁场忽略不计。线圈左边始终在磁场外,右边始终在磁场内,前后两边在磁场内的长度始终相等。某时刻线圈中电流从$ P $流向$ Q $,大小为$ I $.
\begin{enumerate}
\renewcommand{\labelenumi}{\arabic{enumi}.}
% A(\Alph) a(\alph) I(\Roman) i(\roman) 1(\arabic)
%设定全局标号series=example	%引用全局变量resume=example
%[topsep=-0.3em,parsep=-0.3em,itemsep=-0.3em,partopsep=-0.3em]
%可使用leftmargin调整列表环境左边的空白长度 [leftmargin=0em]
\item
求此时线圈所受安培力的大小和方向。
\item 
若此时线圈水平向右运动的速度大小为$ v $,求安培力的功率.



\end{enumerate}
\begin{figure}[h!]
\flushright
\includesvg[width=0.25\linewidth]{picture/svg/GZ-3-tiyou-1238}
\end{figure}



\banswer{
\begin{enumerate}
\renewcommand{\labelenumi}{\arabic{enumi}.}
% A(\Alph) a(\alph) I(\Roman) i(\roman) 1(\arabic)
%设定全局标号series=example	%引用全局变量resume=example
%[topsep=-0.3em,parsep=-0.3em,itemsep=-0.3em,partopsep=-0.3em]
%可使用leftmargin调整列表环境左边的空白长度 [leftmargin=0em]
\item
$ F=nBIL $
由左手定则得安培力方向水平向右。
\item 
安培力的功率为$ P=Fv=nBILv $



\end{enumerate}
}



\newpage
\item
\exwhere{$ 2015 $年理综浙江卷}
小明同学设计了一个“电磁天平”,如图$ 1 $所示,等臂天平的左臂为挂盘,右臂挂有矩形线圈,两臂平衡。线圈的水平边长$ L=0.1\ m $,竖直边长$ H=0.3\ m $,匝数为$ N_{1} $。线圈的下边处于匀强磁场内,磁感应强度$ B_0= $ $ 1.0 \ T $,方向垂直线圈平面向里。线圈中通有可在$ 0 \sim 2.0 \ A $范围内调节的电流$ I $。挂盘放上待测物体后,调节线圈中电流使得天平平衡,测出电流即可测得物体的质量。(重力加速度取$ 10 \ m/s ^{2} $)
\begin{enumerate}
\renewcommand{\labelenumi}{\arabic{enumi}.}
% A(\Alph) a(\alph) I(\Roman) i(\roman) 1(\arabic)
%设定全局标号series=example	%引用全局变量resume=example
%[topsep=-0.3em,parsep=-0.3em,itemsep=-0.3em,partopsep=-0.3em]
%可使用leftmargin调整列表环境左边的空白长度 [leftmargin=0em]
\item
为使电磁天平的量程达到$ 0.5 \ kg $,线圈的匝数$ N_{1} $至少为多少?
\item 
进一步探究电磁感应现象,另选$ N_{2} =100 $ 匝、形状相同的线圈,总电阻$ R=10 \Omega $,不接外电流,两臂平衡,如图$ 2 $所示。保持$ B_{0} $不变,在线圈上部另加垂直纸面向外的匀强磁场,且磁感应强度$ B $随时间均匀变大,磁场区域宽度$ d $ $ =0.1m $。当挂盘中放质量为$ 0.01 \ kg $的物体时,天平平衡,求此时磁感应强度的变化率$ \frac{\Delta B}{\Delta t} $。



\end{enumerate}
\begin{figure}[h!]
\flushright
\begin{subfigure}{0.4\linewidth}
	\centering
	\includesvg[width=0.7\linewidth]{picture/svg/GZ-3-tiyou-1239} 
	\caption{}\label{}
\end{subfigure}
\begin{subfigure}{0.4\linewidth}
	\centering
	\includesvg[width=0.7\linewidth]{picture/svg/GZ-3-tiyou-1240} 
	\caption{}\label{}
\end{subfigure}
\end{figure}




\banswer{
\begin{enumerate}
\renewcommand{\labelenumi}{\arabic{enumi}.}
% A(\Alph) a(\alph) I(\Roman) i(\roman) 1(\arabic)
%设定全局标号series=example	%引用全局变量resume=example
%[topsep=-0.3em,parsep=-0.3em,itemsep=-0.3em,partopsep=-0.3em]
%可使用leftmargin调整列表环境左边的空白长度 [leftmargin=0em]
\item
$ N_1=25 $匝
\item 
$ \frac{\Delta B}{\Delta t}=0.1 \ T/s $



\end{enumerate}
}




\newpage
\item
\exwhere{$ 2014 $年理综重庆卷}
某电子天平原理如图所示,$ E $形磁铁的两侧为$ N $极,中心为$ S $极,两极间的磁感应强度大小均为$ B $,磁极宽度均为$ L $,忽略边缘效应。一正方形线圈套于中心磁极,其骨架与秤盘连为一体,线圈两端$ C $、$ D $与外电路连接。当质量为$ m $的重物放在秤盘上时,弹簧被压缩,秤盘和线圈一起向下运动(骨架与磁极不接触)随后外电路对线圈供电,秤盘和线圈恢复到未放重物时的位置并静止,由此时对应的供电电流$ I $可确定重物的质量。已知线圈匝数为$ n $,线圈电阻为$ R $,重力加速度为$ g $。问:
\begin{enumerate}
\renewcommand{\labelenumi}{\arabic{enumi}.}
% A(\Alph) a(\alph) I(\Roman) i(\roman) 1(\arabic)
%设定全局标号series=example	%引用全局变量resume=example
%[topsep=-0.3em,parsep=-0.3em,itemsep=-0.3em,partopsep=-0.3em]
%可使用leftmargin调整列表环境左边的空白长度 [leftmargin=0em]
\item
线圈向下运动过程中,线圈中感应电流是从$ C $端还是从$ D $端流出?
\item 
供电电流$ I $是从$ C $端还是从$ D $端流入?求重物质量与电流的关系。
\item 
若线圈消耗的最大功率为$ P $,该电子天平能称量的最大质量是多少?



\end{enumerate}
\begin{figure}[h!]
\flushright
 \includesvg[width=0.3\linewidth]{picture/svg/GZ-3-tiyou-1241} 
\end{figure}


\banswer{
\begin{enumerate}
\renewcommand{\labelenumi}{\arabic{enumi}.}
% A(\Alph) a(\alph) I(\Roman) i(\roman) 1(\arabic)
%设定全局标号series=example	%引用全局变量resume=example
%[topsep=-0.3em,parsep=-0.3em,itemsep=-0.3em,partopsep=-0.3em]
%可使用leftmargin调整列表环境左边的空白长度 [leftmargin=0em]
\item
感应电流从$ C $端流出
\item 
外加电流从$ D $点流入$m = \frac { 2 n B L } { g } I$
\item 
$\frac { 2 n B L } { g } \sqrt { \frac { P } { R } }$



\end{enumerate}
}







\newpage
\item
\exwhere{$ 2018 $年天津卷}
真空管道超高速列车的动力系统是一种将电能直接转换成平动动能的装置。图$ 1 $是某种动力系统的简化模型,图中粗实线表示固定在水平面上间距为$ l $的两条平行光滑金属导轨,电阻忽略不计,$ ab $和$ cd $是两根与导轨垂直,长度均为$ l $,电阻均为$ R $的金属棒,通过绝缘材料固定在列车底部,并与导轨良好接触,其间距也为$ l $,列车的总质量为$ m $。列车启动前,$ ab $、$ cd $处于磁感应强度为$ B $的匀强磁场中,磁场方向垂直于导轨平面向下,如图$ 1 $所示,为使列车启动,需在$ M $、$ N $间连接电动势为$ E $的直流电源,电源内阻及导线电阻忽略不计,列车启动后电源自动关闭。
\begin{enumerate}
\renewcommand{\labelenumi}{\arabic{enumi}.}
% A(\Alph) a(\alph) I(\Roman) i(\roman) 1(\arabic)
%设定全局标号series=example	%引用全局变量resume=example
%[topsep=-0.3em,parsep=-0.3em,itemsep=-0.3em,partopsep=-0.3em]
%可使用leftmargin调整列表环境左边的空白长度 [leftmargin=0em]
\item
要使列车向右运行,启动时图$ 1 $中$ M $、$ N $哪个接电源正极,并简要说明理由;
\item 
求刚接通电源时列车加速度$ a $的大小;
\item 
列车减速时,需在前方设置如图$ 2 $所示的一系列磁感应强度为$ B $的匀强磁场区域,磁场宽度和相邻磁场间距均大于$ l $。若某时刻列车的速度为$ v_{0} $,此时$ ab $、$ cd $均在无磁场区域,试讨论:要使列车停下来,前方至少需要多少块这样的有界磁场?



\end{enumerate}
\begin{figure}[h!]
\flushright
\begin{subfigure}{0.4\linewidth}
	\centering
	\includesvg[width=0.7\linewidth]{picture/svg/GZ-3-tiyou-1242} 
	\caption{}\label{}
\end{subfigure}
\begin{subfigure}{0.4\linewidth}
	\centering
	\includesvg[width=0.7\linewidth]{picture/svg/GZ-3-tiyou-1243} 
	\caption{}\label{}
\end{subfigure}
\end{figure}


\banswer{
\begin{enumerate}
\renewcommand{\labelenumi}{\arabic{enumi}.}
% A(\Alph) a(\alph) I(\Roman) i(\roman) 1(\arabic)
%设定全局标号series=example	%引用全局变量resume=example
%[topsep=-0.3em,parsep=-0.3em,itemsep=-0.3em,partopsep=-0.3em]
%可使用leftmargin调整列表环境左边的空白长度 [leftmargin=0em]
\item
$ M $接电源正极,列车要向右运动,安培力方向应向右,根据左手定则,接通电源后,金属棒中电流方向由$ a $到$ b $,由$ c $到$ d $,故$ M $接电源正极。
\item 
$a = \frac { 2 B E l } { m R }$
\item 
讨论:若$\frac { m v _ { 0 } R } { B ^ { 2 } l ^ { 3 } }$恰好为整数,设其为$ n $,则需设置$ n $块有界磁场,若$ \frac{I_{\text{总}}}{I_{0}} $不是整数,设$ \frac{I_{\text{总}}}{I_{0}} $的整数部分为$ N $,则需设置$ N+1 $块有界磁场。



\end{enumerate}
}


\newpage
\item
\exwhere{$ 2017 $年天津卷}
电磁轨道炮利用电流和磁场的作用使炮弹获得超高速度,其原理可用来研制新武器和航天运载器。电磁轨道炮示意如图,图中直流电源电动势为$ E $,电容器的电容为$ C $。两根固定于水平面内的光滑平行金属导轨间距为$ l $,电阻不计。炮弹可视为一质量为$ m $、电阻为$ R $的金属棒$ MN $,垂直放在两导轨间处于静止状态,并与导轨良好接触。首先开关$ S $接$ 1 $,使电容器完全充电。然后将$ S $接至$ 2 $,导轨间存在垂直于导轨平面、磁感应强度大小为$ B $的匀强磁场(图中未画出),$ MN $开始向右加速运动。当$ MN $上的感应电动势与电容器两极板间的电压相等时,回路中电流为零,$ MN $达到最大速度,之后离开导轨。问:
\begin{enumerate}
\renewcommand{\labelenumi}{\arabic{enumi}.}
% A(\Alph) a(\alph) I(\Roman) i(\roman) 1(\arabic)
%设定全局标号series=example	%引用全局变量resume=example
%[topsep=-0.3em,parsep=-0.3em,itemsep=-0.3em,partopsep=-0.3em]
%可使用leftmargin调整列表环境左边的空白长度 [leftmargin=0em]
\item
磁场的方向;
\item 
$ MN $刚开始运动时加速度$ a $的大小;
\item 
$ MN $离开导轨后电容器上剩余的电荷量$ Q $是多少。



\end{enumerate}
\begin{figure}[h!]
\flushright
\includesvg[width=0.65\linewidth]{picture/svg/184}
\end{figure}


\banswer{
\begin{enumerate}
\renewcommand{\labelenumi}{\arabic{enumi}.}
% A(\Alph) a(\alph) I(\Roman) i(\roman) 1(\arabic)
%设定全局标号series=example	%引用全局变量resume=example
%[topsep=-0.3em,parsep=-0.3em,itemsep=-0.3em,partopsep=-0.3em]
%可使用leftmargin调整列表环境左边的空白长度 [leftmargin=0em]
\item
磁场的方向垂直于导轨平面向下;
\item 
$a = \frac { B E l } { m R }$
\item 
$Q _ { 2 } = \frac { B ^ { 2 } l ^ { 2 } C ^ { 2 } E } { B ^ { 2 } l ^ { 2 } C + m }$



\end{enumerate}
}



\newpage
\item
\exwhere{$ 2013 $年北京卷}
对于同一物理问题,常常可以从宏观与微观两个不同角度进行研究,找出其内在联系,从而更加深刻地理解其物理本质。
\begin{enumerate}
\renewcommand{\labelenumi}{\arabic{enumi}.}
% A(\Alph) a(\alph) I(\Roman) i(\roman) 1(\arabic)
%设定全局标号series=example	%引用全局变量resume=example
%[topsep=-0.3em,parsep=-0.3em,itemsep=-0.3em,partopsep=-0.3em]
%可使用leftmargin调整列表环境左边的空白长度 [leftmargin=0em]
\item
一段横截面积为$ S $、长为$ l $的直导线,单位体积内有$ n $个自由电子,电子电量为$ e $。该导线通有电流时,假设自由电子定向移动的速率均为$ v $。
\begin{enumerate}
\item
求导线中的电流$ I $; 
\item 
将该导线放在匀强磁场中,电流方向垂直于磁感应强度$ B $,导线所受安培力大小为$ F_{\text{安}} $,导线内自由电子所受洛伦兹力大小的总和为$ F $,推导$ F _{\text{安}} =F $.



\end{enumerate}

\item 
正方体密闭容器中有大量运动粒子,每个粒子质量为$ m $,单位体积内粒子数量$ n $为恒量。为简化问题,我们假定:粒子大小可以忽略;其速率均为$ v $,且与器壁各面碰撞的机会均等;与器壁碰撞前后瞬间,粒子速度方向都与器壁垂直,且速率不变。利用所学力学知识,导出器壁单位面积所受粒子压力$ f $与$ m $、$ n $和$ v $的关系。

\end{enumerate}
注意:解题过程中需要用到、但题目没有给出的物理量,要在解题时做必要的说明.




\banswer{
\begin{enumerate}
\item
\begin{enumerate}
\item
设$ \Delta t $时间内通过导体横截面的电量为$ \Delta q $,由电流定义有$I = \frac { \Delta q } { \Delta t } = \frac { n e S v \Delta t } { \Delta t } = n e S v$
\item 
略
\end{enumerate}
\item 
$f = \frac { F } { S } = \frac { 1 } { 3 } n m v ^ { 2 }$
\end{enumerate}
}

\end{enumerate}








