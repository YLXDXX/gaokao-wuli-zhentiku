\bta{重力加速度的测定}
\begin{enumerate}
\renewcommand{\labelenumi}{\arabic{enumi}.}
% A(\Alph) a(\alph) I(\Roman) i(\roman) 1(\arabic)
%设定全局标号series=example	%引用全局变量resume=example
%[topsep=-0.3em,parsep=-0.3em,itemsep=-0.3em,partopsep=-0.3em]
%可使用leftmargin调整列表环境左边的空白长度 [leftmargin=0em]
\item
\exwhere{$ 2013 $ 年安徽卷}
根据单摆周期公式 $T=2 \pi \sqrt{\frac{l}{g}}$,可以通过实验测量当地的重力加速度。如图 $ 1 $ 所示,将细线的上端
固定在铁架台上,下端系一小钢球,就做成了单摆。
\begin{figure}[h!]
\centering
\includesvg[width=0.83\linewidth]{picture/svg/GZ-3-tiyou-0542}
\end{figure}

\begin{enumerate}
\renewcommand{\labelenumi}{\arabic{enumi}.}
% A(\Alph) a(\alph) I(\Roman) i(\roman) 1(\arabic)
%设定全局标号series=example	%引用全局变量resume=example
%[topsep=-0.3em,parsep=-0.3em,itemsep=-0.3em,partopsep=-0.3em]
%可使用leftmargin调整列表环境左边的空白长度 [leftmargin=0em]
\item
用游标卡尺测量小钢球直径,求数如图 $ 2 $ 所示,读数为 \tk{$ 18.6 $} $ mm $。


\item 
以下是实验过程中的一些做法,其中正确的有 \tk{ADE} 。
\fivechoices
{摆线要选择细些的、伸缩性小些的,并且尽可能长一些}
{摆球尽量选择质量大些、体积小些的}
{为了使摆的周期大一些,以方便测量,开始时拉开摆球,使摆线相距平衡位置有较大的角度}
{拉开摆球,使摆线偏离平衡位置大于 $ 5 ^{ \circ } $,在释放摆球的同时开始计时,当摆球回到开始位置时停止计时,此时间间隔$ \Delta t $ 即为单摆周期 $ T $}
{拉开摆球,使摆线偏离平衡位置不大于 $ 5 ^{ \circ } $,释放摆球,当摆球振动稳定后,从平衡位置开始计时,记下摆球做 $ 50 $ 次全振动所用的时间$ \Delta t $,则单摆周期 $T=\frac{\Delta t}{50}$}


\end{enumerate}





\newpage
\item 
\exwhere{$ 2011 $ 年理综福建卷}
某实验小组在利用单摆测定当地重力加速度的实验中:

①用游标卡尺测定摆球的直径,测量结果如图所示,则该摆球的直
径为
\tk{$ 0.97 $} 
$ cm $。
\begin{figure}[h!]
\centering
\includesvg[width=0.33\linewidth]{picture/svg/GZ-3-tiyou-0543}
\end{figure}

②小组成员在实验过程中有如下说法,其中正确的是
\tk{C} 。(填选项前的字母)

\fourchoices
{把单摆从平衡位置拉开 $ 30 \degree $的摆角,并在释放摆球的同时开始计时}
{测量摆球通过最低点 $ 100 $ 次的时间 $ t $,则单摆周期为 $ t/100 $}
{用悬线的长度加摆球的直径作为摆长,代入单摆周期公式计算得到的重力加速度值偏大}
{选择密度较小的摆球,测得的重力加速度值误差较小}




\newpage
\item
\exwhere{$ 2012 $ 年理综天津卷}
某同学用实验的方法探究影响单摆周期的因素。
\begin{figure}[h!]
\centering
\includesvg[width=0.23\linewidth]{picture/svg/GZ-3-tiyou-0544}
\end{figure}

①他组装单摆时,在摆线上端的悬点处,用一块开有狭缝的橡皮夹牢摆线,再用铁架台的铁夹将橡
皮夹紧,如图所示。这样做的目的是
\tk{AC} 
(填字母代号)。

\fourchoices
{保证摆动过程中摆长不变}
{可使周期测量得更加准确}
{需要改变摆长时便于调节}
{保证摆球在同一竖直平面内摆动}

②他组装好单摆后在摆球自然悬垂的情况下,用
毫米刻度尺从悬点量到摆球的最低端的长度
$ L=0.9990 \ m $,再用游标卡尺测量摆球直径,结果
游标尺
如图所示,则摆球的直径为 \tk{12.0} 
$ mm $,单摆摆长为
\tk{$ 0.9930 $} 
$ m $。
\begin{figure}[h!]
\centering
\includesvg[width=0.43\linewidth]{picture/svg/GZ-3-tiyou-0545}
\end{figure}

③下列振动图像真实地描述了对摆长约为 $ 1 \ m $ 的单摆进行周期测量的四种操作过程,图中横坐标原
点表示计时开始,$ A $、$ B $、$ C $ 均为 $ 30 $ 次全振动图像,已知 $ \sin 5 ^{ \circ } =0.087 $,$ \sin 15 ^{ \circ } =0.026 $,
这四种操作过程合乎实验要求且误差最小的是 \tk{A} (填字母代号) 
\begin{figure}[h!]
\centering
\includesvg[width=0.83\linewidth]{picture/svg/GZ-3-tiyou-0546}
\end{figure}



\newpage
\item 
\exwhere{$ 2015 $ 年理综天津卷}
某同学利用单摆测量重力加速度。
\begin{figure}[h!]
\centering
\includesvg[width=0.23\linewidth]{picture/svg/GZ-3-tiyou-0547}
\end{figure}

①为了使测量误差尽量小,下列说法正确的是 \tk{BC} 
\fourchoices
{组装单摆须选用密度和直径都较小的摆球}
{组装单摆须选用轻且不易伸长的细线}
{实验时须使摆球在同一竖直面内摆动}
{摆长一定的情况下,摆的振幅尽量大}

②如图所示,在物理支架的竖直立柱上固定有摆长约为 $ 1 \ m $ 的单摆,实验时,由于仅有量程为 $ 20 \ cm $、
精度为 $ 1 \ mm $ 的钢板刻度尺,于是他先使摆球自然下垂,在竖直立柱上与摆球最下端处于同一水平
面的位置做一标记点,测出单摆的周期 $ T_{1} $;然后保持悬点位置不变,设法将摆长缩短一些,再次
使摆球自然下垂,用同样方法在竖直立柱上做另一标记点,并测出单摆周期 $ T_{2} $;最后用钢板刻度
尺量出竖直立柱上两标记点之间的距离$ \Delta L $,用上述测量结果,写出重力加速度的表达式
$ g= $ \tk{$\frac{4 \pi^{2} \Delta L}{T_{1}^{2}-T_{2}^{2}}$} 。





\newpage
\item
\exwhere{$ 2018 $ 年全国\lmd{3}卷}
甲、乙两同学通过下面的实验测量人的反应时间。实验步骤如下:
\begin{figure}[h!]
\centering
\includesvg[width=0.23\linewidth]{picture/svg/GZ-3-tiyou-0548}
\end{figure}

\begin{enumerate}
\renewcommand{\labelenumi}{\arabic{enumi}.}
% A(\Alph) a(\alph) I(\Roman) i(\roman) 1(\arabic)
%设定全局标号series=example	%引用全局变量resume=example
%[topsep=-0.3em,parsep=-0.3em,itemsep=-0.3em,partopsep=-0.3em]
%可使用leftmargin调整列表环境左边的空白长度 [leftmargin=0em]
\item
甲用两个手指轻轻捏住量程为 $ L $ 的木尺上端,让木尺自然下垂。乙把手放在尺的下端(位置
恰好处于 $ L $ 刻度处,但未碰到尺)
,准备用手指夹住下落的尺。


\item 
甲在不通知乙的情况下,突然松手,尺子下落;乙看到尺子下落后
快速用手指夹住尺子。若夹住尺子的位置刻度为 $ L_{1} $,重力加速度大小为
$ g $,则乙的反应时间为
\tk{$\sqrt{\frac{2\left(L-L_{1}\right)}{g}}$} 
(用 $ L $、$ L_{1} $ 和 $ g $ 表示)。

\item 
已知当地的重力加速度大小为 $ g=9.80 \ m/s^{2} $, $ L=30.0 \ cm $, $ L_{1} =10.4 \ cm $。乙的反应时间为 \tk{$ 0.20 $} 
$ s $。(结果保留 $ 2 $ 位有效数字)

\item 
写出一条能提高测量结果准确程度的建议: \tk{多次测量取平均值;初始时乙的手指尽可能接近尺子} 


\end{enumerate}



\newpage
\item 
\exwhere{$ 2018 $ 年海南卷}
学生课外实验小组使用如图所示的实验装置测量重力加速度大小。实
验时,他们先测量分液漏斗下端到水桶底部的距离 $ s $;然后使漏斗中的水一滴一滴地下落,调整阀
门使水滴落到桶底发出声音的同时,下一滴水刚好从漏斗的下端滴
落;用秒表测量第 $ 1 $ 个水滴从漏斗的下端滴落至第 $ n $ 个水滴落到桶底
所用的时间 $ t $。
\begin{figure}[h!]
\centering
\includesvg[width=0.23\linewidth]{picture/svg/GZ-3-tiyou-0549}
\end{figure}

\begin{enumerate}
\renewcommand{\labelenumi}{\arabic{enumi}.}
% A(\Alph) a(\alph) I(\Roman) i(\roman) 1(\arabic)
%设定全局标号series=example	%引用全局变量resume=example
%[topsep=-0.3em,parsep=-0.3em,itemsep=-0.3em,partopsep=-0.3em]
%可使用leftmargin调整列表环境左边的空白长度 [leftmargin=0em]
\item
重力加速度大小可表示为 $ g= $
\tk{$\frac{2 n^{2} s}{t^{2}}$} 
(用 $ s $、$ n $、$ t $ 表示)
;

\item 
如果某次实验中, $ s=0.90 \ m $, $ n=30 $, $ t=13.0 \ s $,则测得的重
力加速度大小
$ g= $
\tk{$ 9.6 $} 
$ m/s^{2} $;(保留 $ 2 $ 位有效数字)

\item 
写出一条能提高测量结果准确程度的建议: \tk{“适当增大 $ n $”或“多次测量取平均值”} 


\end{enumerate}




\newpage
\item 
\exwhere{$ 2012 $ 年物理上海卷}
在“利用单摆测重力加速度”的实验中。
\begin{figure}[h!]
\centering
\includesvg[width=0.13\linewidth]{picture/svg/GZ-3-tiyou-0550}
\end{figure}
\begin{enumerate}
\renewcommand{\labelenumi}{\arabic{enumi}.}
% A(\Alph) a(\alph) I(\Roman) i(\roman) 1(\arabic)
%设定全局标号series=example	%引用全局变量resume=example
%[topsep=-0.3em,parsep=-0.3em,itemsep=-0.3em,partopsep=-0.3em]
%可使用leftmargin调整列表环境左边的空白长度 [leftmargin=0em]
\item
某同学尝试用 $ DIS $ 测量周期。如图,用一个磁性小球代替原先的摆球,在单摆下方放置一个
磁传感器,其轴线恰好位于单摆悬挂点正下方。图中磁传感器的引出端 $ A $ 应接到 \tk{数据采集器} 。使
单摆做小角度摆动,当磁感应强度测量值最大时,磁性小球位于 \tk{最低点(或平衡位置)} 。
若测得连续 $ N $ 个磁感应强度最大值之间的时间间隔为 $ t $,则单摆周期的测量值
为 \tk{$\frac{2 t}{N-1}$} (地磁场和磁传感器的影响可忽略)。


\item 
多次改变摆长使单摆做小角度摆动,测量摆长 $ L $ 及相应的周期 $ T $。此后,
分别取 $ L $ 和 $ T $ 的对数,所得到的 $ \lg T- \lg L $ 图线为 \tk{直线} (填:“直线”、“对
数曲线”或“指数曲线”);读得图线与纵轴交点的纵坐标为 $ c $,由此得到该地重力加速度
$ g= $ \tk{$\frac{4 \pi^{2}}{10^{2 C}}$} 。



\end{enumerate}




\newpage
\item 
\exwhere{$ 2014 $ 年物理上海卷}
某小组在做“用单摆测定重力加速度”实验后,为进一步探究,将单摆的轻质细线改为刚性
重杆。通过查资料得知,这样做成的“复摆”做简谐运动的周期 $T=2 \pi \sqrt{\frac{I _c+m r^{2}}{m g r}}$,式中 $ I_{c} $ 为由该
摆决定的常量,$ m $ 为摆的质量,$ g $ 为重力加速
度,$ r $ 为转轴到重心 $ C $ 的距离。如图$ (a) $,实
验时在杆上不同位置打上多个小孔,将其中
一个小孔穿在光滑水平轴 $ O $ 上,使杆做简谐
运动,测量并记录 $ r $ 和相应的运动周期 $ T $,然
后将不同位置的孔穿在轴上重复实验,实验
数据见表,并测得摆的质量 $ m=0.50 \ kg $。
\begin{figure}[h!]
\centering
\includesvg[width=0.43\linewidth]{picture/svg/GZ-3-tiyou-0551}
\end{figure}



\begin{table}[h!]
\centering 
\begin{tabular}{|c|c|c|c|c|c|c|}
\hline 
$ r/m $ & $ 0.45 $ & $ 0.40 $ & $ 0.35 $ & $ 0.30 $ & $ 0.25 $ & $ 0.20 $
 \\
\hline
$ T/s $ & $ 2.11 $ & $ 2.14 $ & $ 2.20 $ & $ 2.30 $ & $ 2.43 $ & $ 2.64 $\\ 
\hline 
\end{tabular}
\end{table} 


\begin{enumerate}
\renewcommand{\labelenumi}{\arabic{enumi}.}
% A(\Alph) a(\alph) I(\Roman) i(\roman) 1(\arabic)
%设定全局标号series=example	%引用全局变量resume=example
%[topsep=-0.3em,parsep=-0.3em,itemsep=-0.3em,partopsep=-0.3em]
%可使用leftmargin调整列表环境左边的空白长度 [leftmargin=0em]
\item
由实验数据得出图$ (b) $所示的拟合直线,图中纵轴表示
\tk{$ T^{2}r $} 
.




\item 
$ I_{c} $ 的国际单位为 \tk{$ kg \cdot m^{2} $} 
,由拟合直线得到 $ I_{c} $ 的值为
\tk{$ 0.17 $} 
(保留到小数点后二位);




\item 
若摆的质量测量值偏大,重力加速度 $ g $ 的测量值
\tk{不变}
。(选填$ : $“偏大”、“偏小”或“不变")

\end{enumerate}




\newpage
\item 
\exwhere{$ 2015 $ 年理综北京卷}
用单摆测定重力加速度的实验装置如图 $ 2 $
所示。
\begin{figure}[h!]
\centering
\includesvg[width=0.23\linewidth]{picture/svg/GZ-3-tiyou-0552}
\end{figure}



①组装单摆时,应在下列器材中选用
\tk{AD} 
(选填选项前的字母)

\fourchoices
{长度为 $ 1 \ m $ 左右的细线}
{长度为 $ 30 \ cm $ 左右的细线}
{直径为 $ 1.8 \ cm $ 的塑料球}
{直径为 $ 1.8 \ cm $ 的铁球}

②测出悬点 $ O $ 至小球球心的距离(摆长)$ L $ 及单摆完成 $ n $ 次全振动所用的时间 $ t $,则重力加速度
$ g= $ \tk{$\frac{4 \pi^{2} n^{2} L}{t^{2}}$} (用 $ L $、$ n $、$ t $ 表示)


③下表是某同学记录的 $ 3 $ 组实验数据,并做了部分计算处理。
\begin{table}[h!]
\centering 
\begin{tabular}{|c|c|c|c|}
\hline 
组次 & $ 1 $ & $ 2 $ & $ 3 $
 \\
\hline
摆长$ L/cm $ & $ 80.00 $ & $ 90.00 $ & $ 100.00 $
 \\
\hline
$ 50 $次全振动时间$ t/s $ & $ 90.0 $ & $ 95.5 $ & $ 100.5 $
 \\
\hline
振动周期$ T/s $ & $ 1.80 $ & $ 1.91 $ & 
 \\
\hline
重力加速度$ g/(m \cdot s^{-2}) $ & $ 9.74 $ & $ 9.73 $ & \\ 
\hline 
\end{tabular}
\end{table} 




请计算出第 $ 3 $ 组实验中的 $ T=$ \tk{$ 2.01 $} $s $,$ g=$ \tk{$ 9.76 $} $m/s^{2} $。




④用多组实验数据做出 $ T^{2} -L $ 图像,也可以求出重力加速度 $ g $,已知三位同学做出的 $ T^{2} -L $ 图线的示意
图如图 $ 3 $ 中的 $ a $、$ b $、$ c $ 所示,其中 $ a $ 和 $ b $ 平行,$ b $ 和 $ c $ 都过原点,图线
$ b $ 对应的 $ g $ 值最接近当地重力加速度的值。则相对于图线 $ b $,下列分析
正确的是 \tk{B} (选填选项前的字母)。
\begin{figure}[h!]
\centering
\includesvg[width=0.23\linewidth]{picture/svg/GZ-3-tiyou-0553}
\end{figure}

\threechoices
{出现图线 $ a $ 的原因可能是误将悬点到小球下端的距离记为摆长 $ L $}
{出现图线 $ c $ 的原因可能是误将 $ 49 $ 次全振动记为 $ 50 $ 次}
{图线 $ c $ 对应的 $ g $ 值小于图线 $ b $ 对应的 $ g $ 值}


⑤某同学在家里测重力加速度,他找到细线和铁锁,制成一个单摆,如图$ 4 $所示,由于家里只有一根量程为 $ 30 \ cm $ 的刻度尺,于是他在细线上的 $ A $
点做了一个标记,使得悬点 $ O $ 到 $ A $ 点间的细线长度小于刻度尺量程。保持
该标记以下的细线长度不变,通过改变 $ O $、$ A $ 间细线长度以改变摆长。实
验中,当 $ O $、$ A $ 间细线的长度分别为 $ l_{1} $、$ l_{2} $ 时,测得相应单摆的周期为 $ T_{1} $、
$ T_{2} $。由此可得重力加速度 $ g= $ \tk{$\frac{4 \pi^{2}\left(l_{1}-l_{2}\right)}{T_{1}^{2}-T_{2}^{2}}$} (用 $ l_{1} $、$ l_{2} $、$ T_{1} $、$ T_{2} $ 表示)
\begin{figure}[h!]
\centering
\includesvg[width=0.23\linewidth]{picture/svg/GZ-3-tiyou-0554}
\end{figure}


\item 
\exwhere{$ 2019 $ 年物理全国\lmd{3}卷}
甲乙两位同学设计了利用数码相机的连拍功能测重力加速度的实验。实
验中,甲同学负责释放金属小球,乙同学负责在小球自由下落的时候拍照。已知相机每间隔 $ 0.1 \ s $ 拍$ 1 $幅照片。
\begin{enumerate}
\renewcommand{\labelenumi}{\arabic{enumi}.}
% A(\Alph) a(\alph) I(\Roman) i(\roman) 1(\arabic)
%设定全局标号series=example	%引用全局变量resume=example
%[topsep=-0.3em,parsep=-0.3em,itemsep=-0.3em,partopsep=-0.3em]
%可使用leftmargin调整列表环境左边的空白长度 [leftmargin=0em]
\item
若要从拍得的照片中获取必要的信息,在此实验中还必须使用的器材是 \tk{A} 。
(填正确答
案标号)

\fourchoices
{米尺}
{秒表}
{光电门}
{天平}

\item 
简述你选择的器材在本实验中的使用方法。


答: \tk{将米尺竖直放置,使小球下落时尽量靠近米尺 。} 



\item 
实验中两同学由连续 $ 3 $ 幅照片上小球的位置 $ a $、$ b $ 和 $ c $ 得到 $ ab=24.5 \ cm $、$ ac=58.7 \ cm $,则该地的
重力加速度大小为 $ g=$ \tk{$ 9.7 $} $m/s^{2} $。(保留 $ 2 $ 位有效数字)


\end{enumerate}









\end{enumerate}

