\bta{匀变速直线运动}
\begin{enumerate}
\renewcommand{\labelenumi}{\arabic{enumi}.}
% A(\Alph) a(\alph) I(\Roman) i(\roman) 1(\arabic)
%设定全局标号series=example	%引用全局变量resume=example
%[topsep=-0.3em,parsep=-0.3em,itemsep=-0.3em,partopsep=-0.3em]
%可使用leftmargin调整列表环境左边的空白长度 [leftmargin=0em]
\item
\exwhere{$ 2011 $年理综天津卷}
质点做直线运动的位移$ x $与时间$ t $的关系为$ x=5t+t^{2} $(各物理量均采用国际单位制单位),则该质点 \xzanswer{D} 


\fourchoices
{第$ 1 \ s $内的位移是$ 5 \ m $ }
{前$ 2 \ s $内的平均速度是$ 6 \ m/s $}
{任意相邻的$ 1 \ s $ 内位移差都是$ 1 \ m $}
{任意$ 1 \ s $内的速度增量都是$ 2 \ m/s $}



\item 
\exwhere{$ 2011 $年理综重庆卷}
某人估测一竖直枯井深度,从井口静止释放一石头并开始计时,经$ 2 \ s $听到石头落地声,由此可知井深约为(不计声音传播时间,重力加速度$ g $取$ 10 \ m/s ^{2} $) \xzanswer{B} 


\fourchoices
{$ 10 \ m $ }
{$ 20\ m $}
{$ 30\ m $ }
{$ 40\ m $}


\item 
\exwhere{$ 2016 $年新课标 \lmd{3} 卷}
一质点做速度逐渐增大的匀加速直线运动,在时间间隔$ t $内位移为$ s $,动能变为原来的$ 9 $倍。该质点的加速度为 \xzanswer{A} 

\fourchoices
{$ \frac { s } { t ^ { 2 } } $}
{$ \frac { 3 s } { 2 t ^ { 2 } } $}
{$ \frac { 4 s } { t ^ { 2 } } $ }
{$ \frac { 8 s } { t ^ { 2 } } $}


\item 
\exwhere{$ 2016 $年上海卷}
物体做匀加速直线运动,相继经过两段距离均为$ 16 $ $ m $的路程,第一段用时$ 4 $ $ s $,第二段用时$ 2 $ $ s $,则物体的加速度是 \xzanswer{B} 

\fourchoices
{$ \frac{ 2 }{ 3 } \ m/s^{2} $}
{$ \frac{ 4 }{ 3 } \ m/s^{2} $}
{$ \frac{ 8 }{ 9 } \ m/s^{2} $}
{$ \frac{16}{9}\ m/s^{2} $}





\item 
\exwhere{$ 2017 $年海南卷}
汽车紧急刹车后,停止运动的车轮在水平地面上滑动直至停止,在地面上留下的痕迹称为刹车线。由刹车线的长短可知汽车刹车前的速度。已知汽车轮胎与地面之间的动摩擦因数为$ 0.80 $,测得刹车线长$ 25 $ $ m $。汽车在刹车前的瞬间的速度大小为(重力加速度$ g $取$ 10 \ m/s ^{2} $) \xzanswer{B} 


\fourchoices
{$ 10 \ m/s $ }
{$ 20 \ m/s $}
{$ 30 \ m/s $ }
{$ 40 \ m/s $}


\item 
\exwhere{$ 2017 $年浙江选考卷}
汽车以$ 10 \ m/s $的速度在马路上匀速行驶,驾驶员发现正前方$ 15 \ m $处的斑马线上有行人,于是刹车礼让,汽车恰好停在斑马线前,假设驾驶员反应时间为$ 0.5 \ s $。汽车运动的$ v-t $图如图所示,则汽车的加速度大小为 \xzanswer{C} 
\begin{figure}[h!]
\centering
\includesvg[width=0.16\linewidth]{picture/svg/380}
\end{figure}


\fourchoices
{$20\ \mathrm { m }/ \mathrm { s } ^ { 2 } \quad$}
{$6\ \mathrm { m } / \mathrm { s } ^ { 2 } \quad$}
{$5\ \mathrm { m } / \mathrm { s } ^ { 2 } \quad$}
{$4\ \mathrm { m } / \mathrm { s } ^ { 2 }$}



\item 
\exwhere{$ 2013 $年新课标 \lmd{1} 卷}
右表是伽利略$ 1604 $年做斜面实验时的一页手稿左上角的三列数据。表中第二列是时间,第三列是物体沿斜面运动的距离,第一列是伽利略在分析实验数据时添加的。根据表中的数据.伽利略可以得出的结论是 \xzanswer{C} 



\begin{minipage}[h!]{0.7\linewidth}
\vspace{0.3em}
\fourchoices
{物体具有惯性}
{斜面倾角一定时,加速度与质量无关}
{物体运动的距离与时间的平方成正比}
{物体运动的加速度与重力加速度成正比}

\vspace{0.3em}
\end{minipage}
\hfill
\begin{minipage}[h!]{0.3\linewidth}
\flushright
\vspace{0.3em}
\centering 
\begin{tabular}{|c|c|c|}
\hline 
1 & 1 & 32
 \\
\hline
4 & 2 & 130
 \\
\hline
9 & 3 & 298
 \\
\hline
16 & 4 & 526
 \\
\hline
25 & 5 & 824
 \\
\hline
36 & 6 & 1192
 \\
\hline
49 & 7 & 1600
 \\
\hline
64 & 8 & 2104
 \\
\hline
\end{tabular}
\vspace{0.3em}
\end{minipage}



\item 
\exwhere{$ 2013 $年广东卷}
某航母跑道长$ 200 \ m $.飞机在航母上滑行的最大加速度为$ 6 \ m/s ^{2} $,起飞需要的最低速度为$ 50 \ m/s $.那么,飞机在滑行前,需要借助弹射系统获得的最小初速度为
\xzanswer{B} 
\fourchoices
{$ 5 \ m/s $}
{$ 10 \ m/s $}
{$ 15 \ m/s $}
{$ 20 \ m/s $}

\item 
\exwhere{$ 2014 $年物理上海卷}
如图,两光滑斜面在$ B $处连接,小球由$ A $处静止释放,经过$ B $、$ C $两点时速度大小分别为$ 3 \ m/s $和$ 4 \ m/s $, $ AB=BC $。设球经过$ B $点前后速度大小不变,则球在$ AB $、$ BC $段的加速度大小之比为 \tk{$ 7:9 $} ,球由$ A $运动到$ C $的平均速率为 \tk{2.1} $ m/s $。
\begin{figure}[h!]
\centering
\includesvg[width=0.23\linewidth]{picture/svg/381}
\end{figure}

\item 
\exwhere{$ 2011 $年理综安徽卷}
一物体作匀加速直线运动,通过一段位移$ \Delta x $所用的时间为$ t_{1} $,紧接着通过下一段位移$ \Delta x $所用的时间为$ t_{2} $。则物体运动的加速度为 \xzanswer{A} 

\fourchoices
{$ \frac { 2 \Delta x \left( t _ { 1 } - t _ { 2 } \right) } { t _ { 1 } t _ { 2 } \left( t _ { 1 } + t _ { 2 } \right) } $}
{$ \frac { \Delta x \left( t _ { 1 } - t _ { 2 } \right) } { t _ { 1 } t _ { 2 } \left( t _ { 1 } + t _ { 2 } \right) } $}
{$ \frac { 2 \Delta x \left( t _ { 1 } + t _ { 2 } \right) } { t _ { 1 } t _ { 2 } \left( t _ { 1 } - t _ { 2 } \right) } $}
{$ \frac { \Delta x \left( t _ { 1 } + t _ { 2 } \right) } { t _ { 1 } t _ { 2 } \left( t _ { 1 } - t _ { 2 } \right) } $}


\item 
\exwhere{$ 2014 $年物理海南卷}
短跑运动员完成$ 100 \ m $赛跑的过程可简化为匀加速运动和匀速运动两个阶段。一次比赛中,某运动员用$ 11.00 \ s $跑完全程。已知运动员在加速阶段的第$ 2 \ s $内通过的距离为$ 7.5\ m $,求该运动员的加速度及在加速阶段通过的距离。
\banswer{
$a = 5\ \mathrm { m } / \mathrm { s } ^ { 2 }$,	设加速阶段通过的距离为$ S ^{\prime} =10\ m $
}


\newpage
\item 
\exwhere{$ 2013 $年全国卷大纲版}
一客运列车匀速行驶,其车轮在铁轨间的接缝处会产生周期性的撞击。坐在该客车中的某旅客测得从第$ 1 $次到第$ 16 $次撞击声之间的时间间隔为$ 10.0 \ s $。在相邻的平行车道上有一列货车,当该旅客经过货车车尾时,货车恰好从静止开始以恒定加速度沿客车行进方向运动。该旅客在此后的$ 20.0 \ s $内,看到恰好有$ 30 $节货车车厢被他连续超过。已知每根铁轨的长度为$ 25.0\ m $,每节货车车厢的长度为$ 16.0\ m $,货车车厢间距忽略不计。求:
\begin{enumerate}
\renewcommand{\labelenumi}{\arabic{enumi}.}
% A(\Alph) a(\alph) I(\Roman) i(\roman) 1(\arabic)
%设定全局标号series=example	%引用全局变量resume=example
%[topsep=-0.3em,parsep=-0.3em,itemsep=-0.3em,partopsep=-0.3em]
%可使用leftmargin调整列表环境左边的空白长度 [leftmargin=0em]
\item
客车运行速度的大小;
\item 
货车运行加速度的大小。



\end{enumerate}

\banswer{
\begin{enumerate}
\renewcommand{\labelenumi}{\arabic{enumi}.}
% A(\Alph) a(\alph) I(\Roman) i(\roman) 1(\arabic)
%设定全局标号series=example	%引用全局变量resume=example
%[topsep=-0.3em,parsep=-0.3em,itemsep=-0.3em,partopsep=-0.3em]
%可使用leftmargin调整列表环境左边的空白长度 [leftmargin=0em]
\item
$v = 37.5 \ \mathrm { m } / \mathrm { s }$
\item 
$ a=1.35\ m/s^{2} $

\end{enumerate}


}



\item 
\exwhere{$ 2015 $年江苏卷}
如图所示,某“闯关游戏”的笔直通道上每隔 $ 8 $ $ m $设有一个关卡,各关卡同步放行和关闭,放行和关闭的时间分别为 $ 5 $ $ s $ 和 $ 2 $ $ s $. 关卡刚放行时,一同学立即在关卡 $ 1 $ 处以加速度 $ 2 $ $ \ m/s ^{2} $由静止加速到 $ 2 $ $ m /s $,然后匀速向前,则最先挡住他前进的关卡是 \xzanswer{C} 
\begin{figure}[h!]
\centering
\includesvg[width=0.23\linewidth]{picture/svg/382}
\end{figure}

\fourchoices
{关卡 $ 2 $ }
{关卡 $ 3 $ }
{关卡 $ 4 $ }
{关卡 $ 5 $ }


\item 
\exwhere{$ 2014 $年理综新课标\lmd{1}卷}
公路上行驶的两汽车之间应保持一定的安全距离.当前车突然停止时,后车司机可以采取刹车措施,使汽车在安全距离内停下而不会与前车相碰.通常情况下,人的反应时间和汽车系统的反应时间之和为$ 1 $ $ s $,当汽车在晴天干燥沥青路面上以$ 108 $ $ km/h $的速度匀速行驶时,安全距离为$ 120 $ $ m $.设雨天时汽车轮胎与沥青路面间的动摩擦因数为晴天时的$ \frac{ 2 }{ 5 } $,若要求安全距离仍为$ 120 $ $ m $,求汽车在雨天安全行驶的最大速度.

\banswer{
$ 20\ m/s $
}


\newpage		
\item 
\exwhere{$ 2011 $年新课标版}
甲乙两辆汽车都从静止出发做加速直线运动,加速度方向一直不变。在第一段时间间隔内,两辆汽车的加速度大小不变,汽车乙的加速度大小是甲的两倍;在接下来的相同时间间隔内,汽车甲的加速度大小增加为原来的两倍,汽车乙的加速度大小减小为原来的一半。求甲乙两车各自在这两段时间间隔内走过的总路程之比。

\banswer{
$\frac { s } { s ^ { \prime } } = \frac { 5 } { 7 }$
}



\item 
\exwhere{$ 2015 $年理综四川卷}
严重的雾霾天气,对国计民生已造成了严重的影响,汽车尾气是形成雾霾的重要污染源,“铁腕治污”已成为国家的工作重点。地铁列车可实现零排放,大力发展地铁,可以大大减少燃油公交车的使用,减少汽车尾气排放。

若一地铁列车从甲站由静止启动后做直线运动,先匀加速运动$ 20 \ s $达到最高速度$ 72 \ km/h $,再匀速运动$ 80 \ s $,接着匀减速运动$ 15 \ s $到达乙站停住。设列车在匀加速运动阶段牵引力为$ 1 \times 10^{6} \ N $,匀速阶段牵引力的功率为$ 6 \times 10^{3} \ kW $,忽略匀减速运动阶段牵引力所做的功。
\begin{enumerate}
\renewcommand{\labelenumi}{\arabic{enumi}.}
% A(\Alph) a(\alph) I(\Roman) i(\roman) 1(\arabic)
%设定全局标号series=example	%引用全局变量resume=example
%[topsep=-0.3em,parsep=-0.3em,itemsep=-0.3em,partopsep=-0.3em]
%可使用leftmargin调整列表环境左边的空白长度 [leftmargin=0em]
\item
求甲站到乙站的距离;
\item 
如果燃油公交车运行中做的功与该列车从甲站到乙站牵引力做的功相同,求公交车排放气体污染物的质量。(燃油公交车每做$ 1 $焦耳功排放气体污染物$ 3 \times 10^{-6} $克)

\end{enumerate}

\banswer{
\begin{enumerate}
\renewcommand{\labelenumi}{\arabic{enumi}.}
% A(\Alph) a(\alph) I(\Roman) i(\roman) 1(\arabic)
%设定全局标号series=example	%引用全局变量resume=example
%[topsep=-0.3em,parsep=-0.3em,itemsep=-0.3em,partopsep=-0.3em]
%可使用leftmargin调整列表环境左边的空白长度 [leftmargin=0em]
\item
$ s=1950\ m $
\item 
$ M=3\times10^{-9} \times 6.8 \times 10^{8} \ kg = 2.04\ kg $

\end{enumerate}


}



\newpage
\item 
\exwhere{$ 2017 $年新课标 \lmd{2} 卷}
$ 24 $.($ 12 $分)为提高冰球运动员的加速能力,教练员在冰面上与起跑线距离$ s_{0} $和$ s_{1} $($ s_1<s_0 $)处分别设置一个挡板和一面小旗,如图所示。训练时,让运动员和冰球都位于起跑线上,教练员将冰球以速度$ v_{0} $击出,使冰球在冰面上沿垂直于起跑线的方向滑向挡板:冰球被击出的同时,运动员垂直于起跑线从静止出发滑向小旗。训练要求当冰球到达挡板时,运动员至少到达小旗处。假定运动员在滑行过程中做匀加速运动,冰球到达挡板时的速度为$ v_{1} $。重力加速度为$ g $。求:
\begin{enumerate}
\renewcommand{\labelenumi}{\arabic{enumi}.}
% A(\Alph) a(\alph) I(\Roman) i(\roman) 1(\arabic)
%设定全局标号series=example	%引用全局变量resume=example
%[topsep=-0.3em,parsep=-0.3em,itemsep=-0.3em,partopsep=-0.3em]
%可使用leftmargin调整列表环境左边的空白长度 [leftmargin=0em]
\item
冰球与冰面之间的动摩擦因数;
\item 
满足训练要求的运动员的最小加速度。

\end{enumerate}
\begin{figure}[h!]
\flushright
\includesvg[width=0.16\linewidth]{picture/svg/383}
\end{figure}

\banswer{

\begin{enumerate}
\renewcommand{\labelenumi}{\arabic{enumi}.}
% A(\Alph) a(\alph) I(\Roman) i(\roman) 1(\arabic)
%设定全局标号series=example	%引用全局变量resume=example
%[topsep=-0.3em,parsep=-0.3em,itemsep=-0.3em,partopsep=-0.3em]
%可使用leftmargin调整列表环境左边的空白长度 [leftmargin=0em]
\item
$\frac { v _ { 0 } ^ { 2 } - v _ { 1 } ^ { 2 } } { 2 g s _ { 0 } }$
\item 
$\frac { s _ { 1 } \left( v _ { 0 } + v _ { 1 } \right) ^ { 2 } } { 2 s _ { 0 } ^ { 2 } }$



\end{enumerate}



}





\item 
\exwhere{$ 16 $. $ 2013 $年新课标 \lmd{1} 卷}
水平桌面上有两个玩具车$ A $和$ B $,两者用一轻质细橡皮筋相连,在橡皮筋上有一红色标记$ R $。在初始时橡皮筋处于拉直状态,$ A $、$ B $和$ R $分别位于直角坐标系中的$ (0,2l) $、$ (0,-l) $和$ (0,0) $点。已知$ A $从静止开始沿$ y $轴正向做加速度大小为$ a $的匀加速运动,$ B $平行于$ x $轴朝$ x $轴正向匀速运动。在两车此后运动的过程中,标记$ R $在某时刻通过点$ (l $, $ l) $。假定橡皮筋的伸长是均匀的,求$ B $运动速度的大小。

\banswer{
$v = \frac { 1 } { 4 } \sqrt { 6 a l }$
}


\newpage	
\item 
\exwhere{$ 2018 $年浙江卷($ 4 $月选考)}
竖井中的升降机可将地下深处的矿石快速运送到地面。某一竖井的深度为$ 104 \ m $,升降机运行的最大速度为$ 8 \ m/s $,加速度大小不超过$ 1 \ m/s ^{2} $ 。假定升降机到井口的速度为$ 0 $,则将矿石从井底提升到井口的最短时间是 \xzanswer{C} 


\fourchoices
{$ 13 \ s $ }
{$ 16 \ s $}
{$ 21 \ s $ }
{$ 26 \ s $}

\item 
\exwhere{$ 2018 $年浙江卷($ 4 $月选考)}
可爱的企鹅喜欢在冰面上游玩,如图所示,有一企鹅在倾角为$ 37 ^{\circ} $的斜面上,先以加速度$ a=0.5 \ m/s ^{2} $从冰面底部由静止开始沿直线向上“奔跑”,$ t=8 \ s $时,突然卧倒以肚皮贴着冰面向前滑行,最后退滑到出发点,完成一次游戏(企鹅在滑动过程中姿势保持不变)。若企鹅肚皮与冰面间的动摩擦因数$ \mu =0.25 $,已知$ \sin 37 ^{\circ} =0.6 $,$ \cos 37 ^{\circ} =0.8 $ 。求:
\begin{enumerate}
\renewcommand{\labelenumi}{\arabic{enumi}.}
% A(\Alph) a(\alph) I(\Roman) i(\roman) 1(\arabic)
%设定全局标号series=example	%引用全局变量resume=example
%[topsep=-0.3em,parsep=-0.3em,itemsep=-0.3em,partopsep=-0.3em]
%可使用leftmargin调整列表环境左边的空白长度 [leftmargin=0em]
\item
企鹅向上“奔跑”的位移大小;
\item 
企鹅在冰面上滑动的加速度大小;
\item 
企鹅退滑到出发点的速度大小。(计算结果可用根式表示)

\end{enumerate}


\banswer{
\begin{enumerate}
\renewcommand{\labelenumi}{\arabic{enumi}.}
% A(\Alph) a(\alph) I(\Roman) i(\roman) 1(\arabic)
%设定全局标号series=example	%引用全局变量resume=example
%[topsep=-0.3em,parsep=-0.3em,itemsep=-0.3em,partopsep=-0.3em]
%可使用leftmargin调整列表环境左边的空白长度 [leftmargin=0em]
\item
$ x=16\ m $
\item 
上滑过程:$a _ { 1 } = g \sin \theta + \mu g \cos \theta$ 有$ a_1=8\ m/s^{2} $,
\\
下滑过程: $a _ { 2 } = g \sin \theta - \mu g \cos \theta$有$ a_2=4\ m/s^{2} $,
\item 
$v = 2 \sqrt { 34 } \mathrm { m } / \mathrm { s }$

\end{enumerate}


}








\end{enumerate}




