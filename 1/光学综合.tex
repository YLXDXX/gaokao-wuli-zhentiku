\bta{光学综合}


\begin{enumerate}
	%\renewcommand{\labelenumi}{\arabic{enumi}.}
	% A(\Alph) a(\alph) I(\Roman) i(\roman) 1(\arabic)
	%设定全局标号series=example	%引用全局变量resume=example
	%[topsep=-0.3em,parsep=-0.3em,itemsep=-0.3em,partopsep=-0.3em]
	%可使用leftmargin调整列表环境左边的空白长度 [leftmargin=0em]
	\item
\exwhere{$ 2014 $ 年理综天津卷}
一束由两种频率不同的单色组成的复色光从空气射入玻璃三棱镜后,出射光分成 $ a $,$ b $ 两束,如
图所示,则 $ a $、$ b $ 两束光 \xzanswer{AB} 
\begin{figure}[h!]
	\centering
	\includesvg[width=0.23\linewidth]{picture/svg/GZ-3-tiyou-1423}
\end{figure}

\fourchoices
{垂直穿过同一块平板玻璃,$ a $ 光所用的时间比 $ b $ 光长}
{从同种介质射入真空发生全反射时,$ a $ 光临界角比 $ b $ 光的小}
{分别通过同一双缝干涉装置,$ b $ 光形成的相邻亮条纹间距小}
{若照射同一金属都能发生光电效率,$ b $ 光照射时逸出的光电子最大初动能大}



\item 
\exwhere{$ 2014 $ 年理综浙江卷}
关于下列光学现象,说法正确的是 \xzanswer{CD} 

\fourchoices
{水中蓝光的传播速度比红光快}
{光从空气向射入玻璃时可能发生全反射}
{在岸边观察前方水中的一条鱼,鱼的实际深度比看到的要深}
{分别用蓝光和红光在同一装置上做双缝干涉实验,用红光时得到的条纹间距更宽}



\item 
\exwhere{$ 2012 $ 年理综四川卷}
$ a $、$ b $ 两种单色光组成的光束从介质进入空气时,其折射光束如图所示。用 $ a $、$ b $ 两束光 \xzanswer{C} 
\begin{figure}[h!]
	\centering
	\includesvg[width=0.23\linewidth]{picture/svg/GZ-3-tiyou-1424}
\end{figure}

\fourchoices
{先后照射双缝干涉实验装置,在缝后屏上都能出现干涉条纹,由此确定光是横波}
{先后照射某金属,$ a $ 光照射时恰能逸出光电子,$ b $ 光照射时也能逸出光电子}
{从同一介质以相同方向射向空气,其界面为平面,若 $ b $ 光不能进入空气,则 $ a $ 光也不能进入空气}
{从同一介质以相同方向射向空气,其界面为平面,$ a $ 光的反射角比 $ b $ 光的反射角大}




\item 
\exwhere{$ 2012 $ 年理综天津卷}
半圆形玻璃砖横截面如图,$ AB $ 为直径,$ O $ 点为圆心。在该截面内有 $ a $、
$ b $ 两束单色可见光从空气垂直于 $ AB $ 射入玻璃砖, 两入射点到 $ O $ 的距离相
等。两束光在半圆边界上反射和折射的情况如图所示:则 $ a $、$ b $ 两束光 \xzanswer{ACD} 
\begin{figure}[h!]
	\centering
	\includesvg[width=0.23\linewidth]{picture/svg/GZ-3-tiyou-1425}
\end{figure}

\fourchoices
{在同种均匀介质中传播,$ a $ 光的传播速度较大}
{以相同的入射角从空气斜射入水中,$ b $ 光的折射角大}
{若 $ a $ 光照射某金属表面能发生光电效应,$ b $ 光也一定能}
{分别通过同一双缝干涉装置,$ a $ 光的相邻亮条纹间距大}



\item 
\exwhere{$ 2015 $ 年理综天津卷}
中国古人对许多自然现象有深刻认识,唐人张志和在《玄真子$ \cdot $涛之灵》
中写道:“雨色映日而为虹”,从物理学的角度看,虹是太阳光经过
雨滴的两次折射和一次反射形成的,右图是彩虹成因的简化示意
图,其中 $ a $、$ b $ 是两种不同频率的单色光,则两光 \xzanswer{C} 
\begin{figure}[h!]
	\centering
	\includesvg[width=0.23\linewidth]{picture/svg/GZ-3-tiyou-1426}
\end{figure}


\fourchoices
{在同种玻璃中传播,$ a $ 光的传播速度一定大于 $ b $ 光}
{以相同角度斜射到同一玻璃板透过平行表面后,$ b $ 光侧移量大}
{分别照射同一光电管,若 $ b $ 光能引起光电效应,$ a $ 光也一定能}
{以相同的入射角从水中射入空气,在空气中只能看到一种光时,一定是 $ a $ 光}



\item 
\exwhere{$ 2015 $ 年理综北京卷}
真空中放置的平行金属板可以用作光电转换装置,如图所示。光
照前两板都不带电。以光照射 $ A $ 板,则板中的电子可能吸收光的能量而逸出。假设所有逸出的电子
都垂直于 $ A $ 板向 $ B $ 板运动,忽略电子之间的相互作用。保持光照条件不变,$ a $ 和 $ b $ 为接线柱。已知
单位时间内从 $ A $ 板逸出的电子数为 $ N $,电子逸出时的最大动能为 $ E_{km}  $,元电荷为 $ e $。
\begin{enumerate}
	%\renewcommand{\labelenumi}{\arabic{enumi}.}
	% A(\Alph) a(\alph) I(\Roman) i(\roman) 1(\arabic)
	%设定全局标号series=example	%引用全局变量resume=example
	%[topsep=-0.3em,parsep=-0.3em,itemsep=-0.3em,partopsep=-0.3em]
	%可使用leftmargin调整列表环境左边的空白长度 [leftmargin=0em]
	\item
求 $ A $ 板和 $ B $ 板之间的最大电势差 $ U_{m} $,以及将 $ a $、$ b $ 短接时回路中的电流 $ I_{ \text{短} } $ ;


\item 
图示装置可看作直流电源,求其电动势 $ E $ 和内阻 $ r $;


\item 
在 $ a $ 和 $ b $ 之间连接一个外电阻时,该电阻两端的电压为 $ U $。
外电阻上消耗的电功率设为 $ P $;单位时间内到达 $ B $ 板的电子,
在从 $ A $ 板运动到 $ B $ 板的过程中损失的动能之和设为$ \Delta E_{k} $。请推
导证明:$ P= \Delta E_{k} $。

\end{enumerate}
\begin{figure}[h!]
	\flushright
	\includesvg[width=0.25\linewidth]{picture/svg/GZ-3-tiyou-1427}
\end{figure}


\banswer{
	\begin{enumerate}
		%\renewcommand{\labelenumi}{\arabic{enumi}.}
		% A(\Alph) a(\alph) I(\Roman) i(\roman) 1(\arabic)
		%设定全局标号series=example	%引用全局变量resume=example
		%[topsep=-0.3em,parsep=-0.3em,itemsep=-0.3em,partopsep=-0.3em]
		%可使用leftmargin调整列表环境左边的空白长度 [leftmargin=0em]
		\item
		$U_{m}=\frac{E_{K m}}{e}$ \quad $I_{\text {短 }}=N e$
		\item 
		$E=U_{m}=\frac{E_{K m}}{e}$ \quad $r=\frac{E_{K m}}{N e^{2}}$
		\item 
		设单位时间内到达 $B$ 板的电子数为 $N^{\prime},$ 则电路中的电流 
		\begin{equation}\label{key}
			I=\frac{Q^{\prime}}{t}=\frac{N^{\prime} t e}{t}=N^{\prime} e
		\end{equation}
	则外电阻消耗的功率
	\begin{equation}\label{key}
		P=U I=U N^{\prime} e
	\end{equation}
光电子在两极板中运动时,两极板间电压为 $U,$ 每个电子损失的动能
\begin{equation}\label{key}
	\Delta E_{K \text { 每个 }}=e U
\end{equation}
则单位时间内到达 $B$ 板的电子损失的总动能 
\begin{equation}\label{key}
	\Delta E=N^{\prime} \Delta E_{K \text { 每个 }}=N^{\prime} e U
\end{equation}
因此 $P=\Delta E_{K}$。
	\end{enumerate}
}


\item 
\exwhere{$ 2017 $ 年天津卷}
 明代学者方以智在《阳燧倒影》中记载:“凡宝石面凸,则光成一条,有数棱则
必有一面五色”,表明白光通过多棱晶体折射会发生色散现象。如图所
示,一束复色光通过三棱镜后分解成两束单色光 $ a $、$ b $,下列说法正确 \xzanswer{D} 
\begin{figure}[h!]
	\centering
	\includesvg[width=0.23\linewidth]{picture/svg/GZ-3-tiyou-1428}
\end{figure}

\fourchoices
{若增大入射角 $ i $,则 $ b $ 光先消失}
{在该三棱镜中 $ a $ 光波长小于 $ b $ 光}
{$ a $ 光能发生偏振现象,$ b $ 光不能发生}
{若 $ a $、$ b $ 光分别照射同一光电管都能发生光电效应,则 $ a $ 光的遏止电压低}






	
	
	
\end{enumerate}

