\bta{自感现象}

\begin{enumerate}
%\renewcommand{\labelenumi}{\arabic{enumi}.}
% A(\Alph) a(\alph) I(\Roman) i(\roman) 1(\arabic)
%设定全局标号series=example	%引用全局变量resume=example
%[topsep=-0.3em,parsep=-0.3em,itemsep=-0.3em,partopsep=-0.3em]
%可使用leftmargin调整列表环境左边的空白长度 [leftmargin=0em]
\item
\exwhere{$ 2011 $ 年理综北京卷}
某同学为了验证断电自感现象,自己找来带铁心的线圈 $ L $,小灯泡 $ A $,开关 $ S $ 和电池组 $ E $,用
导线将它们连接成如图所示的电路。检查电路后,闭合开关 $ S $,
小灯泡发光;再断开开关 $ S $,小灯泡仅有不显著的延时熄灭现象。
虽经多次重复,仍未见老师演示时出现的小灯泡闪亮现象,他冥
思苦想找不出原因。你认为最有可能造成小灯泡未闪亮的原因是 \xzanswer{C} 
\begin{figure}[h!]
\centering
\includesvg[width=0.23\linewidth]{picture/svg/GZ-3-tiyou-1130}
\end{figure}

\fourchoices
{电源的内阻较大}
{小灯泡电阻偏大}
{线圈电阻偏大}
{线圈的自感系数较大}


\item 
\exwhere{$ 2017 $ 年北京卷}
图 $ 1 $ 和图 $ 2 $ 是教材中演示自感现象的两个电路图,$ L_{1} $ 和 $ L_{2} $ 为电感线圈。实验
时,断开开关 $ S_{1} $ 瞬间,灯 $ A_{1} $ 突然闪亮,随后逐渐变暗;闭合开关 $ S_{2} $,灯 $ A_{2} $ 逐渐变亮,而另一个
相同的灯 $ A_{3} $ 立即变亮,最终 $ A_{2} $ 与 $ A_{3} $ 的亮度相同。下列说法正确的是 \xzanswer{C} 
\begin{figure}[h!]
\centering
\begin{subfigure}{0.4\linewidth}
\centering
\includesvg[width=0.7\linewidth]{picture/svg/GZ-3-tiyou-1131} 
\caption{}\label{}
\end{subfigure}
\begin{subfigure}{0.4\linewidth}
\centering
\includesvg[width=0.7\linewidth]{picture/svg/GZ-3-tiyou-1132} 
\caption{}\label{}
\end{subfigure}
\end{figure}

\fourchoices
{图 $ 1 $ 中,$ A_{1} $ 与 $ L_{1} $ 的电阻值相同}
{图 $ 1 $ 中,闭合 $ S_{1} $,电路稳定后,$ A_{1} $ 中电流大于 $ L_{1} $ 中电流}
{图 $ 2 $ 中,变阻器 $ R $ 与 $ L_{2} $ 的电阻值相同}
{图 $ 2 $ 中,闭合 $ S_{2} $ 瞬间,$ L_{2} $ 中电流与变阻器 $ R $中电流相等}








\end{enumerate}

