\bta{变压器}
\begin{enumerate}
%\renewcommand{\labelenumi}{\arabic{enumi}.}
% A(\Alph) a(\alph) I(\Roman) i(\roman) 1(\arabic)
%设定全局标号series=example	%引用全局变量resume=example
%[topsep=-0.3em,parsep=-0.3em,itemsep=-0.3em,partopsep=-0.3em]
%可使用leftmargin调整列表环境左边的空白长度 [leftmargin=0em]
\item
\exwhere{$ 2019 $ 年物理江苏卷}
某理想变压器原、副线圈的匝数之比为 $ 1 $:$ 10 $,当输入电压增加 $ 20 \ V $ 时,
输出电压 \xzanswer{D} 

\fourchoices
{降低 $ 2 \ V $}
{增加 $ 2 \ V $}
{降低 $ 200 \ V $}
{增加 $ 200 \ V $}



\item 
\exwhere{$ 2015 $ 年江苏卷}
一电器中的变压器可视为理想变压器,它将 $ 220 \ V $ 交变电流改变为 $ 110 \ V $. 已知
变压器原线圈匝数为 $ 800 $,则副线圈匝数为 \xzanswer{B} 

\fourchoices
{$200 $}
{$400 $}
{$1600 $}
{$3200 $}


\item 
\exwhere{$ 2016 $ 年新课标 \lmd{3} 卷}
如图,理想变压器原、副线圈分别接有额定电压相同的灯泡 $ a $ 和 $ b $。当
输入电压 $ U $ 为灯泡额定电压的 $ 10 $ 倍时,两灯泡均能正常发
光。下列说法正确的是 \xzanswer{AD} 
\begin{figure}[h!]
\centering
\includesvg[width=0.23\linewidth]{picture/svg/GZ-3-tiyou-1156}
\end{figure}

\fourchoices
{原、副线圈砸数之比为 $ 9:1 $}
{原、副线圈砸数之比为 $ 1:9 $}
{此时 $ a $ 和 $ b $ 的电功率之比为 $ 9:1 $}
{此时 $ a $ 和 $ b $ 的电功率之比为 $ 1:9 $}

\item 
\exwhere{$ 2016 $ 年江苏卷}
一自耦变压器如图所示,环形铁芯上只绕有一个线圈,将其接在 $ a $、$ b $ 间作为
原线圈.通过滑动触头取该线圈的一部分,接在 $ c $、$ d $ 间作为副线圈。在 $ a $、$ b $ 间输入电压为 $ U_{1} $ 的
交变电流时,$ c $、$ d $ 间的输出电压为 $ U_{2} $,在将滑动触头从 $ M $ 点顺时针旋转到 $ N $ 点的过程中 \xzanswer{C} 
\begin{figure}[h!]
\centering
\includesvg[width=0.23\linewidth]{picture/svg/GZ-3-tiyou-1157}
\end{figure}

\fourchoices
{$ U_{2} > U_{1} $,$ U_{2} $ 降低}
{$ U_{2} > U_{1} $,$ U_{2} $ 升高}
{$ U_{2} < U_{1} $,$ U_{2} $ 降低}
{$ U_{2} < U_{1} $,$ U_{2} $ 升高}


\item 
\exwhere{$ 2016 $ 年四川卷}
如图所示,接在家庭电路上的理想降压变压器给小灯泡 $ L $ 供电,如果将原、副
线圈减少相同匝数,其它条件不变,则 \xzanswer{B} 
\begin{figure}[h!]
\centering
\includesvg[width=0.23\linewidth]{picture/svg/GZ-3-tiyou-1158}
\end{figure}

\fourchoices
{小灯泡变亮}
{小灯泡变暗}
{原、副线圈两端电压的比值不变}
{通过原、副线圈电流的比值不变}


\item 
\exwhere{$ 2016 $ 年天津卷}
$ 5 $、如图所示,理想变压器原线圈接在交流电源上,图中各电表均为理想电表。下
列说法正确的是 \xzanswer{B} 
\begin{figure}[h!]
\centering
\includesvg[width=0.23\linewidth]{picture/svg/GZ-3-tiyou-1159}
\end{figure}

\fourchoices
{当滑动变阻器的滑动触头 $ P $ 向上滑动时,$ R_{1} $ 消耗的功率变大}
{当滑动变阻器的滑动触头 $ P $ 向上滑动时,电压表 $ V $ 示数变大}
{当滑动变阻器的滑动触头 $ P $ 向上滑动时,电流表 $ A_{1} $ 示数变大}
{若闭合开关 $ S $,则电流表 $ A_{1} $ 示数变大,$ A_{2} $ 示数变大}




\item 
\exwhere{$ 2016 $ 年海南卷}
图($ a $)所示,理想变压器的原、副线圈的匝数比为 $ 4:1 $,$ R_T $ 为阻值随温度升高
而减小的热敏电阻,$ R_{1} $ 为定值电阻,电压
表和电流表均为理想交流电表。原线圈所
接电压 $ u $ 随时间 $ t $ 按正弦规律变化,如图
($ b $)所示。下列说法正确的是 \xzanswer{BD} 
\begin{figure}[h!]
\centering
\begin{subfigure}{0.4\linewidth}
\centering
\includesvg[width=0.7\linewidth]{picture/svg/GZ-3-tiyou-1160} 
\caption{}\label{}
\end{subfigure}
\begin{subfigure}{0.4\linewidth}
\centering
\includesvg[width=0.7\linewidth]{picture/svg/GZ-3-tiyou-1161} 
\caption{}\label{}
\end{subfigure}
\end{figure}

\fourchoices
{变压器输入、输出功率之比为 $ 4:1 $}
{变压器原、副线圈中的电流强度之比为 $ 1:4 $}
{$ u $ 随 $ t $ 变化的规律为 $ u=51 \sin (50 \pi t) $ (国际单位制)}
{若热敏电阻 $ R_T $ 的温度升高,则电压表的示数不变,电流表的示数变大}




\item 
\exwhere{$ 2011 $ 年新课标卷}
如图,一理想变压器原副线圈的匝数比为 $ 1:2 $;副线圈电路中接有灯泡,灯泡的额定电压为
$ 220 \ V $,额定功率为 $ 22 \ W $;原线圈电路中接有电压表和电流表。现闭合开关,灯泡正常发光。若用
$ U $ 和 $ I $ 分别表示此时电压表和电流表的读数,则 \xzanswer{A} 
\begin{figure}[h!]
\centering
\includesvg[width=0.23\linewidth]{picture/svg/GZ-3-tiyou-1162}
\end{figure}

\fourchoices
{$ U=110 \ V $,$ I=0.2 \ A $}
{$ U=110 \ V $,$ I=0.05 \ A $}
{$ U=110\sqrt{2} \ V ,I=0.2 \ A $}
{$ U=110\sqrt{2} \ V ,I=0.2\sqrt{2} \ A $}




\item 
\exwhere{$ 2011 $ 年理综广东卷}
图$ (a) $左侧的调压装置可视为理想变压器,负载电路中 $ R=55 \ \Omega $, \ammetermytikz 、 \voltmetermytikz 
为理想电流表和电压表,
若原线圈接入如图$ (b) $所示的正弦交变电压,电压表的示数为 $ 110 \ V $,下列表述正确的是 \xzanswer{AC} 
\begin{figure}[h!]
\centering
\begin{subfigure}{0.4\linewidth}
\centering
\includesvg[width=0.7\linewidth]{picture/svg/GZ-3-tiyou-1163} 
\caption{}\label{}
\end{subfigure}
\begin{subfigure}{0.4\linewidth}
\centering
\includesvg[width=0.7\linewidth]{picture/svg/GZ-3-tiyou-1164} 
\caption{}\label{}
\end{subfigure}
\end{figure}

\fourchoices
{电流表的示数为 $ 2 \ A $}
{原、副线圈匝数比为 $ 1:2 $}
{电压表的示数为电压的有效值}
{原线圈中交变电压的频率为 $ 100 \ Hz $}

\item 
\exwhere{$ 2014 $ 年物理海南卷}
理想变压器上接有三个完全相同的灯泡,其中一个与该变压器的原线圈串联后接入交流电源,
另外两个并联后接在副线圈两端。已知三个灯泡均正常发光。该变压器原、副线圈的匝数之比为 \xzanswer{B} 

\fourchoices
{$ 1:2 $}
{$ 2:l $}
{$ 2:3 $}
{$ 3:2 $}


\item 
\exwhere{$ 2014 $ 年理综山东卷}
如图,将额定电压为 $ 60 \ V $ 的用电器,通过一理想变压器接在正弦交变电源上。闭合开关 $ S $
后,用电器正常工作,交流电压表和交流电流表(均为理想电表)的示数分别为 $ 220 \ V $ 和 $ 2.2 \ A $。以
下判断正确的是 \xzanswer{BD} 
\begin{figure}[h!]
\centering
\includesvg[width=0.23\linewidth]{picture/svg/GZ-3-tiyou-1165}
\end{figure}


\fourchoices
{变压器输入功率为 $ 484 \ W $}
{通过原线圈的电流的有效值为 $ 6.6 \ A $}
{通过副线圈的电流的最大值为 $ 2.2 \ A $}
{变压器原、副线圈匝数比 $ n_{1} : n_{2} =11:3 $}



\item 
\exwhere{$ 2013 $ 年广东卷}
如图 $ 5 $,理想变压器原、副线圈匝数比 $ n_{1} : n_{2} =2:1,V $ 和 $ A $ 均为理想电表,灯泡电阻 $ R_{L} =6 \ \Omega ,AB $
端电压 $ u1=12\sqrt{2} \sin 100 \pi t $($ V $).下列说法正确的是 \xzanswer{D} 
\begin{figure}[h!]
\centering
\includesvg[width=0.23\linewidth]{picture/svg/GZ-3-tiyou-1166}
\end{figure}

\fourchoices
{电流频率为 $ 100 \ Hz $}
{$ V $ 的读数为 $ 24 \ V $}
{$ A $ 的读数为 $ 0.5 \ A $}
{变压器输入功率为 $ 6 \ W $}



\item 
\exwhere{$ 2012 $ 年理综重庆卷}
如图所示,理想变压器的原线圈接入 $ u =11000\sqrt{2} \sin 100 \pi t \ (V) $的交变电压,副线圈通
过电阻 $ r=6 \ \Omega $导线对“$ 220 \ V \quad 880 \ W $”电器 $ R_{L} $ 供电,该电器正常工
作。由此可知 \xzanswer{C} 
\begin{figure}[h!]
\centering
\includesvg[width=0.23\linewidth]{picture/svg/GZ-3-tiyou-1167}
\end{figure}

\fourchoices
{原、副线圈的匝数比为 $ 50:1 $}
{交变电压的频率为 $ 100HZ $}
{副线圈中电流的有效值为 $ 4 \ A $}
{变压器的输入功率为 $ 880 \ W $}



\item 
\exwhere{$ 2011 $ 年海南卷}
如图,理想变压器原线圈与 $ 10 \ V $ 的交流电源相连,副线圈并联
两个小灯泡 $ a $ 和 $ b $,小灯泡 $ a $ 的额定功率为 $ 0.3 \ W $,正常发光时电阻
为 $ 30 \ \Omega $可,已知两灯泡均正常发光,流过原线圈的电流为 $ 0.09 \ A $,
可计算出原、副线圈的匝数比为 \underlinegap ,流过灯泡 $ b $ 的电流为 \underlinegap $ A $。
\begin{figure}[h!]
\centering
\includesvg[width=0.23\linewidth]{picture/svg/GZ-3-tiyou-1168}
\end{figure}


\tk{$ 10:3 $ \quad $ 0.2 \ A $} 




\item 
\exwhere{$ 2011 $ 年理综浙江卷}
如图所示,在铁芯上、下分别绕有匝数 $ n_{1} =800 $ 和 $ n_{2} =200 $ 的两个线圈,上
线圈两端与 $ u=51 \sin 314t \ V $ 的交流电源相连,将下线圈两端接交流电压表,则交
流电压表的读数可能是 \xzanswer{A} 
\begin{figure}[h!]
\centering
\includesvg[width=0.23\linewidth]{picture/svg/GZ-3-tiyou-1169}
\end{figure}

\fourchoices
{$ 2.0 \ V $}
{$ 9.0 \ V $}
{$ 12.7 \ V $}
{$ 144.0 \ V $}


\item 
\exwhere{$ 2012 $ 年物理海南卷}
如图,理想变压器原、副线圈匝数比为 $ 20:1 $,两个标有“$ 12 \ V $,$ 6 \ W $”的小灯泡并联在副线圈的
两端。当两灯泡都正常工作时,原线圈电路中电压表和电流表(可视
为理想的)的示数分别是 \xzanswer{D} 
\begin{figure}[h!]
\centering
\includesvg[width=0.23\linewidth]{picture/svg/GZ-3-tiyou-1170}
\end{figure}


\fourchoices
{$ 120 \ V $,$ 0.10 \ A $}
{$ 240 \ V $,$ 0.025 \ A $}
{$ 120 \ V $,$ 0.05 \ A $}
{$ 240 \ V $,$ 0.05 \ A $}


\item 
\exwhere{$ 2013 $ 年四川卷}
用 $ 220 \ V $ 的正弦交流电通过理想变压器对一负载供电,变压器输出电压是 $ 110 \ V $,通过负载的电
流图像如图所示,则 \xzanswer{A} 
\begin{figure}[h!]
\centering
\includesvg[width=0.23\linewidth]{picture/svg/GZ-3-tiyou-1171}
\end{figure}



\fourchoices
{变压器输入功率约为 $ 3.9 \ W $}
{输出电压的最大值是 $ 110 \ V $}
{变压器原、副线圈匝数比是 $ 1:2 $}
{负载电流的函数表达式:$i=0.05 \sin \left(100 \pi t+\frac{\pi}{2}\right) \ A$}



\item 
\exwhere{$ 2012 $ 年理综四川卷}
如图所示,在铁芯 $ P $ 上绕着两个线圈 $ a $ 和 $ b $,则 \xzanswer{D} 
\begin{figure}[h!]
\centering
\includesvg[width=0.23\linewidth]{picture/svg/GZ-3-tiyou-1172}
\end{figure}


\fourchoices
{线圈 $ a $ 输入正弦交变电流,线圈 $ b $ 可输出恒定电流}
{线圈 $ a $ 输入恒定电流,穿过线圈 $ b $ 的磁通量一定为零}
{线圈 $ b $ 输出的交变电流不对线圈 $ a $ 的磁场造成影响}
{线圈 $ a $ 的磁场变化时,线圈 $ b $ 中一定有电场}


\item 
\exwhere{$ 2013 $ 年天津卷}
普通的交流电流表不能直接接在高压输电线路上测量电流,通常要通过电流互感器来连接。图
中电流互感器 $ ab $ 一侧线圈的匝数较少,工作时电流为 $ I_{ab} $,$ cd $ 一侧线圈的匝数较多,工作时电流为
$ I_{cd} $,为了使电流表能正常工作,则 \xzanswer{B} 
\begin{figure}[h!]
\centering
\begin{subfigure}{0.4\linewidth}
\centering
\includesvg[width=0.7\linewidth]{picture/svg/GZ-3-tiyou-1173} 
\caption{}\label{}
\end{subfigure}
\begin{subfigure}{0.4\linewidth}
\centering
\includesvg[width=0.7\linewidth]{picture/svg/GZ-3-tiyou-1174} 
\caption{}\label{}
\end{subfigure}
\end{figure}



\fourchoices
{$ ab $ 接 $ MN $、$ cd $ 接 $ PQ $,$ I_{ab} < I_{cd} $}
{$ ab $ 接 $ MN $、$ cd $ 接 $ PQ $,$ I_{ab} > I_{cd} $}
{$ ab $ 接 $ PQ $、$ cd $ 接 $ MN $,$ I_{ab} < I_{cd} $}
{$ ab $ 接 $ PQ $、$ cd $ 接 $ MN $,$ I_{ab} > I_{cd} $}


\item 
\exwhere{$ 2013 $年江苏卷}
如图所示,理想变压器原线圈接有交流电源,当副线圈上的滑片$ P $ 处于图示位置时,灯泡$ L $ 能发光. 要
使灯泡变亮,可以采取的方法有 \xzanswer{BC} 
\begin{figure}[h!]
\centering
\includesvg[width=0.23\linewidth]{picture/svg/GZ-3-tiyou-1175}
\end{figure}

\fourchoices
{向下滑动$ P $}
{增大交流电源的电压}
{增大交流电源的频率}
{减小电容器 $ C $ 的电容}


\item 
\exwhere{$ 2012 $ 年理综新课标卷}
自耦变压器铁芯上只绕有一个线圈,原、副线圈都只取该线圈的某部分,一升压式自耦调压变
压器的电路如图所示,其副线圈匝数可调。已知变压器线圈总匝数为 $ 1900 $ 匝;原线圈为 $ 1100 $ 匝,
接在有效值为 $ 220 \ V $ 的交流电源上。当变压器输出电压调至最大时,负载 $ R $ 上的功率为 $ 2.0 \ kW $。设
此时原线圈中电流有效值为 $ I_{1} $,负载两端电压的有效值为 $ U_{2} $,且变压器是理想的,则 $ U_{2} $ 和 $ I_{1} $ 分别
约为 \xzanswer{B} 
\begin{figure}[h!]
\centering
\includesvg[width=0.23\linewidth]{picture/svg/GZ-3-tiyou-1176}
\end{figure}


\fourchoices
{$ 380 \ V $ 和 $ 5.3 \ A $}
{$ 380 \ V $ 和 $ 9.1 \ A $}
{$ 240 \ V $ 和 $ 5.3 \ A $}
{$ 240 \ V $ 和 $ 9.1 \ A $}


\item 
\exwhere{$ 2012 $年理综山东卷}
图甲是某燃气炉点火装置的原理图。转换器将直流电压转换为图乙所示的正弦交变电压,并
加在一理想变压器的原线圈上,变压器原、副线圈的匝数分别为$ n_{1} $、$ n_{2} $。$ V $为交流电压表。当变压
器副线圈电压的瞬时值大于$ 5000 \ V $时,就会在钢针和金属板间引发电火花进而点燃气体。以下判断
正确的是 \xzanswer{BC} 
\begin{figure}[h!]
\centering
\begin{subfigure}{0.4\linewidth}
\centering
\includesvg[width=0.7\linewidth]{picture/svg/GZ-3-tiyou-1177} 
\caption{}\label{}
\end{subfigure}
\begin{subfigure}{0.4\linewidth}
\centering
\includesvg[width=0.7\linewidth]{picture/svg/GZ-3-tiyou-1178} 
\caption{}\label{}
\end{subfigure}
\end{figure}



\fourchoices
{电压表的示数等于$ 5 \ V $}
{电压表的示数等于$\frac{5}{\sqrt{2}} \ V$}
{实现点火的条件是 $\frac{n_{2}}{n_{1}}>1000$}
{实现点火的条件是 $\frac{n_{2}}{n_{1}}<1000$}


\item 
\exwhere{$ 2012 $ 年物理江苏卷}
某同学设计的家庭电路保护装置如图所示,铁芯左侧线圈
$ L_{1} $ 由火线和零线并行绕成. 当右侧线圈 $ L_{2} $ 中产生电流时,电
流经放大器放大后,使电磁铁吸起铁质开关 $ K $,从而切断家庭
电路. 仅考虑 $ L_{1} $ 在铁芯中产生的磁场,下列说法正确的有 \xzanswer{ABD} 
\begin{figure}[h!]
\centering
\includesvg[width=0.23\linewidth]{picture/svg/GZ-3-tiyou-1179}
\end{figure}

\fourchoices
{家庭电路正常工作时,$ L_{2} $ 中的磁通量为零}
{家庭电路中使用的电器增多时,$ L_{2} $ 中的磁通量不变}
{家庭电路发生短路时,开关 $ K $ 将被电磁铁吸起}
{地面上的人接触火线发生触电时,开关 $ K $ 将被电磁铁吸起}



\item 
\exwhere{$ 2012 $ 年理综福建卷}
如图,理想变压器原线圈输入电压 $ u= U_{m} \sin \omega t $,副线圈
电路中 $ R_{0} $ 为定值电阻,$ R $ 是滑动变阻器。$ V_{1} $ 和 $ V_{2} $ 是理想
交流电压表,示数分别用 $ U_{1} $ 和 $ U_{2} $ 表示;$ A_{1} $ 和 $ A_{2} $ 是理想交
流电流表,示数分别用 $ I_{1} $ 和 $ I_{2} $ 表示。下列说法正确的是 \xzanswer{C} 
\begin{figure}[h!]
\centering
\includesvg[width=0.23\linewidth]{picture/svg/GZ-3-tiyou-1180}
\end{figure}


\fourchoices
{$ I_{1} $ 和 $ I_{2} $ 表示电流的瞬间值}
{$ U_{1} $ 和 $ U_{2} $ 表示电压的最大值}
{滑片 $ P $ 向下滑动过程中,$ U_{2} $ 不变、$ I_{1} $ 变大}
{滑片 $ P $ 向下滑动过程中,$ U_{2} $ 变小、$ I_{1} $ 变小}


\item 
\exwhere{$ 2011 $ 年理综山东卷}
为保证用户电压稳定在 $ 220 \ V $,变电所需适时进行调压,图甲为调压变压器示意图。保持输入电
压 $ u_{1} $ 不变,当滑动接头 $ P $ 上下移
动时可改变输出电压。某次检测
得到用户电压 $ u_{2} $ 随时间 $ t $ 变化的
曲线如图乙所示。以下正确的是 \xzanswer{BD} 
\begin{figure}[h!]
\centering
\begin{subfigure}{0.4\linewidth}
\centering
\includesvg[width=0.7\linewidth]{picture/svg/GZ-3-tiyou-1181} 
\caption{}\label{}
\end{subfigure}
\begin{subfigure}{0.4\linewidth}
\centering
\includesvg[width=0.7\linewidth]{picture/svg/GZ-3-tiyou-1182} 
\caption{}\label{}
\end{subfigure}
\end{figure}



\fourchoices
{$u_{2}=190 \sqrt{2} \sin (50 \pi t) \ V$}
{$u_{2}=190 \sqrt{2} \sin (100 \pi t) \ V$}
{为使用户电压稳定在 $ 220 \ V $,应将 $ P $ 适当下移}
{为使用户电压稳定在 $ 220 \ V $,应将 $ P $ 适当上移}


\item 
\exwhere{$ 2014 $ 年理综广东卷}
如图 $ 11 $ 所示的电路中,$ P $ 为滑动变阻器的滑片,保持理想变压器的输入电压 $ U_{1} $ 不变,闭合电
键 $ S $,下列说法正确的是 \xzanswer{BD} 
\begin{figure}[h!]
\centering
\includesvg[width=0.23\linewidth]{picture/svg/GZ-3-tiyou-1183}
\end{figure}

\fourchoices
{$ P $ 向下滑动时,灯 $ L $ 变亮}
{$ P $ 向下滑动时,变压器的输出电压不变}
{$ P $ 向上滑动时,变压器的输入电流变小}
{$ P $ 向上滑动时,变压器的输出功率变大}



\item 
\exwhere{$ 2014 $ 年理综新课标\lmd{2}卷}
如图,一理想变压器原、副线圈的匝数分别为 $ n_{1} $、$ n_{2} $。原线圈通过一理想电流表 $ A $ 接正弦交流
电源,一个二极管和阻值为 $ R $ 的负载电阻串联后接到副线圈的两端;假设该二极管的正向电阻为
零,反向电阻为无穷大;用交流电压表测得 $ a $、$ b $ 端和 $ c $、$ d $ 端的电压分别为 $ U_{ab} $ 和 $ U_{cd} $,则 \xzanswer{BD} 
\begin{figure}[h!]
\centering
\includesvg[width=0.23\linewidth]{picture/svg/GZ-3-tiyou-1184}
\end{figure}

\fourchoices
{$ U_{ab} : U_{cd} = n_{1} : n_{2} $}
{增大负载电阻的阻值 $ R $,电流表的读数变小}
{负载电阻的阻值越小,$ cd $ 间的电压 $ U_{cd} $ 越大}
{将二极管短路,电流表的读数加倍}



\item
\exwhere{$ 2011 $ 年物理江苏卷}
图 $ 1 $ 为一理想变压器,$ ab $ 为原线圈,$ ce $ 为副线圈,$ d $ 为副线圈引出的一个接头,原线
圈输入正弦式交变电压的 $ u-t $ 图象如图 $ 2 $ 所示。若只在 $ ce $ 间接一只 $ R_{ce}=400 \ \Omega $的电阻,或只在 $ de $ 间
接一只 $ R_{de}=225 \ \Omega $的电阻,两种情况下电阻消耗的功率均为 $ 80 \ W $。
\begin{enumerate}
%\renewcommand{\labelenumi}{\arabic{enumi}.}
% A(\Alph) a(\alph) I(\Roman) i(\roman) 1(\arabic)
%设定全局标号series=example	%引用全局变量resume=example
%[topsep=-0.3em,parsep=-0.3em,itemsep=-0.3em,partopsep=-0.3em]
%可使用leftmargin调整列表环境左边的空白长度 [leftmargin=0em]
\item
请写出原线圈输入电压瞬时值 $ u_{ab} $ 的表达式;
\item 
求只在 $ ce $ 间接 $ 400 \ \Omega $的电阻时,原线圈中的电流
$ I_{1} $;
\item 
求 $ ce $ 和 $ de $ 间线圈的匝数比$\frac{n_{c e}}{n_{d e}}$。


\end{enumerate}
\begin{figure}[h!]
\centering
\begin{subfigure}{0.4\linewidth}
\centering
\includesvg[width=0.7\linewidth]{picture/svg/GZ-3-tiyou-1185} 
\caption{}\label{}
\end{subfigure}
\begin{subfigure}{0.4\linewidth}
\centering
\includesvg[width=0.7\linewidth]{picture/svg/GZ-3-tiyou-1186} 
\caption{}\label{}
\end{subfigure}
\end{figure}




\banswer{
\begin{enumerate}
%\renewcommand{\labelenumi}{\arabic{enumi}.}
% A(\Alph) a(\alph) I(\Roman) i(\roman) 1(\arabic)
%设定全局标号series=example	%引用全局变量resume=example
%[topsep=-0.3em,parsep=-0.3em,itemsep=-0.3em,partopsep=-0.3em]
%可使用leftmargin调整列表环境左边的空白长度 [leftmargin=0em]
\item
$u_{a b}=400 \sin 200 \pi t \ (V)$
\item 
$I_{1} \approx 0.28 A \quad\left(\right.$ 或 $\left.\frac{\sqrt{2}}{5} A\right)$
\item 
$\frac{n_{c e}}{n_{d e}}=\frac{4}{3}$
\end{enumerate}
}


\item 
\exwhere{$ 2011 $ 年理综福建卷}
图甲中理想变压器原、副线圈的匝数之比 $ n_{1} : n_{2} =5:1 $,电阻 $ R=20 \ \Omega $,$ L_{1} $、$ L_{2} $ 为规格相同的两
只小灯泡,$ S_{1} $ 为单刀双掷开关。原线圈接
正弦交变电源,输入电压 $ u $ 随时间 $ t $ 的变
化关系如图乙所示。现将 $ S_{1} $ 接 $ 1 $、$ S_{2} $ 闭
合,此时 $ L_{2} $ 正常发光。下列说法正确的
是 \xzanswer{D} 
\begin{figure}[h!]
\centering
\begin{subfigure}{0.4\linewidth}
\centering
\includesvg[width=0.7\linewidth]{picture/svg/GZ-3-tiyou-1187} 
\caption{}\label{}
\end{subfigure}
\begin{subfigure}{0.4\linewidth}
\centering
\includesvg[width=0.7\linewidth]{picture/svg/GZ-3-tiyou-1188} 
\caption{}\label{}
\end{subfigure}
\end{figure}

\fourchoices
{输入电压 $ u $ 的表达式 $ u=20\sqrt{2} \sin (50 \pi t) \ V $}
{只断开 $ S_{2} $ 后,$ L_{1} $、$ L_{2} $ 均正常发光}
{只断开 $ S_{2} $ 后,原线圈的输入功率增大}
{若 $ S_{1} $ 换接到 $ 2 $ 后,$ R $ 消耗的电功率为 $ 0.8 \ W $}



\item 
\exwhere{$ 2015 $ 年理综新课标 \lmd{1} 卷}
一理想变压器的原、副线圈的匝数比为 $ 3:1 $,在原、副线圈的回路中
分别接有阻值相同的电阻,原线圈一侧接在电压为 $ 220 \ V $ 的正弦交流电源上,如图所示。设副线圈
回路中电阻两端的电压为 $ U $,原、副线圈回路中电阻消耗的功率的比值为 $ k $,则 \xzanswer{A} 
\begin{figure}[h!]
\centering
\includesvg[width=0.23\linewidth]{picture/svg/GZ-3-tiyou-1189}
\end{figure}

\fourchoices
{$ U=66 \ V ,k= \frac{ 1 }{ 9 } $}
{$ U=22 \ V ,k= \frac{ 1 }{ 9 } $}
{$ U=66 \ V ,k= \frac{ 1 }{ 3 } $}
{$ U=22 \ V ,k= \frac{ 1 }{ 3 } $}


\item 
\exwhere{$ 2015 $ 年理综天津卷}
如图所示,理想变压器的原线圈连接一只理想交流电流表,副线圈匝
数可以通过滑动触头 $ Q $ 来调节,在副线圈两端连接了定值电阻 $ R_{0} $ 和滑动变阻器 $ R $,$ P $ 为滑动变
阻器的滑动触头,在原线圈上加一电压为 $ U $ 的正弦交流电,则 \xzanswer{BC} 
\begin{figure}[h!]
\centering
\includesvg[width=0.23\linewidth]{picture/svg/GZ-3-tiyou-1190}
\end{figure}

\fourchoices
{保持 $ Q $ 的位置不动,将 $ P $ 向上滑动时,电流表读数变大}
{保持 $ Q $ 的位置不动,将 $ P $ 向上滑动时,电流表读数变小}
{保持 $ P $ 的位置不动,将 $ Q $ 向上滑动时,电流表读数变大}
{保持 $ P $ 的位置不动,将 $ Q $ 向上滑动时,电流表读数变小}


\item 
\exwhere{$ 2015 $ 年理综安徽卷}
图示电路中,变压器为理想变压器,$ a $、$ b $ 接在电压有效值不变的交流电
源两端,$ R_{0} $ 为定值电阻,$ R $ 为滑动变阻器。现将变阻器的滑片从一个位置滑动到另一位置,观察到
电流表 $ A_{1} $ 的示数增大了 $ 0.2 \ A $,电流表 $ A_{2} $ 的示数增大了 $ 0.8 \ A $,则下列说法正确的是 \xzanswer{D} 
\begin{figure}[h!]
\centering
\includesvg[width=0.23\linewidth]{picture/svg/GZ-3-tiyou-1191}
\end{figure}

\fourchoices
{电压表 $ V_{1} $ 示数增大}
{电压表 $ V_{2} $、$ V_{3} $ 示数均增大}
{该变压器起升压作用}
{变阻器滑片是沿 $ c \rightarrow d $ 的方向滑动}

\item 
\exwhere{$ 2015 $ 年海南卷}
如图,一理想变压器原、副线圈匝数比为 $ 4:1 $,原线圈与一可变电阻串联
后,接入一正弦交流电源;副线圈电路中固定电阻的阻值为 $ R_{0} $,负载电阻的阻值 $ R=11 R_{0} $,$ V $ 是理
想电压表;现将负载电阻的阻值减小为 $ R=5 R_{0} $,保持变压器输入电流不变,此时电压表读数为$ 5.0 \ V $,则 \xzanswer{AD} 
\begin{figure}[h!]
\centering
\includesvg[width=0.23\linewidth]{picture/svg/GZ-3-tiyou-1192}
\end{figure}

\fourchoices
{此时原线圈两端电压的最大值约为 $ 34 \ V $}
{此时原线圈两端电压的最大值约为 $ 24 \ V $}
{原线圈两端原来的电压有效值约为 $ 68 \ V $}
{原线圈两端原来 的电压有效值约为 $ 48 \ V $}


\item 
\exwhere{$ 2015 $ 年广东卷}
下图为气流加热装置的示意图,使用电阻丝加热导气管,视变压器为理想变压
器,原线圈接入电压有效值恒定的交流电并保持匝数不变,调节触
头 $ P $,使输出电压有效值由 $ 220 \ V $ 降至 $ 110 \ V $。调节前后 \xzanswer{C} 
\begin{figure}[h!]
\centering
\includesvg[width=0.23\linewidth]{picture/svg/GZ-3-tiyou-1193}
\end{figure}

\fourchoices
{副线圈中的电流比为 $ 1:2 $}
{副线圈输出功率比为 $ 2:1 $}
{副线圈的接入匝数比为 $ 2:1 $}
{原线圈输入功率比为 $ 1:2 $}



\item 
\exwhere{$ 2018 $ 年天津卷}
教学用发电机能够产生正弦式交变电流。利用该发电机(内阻可忽略)通过
理想变压器向定值电阻 $ R $ 供电,电路如图所示,理想交流电流表 $ A $、理想交流电压表 $ V $ 的读数分
别为 $ I $、$ U $,$ R $ 消耗的功率为 $ P $。若发电机线圈的转速变为原来的$ \frac{ 1 }{ 2 } $,则 \xzanswer{B} 
\begin{figure}[h!]
\centering
\includesvg[width=0.23\linewidth]{picture/svg/GZ-3-tiyou-1194}
\end{figure}


\fourchoices
{$ R $ 消耗的功率变为$ \frac{ 1 }{ 2 } P $}
{电压表 $ V $ 的读数为 $ \frac{ 1 }{ 2 } U $}
{电流表 $ A $ 的读数变为 $ 2 I $}
{通过 $ R $ 的交变电流频率不变}



\item 
\exwhere{$ 2017 $ 年北京卷}
如图所示,理想变压器的原线圈接在 $ u=220\sqrt{2} \sin \pi t \ (V) $ 的交流电源上,副
线圈接有 $ R=55 \ \Omega $ 的负载电阻,原、副线圈匝数之比为 $ 2:1 $,电流表、电压表均为理想电表。下列
说法正确的是 \xzanswer{B} 
\begin{figure}[h!]
\centering
\includesvg[width=0.23\linewidth]{picture/svg/GZ-3-tiyou-1195}
\end{figure}


\fourchoices
{原线圈的输入功率为 $ 220\sqrt{2} \ W $}
{电流表的读数为 $ 1 \ A $}
{电压表的读数为 $ 110\sqrt{2} \ V $}
{副线圈输出交流电的周期为 $ 50 \ s $}

\item 
\exwhere{$ 2016 $ 年新课标 \lmd{1} 卷}
一含有理想变压器的电路如图所示,图中电阻 $ R_{1} $、 $ R_{2} $ 和 $ R_{3} $ 的阻值分别是
$ 3 \ \Omega $、$ 1 \ \Omega $ 和 $ 4 \ \Omega $, \ammetermytikz 为理想交流电流表, $ U $ 为正弦交流电
压源,输出电压的有效值恒定。当开关 $ S $ 断开时,电流表
的示数为 $ I $;当 $ S $ 闭合时,电流表的示数为 $ 4I $。该变压器
原、副线圈匝数比为 \xzanswer{B} 
\begin{figure}[h!]
\centering
\includesvg[width=0.23\linewidth]{picture/svg/GZ-3-tiyou-1196}
\end{figure}

\fourchoices
{$ 2 $}
{$ 3 $}
{$ 4 $}
{$ 5 $}








\end{enumerate}

