\bta{第八讲$ \quad $带电粒子在电场中的运动}

\begin{enumerate}[leftmargin=0em]
\renewcommand{\labelenumi}{\arabic{enumi}.}
% A(\Alph) a(\alph) I(\Roman) i(\roman) 1(\arabic)
%设定全局标号series=example	%引用全局变量resume=example
%[topsep=-0.3em,parsep=-0.3em,itemsep=-0.3em,partopsep=-0.3em]
%可使用leftmargin调整列表环境左边的空白长度 [leftmargin=0em]
\item
\exwhere{$ 2015 $年海南卷}
如图,一充电后的平行板电容器的两极板相距$ l $,在正极板附近有一质量为$ M $、电荷量为$ q $($ q > 0 $)的粒子,在负极板附近有另一质量为$ m $、电荷量为$ -q $的粒子,在电场力的作用下,两粒子同时从静止开始运动。已知两粒子同时经过一平行于正极板且与其相距$ \frac{ 2 }{ 5 } l $的平面。若两粒子间相互作用力可忽略,不计重力,则$ M : m $为 \xzanswer{A} 
\begin{figure}[h!]
\centering
\includesvg[width=0.23\linewidth]{picture/svg/082}
\end{figure}

\fourchoices
{$ 3:2 $}
{$ 2:1 $}
{$ 5:2 $}
{$ 3:1 $}




\item
\exwhere{$ 2015 $年理综新课标\lmd{2}卷}
如图,两平行的带电金属板水平放置。若在两板中间$ a $点从静止释放一带电微粒,微粒恰好保持静止状态。现将两板绕过$ a $点的轴(垂直于纸面)逆时针旋转$ 45 ^{ \circ } $,再由$ a $点从静止释放一同样的微粒,该微粒将 \xzanswer{D} 
\begin{figure}[h!]
\centering
\includesvg[width=0.23\linewidth]{picture/svg/083}
\end{figure}


\fourchoices
{保持静止状态 }
{向左上方做匀加速运动}
{向正下方做匀加速运动}
{向左下方做匀加速运动}





\item
\exwhere{$ 2017 $年江苏卷}
如图所示,三块平行放置的带电金属薄板$ A $、$ B $、$ C $中央各有一小孔,小孔分别位于$ O $、$ M $、$ P $点。由$ O $点静止释放的电子恰好能运动到$ P $点,现将$ C $板向右平移到$ P ^{\prime} $点,则由$ O $点静止释放的电子 \xzanswer{A} 
\begin{figure}[h!]
\centering
\includesvg[width=0.23\linewidth]{picture/svg/084}
\end{figure}


\fourchoices
{运动到$ P $点返回}
{运动到$ P $和$ P ^{\prime} $点之间返回}
{运动到$ P ^{\prime} $点返回}
{穿过$ P ^{\prime} $点}





\item
\exwhere{$ 2017 $年天津卷}
如图所示,在点电荷$ Q $产生的电场中,实线$ MN $是一条方向未标出的电场线,虚线$ AB $是一个电子只在静电力作用下的运动轨迹。设电子在$ A $、$ B $两点的加速度大小分别为$ a_A $、$ a_B $,电势能分别为$ E_{pA} $、$ E_{pB} $。下列说法正确的是 \xzanswer{BC} 


\begin{minipage}[h!]{0.7\linewidth}
\vspace{0.3em}
\fourchoices
{电子一定从$ A $向$ B $运动}
{若$ a_A>a_B $,则$ Q $靠近$ M $端且为正电荷}
{无论$ Q $为正电荷还是负电荷一定有$ E_{pA}<E_{pB} $}
{$ B $点电势可能高于$ A $点电势}
\vspace{0.3em}
\end{minipage}
\hfill
\begin{minipage}[h!]{0.3\linewidth}
\flushright
\vspace{0.3em}
\includesvg[width=0.7\linewidth]{picture/svg/085}
\vspace{0.3em}
\end{minipage}




\item
\exwhere{$ 2013 $年广东卷}
喷墨打印机的简化模型如图所示,重力可忽略的墨汁微滴,经带电室带负电后,以速度$ v $垂直匀强电场飞入极板间,最终打在纸上,则微滴在极板间电场中 \xzanswer{C} 
\begin{figure}[h!]
\centering
\includesvg[width=0.23\linewidth]{picture/svg/086}
\end{figure}



\fourchoices
{向负极板偏转}
{电势能逐渐增大}
{运动轨迹是抛物线}
{运动轨迹与带电量无关}





\item
\exwhere{$ 2011 $年理综安徽卷}
图$ (a) $为示波管的原理图。如果在电极$ YY ^{\prime} $ 之间所加的电压图按图$ (b) $所示的规律变化,在电极$ XX ^{\prime} $之间所加的电压按图$ (c) $所示的规律变化,则在荧光屏上会看到的图形是 \xzanswer{B} 
\begin{figure}[h!]
\centering
\includesvg[width=0.83\linewidth]{picture/svg/087}\\
\includesvg[width=0.83\linewidth]{picture/svg/088}
\end{figure}



\item
\exwhere{$ 2014 $年理综山东卷}
如图,场强大小为$ E $、方向竖直向下的匀强电场中有一矩形区域$ abcd $,水平边$ ab $长为$ s $,竖直边$ ad $长为$ h $。质量均为$ m $、带电量分别为$ +q $和-$ q $的两粒子,由$ a $、$ c $两点先后沿$ ab $和$ cd $方向以速率$ v_{0} $进入矩形区(两粒子不同时出现在电场中)。不计重力。若两粒子轨迹恰好相切,则$ v_{0} $等于 \xzanswer{B} 
\begin{figure}[h!]
\centering
\includesvg[width=0.23\linewidth]{picture/svg/089}
\end{figure}
\fourchoices
{$\frac { s } { 2 } \sqrt { \frac { 2 q E } { m h } }$}
{$\frac { s } { 2 } \sqrt { \frac { q E } { m h } }$}
{$\frac { s } { 4 } \sqrt { \frac { 2 q E } { m h } }$}
{$\frac { s } { 4 } \sqrt { \frac { q E } { m h } }$}





\item
\exwhere{$ 2011 $年物理江苏卷}
一粒子从$ A $点射入电场,从$ B $点射出,电场的等势面和粒子的运动轨迹如图所示,图中左侧前三个等势面彼此平行,不计粒子的重力。下列说法正确的有 \xzanswer{AB} 
\begin{figure}[h!]
\centering
\includesvg[width=0.23\linewidth]{picture/svg/090}
\end{figure}


\fourchoices
{粒子带负电荷}
{粒子的加速度先不变,后变小}
{粒子的速度不断增大}
{粒子的电势能先减小,后增大}





\item
\exwhere{$ 2011 $年理综安徽卷}
如图($ a $)所示,两平行正对的金属板$ A $、$ B $间加有如图($ b $)所示的交变电压,一重力可忽略不计的带正电粒子被固定在两板的正中间$ P $处。若在$ t_{0} $时刻释放该粒子,粒子会时而向$ A $板运动,时而向$ B $板运动,并最终打在$ A $板上。则$ t_{0} $可能属于的时间段是 \xzanswer{B} 
\begin{figure}[h!]
\centering
\includesvg[width=0.35\linewidth]{picture/svg/091}
\end{figure}
\fourchoices
{$0 < t _ { 0 } < \frac { T } { 4 }$}
{$\frac { T } { 2 } < t _ { 0 } < \frac { 3 T } { 4 }$}
{$\frac { 3 T } { 4 } < t _ { 0 } < T$}
{$T < t _ { 0 } < \frac { 9 T } { 8 }$}




\item
\exwhere{$ 2015 $年理综天津卷}
如图所示,氕核、氘核、氚核三种粒子从同一位置无初速度地飘入电场线水平向右的加速电场$ E_{1} $,之后进入电场线竖直向下的匀强电场$ E_{2} $发生偏转,最后打在屏上。整个装置处于真空中,不计粒子重力及其相互作用,那么 \xzanswer{AD} 
\begin{figure}[h!]
\centering
\includesvg[width=0.23\linewidth]{picture/svg/092}
\end{figure}


\fourchoices
{偏转电场$ E_{2} $对三种粒子做功一样多}
{三种粒子打到屏上时的速度一样大}
{三种粒子运动到屏上所用时间相同}
{三种粒子一定打到屏上的同一位置}





\item
\exwhere{$ 2016 $年海南卷}
如图,平行板电容器两极板的间距为$ d $,极板与水平面成$ 45 ^{ \circ } $角,上极板带正电。一电荷量为$ q $($ q>0 $)的粒子在电容器中靠近下极板处。以初动能$ E_{k0} $竖直向上射出。不计重力,极板尺寸足够大,若粒子能打到上极板,则两极板间电场强度的最大值为 \xzanswer{B} 
\begin{figure}[h!]
\centering
\includesvg[width=0.23\linewidth]{picture/svg/093}
\end{figure}
\fourchoices
{$\frac { E _ { \mathrm { k } 0 } } { 4 q d }$}
{$\frac { E _ { \mathrm { k } 0 } } { 2 q d }$}
{$\frac { \sqrt { 2 } E _ { \mathrm { k } 0 } } { 2 q d }$}
{$\frac { \sqrt { 2 } E _ { \mathrm { k } 0 } } { q d }$}





\item
\exwhere{$ 2018 $年江苏卷}
如图所示,水平金属板$ A $、$ B $分别与电源两极相连,带电油滴处于静止状态。现将$ B $板右端向下移动一小段距离,两金属板表面仍均为等势面,则该油滴 \xzanswer{D} 
\begin{figure}[h!]
\centering
\includesvg[width=0.23\linewidth]{picture/svg/094}
\end{figure}


\fourchoices
{仍然保持静止}
{竖直向下运动}
{向左下方运动}
{向右下方运动}





\item
\exwhere{$ 2018 $年天津卷}
如图所示,实线表示某电场的电场线(方向未标出),虚线是一带负电的粒子只在电场力作用下的运动轨迹,设$ M $点和$ N $点的电势分别为$ \varphi_{M} $、$ \varphi_{N} $,粒子在$ M $和$ N $时加速度大小分别为$ a_{M} $、$ a_{N} $,速度大小分别为$ v_{M} $、$ v_{N} $,电势能分别为$ E_{pM} $、$ E_{pN} $。下列判断正确的是 \xzanswer{D} 

\begin{minipage}[h!]{0.7\linewidth}
\vspace{0.3em}
\fourchoices
{$v _ { M } < v _ { N } , a _ { M } < a _ { N }$}
{$v _ { M } < v _ { N } , \varphi _ { M } < \varphi _ { N }$}
{$\varphi _ { M } < \varphi _ { N } , E _ { \mathrm { PM } } < E _ { \mathrm { PN } }$}
{$a _ { M } < a _ { N } , E _ { \mathrm { PM } } < E _ { \mathrm { PN } }$}

\vspace{0.3em}
\end{minipage}
\hfill
\begin{minipage}[h!]{0.3\linewidth}
\flushright
\vspace{0.3em}
\includesvg[width=0.8\linewidth]{picture/svg/095}
\vspace{0.3em}
\end{minipage}



\item
\exwhere{$ 2018 $年全国\lmd{3}卷}
如图,一平行板电容器连接在直流电源上,电容器的极板水平;两微粒$ a $、$ b $所带电荷量大小相等、符号相反,使它们分别静止于电容器的上、下极板附近,与极板距离相等。现同时释放$ a $、$ b $,它们由静止开始运动。在随后的某时刻,$ a $、$ b $经过电容器两极板间下半区域的同一水平面。$ a $、$ b $间的相互作用和重力可忽略。下列说法正确的是 \xzanswer{BD} 
\begin{figure}[h!]
\centering
\includesvg[width=0.23\linewidth]{picture/svg/096}
\end{figure}

\fourchoices
{$ a $的质量比$ b $的大}
{在$ t $时刻,$ a $的动能比$ b $的大}
{在$ t $时刻,$ a $和$ b $的电势能相等}
{在$ t $时刻,$ a $和$ b $的动量大小相等}


\item
\exwhere{$ 2019 $年$ 4 $月浙江物理选考}
质子疗法是用一定能量的质子束照射肿瘤杀死癌细胞。现用一直线加速器来加速质子,使其从静止开始被加速到$ 1.0 \times 10^7 \ m/s $。已知加速电场的场强为$ 1.3 \times 10^5 \ N/C $,质子的质量为$ 1.67 \times 10^{-27} \ kg $,电荷量为$ 1.6 \times 10^{-19}C $,则下列说法正确的是 \xzanswer{D} 



\fourchoices
{加速过程中质子电势能增加}
{质子所受到的电场力约为$ 2 \times 10^{-15} \ N $}
{质子加速需要的时间约为$ 8 \times 10^{-6} \ s $}
{加速器加速的直线长度约为$ 4 \ m $}





\item
\exwhere{$ 2019 $年物理江苏卷}
一匀强电场的方向竖直向上,$ t=0 $时刻,一带电粒子以一定初速度水平射入该电场,电场力对粒子做功的功率为$ P $,不计粒子重力,则$ P-t $关系图象是 \xzanswer{A} 
\begin{figure}[h!]
\centering
\includesvg[width=0.83\linewidth]{picture/svg/102}
\end{figure}




\item
\exwhere{$ 2016 $年四川卷}
中国科学家$ 2015 $年$ 10 $月宣布中国将在$ 2020 $年开始建造世界上最大的粒子加速器。加速器是人类揭示物质本源的关键设备,在放射治疗、食品安全、材料科学等方面有广泛应用。\\
如图所示,某直线加速器由沿轴线分布的一系列金属圆管(漂移管)组成,相邻漂移管分别接在高频脉冲电源的两极。质子从$ K $点沿轴线进入加速器并依此向右穿过各漂移管,在漂移管内做匀速直线运动,在漂移管间被电场加速,加速电压视为不变。设质子进入漂移管$ B $时速度为$ 8 \times 10^6 \ m/s $,进入漂移管$ E $时速度为$ 1 \times 10^7 \ m/s $,电源频率为$ 1 \times 10^7Hz $,漂移管间缝隙很小,质子在每个管内运动时间视为电源周期的$ 1/2 $.质子的荷质比取$ 1 \times 10^8\ C/kg $。求:
\begin{enumerate}
\renewcommand{\labelenumi}{\arabic{enumi}.}
% A(\Alph) a(\alph) I(\Roman) i(\roman) 1(\arabic)
%设定全局标号series=example	%引用全局变量resume=example
%[topsep=-0.3em,parsep=-0.3em,itemsep=-0.3em,partopsep=-0.3em]
%可使用leftmargin调整列表环境左边的空白长度 [leftmargin=0em]
\item
漂移管$ B $的长度;
\item 
相邻漂移管间的加速电压。


\end{enumerate}
\begin{figure}[h!]
\flushright
\includesvg[width=0.4\linewidth]{picture/svg/097}
\end{figure}

\banswer{
\begin{enumerate}
\renewcommand{\labelenumi}{\arabic{enumi}.}
% A(\Alph) a(\alph) I(\Roman) i(\roman) 1(\arabic)
%设定全局标号series=example	%引用全局变量resume=example
%[topsep=-0.3em,parsep=-0.3em,itemsep=-0.3em,partopsep=-0.3em]
%可使用leftmargin调整列表环境左边的空白长度 [leftmargin=0em]
\item
$ 0.4\ m $
\item 
$ 6\times 10^{4}\ V $		

\end{enumerate}
}

\newpage
\item
\exwhere{$ 2016 $年北京卷}
如图所示,电子由静止开始经加速电场加速后,沿平行于版面的方向射入偏转电场,并从另一侧射出。已知电子质量为$ m $,电荷量为$ e $,加速电场电压为$ U_{0} $,偏转电场可看做匀强电场,极板间电压为$ U $,极板长度为$ L $,板间距为$ d $。
\begin{enumerate}
\renewcommand{\labelenumi}{\arabic{enumi}.}
% A(\Alph) a(\alph) I(\Roman) i(\roman) 1(\arabic)
%设定全局标号series=example	%引用全局变量resume=example
%[topsep=-0.3em,parsep=-0.3em,itemsep=-0.3em,partopsep=-0.3em]
%可使用leftmargin调整列表环境左边的空白长度 [leftmargin=0em]
\item
忽略电子所受重力,求电子射入偏转电场时初速度$ v_{0} $和从电场射出时沿垂直版面方向的偏转距离$ \Delta y $;
\item 
分析物理量的数量级,是解决物理问题的常用方法。在解决($ 1 $)问时忽略了电子所受重力,请利用下列数据分析说明其原因。已知$ U=2.0 \times 10^2 \ V $,$ d=4.0 \times 10^{-2}m $,$ m=9.1 \times 10^{-31} \ kg $,$ e=1.6 \times 10^{-19}C $,重力加速度$ g=10 \ \ m/s ^{2} $。
\item 
极板间既有静电场,也有重力场。电势反映了静电场各点的能的性质,请写出电势$ \varphi $的定义式。类比电势的定义方法,在重力场中建立“重力势”$ \varphi _G $的概念,并简要说明电势和“重力势”的共同特点。


\end{enumerate}
\begin{figure}[h!]
\flushright
\includesvg[width=0.33\linewidth]{picture/svg/098}
\end{figure}
\banswer{
\begin{enumerate}
\renewcommand{\labelenumi}{\arabic{enumi}.}
% A(\Alph) a(\alph) I(\Roman) i(\roman) 1(\arabic)
%设定全局标号series=example	%引用全局变量resume=example
%[topsep=-0.3em,parsep=-0.3em,itemsep=-0.3em,partopsep=-0.3em]
%可使用leftmargin调整列表环境左边的空白长度 [leftmargin=0em]
\item
$\frac { U L ^ { 2 } } { 4 U _ { 0 } d }$
\item 
不需要考虑电子所受的重力
\item 
$\varphi = \frac { E _ { \mathrm { P } } } { q }$,电势$ \varphi $ 和重力势$ \varphi _G $都是反映场的能的性质的物理量,仅仅由场自身的因素决定。



\end{enumerate}
}


\newpage
\item
\exwhere{$ 2011 $年理综福建卷}
反射式速调管是常用的微波器件之一,它利用电子团在电场中的振荡来产生微波,其振荡原理与下述过程类似。如图所示,在虚线$ MN $两侧分别存在着方向相反的两个匀强电场,一带电微粒从$ A $点由静止开始,在电场力作用下沿直线在$ A $、$ B $两点间往返运动。已知电场强度的大小分别是$ E_{1} =2.0 \times 10^3 \ N/C $和$ E_{2} =4.0 \times 10^3 \ N/C $,方向如图所示。带电微粒质量$ m=1.0 \times 10^{-20 }\ kg $,带电量$ q=-1.0 \times 10^{-9}C $,$ A $点距虚线$ MN $的距离$ d_1=1.0 \ cm $,不计带电微粒的重力,忽略相对论效应。求:
\begin{enumerate}
\renewcommand{\labelenumi}{\arabic{enumi}.}
% A(\Alph) a(\alph) I(\Roman) i(\roman) 1(\arabic)
%设定全局标号series=example	%引用全局变量resume=example
%[topsep=-0.3em,parsep=-0.3em,itemsep=-0.3em,partopsep=-0.3em]
%可使用leftmargin调整列表环境左边的空白长度 [leftmargin=0em]
\item
$ B $点到虚线$ MN $的距离$ d_{2} $;
\item 
带电微粒从$ A $点运动到$ B $点所经历的时间$ t $。



\end{enumerate}
\begin{figure}[h!]
\flushright
\includesvg[width=0.35\linewidth]{picture/svg/099}
\end{figure}
\banswer{
\begin{enumerate}
\renewcommand{\labelenumi}{\arabic{enumi}.}
% A(\Alph) a(\alph) I(\Roman) i(\roman) 1(\arabic)
%设定全局标号series=example	%引用全局变量resume=example
%[topsep=-0.3em,parsep=-0.3em,itemsep=-0.3em,partopsep=-0.3em]
%可使用leftmargin调整列表环境左边的空白长度 [leftmargin=0em]
\item
$ 0.50\ cm $
\item 
$ 1.5\time10^{-8}\ s $


\end{enumerate}
}



\item
\exwhere{$ 2015 $年理综新课标\lmd{2}卷}
如图,一质量为$ m $、电荷量为$ q $($ q>0 $)的粒子在匀强电场中运动,$ A $、$ B $为其运动轨迹上的两点。已知该粒子在$ A $点的速度大小为$ v_{0} $,方向与电场方向的夹角为$ 60 ^{ \circ } $;它运动到$ B $点时速度方向与电场方向的夹角为$ 30 ^{ \circ } $。不计重力。求$ A $、$ B $两点间的电势差。
\begin{figure}[h!]
\flushright
\includesvg[width=0.23\linewidth]{picture/svg/100}
\end{figure}


\banswer{
$U _ { A B } = \frac { m v _ { 0 } ^ { 2 } } { q }$
}




\item
\exwhere{$ 2013 $年新课标\lmd{2}卷}
如图,匀强电场中有一半径为$ r $的光滑绝缘圆轨道,轨道平面与电场方向平行。$ a $、$ b $为轨道直径的两端,该直径与电场方向平行。一电荷量为$ q $($ q>0 $)的质点沿轨道内侧运动,经过$ a $点和$ b $点时对轨道压力的大小分别为$ N_a $和$ N_b $,不计重力,求电场强度的大小$ E $、质点经过$ a $点和$ b $点时的动能。
\begin{figure}[h!]
\flushright
\includesvg[width=0.25\linewidth]{picture/svg/101}
\end{figure}

\banswer{
$E = \frac { 1 } { 6 q } \left( N _ { b } - N _ { a } \right)$,$E _ { k a } = \frac { r } { 12 } \left( N _ { b } + 5 N _ { a } \right)$,$E _ { k b } = \frac { r } { 12 } \left( 5 N _ { b } + N _ { a } \right)$.
}






\item
\exwhere{$ 2019 $年物理全国\lmd{2}卷}
如图,两金属板$ P $、$ Q $水平放置,间距为$ d $。两金属板正中间有一水平放置的金属网$ G $,$ PQG $的尺寸相同。$ G $接地,$ PQ $的电势均为$ \varphi $($ \varphi>0 $)。质量为$ m $,电荷量为$ q $($ q>0 $)的粒子自$ G $的左端上方距离$ G $为$ h $的位置,以速度$ v_{0} $平行于纸面水平射入电场,重力忽略不计。
\begin{enumerate}
\renewcommand{\labelenumi}{\arabic{enumi}.}
% A(\Alph) a(\alph) I(\Roman) i(\roman) 1(\arabic)
%设定全局标号series=example	%引用全局变量resume=example
%[topsep=-0.3em,parsep=-0.3em,itemsep=-0.3em,partopsep=-0.3em]
%可使用leftmargin调整列表环境左边的空白长度 [leftmargin=0em]
\item
求粒子第一次穿过$ G $时的动能,以及她从射入电场至此时在水平方向上的位移大小;
\item 
若粒子恰好从$ G $的下方距离$ G $也为$ h $的位置离开电场,则金属板的长度最短应为多少?


\end{enumerate}
\begin{figure}[h!]
\flushright
\includesvg[width=0.25\linewidth]{picture/svg/103}
\end{figure}


\banswer{
\begin{enumerate}
\renewcommand{\labelenumi}{\arabic{enumi}.}
% A(\Alph) a(\alph) I(\Roman) i(\roman) 1(\arabic)
%设定全局标号series=example	%引用全局变量resume=example
%[topsep=-0.3em,parsep=-0.3em,itemsep=-0.3em,partopsep=-0.3em]
%可使用leftmargin调整列表环境左边的空白长度 [leftmargin=0em]
\item
$E _ { \mathrm { k } } = \frac { 1 } { 2 } m v _ { 0 } ^ { 2 } + \frac { 2 \varphi } { d } q h$;$l = v _ { 0 } \sqrt { \frac { m d h } { q \varphi } }$
\item 
$L = 2 l = 2 v _ { 0 } \sqrt { \frac { m d h } { q \varphi } }$



\end{enumerate}
}



\end{enumerate}





