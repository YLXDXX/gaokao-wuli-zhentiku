\bta{部分电路欧姆定律}

\begin{enumerate}
%\renewcommand{\labelenumi}{\arabic{enumi}.}
% A(\Alph) a(\alph) I(\Roman) i(\roman) 1(\arabic)
%设定全局标号series=example	%引用全局变量resume=example
%[topsep=-0.3em,parsep=-0.3em,itemsep=-0.3em,partopsep=-0.3em]
%可使用leftmargin调整列表环境左边的空白长度 [leftmargin=0em]
\item
\exwhere{$ 2019 $ 年 $ 4 $ 月浙江物理选考}
电动机与小电珠串联接人电路,电动机正常工作时,小电珠的电阻
为 $ R_{1} $,两端电压为 $ U_{1} $,流过的电流为 $ I_{1} $;电动机的内电阻为 $ R_{2} $,两端电压为 $ U_{2} $,流过的电流为 $ I_{2} $。
则 \xzanswer{D} 


\fourchoices
{$I_{1}<I_{2}$}
{$\frac{U_{1}}{U_{2}}>\frac{R_{1}}{R_{2}}$}
{$\frac{U_{1}}{U_{2}}=\frac{R_{1}}{R_{2}}$}
{$\frac{U_{1}}{U_{2}}<\frac{R_{1}}{R_{2}}$}


\item 
\exwhere{$ 2018 $ 年浙江卷($ 4 $ 月选考)}
杭州市正将主干道上的部分高压钠灯换成 $ LED $ 灯,已知高压钠灯
功率为 $ 400 \ W $,$ LED $ 灯功率为 $ 180 \ W $,若更换 $ 4000 $ 盏,则一个月可节约的电能约为 \xzanswer{B} 

\fourchoices
{$ 9 \times 10^{2} \ kW \cdot h $}
{$ 3 \times 10^{5} \ kW \cdot h $}
{$ 6 \times 10^{5} \ kW \cdot h $}
{$ 1 \times 10^{12} \ kW \cdot h $}



\item 
\exwhere{$ 2012 $ 年理综浙江卷}
功率为 $ 10 \ W $ 的发光二极管($ LED $ 灯)的亮度与功率为 $ 60 \ W $ 的白炽灯相当。根据国家节能战略,
$ 2016 $ 年前普通白炽灯应被淘汰。假设每户家庭有 $ 2 $ 只 $ 60 \ W $ 的白炽灯,均用 $ 10 \ W $ 的 $ LED $ 灯替代,
估算出全国一年节省的电能最接近 \xzanswer{B} 

\fourchoices
{$ 8 \times 10^{8} \ kW \cdot h $}
{$ 8 \times 10 ^{10} \ kW \cdot h $}
{$ 8 \times 10 ^{11} \ kW \cdot h $}
{$ 8 \times 10 ^{13} \ kW \cdot h $}



\item 
\exwhere{$ 2012 $ 年物理上海卷}
当电阻两端加上某一稳定电压时,通过该电阻的电荷量为 $ 0.3 \ C $,消耗的电能为 $ 0.9 \ J $。为在相同
时间内使 $ 0.6 \ C $ 的电荷量通过该电阻,在其两端需加的电压和消耗的电能分别是 \xzanswer{D} 

\fourchoices
{$ 3 \ V $,$ 1.8 \ J $}
{$ 3 \ V $,$ 3.6 \ J $}
{$ 6 \ V $,$ l.8 \ J $}
{$ 6 \ V $,$ 3.6 \ J $}



\item
\exwhere{$ 2013 $ 年安徽卷}
用图示的电路可以测量电阻的阻值。图中 $ R_{x} $ 是待测电阻,$ R_{0} $ 是定值, $ G $ 是灵敏度很高的电流表,
$ MN $ 是一段均匀的电阻丝。闭合开关,改变滑动头 $ P $ 的位置,当通过电流表 $ G $ 的电流为零时,测得
$ MP= l_{1} $, $ PN= l_{2} $,则 $ R_{x} $ 的阻值为 \xzanswer{C} 
\begin{figure}[h!]
\centering
\includesvg[width=0.23\linewidth]{picture/svg/GZ-3-tiyou-1101}
\end{figure}


\fourchoices
{$\frac{l_{1}}{l_{2}} R_{0}$}
{$\frac{l_{1}}{l_{1}+l_{2}} R_{0}$}
{$\frac{l_{2}}{l_{1}} R_{0}$}
{$\frac{l_{2}}{l_{1}+l_{2}} R_{0}$}


\item 
\exwhere{$ 2015 $ 年上海卷}
重离子肿瘤治疗装置中的回旋加速器可发射$ +5 $ 价重离子束,其束流强度为
$ 1.2 \times 10^{-5} \ A $,则在 $ 1 \ s $ 内发射的重离子个数为($ e=1.6 \times 10^{-19} \ C $) \xzanswer{B} 

\fourchoices
{$ 3.0 \times 10^{12} $}
{$ 1.5 \times 10^{13} $}
{$ 7.5 \times 10^{13} $}
{$ 3.75 \times 10^{14} $}


\item 
\exwhere{$ 2015 $ 年理综安徽卷}
一根长为 $ L $、横截面积为 $ S $ 的金属棒,其材料的电阻率为$ \rho $,棒内单位体
积自由电子数为 $ n $,电子的质量为 $ m $,电荷量为 $ e $。在棒两端加上恒定的电压时,棒内产生电流,
自由电子定向运动的平均速率为 $ v $,则金属棒内的电场强度大小为 \xzanswer{C} 
\begin{figure}[h!]
\centering
\includesvg[width=0.23\linewidth]{picture/svg/GZ-3-tiyou-1102}
\end{figure}

\fourchoices
{$\frac{m v^{2}}{2 e L}$}
{$\frac{m v^{2} S n}{e}$}
{$ \rho nev $}
{$\frac{\rho e v}{S L}$}


\item
\exwhere{$ 2011 $ 年理综全国卷}
通常一次闪电过程历时约 $ 0.2 \sim 0.3 \ s $,它由若干个相继发生的闪击构成。每个闪击持续时间仅
$ 40 \sim 80 \ \mu s $,电荷转移主要发生在第一个闪击过程中。在某一次闪电前云地之间的电势差约为
$ 1.0 \times 10^{9} \ V $,云地间距离约为 $ 1 \ km $;第一个闪击过程中云地间转移的电荷量约为 $ 6 \ C $,闪击持续时间
约为 $ 60 \ \mu s $。假定闪电前云地间的电场是均匀的。根据以上数据,下列判断正确的是 \xzanswer{AC} 

\fourchoices
{闪电电流的瞬时值可达到 $ 1 \times 10^{5} \ A $}
{整个闪电过程的平均功率约为 $ l \times 10^{14} \ W $}
{闪电前云地间的电场强度约为 $ l \times 10^{6} \ V/m $}
{整个闪电过程向外释放的能量约为 $ 6 \times 10^{6} \ J $}


\item 
\exwhere{$ 2015 $ 年理综北京卷}
如图所示,其中电流表 $ A $ 的量程为 $ 0.6 \ A $,表盘均匀划分为 $ 30 $ 个小格,每
一小格表示 $ 0.02 \ A $,$ R_{1} $ 的阻值等于电流表内阻的 $ 1/2 $;$ R_{2} $ 的阻值等于电流表内阻的 $ 2 $ 倍。若用电流
表 $ A $ 的表盘刻度表示流过接线柱 $ 1 $ 的电流值,则下列分析正确
的是 \xzanswer{C} 
\begin{figure}[h!]
\centering
\includesvg[width=0.23\linewidth]{picture/svg/GZ-3-tiyou-1103}
\end{figure}


\fourchoices
{将接线柱 $ 1 $、$ 2 $ 接入电路时,每一小格表示 $ 0.04 \ A $}
{将接线柱 $ 1 $、$ 2 $ 接入电路时,每一小格表示 $ 0.02 \ A $}
{将接线柱 $ 1 $、$ 3 $ 接入电路时,每一小格表示 $ 0.06 \ A $}
{将接线柱 $ 1 $、$ 3 $ 接入电路时,每一小格表示 $ 0.01 \ A $}




\item 
\exwhere{$ 2012 $ 年理综四川卷}
四川省“十二五”水利发展规划指出,若按现有供水能力测算,我省供水缺口极大,蓄引提水是目
前解决供水问题的重要手段之一。某地要把河水抽高 $ 20 \ m $,进入蓄水池,用一台电动机通过传动效
率为 $ 80 \% $的皮带,带动效率为 $ 60 \% $的离心水泵工作。工作电压为 $ 380 \ V $,此时输入电动机的电功率
为 $ 19 \ kW $,电动机的内阻为 $ 0.4 \ \Omega $。已知水的密度为 $ 1 \times 10^{3} \ kg/m^{3} $,重力加速度取 $ 10 \ m/s^{2} $。求:
\begin{enumerate}
%\renewcommand{\labelenumi}{\arabic{enumi}.}
% A(\Alph) a(\alph) I(\Roman) i(\roman) 1(\arabic)
%设定全局标号series=example	%引用全局变量resume=example
%[topsep=-0.3em,parsep=-0.3em,itemsep=-0.3em,partopsep=-0.3em]
%可使用leftmargin调整列表环境左边的空白长度 [leftmargin=0em]
\item
电动机内阻消耗的热功率;

\item 
将蓄水池蓄入 $ 864 \ m^{3} $ 的水需要的时间(不计进、出水口的水流速度)。



\end{enumerate}


\banswer{
\begin{enumerate}
%\renewcommand{\labelenumi}{\arabic{enumi}.}
% A(\Alph) a(\alph) I(\Roman) i(\roman) 1(\arabic)
%设定全局标号series=example	%引用全局变量resume=example
%[topsep=-0.3em,parsep=-0.3em,itemsep=-0.3em,partopsep=-0.3em]
%可使用leftmargin调整列表环境左边的空白长度 [leftmargin=0em]
\item
$P_{r}=1 \times 10^{3} \ W$
\item 
$t=2 \times 10^{4} \ s$
\end{enumerate}
}



\item 
\exwhere{$ 2016 $ 年新课标 \lmd{2} 卷}
阻值相等的四个电阻、电容器 $ C $ 及电池 $ E $
(内阻可忽略)连接成如图所示电路。开关 $ S $ 断开且电流稳定时,$ C $
所带的电荷量为 $ Q_{1} $;闭合开关 $ S $,电流再次稳定后,$ C $ 所带的电荷量
为 $ Q_{2} $。$ Q_{1} $ 与 $ Q_{2} $ 的比值为 \xzanswer{C} 
\begin{figure}[h!]
\centering
\includesvg[width=0.23\linewidth]{picture/svg/GZ-3-tiyou-1104}
\end{figure}

\fourchoices
{$ \frac{ 2 }{ 5 } $}
{$ \frac{ 1 }{ 2 } $}
{$ \frac{ 3 }{ 5 } $}
{$ \frac{ 2 }{ 3 } $}





\item 
\exwhere{$ 2016 $ 年上海卷}
如图所示电路中,电源内阻忽略不计。闭合电建,电压表示数为 $ U $,电流表示
数为 $ I $;在滑动变阻器 $ R_{1} $ 的滑片 $ P $ 由 $ a $ 端滑到 $ b $ 端的过程中 \xzanswer{BC} 
\begin{figure}[h!]
\centering
\includesvg[width=0.23\linewidth]{picture/svg/GZ-3-tiyou-1105}
\end{figure}

\fourchoices
{$ U $ 先变大后变小}
{$ I $ 先变小后变大}
{$ U $ 与 $ I $ 比值先变大后变小}
{$ U $ 变化量与 $ I $ 变化量比值等于 $ R_{3} $}






\end{enumerate}

