\bta{第四讲$ \quad $电场力做功和电势能}


\begin{enumerate}[leftmargin=0em]
\renewcommand{\labelenumi}{\arabic{enumi}.}
% A(\Alph) a(\alph) I(\Roman) i(\roman) 1(\arabic)
%设定全局标号series=example	%引用全局变量resume=example
%[topsep=-0.3em,parsep=-0.3em,itemsep=-0.3em,partopsep=-0.3em]
%可使用leftmargin调整列表环境左边的空白长度 [leftmargin=0em]
\item
\exwhere{$ 2013 $年上海卷}
两异种点电荷电场中的部分等势面如图所示,已知$ A $点电势高于$ B $点电势。若位于$ a $、$ b $处点电荷的电荷量大小分别为$ q_a $和$ q_b $,则 \xzanswer{B} 
\begin{figure}[h!]
\centering
\includesvg[width=0.19\linewidth]{picture/svg/026}
\end{figure}

\fourchoices
{$ a $处为正电荷,$q _ { \mathrm { a } } < q _ { \mathrm { b } }$}
{$ a $处为正电荷,$q _ { \mathrm { a } } > q _ { \mathrm { b } }$}
{$ a $处为负电荷,$q _ { \mathrm { a } } < q _ { \mathrm { b } }$}
{$ a $处为负电荷,$q _ { \mathrm { a } } > q _ { \mathrm { b } }$}




\item
\exwhere{$ 2013 $年天津卷}
两个带等量正电的点电荷,固定在图中$ P $、$ Q $两点,$ MN $为$ PQ $连线的中垂线,交$ PQ $于$ O $点,$ A $点为$ MN $上的一点。一带负电的试探电荷$ q $,从$ A $点由静止释放,只在静电力作用下运动。取无限远处的电势为零,则 \xzanswer{BC} 
\begin{figure}[h!]
\centering
\includesvg[width=0.19\linewidth]{picture/svg/027}
\end{figure}


\fourchoices
{$ q $由$ A $向$ O $的运动是匀加速直线运动 }
{$ q $由$ A $向$ O $运动的过程电势能逐渐减小}
{$ q $运动到$ O $点时的动能最大 }
{$ q $运动到$ O $点时电势能为零}




\item
\exwhere{$ 2016 $年新课标\lmd{2}卷}
如图,$ P $为固定的点电荷,虚线是以$ P $为圆心的两个圆。带电粒子$ Q $在$ P $的电场中运动,运动轨迹与两圆在同一平面内,$ a $、$ b $、$ c $为轨迹上的三个点。若$ Q $仅受$ P $的电场力作用,其在$ a $、$ b $、$ c $点的加速度大小分别为$ a_a $、$ a_b $、$ a_c $,速度大小分别为$ v_a $、$ v_b $、$ v_c $,则 \xzanswer{D} 
\begin{figure}[h!]
\centering
\includesvg[width=0.19\linewidth]{picture/svg/028}
\end{figure}

\fourchoices
{$a _ { a } > a _ { b } > a _ { c } , \quad v _ { a } > v _ { c } > v _ { b }$}
{$a _ { a } > a _ { b } > a _ { c } , \quad v _ { b } > v _ { c } > v _ { a }$}
{$a _ { b } > a _ { c } > a _ { a } , \quad v _ { b } > v _ { c } > v _ { a }$}
{$a _ { b } > a _ { c } > a _ { a } , \quad v _ { a } > v _ { c } > v _ { b }$}






\item
\exwhere{$ 2016 $年江苏卷}
一金属容器置于绝缘板上,带电小球用绝缘细线悬挂于容器中,容器内的电场线分布如图所示。容器内表面为等势面,$ A $、$ B $为容器内表面上的两点,下列说法正确的是 \xzanswer{C} 
\begin{figure}[h!]
\centering
\includesvg[width=0.19\linewidth]{picture/svg/029}
\end{figure}


\fourchoices
{$ A $点的电场强度比$ B $点的大}
{小球表面的电势比容器内表面的低}
{$ B $点的电场强度方向与该处内表面垂直}
{将检验电荷从$ A $点沿不同路径移到$ B $点,电场力所做的功不同}





\item
\exwhere{$ 2016 $年上海卷}
如图,质量为$ m $的带电小球$ A $用绝缘细线悬挂于$ O $点,处于静止状态。施加一水平向右的匀强电场后,$ A $向右摆动,摆动的最大角度为$ 60 ^{ \circ } $,则$ A $受到的电场力大小为 \tk{$\frac { \sqrt { 3 } } { 3 } m g$} 。 在改变电场强度的大小和方向后,小球$ A $的平衡位置在$ \alpha=60 ^{ \circ } $处,然后再将$ A $的质量改变为$ 2 \ m $,其新的平衡位置在$ \alpha=30 ^{ \circ } $处,$ A $受到的电场力大小为 \tk{$m g$} 。
\begin{figure}[h!]
\centering
\includesvg[width=0.13\linewidth]{picture/svg/030}
\end{figure}



\item
\exwhere{$ 2016 $年海南卷}
如图,一带正电的点电荷固定于$ O $点,两虚线圆均以$ O $为圆心,两实线分别为带电粒子$ M $和$ N $先后在电场中运动的轨迹,$ a $、$ b $、$ c $、$ d $、$ e $为轨迹和虚线圆的交点。不计重力。下列说法说法正确的是 \xzanswer{ABC} 
\begin{figure}[h!]
\centering
\includesvg[width=0.19\linewidth]{picture/svg/031}
\end{figure}


\fourchoices
{$ M $带负电荷,$ N $带正电荷}
{$ M $在$ b $点的动能小于它在$ a $点的动能}
{$ N $在$ d $点的电势能等于它在$ e $点的电势能}
{$ N $在从$ c $点运动到$ d $点的过程中克服电场力做功}






\item
\exwhere{$ 2013 $年江苏卷}
将一电荷量为$ +Q $ 的小球放在不带电的金属球附近,所形成的电场线分布如图所示,金属球表面的电势处处相等 $. a $、$ b $ 为电场中的两点,则 \xzanswer{ABD} 
\begin{figure}[h!]
\centering
\includesvg[width=0.19\linewidth]{picture/svg/032}
\end{figure}


\fourchoices
{$a $ 点的电场强度比$ b $ 点的大}
{$a $ 点的电势比$ b $ 点的高}
{检验电荷$ -q $ 在$ a $ 点的电势能比在$ b $ 点的大}
{将检验电荷$ -q $ 从$ a $ 点移到$ b $ 点的过程中,电场力做负功}





\item
\exwhere{$ 2012 $年理综天津卷}
两个固定带等量异号点电荷所产生电场等势面如图中虚线所示,一带负电的粒子以某一速度从图中$ A $点沿图示方向进入电场在纸面内飞行,最后离开电场,粒子只受静电力作用,则粒子在电场中 \xzanswer{C} 
\begin{figure}[h!]
\centering
\includesvg[width=0.19\linewidth]{picture/svg/033}
\end{figure}



\fourchoices
{做直线运动,电势能先变小后变大}
{做直线运动,电势能先变大后变小}
{做曲线运动,电势能先变小后变大}
{做曲线运动,电势能先变大后变小}




\item
\exwhere{$ 2012 $年理综福建卷}
如图,在点电荷$ Q $产生的电场中,将两个带正电的试探电荷$ q_{1} $、$ q_{2} $分别置于$ A $、$ B $两点,虚线为等势线。取无穷远处为零电势点,若将$ q_{1} $、$ q_{2} $移动到无穷远的过程中外力克服电场力做的功相等,则下列说法正确的是 \xzanswer{C} 
\begin{figure}[h!]
\centering
\includesvg[width=0.19\linewidth]{picture/svg/034}
\end{figure}


\fourchoices
{$ A $点电势大于$ B $点电势}
{$ A $、$ B $两点的电场强度相等}
{$ q_{1} $的电荷量小于$ q_{2} $的电荷量}
{$ q_{1} $在$ A $点的电势能小于$ q_{2} $在$ B $点的电势能}




\item
\exwhere{$ 2012 $年物理海南卷}
如图,直线上有$ O $、$ a $、$ b $、$ c $四点,$ ab $间的距离与$ bc $间的距离相等。在$ O $点处有固定点电荷。已知$ b $点电势高于$ c $点电势。若一带负电荷的粒子仅在电场力作用下先从$ c $点运动到$ b $点,再从$ b $点运动到$ a $点,则 \xzanswer{C} 
\begin{figure}[h!]
\centering
\includesvg[width=0.19\linewidth]{picture/svg/035}
\end{figure}


\fourchoices
{两过程中电场力做的功相等}
{前一过程中电场力做的功大于后一过程中电场力做的功}
{前一过程中,粒子电势能不断减小}
{后一过程,粒子动能不断减小}





\item
\exwhere{$ 2011 $年上海卷}
如图,在竖直向下,场强为$ E $的匀强电场中,长为$ l $的绝缘轻杆可绕固定轴$ O $在竖直面内无摩擦转动,两个小球$ A $、$ B $固定于杆的两端,$ A $、$ B $的质量分别为$ m_{1} $和$ m_{2} $ $ (m_{1}<m_{2}) $,$ A $带负电,电量为$ q_{1} $,$ B $带正电,电量为$ q_{2} $。杆从静止开始由水平位置转到竖直位置,在此过程中电场力做功为\tk{$\left( q _ { 1 } + q _ { 2 } \right) E l / 2$},在竖直位置处两球的总动能为\tk{$\left[ \left( q _ { 1 } + q _ { 2 } \right) E + \left( m _ { 2 } - m _ { 1 } \right) g \right] l / 2$}。
\begin{figure}[h!]
\centering
\includesvg[width=0.19\linewidth]{picture/svg/036}
\end{figure}


\item
\exwhere{$ 2014 $年理综大纲卷}
地球表面附近某区域存在大小为$ 150 \ N/C $、方向竖直向下的电场。一质量为$ 1.00 \times 10 ^{-4} \ kg $、带电量为$ - 1.00 \times 10^{-7}\ C $的小球从静止释放,在电场区域内下落$ 10.0\ m $。对此过程,该小球的电势能和动能的改变量分别为(重力加速度大小取$ 9.80 \ \ \ m/s ^{2} $,忽略空气阻力) \xzanswer{D} 
\fourchoices
{$- 1.50 \times 10 ^ { - 4 } \mathrm { J } \text{和} 9.95 \times 10 ^ { - 3 } \mathrm { J }$}
{$1.50 \times 10 ^ { - 4 } \mathrm { J } \text{和} -9.95 \times 10 ^ { - 3 } \mathrm { J }$}
{$ - 1.50 \times 10 ^ { - 4 } \mathrm { J } \text{和} -9.95 \times 10 ^ { - 3 } \mathrm { J }$}
{$ 1.50 \times 10 ^ { - 4 } \mathrm { J } \text{和} 9.95 \times 10 ^ { - 3 } \mathrm { J }$}





\item
\exwhere{$ 2014 $年物理上海卷}
静电场在$ x $轴上的场强$ E $随$ x $的变化关系如图所示,$ x $轴正向为场强正方向,带正电的点电荷沿$ x $轴运动,则点电荷 \xzanswer{BC} 
\begin{figure}[h!]
\centering
\includesvg[width=0.24\linewidth]{picture/svg/037}
\end{figure}

\fourchoices
{在$ x_{2} $和$ x_4 $处电势能相等}
{由$ x_{1} $运动到$ x_3 $的过程中电势能增大}
{由$ x_{1} $运动到$ x_4 $的过程中电场力先增大后减小}
{由$ x_{1} $运动到$ x_4 $的过程中电场力先减小后增大}



\item
\exwhere{$ 2014 $年理综重庆卷}
如图所示为某示波管内的聚焦电场,实线和虚线分别表示电场线和等势线。两电子分别从$ a $、$ b $两点运动到$ c $点,设电场力对两电子做的功分别为$ W_a $和$ W_b $,$ a $、$ b $两点的电场强度大小分别为$ E_a $和$ E_b $,则 \xzanswer{A} 
\begin{figure}[h!]
\centering
\includesvg[width=0.24\linewidth]{picture/svg/038}
\end{figure}

\fourchoices
{$W _ { a } = W _ { b } , \quad E _ { a } > E _ { b }$}
{$W _ { a } \neq W _ { b } , \quad E _ { a } > E _ { b }$}
{$W _ { a } = W _ { b } , \quad E _ { a } < E _ { b }$}
{$W _ { a } \neq W _ { b } , \quad E _ { a } < E _ { b }$}




\item
\exwhere{$ 2014 $年理综山东卷}
如图,半径为$ R $的均匀带正电的薄球壳,其上有一小孔$ A $。已知壳内的场强处处为零;壳外空间的电场,与将球壳上的全部电荷集中于球心$ O $时在壳外产生的电场一样。一带正电的试探电荷(不计重力)从球心以初动能$ E_{k0} $沿$ OA $方向射出。下列关于试探电荷的动能$ E_{k} $与离开球心的距离$ r $的关系,可能正确的是 \xzanswer{A} 
\begin{figure}[h!]
\centering
\includesvg[width=0.89\linewidth]{picture/svg/039}
\end{figure}



\item
\exwhere{$ 2018 $年全国卷\lmd{1}}
图中虚线$ a $、$ b $、$ c $、$ d $、$ f $代表匀强电场内间距相等的一组等势面,已知平面$ b $上的电势为$ 2 \ V $。一电子经过$ a $时的动能为$ 10\ eV $,从$ a $到$ d $的过程中克服电场力所做的功为$ 6\ eV $。下列说法正确的是 \xzanswer{AB} 
\begin{figure}[h!]
\centering
\includesvg[width=0.19\linewidth]{picture/svg/040}
\end{figure}


\fourchoices
{平面$ c $上的电势为零}
{该电子可能到达不了平面$ f $}
{该电子经过平面$ d $时,其电势能为$ 4\ eV $ }
{该电子经过平面$ b $时的速率是经过$ d $时的$ 2 $倍}





\item
\exwhere{$ 2018 $年全国卷\lmd{2}}
如图,同一平面内的$ a $、$ b $、$ c $、$ d $四点处于匀强电场中,电场方向与此平面平行,$ M $为$ a $、$ c $连线的中点,$ N $为$ b $、$ d $连线的中点。一电荷量为$ q $($ q>0 $)的粒子从$ a $点移动到$ b $点,其电势能减小$ W_1 $;若该粒子从$ c $点移动到$ d $点,其电势能减小$ W_2 $。下列说法正确的是 \xzanswer{BD} 
\begin{figure}[h!]
\centering
\includesvg[width=0.19\linewidth]{picture/svg/041}
\end{figure}


\fourchoices
{此匀强电场的场强方向一定与$ a $、$ b $两点连线平行}
{若该粒子从$ M $点移动到$ N $点,则电场力做功一定为$\frac { W _ { 1 } + W _ { 2 } } { 2 }$}
{若$ c $、$ d $之间的距离为$ L $,则该电场的场强大小一定为$\frac { W _ { 2 } } { q L }$}
{若$ W_1=W_2 $,则$ a $、$ M $两点之间的电势差一定等于$ b $、$ N $两点之间的电势差}





\item
\exwhere{$ 2014 $年理综安徽卷}
一带电粒子在电场中仅受静电力作用,做初速度为零的直线运动。取该直线为$ x $轴,起始点$ O $为坐标原点,其电势能$ E_{P} $与位移$ x $的关系如右图所示。下列图象中合理的是 \xzanswer{D} 
\begin{figure}[h!]
\centering
\includesvg[width=0.24\linewidth]{picture/svg/081}\\
\includesvg[width=0.91\linewidth]{picture/svg/042}
\end{figure}



\item
\exwhere{$ 2015 $年理综新课标\lmd{1}卷}
如图,直线$ a $、$ b $和$ c $、$ d $是处于匀强电场中的两组平行线,$ M $、$ N $、$ P $、$ Q $是它们的交点,四点处的电势分别为$ \varphi _M $、$ \varphi _N $、$ \varphi _P $、$ \varphi _Q $。一电子由$ M $点分别运动到$ N $点和$ P $点的过程中,电场力所做的负功相等,则 \xzanswer{B} 
\begin{figure}[h!]
\centering
\includesvg[width=0.19\linewidth]{picture/svg/043}
\end{figure}


\fourchoices
{直线$ a $位于某一等势面内,$ \varphi _M > \varphi _Q $}
{直线$ c $位于某一等势面内,$ \varphi _M > \varphi _N $ }
{若电子由$ M $点运动到$ Q $点,电场力做正功}
{若电子由$ P $点运动到$ Q $点,电场力做负功}





\item
\exwhere{$ 2015 $年理综安徽卷}
在$ xOy $平面内,有沿$ y $轴负方向的匀强电场,场强大小为$ E $(图中未画出),由$ A $点斜射出一质量为$ m $,带电量为$ +q $的粒子,$ B $和$ C $是粒子运动轨迹上的两点,如图所示,其中$ l_{0} $为常数。粒子所受重力忽略不计,求:
\begin{enumerate}
\renewcommand{\labelenumi}{\arabic{enumi}.}
% A(\Alph) a(\alph) I(\Roman) i(\roman) 1(\arabic)
%设定全局标号series=example	%引用全局变量resume=example
%[topsep=-0.3em,parsep=-0.3em,itemsep=-0.3em,partopsep=-0.3em]
%可使用leftmargin调整列表环境左边的空白长度 [leftmargin=0em]
\item
粒子从$ A $到$ C $过程中电场力对它做的功;
\item 
粒子从$ A $到$ C $过程所经历的时间;
\item 
粒子经过$ C $点时的速率。



\end{enumerate}
\begin{figure}[h!]
\flushright
\includesvg[width=0.39\linewidth]{picture/svg/044}
\end{figure}
\banswer{
\begin{enumerate}
\renewcommand{\labelenumi}{\arabic{enumi}.}
% A(\Alph) a(\alph) I(\Roman) i(\roman) 1(\arabic)
%设定全局标号series=example	%引用全局变量resume=example
%[topsep=-0.3em,parsep=-0.3em,itemsep=-0.3em,partopsep=-0.3em]
%可使用leftmargin调整列表环境左边的空白长度 [leftmargin=0em]
\item
$W = 3 q E l _ { 0 }$
\item 
$t _ { A C } = 3 \sqrt { \frac { 2 m l _ { 0 } } { q E } }$
\item 
$v _ { C } = \sqrt { \frac { 17 q E l _ { 0 } } { 2 m } }$



\end{enumerate}
}




\item
\exwhere{$ 2015 $年理综四川卷}
如图所示,半圆槽光滑、绝缘、固定,圆心是$ O $,最低点是$ P $,直径$ MN $水平,$ a $、$ b $是两个完全相同的带正电小球(视为点电荷),$ b $固定在$ M $点,$ a $从$ N $点静止释放,沿半圆槽运动经过$ P $点到达某点$ Q $(图中未画出)时速度为零。则小球$ a $ \xzanswer{BC} 
\begin{figure}[h!]
\centering
\includesvg[width=0.19\linewidth]{picture/svg/045}
\end{figure}


\fourchoices
{从$ N $到$ Q $的过程中,重力与库仑力的合力先增大后减小}
{从$ N $到$ P $的过程中,速率先增大后减小}
{从$ N $到$ Q $的过程中,电势能一直增加}
{从$ P $到$ Q $的过程中,动能减少量小于电势能增加量}





\item
\exwhere{$ 2015 $年广东卷}
$ 21 $.图$ 8 $所示的水平匀强电场中,将两个带电小球$ M $和$ N $分别沿图示路径移动到同一水平线上的不同位置,释放后,$ M $、$ N $保持静止,不计重力,则 \xzanswer{BD} 
\begin{figure}[h!]
\centering
\includesvg[width=0.19\linewidth]{picture/svg/046}
\end{figure}


\fourchoices
{$ M $的带电量比$ N $的大 }
{$ M $带负电荷,$ N $带正电荷}
{静止时$ M $受到的合力比$ N $的大 }
{移动过程中匀强电场对$ M $做负功}




\item
\exwhere{$ 2015 $年上海卷}
两个正、负点电荷周围电场线分布如图所示。$ P $、$ Q $为电场中两点,则 \xzanswer{D} 
\begin{figure}[h!]
\centering
\includesvg[width=0.19\linewidth]{picture/svg/047}
\end{figure}


\fourchoices
{正电荷由$ P $静止释放能运动到$ Q $}
{正电荷在$ P $的加速度小于在$ Q $的加速度}
{负电荷在$ P $的电势能高于在$ Q $的电势能}
{负电荷从$ P $移动到$ Q $,其间必有一点电势能为零}




\item
\exwhere{$ 2015 $年海南卷}
如图,两电荷量分别为$ Q $($ Q > 0 $)和$ -Q $的点电荷对称地放置在$ x $轴上原点$ O $的两侧,$ a $点位于$ x $轴上$ O $点与点电荷$ Q $之间,$ b $位于$ y $轴$ O $点上方。取无穷远处的电势为零,下列说法正确的是 \xzanswer{BC} 
\begin{figure}[h!]
\centering
\includesvg[width=0.19\linewidth]{picture/svg/048}
\end{figure}

\fourchoices
{$ b $点的电势为零,电场强度也为零}
{正的试探电荷在$ a $点的电势能大于零,所受电场力方向向右}
{将正的试探电荷从$ O $点移到$ a $点,必须克服电场力做功}
{将同一正的试探电荷先后从$ O $、$ b $两点移到$ a $点,后者电势能的变化较大}





\item
\exwhere{$ 2019 $年物理江苏卷}
如图所示,$ ABC $为等边三角形,电荷量为$ +q $的点电荷固定在$ A $点.先将一电荷量也为$ +q $的点电荷$ Q_{1} $从无穷远处(电势为$ 0 $)移到$ C $点,此过程中,电场力做功为$ -W $.再将$ Q_{1} $从$ C $点沿$ CB $移到$ B $点并固定.最后将一电荷量为$ -2q $的点电荷$ Q_{2} $从无穷远处移到$ C $点.下列说法正确的有 \xzanswer{ABD} 
\begin{figure}[h!]
\centering
\includesvg[width=0.19\linewidth]{picture/svg/049}
\end{figure}



\fourchoices
{$ Q_{1} $移入之前,$ C $点的电势为$ \frac{W}{q} $}
{$ Q_{1} $从$ C $点移到$ B $点的过程中,所受电场力做的功为$ 0 $}
{$ Q_{2} $从无穷远处移到$ C $点的过程中,所受电场力做的功为$ 2 \ W $}
{$ Q_{2} $在移到$ C $点后的电势能为$ -4 \ W $}



\newpage 
\item
\exwhere{$ 2014 $年物理上海卷}
如图,一对平行金属板水平放置,板间距为$ d $,上极板始终接地。长度为$ d/2 $、质量均匀的绝缘杆,上端可绕上板中央的固定轴$ O $在竖直平面内转动,下端固定一带正电的轻质小球,其电荷量为$ q $。当两板间电压为$ U_{1} $时,杆静止在与竖直方向$ OO ^{\prime} $夹角$ \theta =30 ^{\circ} $的位置;若两金属板在竖直平面内同时绕$ O $、$ O ^{\prime} $ 顺时针旋转 $ \alpha= 15 ^{\circ} $至图中虚线位置时,为使杆仍在原位置静止,需改变两板间电压。假定两板间始终为匀强电场。求$ : $
\begin{enumerate}
\renewcommand{\labelenumii}{(\arabic{enumii})}

\item 
绝缘杆所受的重力$ G $;

\item 
两板旋转后板间电压$ U_{2} $。

\item 
在求前后两种情况中带电小球的电势能$ W_1 $与$ W_2 $时,某同学认为由于在两板旋转过程中带电小球位置未变,电场力不做功,因此带电小球的电势能不变。你若认为该同学的结论正确,计算该电势能;你若认为该同学的结论错误,说明理由并求$ W_1 $与$ W_2 $。
\end{enumerate}
\begin{figure}[h!]
\flushright
\includesvg[width=0.25\linewidth]{picture/svg/050}
\end{figure}


\banswer{
\begin{enumerate}
\renewcommand{\labelenumi}{\arabic{enumi}.}
% A(\Alph) a(\alph) I(\Roman) i(\roman) 1(\arabic)
%设定全局标号series=example	%引用全局变量resume=example
%[topsep=-0.3em,parsep=-0.3em,itemsep=-0.3em,partopsep=-0.3em]
%可使用leftmargin调整列表环境左边的空白长度 [leftmargin=0em]
\item
$G = \frac { 2 q U _ { 1 } } { d }$
\item 
$\frac { \sqrt { 3 } + 1 } { 4 } U _ { 1 }$
\item 
错误,$W _ { 1 } = q U _ { 1 } ^ { \prime } = \frac { \sqrt { 3 } } { 4 } q U _ { 1 }$,$W _ { 2 } = q U _ { 2 } ^ { \prime } = \frac { 1 } { 4 } q U _ { 1 }$.



\end{enumerate}
}



\end{enumerate}






