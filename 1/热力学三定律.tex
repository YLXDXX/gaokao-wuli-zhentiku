\bta{热力学三定律}
\btd{第一定律}
\begin{enumerate}[leftmargin=0em]
\renewcommand{\labelenumi}{\arabic{enumi}.}
% A(\Alph) a(\alph) I(\Roman) i(\roman) 1(\arabic)
%设定全局标号series=example	%引用全局变量resume=example
%[topsep=-0.3em,parsep=-0.3em,itemsep=-0.3em,partopsep=-0.3em]
%可使用leftmargin调整列表环境左边的空白长度 [leftmargin=0em]
\item
\exwhere{$ 2014 $年理综广东卷}
用密封性好、充满气体的塑料袋包裹易碎品,如图所示,充气袋四周被挤压时,假设袋内气体与外界无热交换,则袋内气体 \xzanswer{AC} 
\begin{figure}[h!]
\centering
\includesvg[width=0.23\linewidth]{picture/svg/269}
\end{figure}

\fourchoices
{体积减小,内能增大}
{体积减小,压强减小}
{对外界做负功,内能增大}
{对外界做正功,压强减小}


\item 
\exwhere{$ 2011 $年理综广东卷}
图为某种椅子与其升降部分的结构示意图。$ M $、$ N $两筒间密闭了一定质量的气体,$ M $可沿$ N $的内壁上下滑动,设筒内气体不与外界发生热交换,在$ M $向下滑动的过程中 \xzanswer{A} 
\begin{figure}[h!]
\centering
\includesvg[width=0.23\linewidth]{picture/svg/270}
\end{figure}

\fourchoices
{外界对气体做功,气体内能增大}
{外界对气体做功,气体内能减小}
{气体对外界做功,气体内能增大}
{气体对外界做功,气体内能减小}


\item 
\exwhere{$ 2012 $年理综广东卷}
景颇族的祖先发明的点火器如图$ 1 $所示,用牛角做套筒,木质推杆前端粘着艾绒。猛推推杆,艾绒即可点燃,对筒内封闭的气体,在此压缩过程中 \xzanswer{B} 
\begin{figure}[h!]
\centering
\includesvg[width=0.23\linewidth]{picture/svg/271}
\end{figure}


\fourchoices
{气体温度升高,压强不变}
{气体温度升高,压强变大}
{气体对外界做正功,其体内能增加}
{外界对气体做正功,气体内能减少}


\item 
\exwhere{$ 2011 $年理综重庆卷}
某汽车后备箱内安装有撑起箱盖的装置,它主要由汽缸和活塞组成。开箱时,密闭于汽缸内的压缩气体膨胀,将箱盖顶起,如图所示。在此过程中,若缸内气体与外界无热交换,忽略气体分子间相互作用,则缸内气体 \xzanswer{A} 
\begin{figure}[h!]
\centering
\includesvg[width=0.23\linewidth]{picture/svg/272}
\end{figure}

\fourchoices
{对外做正功,分子的平均动能减小}
{对外做正功,内能增大}
{对外做负功,分子的平均动能增大}
{对外做负功,内能减小}

\item 
\exwhere{$ 2015 $年理综北京卷}
下列说法正确的是 \xzanswer{C} 

\fourchoices
{物体放出热量,其内能一定减小}
{物体对外做功,其内能一定减小}
{物体吸收热量,同时对外做功,其内能可能增加}
{物体放出热量,同时对外做功,其内能可能不变}

\begin{enumerate}[leftmargin=-2em]
\renewcommand{\labelenumii}{}
% A(\Alph) a(\alph) I(\Roman) i(\roman) 1(\arabic)
%可使用leftmargin调整列表环境左边的空白长度
\item
\btd{第二定律}
\end{enumerate}


\item 
\exwhere{$ 2013 $年全国卷大纲卷}
根据热力学定律,下列说法正确的是 \xzanswer{AB} 

\fourchoices
{电冰箱的工作过程表明,热量可以从低温物体向高温物体传递}
{空调机在制冷过程中,从室内吸收的热量少于向室外放出的热量}
{科技的进步可以使内燃机成为单一热源的热机}
{对能源的过度消耗使自然界的能量不断减少,形成“能源危机”}


\item
\exwhere{$ 2011 $年理综全国卷}
关于一定量的气体,下列叙述正确的是 \xzanswer{AD} 


\fourchoices
{气体吸收的热量可以完全转化为功 }
{气体体积增大时,其内能一定减少}
{气体从外界吸收热量,其内能一定增加}
{外界对气体做功,气体内能可能减少}







\end{enumerate}

