
\bta{第三讲$ \quad $电场强度和电场力}
\begin{enumerate}[leftmargin=0em]
\renewcommand{\labelenumi}{\arabic{enumi}.}
% A(\Alph) a(\alph) I(\Roman) i(\roman) 1(\arabic)
%设定全局标号series=example	%引用全局变量resume=example
%[topsep=-0.3em,parsep=-0.3em,itemsep=-0.3em,partopsep=-0.3em]
%可使用leftmargin调整列表环境左边的空白长度 [leftmargin=0em]
\item
\exwhere{$ 2019 $年物理全国卷\lmd{1}}
如图,空间存在一方向水平向右的匀强电场,两个带电小球$ P $和$ Q $用相同的绝缘细绳悬挂在水平天花板下,两细绳都恰好与天花板垂直,则 \xzanswer{D} 
\begin{figure}[h!]
\centering
\includesvg[width=0.22\linewidth]{picture/svg/006}
\end{figure}

\fourchoices
{$ P $和$ Q $都带正电荷}
{$ P $和$ Q $都带负电荷}
{$ P $带正电荷,$ Q $带负电荷}
{$ P $带负电荷,$ Q $带正电荷}


\item 
\exwhere{$ 2019 $年物理全国卷\lmd{2}}
静电场中,一带电粒子仅在电场力的作用下自$ M $点由静止开始运动,$ N $为粒子运动轨迹上的另外一点,则 \xzanswer{AC} 



\fourchoices
{运动过程中,粒子的速度大小可能先增大后减小}
{在$ M $、$ N $两点间,粒子的轨迹一定与某条电场线重合}
{粒子在$ M $点的电势能不低于其在$ N $点的电势能}
{粒子在$ N $点所受电场力的方向一定与粒子轨迹在该点的切线平行}





\item
\exwhere{$ 2012 $年海南卷}
$ N $($ N > 1 $)个电荷量均为$ q $($ q > 0 $)的小球,均匀分布在半径为$ R $的圆周上,示意如图。若移去位于圆周上$ P $点的一个小球,则圆心$ O $点处的电场强度大小为\tk{$k \frac { q } { R ^ { 2 } }$},方向\tk{沿$ OP $指向$ P $点}。(已知静电力常量为$ k $)
\begin{figure}[h!]
\centering
\includesvg[width=0.19\linewidth]{picture/svg/007}
\end{figure}

\newpage
\item
\exwhere{$ 2011 $年新课标版}
一带负电荷的质点,在电场力作用下沿曲线$ abc $从$ a $运动到$ c $,已知质点的速率是递减的。关于$ b $点电场强度$ E $的方向,下列图示中可能正确的是(虚线是曲线在$ b $点的切线) \xzanswer{D} 


\fourchoices
{\includesvg[width=0.25\linewidth]{picture/svg/008}}
{\includesvg[width=0.25\linewidth]{picture/svg/009}}
{\includesvg[width=0.25\linewidth]{picture/svg/010}}
{\includesvg[width=0.25\linewidth]{picture/svg/011}}





\item
\exwhere{$ 2013 $年海南卷}
如图,电荷量为$ q_{1} $和$ q_{2} $的两个点电荷分别位于$ P $点和$ Q $点。已知在$ P $、$ Q $连线上某点$ R $处的电场强度为零,且$ PR=2RQ $。则 \xzanswer{B} 
\begin{figure}[h!]
\centering
\includesvg[width=0.19\linewidth]{picture/svg/012}
\end{figure}


\fourchoices
{$ q_{1} =2 q_{2} $ }
{$ q_{1} =4 q_{2} $}
{$ q_{1} =-2 q_{2} $}
{$ q_{1} =-4 q_{2} $}





\item
\exwhere{$ 2017 $年海南卷}
关于静电场的电场线,下列说法正确的是 \xzanswer{C} 


\fourchoices
{电场强度较大的地方电场线一定较疏}
{沿电场线方向,电场强度一定越来越小}
{沿电场线方向,电势一定越来越低}
{电场线一定是带电粒子在电场中运动的轨迹}





\item
\exwhere{$ 2017 $年海南卷}
如图,平行板电容器的两极板竖直放置并分别与电源的正负极相连,一带电小球经绝缘轻绳悬挂于两极板之间,处于静止状态。现保持右极板不动,将左极板向左缓慢移动。关于小球所受的电场力大小$ F $和绳子的拉力大小$ T $,下列判断正确的是 \xzanswer{A} 
\begin{figure}[h!]
\centering
\includesvg[width=0.19\linewidth]{picture/svg/013}
\end{figure}

\fourchoices
{$ F $逐渐减小,$ T $逐渐减小}
{$ F $逐渐增大,$ T $逐渐减小}
{$ F $逐渐减小,$ T $逐渐增大}
{$ F $逐渐增大,$ T $逐渐增大}


\item
\exwhere{$ 2013 $年理综新课标\lmd{2}卷}
如图,在光滑绝缘水平面上,三个带电小球$ a $,$ b $和$ c $分别位于边长为$ l $的正三角形的三个顶点上;$ a $、$ b $带正电,电荷量均为$ q $,$ c $带负电。整个系统置于方向水平的匀强电场中。已知静电力常量为$ k $。若 三个小球均处于静止状态,则匀强电场场强的大小为 \xzanswer{B} 
\begin{figure}[h!]
\centering
\includesvg[width=0.19\linewidth]{picture/svg/015}
\end{figure}
\fourchoices
{$\frac { \sqrt { 3 } k q } { 3 l ^ { 2 } }$}
{$\frac { \sqrt { 3 } k q } { l ^ { 2 } }$}
{$\frac { 3 k q } { l ^ { 2 } }$}
{$\frac { 2 \sqrt { 3 } k q } { l ^ { 2 } }$}



\item
\exwhere{$ 2017 $年北京卷}
如图所示,长$ l=1 $ $ m $的轻质细绳上端固定,下端连接一个可视为质点的带电小球,小球静止在水平向右的匀强电场中,绳与竖直方向的夹角$ \theta =37 ^{ \circ } $。已知小球所带电荷量$ q=1.0 \times 10 ^{-6} $ $ C $,匀强电场的场强$ E=3.0 \times 10^3 $ $ N/C $,取重力加速度$ g=10 $ $ m/s ^{2} $,$ \sin $ $ 37 ^{ \circ } =0.6 $,$ \cos $ $ 37 ^{ \circ } =0.8 $.,求:
\begin{minipage}[h!]{0.7\linewidth}
\vspace{0.3em}
\begin{enumerate}
\renewcommand{\labelenumi}{\arabic{enumi}.}
% A(\Alph) a(\alph) I(\Roman) i(\roman) 1(\arabic)
%设定全局标号series=example	%引用全局变量resume=example
%[topsep=-0.3em,parsep=-0.3em,itemsep=-0.3em,partopsep=-0.3em]
%可使用leftmargin调整列表环境左边的空白长度 [leftmargin=0em]
\item
小球所受电场力$ F $的大小。
\item 
小球的质量$ m $。
\item 
将电场撤去,小球回到最低点时速度$ v $的大小。



\end{enumerate}
\vspace{0.3em}
\end{minipage}
\hfill
\begin{minipage}[h!]{0.3\linewidth}
\flushright
\vspace{0.3em}
\includesvg[width=0.7\linewidth]{picture/svg/014}
\vspace{0.3em}
\end{minipage}

\banswer{
\begin{enumerate}
\renewcommand{\labelenumi}{\arabic{enumi}.}
% A(\Alph) a(\alph) I(\Roman) i(\roman) 1(\arabic)
%设定全局标号series=example	%引用全局变量resume=example
%[topsep=-0.3em,parsep=-0.3em,itemsep=-0.3em,partopsep=-0.3em]
%可使用leftmargin调整列表环境左边的空白长度 [leftmargin=0em]
\item
$F = q E = 1.0 \times 10 ^ { - 6 } \times 3.0 \times 10 ^ { 3 } \mathrm { N } = 3.0 \times 10 ^ { - 3 } \mathrm { N }$
\item 
$m = 4.0 \times 10 ^ { - 4 } \mathrm { kg }$
\item 
$v = \sqrt { 2 g l \left( 1 - \cos 37 ^ { \circ } \right) } = 2.0 \mathrm { m } / \mathrm { s }$



\end{enumerate}
}







\item
\exwhere{$ 2016 $年浙江卷}
如图所示,把$ A $、$ B $两个相同的导电小球分别用长为$ 0.10 $ $ m $的绝缘细线悬挂于$ O_A $和$ O_B $两点。用丝绸摩擦过的玻璃棒与$ A $球接触,棒移开后将悬点$ O_B $移到$ O_A $点固定。两球接触后分开,平衡时距离为$ 0.12 $ $ m $。已测得每个小球质量是$8.0 \times 10 ^ { 4 } \mathrm { kg }$,带电小球可视为点电荷,重力加速度$ g=10\ m/s^{2} $,静电力常量$k = 9.0 \times 10 ^ { 9 } \mathrm { N } \cdot \mathrm { m } ^ { 2 } / \mathrm { C } ^ { 2 }$.则 \xzanswer{ACD} 

\begin{minipage}[h!]{0.7\linewidth}
\vspace{0.3em}
\fourchoices
{两球所带电荷量相等}
{$ A $球所受的静电力为$1.0 \times 10 ^ { - 2 } \mathrm { N }$}
{$ B $球所带的电荷量为$4 \sqrt { 6 } \times 10 ^ { - 8 } \mathrm { C }$ }
{$ A $、$ B $两球连线中点处的电场强度为$ 0 $ }
\vspace{0.3em}
\end{minipage}
\hfill
\begin{minipage}[h!]{0.3\linewidth}
\flushright
\vspace{0.3em}
\includesvg[width=0.4\linewidth]{picture/svg/016}
\vspace{0.3em}
\end{minipage}


\item
\exwhere{$ 2013 $年理综新课标\lmd{1}卷}
如图,一半径为$ R $的圆盘上均匀分布着电荷量为$ Q $的电荷,在垂直于圆盘且过圆心$ c $的轴线上有$ a $、 $ b $、$ d $三个点,$ a $和$ b $、$ b $和$ c $、 $ c $和$ d $间的距离均为$ R $,在$ a $点处有一电荷量为$ q $ $ (q>0) $的固定点电荷。已知$ b $点处的场强为零,则$ d $点处场强的大小为$ (k $为静电力常量) \xzanswer{B} 
\begin{figure}[h!]
\centering
\includesvg[width=0.23\linewidth]{picture/svg/018}
\end{figure}
\fourchoices
{$k \frac { 3 q } { R ^ { 2 } }$}
{$k \frac { 10 q } { 9 R ^ { 2 } }$}
{$k \frac { Q + q } { R ^ { 2 } }$}
{$k \frac { 9 Q + q } { 9 R ^ { 2 } }$}


\item
\exwhere{$ 2014 $年福建卷}
如图,真空中$ xOy $平面直角坐标系上的$ ABC $三点构成等边三角形,边长$ L=2 . 0 \ m $。若将电荷量均为$ q=+2 . 0 \times 10^{-6} $的两点电荷分别固定在$ A $、$ B $点,已知静电力常量$k = 9.0 \times 10 ^ { 9 } \mathrm { N } \cdot \mathrm { m } ^ { 2 } / \mathrm { C } ^ { 2 }$.求:
\begin{enumerate}
\renewcommand{\labelenumi}{\arabic{enumi}.}
% A(\Alph) a(\alph) I(\Roman) i(\roman) 1(\arabic)
%设定全局标号series=example	%引用全局变量resume=example
%[topsep=-0.3em,parsep=-0.3em,itemsep=-0.3em,partopsep=-0.3em]
%可使用leftmargin调整列表环境左边的空白长度 [leftmargin=0em]
\item
两点电荷间的库仑力大小;
\item 
$ C $点的电场强度的大小和方向。

\end{enumerate}
\begin{figure}[h!]
\flushright
\includesvg[width=0.19\linewidth]{picture/svg/017}
\end{figure}


\banswer{
\begin{enumerate}
\renewcommand{\labelenumi}{\arabic{enumi}.}
% A(\Alph) a(\alph) I(\Roman) i(\roman) 1(\arabic)
%设定全局标号series=example	%引用全局变量resume=example
%[topsep=-0.3em,parsep=-0.3em,itemsep=-0.3em,partopsep=-0.3em]
%可使用leftmargin调整列表环境左边的空白长度 [leftmargin=0em]
\item
$F = 9.0 \times 10 ^ { - 3 } \mathrm { N }$
\item 
$E = 7.8 \times 10 ^ { 3 } \mathrm { N } / \mathrm { C }$,方向沿$ y $轴正方向.



\end{enumerate}
}










\item
\exwhere{$ 2013 $年江苏卷}
下列选项中的各$ \frac{ 1 }{ 4 } $圆环大小相同,所带电荷量已在图中标出,且电荷均匀分布,各$ \frac{ 1 }{ 4 } $圆环间彼此绝缘. 坐标原点$ O $处电场强度最大的是 \xzanswer{B} 
\begin{figure}[h!]
\centering
\includesvg[width=0.79\linewidth]{picture/svg/019}
\end{figure}


\item
\exwhere{$ 2013 $年安徽卷}
如图所示, $ xOy $平面是无穷大导体的表面,该导体充满$ z<0 $的空间,$ z>0 $的空间为真空。将电荷为$ q $的点电荷置于$ z $轴上$ z=h $处,则在$ xOy $平面上会产生感应电荷。空间任意一点处的电场皆是由点电荷$ q $和导体表面上的感应电荷共同激发的。已知静电平衡时导体内部场强处处为零,则在$ z $轴上$ z=\frac{h}{2} $处的场强大小为($ k $为静电力常量) \xzanswer{D} 

\begin{minipage}[h!]{0.7\linewidth}
\vspace{0.3em}
\fourchoices
{$k \frac { 4 q } { h ^ { 2 } }$}
{$k \frac { 4 q } { 9 h ^ { 2 } }$}
{$k \frac { 32 q } { 9 h ^ { 2 } }$}
{$k \frac { 40 q } { 9 h ^ { 2 } }$}
\vspace{0.3em}
\end{minipage}
\hfill
\begin{minipage}[h!]{0.3\linewidth}
\flushright
\vspace{0.3em}
\includesvg[width=0.7\linewidth]{picture/svg/020}
\vspace{0.3em}
\end{minipage}





\item
\exwhere{$ 2011 $年重庆卷}
如图所示,电量为$ +q $和$ -q $的点电荷分别位于正方体的顶点,正方体范围内电场强度为零的点有 \xzanswer{D} 
\begin{figure}[h!]
\centering
\includesvg[width=0.21\linewidth]{picture/svg/021}
\end{figure}

\fourchoices
{体中心、各面中心和各边中点}
{体中心和各边中点}
{各面中心和各边中点 }
{体中心和各面中心}





\item
\exwhere{$ 2012 $年安徽卷}
如图$ 1 $所示,半径为$ R $的均匀带电圆形平板,单位面积带电量为$\sigma$,其轴线上任意一点$ P $(坐标为)的电场强度可以由库仑定律和电场强度的叠加原理求出:$E = 2 \pi k \sigma \left[ 1 - \frac { x } { \left( R ^ { 2 } + x ^ { 2 } \right) ^ { 1 / 2 } } \right]$,方向沿$ x $轴。现考虑单位面积带电量为$\sigma _ { 0 } $的无限大均匀带电平板,从其中间挖去一半径为$ r $的圆板,如图$ 2 $所示。则圆孔轴线上任意一点$ Q $(坐标为$ x $)的电场强度为 \xzanswer{A} 
\begin{figure}[h!]
\centering
\includesvg[width=0.29\linewidth]{picture/svg/022}
\end{figure}

\fourchoices
{$2 \pi \kappa \sigma _ { 0 } \frac { x } { \left( r ^ { 2 } + x ^ { 2 } \right) ^ { 1 / 2 } }$}
{$2 \pi \kappa \sigma _ { 0 } \frac { r } { \left( r ^ { 2 } + x ^ { 2 } \right) ^ { 1 / 2 } }$}
{$2 \pi \kappa \sigma _ { 0 } \frac { x } { r }$}
{$2 \pi \kappa \sigma _ { 0 } \frac { r } { x }$}





\item
\exwhere{$ 2015 $年山东卷}
直角坐标系$ xOy $中,$ M $、$ N $两点位于$ x $轴上,$ G $、$ H $两点坐标如图。$ M $、$ N $两点各固定一负点电荷,一电量为$ Q $的正点电荷置于$ O $点时,$ G $点处的电场强度恰好为零。静电力常量用$ k $表示。若将该正点电荷移到$ G $点,则$ H $点处场强的大小和方向分别 \xzanswer{B} 
\begin{figure}[h!]
\centering
\includesvg[width=0.29\linewidth]{picture/svg/023}
\end{figure}

\fourchoices
{$\frac { 3 k Q } { 4 a ^ { 2 } }$,沿$ y $轴正向}
{$\frac { 3 k Q } { 4 a ^ { 2 } }$,沿$ y $轴负向}
{$\frac { 5 k Q } { 4 a ^ { 2 } }$,沿$ y $轴正向 }
{$\frac { 5 k Q } { 4 a ^ { 2 } }$,沿$ y $轴负向}





\item
\exwhere{$ 2015 $年理综安徽卷}
已知均匀带电的无穷大平面在真空中激发电场的场强大小为$\frac { \sigma } { 2 \varepsilon _ { 0 } }$,其中$ \sigma $	为平面上单位面积所带的电荷量,$ \varepsilon _0 $为常量。如图所示的平行板电容器,极板正对面积为$ S $,其间为真空,带电量为$ Q $。不计边缘效应时,极板可看作无穷大导体板,则极板间的电场强度大小和两极板间相互的静电引力大小分别为 \xzanswer{D} 
\begin{figure}[h!]
\centering
\includesvg[width=0.13\linewidth]{picture/svg/024}
\end{figure}
\fourchoices
{$\frac { Q } { \varepsilon _ { 0 } S } \text{和} \frac { Q ^ { 2 } } { \varepsilon _ { 0 } S }$}
{$\frac { Q } { 2 \varepsilon _ { 0 } S } \text{和} \frac { Q ^ { 2 } } { \varepsilon _ { 0 } S }$}
{$\frac { Q } { 2\varepsilon _ { 0 } S } \text{和} \frac { Q ^ { 2 } } { 2\varepsilon _ { 0 } S }$}
{$\frac { Q } { \varepsilon _ { 0 } S } \text{和} \frac { Q ^ { 2 } } { 2\varepsilon _ { 0 } S }$}



\item
\exwhere{$ 2015 $年理综浙江卷}
如图所示为静电力演示仪,两金属极板分别固定于绝缘支架上,且正对平行放置。工作时两板分别接高压直流电源的正负极,表面镀铝的乒乓球用绝缘细线悬挂在金属极板中间,则 \xzanswer{D} 
\begin{figure}[h!]
\centering
\includesvg[width=0.29\linewidth]{picture/svg/025}
\end{figure}


\fourchoices
{乒乓球的左侧感应出负电荷}
{乒乓球受到扰动后,会被吸在左极板上}
{乒乓球共受到电场、重力和库仑力三个力的作用}
{用绝缘棒将乒乓球拨到与右极板接触,放开后乒乓球会在两极板间来回碰撞}








\end{enumerate}



