\bta{交流电的变化规律}




\begin{enumerate}
%\renewcommand{\labelenumi}{\arabic{enumi}.}
% A(\Alph) a(\alph) I(\Roman) i(\roman) 1(\arabic)
%设定全局标号series=example	%引用全局变量resume=example
%[topsep=-0.3em,parsep=-0.3em,itemsep=-0.3em,partopsep=-0.3em]
%可使用leftmargin调整列表环境左边的空白长度 [leftmargin=0em]
\item
\exwhere{$ 2012 $ 年理综广东卷}
某小型发电机产生的交变电动势为 $ e=50 \sin 100 \pi t $($ V $)
,对此电动势,下列表述正确的有 \xzanswer{CD} 

\fourchoices
{最大值是 $ 50\sqrt{2} \ V $}
{频率是 $ 100 \ Hz $}
{有效值是 $ 25\sqrt{2} \ V $}
{周期是 $ 0.02 \ s $}


\item 
\exwhere{$ 2013 $ 年山东卷}
如图甲所示为交流发电机的示意图,两磁极 $ N $、$ S $ 间的磁场可视为水平方向匀强磁场,$ A $ 为交
流电流表。线圈绕垂直于磁场的水平轴 $ OO ^{\prime} $沿逆时针方向匀速转动,从图示位置开始计时,产生
的交变电流随时间变化的图像如图乙所
示。以下判断正确的是 \xzanswer{AC} 
\begin{figure}[h!]
\centering
\begin{subfigure}{0.4\linewidth}
\centering
\includesvg[width=0.7\linewidth]{picture/svg/GZ-3-tiyou-1135} 
\caption{}\label{}
\end{subfigure}
\begin{subfigure}{0.4\linewidth}
\centering
\includesvg[width=0.7\linewidth]{picture/svg/GZ-3-tiyou-1136} 
\caption{}\label{}
\end{subfigure}
\end{figure}




\fourchoices
{电流表的示数为 $ 10 \ A $}
{线圈转动的角速度为 $ 50 \pi \ rad/s $}
{$ 0.01 \ s $ 时,线圈平面与磁场方向平行}
{$ 0.02 \ s $ 时,电阻 $ R $ 中电流的方向自右向左}



\item 
\exwhere{$ 2011 $ 年理综天津卷}
在匀强磁场中,一矩形金属线框绕与磁感线垂直的转轴匀速转动,如图 $ 1 $ 所示。产生的交变电
动势的图象如图 $ 2 $ 所示,则 \xzanswer{B} 
\begin{figure}[h!]
\centering
\begin{subfigure}{0.4\linewidth}
\centering
\includesvg[width=0.7\linewidth]{picture/svg/GZ-3-tiyou-1137} 
\caption{}\label{}
\end{subfigure}
\begin{subfigure}{0.4\linewidth}
\centering
\includesvg[width=0.7\linewidth]{picture/svg/GZ-3-tiyou-1138} 
\caption{}\label{}
\end{subfigure}
\end{figure}

\fourchoices
{$ t=0.005 \ s $ 时线框的磁通量变化率为零}
{$ t=0.01 \ s $ 时线框平面与中性面重合}
{线框产生的交变电动势有效值为 $ 311 \ V $}
{线框产生的交变电动势频率为 $ 100 \ Hz $}



\item 
\exwhere{$ 2013 $ 年福建卷}
如图,实验室一台手摇交流发电机,内阻 $ r=1.0 \ \Omega $,外接 $ R=9.0 \ \Omega $的电阻。闭合开关 $ S $,当发电
机转子以某一转速匀速转动时,产生的电动势 $ e=10\sqrt{2} \sin 10 \pi t(V) $,则 \xzanswer{D} 
\begin{figure}[h!]
\centering
\includesvg[width=0.23\linewidth]{picture/svg/GZ-3-tiyou-1139}
\end{figure}

\fourchoices
{该交变电流的频率为 $ 10 \ Hz $}
{该电动势的有效值为 $ 10\sqrt{2} \ V $}
{外接电阻 $ R $ 所消耗的电功率为 $ 10 \ W $}
{电路中理想交流电流表 $ A $ 的示数为 $ 1.0 \ A $}


\item 
\exwhere{$ 2013 $ 年海南卷}
通过一阻值 $ R=100 \ \Omega $的电阻的交变电流如图所示,其周期为 $ 1 \ s $。电阻两端电压的有效值为 \xzanswer{B} 
\begin{figure}[h!]
\centering
\includesvg[width=0.23\linewidth]{picture/svg/GZ-3-tiyou-1140}
\end{figure}

\fourchoices
{$ 12 \ V $}
{$ 4\sqrt{10} \ V $}
{$ 15 \ V $}
{$ 8\sqrt{5} \ V $}

\item 
\exwhere{$ 2012 $ 年理综全国卷}
一台电风扇的额定电压为交流 $ 220 \ V $。在其正常工作过程中,用交流电流表测得某一段时间内的
工作电流 $ I $ 随时间 $ t $ 的变化如图所示。这段时间内电风扇的用电量为 \xzanswer{B} 
\begin{figure}[h!]
\centering
\includesvg[width=0.23\linewidth]{picture/svg/GZ-3-tiyou-1141}
\end{figure}

\fourchoices
{$ 3.9 \times 10^{-2} $ 度}
{$ 5.5 \times 10^{-2} $ 度}
{$ 7.8 \times 10^{-2} $ 度}
{$ 11.0 \times 10^{-2} $ 度}



\item 
\exwhere{$ 2012 $ 年理综北京卷}
一个小型电热器若接在输出电压为 $ 10 \ V $ 的直流电源上.消耗电功率为 $ P $;若把它接在某个正弦交
流电源上,其消耗的电功率为$ \frac{P}{2} $。如果电热器电阻不变,则此交流电源输出电压的最大值为 \xzanswer{C} 

\fourchoices
{$ 5 \ V $}
{$ 5\sqrt{2} \ V $}
{$ 10 \ V $}
{$ 10\sqrt{2} \ V $}

\item 
\exwhere{$ 2014 $ 年理综天津卷}
如图 $ 1 $ 所示,在匀强磁场中,一矩形金属线圈两次分别以不同的转速,绕与磁感线垂直的轴匀
速转动,产生的交变电动势图象如图 $ 2 $ 中曲
线 $ a $、$ b $ 所示,则 \xzanswer{AC} 
\begin{figure}[h!]
\centering
\begin{subfigure}{0.4\linewidth}
\centering
\includesvg[width=0.4\linewidth]{picture/svg/GZ-3-tiyou-1142} 
\caption{}\label{}
\end{subfigure}
\begin{subfigure}{0.4\linewidth}
\centering
\includesvg[width=0.7\linewidth]{picture/svg/GZ-3-tiyou-1143} 
\caption{}\label{}
\end{subfigure}
\end{figure}



\fourchoices
{两次 $ t=0 $ 时刻线圈平面均与中性面重合}
{曲线 $ a $、$ b $ 对应的线圈转速之比为 $ 2:3 $}
{曲线 $ a $ 表示的交变电动势频率为 $ 25 \ Hz $}
{曲线 $ b $ 表示的交变电动势有效值为 $ 10 \ V $}


\item 
\exwhere{$ 2015 $ 年理综四川卷}
小型发电机线圈共 $ N $ 匝,每匝可简化为矩形线圈 $ abcd $,磁极间的磁场视
为匀强磁场,方向垂直于线圈中心轴 $ OO ^{\prime} $,线圈绕 $ OO ^{\prime} $匀速转
动,如图所示。矩形线圈 $ ab $ 边和 $ cd $ 边产生的感应电动势的最大
值都为 $ e_{0} $,不计线圈电阻,则发电机输出电压 \xzanswer{D} 
\begin{figure}[h!]
\centering
\includesvg[width=0.23\linewidth]{picture/svg/GZ-3-tiyou-1144}
\end{figure}

\fourchoices
{峰值是 $ e_{0} $}
{峰值是 $ 2e_{0} $}
{有效值是$\frac{\sqrt{2}}{2} N e_{0}$}
{有效值是$\sqrt{2} \mathrm{Ne}_{0}$}


\item 
\exwhere{$ 2011 $ 年理综安徽卷}
如图所示的区域内有垂直于纸面的匀强磁场,磁感应强度为 $ B $。电阻
为 $ R $、半径为 $ L $、圆心角为 $ 45 \degree $的扇形闭合导线框绕垂直于纸面的 $ O $ 轴以
角速度$ \omega $匀速转动($ O $ 轴位于磁场边界)。则线框内产生的感应电流的有效
值为 \xzanswer{D} 
\begin{figure}[h!]
\centering
\includesvg[width=0.23\linewidth]{picture/svg/GZ-3-tiyou-1145}
\end{figure}


\fourchoices
{$\frac{B L^{2} \omega}{2 R}$}
{$\frac{\sqrt{2} B L^{2} \omega}{2 R}$}
{$\frac{\sqrt{2} B L^{2} \omega}{4 R}$}
{$\frac{B L^{2} \omega}{4 R}$}



\item
\exwhere{$ 2011 $ 年理综四川卷}
如图所示,在匀强磁场中匀速转动的矩形线圈的周期为 $ T $,转轴 $ O_{1} O_{2} $ 垂直于磁场方向,线圈
电阻为 $ 2 \ \Omega $。从线圈平面与磁场方向平行时开始计时,线圈转过 $ 60 \degree $时的感应
电流为 $ 1 \ A $。那么 \xzanswer{AC} 
\begin{figure}[h!]
\centering
\includesvg[width=0.23\linewidth]{picture/svg/GZ-3-tiyou-1146}
\end{figure}


\fourchoices
{线圈消耗的电功率为 $ 4 \ W $}
{线圈中感应电流的有效值为 $ 2 \ A $}
{任意时刻线圈中的感应电动势为 $e=4 \cos \frac{2 \pi}{T} t$}
{任意时刻穿过线圈的磁通量为 $\Phi=\frac{T}{\pi} \sin \frac{2 \pi}{T} t$}



\item 
\exwhere{$ 2018 $ 年全国\lmd{3}卷}
一电阻接到方波交流电源上,在一个周期
内产生的热量为 $ Q_{ \text{方} } $ ;若该电阻接到正弦交流电源上,在一个周期内
产生的热量为 $ Q_{ \text{正} } $。该电阻上电压的峰值均为 $ u_{0} $,周期均为 $ T $,如图
所示。则 $ Q _{ \text{方} } :Q_{ \text{正} } $ 等于 \xzanswer{D} 
\begin{figure}[h!]
\centering
\begin{subfigure}{0.4\linewidth}
\centering
\includesvg[width=0.7\linewidth]{picture/svg/GZ-3-tiyou-1147} 
\caption{}\label{}
\end{subfigure}
\begin{subfigure}{0.4\linewidth}
\centering
\includesvg[width=0.7\linewidth]{picture/svg/GZ-3-tiyou-1148} 
\caption{}\label{}
\end{subfigure}
\end{figure}



\fourchoices
{$ 1:\sqrt{2} $}
{$ \sqrt{2}:1 $}
{$ 1:2 $}
{$ 2:1 $}


\item 
\exwhere{$ 2012 $ 年物理江苏卷}
某兴趣小组设计了一种发电装置,如图所示. 在磁极和圆柱状铁芯之间形成的两磁场区域
的圆心角$ \alpha $均为
$ \frac{ 4 }{ 9 } \pi $,磁场均沿半径方向. 匝数为
$ N $ 的矩形线圈 $ abcd $ 的边长 $ ab=cd=l $、$ bc=ad=2l $。
线圈以角速度$ \omega $绕中心轴匀速转动,$ bc $ 和 $ ad $ 边同
时进入磁场. 在磁场中,两条边所经过处的磁感
应强度大小均为 $ B $、方向始终与两边的运动方
向垂直. 线圈的总电阻为 $ r $,外接电阻为 $ R $. 求:
\begin{enumerate}
%\renewcommand{\labelenumi}{\arabic{enumi}.}
% A(\Alph) a(\alph) I(\Roman) i(\roman) 1(\arabic)
%设定全局标号series=example	%引用全局变量resume=example
%[topsep=-0.3em,parsep=-0.3em,itemsep=-0.3em,partopsep=-0.3em]
%可使用leftmargin调整列表环境左边的空白长度 [leftmargin=0em]
\item
线圈切割磁感线时,感应电动势的大小 $ E_{m} $;

\item 
线圈切割磁感线时,$ bc $ 边所受安培力的大小$ F $;


\item 
外接电阻上电流的有效值 $ I $。


\end{enumerate}
\begin{figure}[h!]
\centering
\includesvg[width=0.23\linewidth]{picture/svg/GZ-3-tiyou-1149}
\end{figure}

\banswer{
\begin{enumerate}
%\renewcommand{\labelenumi}{\arabic{enumi}.}
% A(\Alph) a(\alph) I(\Roman) i(\roman) 1(\arabic)
%设定全局标号series=example	%引用全局变量resume=example
%[topsep=-0.3em,parsep=-0.3em,itemsep=-0.3em,partopsep=-0.3em]
%可使用leftmargin调整列表环境左边的空白长度 [leftmargin=0em]
\item
$E_{m}=2 N B l^{2} \omega$
\item 
$F=\frac{4 N^{2} B^{2} l^{3} \omega}{r+R}$
\item 
$I=\frac{4 N B l^{2} \omega}{3(r+R)}$
\end{enumerate}
}


\item 
\exwhere{$ 2017 $ 年天津卷}
在匀强磁场中,一个 $ 100 $ 匝的闭合矩形金属线圈,绕与磁感线垂直的固定轴
匀速转动,穿过该线圈的磁通量随时间按图示正弦规律变化。设线圈总电阻为 $ 2 \ \Omega $,则 \xzanswer{AD} 
\begin{figure}[h!]
\centering
\includesvg[width=0.23\linewidth]{picture/svg/GZ-3-tiyou-1150}
\end{figure}

\fourchoices
{$ t=0 $ 时,线圈平面平行于磁感线}
{$ t=1 \ s $ 时,线圈中的电流改变方向}
{$ t=1.5 \ s $ 时,线圈中的感应电动势最大}
{一个周期内,线圈产生的热量为 $ 8 \pi^{2} \ J $}



\item 
\exwhere{$ 2019 $ 年物理天津卷}
单匝闭合矩形线框电阻为 $ R $,在匀强磁场中绕与磁感线垂直的轴匀速转
动,穿过线框的磁通量 $ \Phi $ 与时间 $ t $ 的关系图像如图所示。下列说法正确的是 \xzanswer{BC} 
\begin{figure}[h!]
\centering
\begin{subfigure}{0.4\linewidth}
\centering
\includesvg[width=0.7\linewidth]{picture/svg/GZ-3-tiyou-1151} 
\caption{}\label{}
\end{subfigure}
\begin{subfigure}{0.4\linewidth}
\centering
\includesvg[width=0.7\linewidth]{picture/svg/GZ-3-tiyou-1152} 
\caption{}\label{}
\end{subfigure}
\end{figure}



\fourchoices
{$ \frac{T}{2} $时刻线框平面与中性面垂直}
{线框的感应电动势有效值为$\frac{\sqrt{2} \pi \Phi_{m}}{T}$}
{线框转一周外力所做的功为$\frac{2 \pi^{2} \Phi_{m}^{2}}{R T}$}
{从 $t=0$ 到 $t=\frac{T}{4}$ 过程中线框的平均感应电动势为 $\frac{\pi \Phi_{m}}{T}$}




\item 
\exwhere{$ 2012 $ 年理综安徽卷}
图 $ 1 $ 是交流发电机模型示意图。在磁感应强度为 $ B $ 的匀强磁场中,有一矩形线圈 $ abcd $ 可绕线圈平
面内垂直于磁感线的轴
$ OO ^{\prime} $ 转动,由线圈引起的
导线 $ ae $ 和 $ df $ 分别与两个
跟线圈一起绕 $ OO ^{\prime} $ 转动
的金属圆环相连接,金属
圆环又分别与两个固定的
电刷保持滑动接触,这样
矩形线圈在转动中就可以保持和外电路电阻 $ R $ 形成闭合电路。图 $ 2 $ 是线圈的主视图,导线 $ ab $ 和 $ cd $
分别用它们的横截面来表示。已知 $ ab $ 长度为 $ L_{1} ,bc $ 长度为 $ L_{2} $,线圈以恒定角速度$ \omega $逆时针转动。(只
考虑单匝线圈)
\begin{enumerate}
%\renewcommand{\labelenumi}{\arabic{enumi}.}
% A(\Alph) a(\alph) I(\Roman) i(\roman) 1(\arabic)
%设定全局标号series=example	%引用全局变量resume=example
%[topsep=-0.3em,parsep=-0.3em,itemsep=-0.3em,partopsep=-0.3em]
%可使用leftmargin调整列表环境左边的空白长度 [leftmargin=0em]
\item
线圈平面处于中性面位置时开始计时,试推导 $ t $ 时刻整个线圈中的感应电动势 $ e_{1} $ 的表达式;

\item 
线圈平面处于与中性面成 $ \phi_{0} $ 夹角位置时开始计时,如图 $ 3 $ 所示,试写出 $ t $ 时刻整个线圈中的
感应电动势 $ e_{2} $ 的表达式;

\item 
若线圈电阻为 $ r $,求线圈每转动一周电阻 $ R $ 上产生的焦耳热。(其它电阻均不计)




\end{enumerate}
\begin{figure}[h!]
\centering
\begin{subfigure}{0.4\linewidth}
\centering
\includesvg[width=0.7\linewidth]{picture/svg/GZ-3-tiyou-1153} 
\caption{}\label{}
\end{subfigure}
\begin{subfigure}{0.4\linewidth}
\centering
\includesvg[width=0.7\linewidth]{picture/svg/GZ-3-tiyou-1154} 
\caption{}\label{}
\end{subfigure}
\begin{subfigure}{0.4\linewidth}
\centering
\includesvg[width=0.7\linewidth]{picture/svg/GZ-3-tiyou-1155} 
\caption{}\label{}
\end{subfigure}
\end{figure}


\banswer{
\begin{enumerate}
%\renewcommand{\labelenumi}{\arabic{enumi}.}
% A(\Alph) a(\alph) I(\Roman) i(\roman) 1(\arabic)
%设定全局标号series=example	%引用全局变量resume=example
%[topsep=-0.3em,parsep=-0.3em,itemsep=-0.3em,partopsep=-0.3em]
%可使用leftmargin调整列表环境左边的空白长度 [leftmargin=0em]
\item
$e_{1}=B \omega L_{1} L_{2} \sin \omega t$
\item 
$e_{2}=B \omega L_{1} L_{2} \sin \left(\omega t+\varphi_{0}\right)$
\item 
$W=I^{2} R T=\pi \omega R\left(\frac{B L_{1} L_{2}}{R+r}\right)^{2}$
\end{enumerate}
}








\end{enumerate}

