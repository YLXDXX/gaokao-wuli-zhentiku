\bta{带电粒子在磁场中的圆周运动(上)}

\begin{enumerate}[leftmargin=0em]
\renewcommand{\labelenumi}{\arabic{enumi}.}
% A(\Alph) a(\alph) I(\Roman) i(\roman) 1(\arabic)
%设定全局标号series=example	%引用全局变量resume=example
%[topsep=-0.3em,parsep=-0.3em,itemsep=-0.3em,partopsep=-0.3em]
%可使用leftmargin调整列表环境左边的空白长度 [leftmargin=0em]
\item
\exwhere{$ 2013 $年新课标\lmd{1}卷}
如图,半径为$ R $的圆是一圆柱形匀强磁场区域的横截面(纸面),磁感应强度大小为$ B $,方向垂直于纸面向外。一电荷量为$ q $($ q>0 $)、质量为$ m $的粒子沿平行于直径$ ab $的方向射入磁场区域,射入点与$ ab $的距离为$ R/2 $,已知粒子射出磁场与射入磁场时运动方向间的夹角为$ 60 ^{ \circ } $,则粒子的速率为(不计重力) \xzanswer{B} 

\begin{figure}[h!]
\centering
\includesvg[width=0.18\linewidth]{picture/svg/197}
\end{figure}


\fourchoices
{$\frac { q B R } { 2 m } \quad$}
{$\frac { q B R } { m } \quad$}
{$\frac { 3 q B R } { 2 m } \quad$}
{$\frac { 2 q B R } { m }$}




\item
\exwhere{$ 2013 $年新课标\lmd{2}卷}
空间有一圆柱形匀强磁场区域,该区域的横截面的半径为$ R $,磁场方向垂直横截面。一质量为$ m $、电荷量为$ q $($ q $>$ 0 $)的粒子以速率$ v_{0} $沿横截面的某直径射入磁场,离开磁场时速度方向偏离入射方向$ 60 ^{ \circ } $。不计重力,该磁场的磁感应强度大小为 \xzanswer{A} 

\fourchoices
{$\frac { \sqrt { 3 } m v _ { 0 } } { 3 q R } \quad$}
{$\frac { m v _ { 0 } } { q R } \quad$}
{$\frac { \sqrt { 3 } m v _ { 0 } } { q R } \quad$}
{$\frac { 3 m v _ { 0 } } { q R }$}




\item
\exwhere{$ 2012 $年理综全国卷}
质量分别为$ m_{1} $和$ m_{2} $、电荷量分别为$ q_{1} $和$ q_{2} $的两粒子在同一匀强磁场中做匀速圆周运动,已知两粒子的动量大小相等。下列说法正确的是 \xzanswer{A} 


\fourchoices
{若$ q_{1} = q_{2} $,则它们作圆周运动的半径一定相等}
{若$ m_{1}= m_{2} $,则它们作圆周运动的半径一定相等}
{若$ q_{1} \neq q_{2} $,则它们作圆周运动的周期一定不相等}
{若$ m_{1} \neq m_{2} $,则它们作圆周运动的周期一定不相等}




\item
\exwhere{$ 2012 $年理综北京卷}
处于匀强磁场中的一个带电粒子,仅在磁场力作用下做匀速圆周运动。将该粒子的运动等效为环形电流,那么此电流值 \xzanswer{D} 


\fourchoices
{与粒子电荷量成正比}
{与粒子速率成正比}
{与粒子质量成正比}
{与磁感应强度成正比}


\item
\exwhere{$ 2012 $年理综广东卷}
质量和电量都相等的带电粒子$ M $和$ N $,以不同的速率经小孔$ S $垂直进入匀强磁场,运行的半圆轨迹如图中虚线所示,下列表述正确的是 \xzanswer{A} 
\begin{figure}[h!]
\centering
\includesvg[width=0.28\linewidth]{picture/svg/198}
\end{figure}


\fourchoices
{$ M $带负电,$ N $带正电}
{$ M $的速率小于$ N $的速率}
{洛仑兹力对$ M $、$ N $做正功}
{$ M $的运行时间大于$ N $的运行时间}



\item
\exwhere{$ 2013 $年理综广东卷}
如图,两个初速度大小相同的同种离子$ a $和$ b $,从$ O $点沿垂直磁场方向进入匀强磁场,最后打到屏$ P $上。不计重力。下列说法正确的有 \xzanswer{AD} 


\begin{minipage}[h!]{0.7\linewidth}
\vspace{0.3em}
\fourchoices
{$ a $、$ b $均带正电}
{$ a $在磁场中飞行的时间比$ b $的短}
{$ a $在磁场中飞行的路程比$ b $的短}
{$ a $在$ P $上的落点与$ O $点的距离比$ b $的近}

\vspace{0.3em}
\end{minipage}
\hfill
\begin{minipage}[h!]{0.3\linewidth}
\flushright
\vspace{0.3em}
\includesvg[width=0.57\linewidth]{picture/svg/199}
\vspace{0.3em}
\end{minipage}



\item
\exwhere{$ 2014 $年理综北京卷}
带电粒子$ a $、$ b $在同一匀强磁场中做匀速圆周运动,他们的动量大小相等,$ a $运动的半径大于$ b $运动的半径。若$ a $、$ b $的电荷量分别为$ q_a $、$ q_b $,质量分别为$ m_a $、$ m_b $,周期分别为$ T_a $、$ T_b $。则一定有 \xzanswer{A} 
\fourchoices
{$q _ { a } < q _ { b } $}
{$m _ { a } < m _ { b } $}
{$ T _ { a } < T _ { b } $}
{$\frac { q _ { a } } { m _ { a } } < \frac { q _ { b } } { m _ { b } }$}



\item
\exwhere{$ 2014 $年理综安徽卷}
“人造小太阳”托卡马克装置使用强磁场约束高温等离子体,使其中的带电粒子被尽可能限制在装置内部,而不与装置器壁碰撞。已知等离子体中带电粒子的平均动能与等离子体的温度$ T $成正比,为约束更高温度的等离子体,则需要更强的磁场,以使带电粒子的运动半径不变。由此可判断所需的磁感应强度$ B $正比于 \xzanswer{A} 
\fourchoices
{$\sqrt { T } \quad$}
{$T \quad$}
{$\sqrt { T ^ { 3 } } \quad$}
{$T ^ { 2 }$}



\item
\exwhere{$ 2015 $年理综新课标\lmd{1}卷}
两相邻匀强磁场区域的磁感应强度大小不同、方向平行。一速度方向与磁感应强度方向垂直的带电粒子(不计重力),从较强磁场区域进入到较弱磁场区域后,粒子的 \xzanswer{D} 


\fourchoices
{轨道半径减小,角速度增大}
{轨道半径减小,角速度减小}
{轨道半径增大,角速度增大}
{轨道半径增大,角速度减小}


\item
\exwhere{$ 2016 $年新课标\lmd{2}卷}
一圆筒处于磁感应强度大小为$ B $的匀强磁场中,磁场方向与筒的轴平行,筒的横截面如图所示。图中直径$ MN $的两端分别开有小孔,筒绕其中心轴以角速度$ \omega $顺时针转动。在该截面内,一带电粒子从小孔$ M $射入筒内,射入时的运动方向与$ MN $成$ 30 ^{ \circ } $角。当筒转过$ 90 ^{ \circ } $时,该粒子恰好从小孔$ N $飞出圆筒。不计重力。若粒子在筒内未与筒壁发生碰撞,则带电粒子的比荷为 \xzanswer{A} 
\begin{figure}[h!]
\centering
\includesvg[width=0.23\linewidth]{picture/svg/200}
\end{figure}
\fourchoices
{$\frac { \omega } { 3 B } \quad $}
{$ \cdot \frac { \omega } { 2 B } \quad $}
{$ \cdot \frac { \omega } { B } \quad $}
{$ \cdot \frac { 2 \omega } { B }$}


\item
\exwhere{$ 2012 $年物理江苏卷}
如图所示,$ MN $是磁感应强度为$ B $的匀强磁场的边界. 一质量为$ m $、电荷量为$ q $的粒子在纸面内从$ O $点射入磁场. 若粒子速度为$ v_{0} $,最远能落在边界上的$ A $点. 下列说法正确的有 \xzanswer{BC} 
\begin{figure}[h!]
\centering
\includesvg[width=0.25\linewidth]{picture/svg/238}
\end{figure}



\fourchoices
{若粒子落在$ A $点的左侧,其速度一定小于$ v_{0} $}
{若粒子落在$ A $点的右侧,其速度一定大于$ v_{0} $}
{若粒子落在$ A $点左右两侧$ d $的范围内,其速度不可能小于$v _ { 0 } - \frac { q B d } { 2 m }$}
{若粒子落在$ A $点左右两侧$ d $的范围内,其速度不可能大于$v _ { 0 } + \frac { q B d } { 2 m }$}




\item
\exwhere{$ 2012 $年理综安徽卷}
如图所示,圆形区域内有垂直于纸面向里的匀强磁场,一个带电粒子以速度$ v $从$ A $点沿直径$ AOB $方向射入磁场,经过$ \Delta $$ t $时间从$ C $点射出磁场,$ OC $与$ OB $成$ 60 ^{ \circ } $角。现将带电粒子的速度变为$ v/3 $,仍从$ A $点沿原方向射入磁场,不计重力,则粒子在磁场中的运动时间变为 \xzanswer{B} 
\begin{figure}[h!]
\centering
\includesvg[width=0.23\linewidth]{picture/svg/201}
\end{figure}
\fourchoices
{$ \frac { 1 } { 2 } \Delta t$}
{$ 2 \Delta t$}
{$ \frac { 1 } { 3 } \Delta t$}
{$ 3 \Delta t$}



\item
\exwhere{$ 2016 $年四川卷}
如图所示,正六边形$ abcdef $区域内有垂直于纸面的匀强磁场。一带正电的粒子从$ f $点沿$ fd $方向射入磁场区域,当速度大小为$ v_b $时,从$ b $点离开磁场,在磁场中运动的时间为$ t_b $,当速度大小为$ v_c $时,从$ c $点离开磁场,在磁场中运动的时间为$ t_c $,不计粒子重力。则 \xzanswer{A} 
\begin{figure}[h!]
\centering
\includesvg[width=0.23\linewidth]{picture/svg/202}
\end{figure}


\fourchoices
{$v _ { b }: v _ { c } = 1: 2 , \quad t _ { b }: t _ { c } = 2: 1$}
{$v _ { b }: v _ { c } = 2: 2 , \quad t _ { b }: t _ { c } = 1: 2$}
{$v _ { b }: v _ { c } = 2: 1 , \quad t _ { b }: t _ { c } = 2: 1$}
{$v _ { b }: v _ { c } = 1: 2 , \quad t _ { b }: t _ { c } = 1: 2$}

\item
\exwhere{$ 2016 $年新课标\lmd{3}卷}
平面$ OM $和平面$ ON $之间的夹角为$ 30 ^{ \circ } $,其横截面(纸面)如图所示,平面$ OM $上方存在匀强磁场,磁感应强度大小为$ B $,方向垂直于纸面向外。一带电粒子的质量为$ m $,电荷量为$ q $($ q>0 $)。粒子沿纸面以大小为$ v $的速度从	$ OM $的某点向左上方射入磁场,速度与$ OM $成$ 30 ^{ \circ } $角。已知粒子在磁场中的运动轨迹与$ ON $只有一个交点,并从$ OM $上另一点射出磁场。不计重力。粒子离开磁场的出射点到两平面交线$ O $的距离为 \xzanswer{D} 
\begin{figure}[h!]
\centering
\includesvg[width=0.32\linewidth]{picture/svg/205}
\end{figure}

\fourchoices
{$\frac { m v } { 2 B q } \quad$}
{$\frac { \sqrt { 3 } m v } { B q } \quad$}
{$\frac { 2 m v } { B q } \quad$}
{$\frac { 4 m v } { B q }$}





\item
\exwhere{$ 2015 $年理综新课标\lmd{2}卷}
有两个匀强磁场区域$ I $和 $ II $,$ I $中的磁感应强度是$ II $中的$ k $倍,两个速率相同的电子分别在两磁场区域做圆周运动。与$ I $中运动的电子相比,$ II $中的电子 \xzanswer{AC} 


\fourchoices
{运动轨迹的半径是$ I $中的$ k $倍}
{加速度的大小是$ I $中的$ k $倍}
{做圆周运动的周期是$ I $中的$ k $倍}
{做圆周运动的角速度与$ I $中的相等}




\item
\exwhere{$ 2015 $年广东卷}
在同一匀强磁场中,$ \alpha $粒子($ ^{4}_{2} He $)和质子($ ^{1}_{1}H $)做匀速圆周运动,若它们的动量大小相等,则$ \alpha $粒子和质子 \xzanswer{B} 


\fourchoices
{运动半径之比是$ 2 $∶$ 1 $}
{运动周期之比是$ 2 $∶$ 1 $}
{运动速度大小之比是$ 4 $∶$ 1 $}
{受到的洛伦兹力之比是$ 2 $∶$ 1 $}


\item
\exwhere{$ 2019 $年物理全国\lmd{2}卷}
如图,边长为$ l $的正方形$ abcd $内存在匀强磁场,磁感应强度大小为$ B $,方向垂直于纸面($ abcd $所在平面)向外。$ ab $边中点有一电子发源$ O $,可向磁场内沿垂直于$ ab $边的方向发射电子。已知电子的比荷为$ k $。则从$ a $、$ d $两点射出的电子的速度大小分别为 \xzanswer{B} 
\begin{figure}[h!]
\centering
\includesvg[width=0.23\linewidth]{picture/svg/216}
\end{figure}

\fourchoices
{$\frac { 1 } { 4 } k B l , \quad \frac { \sqrt { 5 } } { 4 } k B l$}
{$\frac { 1 } { 4 } k B l , \quad \frac { 5 } { 4 } k B l$}
{$\frac { 1 } { 2 } k B l , \quad \frac { \sqrt { 5 } } { 4 } k B l$}
{$\frac { 1 } { 2 } k B l , \quad \frac { 5 } { 4 } k B l$}




\item
\exwhere{$ 2019 $年物理北京卷}
如图所示,正方形区域内存在垂直纸面的匀强磁场。一带电粒子垂直磁场边界从$ a $点射入,从$ b $点射出。下列说法正确的是 \xzanswer{C} 
\begin{figure}[h!]
\centering
\includesvg[width=0.23\linewidth]{picture/svg/217}
\end{figure}



\fourchoices
{粒子带正电}
{粒子在$ b $点速率大于在$ a $点速率}
{若仅减小磁感应强度,则粒子可能从$ b $点右侧射出}
{若仅减小入射速率,则粒子在磁场中运动时间变短}




\item
\exwhere{$ 2019 $年物理全国\lmd{3}卷}
如图,在坐标系的第一和第二象限内存在磁感应强度大小分别为$ B $和$ \frac{ 1 }{ 2 } B $、方向均垂直于纸面向外的匀强磁场。一质量为$ m $、电荷量为$ q $($ q>0 $)的粒子垂直于$ x $轴射入第二象限,随后垂直于$ y $轴进入第一象限,最后经过$ x $轴离开第一象限。粒子在磁场中运动的时间为 \xzanswer{B} 

\begin{figure}[h!]
\centering
\includesvg[width=0.23\linewidth]{picture/svg/218}
\end{figure}

\fourchoices
{$\frac { 5 \pi m } { 6 q B } \quad$}
{$\frac { 7 \pi m } { 6 q B } \quad$}
{$\frac { 11 \pi m } { 6 q B } \quad$}
{$\frac { 13 \pi m } { 6 q B }$}

\item
\exwhere{$ 2015 $年理综四川卷}
如图所示,$ S $处有一电子源,可向纸面内任意方向发射电子,平板$ MN $垂直于纸面,在纸面内的长度$ L=9.1 \ cm $,中点$ O $与$ S $间的距离$ d=4.55 \ cm $,$ MN $与$ SO $直线的夹角为$ \theta $,板所在平面有电子源的一侧区域有方向垂直于纸面向外的匀强磁场,磁感应强度$ B=2.0 \times 10^{-4} \ T $,电子质量$ m=9.1 \times 10^{-31} \ kg $,电量$ e=1.6 \times 10^{-19}\ C $,不计电子重力。电子源发射速度$ v=1.6 \times 10^{6} \ m/s $的一个电子,该电子打在板上可能位置
的区域的长度为$ l $,则 \xzanswer{AD} 
\begin{figure}[h!]
\centering
\includesvg[width=0.23\linewidth]{picture/svg/207}
\end{figure}


\fourchoices
{$ \theta =90 ^{\circ} $时,$ l=9.1 \ cm $}
{$ \theta =60^{\circ} $时,$ l=9.1 \ cm $}
{$ \theta =45^{\circ} $时,$ l=4.55 \ cm $}
{$ \theta =30^{\circ} $时,$ l=4.55 \ cm $}




\newpage
\item
\exwhere{$ 2013 $年北京卷}
如图所示,两平行金属板间距为$ d $,电势差为$ U $,板间电场可视为匀强电场;金属板下方有一磁感应强度为$ B $的匀强磁场。带电量为$ +q $、质量为$ m $的粒子,由静止开始从正极板出发,经电场加速后射出,并进入磁场做匀速圆周运动。忽略重力的影响,求:
\begin{enumerate}
\renewcommand{\labelenumi}{\arabic{enumi}.}
% A(\Alph) a(\alph) I(\Roman) i(\roman) 1(\arabic)
%设定全局标号series=example	%引用全局变量resume=example
%[topsep=-0.3em,parsep=-0.3em,itemsep=-0.3em,partopsep=-0.3em]
%可使用leftmargin调整列表环境左边的空白长度 [leftmargin=0em]
\item
匀强电场场强$ E $的大小;
\item 
粒子从电场射出时速度$ v $的大小;
\item 
粒子在磁场中做匀速圆周运动的半径$ R $。


\end{enumerate}
\begin{figure}[h!]
\flushright
\includesvg[width=0.36\linewidth]{picture/svg/203}
\end{figure}

\banswer{
\begin{enumerate}
\renewcommand{\labelenumi}{\arabic{enumi}.}
% A(\Alph) a(\alph) I(\Roman) i(\roman) 1(\arabic)
%设定全局标号series=example	%引用全局变量resume=example
%[topsep=-0.3em,parsep=-0.3em,itemsep=-0.3em,partopsep=-0.3em]
%可使用leftmargin调整列表环境左边的空白长度 [leftmargin=0em]
\item
$\frac { U } { d }$
\item 
$\sqrt { \frac { 2 q U } { m } }$
\item 
$\frac { 1 } { B } \sqrt { \frac { 2 m U } { q } }$



\end{enumerate}
}




\newpage
\item
\exwhere{$ 2016 $年北京卷}
如图所示,质量为$ m $,电荷量为$ q $的带电粒子,以初速度$ v $沿垂直磁场方向射入磁感应强度为$ B $的匀强磁场,在磁场中做匀速圆周运动。不计带电粒子所受重力。
\begin{enumerate}
\renewcommand{\labelenumi}{\arabic{enumi}.}
% A(\Alph) a(\alph) I(\Roman) i(\roman) 1(\arabic)
%设定全局标号series=example	%引用全局变量resume=example
%[topsep=-0.3em,parsep=-0.3em,itemsep=-0.3em,partopsep=-0.3em]
%可使用leftmargin调整列表环境左边的空白长度 [leftmargin=0em]
\item
求粒子做匀速圆周运动的半径$ R $和周期$ T $;
\item 
为使该粒子做匀速直线运动,还需要同时存在一个与磁场方向垂直的匀强电场,求电场强度$ E $的大小。


\end{enumerate}

\begin{figure}[h!]
\flushright
\includesvg[width=0.2\linewidth]{picture/svg/204}
\end{figure}

\banswer{
\begin{enumerate}
\renewcommand{\labelenumi}{\arabic{enumi}.}
% A(\Alph) a(\alph) I(\Roman) i(\roman) 1(\arabic)
%设定全局标号series=example	%引用全局变量resume=example
%[topsep=-0.3em,parsep=-0.3em,itemsep=-0.3em,partopsep=-0.3em]
%可使用leftmargin调整列表环境左边的空白长度 [leftmargin=0em]
\item
$R = \frac { m v } { B q } , \quad T = \frac { 2 \pi m } { B q }$
\item 
$E = v B$



\end{enumerate}
}







\newpage
\item
\exwhere{$ 2016 $年海南卷}
如图,$ A $、$ C $两点分别位于$ x $轴和$ y $轴上,$ \angle OCA=30 ^{ \circ } $,$ OA $的长度为$ L $。在Δ$ OCA $区域内有垂直于$ xOy $平面向里的匀强磁场。质量为$ m $、电荷量为$ q $的带正电粒子,以平行于$ y $轴的方向从$ OA $边射入磁场。已知粒子从某点射入时,恰好垂直于$ OC $边射出磁场,且粒子在磁场中运动的时间为$ t_{0} $。不计重力。
\begin{enumerate}
\renewcommand{\labelenumi}{\arabic{enumi}.}
% A(\Alph) a(\alph) I(\Roman) i(\roman) 1(\arabic)
%设定全局标号series=example	%引用全局变量resume=example
%[topsep=-0.3em,parsep=-0.3em,itemsep=-0.3em,partopsep=-0.3em]
%可使用leftmargin调整列表环境左边的空白长度 [leftmargin=0em]
\item
求磁场的磁感应强度的大小;
\item 
若粒子先后从两不同点以相同的速度射入磁场,恰好从$ OC $边上的同一点射出磁场,求该粒子这两次在磁场中运动的时间之和;
\item 
若粒子从某点射入磁场后,其运动轨迹与$ AC $边相切,且在磁场内运动的时间为$ \frac{ 5 }{ 3 } t_{0} $,求粒子此次入射速度的大小。



\end{enumerate}
\begin{figure}[h!]
\flushright
\includesvg[width=0.25\linewidth]{picture/svg/206}
\end{figure}

\banswer{
\begin{enumerate}
\renewcommand{\labelenumi}{\arabic{enumi}.}
% A(\Alph) a(\alph) I(\Roman) i(\roman) 1(\arabic)
%设定全局标号series=example	%引用全局变量resume=example
%[topsep=-0.3em,parsep=-0.3em,itemsep=-0.3em,partopsep=-0.3em]
%可使用leftmargin调整列表环境左边的空白长度 [leftmargin=0em]
\item
$B = \frac { \pi m } { 2 q t _ { 0 } }$
\item 
$ 2t_{0} $
\item 
$v _ { 0 } = \frac { \sqrt { 3 } \pi L } { 7 t _ { 0 } }$


\end{enumerate}
}






\newpage
\item
\exwhere{$ 2018 $年江苏卷}
如图所示,真空中四个相同的矩形匀强磁场区域,高为$ 4d $,宽为$ d $,中间两个磁场区域间隔为$ 2d $,中轴线与磁场区域两侧相交于$ O $、$ O ^{\prime} $点,各区域磁感应强度大小相等.某粒子质量为$ m $、电荷量为$ +q $,从$ O $沿轴线射入磁场.当入射速度为$ v_{0} $时,粒子从$ O $上方$ \frac{d}{2} $处射出磁场.取$ \sin 53 ^{ \circ } =0.8 $,$ \cos 53 ^{ \circ } =0.6 $.
\begin{enumerate}
\renewcommand{\labelenumi}{\arabic{enumi}.}
% A(\Alph) a(\alph) I(\Roman) i(\roman) 1(\arabic)
%设定全局标号series=example	%引用全局变量resume=example
%[topsep=-0.3em,parsep=-0.3em,itemsep=-0.3em,partopsep=-0.3em]
%可使用leftmargin调整列表环境左边的空白长度 [leftmargin=0em]
\item
求磁感应强度大小$ B $;
\item 
入射速度为$ 5v_0 $时,求粒子从$ O $运动到$ O ^{\prime} $的时间$ t $;
\item 
入射速度仍为$ 5v_0 $,通过沿轴线$ OO ^{\prime} $平移中间两个磁场(磁场不重叠),可使粒子从$ O $运动到$ O ^{\prime} $的时间增加$ \Delta $$ t $,求$ \Delta $$ t $的最大值。

\end{enumerate}
\begin{figure}[h!]
\flushright
\includesvg[width=0.4\linewidth]{picture/svg/208}
\end{figure}

\banswer{
\begin{enumerate}
\renewcommand{\labelenumi}{\arabic{enumi}.}
% A(\Alph) a(\alph) I(\Roman) i(\roman) 1(\arabic)
%设定全局标号series=example	%引用全局变量resume=example
%[topsep=-0.3em,parsep=-0.3em,itemsep=-0.3em,partopsep=-0.3em]
%可使用leftmargin调整列表环境左边的空白长度 [leftmargin=0em]
\item
$B = \frac { 4 m v _ { 0 } } { q d }$
\item 
在一个矩形磁场中的运动时间$t _ { 1 } = \frac { \alpha } { 360 ^ { \circ } } \frac { 2 \pi m } { q B }$,直线运动的时间$t _ { 2 } = \frac { 2 d } { v }$,$t = 4 t _ { 1 } + t _ { 2 } = \left( \frac { 53 \pi + 72 } { 180 } \right) \frac { d } { v _ { 0 } }$
\item 
$\Delta t _ { \mathrm { m } } = \frac { \Delta s _ { \mathrm { m } } } { v } = \frac { d } { 5 v _ { 0 } }$


\end{enumerate}
}


\newpage
\item
\exwhere{$ 2018 $年海南卷}
如图,圆心为$ O $、半径为$ r $的圆形区域外存在匀强磁场,磁场方向垂直于纸面向外,磁感应强度大小为$ B $。$ P $是圆外一点,$ OP=3r $。一质量为$ m $、电荷量为$ q $($ q>0 $)的粒子从$ P $点在纸面内垂直于$ OP $射出。已知粒子运动轨迹经过圆心$ O $,不计重力。求
\begin{enumerate}
\renewcommand{\labelenumi}{\arabic{enumi}.}
% A(\Alph) a(\alph) I(\Roman) i(\roman) 1(\arabic)
%设定全局标号series=example	%引用全局变量resume=example
%[topsep=-0.3em,parsep=-0.3em,itemsep=-0.3em,partopsep=-0.3em]
%可使用leftmargin调整列表环境左边的空白长度 [leftmargin=0em]
\item
粒子在磁场中做圆周运动的半径;
\item 
粒子第一次在圆形区域内运动所用的时间。



\end{enumerate}
\begin{figure}[h!]
\flushright
\includesvg[width=0.27\linewidth]{picture/svg/209}
\end{figure}


\banswer{
\begin{enumerate}
\renewcommand{\labelenumi}{\arabic{enumi}.}
% A(\Alph) a(\alph) I(\Roman) i(\roman) 1(\arabic)
%设定全局标号series=example	%引用全局变量resume=example
%[topsep=-0.3em,parsep=-0.3em,itemsep=-0.3em,partopsep=-0.3em]
%可使用leftmargin调整列表环境左边的空白长度 [leftmargin=0em]
\item
$R = \frac { 4 } { 3 } r$
\item 
$t = \frac { 3 m } { 2 q B }$



\end{enumerate}
}




\newpage
\item
\exwhere{$ 2018 $年全国\lmd{3}卷}
如图,从离子源产生的甲、乙两种离子,由静止经加速电压加速后在纸面内水平向右运动,自$ M $点垂直于磁场边界射入匀强磁场,磁场方向垂直于纸面向里,磁场左边界竖直。已知甲种离子射入磁场的速度大小为$ v_{1} $,并在磁场边界的$ N $点射出;乙种离子在$ MN $的中点射出;$ MN $长为$ l $。不计重力影响和离子间的相互作用。求
\begin{enumerate}
\renewcommand{\labelenumi}{\arabic{enumi}.}
% A(\Alph) a(\alph) I(\Roman) i(\roman) 1(\arabic)
%设定全局标号series=example	%引用全局变量resume=example
%[topsep=-0.3em,parsep=-0.3em,itemsep=-0.3em,partopsep=-0.3em]
%可使用leftmargin调整列表环境左边的空白长度 [leftmargin=0em]
\item
磁场的磁感应强度大小;
\item 
甲、乙两种离子的比荷之比。



\end{enumerate}
\begin{figure}[h!]
\flushright
\includesvg[width=0.33\linewidth]{picture/svg/210}
\end{figure}

\banswer{
\begin{enumerate}
\renewcommand{\labelenumi}{\arabic{enumi}.}
% A(\Alph) a(\alph) I(\Roman) i(\roman) 1(\arabic)
%设定全局标号series=example	%引用全局变量resume=example
%[topsep=-0.3em,parsep=-0.3em,itemsep=-0.3em,partopsep=-0.3em]
%可使用leftmargin调整列表环境左边的空白长度 [leftmargin=0em]
\item
$B = \frac { 4 U } { l v _ { 1 } }$
\item 
甲、乙两种离子的比荷之比为$\frac { q _ { 1 } } { m _ { 1 } }: \frac { q _ { 2 } } { m _ { 2 } } = 1: 4$



\end{enumerate}
}



\newpage
\item
\exwhere{$ 2013 $年天津卷}
一圆筒的横截面如图所示,其圆心为$ O $。筒内有垂直于纸面向里的匀强磁场,磁感应强度为$ B $。圆筒下面有相距为$ d $的平行金属板$ M $、$ N $,其中$ M $板带正电荷,$ N $板带等量负电荷。质量为$ m $、电荷量为$ q $的带正电粒子自$ M $板边缘的$ P $处由静止释放,经$ N $板的小孔$ S $以速度$ v $沿半径$ SO $方向射入磁场中。粒子与圈筒发生两次碰撞后仍从$ S $孔射出,设粒子与圆筒碰撞过程中没有动能损失,且电荷量保持不变,在不计重力的情况下,求:
\begin{enumerate}
\renewcommand{\labelenumi}{\arabic{enumi}.}
% A(\Alph) a(\alph) I(\Roman) i(\roman) 1(\arabic)
%设定全局标号series=example	%引用全局变量resume=example
%[topsep=-0.3em,parsep=-0.3em,itemsep=-0.3em,partopsep=-0.3em]
%可使用leftmargin调整列表环境左边的空白长度 [leftmargin=0em]
\item
$ M $、$ N $间电场强度$ E $的大小;
\item 
圆筒的半径$ R $;
\item 
保持$ M $、$ N $间电场强度$ E $不变,仅将$ M $板向上平移$ 2d/3 $,粒子仍从$ M $板边缘的$ P $处由静止释放,粒子自进入圆筒至从$ S $孔射出期间,与圆筒的碰撞次数$ n $。



\end{enumerate}
\begin{figure}[h!]
\flushright
\includesvg[width=0.25\linewidth]{picture/svg/211}
\end{figure}

\banswer{
\begin{enumerate}
\renewcommand{\labelenumi}{\arabic{enumi}.}
% A(\Alph) a(\alph) I(\Roman) i(\roman) 1(\arabic)
%设定全局标号series=example	%引用全局变量resume=example
%[topsep=-0.3em,parsep=-0.3em,itemsep=-0.3em,partopsep=-0.3em]
%可使用leftmargin调整列表环境左边的空白长度 [leftmargin=0em]
\item
$E = \frac { m v ^ { 2 } } { 2 q d }$
\item 
$r ^ { \prime } = \frac { \sqrt { 3 } m v } { 3 q B }$
\item 
$n = 3$



\end{enumerate}
}


\newpage
\item
\exwhere{$ 2012 $年物理海南卷}
图$ (a) $所示的$ xoy $平面处于匀强磁场中,磁场方向与$ xOy $平面(纸面)垂直,磁感应强度$ B $随时间$ t $变化的周期为$ T $,变化图线如图($ b $)所示。当$ B $为+$ B_{0} $时,磁感应强度方向指向纸外。在坐标原点$ O $有一带正电的粒子$ P $,其电荷量与质量之比恰好等于$\frac { 2 \pi } { T B _ { 0 } }$。不计重力。设$ P $在某时刻$ t_{0} $以某一初速度沿$ y $轴正向自$ O $点开始运动,将它经过时间$ T $到达的点记为$ A $。
\begin{enumerate}
\renewcommand{\labelenumi}{\arabic{enumi}.}
% A(\Alph) a(\alph) I(\Roman) i(\roman) 1(\arabic)
%设定全局标号series=example	%引用全局变量resume=example
%[topsep=-0.3em,parsep=-0.3em,itemsep=-0.3em,partopsep=-0.3em]
%可使用leftmargin调整列表环境左边的空白长度 [leftmargin=0em]
\item
若$ t_{0} = 0 $,则直线$ OA $与$ x $轴的夹角是多少?
\item 
若$ t_{0}=\frac{T}{4} $,则直线$ OA $与$ x $轴的夹角是多少?
\item 
为了使直线$ OA $与$ x $轴的夹角为$ \frac{\pi}{4} $,在$ 0 < t_{0}<\frac{T}{4} $的范围内,$ t_{0} $应取何值?



\end{enumerate}
\begin{figure}[h!]
\flushright
\includesvg[width=0.44\linewidth]{picture/svg/212}
\end{figure}

\banswer{
\begin{enumerate}
\renewcommand{\labelenumi}{\arabic{enumi}.}
% A(\Alph) a(\alph) I(\Roman) i(\roman) 1(\arabic)
%设定全局标号series=example	%引用全局变量resume=example
%[topsep=-0.3em,parsep=-0.3em,itemsep=-0.3em,partopsep=-0.3em]
%可使用leftmargin调整列表环境左边的空白长度 [leftmargin=0em]
\item
$ OA $与$ x $轴的夹角$ \theta=0 ^{\circ} $
\item 
$ OA $与$ x $轴的夹角$ \theta=\frac{\pi}{2} $
\item 
$t _ { 0 } = \frac { T } { 8 }$



\end{enumerate}
}



\newpage
\item
\exwhere{$ 2011 $年新课标版}
如图,在区域$ I $($ 0 \leq x \leq d $)和区域$ II $($ d<x \leq 2d $)内分别存在匀强磁场,磁感应强度大小分别为$ B $和$ 2B $,方向相反,且都垂直于$ Oxy $平面。一质量为$ m $、带电荷量$ q $($ q > 0 $)的粒子$ a $于某时刻从$ y $轴上的$ P $点射入区域$ I $,其速度方向沿$ x $轴正向。已知$ a $在离开区域$ I $时,速度方向与$ x $轴正方向的夹角为$ 30 ^{ \circ } $;此时,另一质量和电荷量均与$ a $相同的粒子$ b $也从$ P $点沿$ x $轴正向射入区域$ I $,其速度大小是$ a $的$ 1/3 $。不计重力和两粒子之间的相互作用力。求:
\begin{enumerate}
\renewcommand{\labelenumi}{\arabic{enumi}.}
% A(\Alph) a(\alph) I(\Roman) i(\roman) 1(\arabic)
%设定全局标号series=example	%引用全局变量resume=example
%[topsep=-0.3em,parsep=-0.3em,itemsep=-0.3em,partopsep=-0.3em]
%可使用leftmargin调整列表环境左边的空白长度 [leftmargin=0em]
\item
粒子$ a $射入区域$ I $时速度的大小;
\item 
当$ a $离开区域$ II $时,$ a $、$ b $两粒子的$ y $坐标之差。



\end{enumerate}
\begin{figure}[h!]
\flushright
\includesvg[width=0.25\linewidth]{picture/svg/213}
\end{figure}

\banswer{
\begin{enumerate}
\renewcommand{\labelenumi}{\arabic{enumi}.}
% A(\Alph) a(\alph) I(\Roman) i(\roman) 1(\arabic)
%设定全局标号series=example	%引用全局变量resume=example
%[topsep=-0.3em,parsep=-0.3em,itemsep=-0.3em,partopsep=-0.3em]
%可使用leftmargin调整列表环境左边的空白长度 [leftmargin=0em]
\item
$v _ { a 1 } = \frac { 2 q B d } { m }$
\item 
$y _ { p _ { a } } - y _ { p _ { b } } = \frac { 2 } { 3 } ( \sqrt { 3 } - 2 ) d$


\end{enumerate}
}




\newpage
\item
\exwhere{$ 2013 $年海南卷}
如图,纸面内有$ E $、$ F $、$ G $三点,$ \angle GEF=30 ^{\circ} $,$ \angle EFG=135 ^{\circ} $,空间有一匀强磁场,磁感应强度大小为$ B $,方向垂直于纸面向外。先使带有电荷量为$ q(q>0) $的点电荷$ a $在纸面内垂直于$ EF $从$ F $点射出,其轨迹经过$ G $点;再使带有同样电荷量的点电荷$ b $在纸面内与$ EF $成一定角度从$ E $点射出,其轨迹也经过$ G $点。两点电荷从射出到经过$ G $点所用的时间相同,且经过$ G $点时的速度方向也相同。已知点电荷$ a $的质量为$ m $,轨道半径为$ R $,不计重力,求:
\begin{enumerate}
\renewcommand{\labelenumi}{\arabic{enumi}.}
% A(\Alph) a(\alph) I(\Roman) i(\roman) 1(\arabic)
%设定全局标号series=example	%引用全局变量resume=example
%[topsep=-0.3em,parsep=-0.3em,itemsep=-0.3em,partopsep=-0.3em]
%可使用leftmargin调整列表环境左边的空白长度 [leftmargin=0em]
\item
点电荷$ a $从射出到经过$ G $点所用的时间;
\item 
点电荷$ b $的速度大小。



\end{enumerate}
\begin{figure}[h!]
\flushright
\includesvg[width=0.25\linewidth]{picture/svg/214}
\end{figure}

\banswer{
\begin{enumerate}
\renewcommand{\labelenumi}{\arabic{enumi}.}
% A(\Alph) a(\alph) I(\Roman) i(\roman) 1(\arabic)
%设定全局标号series=example	%引用全局变量resume=example
%[topsep=-0.3em,parsep=-0.3em,itemsep=-0.3em,partopsep=-0.3em]
%可使用leftmargin调整列表环境左边的空白长度 [leftmargin=0em]
\item
$t=\frac { \pi m } { 2 B q }$
\item 
$v_{b}=\frac { 4 q B R } { 3 m }$


\end{enumerate}
}



\newpage
\item
\exwhere{$ 2019 $年$ 4 $月浙江物理选考加试题}
有一种质谱仪由静电分析器和磁分析器组成,其简化原理如图所示。左侧静电分析器中有方向指向圆心$ O $、与$ O $点等距离各点的场强大小相同的径向电场,右侧的磁分析器中分布着方向垂直于纸面向外的匀强磁场,其左边界与静电分析器的右边界平行,两者间距近似为零。离子源发出两种速度均为$ v_{0} $、电荷量均为$ q $、质量分别为$ m $和$ 0.5m $的正离子束,从$ M $点垂直该点电场方向进入静电分析器。在静电分析器中,质量为$ m $的离子沿半径为$ r_{0} $的四分之一圆弧轨道做匀速圆周运动,从$ N $点水平射出,而质量为$ 0.5m $的离子恰好从$ ON $连线的中点$ P $与水平方向成$ \theta $角射出,从静电分析器射出的这两束离子垂直磁场方向射入磁分析器中,最后打在放置于磁分析器左边界的探测板上,其中质量为$ m $的离子打在$ O $点正下方的$ Q $点。已知$ OP=0.5r_0 $,$ OQ=r_0 $,$ N $、$ P $两点间的电势差$U _ { N P } = \frac { m v ^ { 2 } } { q }$,$\cos \theta = \sqrt { \frac { 4 } { 5 } }$,不计重力和离子间相互作用。
\begin{enumerate}
\renewcommand{\labelenumi}{\arabic{enumi}.}
% A(\Alph) a(\alph) I(\Roman) i(\roman) 1(\arabic)
%设定全局标号series=example	%引用全局变量resume=example
%[topsep=-0.3em,parsep=-0.3em,itemsep=-0.3em,partopsep=-0.3em]
%可使用leftmargin调整列表环境左边的空白长度 [leftmargin=0em]
\item
求静电分析器中半径为$ r_{0} $处的电场强度$ E_{0} $和磁分析器中的磁感应强度$ B $的大小;
\item 
求质量为$ 0.5m $的离子到达探测板上的位置与$ O $点的距离$ l $(用$ r_{0} $表示);
\item 
若磁感应强度在($ B - \triangle B $)到($ B + \triangle B $)之间波动,要在探测板上完全分辨出质量为$ m $和$ 0.5m $的两束离子,求$\frac { \Delta B } { B }$的最大值。



\end{enumerate}
\begin{figure}[h!]
\flushright
\includesvg[width=0.25\linewidth]{picture/svg/215}
\end{figure}

\banswer{
\begin{enumerate}
\renewcommand{\labelenumi}{\arabic{enumi}.}
% A(\Alph) a(\alph) I(\Roman) i(\roman) 1(\arabic)
%设定全局标号series=example	%引用全局变量resume=example
%[topsep=-0.3em,parsep=-0.3em,itemsep=-0.3em,partopsep=-0.3em]
%可使用leftmargin调整列表环境左边的空白长度 [leftmargin=0em]
\item
$E _ { 0 } = \frac { m v _ { 0 } ^ { 2 } } { q r _ { 0 } } , \quad B = \frac { m v _ { 0 } } { q r _ { 0 } }$
\item 
$1.5 r _ { 0 }$
\item 
$12 \%$



\end{enumerate}
}





\newpage
\item
\exwhere{$ 2019 $年物理江苏卷}
如图所示,匀强磁场的磁感应强度大小为$ B $.磁场中的水平绝缘薄板与磁场的左、右边界分别垂直相交于$ M $、$ N $,$ MN=L $,粒子打到板上时会被反弹(碰撞时间极短),反弹前后水平分速度不变,竖直分速度大小不变、方向相反.质量为$ m $、电荷量为$ -q $的粒子速度一定,可以从左边界的不同位置水平射入磁场,在磁场中做圆周运动的半径为$ d $,且$ d<L $,粒子重力不计,电荷量保持不变。
\begin{enumerate}
\renewcommand{\labelenumi}{\arabic{enumi}.}
% A(\Alph) a(\alph) I(\Roman) i(\roman) 1(\arabic)
%设定全局标号series=example	%引用全局变量resume=example
%[topsep=-0.3em,parsep=-0.3em,itemsep=-0.3em,partopsep=-0.3em]
%可使用leftmargin调整列表环境左边的空白长度 [leftmargin=0em]
\item
求粒子运动速度的大小$ v $;
\item 
欲使粒子从磁场右边界射出,求入射点到$ M $的最大距离$ d_m $;
\item 
从$ P $点射入的粒子最终从$ Q $点射出磁场,$ PM=d $,$ QN=\frac{d}{2} $,求粒子从$ P $到$ Q $的运动时间$ t $.



\end{enumerate}
\begin{figure}[h!]
\flushright
\includesvg[width=0.48\linewidth]{picture/svg/219}
\end{figure}

\banswer{
\begin{enumerate}
\renewcommand{\labelenumi}{\arabic{enumi}.}
% A(\Alph) a(\alph) I(\Roman) i(\roman) 1(\arabic)
%设定全局标号series=example	%引用全局变量resume=example
%[topsep=-0.3em,parsep=-0.3em,itemsep=-0.3em,partopsep=-0.3em]
%可使用leftmargin调整列表环境左边的空白长度 [leftmargin=0em]
\item
$v = \frac { q B d } { m }$
\item 
$d _ { \mathrm { m } } = \frac { 2 + \sqrt { 3 } } { 2 } d$
\item 
A.当$L = n d + \left( 1 - \frac { \sqrt { 3 } } { 2 } \right) d$时,$t = \left( \frac { L } { d } + \frac { 3 \sqrt { 3 } - 4 } { 6 } \right) \frac { \pi m } { 2 q B }$ ,B.当$L = n d + \left( 1 + \frac { \sqrt { 3 } } { 2 } \right) d$时, $t = \left( \frac { L } { d } - \frac { 3 \sqrt { 3 } - 4 } { 6 } \right) \frac { \pi m } { 2 q B }$.



\end{enumerate}
}



\end{enumerate}




