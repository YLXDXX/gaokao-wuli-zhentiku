\bta{电表内阻的测定}

\begin{enumerate}[leftmargin=0em]
\renewcommand{\labelenumi}{\arabic{enumi}.}
% A(\Alph) a(\alph) I(\Roman) i(\roman) 1(\arabic)
%设定全局标号series=example	%引用全局变量resume=example
%[topsep=-0.3em,parsep=-0.3em,itemsep=-0.3em,partopsep=-0.3em]
%可使用leftmargin调整列表环境左边的空白长度 [leftmargin=0em]
\item
\exwhere{$ 2016 $年理综新课标 \lmd{2} 卷}
某同学利用图($ a $)所示电路测量量程为$ 2.5 $ $ V $的电压表 \voltmetermytikz 的内阻(内阻为数千欧姆),可供选择的器材有:电阻箱$ R $(最大阻值$ 99 $ $ 999.9 $ $ \Omega $),滑动变阻器$ R_{1} $(最大阻值$ 50 $ $ \Omega $),滑动变阻器$ R_{2} $(最大阻值$ 5 $ $ k \Omega $),直流电源$ E $(电动势$ 3 $ $ V $)。开关$ 1 $个,导线若干。
\begin{figure}[h!]
\centering
\includesvg[width=0.23\linewidth]{picture/svg/643} \qquad \qquad 
\includesvg[width=0.23\linewidth]{picture/svg/644}
\end{figure}

实验步骤如下:

①按电路原理图($ a $)连接线路;

②将电阻箱阻值调节为$ 0 $,将滑动变阻器的滑片移到与图($ a $)中最左端所对应的位置,闭合开关$ S $;

③调节滑动变阻器,使电压表满偏;

④保持滑动变阻器滑片的位置不变,调节电阻箱阻值,使电压表的示数为$ 2.00 $ $ V $,记下电阻箱的阻值。

回答下列问题:

\begin{enumerate}
\renewcommand{\labelenumi}{\arabic{enumi}.}
% A(\Alph) a(\alph) I(\Roman) i(\roman) 1(\arabic)
%设定全局标号series=example	%引用全局变量resume=example
%[topsep=-0.3em,parsep=-0.3em,itemsep=-0.3em,partopsep=-0.3em]
%可使用leftmargin调整列表环境左边的空白长度 [leftmargin=0em]
\item
实验中应选择滑动变阻器\tk{$ R_{1} $}(填“$ R_{1} $”或“$ R_{2} $”)。
\item 
根据图($ a $)所示电路将图($ b $)中实物图连线。
\item 
实验步骤④中记录的电阻箱阻值为$ 630.0 $ $ \Omega $,若认为调节电阻箱时滑动变阻器上的分压不变,计算可得电压表的内阻为\tk{2520}$ \Omega $(结果保留到个位)。
如果此电压表是由一个表头和电阻串联构成的,可推断该表头的满刻度电流为 \xzanswer{D} 
(填正确答案标号)。
\fourchoices
{$ 100 $ $ \mu A $}
{$ 250 $ $ \mu A $}
{$ 500 $ $ \mu A $}
{$ 1 $ $ mA $}

\end{enumerate}

\banswer{
($ 2 $)连线如下图
\begin{figure}[h!]
\centering
\includesvg[width=0.23\linewidth]{picture/svg/645}
\end{figure}

}

\newpage
\item 
\exwhere{$ 2015 $年理综新课标$ \lmd{2} $卷}
电压表满偏时通过该表的电流是半偏时通过该表的电流的两倍。某同学利用这一事实测量电压表的内阻(半偏法)实验室提供的材料器材如下:

待测电压表$ V $(量程$ 3V $,内阻约为$ 3000 \ \Omega $),电阻箱$ R_{0} $(最大阻值为$ 99999.9 \ \Omega ) $,滑动变阻器$ R_{1} $(最大阻值$ 100 \ \Omega $,额定电流$ 2A $),电源$ E $(电动势$ 6V $,内阻不计),开关两个,导线若干。
\begin{figure}[h!]
\centering
\includesvg[width=0.23\linewidth]{picture/svg/646}
\end{figure}

\begin{enumerate}
\renewcommand{\labelenumi}{\arabic{enumi}.}
% A(\Alph) a(\alph) I(\Roman) i(\roman) 1(\arabic)
%设定全局标号series=example	%引用全局变量resume=example
%[topsep=-0.3em,parsep=-0.3em,itemsep=-0.3em,partopsep=-0.3em]
%可使用leftmargin调整列表环境左边的空白长度 [leftmargin=0em]
\item
虚线框内为同学设计的测量电压表内阻的电路图的一部分,将电路图补充完整。
\item 
根据设计的电路,写出实验步骤
。

 \hfullline 
 
 \item 
 将这种方法测出的电压表内阻记为$ R ^{\prime} _ V $,与电压表内阻的真实值$ R_v $相比,$ R ^{\prime} _ V $ \tk{$ > $} $ R_v $ (填“$ > $”“$ = $”或“$ < $”)
 主要理由是 \tk{断开$ S_{2} $,调节电阻箱使电压表成半偏状态,电压表所在支路总电阻增大,分得的电压也增大;此时$ R_{0} $两端的电压大于电压表的半偏电压,故$ R ^{\prime} _ V>R_v $} 。


\end{enumerate}


\banswer{
(1)实验电路如右图所示:
\includesvg[width=0.23\linewidth]{picture/svg/647}


}





\newpage	
\item 
\exwhere{$ 2011 $年上海卷}
实际电流表有内阻,可等效为理想电流表与电阻的串联。测量实际电流表$ G_{1} $内阻$ r_{1} $的电路如图所示。
\begin{figure}[h!]
\centering
\includesvg[width=0.63\linewidth]{picture/svg/648}
\end{figure}

供选择的仪器如下:

①待测电流表$ G_{1} $ $ (0 \sim 5 \ mA $,内阻约$ 300 \ \Omega ) $,

②电流表$ G_{2} $ $ (0 \sim 10 mA $,内阻约$ 100 \ \Omega ) $,

③定值电阻$ R_{1} $ $ (300 \ \Omega ) $, ④定值电阻$ R_{2} $ $ (10 \ \Omega ) $,⑤滑动变阻器$ R_{3} $ $ (0 \sim 1000 \ \Omega ) $, 
 
⑥滑动变阻器$ R_{4} $ $ (0 \sim 20 \ \Omega ) $,

⑦干电池($ 1.5V $),⑧电键$ S $及导线若干。

\begin{enumerate}
\renewcommand{\labelenumi}{\arabic{enumi}.}
% A(\Alph) a(\alph) I(\Roman) i(\roman) 1(\arabic)
%设定全局标号series=example	%引用全局变量resume=example
%[topsep=-0.3em,parsep=-0.3em,itemsep=-0.3em,partopsep=-0.3em]
%可使用leftmargin调整列表环境左边的空白长度 [leftmargin=0em]
\item
定值电阻应选\tk{③},滑动变阻器应选\tk{⑥}。(在空格内填写序号)
\item 
用连线连接实物图。
\item 
补全实验步骤:

①按电路图连接电路,\tk{将滑动触头移至最左端};

②闭合电键$ S $,移动滑动触头至某一位置,记录$ G_{1} $,$ G_{2} $的读数$ I_{1} $,$ I_{2} $;

③;\tk{多次移动滑动触头,记录相应的$ G_{1} $、$ G_{2} $读数$ I_{1} $、$ I_{2} $} ;

④以$ I_{2} $为纵坐标,$ I_{1} $为横坐标,作出相应图线,如图所示。
\begin{figure}[h!]
\centering
\includesvg[width=0.23\linewidth]{picture/svg/649}
\end{figure}

\item 
根据$ I_2-I_1 $图线的斜率$ k $及定值电阻,写出待测电流表内阻的表达式\tk{$ r_{1} =(k $-$ 1) R_{1} $}。

\end{enumerate}

\banswer{
($ 2 $)如图
\includesvg[width=0.23\linewidth]{picture/svg/650}

}








\end{enumerate}

