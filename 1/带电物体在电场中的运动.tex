\bta{第九讲$ \quad $带电物体在电场中的运动}

\begin{enumerate}[leftmargin=0em]
\renewcommand{\labelenumi}{\arabic{enumi}.}
% A(\Alph) a(\alph) I(\Roman) i(\roman) 1(\arabic)
%设定全局标号series=example	%引用全局变量resume=example
%[topsep=-0.3em,parsep=-0.3em,itemsep=-0.3em,partopsep=-0.3em]
%可使用leftmargin调整列表环境左边的空白长度 [leftmargin=0em]
\item
\exwhere{$ 2012 $年理综新课标卷}
如图,平行板电容器的两个极板与水平地面成一角度,两极板与一直流电源相连。若一带电粒子恰能沿图中所示水平直线通过电容器,则在此过程中,该粒子 \xzanswer{BD} 

\begin{minipage}[h!]{0.6\linewidth}
\vspace{0.3em}
\fourchoices
{所受重力与电场力平衡}
{电势能逐渐增加}
{动能逐渐增加}
{做匀变速直线运动}
\vspace{0.3em}
\end{minipage}
\hfill
\begin{minipage}[h!]{0.3\linewidth}
\flushright
\vspace{0.3em}
\includesvg[width=0.7\linewidth]{picture/svg/104}
\vspace{0.3em}
\end{minipage}




\item
\exwhere{$ 2013 $年新课标\lmd{1}卷}
一水平放置的平行板电容器的两极板间距为$ d $,极板分别与电池两极相连,上极板中心有一小孔(小孔对电场的影响可忽略不计)。小孔正上方$ \frac{d}{2} $处的$ P $点有一带电粒子,该粒子从静止开始下落,经过小孔进入电容器,并在下极板处(未与极板接触)返回。若将下极板向上平移$ \frac{d}{3} $,则从$ P $点开始下落的相同粒子 \xzanswer{D} 



\fourchoices
{打到下极板上}
{在下极板处返回}
{在距上极板$ \frac{d}{2} $处返回}
{在距上极板$ \frac{2}{5}d $处返回}




\item
\exwhere{$ 2011 $年理综四川卷}
质量为$ m $的带正电小球由空中$ A $点无初速度自由下落,在$ t $秒末加上竖直向上、范围足够大的匀强电场,再经过$ t $秒小球又回到$ A $点,不计空气阻力且小球从未落地。则 \xzanswer{BD} 


\fourchoices
{整个过程中小球电势能变化了$\frac { 3 } { 2 } m g ^ { 2 } t ^ { 2 }$}
{整个过程中小球动量增量的大小为$ 2mgt $}
{从加电场开始到小球运动到最低点时小球动能变化了$ mg^{2}t^{2} $}
{从$ A $点到最低点小球重力势能变化了$\frac { 2 } { 3 } m g ^ { 2 } t ^ { 2 }$}




\item
\exwhere{$ 2014 $年理综天津卷}
如图所示,平行金属板$ A $、$ B $水平正对放置,分别带等量异号电荷.一带电微粒水平射入板间,在重力和电场力共作用下运动,轨迹如图中虚线所示,那么 \xzanswer{C} 


\begin{minipage}[h!]{0.7\linewidth}
\vspace{0.3em}
\fourchoices
{若微粒带正电荷,则$ A $板一定带正电荷}
{微粒从$ M $点运动到$ N $点电势能一定增加}
{微粒从$ M $点运动到$ N $点动能一定增加}
{微粒从$ M $点运动到$ N $点机械能一定增加}
\vspace{0.3em}
\end{minipage}
\hfill
\begin{minipage}[h!]{0.3\linewidth}
\flushright
\vspace{0.3em}
\includesvg[width=0.7\linewidth]{picture/svg/105}
\vspace{0.3em}
\end{minipage}





\item
\exwhere{$ 2015 $年江苏卷}
一带正电的小球向右水平抛入范围足够大的匀强电场,电场方向水平向左。 不计空气阻力,则小球 \xzanswer{BC} 


\begin{minipage}[h!]{0.7\linewidth}
\vspace{0.3em}
\fourchoices
{做直线运动}
{做曲线运动}
{速率先减小后增大}
{速率先增大后减小}
\vspace{0.3em}
\end{minipage}
\hfill
\begin{minipage}[h!]{0.3\linewidth}
\flushright
\vspace{0.3em}
\includesvg[width=0.7\linewidth]{picture/svg/106}
\vspace{0.3em}
\end{minipage}




\item
\exwhere{$ 2015 $年理综山东卷}
如图甲,两水平金属板间距为$ d $,板间电场强度的变化规律如图乙所示。$ t=0 $时刻,质量为$ m $的带电微粒以初速度为$ v_{0} $沿中线射入两板间,$ 0 \sim \frac{T}{3} $时间内微粒匀速运动,$ T $时刻微粒恰好经金属板边缘飞出,微粒运动过程中未与金属板接触,重力加速度的大小为$ g $,关于微粒在$ 0 \sim T $时间内运动的描述,正确的是 \xzanswer{BC} 
\begin{figure}[h!]
\centering
\includesvg[width=0.37\linewidth]{picture/svg/107}
\end{figure}



\fourchoices
{末速度大小为 $ \sqrt{2}v_{0} $}
{末速度沿水平方向}
{重力势能减少了$ \frac{ 1 }{ 2 } mgd $}
{克服电场力做功为$ mgd $ }





\item
\exwhere{$ 2017 $年浙江选考卷}
如图所示,在竖立放置间距为$ d $的平行板电容器中,存在电场强度为$ E $的匀强电场。有一质量为$ m $,电荷量为$ +q $的点电荷从两极板正中间处静止释放,重力加速度为$ g $。则点电荷运动到负极板的过程 \xzanswer{B} 
\begin{figure}[h!]
\centering
\includesvg[width=0.23\linewidth]{picture/svg/108}
\end{figure}



\fourchoices
{ 加速度大小为$a = \frac { q E } { m } + g$}
{所需的时间为$t = \sqrt { \frac { d m } { E q } }$}
{下降的高度为$y = \frac { d } { 2 }$}
{电场力所做的功为$ W=Eqd $}





\item
\exwhere{$ 2019 $年$ 4 $月浙江物理选考}
用长为$ 1.4 \ m $的轻质柔软绝缘细线,拴一质量为$ 1.0 \times 10^{-2} \ kg $、电荷量为$ 2.0 \times 10^{-8} \ C $的小球,细线的上端固定于$ O $点。现加一水平向右的匀强电场,平衡时细线与铅垂线成$ 37 ^{\circ} $,如图所示。现向左拉小球使细线水平且拉直,静止释放,则($ \sin 37 ^{\circ} =0.6 $) \xzanswer{C} 
\begin{figure}[h!]
\centering
\includesvg[width=0.23\linewidth]{picture/svg/109}
\end{figure}



\fourchoices
{该匀强电场的场强为$ 3.75 \times 10^7 \ N/C $}
{平衡时细线的拉力为$ 0.17 \ N $}
{ 经过$ 0.5 \ s $,小球的速度大小为$ 6.25 \ m/s $}
{小球第一次通过$ O $点正下方时,速度大小为$ 7 \ m/s $}




\item
\exwhere{$ 2019 $年物理天津卷}
如图所示,在水平向右的匀强电场中,质量为$ m $的带电小球,以初速度$ v $从$ M $点竖直向上运动,通过$ N $点时,速度大小为$ 2v $,方向与电场方向相反,则小球从$ M $运动到$ N $的过程 \xzanswer{B} 
\begin{figure}[h!]
\centering
\includesvg[width=0.23\linewidth]{picture/svg/110}
\end{figure}




\fourchoices
{动能增加$ \frac{ 1 }{ 2 } mv^{2} $}
{机械能增加$ 2mv^{2} $}
{重力势能增加$ \frac{ 3 }{ 2 } mv^{2} $}
{电势能增加$ 2mv^{2} $}




\item
\exwhere{$ 2019 $年物理全国\lmd{3}卷}
空间存在一方向竖直向下的匀强电场,$ O $、$ P $是电场中的两点。从$ O $点沿水平方向以不同速度先后发射两个质量均为$ m $的小球$ A $、$ B $。$ A $不带电,$ B $的电荷量为$ q $($ q>0 $)。$ A $从$ O $点发射时的速度大小为$ v_{0} $,到达$ P $点所用时间为$ t $;$ B $从$ O $点到达$ P $点所用时间为$ \frac{t}{2} $。重力加速度为$ g $,求:
\begin{enumerate}
\renewcommand{\labelenumi}{\arabic{enumi}.}
% A(\Alph) a(\alph) I(\Roman) i(\roman) 1(\arabic)
%设定全局标号series=example	%引用全局变量resume=example
%[topsep=-0.3em,parsep=-0.3em,itemsep=-0.3em,partopsep=-0.3em]
%可使用leftmargin调整列表环境左边的空白长度 [leftmargin=0em]
\item
电场强度的大小;
\item 
$ B $运动到$ P $点时的动能。



\end{enumerate}


\banswer{
\begin{enumerate}
\renewcommand{\labelenumi}{\arabic{enumi}.}
% A(\Alph) a(\alph) I(\Roman) i(\roman) 1(\arabic)
%设定全局标号series=example	%引用全局变量resume=example
%[topsep=-0.3em,parsep=-0.3em,itemsep=-0.3em,partopsep=-0.3em]
%可使用leftmargin调整列表环境左边的空白长度 [leftmargin=0em]
\item
$E = \frac { 3 m g } { q }$
\item 
$E _ { \mathrm { k } } = 2 m \left( v _ { 0 } ^ { 2 } + g ^ { 2 } t ^ { 2 } \right)$



\end{enumerate}
}


\item
\exwhere{$ 2013 $年四川卷}
在如图所示的竖直平面内,物体$ A $和带正电的物体$ B $用跨过定滑轮的绝缘轻绳连接,分别静止于倾角$ \theta =37 ^{\circ} $的光滑斜面上的$ M $点和粗糙绝缘水平面上,轻绳与对应平面平行。劲度系数$ k=5 \ N/m $的轻弹簧一端固定在$ O $点,一端用另一轻绳穿过固定的光滑小环$ D $与$ A $相连,弹簧处于原长,轻绳恰好拉直,$ DM $垂直于斜面。水平面处于场强$ E=5 \times 10^4 \ N/C $、方向水平向右的匀强电场中。已知$ A $、$ B $的质量分别为$ m_A=0.1 \ kg $和$ m_B=0.2 \ kg $,$ B $所带电荷量$ q=+4 \times 10^{-6}\ C $。设两物体均视为质点,不计滑轮质量和摩擦,绳不可伸长,弹簧始终在弹性限度内,$ B $电量不变。取$ g=10 \ m/s ^{2} $,$ \sin 37 ^{\circ} =0.6 $,$ \cos 37 ^{\circ} =0.8 $。
\begin{enumerate}
\renewcommand{\labelenumi}{\arabic{enumi}.}
% A(\Alph) a(\alph) I(\Roman) i(\roman) 1(\arabic)
%设定全局标号series=example	%引用全局变量resume=example
%[topsep=-0.3em,parsep=-0.3em,itemsep=-0.3em,partopsep=-0.3em]
%可使用leftmargin调整列表环境左边的空白长度 [leftmargin=0em]
\item
求$ B $所受静摩擦力的大小;
\item 
现对$ A $施加沿斜面向下的拉力$ F $,使$ A $以加速度$ a=0.6 \ m/s ^{2} $开始做匀加速直线运动。$ A $从$ M $到$ N $的过程中,$ B $的电势能增加了$ \Delta E_p=0.06\ J $。已知$ DN $沿竖直方向,$ B $与水平面间的动摩擦因数$ \mu =0.4 $。求$ A $到达$ N $点时拉力$ F $的瞬时功率。



\end{enumerate}
\begin{figure}[h!]
\flushright
\includesvg[width=0.38\linewidth]{picture/svg/111}
\end{figure}


\banswer{
\begin{enumerate}
\renewcommand{\labelenumi}{\arabic{enumi}.}
% A(\Alph) a(\alph) I(\Roman) i(\roman) 1(\arabic)
%设定全局标号series=example	%引用全局变量resume=example
%[topsep=-0.3em,parsep=-0.3em,itemsep=-0.3em,partopsep=-0.3em]
%可使用leftmargin调整列表环境左边的空白长度 [leftmargin=0em]
\item
$ f_{0}=-0.4\ N $
\item 
$ P=0.528\ W $


\end{enumerate}
}


\newpage
\item
\exwhere{$ 2014 $年理综新课标\lmd{1}卷}
如图所示,$ O $,$ A $,$ B $为同一竖直平面内的三个点,$ OB $沿竖直方向,$ \angle BOA = 60 ^{ \circ } $,$ OB = \frac{ 3 }{ 2 } OA $,将一质量为$ m $的小球以一定的初动能自$ O $点水平向右抛出,小球在运动过程中恰好通过$ A $点,使此小球带电,电荷量为$ q(q>0) $,同时加一匀强电场,场强方向与$ \triangle OAB $所在平面平行.现从$ O $点以同样的初动能沿某一方向抛出此带电小球,该小球通过了$ A $点,到达$ A $点时的动能是初动能的$ 3 $倍;若该小球从$ O $点以同样的初动能沿另一方向抛出,恰好通过$ B $点,且到达$ B $点时的动能为初动能的$ 6 $倍,重力加速度大小为$ g $.求:
\begin{enumerate}
\renewcommand{\labelenumi}{\arabic{enumi}.}
% A(\Alph) a(\alph) I(\Roman) i(\roman) 1(\arabic)
%设定全局标号series=example	%引用全局变量resume=example
%[topsep=-0.3em,parsep=-0.3em,itemsep=-0.3em,partopsep=-0.3em]
%可使用leftmargin调整列表环境左边的空白长度 [leftmargin=0em]
\item
无电场时,小球到达$ A $点时的动能与初动能的比值;
\item 
电场强度的大小和方向.



\end{enumerate}
\begin{figure}[h!]
\flushright
\includesvg[width=0.17\linewidth]{picture/svg/112}
\end{figure}

\banswer{
\begin{enumerate}
\renewcommand{\labelenumi}{\arabic{enumi}.}
% A(\Alph) a(\alph) I(\Roman) i(\roman) 1(\arabic)
%设定全局标号series=example	%引用全局变量resume=example
%[topsep=-0.3em,parsep=-0.3em,itemsep=-0.3em,partopsep=-0.3em]
%可使用leftmargin调整列表环境左边的空白长度 [leftmargin=0em]
\item
$\frac { E _ { k A } } { E _ { k 0 } } = \frac { 7 } { 3 }$
\item 
$E = \frac { \sqrt { 3 } m g } { 6 q }$


\end{enumerate}
}



\newpage 
\item
\exwhere{$ 2015 $年理综四川卷}
如图所示,粗糙、绝缘的直轨道$ OB $固定在水平桌面上,$ B $端与桌面边缘对齐,$ A $是轨道上一点,过$ A $点并垂直于轨道的竖直面右侧有大小$ E $=$ 1.5 \times 10^6 \ N/C $,方向水平向右的匀强电场。带负电的小物体$ P $电荷量是$ 2.0 \times 10^{-6}\ C $,质量$ m = 0.25 \ kg $,与轨道间动摩擦因数$ \mu =0.4 $,$ P $从$ O $点由静止开始向右运动,经过$ 0.55 \ s $到达$ A $点,到达$ B $点时速度是$ 5 \ m/s $,到达空间$ D $点时速度与竖直方向的夹角为$ \alpha $,且$ \tan \alpha = 1.2 $。$ P $在整个运动过程中始终受到水平向右的某外力$ F $作用,$ F $大小与$ P $的速率$ v $的关系如表所示。
\begin{table}[h!]
\centering 
\begin{tabular}{|c|c|c|c|}
\hline 
v($ ms^{-1} $) &$ 0\leq v \leq 2 $&$ 2<v<5 $ &$ v\geq 5 $ \\
\hline
F/N & 2 & 6 & 3\\ 
\hline 
\end{tabular}
\end{table} 

$ P $视为质点,电荷量保持不变,忽略空气阻力,取$ g $=$ 10 \ m/s ^{2} $,求:
\begin{enumerate}
\renewcommand{\labelenumi}{\arabic{enumi}.}
% A(\Alph) a(\alph) I(\Roman) i(\roman) 1(\arabic)
%设定全局标号series=example	%引用全局变量resume=example
%[topsep=-0.3em,parsep=-0.3em,itemsep=-0.3em,partopsep=-0.3em]
%可使用leftmargin调整列表环境左边的空白长度 [leftmargin=0em]
\item
小物体$ P $从开始运动至速率为$ 2 \ m/s $所用的时间;
\item 
小物体$ P $从$ A $运动至$ D $的过程,电场力做的功。


\end{enumerate}
\begin{figure}[h!]
\flushright
\includesvg[width=0.4\linewidth]{picture/svg/113}
\end{figure}

\banswer{
\begin{enumerate}
\renewcommand{\labelenumi}{\arabic{enumi}.}
% A(\Alph) a(\alph) I(\Roman) i(\roman) 1(\arabic)
%设定全局标号series=example	%引用全局变量resume=example
%[topsep=-0.3em,parsep=-0.3em,itemsep=-0.3em,partopsep=-0.3em]
%可使用leftmargin调整列表环境左边的空白长度 [leftmargin=0em]
\item
$ t_{1}=0.5\ m/s $
\item 
$ W=-9.25\ J $

\end{enumerate}
}


\newpage 
\item
\exwhere{$ 2016 $年上海卷}
如图($ a $),长度$ L=0.8\ m $的光滑杆左端固定一带正电的点电荷$ A $,其电荷量$ Q=1.8\times 10^{-7}\ C $;一质量$ m=0.02 \ kg $,带电量为$ q $的小球$ B $套在杆上。将杆沿水平方向固定于某非均匀外电场中,以杆左端为原点,沿杆向右为$ x $轴正方向建立坐标系。点电荷$ A $对小球$ B $的作用力随$ B $位置$ x $的变化关系如图($ b $)中曲线\lmd{1}所示,小球$ B $所受水平方向的合力随$ B $位置$ x $的变化关系如图($ b $)中曲线\lmd{2}所示,其中曲线\lmd{2}在
$ 0.16 \leq x \leq 0.20 $和$ x \geq 0.40 $范围可近似看作直线。求:(静电力常量$ k=9\times10^{9}\ N\cdot m^{2}/C^{2} $)
\begin{enumerate}
\renewcommand{\labelenumi}{\arabic{enumi}.}
% A(\Alph) a(\alph) I(\Roman) i(\roman) 1(\arabic)
%设定全局标号series=example	%引用全局变量resume=example
%[topsep=-0.3em,parsep=-0.3em,itemsep=-0.3em,partopsep=-0.3em]
%可使用leftmargin调整列表环境左边的空白长度 [leftmargin=0em]
\item
小球$ B $所带电量$ q $;
\item 
非均匀外电场在$ x=0.3\ m $处沿细杆方向的电场强度大小$ E $;
\item 
在合电场中,$ x=0.4\ m $与$ x=0.6\ m $之间的电势差$ U $;
\item 
已知小球在$ x=0.2\ m $处获得$ v=0.4 \ m/s $的初速度时,最远可以运动到$ x=0.4\ m $。若小球在$ x=0.16\ m $处受到方向向右,大小为$ 0.04 \ N $的恒力作用后,由静止开始运动,为使小球能离开细杆,恒力作用的最小距离$ s $是多少?
\end{enumerate}

\begin{figure}[h!]
\flushright
\includesvg[width=0.25\linewidth]{picture/svg/114} \qquad 
\includesvg[width=0.55\linewidth]{picture/svg/115}
\end{figure}
\banswer{
\begin{enumerate}
\renewcommand{\labelenumi}{\arabic{enumi}.}
% A(\Alph) a(\alph) I(\Roman) i(\roman) 1(\arabic)
%设定全局标号series=example	%引用全局变量resume=example
%[topsep=-0.3em,parsep=-0.3em,itemsep=-0.3em,partopsep=-0.3em]
%可使用leftmargin调整列表环境左边的空白长度 [leftmargin=0em]
\item
$1 \times 10 ^ { - 6 } C$
\item 
$3 \times 10 ^ { 4 } \mathrm { N } / \mathrm { C }$
\item 
$ U=800\ V $
\item 
$ 0.065\ m $

\end{enumerate}
}


\newpage 
\item
\exwhere{$ 2017 $年新课\lmd{2}卷}
如图,两水平面(虚线)之间的距离为$ H $,其间的区域存在方向水平向右的匀强电场。自该区域上方的$ A $点将质量为$ m $、电荷量分别为$ q $和$ -q $($ q>0 $)的带电小球$ M $、$ N $先后以相同的初速度沿平行于电场的方向射出。小球在重力作用下进入电场区域,并从该区域的下边界离开。已知$ N $离开电场时的速度方向竖直向下;$ M $在电场中做直线运动,刚离开电场时的动能为$ N $刚离开电场时的动能的$ 1.5 $倍。不计空气阻力,重力加速度大小为$ g $。求:
\begin{enumerate}
\renewcommand{\labelenumi}{\arabic{enumi}.}
% A(\Alph) a(\alph) I(\Roman) i(\roman) 1(\arabic)
%设定全局标号series=example	%引用全局变量resume=example
%[topsep=-0.3em,parsep=-0.3em,itemsep=-0.3em,partopsep=-0.3em]
%可使用leftmargin调整列表环境左边的空白长度 [leftmargin=0em]
\item
$ M $与$ N $在电场中沿水平方向的位移之比;
\item 
$ A $点距电场上边界的高度;
\item 
该电场的电场强度大小。



\end{enumerate}
\begin{figure}[h!]
\flushright
\includesvg[width=0.25\linewidth]{picture/svg/116}
\end{figure}

\banswer{
\begin{enumerate}
\renewcommand{\labelenumi}{\arabic{enumi}.}
% A(\Alph) a(\alph) I(\Roman) i(\roman) 1(\arabic)
%设定全局标号series=example	%引用全局变量resume=example
%[topsep=-0.3em,parsep=-0.3em,itemsep=-0.3em,partopsep=-0.3em]
%可使用leftmargin调整列表环境左边的空白长度 [leftmargin=0em]
\item
$ 3:1 $
\item 
$ \frac{ 1 }{ 3 } H $
\item 
$E = \frac { m g } { \sqrt { 2 } q }$

\end{enumerate}
}


\newpage
\item
\exwhere{$ 2017 $年新课标\lmd{1}卷}
真空中存在电场强度大小为$ E_{1} $的匀强电场,一带电油滴在该电场中竖直向上做匀速直线运动,速度大小为$ v_{0} $,在油滴处于位置$ A $时,将电场强度的大小突然增大到某值,但保持其方向不变。持续一段时间$ t_{1} $后,又突然将电场反向,但保持其大小不变;再持续同样一段时间后,油滴运动到$ B $点。重力加速度大小为$ g $。
\begin{enumerate}
\renewcommand{\labelenumi}{\arabic{enumi}.}
% A(\Alph) a(\alph) I(\Roman) i(\roman) 1(\arabic)
%设定全局标号series=example	%引用全局变量resume=example
%[topsep=-0.3em,parsep=-0.3em,itemsep=-0.3em,partopsep=-0.3em]
%可使用leftmargin调整列表环境左边的空白长度 [leftmargin=0em]
\item
油滴运动到$ B $点时的速度;
\item 
求增大后的电场强度的大小;为保证后来的电场强度比原来的大,试给出相应的$ t_{1} $和$ v_{0} $应满足的条件。已知不存在电场时,油滴以初速度$ v_{0} $做竖直上抛运动的最大高度恰好等于$ B $、$ A $两点间距离的两倍。

\end{enumerate}

\banswer{
\begin{enumerate}
\renewcommand{\labelenumi}{\arabic{enumi}.}
% A(\Alph) a(\alph) I(\Roman) i(\roman) 1(\arabic)
%设定全局标号series=example	%引用全局变量resume=example
%[topsep=-0.3em,parsep=-0.3em,itemsep=-0.3em,partopsep=-0.3em]
%可使用leftmargin调整列表环境左边的空白长度 [leftmargin=0em]
\item
$v _ { 2 } = v _ { 0 } - 2 g t _ { 1 }$
\item $E _ { 2 } = \left[ 2 - 2 \frac { v _ { 0 } } { g t _ { 1 } } + \frac { 1 } { 4 } \left( \frac { v _ { 0 } } { g t _ { 1 } } \right) ^ { 2 } \right] E _ { 1 } \quad t _ { 1 } > \left( \frac { \sqrt { 5 } } { 2 } + 1 \right) \frac { v _ { 0 } } { g }$



\end{enumerate}
}






\end{enumerate}









