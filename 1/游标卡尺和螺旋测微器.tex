\bta{游标卡尺和螺旋测微器}



\begin{enumerate}
\renewcommand{\labelenumi}{\arabic{enumi}.}
% A(\Alph) a(\alph) I(\Roman) i(\roman) 1(\arabic)
%设定全局标号series=example	%引用全局变量resume=example
%[topsep=-0.3em,parsep=-0.3em,itemsep=-0.3em,partopsep=-0.3em]
%可使用leftmargin调整列表环境左边的空白长度 [leftmargin=0em]
\item
\exwhere{$ 2017 $ 年海南卷}
某同学用游标卡尺分别测量金属圆管的内、外壁直径,游标卡尺的示
数分别如图($ a $)和图($ b $)
所示。
\begin{figure}[h!]
\centering
\includesvg[width=0.83\linewidth]{picture/svg/GZ-3-tiyou-0436}
\end{figure}


由图可读出,圆管内壁的直
径为 \tk{2.23} $ cm $,圆管外壁
的直径为 \tk{2.99} $ cm $;由此可计算出金属圆管横截面的面积。


\item 
\exwhere{$ 2011 $ 年理综天津卷}
用螺旋测微器测量某金属丝直径的结果如图所示。该金属丝的
直径是
\tk{本题读数 $ 1.704 \sim 1.708 $ 都算正确。} 
$ mm $。
\begin{figure}[h!]
\centering
\includesvg[width=0.2\linewidth]{picture/svg/GZ-3-tiyou-0437}
\end{figure}









\item
\exwhere{$ 2013 $ 年山东卷}
图甲为一游标卡尺的结构示意图,当测量一钢笔帽的内径时,应该用游标卡尺的 \tk{A} (填“$ A $”、“$ B $”
或“$ C $”)进行测量;示数如图乙所示,该钢笔帽的内径为 \tk{$ 11.30 \ mm $($ 11.25 \ mm $ 或 $ 11.35 \ mm $)} $ mm $。
\begin{figure}[h!]
\centering
\includesvg[width=0.83\linewidth]{picture/svg/GZ-3-tiyou-0438}
\end{figure}




\newpage
\item
\exwhere{$ 2012 $ 年理综新课标卷}
某同学利用螺旋测微器测量一金属板的厚
度。该螺旋测微器校零时的示数如图($ a $)所示,
测量金属板厚度时的示数如图($ b $)所示。图($ a $)
所示读数为 \tk{$ 0.010 $} $ mm $,图($ b $)所示读数为
\tk{$ 6.870 $} 
$ mm $, 所 测 金 属 板 的 厚 度 为
\tk{$ 6.860 $} 
$ mm $。
\begin{figure}[h!]
\centering
\includesvg[width=0.43\linewidth]{picture/svg/GZ-3-tiyou-0439}
\end{figure}


\banswer{

}

\item 
\exwhere{$ 2014 $ 年理综福建卷}
某同学测定一金属杆的长度和直径,示数如图甲、乙所示,则该金属杆的长度和直径分别为
\tk{$ 60.10 $} 
$ cm $ 和
\tk{$ 4.20 $} 
$ mm $。
\begin{figure}[h!]
\centering
\includesvg[width=0.83\linewidth]{picture/svg/GZ-3-tiyou-0440}
\end{figure}


\banswer{

}

\item 
\exwhere{$ 2015 $ 年海南卷}
某同学利用游标卡尺和螺旋测微器分别测量一圆柱体工件的直径和高度,测量
结果如图($ a $)和($ b $)所示。该工件的直径为 \tk{$ 1.220 \ cm $} $ cm $,高度为 \tk{$ 6.861 \ mm $} $ mm $。
\begin{figure}[h!]
\centering
\includesvg[width=0.83\linewidth]{picture/svg/GZ-3-tiyou-0441}
\end{figure}

\banswer{

}



\newpage
\item 
\exwhere{$ 2014 $ 年物理海南卷}
现有一合金制成的圆柱体,为测量该合金的电阻率,现用伏安法测圆柱体两端之间的电阻,用
螺旋测微器测量该圆柱体的直径,用游标卡尺测量该
圆柱体的长度。螺旋测微器和游标卡尺的示数如图($ a $)
和图($ b $)所示。
\begin{figure}[h!]
\centering
\includesvg[width=0.53\linewidth]{picture/svg/GZ-3-tiyou-0442}
\end{figure}

\begin{enumerate}
\renewcommand{\labelenumi}{\arabic{enumi}.}
% A(\Alph) a(\alph) I(\Roman) i(\roman) 1(\arabic)
%设定全局标号series=example	%引用全局变量resume=example
%[topsep=-0.3em,parsep=-0.3em,itemsep=-0.3em,partopsep=-0.3em]
%可使用leftmargin调整列表环境左边的空白长度 [leftmargin=0em]
\item
由上图读得圆柱体的直径为
\tk{$ 1.844(2 $ 分。在 $ 1.842-1.846 $ 范围内的均给分)} 
$ mm $,长度
为
\tk{$ 4.240 $} 
$ cm $.


\item 
若流经圆柱体的电流为 $ I $ ,圆柱体两端之间的电压为 $ U $,圆柱体的直径和长度分别用 $ D $、$ L $ 表
示,则用 $ D $、$ L $、 $ I $ 、$ U $ 表示的电阻率的关系式为$ \rho = $ \tk{$\frac{\pi D^{2} U}{4 I L}$} 。

\end{enumerate}







\end{enumerate}

