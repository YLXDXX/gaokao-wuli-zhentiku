\bta{电容和电感对交流电路的影响}

\begin{enumerate}
%\renewcommand{\labelenumi}{\arabic{enumi}.}
% A(\Alph) a(\alph) I(\Roman) i(\roman) 1(\arabic)
%设定全局标号series=example	%引用全局变量resume=example
%[topsep=-0.3em,parsep=-0.3em,itemsep=-0.3em,partopsep=-0.3em]
%可使用leftmargin调整列表环境左边的空白长度 [leftmargin=0em]
\item
\exwhere{$ 2017 $ 年江苏卷}
某音响电路的简化电路图如图所示,输入信号既有高频成分,也有低频成分,
则 \xzanswer{BD} 
\begin{figure}[h!]
\centering
\includesvg[width=0.23\linewidth]{picture/svg/GZ-3-tiyou-1133}
\end{figure}

\fourchoices
{电感 $ L_{1} $ 的作用是通高频}
{电容 $ G_{2} $ 的作用是通高频}
{扬声器甲用于输出高频成分}
{扬声器乙用于输出高频成分}




\item 
\exwhere{$ 2017 $ 年海南卷}
如图,电阻 $ R $、电容 $ C $ 和电感 $ L $ 并联后,接入输出电压有效值、频率可调的交
流电源。当电路中交流电的频率为 $ f $ 时,通过 $ R $、$ C $ 和 $ L $ 的电流有
效值恰好相等。若将频率降低为
$ \frac{ 1 }{ 2 } f $,分别用 $ I_{1} $、$ I_{2} $ 和 $ I_{3} $ 表示此时
通过 $ R $、$ C $ 和 $ L $ 的电流有效值,则 \xzanswer{BC} 
\begin{figure}[h!]
\centering
\includesvg[width=0.23\linewidth]{picture/svg/GZ-3-tiyou-1134}
\end{figure}

\fourchoices
{$ I_{1} > I_{3} $}
{$ I_{1} > I_{2} $}
{$ I_{3} > I_{2} $}
{$ I_{2} = I_{3} $}









\end{enumerate}

