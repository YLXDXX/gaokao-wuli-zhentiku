\bta{图像在物理解题中的应用}

\begin{enumerate}[leftmargin=0em]
\renewcommand{\labelenumi}{\arabic{enumi}.}
% A(\Alph) a(\alph) I(\Roman) i(\roman) 1(\arabic)
%设定全局标号series=example	%引用全局变量resume=example
%[topsep=-0.3em,parsep=-0.3em,itemsep=-0.3em,partopsep=-0.3em]
%可使用leftmargin调整列表环境左边的空白长度 [leftmargin=0em]
\item
\exwhere{$ 2019 $年物理全国\lmd{3}卷}
从地面竖直向上抛出一物体,物体在运动过程中除受到重力外,还受到一大小不变、方向始终与运动方向相反的外力作用。距地面高度$ h $在$ 3 \ m $以内时,物体上升、下落过程中动能$ E_{k} $随$ h $的变化如图所示。重力加速度取$ 10 \ m/s^{2} $。该物体的质量为 \xzanswer{C} 
\begin{figure}[h!]
\centering
\includesvg[width=0.23\linewidth]{picture/svg/795}
\end{figure}

\fourchoices
{$ 2 \ kg $}
{$ 15 \ kg $	}
{$ 1 \ kg $}
{$ 0.5 \ kg $}


\item 
\exwhere{$ 2019 $年物理全国\lmd{2}卷}
从地面竖直向上抛出一物体,其机械能$ E_{ \text{总} } $等于动能$ E_{k} $与重力势能$ E_{p} $之和。取地面为重力势能零点,该物体的$ E_{ \text{总} } $和$ E_{p} $随它离开地面的高度$ h $的变化如图所示。重力加速度取$ 10 $ $ m/s^{2} $。由图中数据可得 \xzanswer{AD} 
\begin{figure}[h!]
\centering
\includesvg[width=0.23\linewidth]{picture/svg/796}
\end{figure}


\fourchoices
{物体的质量为$ 2 $ $ kg $}
{$ h=0 $时,物体的速率为$ 20 $ $ m/s $}
{$ h=2 $ $ m $时,物体的动能$ E_{k} =40 $ $ J $}
{从地面至$ h=4 $ $ m $,物体的动能减少$ 100 $ $ J $}


\item 
\exwhere{$ 2011 $年海南卷}
一物体自$ t=0 $时开始做直线运动,其速度图线如图所示。
下列选项正确的是 \xzanswer{BC} 
\begin{figure}[h!]
\centering
\includesvg[width=0.23\linewidth]{picture/svg/797}
\end{figure}

\fourchoices
{在$ 0 \sim 6\ s $内,物体离出发点最远为$ 30 \ m $}
{在$ 0 \sim 6\ s $内,物体经过的路程为$ 40 \ m $}
{在$ 0 \sim 4\ s $内,物体的平均速率为$ 7.5 \ m/s $}
{在$ 5 \sim 6\ s $内,物体所受的合外力做负功}


\item 
\exwhere{$ 2013 $年浙江卷}

$ 17 $.如图所示,水平板上有质量$ m=1.0 \ kg $的物块,受到随时间$ t $变化的水平拉力$ F $作用,用力传感器测出相应时刻物块所受摩擦力$ F_{f} $的大小。取重力加速度$ g=10 \ m/s^{2} $。下列判断正确的是 \xzanswer{D} 
\begin{figure}[h!]
\centering
\includesvg[width=0.23\linewidth]{picture/svg/798} \qquad 
 \includesvg[width=0.43\linewidth]{picture/svg/799} 
\end{figure}

\fourchoices
{$ 5\ s $内拉力对物块做功为零}
{$ 4\ s $末物块所受合力大小为$ 4.0 \ N $}
{物块与木板之间的动摩擦因数为$ 0.4 $}
{$ 6\ s \sim 9\ s $内物块的加速度的大小为$ 2.0 \ m/s^{2} $}



\item
\exwhere{$ 2012 $年理综天津卷}
如图甲所示,静止在水平地面的物块$ A $,受到水平向右的拉力$ F $的作用,$ F $与时间$ t $的关系如图乙所示,设物块与地面的静摩擦力最大值$ f_m $与滑动摩擦力大小相等,则 \xzanswer{BD} 
\begin{figure}[h!]
\centering
\includesvg[width=0.43\linewidth]{picture/svg/800}
\end{figure}

\fourchoices
{$ 0 \sim t_{1} $时间内$ F $的功率逐渐增大}
{$ t_{2} $时刻物块$ A $的加速度最大}
{$ t_{2} $时刻后物块$ A $做反向运动}
{$ t_{3} $时刻物块$ A $的动能最大}



\item
\exwhere{$ 2012 $年物理海南卷}
如图,表面处处同样粗糙的楔形木块$ abc $固定在水平地面上,$ ab $面和$ bc $面与地面的夹角分别为$ \alpha $和$ \beta $,且$ \alpha $>$ \beta $。一初速度为$ v_{0} $的小物块沿斜面$ ab $向上运动,经时间$ to $后到达顶点$ b $时,速度刚好为零;然后让小物块立即从静止开始沿斜面$ bc $下滑。在小物块从$ a $运动到$ c $的过程中,可能正确描述其速度大小$ v $与时间$ t $的关系的图像是 \xzanswer{C} 
\begin{figure}[h!]
\centering
\includesvg[width=0.23\linewidth]{picture/svg/801}\\
 \includesvg[width=0.83\linewidth]{picture/svg/802} 
\end{figure}



\item 
\exwhere{$ 2014 $年物理上海卷}
如图,竖直平面内的轨道 \lmd{1} 和 \lmd{2} 都由两段细直杆连接而成,两轨道长度相等。用相同的水平恒力将穿在轨道最低点$ B $的静止小球,分别沿 \lmd{1} 和 \lmd{2} 推至最高点$ A $,所需时间分别为$ t_{1} $、$ t_{2} $;动能增量分别为$ \triangle E_{k1} $、$ \triangle E_{k2} $,假定球在经过轨道转折点前后速度大小不变,且球与 \lmd{1} 、 \lmd{2} 轨道间的动摩擦因数相等,则 \xzanswer{B} 
\begin{figure}[h!]
\centering
\includesvg[width=0.23\linewidth]{picture/svg/804}
\end{figure}

\fourchoices
{$\Delta E _ { k 1 } > \Delta E _ { k 2 } \quad t _ { 1 } > t _ { 2 }$}
{$\Delta E _ { k 1 } = \Delta E _ { k 2 } \quad t _ { 1 } > t _ { 2 }$}
{$\Delta E _ { k 1 } > \Delta E _ { k 2 } \quad t _ { 1 } < t _ { 2 }$}
{$\Delta E _ { k 1 } = \Delta E _ { k 2 } \quad t _ { 1 } < t _ { 2 }$}





\item 
\exwhere{$ 2014 $年理综四川卷}
如右图所示,水平传送带以速度$ v_{1} $匀速运动,小物体$ P $、$ Q $由通过定滑轮且不可伸长的轻绳相连,$ t $ $ = 0 $时刻$ P $在传送带左端具有速度$ v_{2} $,$ P $与定滑轮间的绳水平,$ t $ $ = t_{0} $时刻$ P $离开传送带。不计定滑轮质量和摩擦,绳足够长。正确描述小物体$ P $速度随时间变化的图像可能是 \xzanswer{BC} 
\begin{figure}[h!]
\centering
\includesvg[width=0.23\linewidth]{picture/svg/805}\\
 \includesvg[width=0.83\linewidth]{picture/svg/806} 
\end{figure}









\item 
\exwhere{$ 2014 $年理综福建卷}
如右图,滑块以初速度$ v_{0} $沿表面粗糙且足够长的固定斜面,从顶端下滑,直至速度为零。对于该运动过程,若用$ h $、$ s $、$ v $、$ a $分别表示滑块的下降高度、位移、速度和加速度的大小,$ t $表示时间,则下列图像最能正确描述这一运动规律的是 \xzanswer{B} 
\begin{figure}[h!]
\centering
\includesvg[width=0.23\linewidth]{picture/svg/807}\\
 \includesvg[width=0.83\linewidth]{picture/svg/808} 
\end{figure}

\item 
\exwhere{$ 2016 $年海南卷}
沿固定斜面下滑的物体受到与斜面平行向上的拉力$ F $的作用,其下滑的速度$ - $时间图线如图所示。已知物体与斜面之间的动摩擦因数为常数,在$ 0 \sim 5\ s $,$ 5 \sim 10\ s $,$ 10 \sim 15\ s $内$ F $的大小分别为$ F_{1} $、$ F_{2} $和$ F_{3} $,则 \xzanswer{A} 
\begin{figure}[h!]
\centering
\includesvg[width=0.23\linewidth]{picture/svg/809}
\end{figure}

\fourchoices
{$ F_{1} < F_{2} $ }
{$ F_{2} >F_3 $}
{$ F_{1} >F3 $ }
{$ F_{1} =F_3 $}



\item 
\exwhere{$ 2018 $年海南卷}
如图($ a $),一长木板静止于光滑水平桌面上,$ t=0 $时,小物块以速度$ v_{0} $滑到长木板上,图($ b $)为物块与木板运动的$ v-t $图像,图中$ t_{1} $、$ v_{0} $、$ v_{1} $已知。重力加速度大小为$ g $。由此可求得 \xzanswer{BC} 
\begin{figure}[h!]
\centering
\includesvg[width=0.23\linewidth]{picture/svg/810} \qquad 
 \includesvg[width=0.23\linewidth]{picture/svg/811} 
\end{figure}

\fourchoices
{木板的长度}
{物块与木板的质量之比}
{物块与木板之间的动摩擦因数}
{从$ t=0 $开始到$ t_{1} $时刻,木板获得的动能}

\item 
\exwhere{$ 2018 $年全国\lmd{3}卷}
地下矿井中的矿石装在矿车中,用电机通过竖井运送到地面。某竖井中矿车提升的速度大小$ v $随时间$ t $的变化关系如图所示,其中图线①②分别描述两次不同的提升过程,它们变速阶段加速度的大小都相同;两次提升的高度相同,提升的质量相等。不考虑摩擦阻力和空气阻力。对于第①次和第②次提升过程 \xzanswer{AC} 
\begin{figure}[h!]
\centering
\includesvg[width=0.23\linewidth]{picture/svg/812}
\end{figure}

\fourchoices
{矿车上升所用的时间之比为$ 4:5 $}
{电机的最大牵引力之比为$ 2:1 $}
{电机输出的最大功率之比为$ 2:1 $}
{电机所做的功之比为$ 4:5 $}

\item 
\exwhere{$ 2013 $年新课标 \lmd{1} 卷}
$ 2012 $年$ 11 $曰,“歼$ 15 $”舰载机在“辽宁号”航空母舰上着舰成功。图($ a $)为利用阻拦系统让舰载机在飞行甲板上快速停止的原理示意图。飞机着舰并成功钩住阻拦索后,飞机的动力系统立即关闭,阻拦系统通过阻拦索对飞机施加一作用力,使飞机在甲板上短距离滑行后停止。某次降落,以飞机着舰为计时零点,飞机在$ t=0.4\ s $时恰好钩住阻拦索中间位置,其着舰到停止的速度一时间图线如图$ (b) $所示。假如无阻拦索,飞机从着舰到停止需要的滑行距离约为$ 1000\ m $。已知航母始终静止,重力加速度的大小为$ g $。则 \xzanswer{AC} 
\begin{figure}[h!]
\centering
\includesvg[width=0.5\linewidth]{picture/svg/813}
\end{figure}

\fourchoices
{从着舰到停止,飞机在甲板上滑行的距离约为无阻拦索时的$ 1/10 $}
{在$ 0.4\ s \sim 2.5\ s $时间内,阻拦索的张力几乎不随时间变化}
{在滑行过程中,飞行员所承受的加速度大小会超过$ 2.5 $ $ g $}
{在$ 0.4\ s \sim 2.5\ s $时间内,阻拦系统对飞机做功的功率儿乎不变}



\item 
\exwhere{$ 2015 $年理综新课标 \lmd{1} 卷}
如图($ a $),一物块在$ t=0 $时刻滑上一固定斜面,其运动的ν$ -t $图线如图($ b $)所示。若重力加速度及图中的$ v_{0} $,$ v_{1} $,$ t_{1} $均为已知量,则可求出 \xzanswer{ACD} 
\begin{figure}[h!]
\centering
\includesvg[width=0.4\linewidth]{picture/svg/814}
\end{figure}

\fourchoices
{斜面的倾角}
{物块的质量}
{物块与斜面间的动摩擦因数}
{物块沿斜面向上滑行的最大高度}

\item 
\exwhere{$ 2013 $年广东卷}
如图,游乐场中,从高处$ A $到水面$ B $处有两条长度相同的光滑轨道。甲、乙两小孩沿不同轨道同时从$ A $处自由滑向$ B $处,下列说法正确的有 \xzanswer{BD} 
\begin{figure}[h!]
\centering
\includesvg[width=0.23\linewidth]{picture/svg/815}
\end{figure}

\fourchoices
{甲的切向加速度始终比乙的大}
{甲、乙在同一高度的速度大小相等}
{甲、乙在同一时刻总能到达同一高度}
{甲比乙先到达$ B $处}

\item 
\exwhere{$ 2015 $年上海卷}
一颗子弹以水平速度$ v_{0} $穿透一块在光滑水平面上迎面滑来的木块后,二者运动方向均不变。设子弹与木块间相互作用力恒定,木块最后速度为$ v $,则 \xzanswer{AC} 


\fourchoices
{$ v_{0} $越大,$ v $越大}
{$ v_{0} $越小,$ v $越大}
{子弹质量越大,$ v $越大}
{木块质量越小,$ v $越大}


\newpage
\item 
\exwhere{$ 2013 $年安徽卷}
一物体放在水平地面上,如图$ \lmd{1} $所示,已知物体所受水平拉力$ F $随时间的变化情况如图$ 2 $所示,物体相应的速度随时间的变化关系如图$ 3 $所示。求:
\begin{enumerate}
\renewcommand{\labelenumi}{\arabic{enumi}.}
% A(\Alph) a(\alph) I(\Roman) i(\roman) 1(\arabic)
%设定全局标号series=example	%引用全局变量resume=example
%[topsep=-0.3em,parsep=-0.3em,itemsep=-0.3em,partopsep=-0.3em]
%可使用leftmargin调整列表环境左边的空白长度 [leftmargin=0em]
\item
$ 0 \sim 8\ s $时间内拉力的冲量;
\item 
$ 0 \sim 6\ s $时间内物体的位移;
\item 
$ 0 \sim 10\ s $时间内,物体克服摩擦力所做的功。



\end{enumerate}
\begin{figure}[h!]
\flushright
\includesvg[width=0.55\linewidth]{picture/svg/803}
\end{figure}



\banswer{
\begin{enumerate}
\renewcommand{\labelenumi}{\arabic{enumi}.}
% A(\Alph) a(\alph) I(\Roman) i(\roman) 1(\arabic)
%设定全局标号series=example	%引用全局变量resume=example
%[topsep=-0.3em,parsep=-0.3em,itemsep=-0.3em,partopsep=-0.3em]
%可使用leftmargin调整列表环境左边的空白长度 [leftmargin=0em]
\item
$ 18 \ N \cdot S $
\item 
$ 6 \ m $
\item 
$ 30 \ J $
\end{enumerate}


}



\item 
\exwhere{$ 2014 $年理综新课标 \lmd{2}卷}
$ 2012 $年$ 10 $月,奥地利极限运动员菲利克斯$ \cdot $鲍姆加特纳乘气球升至约$ 39 \ km $的高空后跳下,经过$ 4 $分$ 20 $秒到达距地面约$ 1.5 \ km $高度处,打开降落伞并成功落地,打破了跳伞运动的多项世界纪录,取重力加速度的大小$ g=10 \ m/s^{2} $.

\begin{enumerate}
\renewcommand{\labelenumi}{\arabic{enumi}.}
% A(\Alph) a(\alph) I(\Roman) i(\roman) 1(\arabic)
%设定全局标号series=example	%引用全局变量resume=example
%[topsep=-0.3em,parsep=-0.3em,itemsep=-0.3em,partopsep=-0.3em]
%可使用leftmargin调整列表环境左边的空白长度 [leftmargin=0em]
\item
忽略空气阻力,求该运动员从静止开始下落到$ 1.5 \ km $高度处所需要的时间及其在此处速度的大小;
\item 
实际上物体在空气中运动时会受到空气阻力,高速运动受阻力大小可近似表示为$ f=kv^{2} $,其中$ v $为速率,$ k $为阻力系数,其数值与物体的形状,横截面积及空气密度有关,已知该运动员在某段时间内高速下落的$ v-t $图象如图所示,着陆过程中,运动员和所携装备的总质量$ m=100 \ kg $,试估算该运动员在达到最大速度时所受阻力的阻力系数(结果保留$ 1 $位有效数字).



\end{enumerate}
\begin{figure}[h!]
\flushright
\includesvg[width=0.33\linewidth]{picture/svg/816}
\end{figure}

\banswer{
\begin{enumerate}
\renewcommand{\labelenumi}{\arabic{enumi}.}
% A(\Alph) a(\alph) I(\Roman) i(\roman) 1(\arabic)
%设定全局标号series=example	%引用全局变量resume=example
%[topsep=-0.3em,parsep=-0.3em,itemsep=-0.3em,partopsep=-0.3em]
%可使用leftmargin调整列表环境左边的空白长度 [leftmargin=0em]
\item
$v = 8.7 \times 10 ^ { 2 } \ \mathrm { m } / \mathrm { s }$
\item 
$k = 0.008\ \mathrm { kg } / \mathrm { m }$

\end{enumerate}


}


\newpage
\item 
\exwhere{$ 2012 $年理综北京卷}
摩天大楼中一部直通高层的客运电梯,行程超过百米。电梯的简化模型如$ 1 $所示。考虑安全、舒适、省时等因索,电梯的加速度$ a $随时间$ t $变化的。已知电梯在$ t=0 $时由静止开始上升,$ a - t $图像如图$ 2 $所示。电梯总质最$ m=2.0 \times 10^3 \ kg $。忽略一切阻力,重力加速度$ g $取$ 10 \ m/s^{2} $。


\begin{enumerate}
\renewcommand{\labelenumi}{\arabic{enumi}.}
% A(\Alph) a(\alph) I(\Roman) i(\roman) 1(\arabic)
%设定全局标号series=example	%引用全局变量resume=example
%[topsep=-0.3em,parsep=-0.3em,itemsep=-0.3em,partopsep=-0.3em]
%可使用leftmargin调整列表环境左边的空白长度 [leftmargin=0em]
\item
求电梯在上升过程中受到的最大拉力$ F_{1} $和最小拉力$ F_{2} $;
\item 
类比是一种常用的研究方法。对于直线运动,教科书中讲解了由$ v-t $图像求位移的方法。请你借鉴此方法,对比加速度和速度的定义,根据图$ 2 $所示$ a-t $图像,求电梯在第$ 1s $内的速度改变量$ \triangle v_{1} $和第$ 2s $末的速率$ v_{2} $;
\item 
求电梯以最大速率上升时,拉力做功的功率$ P $;再求在$ 0 \sim 11\ s $时间内,拉力和重力对电梯所做的总功$ W $。


\end{enumerate}
\begin{figure}[h!]
\flushright 
\includesvg[width=0.55\linewidth]{picture/svg/817}
\end{figure}

\banswer{
\begin{enumerate}
\renewcommand{\labelenumi}{\arabic{enumi}.}
% A(\Alph) a(\alph) I(\Roman) i(\roman) 1(\arabic)
%设定全局标号series=example	%引用全局变量resume=example
%[topsep=-0.3em,parsep=-0.3em,itemsep=-0.3em,partopsep=-0.3em]
%可使用leftmargin调整列表环境左边的空白长度 [leftmargin=0em]
\item
$F _ { 1 } = m a _ { \min } + m g = 2.0 \times 10 ^ { 3 } \times ( - 1.0 + 10 ) = 1.8 \times 10 ^ { 4 } \mathrm { N }$
\item 
$\Delta v _ { 1 } = \frac { 1 } { 2 } \times 1 \times 1.0 = 0.5 \mathrm { m } / \mathrm { s }$ \qquad $v _ { 2 } = \frac { 1 } { 2 } \times ( 1 + 2 ) \times 1.0 = 1.5 \mathrm { m } / \mathrm { s }$
\item 
$P = F v _ { \max } = m g v _ { \max } = 2 \times 10 ^ { 3 } \times 10 \times 10 = 2 \times 10 ^ { 5 } \mathrm { W }$ \qquad $W = \frac { 1 } { 2 } m v _ { \max } ^ { 2 } = \frac { 1 } { 2 } \times 2 \times 10 ^ { 3 } \times 10 ^ { 2 } = 1 \times 10 ^ { 5 } \mathrm { J }$



\end{enumerate}


}







\end{enumerate}

