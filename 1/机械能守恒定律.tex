\bta{机械能守恒定律}

\begin{enumerate}[leftmargin=0em]
\renewcommand{\labelenumi}{\arabic{enumi}.}
% A(\Alph) a(\alph) I(\Roman) i(\roman) 1(\arabic)
%设定全局标号series=example	%引用全局变量resume=example
%[topsep=-0.3em,parsep=-0.3em,itemsep=-0.3em,partopsep=-0.3em]
%可使用leftmargin调整列表环境左边的空白长度 [leftmargin=0em]
\item
\exwhere{$ 2012 $年理综浙江卷}
由光滑细管组成的轨道如图所示,其中$ AB $段和$ BC $段是半径为$ R $的四分之一圆弧,轨道固定在竖直平面内。一质量为$ m $的小球,从距离水平地面高为$ H $的管口$ D $处静止释放,最后能够从$ A $端水平抛出落到地面上。下列说法正确的是 \xzanswer{BC} 
\begin{figure}[h!]
\centering
\includesvg[width=0.3\linewidth]{picture/svg/765}
\end{figure}

\fourchoices
{小球落到地面时相对于$ A $点的水平位移值为$2 \sqrt { R H - 2 R ^ { 2 } }$}
{小球落到地面时相对于$ A $点的水平位移值为$\sqrt { 2 R H - 4 R ^ { 2 } }$}
{小球能从细管$ A $端水平抛出的条件是$ H>2R $}
{小球能从细管$ A $端水平抛出的最小高度$H _ { \min } = \frac { 5 } { 2 } R$}




\item 
\exwhere{$ 2012 $年物理上海卷}
如图,可视为质点的小球$ A $、$ B $用不可伸长的细软轻线连接,跨过固定在地面上、半径为$ R $的光滑圆柱,$ A $的质量为$ B $的两倍。当$ B $位于地面时,$ A $恰与圆柱轴心等高。将$ A $由静止释放,$ B $上升的最大高度是 \xzanswer{C} 
\begin{figure}[h!]
\centering
\includesvg[width=0.23\linewidth]{picture/svg/766}
\end{figure}


\fourchoices
{$ 2R $}
{$ 5R/3 $}
{$ 4R/3 $ }
{$ 2R/3 $}





\item
\exwhere{$ 2014 $年理综安徽卷}
如图所示,有一内壁光滑的闭合椭圆形管道,置于竖直平面内,$ MN $是通过椭圆中心$ O $点的水平线。已知一小球从$ M $点出发,初速率为$ v_{0} $,沿管道$ MPN $运动,到$ N $点的速率为$ v_{1} $,所需的时间为$ t_{1} $;若该小球仍由$ M $点以初速率$ v_{0} $出发,而沿管道$ MQN $运动,到$ N $点的速率为$ v_{2} $,所需时间为$ t_{2} $。则 \xzanswer{A} 
\begin{figure}[h!]
\centering
\includesvg[width=0.23\linewidth]{picture/svg/767}
\end{figure}

\fourchoices
{$ v_{1} = v_{2} $,$ t_{1} > t_{2} $}
{$ v_{1} < v_{2} $,$ t_{1} > t_{2} $}
{$ v_{1} = v_{2} $,$ t_{1} < t_{2} $}
{$ v_{1} < v_{2} $,$ t_{1} < t_{2} $}



\item 
\exwhere{$ 2011 $年新课标卷}
一蹦极运动员身系弹性蹦极绳从水面上方的高台下落,到最低点时距水面还有数米距离。假定空气阻力可忽略,运动员可视为质点,下列说法正确的是 \xzanswer{ABC} 

\fourchoices
{运动员到达最低点前重力势能始终减小}
{蹦极绳张紧后的下落过程中,弹性力做负功,弹性势能增加}
{蹦极过程中,运动员、地球和蹦极绳所组成的系统机械能守恒}
{蹦极过程中,重力势能的改变与重力势能零点的选取有关}



\item 
\exwhere{$ 2015 $年理综四川卷}
在同一位置以相同的速率把三个小球分别沿水平、斜向上、斜向下方向抛出,不计空气阻力,则落在同一水平地面地面时的速度大小 \xzanswer{A} 

\fourchoices
{一样大}
{水平抛的最大}
{斜向上抛的最大}
{斜向下抛的最大}


\item
\exwhere{$ 2015 $年理综新课标$ \lmd{2} $卷}
如图,滑块$ a $、$ b $的质量均为$ m $,$ a $套在固定直杆上,与光滑水平地面相距$ h $,$ b $放在地面上,$ a $、$ b $通过铰链用刚性轻杆连接,由静止开始运动。不计摩擦,$ a $、$ b $可视为质点,重力加速度大小为$ g $。则 \xzanswer{BD} 
\begin{figure}[h!]
\centering
\includesvg[width=0.23\linewidth]{picture/svg/768}
\end{figure}

\fourchoices
{$ a $落地前,轻杆对$ b $一直做正功}
{$ a $落地时速度大小为$\sqrt{2gh}$}
{$ a $下落过程中,其加速度大小始终不大于$ g $}
{$ a $落地前,当$ a $的机械能最小时,$ b $对地面的压力大小为$ mg $}


\item
$ 2017 $年江苏卷$ 9 $.如图所示,三个小球$ A $、$ B $、$ C $的质量均为$ m $,$ A $与$ B $、$ C $间通过铰链用轻杆连接,杆长为$ L $,$ B $、$ C $置于水平地面上,用一轻质弹簧连接,弹簧处于原长。现$ A $由静止释放下降到最低点,两轻杆间夹角$ \alpha $由$ 60 ^{ \circ } $变为$ 120 ^{ \circ } $,$ A $、$ B $、$ C $在同一竖直平面内运动,弹簧在弹性限度内,忽略一切摩擦,重力加速度为$ g $. 则此下降过程中 \xzanswer{AB} 
\begin{figure}[h!]
\centering
\includesvg[width=0.23\linewidth]{picture/svg/769}
\end{figure}


\fourchoices
{$ A $的动能达到最大前,$ B $受到地面的支持力小于$ \frac{ 3 }{ 2 } mg $}
{$ A $的动能最大时,$ B $受到地面的支持力等于$ \frac{ 3 }{ 2 } mg $}
{弹簧的弹性势能最大时,$ A $的加速度方向竖直向下}
{弹簧的弹性势能最大值为$ \frac{\sqrt{3}}{2}mgL $}




\item 
\exwhere{$ 2011 $年理综北京卷}
如图所示,长度为$ l $的轻绳上端固定在$ O $点,下端系一质量为$ m $的小球(小球的大小可以忽略)。
\begin{enumerate}
\renewcommand{\labelenumi}{\arabic{enumi}.}
% A(\Alph) a(\alph) I(\Roman) i(\roman) 1(\arabic)
%设定全局标号series=example	%引用全局变量resume=example
%[topsep=-0.3em,parsep=-0.3em,itemsep=-0.3em,partopsep=-0.3em]
%可使用leftmargin调整列表环境左边的空白长度 [leftmargin=0em]
\item
在水平拉力$ F $的作用下,轻绳与竖直方向的夹角为$ \alpha $,小球保持静止。画出此时小球的受力图,并求力$ F $的大小; 
\item 
由图示位置无初速度释放小球,求当小球通过最低点时的速度大小及轻绳对小球的拉力。不计空气阻力。



\end{enumerate}
\begin{figure}[h!]
\flushright
\includesvg[width=0.21\linewidth]{picture/svg/770}
\end{figure}


\banswer{
\begin{enumerate}
\renewcommand{\labelenumi}{\arabic{enumi}.}
% A(\Alph) a(\alph) I(\Roman) i(\roman) 1(\arabic)
%设定全局标号series=example	%引用全局变量resume=example
%[topsep=-0.3em,parsep=-0.3em,itemsep=-0.3em,partopsep=-0.3em]
%可使用leftmargin调整列表环境左边的空白长度 [leftmargin=0em]
\item
$F = m g \tan \alpha$
\item 
$T ^ { \prime } = m g + \frac { m v ^ { 2 } } { l } = m g ( 3 - 2 \cos \alpha )$



\end{enumerate}


}




\item 
\exwhere{$ 2014 $年理综福建卷}
图为某游乐场内水上滑梯轨道示意图。整个轨道在同一竖直平面内,表面粗糙的$ AB $段轨道与四分之一光滑圆弧轨道$ BC $在$ B $点水平相切。点$ A $距水面的高度为$ H $,圆弧轨道$ BC $的半径为$ R $,圆心$ O $恰在水面。一质量为$ m $的游客(视为质点)可从轨道$ AB $上任意位置滑下,不计空气阻力。
\begin{enumerate}
\renewcommand{\labelenumi}{\arabic{enumi}.}
% A(\Alph) a(\alph) I(\Roman) i(\roman) 1(\arabic)
%设定全局标号series=example	%引用全局变量resume=example
%[topsep=-0.3em,parsep=-0.3em,itemsep=-0.3em,partopsep=-0.3em]
%可使用leftmargin调整列表环境左边的空白长度 [leftmargin=0em]
\item
若游客从$ A $点由静止开始滑下,到$ B $点时沿切线方向滑离轨道落在水面$ D $点,$ OD=2R $,求游客滑到$ B $点时的速度$ v_{B} $大小及运动过程轨道摩擦力对其所做的功$ Wf $;
\item 
若游客从$ AB $段某处滑下,恰好停在$ B $点,又因受到微小扰动,继续沿圆弧轨道滑到$ P $点后滑离轨道,求$ P $点离水面的高度$ h $。(提示:在圆周运动过程中任一点,质点所受的向心力与其速率的关系为)



\end{enumerate}
\begin{figure}[h!]
\flushright
\includesvg[width=0.3\linewidth]{picture/svg/771}
\end{figure}


\banswer{
\begin{enumerate}
\renewcommand{\labelenumi}{\arabic{enumi}.}
% A(\Alph) a(\alph) I(\Roman) i(\roman) 1(\arabic)
%设定全局标号series=example	%引用全局变量resume=example
%[topsep=-0.3em,parsep=-0.3em,itemsep=-0.3em,partopsep=-0.3em]
%可使用leftmargin调整列表环境左边的空白长度 [leftmargin=0em]
\item
$W _ { f } = - ( m g H - 2 m g R )$
\item 
$h = \frac { 2 } { 3 } R$


\end{enumerate}


}


\newpage
\item 
\exwhere{$ 2015 $年上海卷}
质量为$ m $的小球在竖直向上的恒定拉力作用下,由静止开始从水平地面向上运动,经一段时间,拉力做功为$ W $,此后撤去拉力,球又经相同时间回到地面。以地面为零势能面,不计空气阻力。求:
\begin{enumerate}
\renewcommand{\labelenumi}{\arabic{enumi}.}
% A(\Alph) a(\alph) I(\Roman) i(\roman) 1(\arabic)
%设定全局标号series=example	%引用全局变量resume=example
%[topsep=-0.3em,parsep=-0.3em,itemsep=-0.3em,partopsep=-0.3em]
%可使用leftmargin调整列表环境左边的空白长度 [leftmargin=0em]
\item
球回到地面时的动能$ E_{k} t $;
\item 
撤去拉力前球的加速度大小$ a $及拉力的大小$ F $;
\item 
球动能为$ W/5 $时的重力势能$ E_{p} $。



\end{enumerate}

\banswer{
\begin{enumerate}
\renewcommand{\labelenumi}{\arabic{enumi}.}
% A(\Alph) a(\alph) I(\Roman) i(\roman) 1(\arabic)
%设定全局标号series=example	%引用全局变量resume=example
%[topsep=-0.3em,parsep=-0.3em,itemsep=-0.3em,partopsep=-0.3em]
%可使用leftmargin调整列表环境左边的空白长度 [leftmargin=0em]
\item
$ W $
\item 
$F = \frac { 4 } { 3 } m g$
\item 
$\frac { 3 } { 5 } W$ 或 $\frac { 4 } { 5 } W$
\end{enumerate}


}



\item 
\exwhere{$ 2018 $年江苏卷}
如图所示,钉子$ A $、$ B $相距$ 5l $,处于同一高度。细线的一端系有质量为$ M $的小物块,另一端绕过$ A $固定于$ B $。质量为$ m $的小球固定在细线上$ C $点,$ B $、$ C $间的线长为$ 3l $。用手竖直向下拉住小球,使小球和物块都静止,此时$ BC $与水平方向的夹角为$ 53 ^{ \circ } $。松手后,小球运动到与$ A $、$ B $相同高度时的速度恰好为零,然后向下运动。忽略一切摩擦,重力加速度为$ g $,取$ \sin 53 ^{ \circ } =0.8 $,$ \cos 53 ^{ \circ } =0.6 $。求:
\begin{enumerate}
\renewcommand{\labelenumi}{\arabic{enumi}.}
% A(\Alph) a(\alph) I(\Roman) i(\roman) 1(\arabic)
%设定全局标号series=example	%引用全局变量resume=example
%[topsep=-0.3em,parsep=-0.3em,itemsep=-0.3em,partopsep=-0.3em]
%可使用leftmargin调整列表环境左边的空白长度 [leftmargin=0em]
\item
小球受到手的拉力大小$ F $;
\item 
物块和小球的质量之比$ M : m $;
\item 
小球向下运动到最低点时,物块$ M $所受的拉力大小$ T $。



\end{enumerate}
\begin{figure}[h!]
\flushright
\includesvg[width=0.25\linewidth]{picture/svg/772}
\end{figure}

\banswer{
\begin{enumerate}
\renewcommand{\labelenumi}{\arabic{enumi}.}
% A(\Alph) a(\alph) I(\Roman) i(\roman) 1(\arabic)
%设定全局标号series=example	%引用全局变量resume=example
%[topsep=-0.3em,parsep=-0.3em,itemsep=-0.3em,partopsep=-0.3em]
%可使用leftmargin调整列表环境左边的空白长度 [leftmargin=0em]
\item
$F = \frac { 5 } { 3 } M g - m g$
\item 
$\frac { M } { m } = \frac { 6 } { 5 }$
\item 
得$T = \frac { 8 m M g } { 5 ( m + M ) }$($T = \frac { 48 } { 55 } m g$或$T = \frac { 8 } { 11 } M g$)

\end{enumerate}


}



\newpage
\item 
\exwhere{$ 2016 $年新课标$ \lmd{3} $卷}
如图,在竖直平面内有由$ \frac{ 1 }{ 4 } $圆弧$ AB $和$ \frac{ 1 }{ 2 } $圆弧$ BC $组成的光滑固定轨道,两者在最低点$ B $平滑连接。$ AB $弧的半径为$ R $,$ BC $弧的半径为$ \frac{R}{2} $。一小球在$ A $点正上方与$ A $相距$ \frac{R}{4} $处由静止开始自由下落,经$ A $点沿圆弧轨道运动。
\begin{enumerate}
\renewcommand{\labelenumi}{\arabic{enumi}.}
% A(\Alph) a(\alph) I(\Roman) i(\roman) 1(\arabic)
%设定全局标号series=example	%引用全局变量resume=example
%[topsep=-0.3em,parsep=-0.3em,itemsep=-0.3em,partopsep=-0.3em]
%可使用leftmargin调整列表环境左边的空白长度 [leftmargin=0em]
\item
求小球在$ B $、$ A $两点的动能之比;
\item 
通过计算判断小球能否沿轨道运动到$ C $点。




\end{enumerate}
\begin{figure}[h!]
\flushright
\includesvg[width=0.2\linewidth]{picture/svg/773}
\end{figure}

\banswer{
\begin{enumerate}
\renewcommand{\labelenumi}{\arabic{enumi}.}
% A(\Alph) a(\alph) I(\Roman) i(\roman) 1(\arabic)
%设定全局标号series=example	%引用全局变量resume=example
%[topsep=-0.3em,parsep=-0.3em,itemsep=-0.3em,partopsep=-0.3em]
%可使用leftmargin调整列表环境左边的空白长度 [leftmargin=0em]
\item
$\frac { E _ { K 4 } } { E _ { K B } } = 5$
\item 
小球恰好可以沿轨道运动到C点


\end{enumerate}


}




\item 
\exwhere{$ 2016 $年江苏卷}
如图所示,倾角为$ \alpha $的斜面$ A $被固定在水平面上,细线的一端固定于墙面,另一端跨过斜面顶端的小滑轮与物块$ B $相连,$ B $静止在斜面上。滑轮左侧的细线水平,右侧的细线与斜面平行。$ A $、$ B $的质量均为$ m $.撤去固定$ A $的装置后,$ A $、$ B $均做直线运动。不计一切摩擦,重力加速度为$ g $。求:
\begin{enumerate}
\renewcommand{\labelenumii}{(\arabic{enumii})}
\item 
$ A $固定不动时,$ A $对$ B $支持力的大小$ N $;


\item 
$ A $滑动的位移为$ x $时,$ B $的位移大小$ s $;


\item 
$ A $滑动的位移为$ x $时的速度大小$ v_{A} $.

\end{enumerate}
\begin{figure}[h!]
\flushright
\includesvg[width=0.3\linewidth]{picture/svg/774}
\end{figure}


\banswer{
\begin{enumerate}
\renewcommand{\labelenumi}{\arabic{enumi}.}
% A(\Alph) a(\alph) I(\Roman) i(\roman) 1(\arabic)
%设定全局标号series=example	%引用全局变量resume=example
%[topsep=-0.3em,parsep=-0.3em,itemsep=-0.3em,partopsep=-0.3em]
%可使用leftmargin调整列表环境左边的空白长度 [leftmargin=0em]
\item
$ mg \cos \alpha $
\item 
$\sqrt { 2 ( 1 - \cos \alpha ) } \cdot x$
\item 
$\sqrt { \frac { 2 g x \sin \alpha } { 3 - 2 \cos \alpha } }$

\end{enumerate}


}




\newpage
\item 
\exwhere{$ 2017 $年新课标 \lmd{1} 卷}
一质量为$ 8.00 \times 10^4 $ $ kg $的太空飞船从其飞行轨道返回地面。飞船在离地面高度$ 1.60 \times 10^5 $ $ m $处以$ 7.5 \times 10^3 $ $ m/s $的速度进入大气层,逐渐减慢至速度为$ 100 $ $ m/s $时下落到地面。取地面为重力势能零点,在飞船下落过程中,重力加速度可视为常量,大小取为$ 9.8 $ $ m/s^{2} $。(结果保留$ 2 $位有效数字)
\begin{enumerate}
\renewcommand{\labelenumi}{\arabic{enumi}.}
% A(\Alph) a(\alph) I(\Roman) i(\roman) 1(\arabic)
%设定全局标号series=example	%引用全局变量resume=example
%[topsep=-0.3em,parsep=-0.3em,itemsep=-0.3em,partopsep=-0.3em]
%可使用leftmargin调整列表环境左边的空白长度 [leftmargin=0em]
\item
分别求出该飞船着地前瞬间的机械能和它进入大气层时的机械能;
\item 
求飞船从离地面高度$ 600 $ $ m $处至着地前瞬间的过程中克服阻力所做的功,已知飞船在该处的速度大小是其进入大气层时速度大小的$ 2.0 \% $。




\end{enumerate}


\banswer{
\begin{enumerate}
\renewcommand{\labelenumi}{\arabic{enumi}.}
% A(\Alph) a(\alph) I(\Roman) i(\roman) 1(\arabic)
%设定全局标号series=example	%引用全局变量resume=example
%[topsep=-0.3em,parsep=-0.3em,itemsep=-0.3em,partopsep=-0.3em]
%可使用leftmargin调整列表环境左边的空白长度 [leftmargin=0em]
\item
$ 4.0 \times 10^8 \ J $ \qquad $ 2.4 \times 10^{12} \ J $
\item 
$ 9.7 \times 10^{8} \ J $


\end{enumerate}


}











\end{enumerate}

