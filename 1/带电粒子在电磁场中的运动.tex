\bta{带电粒子在电磁场中的运动}



\begin{enumerate}[leftmargin=0em]
\renewcommand{\labelenumi}{\arabic{enumi}.}
% A(\Alph) a(\alph) I(\Roman) i(\roman) 1(\arabic)
%设定全局标号series=example	%引用全局变量resume=example
%[topsep=-0.3em,parsep=-0.3em,itemsep=-0.3em,partopsep=-0.3em]
%可使用leftmargin调整列表环境左边的空白长度 [leftmargin=0em]
\item
\exwhere{$ 2019 $年物理全国\lmd{1}卷}
如图,在直角三角形$ OPN $区域内存在匀强磁场,磁感应强度大小为$ B $、方向垂直于纸面向外。一带正电的粒子从静止开始经电压$ U $加速后,沿平行于$ x $辅的方向射入磁场;一段时间后,该粒子在$ OP $边上某点以垂直于$ x $轴的方向射出。已知$ O $点为坐标原点,$ N $点在$ y $轴上,$ OP $与$ x $轴的夹角为$ 30 ^{ \circ } $,粒子进入磁场的入射点与离开磁场的出射点之间的距离为$ d $,不计重力。求:
\begin{enumerate}
\renewcommand{\labelenumi}{\arabic{enumi}.}
% A(\Alph) a(\alph) I(\Roman) i(\roman) 1(\arabic)
%设定全局标号series=example	%引用全局变量resume=example
%[topsep=-0.3em,parsep=-0.3em,itemsep=-0.3em,partopsep=-0.3em]
%可使用leftmargin调整列表环境左边的空白长度 [leftmargin=0em]
\item
带电粒子的比荷;
\item 
带电粒子从射入磁场到运动至$ x $轴的时间。



\end{enumerate}
\begin{figure}[h!]
\flushright
\includesvg[width=0.25\linewidth]{picture/svg/239}
\end{figure}

\banswer{
\begin{enumerate}
\renewcommand{\labelenumi}{\arabic{enumi}.}
% A(\Alph) a(\alph) I(\Roman) i(\roman) 1(\arabic)
%设定全局标号series=example	%引用全局变量resume=example
%[topsep=-0.3em,parsep=-0.3em,itemsep=-0.3em,partopsep=-0.3em]
%可使用leftmargin调整列表环境左边的空白长度 [leftmargin=0em]
\item
$\frac { 4 U } { d ^ { 2 } B ^ { 2 } }$
\item 
$\left( \frac { \pi } { 8 } + \frac { \sqrt { 3 } } { 12 } \right) \cdot \frac { d ^ { 2 } B } { U }$或$\frac { B d ^ { 2 } } { 4 U } \left( \frac { \pi } { 2 } + \frac { \sqrt { 3 } } { 3 } \right)$


\end{enumerate}
}




\newpage
\item
\exwhere{$ 2017 $年天津卷}
平面直角坐标系$ xOy $中,第 \lmd{1} 象限存在垂直于平面向里的匀强磁场,第 \lmd{3} 现象存在沿$ y $轴负方向的匀强电场,如图所示。一带负电的粒子从电场中的$ Q $点以速度$ v_{0} $沿$ x $轴正方向开始运动,$ Q $点到$ y $轴的距离为到$ x $轴距离的$ 2 $倍。粒子从坐标原点$ O $离开电场进入磁场,最终从$ x $轴上的$ P $点射出磁场,$ P $点到$ y $轴距离与$ Q $点到$ y $轴距离相等。不计粒子重力,求:
\begin{enumerate}
\renewcommand{\labelenumi}{\arabic{enumi}.}
% A(\Alph) a(\alph) I(\Roman) i(\roman) 1(\arabic)
%设定全局标号series=example	%引用全局变量resume=example
%[topsep=-0.3em,parsep=-0.3em,itemsep=-0.3em,partopsep=-0.3em]
%可使用leftmargin调整列表环境左边的空白长度 [leftmargin=0em]
\item
粒子到达$ O $点时速度的大小和方向;
\item 
电场强度和磁感应强度的大小之比。


\end{enumerate}
\begin{figure}[h!]
\flushright
\includesvg[width=0.4\linewidth]{picture/svg/240}
\end{figure}
\banswer{
\begin{enumerate}
\renewcommand{\labelenumi}{\arabic{enumi}.}
% A(\Alph) a(\alph) I(\Roman) i(\roman) 1(\arabic)
%设定全局标号series=example	%引用全局变量resume=example
%[topsep=-0.3em,parsep=-0.3em,itemsep=-0.3em,partopsep=-0.3em]
%可使用leftmargin调整列表环境左边的空白长度 [leftmargin=0em]
\item
$v = \sqrt { 2 } v _ { 0 }$,方向与$ x $轴方向的夹角为$ 45 ^{\circ} $角斜向上;
\item 
$\frac { E } { B } = \frac { v _ { 0 } } { 2 }$



\end{enumerate}
}



\newpage
\item
\exwhere{$ 2014 $年物理海南卷}
如图,在$ x $轴上方存在匀强磁场,磁感应强度大小为$ B $,方向垂直于纸面向外;在$ x $轴下方存在匀强电场,电场方向与$ xoy $平面平行,且与$ x $轴成$ 45 ^{\circ} $夹角。一质量为$ m $、电荷量为$ q $($ q $>$ 0 $)的粒子以速度$ v_{0} $从$ y $轴上$ P $点沿$ y $轴正方向射出,一段时间后进入电场,进入电场时的速度方向与电场方向相反;又经过一段时间$ T_{0} $,磁场方向变为垂直纸面向里,大小不变,不计重力。
\begin{enumerate}
\renewcommand{\labelenumi}{\arabic{enumi}.}
% A(\Alph) a(\alph) I(\Roman) i(\roman) 1(\arabic)
%设定全局标号series=example	%引用全局变量resume=example
%[topsep=-0.3em,parsep=-0.3em,itemsep=-0.3em,partopsep=-0.3em]
%可使用leftmargin调整列表环境左边的空白长度 [leftmargin=0em]
\item
求粒子从$ P $点出发至第一次到达$ x $轴时所需的时间;

\item 
若要使粒子能够回到$ P $点,求电场强度的最大值。

\end{enumerate}
\begin{figure}[h!]
\flushright
\includesvg[width=0.37\linewidth]{picture/svg/241}
\end{figure}

\banswer{
\begin{enumerate}
\renewcommand{\labelenumi}{\arabic{enumi}.}
% A(\Alph) a(\alph) I(\Roman) i(\roman) 1(\arabic)
%设定全局标号series=example	%引用全局变量resume=example
%[topsep=-0.3em,parsep=-0.3em,itemsep=-0.3em,partopsep=-0.3em]
%可使用leftmargin调整列表环境左边的空白长度 [leftmargin=0em]
\item
$t _ { 1 } = \frac { 5 \pi m } { 4 q B }$
\item 
电场强度最大值为$E = \frac { 2 m v _ { 0 } } { q T _ { 0 } }$



\end{enumerate}
}



\newpage
\item
\exwhere{$ 2014 $年理综大纲卷}
如图,在第一象限存在匀强磁场,磁感应强度方向垂直于纸面$ (xy $平面)向外;在第四象限存在匀强电场,方向沿$ x $轴负向。在$ y $轴正半轴上某点以与$ x $轴正向平行、大小为$ v_{0} $的速度发射出一带正电荷的粒子,该粒子在$ (d $,$ 0) $点沿垂直于$ x $轴的方向进人电场。不计重力。若该粒子离开电场时速度方向与$ y $轴负方向的夹角为$ \theta $,求:
\begin{enumerate}
\renewcommand{\labelenumi}{\arabic{enumi}.}
% A(\Alph) a(\alph) I(\Roman) i(\roman) 1(\arabic)
%设定全局标号series=example	%引用全局变量resume=example
%[topsep=-0.3em,parsep=-0.3em,itemsep=-0.3em,partopsep=-0.3em]
%可使用leftmargin调整列表环境左边的空白长度 [leftmargin=0em]
\item
电场强度大小与磁感应强度大小的比值;
\item 
该粒子在电场中运动的时间。


\end{enumerate}
\begin{figure}[h!]
\flushright
\includesvg[width=0.29\linewidth]{picture/svg/242}
\end{figure}

\banswer{
\begin{enumerate}
\renewcommand{\labelenumi}{\arabic{enumi}.}
% A(\Alph) a(\alph) I(\Roman) i(\roman) 1(\arabic)
%设定全局标号series=example	%引用全局变量resume=example
%[topsep=-0.3em,parsep=-0.3em,itemsep=-0.3em,partopsep=-0.3em]
%可使用leftmargin调整列表环境左边的空白长度 [leftmargin=0em]
\item
$\frac { E } { B } = \frac { 1 } { 2 } v _ { 0 } \tan ^ { 2 } \theta$
\item 
$t=\frac { 2 d } { v _ { 0 } \tan \theta }$



\end{enumerate}
}



\newpage
\item
\exwhere{$ 2011 $年理综全国卷}
$ 25 $.($ 19 $分)如图,与水平面成$ 45 ^{ \circ } $角的平面$ MN $将空间分成$ I $和$ II $两个区域。一质量为$ m $、电荷量为$ q $($ q>0 $)的粒子以速度$ v_{0} $从平面$ MN $上的$ P_{0} $点水平向右射入 \lmd{1} 区。粒子在 \lmd{1} 区运动时,只受到大小不变、方向竖直向下的电场作用,电场强度大小为$ E $;在 \lmd{2} 区运动时,只受到匀强磁场的作用,磁感应强度大小为$ B $,方向垂直于纸面向里。求粒子首次从 \lmd{2} 区离开时到出发点$ P_{0} $的距离。粒子的重力可以忽略。

\begin{figure}[h!]
\flushright
\includesvg[width=0.25\linewidth]{picture/svg/243}
\end{figure}

\banswer{
$l = \frac { \sqrt { 2 } m v _ { 0 } } { q } \left( \frac { 2 v _ { 0 } } { E } + \frac { 1 } { B } \right)$
}


\newpage
\item
\exwhere{$ 2013 $年四川卷}
如图所示,竖直平面(纸面)内有直角坐标系$ xOy $,$ x $轴沿水平方向。在$ x \leq 0 $的区域内存在方向垂直于纸面向里,磁感应强度大小为$ B_{1} $的匀强磁场。在第二象限紧贴$ y $轴固定放置长为$ l $、表面粗糙的不带电绝缘平板,平板平行于$ x $轴且与$ x $轴相距$ h $。在第一象限内的某区域存在方向相互垂直的匀强磁场(磁感应强度大小为$ B_{2} $、方向垂直于纸面向外)和匀强电场(图中未画出)。一质量为$ m $、不带电的小球$ Q $从平板下侧$ A $点沿$ x $轴正向抛出;另一质量也为$ m $、带电量为$ q $的小球$ P $从$ A $点紧贴平板沿$ x $轴正向运动,变为匀速运动后从$ y $轴上的$ D $点进入电磁场区域做匀速圆周运动,经$ 1/4 $圆周离开电磁场区域,沿$ y $轴负方向运动,然后从$ x $轴上的$ K $点进入第四象限。小球$ P $、$ Q $相遇在第四象限的某一点,且竖直方向速度相同。设运动过程中小球$ P $电量不变,小球$ P $和$ Q $始终在纸面内运动且均看作质点,重力加速度为$ g $。求:
\begin{enumerate}
\renewcommand{\labelenumi}{\arabic{enumi}.}
% A(\Alph) a(\alph) I(\Roman) i(\roman) 1(\arabic)
%设定全局标号series=example	%引用全局变量resume=example
%[topsep=-0.3em,parsep=-0.3em,itemsep=-0.3em,partopsep=-0.3em]
%可使用leftmargin调整列表环境左边的空白长度 [leftmargin=0em]
\item
匀强电场的场强大小,并判断$ P $球所带电荷的正负;
\item 
小球$ Q $的抛出速度$ v_{0} $的取值范围;
\item 
$ B_{1} $是$ B_{2} $的多少倍?





\end{enumerate}
\begin{figure}[h!]
\flushright
\includesvg[width=0.4\linewidth]{picture/svg/244}
\end{figure}


\banswer{
\begin{enumerate}
\renewcommand{\labelenumi}{\arabic{enumi}.}
% A(\Alph) a(\alph) I(\Roman) i(\roman) 1(\arabic)
%设定全局标号series=example	%引用全局变量resume=example
%[topsep=-0.3em,parsep=-0.3em,itemsep=-0.3em,partopsep=-0.3em]
%可使用leftmargin调整列表环境左边的空白长度 [leftmargin=0em]
\item
P带正电,$E = \frac { m g } { q }$
\item 
$0 < v _ { 0 } \leq \left( \frac { m ^ { 2 } g } { q ^ { 2 } B _ { 1 } B _ { 2 } } + l \right) \frac { \sqrt { 2 g h } } { 2 h }$
\item 
$B _ { 1 } = 0.5 B _ { 2 }$



\end{enumerate}
}



\newpage
\item
\exwhere{$ 2014 $年理综重庆卷}
如图所示,在无限长的竖直边界$ NS $和$ MT $间充满匀强电场,同时该区域上、下部分分别充满方向垂直于$ NSTM $平面向外和向内的匀强磁场,磁感应强度大小分别为$ B $和$ 2B $,$ KL $为上下磁场的水平分界线,在$ NS $和$ MT $边界上,距$ KL $高$ h $处分别有$ P $、$ Q $两点,$ NS $和$ MT $间距为$ 1.8h $。质量为$ m $、带电量为$ +q $的粒子从$ P $点垂直于$ NS $边界射入该区域,在两边界之间做圆周运动,重力加速度为$ g $。
\begin{enumerate}
\renewcommand{\labelenumi}{\arabic{enumi}.}
% A(\Alph) a(\alph) I(\Roman) i(\roman) 1(\arabic)
%设定全局标号series=example	%引用全局变量resume=example
%[topsep=-0.3em,parsep=-0.3em,itemsep=-0.3em,partopsep=-0.3em]
%可使用leftmargin调整列表环境左边的空白长度 [leftmargin=0em]
\item
求该电场强度的大小和方向。
\item 
要使粒子不从$ NS $边界飞出,求粒子入射速度的最小值。
\item 
若粒子能经过$ Q $点从$ MT $边界飞出,求粒子入射速度的所有可能值。



\end{enumerate}
\begin{figure}[h!]
\flushright
\includesvg[width=0.25\linewidth]{picture/svg/245}
\end{figure}


\banswer{
\begin{enumerate}
\renewcommand{\labelenumi}{\arabic{enumi}.}
% A(\Alph) a(\alph) I(\Roman) i(\roman) 1(\arabic)
%设定全局标号series=example	%引用全局变量resume=example
%[topsep=-0.3em,parsep=-0.3em,itemsep=-0.3em,partopsep=-0.3em]
%可使用leftmargin调整列表环境左边的空白长度 [leftmargin=0em]
\item
方向向上,$ E=\frac{mg}{q} $
\item 
$v _ { \min } = ( 9 - 6 \sqrt { 2 } ) \frac { B q h } { m }$
\item 
可能的速度有三个:$\frac { 0.68 B q h } { m } , \frac { 0.545 B q h } { m } , \frac { 0.52 B q h } { m }$


\end{enumerate}
}


\newpage
\item
\exwhere{$ 2015 $年理综天津卷}
现代科学仪器常利用电场、磁场控制带电粒子的运动。真空中存在着如图所示的多层紧密相邻的匀强电场和匀强磁场,电场和磁场的宽度均为$ d $。电场强度为$ E $,方向水平向右;磁感应强度为$ B $,方向垂直纸面向里。电场、磁场的边界互相平行且与电场方向垂直,一个质量为$ m $、电荷量为$ q $的带正电粒子在第$ 1 $层电场左侧边界某处由静止释放,粒子始终在电场、磁场中运动,不计粒子重力及运动时的电磁辐射.
\begin{enumerate}
\renewcommand{\labelenumi}{\arabic{enumi}.}
% A(\Alph) a(\alph) I(\Roman) i(\roman) 1(\arabic)
%设定全局标号series=example	%引用全局变量resume=example
%[topsep=-0.3em,parsep=-0.3em,itemsep=-0.3em,partopsep=-0.3em]
%可使用leftmargin调整列表环境左边的空白长度 [leftmargin=0em]
\item
求粒子在第$ 2 $层磁场中运动时速度$ v_{2} $的大小与轨迹半径$ r_{2} $;
\item 
粒子从第$ n $层磁场右侧边界穿出时,速度的方向与水平方向的夹角为$ \theta _n $,试求$ \sin \theta _n $;
\item 
若粒子恰好不能从第$ n $层磁场右侧边界穿出,试问在其他条件不变的情况下,也进入第$ n $层磁场,但比荷较该粒子大的粒子能否穿出该层磁场右侧边界,请简要推理说明之。




\end{enumerate}
\begin{figure}[h!]
\flushright
\includesvg[width=0.42\linewidth]{picture/svg/246}
\end{figure}

\banswer{
\begin{enumerate}
\renewcommand{\labelenumi}{\arabic{enumi}.}
% A(\Alph) a(\alph) I(\Roman) i(\roman) 1(\arabic)
%设定全局标号series=example	%引用全局变量resume=example
%[topsep=-0.3em,parsep=-0.3em,itemsep=-0.3em,partopsep=-0.3em]
%可使用leftmargin调整列表环境左边的空白长度 [leftmargin=0em]
\item
$r _ { 2 } = \frac { 2 } { B } \sqrt { \frac { m E d } { q } }$
\item 
$\sin \theta _ { n } = B \sqrt { \frac { n q d } { 2 m E } }$
\item 
若粒子恰好不能从第n层磁场右侧边界穿出,则:$\theta _ { n } = \frac { \pi } { 2 } , \quad \sin \theta _ { n } = 1$,在其他条件不变的情况下,换用比荷更大的粒子,设其比荷为$\frac { q ^ { \prime } } { m ^ { \prime } }$,假设能穿出第n层磁场右侧边界,粒子穿出时速度方向与水平方向的夹角为$\theta _ { n } ^ { \prime }$,由于$\frac { q ^ { \prime } } { m ^ { \prime } } = \frac { q } { m }$,则导致:$\sin \theta _ { n } ^ { \prime } > 1$ 
说明$\theta _ { n } ^ { \prime }$不存在,即原假设不成立,所以比荷较该粒子大的粒子不能穿出该层磁场右侧边界。



\end{enumerate}
}




\newpage
\item 
\exwhere{$ 2015 $年理综重庆卷}
下图为某种离子加速器的设计方案。两个半圆形金属盒内存在相同的垂直于纸面向外的匀强磁场。其中$ MN $和$ M ^{\prime} N ^{\prime} $是间距为$ h $的两平行极板,其上分别有正对的两个小孔$ O $和$ O ^{\prime} $,
$O ^ { \prime } N ^ { \prime } = O N = d$,$ P $为靶点,$ O ^{\prime} P=kd $($ k $为大于$ 1 $的整数).极板间存在方向向上的匀强电场,两极板间电压为$ U $. 质量为$ m $、带电量为$ q $的正离子从$ O $点由静止开始加速,经$ O $' 进入磁场区域。当离子打到极板上$ O ^{\prime} N ^{\prime} $区域(含$ N ^{\prime} $点)或外壳上时将会被吸收。两虚线之间的区域无电场和磁场存在,离子可匀速穿过。忽略相对论效应和离子所受的重力。求:
\begin{enumerate}
\renewcommand{\labelenumi}{\arabic{enumi}.}
% A(\Alph) a(\alph) I(\Roman) i(\roman) 1(\arabic)
%设定全局标号series=example	%引用全局变量resume=example
%[topsep=-0.3em,parsep=-0.3em,itemsep=-0.3em,partopsep=-0.3em]
%可使用leftmargin调整列表环境左边的空白长度 [leftmargin=0em]
\item
离子经过电场仅加速一次后能打到$ P $点所需的磁感应强度大小;
\item 
能使离子打到$ P $点的磁感应强度的所有可能值;
\item 
打到$ P $点的能量最大的离子在磁场中运动的时间和在电场中运动的时间。





\end{enumerate}
\begin{figure}[h!]
\flushright
\includesvg[width=0.28\linewidth]{picture/svg/247}
\end{figure}

\banswer{
\begin{enumerate}
\renewcommand{\labelenumi}{\arabic{enumi}.}
% A(\Alph) a(\alph) I(\Roman) i(\roman) 1(\arabic)
%设定全局标号series=example	%引用全局变量resume=example
%[topsep=-0.3em,parsep=-0.3em,itemsep=-0.3em,partopsep=-0.3em]
%可使用leftmargin调整列表环境左边的空白长度 [leftmargin=0em]
\item
$B = \frac { 2 \sqrt { 2 q U m } } { q k d }$
\item 
$B = \frac { 2 \sqrt { 2 n q U m } } { q k d } , \quad \left( n = 1,2,3 , \cdots , k ^ { 2 } - 1 \right)$
\item 
$t_{\text{磁}}= \frac { \left( 2 k ^ { 2 } - 3 \right) \pi m k d } { 2 \sqrt { 2 q m U \left( k ^ { 2 } - 1 \right) } }$ , $t_{\text{电}}= h \sqrt { \frac { 2 \left( k ^ { 2 } - 1 \right) m } { q U } }$



\end{enumerate}
}




\newpage
\item
\exwhere{$ 2016 $年江苏卷}
回旋加速器的工作原理如图$ 1 $所示,置于真空中的$ D $形金属盒半径为$ R $,两盒间狭缝的间距为$ d $,磁感应强度为$ B $的匀强磁场与盒面垂直,被加速粒子的质量为$ m $,电荷量为$ +q $,加在狭缝间的交变电压如图$ 2 $所示,电压值的大小为$ U_{0} $.周期$T = \frac { 2 \pi m } { q B }$。一束该种粒子在$t = 0 \sim \frac { T } { 2 }$时间内从$ A $处均匀地飘入狭缝,其初速度视为零。现考虑粒子在狭缝中的运动时间,假设能够出射的粒子每次经过狭缝均做加速运动,不考虑粒子间的相互作用。求: 
\begin{enumerate}
\renewcommand{\labelenumii}{(\arabic{enumii})}

\item 
出射粒子的动能$ E_m $;

\item 
粒子从飘入狭缝至动能达到$ E_m $所需的总时间$ t_{0} $;

\item 
要使飘入狭缝的粒子中有超过$ 99 \% $能射出,$ d $应满足的条件。


\end{enumerate}
\begin{figure}[h!]
\flushright
\includesvg[width=0.55\linewidth]{picture/svg/248}
\end{figure}

\banswer{
\begin{enumerate}
\renewcommand{\labelenumi}{\arabic{enumi}.}
% A(\Alph) a(\alph) I(\Roman) i(\roman) 1(\arabic)
%设定全局标号series=example	%引用全局变量resume=example
%[topsep=-0.3em,parsep=-0.3em,itemsep=-0.3em,partopsep=-0.3em]
%可使用leftmargin调整列表环境左边的空白长度 [leftmargin=0em]
\item
$\frac { q ^ { 2 } B ^ { 2 } R ^ { 2 } } { 2 m }$
\item 
$\frac { \pi B R ^ { 2 } + 2 B R d } { 2 U _ { 0 } } - \frac { \pi m } { q B }$
\item 
$d < \frac { \pi m U _ { 0 } } { 100 q B ^ { 2 } R }$



\end{enumerate}
}


\newpage
\item
\exwhere{$ 2016 $年四川卷}
如图所示,图面内有竖直线$ DD ^{\prime} $,过$ DD ^{\prime} $且垂直于图面的平面将空间分成 \lmd{1} 、 \lmd{2} 两区域。区域 \lmd{1} 有方向竖直向上的匀强电场和方向垂直图面的匀强磁场$ B $(图中未画出);区域 \lmd{2} 有固定在水平面上高$ h=2l $、倾角$ \alpha=\frac{\pi}{4} $的光滑绝缘斜面,斜面顶端与直线$ DD ^{\prime} $距离$ s=4l $,区域 \lmd{2} 可加竖直方向的大小不同的匀强电场(图中未画出);$ C $点在$ DD ^{\prime} $上,距地面高$ h=3l $。零时刻,质量为$ m $、带电荷量为$ q $的小球$ P $在$ K $点具有大小$v _ { 0 } = \sqrt { g l }$、方向与水平面夹角$ \theta=\frac{\pi}{3} $的速度,在区域 \lmd{1} 内做半径$ r=\frac{3l}{\pi} $的匀速圆周运动,经$ C $点水平进入区域 \lmd{2} 。某时刻,不带电的绝缘小球$ A $由斜面顶端静止释放,在某处与刚运动到斜面的小球$ P $相遇。小球视为质点,不计空气阻力及小球$ P $所带电量对空间电磁场的影响。$ l $已知,$ g $为重力加速度。
\begin{enumerate}
\renewcommand{\labelenumi}{\arabic{enumi}.}
% A(\Alph) a(\alph) I(\Roman) i(\roman) 1(\arabic)
%设定全局标号series=example	%引用全局变量resume=example
%[topsep=-0.3em,parsep=-0.3em,itemsep=-0.3em,partopsep=-0.3em]
%可使用leftmargin调整列表环境左边的空白长度 [leftmargin=0em]
\item
求匀强磁场的磁感应强度$ B $的大小;
\item 
若小球$ A $、$ P $在斜面底端相遇,求释放小球$ A $的时刻$ t_{A} $;
\item 
若小球$ A $、$ P $在时刻$t = \beta \sqrt { \frac{l}{g} }$($ \beta $为常数)相遇于斜面某处,求此情况下区域 \lmd{2} 的匀强电场的场强$ E $,并讨论场强$ E $的极大值和极小值及相应的方向。




\end{enumerate}
\begin{figure}[h!]
\flushright
\includesvg[width=0.4\linewidth]{picture/svg/249}
\end{figure}

\banswer{
\begin{enumerate}
\renewcommand{\labelenumi}{\arabic{enumi}.}
% A(\Alph) a(\alph) I(\Roman) i(\roman) 1(\arabic)
%设定全局标号series=example	%引用全局变量resume=example
%[topsep=-0.3em,parsep=-0.3em,itemsep=-0.3em,partopsep=-0.3em]
%可使用leftmargin调整列表环境左边的空白长度 [leftmargin=0em]
\item
$B = \frac { \pi m } { 3 q } \sqrt { \frac { g } { l } }$
\item 
$( 3 - 2 \sqrt { 2 } ) \sqrt { \frac { l } { g } }$
\item 
$E = \frac { \left( \beta ^ { 2 - 11 } \right) m g } { q ( \beta - 1 )^{2} }$,方向向上.\\
$3 \leq \beta \leq 5$,得$E _ { \min } = 0$,$E _ { \max } = \frac { 7 m g } { 8 q }$,方向竖直向上。



\end{enumerate}
}




\end{enumerate}







