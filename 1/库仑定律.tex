\bta{第二讲$ \quad $库仑定律}


\begin{enumerate}[leftmargin=0em]
\renewcommand{\labelenumi}{\arabic{enumi}.}
% A(\Alph) a(\alph) I(\Roman) i(\roman) 1(\arabic)
%设定全局标号series=example	%引用全局变量resume=example
%[topsep=-0.3em,parsep=-0.3em,itemsep=-0.3em,partopsep=-0.3em]
%可使用leftmargin调整列表环境左边的空白长度 [leftmargin=0em]
\item
\exwhere{$ 2012 $年物理江苏卷}
真空中,A、B两点与点电荷Q的距离分别为r和3r,则A、B两点的电场强度大小之比为 \xzanswer{C} 
\fourchoices
{$ 3:1 $}
{$ 1:3 $}
{$ 9:1 $}
{$ 1:9 $}


\item
\exwhere{2012年物理上海卷}
$ A $、$ B $、$ C $三点在同一直线上,$ AB:BC=1:2 $,$ B $点位于$ A $、$ C $之间,在$ B $处固定一电荷量为$ Q $的点电荷。当在$ A $处放一电荷量为+$ q $的点电荷时,它所受到的电场力为$ F $;移去$ A $处电荷,在$ C $处放电荷量为$ -2q $的点电荷,其所受电场力为 \xzanswer{B} 
\fourchoices
{$ -\frac{F}{2} $}
{$ \frac{F}{2} $}
{$ -F $}
{$ F $}


\item
\exwhere{$ 2011 $年海南卷}
三个相同的金属小球$ 1 $、$ 2 $、$ 3 $分别置于绝缘支架上,各球之间的距离远大于小球的直径。球$ 1 $的带电量为$ q $,球$ 2 $的带电量为$ nq $,球$ 3 $不带电且离球$ 1 $和球$ 2 $很远,此时球$ 1 $、$ 2 $之间作用力的大小为$ F $。现使球$ 3 $先与球$ 2 $接触,再与球$ 1 $接触,然后将球$ 3 $移至远处,此时$ 1 $、$ 2 $之间作用力的大小仍为$ F $,方向不变。由此可知 \xzanswer{D} 

\fourchoices
{$ n=3 $}
{$ n=4 $}
{$ n=5 $}
{$ n=6 $}





\item
\exwhere{$ 2015 $年理综安徽卷}
由库仑定律可知,真空中两个静止的点电荷,带电量分别为$ q_{1} $和$ q_{2} $,其间距离为$ r $时,它们之间相互作用力的大小为$F = k \frac { q _ { 1 } q _ { 2 } } { r ^ { 2 } }$,式中$ k $为静电力常量。若用国际单位制的基本单位表示,$ k $的单位应为 \xzanswer{B} 
\fourchoices
{$\mathrm { kg } \cdot \mathrm { A } ^ { 2 } \cdot \mathrm { m } ^ { 2 }$}
{$\mathrm { kg } \cdot \mathrm { A } ^ { - 2 } \cdot \mathrm { m } ^ { 3 } \cdot \mathrm { s } ^ { - 4 }$}
{$\mathrm { kg } \cdot \mathrm { m } ^ { 2 } \cdot \mathrm { C } ^ { - 2 }$}
{$\mathrm { N } \cdot \mathrm { m } ^ { 2 } \cdot \mathrm { A } ^ { - 2 }$}





\item
\exwhere{$ 2015 $年理综安徽卷}
图示是$ \alpha $粒子(氦原子核)被重金属原子核散射的运动轨迹,$ M $、$ N $、$ P $、$ Q $是轨迹上的四点,在散射过程中可以认为重金属原子核静止不动。图中所标出的$ \alpha $粒子在各点处的加速度方向正确的是 \xzanswer{C} 
\begin{figure}[h!]
\centering
\includesvg[width=0.33\linewidth]{picture/svg/001}
\end{figure}


\fourchoices
{$ M $}
{$ N $}
{$ P $}
{$ Q $}





\item
\exwhere{$ 2018 $年浙江卷($ 4 $月选考)}
真空中两个完全相同、带等量同种电荷的金属小球$ A $和$ B $(可视为点电荷),分别固定在两处,它们之间的静电力为$ F $。用一个不带电的同样金属球$ C $先后与$ A $、$ B $球接触,然后移开球$ C $,此时$ A $、$ B $球间的静电力为 \xzanswer{C} 
\fourchoices
{$ \frac{F}{3} $}
{$ \frac{F}{4} $}
{$ \frac{3F}{8} $}
{$ \frac{F}{2} $}






\item
\exwhere{$ 2018 $年全国卷\lmd{1}}
如图,三个固定的带电小球$ a $、$ b $和$ c $,相互间的距离分别为$ ab=5 \ cm $,,$ bc=3 \ cm $,$ ca=4 \ cm $。小球$ c $所受库仑力的合力的方向平行于$ a $、$ b $的连线。设小球$ a $、$ b $所带电荷量的比值的绝对值为$ k $,则 \xzanswer{D} 
\begin{figure}[h!]
\centering
\includesvg[width=0.24\linewidth]{picture/svg/002}
\end{figure}

\fourchoices
{$ a $、$ b $的电荷同号,$k = \frac { 16 } { 9 }$}
{$ a $、$ b $的电荷异号,$k = \frac { 16 } { 9 }$}
{$ a $、$ b $的电荷同号,$k = \frac { 64 } { 27 }$}
{$ a $、$ b $的电荷异号,$k = \frac { 64 } { 27 }$}





\item
\exwhere{$ 2014 $年物理上海卷}
如图,竖直绝缘墙上固定一带电小球$ A $,将带电小球$ B $用轻质绝缘丝线悬挂在$ A $的正上方$ C $处,图中$ AC=h $。当$ B $静止在与竖直方向夹角$ \theta =30 ^{ \circ } $方向时,$ A $对$ B $的静电力为$ B $所受重力的$\frac { \sqrt { 3 } } { 3 }$倍,则丝线$ BC $长度为\tk{$\frac { \sqrt { 3 } } { 3 } h $或$ \frac { 2 \sqrt { 3 } } { 3 } h$}。若$ A $对$ B $的静电力为$ B $所受重力的$ 0.5 $倍,改变丝线长度,使$ B $仍能在$ \theta =30 ^{ \circ } $处平衡。以后由于$ A $漏电,$ B $在竖直平面内缓慢运动,到$ \theta =0 ^{ \circ } $处$ A $的电荷尚未漏完,在整个漏电过程中,丝线上拉力大小的变化情况是\tk{先不变后增大}。
\begin{figure}[h!]
\centering
\includesvg[width=0.19\linewidth]{picture/svg/003}
\end{figure}




\item
\exwhere{$ 2014 $年理综浙江卷}
如图所示,水平地面上固定一个光滑绝缘斜面,斜面与水平面的夹角为$ \theta $。一根轻质绝缘细线的一端固定在斜面顶端,另一端系有一个带电小球$ A $,细线与斜面平行。小球$ A $的质量为$ m $、电量为$ q $。小球$ A $的右侧固定放置带等量同种电荷的小球$ B $,两球心的高度相同、间距为$ d $。静电力常量为$ k $,重力加速度为$ g $,两带电小球可视为点电荷。小球$ A $静止在斜面上,则 \xzanswer{AC} 
\begin{figure}[h!]
\centering
\includesvg[width=0.29\linewidth]{picture/svg/004}
\end{figure}

\fourchoices
{小球$ A $与$ B $之间库仑力的大小为$k \frac { q ^ { 2 } } { d ^ { 2 } }$}
{当$\frac { q } { d } = \sqrt { \frac { m g \sin \theta } { k } }$时,细线上的拉力为$ 0 $}
{当$\frac { q } { d } = \sqrt { \frac { m g \tan \theta } { k } }$时,细线上的拉力为$ 0 $}
{当$\frac { q } { d } = \sqrt { \frac { m g } { k \tan \theta } }$时,斜面对小球$ A $的支持力为$ 0 $}






\item
\exwhere{$ 2015 $年理综浙江卷}
如图所示,用两根长度相同的绝缘细线把一个质量为$ 0.1 \ kg $的小球$ A $悬挂到水平板的$ M $、$ N $两点, $ A $上带有$Q = 3.0 \times 10 ^ { - 6 } \mathrm { C }$的正电荷。两线夹角为$ 120 ^{ \circ } $,两线上的拉力大小分别为$ F_1 $和$ F_2 $。$ A $的正下方$ 0.3\ m $处放有一带等量异种电荷的小球$ B $,$ B $与绝缘支架的总质量为$ 0.2 \ kg $(重力加速度取$ g=10 \ \ m/s ^{2} $;静电力常量$k = 9.0 \times 10 ^ { 9 } \mathrm { N } \cdot \mathrm { m } ^ { 2 } / \mathrm { C } ^ { 2 }$,$ A $、$ B $球可视为点电荷),则 \xzanswer{BC} 
\begin{figure}[h!]
\centering
\includesvg[width=0.29\linewidth]{picture/svg/005}
\end{figure}

\fourchoices
{支架对地面的压力大小为$ 2.0 \ N $}
{两线上的拉力大小$ F_1=F_2=1.9 \ N $}
{将$ B $水平右移,使$ M $、$ A $、$ B $在同一直线上,此时两线上的拉力大小$ F_1 $ $ = $ $ 1.225 \ N $, $ F_2=1.0 \ N $}
{将$ B $移到无穷远处,两线上的拉力大小$ F_1=F_2=0.866 \ N $}







\end{enumerate}




