\bta{洛伦兹力}
\begin{enumerate}
\renewcommand{\labelenumi}{\arabic{enumi}.}
% A(\Alph) a(\alph) I(\Roman) i(\roman) 1(\arabic)
%设定全局标号series=example	%引用全局变量resume=example
%[topsep=-0.3em,parsep=-0.3em,itemsep=-0.3em,partopsep=-0.3em]
%可使用leftmargin调整列表环境左边的空白长度 [leftmargin=0em]
\item
\exwhere{$ 2015 $年海南卷}
如图,$ a $是竖直平面$ P $上的一点,$ P $前有一条形磁铁垂直于$ P $,且$ S $极朝向$ a $点,$ P $后一电子在偏转线圈和条形磁铁的磁场的共同作用下,在水平面内向右弯曲经过$ a $点。在电子经过$ a $点的瞬间。条形磁铁的磁场对该电子的作用力的方向 \xzanswer{A} 
\begin{figure}[h!]
\centering
\includesvg[width=0.23\linewidth]{picture/svg/185}
\end{figure}


\fourchoices
{向上}
{向下}
{向左}
{向右}





\item
\exwhere{$ 2013 $年安徽卷}
图中$ a $,$ b $,$ c $,$ d $为四根与纸面垂直的长直导线,其横截面位于正方形的四个顶点上,导线中通有大小相同的电流,方向如图所示。一带正电的粒子从正方形中心$ O $点沿垂直于纸面的方向向外运动,它所受洛伦兹力的方向是 \xzanswer{B} 

\begin{minipage}[h!]{0.7\linewidth}
\vspace{0.3em}
\fourchoices
{向上}
{向下}
{向左}
{向右}
\vspace{0.3em}
\end{minipage}
\hfill
\begin{minipage}[h!]{0.3\linewidth}
\flushright
\vspace{0.3em}
\includesvg[width=0.7\linewidth]{picture/svg/186}
\vspace{0.3em}
\end{minipage}





\item
\exwhere{$ 2011 $年海南卷}
空间存在方向垂直于纸面向里的匀强磁场,图中的正方形为其边界。一细束由两种粒子组成的粒子流沿垂直于磁场的方向从$ O $点入射。这两种粒子带同种电荷,它们的电荷量、质量均不同,但其比荷相同,且都包含不同速率的粒子。不计重力。下列说法正确的是 \xzanswer{BD} 
\begin{figure}[h!]
\centering
\includesvg[width=0.23\linewidth]{picture/svg/187}
\end{figure}


\fourchoices
{入射速度不同的粒子在磁场中的运动时间一定不同}
{入射速度相同的粒子在磁场中的运动轨迹一定相同}
{在磁场中运动时间相同的粒子,其运动轨迹一定相同}
{在磁场中运动时间越长的粒子,其轨迹所对的圆心角一定越大}





\item
\exwhere{$ 2014 $年理综新课标\lmd{1}卷}
如图,$ MN $为铝质薄平板,铝板上方和下方分别有垂直于图平面的匀强磁场(未画出)。一带电粒子从紧贴铝板上表面的$ P $点垂直于铝板向上射出,从$ Q $点穿过铝板后到达$ PQ $的中点$ O $,已知粒子穿越铝板时,其动能损失一半,速度方向和电荷量不变,不计重力。铝板上方和下方的磁感应强度大小之比为 \xzanswer{D} 
\begin{figure}[h!]
\centering
\includesvg[width=0.23\linewidth]{picture/svg/188}
\end{figure}


\fourchoices
{2}
{$\sqrt{2}$}
{$ 1 $}
{$ \frac{\sqrt{2}}{2} $}






\item
\exwhere{$ 2011 $年理综浙江卷}
利用如图所示装置可以选择一定速度范围内的带电粒子。图中板$ MN $上方是磁感应强度大小为$ B $、方向垂直纸面向里的匀强磁场,板上有两条宽度分别为$ 2d $和$ d $的缝,两缝近端相距为$ L $。一群质量为$ m $、电荷量为$ q $,具有不同速度的的粒子从宽度为$ 2d $的缝垂直于板$ MN $进入磁场,对于能够从宽度$ d $的缝射出的粒子,下列说法正确的是 \xzanswer{BC} 
\begin{figure}[h!]
\centering
\includesvg[width=0.27\linewidth]{picture/svg/189}
\end{figure}



\fourchoices
{粒子带正电}
{射出粒子的最大速度为$\frac { q B ( L + 3 d ) } { 2 m }$}
{保持$ d $和$ L $不变,增大$ B $,射出粒子的最大速度与最小速度之差增大}
{保持$ d $和$ B $不变,增大$ L $,射出粒子的最大速度与最小速度之差增大}



\item
\exwhere{$ 2014 $年理综新课标\lmd{2}卷}
图为某磁谱仪部分构件的示意图。图中,永磁铁提供匀强磁场,硅微条径迹探测器可以探测粒子在其中运动的轨迹。宇宙射线中有大量的电子、正电子和质子。当这些粒子从上部垂直进入磁场时,下列说法正确的是: \xzanswer{AC} 
\begin{figure}[h!]
\centering
\includesvg[width=0.23\linewidth]{picture/svg/190}
\end{figure}


\fourchoices
{电子与正电子的偏转方向一定不同}
{电子和正电子在磁场中的运动轨迹一定相同}
{仅依据粒子的运动轨迹无法判断此粒子是质子还是正电子}
{粒子的动能越大,它在磁场中运动轨迹的半径越小}



\newpage
\item
\exwhere{$ 2017 $年北京卷}
发电机和电动机具有装置上的类似性,源于它们机理上的类似性。直流发电机和直流电动机的工作原理可以简化为如图$ 1 $、图$ 2 $所示的情景。在竖直向下的磁感应强度为$ B $的匀强磁场中,两根光滑平行金属轨道$ MN $、$ PQ $固定在水平面内,相距为$ L $,电阻不计。电阻为$ R $的金属导体棒$ ab $垂直于$ MN $、$ PQ $放在轨道上,与轨道接触良好,以速度$ v $($ v $平行于$ MN $)向右做匀速运动。
图$ 1 $轨道端点$ MP $间接有阻值为$ r $的电阻,导体棒$ ab $受到水平向右的外力作用。图$ 2 $轨道端点$ MP $间接有直流电源,导体棒$ ab $通过滑轮匀速提升重物,电路中的电流为$ I $。
\begin{enumerate}
\renewcommand{\labelenumi}{\arabic{enumi}.}
% A(\Alph) a(\alph) I(\Roman) i(\roman) 1(\arabic)
%设定全局标号series=example	%引用全局变量resume=example
%[topsep=-0.3em,parsep=-0.3em,itemsep=-0.3em,partopsep=-0.3em]
%可使用leftmargin调整列表环境左边的空白长度 [leftmargin=0em]
\item
求在$ \Delta t $时间内,图$ 1 $“发电机”产生的电能和图$ 2 $“电动机”输出的机械能。
\item 
从微观角度看,导体棒$ ab $中的自由电荷所受洛伦兹力在上述能量转化中起着重要作用。为了方便,可认为导体棒中的自由电荷为正电荷。
\begin{enumerate}
\renewcommand{\labelenumiii}{\alph{enumiii}.}
% A(\Alph) a(\alph) I(\Roman) i(\roman) 1(\arabic)
%设定全局标号series=example	%引用全局变量resume=example
%[topsep=-0.3em,parsep=-0.3em,itemsep=-0.3em,partopsep=-0.3em]
%可使用leftmargin调整列表环境左边的空白长度 [leftmargin=0em]
\item
请在图$ 3 $(图$ 1 $的导体棒$ ab $)、图$ 4 $(图$ 2 $的导体棒$ ab $)中,分别画出自由电荷所受洛伦兹力的示意图。
\item 
我们知道,洛伦兹力对运动电荷不做功。那么,导体棒$ ab $中的自由电荷所受洛伦兹力是如何在能量转化过程中起到作用的呢?请以图$ 2 $“电动机”为例,通过计算分析说明。



\end{enumerate}

\end{enumerate}
\begin{figure}[h!]
\flushright
\includesvg[width=0.45\linewidth]{picture/svg/191}\qquad
\includesvg[width=0.25\linewidth]{picture/svg/192}
\end{figure}


\banswer{
\begin{enumerate}
\renewcommand{\labelenumi}{\arabic{enumi}.}
% A(\Alph) a(\alph) I(\Roman) i(\roman) 1(\arabic)
%设定全局标号series=example	%引用全局变量resume=example
%[topsep=-0.3em,parsep=-0.3em,itemsep=-0.3em,partopsep=-0.3em]
%可使用leftmargin调整列表环境左边的空白长度 [leftmargin=0em]
\item
$E _ { 1 } = \frac { B ^ { 2 } L ^ { 2 } v ^ { 2 } } { R + r } \cdot \Delta t , \quad E _ { 2 } = I L B v \Delta t$;
\item 
\begin{enumerate}
\renewcommand{\labelenumiii}{\alph{enumiii}.}
% A(\Alph) a(\alph) I(\Roman) i(\roman) 1(\arabic)
%设定全局标号series=example	%引用全局变量resume=example
%[topsep=-0.3em,parsep=-0.3em,itemsep=-0.3em,partopsep=-0.3em]
%可使用leftmargin调整列表环境左边的空白长度 [leftmargin=0em]
\item
略
\item 
略(导体棒中一个自由电荷所受的洛伦兹力做功为零,大量自由电荷所受洛伦兹力做功的宏观表现是将电能转化为等量的机械能,在此过程中洛伦兹力通过两个分力做功起到“传递”能量的作用。)


\end{enumerate}



\end{enumerate}
}




\end{enumerate}








