
\bta{直线运动的图像问题}

\begin{enumerate}[leftmargin=0em]
\renewcommand{\labelenumi}{\arabic{enumi}.}
% A(\Alph) a(\alph) I(\Roman) i(\roman) 1(\arabic)
%设定全局标号series=example	%引用全局变量resume=example
%[topsep=-0.3em,parsep=-0.3em,itemsep=-0.3em,partopsep=-0.3em]
%可使用leftmargin调整列表环境左边的空白长度 [leftmargin=0em]
\item
\exwhere{$ 2019 $年$ 4 $月浙江物理选考}
甲、乙两物体零时刻开始从同一地点向同一方向做直线运动,位移$ - $时间图象如图所示,则在$ 0 \sim t_{1} $时间内 \xzanswer{B} 
\begin{figure}[h!]
\centering
\includesvg[width=0.17\linewidth]{picture/svg/389}
\end{figure}


\fourchoices
{甲的速度总比乙大}
{甲、乙位移相同}
{甲经过的路程比乙小}
{甲、乙均做加速运动}



\item 
\exwhere{$ 2015 $年理综福建卷}
一摩托车由静止开始在平直的公路上行驶,其运动过程的$ v-t $图象如图所示。求:
\begin{enumerate}
\renewcommand{\labelenumii}{(\arabic{enumii})}

\item 
摩托车在$ 0 \sim 20 \ s $这段时间的加速度大小$ a $;

\item 
摩托车在$ 0 \sim 75 \ s $这段时间的平均速度大小$ \bar{v} $。





\end{enumerate}
\begin{figure}[h!]
\flushright 
\includesvg[width=0.26\linewidth]{picture/svg/390}
\end{figure}


\banswer{
\begin{enumerate}
\renewcommand{\labelenumi}{\arabic{enumi}.}
% A(\Alph) a(\alph) I(\Roman) i(\roman) 1(\arabic)
%设定全局标号series=example	%引用全局变量resume=example
%[topsep=-0.3em,parsep=-0.3em,itemsep=-0.3em,partopsep=-0.3em]
%可使用leftmargin调整列表环境左边的空白长度 [leftmargin=0em]
\item
$ a=1.5\ m/s^{2} $

\item 
$ \bar{v}=20\ m/s $

\end{enumerate}


}



\item 
\exwhere{$ 2013 $年上海卷}
汽车以恒定功率沿公路做直线运动,途中通过一块沙地。汽车在公路及沙地上所受阻力均为恒力,且在沙地上受到的阻力大于在公路上受到的阻力。汽车在驶入沙地前已做匀速直线运动,它在驶入沙地到驶出沙地后的一段时间内,位移$ s $随时间$ t $的变化关系可能是 \xzanswer{A} 
\begin{figure}[h!]
\centering
\includesvg[width=0.83\linewidth]{picture/svg/391}
\end{figure}




\newpage	
\item
\exwhere{$ 2014 $年物理江苏卷}
一汽车从静止开始做匀加速直线运动, 然后刹车做匀减速直线运动,直到停止。 下列速度 $ v $ 和位移 $ x $ 的关系图像中,能描述该过程的是 \xzanswer{A} 
\begin{figure}[h!]
\centering
\includesvg[width=0.83\linewidth]{picture/svg/392}
\end{figure}

\item 
\exwhere{$ 2014 $年理综大纲卷}
—质点沿$ x $轴做直线运动,其$ v-t $图像如图所示。质点在$ t = 0 $时位于$ x= 5 \ m $处,开始沿$ x $轴正向运动。当$ t $=$ 8 \ s $时,质点在$ x $轴上的位置为 \xzanswer{B} 
\begin{figure}[h!]
\centering
\includesvg[width=0.26\linewidth]{picture/svg/393}
\end{figure}

\fourchoices
{$ x $=$ 3 \ m $ }
{$ x $=$ 8 \ m $}
{$ x $=$ 9 \ m $ }
{$ x $=$ 14 \ m $}


\item 
\exwhere{$ 2014 $年理综天津卷}
质点做直线运动的速度—时间图象如图所示,
该质点 \xzanswer{D} 
\begin{figure}[h!]
\centering
\includesvg[width=0.23\linewidth]{picture/svg/394}
\end{figure}


\fourchoices
{在第$ 1 $秒末速度方向发生了改变}
{在第$ 2 $秒末加速度方向发生了改变}
{在前$ 2 $秒内发生的位移为零}
{第$ 3 $秒末和第$ 5 $秒末的位置相同}


\item 
\exwhere{$ 2014 $年理综广东卷}
下图是物体做直线运动的$ v-t $图象,由图可知,该物体 \xzanswer{B} 
\begin{figure}[h!]
\centering
\includesvg[width=0.2\linewidth]{picture/svg/395}
\end{figure}


\fourchoices
{第$ 1 $ $ s $内和第$ 3 $ $ s $内的运动方向相反}
{第$ 3 $ $ s $内和第$ 4 $ $ s $内的加速度相同}
{第$ 1 $ $ s $内和第$ 4 \ s $内的位移大小不等}
{$ 0 \sim 2 \ s $内和$ 0 \sim 4 \ s $内的平均速度大小相等}





\item 
\exwhere{$ 2014 $年理综新课标 \lmd{2} 卷}
甲乙两汽车在一平直公路上同向行驶。在$ t=0 $到$ t=t_1 $的时间内,它们的$ v-t $图像如图所示。在这段时间内 \xzanswer{A} 
\begin{figure}[h!]
\centering
\includesvg[width=0.2\linewidth]{picture/svg/398}
\end{figure}

\fourchoices
{汽车甲的平均速度比乙大}
{汽车乙的平均速度等于}
{甲乙两汽车的位移相同}
{汽车甲的加速度大小逐渐减小,汽车乙的加速度大小逐渐增大}

\item 
\exwhere{$ 2013 $年海南卷}
一物体做直线运动,其加速度随时间变化的$ a-t $图象如图所示。下列$ v-t $图象中,可能正确描述此物体运动的是 \xzanswer{D} 
\begin{figure}[h!]
\centering
\includesvg[width=0.23\linewidth]{picture/svg/396}\\
\includesvg[width=0.83\linewidth]{picture/svg/405}
\end{figure}



\item 
\exwhere{$ 2016 $年新课标 \lmd{1} 卷}
甲、乙两车在平直公路上同向行驶,其$ v-t $图像如图所示。已知两车在$ t=3 \ s $时并排行驶,则 \xzanswer{BD} 
\begin{figure}[h!]
\centering
\includesvg[width=0.23\linewidth]{picture/svg/399}
\end{figure}


\fourchoices
{在$ t=1 \ s $时,甲车在乙车后}
{在$ t=0 $时,甲车在乙车前$ 7.5\ m $}
{两车另一次并排行驶的时刻是$ t=2 \ s $}
{甲、乙车两次并排行驶的位置之间沿公路方向的距离为$ 40 \ m $}

\item 
\exwhere{$ 2018 $年全国 \lmd{2} 卷}
甲、乙两汽车在同一条平直公路上同向运动,其速度-时间图像分别如图中甲、乙两条曲线所示。已知两车在$ t_{2} $时刻并排行驶。下列说法正确的是 \xzanswer{BD} 
\begin{figure}[h!]
\centering
\includesvg[width=0.23\linewidth]{picture/svg/400}
\end{figure}


\fourchoices
{两车在$ t_{1} $时刻也并排行驶}
{在$ t_{1} $时刻甲车在后,乙车在前}
{甲车的加速度大小先增大后减小}
{乙车的加速度大小先减小后增大}


\item 
\exwhere{$ 2018 $年全国 \lmd{3} 卷}
甲、乙两车在同一平直公路上同向运动,甲做匀加速直线运动,乙做匀速直线运动。甲、乙两车的位置$ x $随时间$ t $的变化如图所示。下列说法正确的是 \xzanswer{CD} 
\begin{figure}[h!]
\centering
\includesvg[width=0.23\linewidth]{picture/svg/401}
\end{figure}


\fourchoices
{在$ t_{1} $时刻两车速度相等}
{从$ 0 $到$ t_{1} $时间内,两车走过的路程相等}
{从$ t_{1} $到$ t_{2} $时间内,两车走过的路程相等}
{在$ t_{1} $到$ t_{2} $时间内的某时刻,两车速度相等}


\item 
\exwhere{$ 2016 $年江苏卷}
小球从一定高度处由静止下落,与地面碰撞后回到原高度再次下落,重复上述运动,取小球的落地点为原点建立坐标系,竖直向上为正方向,下列速度$ v $和位置$ x $的关系图像中,能描述该过程的是 \xzanswer{A} 
\begin{figure}[h!]
\centering
\includesvg[width=0.83\linewidth]{picture/svg/402}
\end{figure}


\item 
\exwhere{$ 2013 $年四川卷}
甲、乙两物体在$ t=0 $时刻经过同一位置沿$ x $轴运动,其$ v-t $图像如图所示。则 \xzanswer{BD} 


\begin{minipage}[h!]{0.7\linewidth}
\vspace{0.3em}
\fourchoices
{甲、乙在$ t=0 \ s $到$ t=1 \ s $之间沿同一方向运动}
{乙在$ t=0 $到$ t=7 \ s $之间的位移为零}
{甲在$ t=0 $到$ t=4 \ s $之间做往复运动}
{甲、乙在$ t=6 \ s $时的加速度方向相同}

\vspace{0.3em}
\end{minipage}
\hfill
\begin{minipage}[h!]{0.3\linewidth}
\flushright
\vspace{0.3em}
\includesvg[width=0.9\linewidth]{picture/svg/403}
\vspace{0.3em}
\end{minipage}


\item 
\exwhere{$ 2013 $年全国卷大纲版}
将甲乙两小球以同样的速度在距离地面不同高度处竖直向上抛出,抛出时间间隔$ 2 \ s $,它们运动的图象分别如直线甲、乙所示。则 \xzanswer{BD} 


\begin{minipage}[h!]{0.7\linewidth}
\vspace{0.3em}
\fourchoices
{$ t=2 \ s $时,两球高度相差一定为$ 40 \ m $}
{$ t=4 \ s $时,两球相对于各自抛出点的位移相等}
{两球从抛出至落到地面所用的时间间隔相等}
{甲球从抛出至达到最高点的时间间隔与乙球的相等}

\vspace{0.3em}
\end{minipage}
\hfill
\begin{minipage}[h!]{0.3\linewidth}
\flushright
\vspace{0.3em}
\includesvg[width=0.7\linewidth]{picture/svg/404}
\vspace{0.3em}
\end{minipage}


\newpage
\item 
\exwhere{$ 2017 $年浙江选考卷}
游船从码头沿直线行驶到湖对岸,小明对过程进行观察,记录数据如下表:
\begin{table}[h!]
\centering 
\begin{tabular}{|c|c|c|}
\hline 
运动过程 & 运动时间 & 运动状态
 \\
\hline
匀加速运动 & $ 0 \sim 40\ s $ & 初速度$ v_{0}=0 $;末速度$ v=4.2\ m/s $
 \\
\hline
匀速运动 & $ 40 \sim 640 \ s $ & $ v=4.2\ m/s $ 
 \\
\hline
匀减速运动 & $ 640 \sim 720 \ s $ & 靠岸时的速度$ v_{t}=0.2\ m/s $\\ 
\hline 
\end{tabular}
\end{table} 

\begin{enumerate}
\renewcommand{\labelenumi}{\arabic{enumi}.}
% A(\Alph) a(\alph) I(\Roman) i(\roman) 1(\arabic)
%设定全局标号series=example	%引用全局变量resume=example
%[topsep=-0.3em,parsep=-0.3em,itemsep=-0.3em,partopsep=-0.3em]
%可使用leftmargin调整列表环境左边的空白长度 [leftmargin=0em]
\item
求游船匀加速运动过程中加速度大小$ a_{1} $,及位移大小$ x_{1} $;
\item 
若游船和游客总质量,求游船匀减速运动过程中所受合力的大小$ F $;
\item 
求游船在整个行驶过程中的平均速度大小 。



\end{enumerate}


\banswer{
\begin{enumerate}
\renewcommand{\labelenumi}{\arabic{enumi}.}
% A(\Alph) a(\alph) I(\Roman) i(\roman) 1(\arabic)
%设定全局标号series=example	%引用全局变量resume=example
%[topsep=-0.3em,parsep=-0.3em,itemsep=-0.3em,partopsep=-0.3em]
%可使用leftmargin调整列表环境左边的空白长度 [leftmargin=0em]
\item
$a _ { 1 } = \frac { \Delta v } { \Delta t } = \frac { 4.2 - 0 } { 40 } = 0.105 \ \mathrm { m } / \mathrm { s } ^ { 2 }$\\
$x _ { 1 } = \frac { 1 } { 2 } v t = \frac { 1 } { 2 } \times 4.2 \times 40 = 84 \mathrm { m }$

\item 
$ F=400\ N $
\item 
$ 3.86\ m/s $

\end{enumerate}


}





\end{enumerate}





