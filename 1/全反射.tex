\bta{全反射}


\begin{enumerate}
	%\renewcommand{\labelenumi}{\arabic{enumi}.}
	% A(\Alph) a(\alph) I(\Roman) i(\roman) 1(\arabic)
	%设定全局标号series=example	%引用全局变量resume=example
	%[topsep=-0.3em,parsep=-0.3em,itemsep=-0.3em,partopsep=-0.3em]
	%可使用leftmargin调整列表环境左边的空白长度 [leftmargin=0em]
	\item
\exwhere{$ 2019 $ 年物理江苏卷}
如图所示,某 $ L $ 形透明材料的折射率 $ n=2 $.现沿 $ AB $ 方向切去一角,$ AB $ 与
水平方向的夹角为$ \theta $.为使水平方向的光线射到 $ AB $ 面时不会射入空气,求$ \theta $的最大值.
\begin{figure}[h!]
	\flushright
	\includesvg[width=0.25\linewidth]{picture/svg/GZ-3-tiyou-1455}
\end{figure}

\banswer{
	$ \theta=60 \degree $
}



\item 
\exwhere{$ 2014 $ 年理综四川卷}
如图所示,口径较大、充满水的薄壁圆柱形玻璃缸底有一发光小球,则 \xzanswer{D} 
\begin{figure}[h!]
	\centering
	\includesvg[width=0.23\linewidth]{picture/svg/GZ-3-tiyou-1456}
\end{figure}


\fourchoices
{小球必须位于缸底中心才能从侧面看到小球}
{小球所发的光能从水面任何区域射出}
{小球所发的光频率变大}
{小球所发的光从水中进入空气后传播速度变大}



\item 
\exwhere{$ 2014 $ 年理综福建卷}
如图,一束光由空气射向半圆柱体玻璃砖,$ O $ 点为该玻璃砖截面的圆心,下图能正确描述其光
路图的是 \xzanswer{A} 

\pfourchoices
{\includesvg[width=4.3cm]{picture/svg/GZ-3-tiyou-1457}}
{\includesvg[width=4.3cm]{picture/svg/GZ-3-tiyou-1458}}
{\includesvg[width=4.3cm]{picture/svg/GZ-3-tiyou-1459}}
{\includesvg[width=4.3cm]{picture/svg/GZ-3-tiyou-1460}}



\item 
\exwhere{$ 2013 $ 年浙江卷}
与通常观察到的月全食不同,小虎同学在 $ 2012 $ 年 $ 12 $ 月 $ 10 $ 日晚观看月全食时,看到整个月亮
是暗红的。小虎画了月全食的示意图,并提出了如下
猜想,其中最为合理的是 \xzanswer{C} 
\begin{figure}[h!]
	\centering
	\includesvg[width=0.23\linewidth]{picture/svg/GZ-3-tiyou-1461}
\end{figure}


\fourchoices
{地球上有人用红色激光照射月球}
{太阳照射到地球的红光反射到月球}
{太阳光中的红光经地球大气层折射到月球}
{太阳光中的红光在月球表面形成干涉条纹}


\item 
\exwhere{$ 2013 $ 年天津卷}
固定的半圆形玻璃砖的横截面如图。$ O $ 点为圆心,$ OO ^{\prime} $为直径 $ MN $ 的垂线。足够大的光屏 $ PQ $ 紧
靠玻瑞砖右侧且垂直于 $ MN $。由 $ A $、$ B $ 两种单色光组成的一束光沿半径方向射向 $ O $ 点,入射光线与
$ OO ^{\prime} $夹角$ \theta $较小时,光屏 $ NQ $ 区城出现两个光斑,逐渐增大$ \theta $角.当$ \theta = \alpha $时,光屏 $ NQ $ 区城 $ A $ 光的光
斑消失,继续增大$ \theta $角,当$ \theta = \beta $时,光屏 $ NQ $ 区域 $ B $ 光的光斑消失,则 \xzanswer{AD} 
\begin{figure}[h!]
	\centering
	\includesvg[width=0.23\linewidth]{picture/svg/GZ-3-tiyou-1462}
\end{figure}

\fourchoices
{玻璃砖对 $ A $ 光的折射率比对 $ B $ 光的大}
{$ A $ 光在玻璃砖中传播速度比 $ B $ 光的大}
{$ \alpha < \theta < \beta $时,光屏上只有 $ 1 $ 个光斑}
{$ \beta < \theta < \pi /2 $ 时,光屏上只有 $ 1 $ 个光斑}

\item 
\exwhere{$ 2011 $ 年理综重庆卷}
在一次讨论中,老师问道:“假如水中相同深度处有 $ a $、$ b $、$ c $ 三种不同颜色的单色点光源,有人在水
面上方同等条件下观测发现,$ b $ 在水下的像最深,$ c $ 照亮水面的面积比 $ a $ 的大。关于这三种光在水中的性
质,同学们能做出什么判断?”有同学回答如下:\\
①$ c $ 光的频率最大;\\②$ a $ 光的传播速度最小;\\③$ b $ 光的折
射率最大;\\④$ a $ 光的波长比 $ b $ 光的短。\\ 
根据老师的假定,以上回答正确的是 \xzanswer{C} 

\fourchoices
{①②}
{①③}
{②④}
{③④}


	
	
\end{enumerate}

