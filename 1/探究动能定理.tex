\bta{探究动能定理}
\begin{enumerate}
\renewcommand{\labelenumi}{\arabic{enumi}.}
% A(\Alph) a(\alph) I(\Roman) i(\roman) 1(\arabic)
%设定全局标号series=example	%引用全局变量resume=example
%[topsep=-0.3em,parsep=-0.3em,itemsep=-0.3em,partopsep=-0.3em]
%可使用leftmargin调整列表环境左边的空白长度 [leftmargin=0em]
\item
\exwhere{$ 2019 $ 年物理江苏卷}
某兴趣小组用如题 $ 1 $ 图所示 的装置验证动能定理.
\begin{figure}[h!]
\centering
\includesvg[width=0.43\linewidth]{picture/svg/GZ-3-tiyou-0483}
\end{figure}

\begin{enumerate}
\renewcommand{\labelenumi}{\arabic{enumi}.}
% A(\Alph) a(\alph) I(\Roman) i(\roman) 1(\arabic)
%设定全局标号series=example	%引用全局变量resume=example
%[topsep=-0.3em,parsep=-0.3em,itemsep=-0.3em,partopsep=-0.3em]
%可使用leftmargin调整列表环境左边的空白长度 [leftmargin=0em]
\item
有两种工作频率均为 $ 50 \ Hz $ 的打点计时器供实验选用:

A.电磁打点计时器

B.电火花打点计时器


为使纸带在运动时受到的阻力较小,应选择 \tk{B} (选填“$ A $”或“$ B $”).

\item 
保持长木板水平,将纸带固定在小车后端,纸带穿过打点计时器的限位孔.实验中,为消除
摩擦力的影响,在砝码盘中慢慢加入沙子,直到小车开始运动.同学甲认为此时摩擦力的影响已得
到消除.同学乙认为还应从盘中取出适量沙子,直至轻推小车观察到小车做匀速运动.看法正确的
同学是 \tk{乙} (选填“甲”或“乙”).


\item 
消除摩擦力 的影响后,在砝码盘中加入砝码.接通打点计时器电源,松开小车,小车运动.纸
带被打出一系列点,其中的一段如题 $ 2 $ 图所示.图中纸带按实际尺寸画出,纸带上 $ A $ 点的速度
$ v_{A} =$ \tk{$ 0.31 $($ 0.30 \sim 0.33 $ 都算对)} $m/s $.
\begin{figure}[h!]
\centering
\includesvg[width=0.63\linewidth]{picture/svg/GZ-3-tiyou-0484}
\end{figure}


\item 
测出小车的质量为 $ M $,再测出纸带上起点到 $ A $ 点的距离为 $ L $.小车动能的变化量可用$ \Delta E_{k} = \frac{ 1 }{ 2 } M v_{A} ^{2} $
算出.砝码盘中砝码的质量为 $ m $,重力加速度为 $ g $;实验中,小车的质量应 \tk{远大于} (选填“远大于”“远
小于”或“接近”)砝码、砝码盘和沙子的总质量,小车所受合力做的功可用 $ W=mgL $ 算出.多次测量,
若 $ W $ 与$ \Delta E_{k} $ 均基本相等则验证了动能定理.


\end{enumerate}



\newpage
\item
\exwhere{$ 2013 $ 年福建卷}
在“探究恒力做功与动能改变的关系”实验中(装置如图甲):
\begin{figure}[h!]
\centering
\includesvg[width=0.43\linewidth]{picture/svg/GZ-3-tiyou-0485} \qquad 
 \includesvg[width=0.43\linewidth]{picture/svg/GZ-3-tiyou-0486} 
\end{figure}


①下列说法哪一项是正确的
\tk{C} 
。(填选项前字母)

\threechoices
{平衡摩擦力时必须将钩码通过细线挂在小车上}
{为减小系统误差,应使钩码质量远大于小车质量}
{实验时,应使小车靠近打点计时器由静止释放}

②图乙是实验中获得的一条纸带的一部分,选取 $ O $、$ A $、
$ B $、$ C $ 计数点,已知打点计时器使用的交流电频率为 $ 50 \ Hz $,则打 $ B $ 点时小车的瞬时速度大小为
\tk{$ 0.653 $} 
$ m/s $
(保留三位有效数字)
。


\newpage
\item 
\exwhere{$ 2014 $ 年理综天津卷}
某同学把附有滑轮的长木板平放在实验桌上,将细绳一端拴在小车上,另一端绕过定滑轮,
挂上适当的钩码,使小车在钩码的牵引下运动,以此定量探究绳拉力与小车动能变化的关系.此外
还准备了打点计时器及配套的电源、导线、复写纸、纸带、小木块等.组装的实验装置如图所示.
\begin{figure}[h!]
\centering
\includesvg[width=0.36\linewidth]{picture/svg/GZ-3-tiyou-0487}
\end{figure}


①若要完成该实验,必需的实验器材还有哪些 \tk{刻度尺、天平(包括砝码)} .

②实验开始时,他先调节木板上定滑轮的高度,使牵引小车的细绳与木板平行.他这样做的目的是
下列的哪个 \tk{D} (填字母代号).
\fourchoices
{避免小车在运动过程中发生抖动}
{可使打点计时器在纸带上打出的点迹清晰}
{可以保证小车最终能够实现匀速直线运动}
{可在平衡摩擦力后使细绳拉力等于小车受的合力}

③平衡摩擦力后,当他用多个钩码牵引小车时,发现小车运动过快,致使打出的纸带上点数较少,
难以选到合适的点计算小车速度.在保证所挂钩码数目不变的条件下,请你利用本实验的器材提出
一个解决办法: \tk{可在小车上加适量的砝码(或钩码)} .


④他将钩码重力做的功当作细绳拉力做的功,经多次实验发现拉力做功总是要比小车动能增量大一
些.这一情况可能是下列哪些原因造成的 \tk{CD} (填字母代号).
\fourchoices
{在接通电源的同时释放了小车}
{小车释放时离打点计时器太近}
{阻力未完全被小车重力沿木板方向的分力平衡掉}
{钩码做匀加速运动,钩码重力大于细绳拉力}


\newpage
\item 
\exwhere{$ 2017 $ 年北京卷}
如图 $ 1 $ 所示,用质量为 $ m $ 的重物通过滑轮牵引小车,使它在长木板上运动,打点计时器在纸带上记
录小车的运动情况。利用该装置可以完成“探究动能
定理”的实验。
\begin{figure}[h!]
\centering
\includesvg[width=0.3\linewidth]{picture/svg/GZ-3-tiyou-0489} \qquad 
 \includesvg[width=0.5\linewidth]{picture/svg/GZ-3-tiyou-0490} 
\end{figure}

\begin{enumerate}
\renewcommand{\labelenumi}{\arabic{enumi}.}
% A(\Alph) a(\alph) I(\Roman) i(\roman) 1(\arabic)
%设定全局标号series=example	%引用全局变量resume=example
%[topsep=-0.3em,parsep=-0.3em,itemsep=-0.3em,partopsep=-0.3em]
%可使用leftmargin调整列表环境左边的空白长度 [leftmargin=0em]
\item
打点计时器使用的电源是
\tk{B} 
(选填选
项前的字母)
。


\twochoices
{直流电源}
{交流电源}

\item 

实验中,需要平衡摩擦力和其他阻力。正确操
作方法是
\tk{A} 
(选填选项前的字母)。

\twochoices
{把长木板右端垫高}
{改变小车的质}



在不挂重物且
\tk{B} 
(选填选项前的字母)的情
况下,轻推一下小车,若小车拖着纸带做匀速运动,表明已经消除了摩擦力和其他阻力的影响。

\twochoices
{计时器不打点}
{计时器打点}


\item 
接通电源,释放小车,打点计时器在
纸带上打下一系列点,将打下的第一个点标
为 $ O $。在纸带上依次取 $ A $、$ B $、$ C \cdots \cdots $若干个
计数点,已知相邻计数点间的时间间隔为 $ T $。
测得 $ A $、$ B $、$ C \cdots \cdots $各点到 $ O $ 点的距离为 $ x_{1} $、
$ x_{2} $、$ x_{3} \cdots \cdots $,如图 $ 2 $ 所示。



实验中,重物质量远小于小车质量,可认为
小车所受的拉力大小为 $ mg $,从打 $ O $ 点到打
$ B $点的过程中,拉力对小车做的功
$ W= $
\tk{$ mgx_{2} $} 
,打 $ B $ 点时小车的速度 $ v= $
\tk{$\frac{x_{3}-x_{1}}{2 T}$}。



\item 
以 $ v^{2} $ 为纵坐标,$ W $ 为横坐标,利用实验数据做出如图 $ 3 $ 所示的 $ v^{2} - W $ 图像。由此图像可得 $ v^{2} $
随 $ W $ 变化的表达式为 \tk{$v^{2}=k W$} 。根据功与能的关系,动能的表达式中可能包含 $ v^{2} $ 这个因
子;分析实验结果的单位关系,与图线斜率有关的物理量应是 \tk{$k=4.7 \mathrm{kg}^{-1}$} 。


\item 
假设已经完全消除了摩擦力和其他阻力的影响,若重物质量不满足远小于小车质量的条件,
则从理论上分
析,图 $ 4 $ 中正确
反映 $ v^{2} - W $ 关系
的是 \tk{D} 。
\begin{figure}[h!]
\centering
\includesvg[width=0.83\linewidth]{picture/svg/GZ-3-tiyou-0491}
\end{figure}


\end{enumerate}



\newpage
\item 
\exwhere{$ 2017 $ 年江苏卷}
利用如图所示的实验装置探究恒力做功与物体动能变化的关系。
小车的质量为 $ M=200.0 \ g $,钩码的质量为 $ m=10.0 \ g $,打点计时器的电源为 $ 50 \ Hz $ 的交流电.
\begin{figure}[h!]
\centering
\includesvg[width=0.4\linewidth]{picture/svg/GZ-3-tiyou-0492}
\end{figure}


\begin{enumerate}
\renewcommand{\labelenumi}{\arabic{enumi}.}
% A(\Alph) a(\alph) I(\Roman) i(\roman) 1(\arabic)
%设定全局标号series=example	%引用全局变量resume=example
%[topsep=-0.3em,parsep=-0.3em,itemsep=-0.3em,partopsep=-0.3em]
%可使用leftmargin调整列表环境左边的空白长度 [leftmargin=0em]
\item
挂钩码前,为了消除摩擦力的影响,应调节木板右侧
的高度,直至向左轻推小车观察到 \tk{小车做匀速运动} .

\item 
挂上钩码,按实验要求打出的一条纸带如图所
示.选择某一点为 $ O $,一次每隔 $ 4 $ 个计时点取一个计数点.用
刻度尺量出相邻计数点间的距离 $ \Delta x $,记录在纸带上.计算打
出各计数点时小车的速度 $ v $,其中打出计数点“$ 1 $”时小车的速度 $ v_{1}= $ \tk{$ 0.228 $} $ m/s $.
\begin{figure}[h!]
\centering
\includesvg[width=0.83\linewidth]{picture/svg/GZ-3-tiyou-0493}
\end{figure}


\item 
将钩码的重力视位小车受到的拉力,取 $ g=9.80 \ m/s $,利用 $ W=mg\Delta x $ 算出拉力对小车做的功 $ W $,
利用 $ E_{k} = \frac{ 1 }{ 2 } M v^{2} $ 算出小车动能,并求出动能的变化量$ \triangle E_{k} $.计算结果见下表.
\begin{table}[h!]
\centering 
\begin{tabular}{|c|c|c|c|c|c|}
\hline 
$W / \times 10^{-3} \mathbf{J}$ & $ 2.45 $ & $ 2.92 $ & $ 3.35 $ & $ 3.81 $ & $ 4.26 $
 \\
\hline
$\Delta E_{\mathrm{k}} / \times 10^{-3} \mathrm{J}$ & $ 2.31 $ & $ 2.73 $ & $ 3.12 $ & $ 3.61 $ & $ 4.00 $\\ 
\hline 
\end{tabular}
\end{table} 




请根据表中的数据,在答题卡的方格纸上作出 $ \Delta E_{k} -W $ 图象.
\begin{figure}[h!]
\centering
\includesvg[width=0.43\linewidth]{picture/svg/GZ-3-tiyou-0494}
\end{figure}

\banswer{
 \includesvg[width=0.23\linewidth]{picture/svg/GZ-3-tiyou-0495} 
}


\item 
实验结果表明,$ \triangle E_{k} $ 总是略小于 $ W $, 某同学猜想是由
于小车所受拉力小于钩码重力造成的。用题中小车和钩码质
量的数据可算出小车受到的实际拉力 $ F= $ \tk{$ 0.093 \ N $} .


\end{enumerate}








\end{enumerate}

