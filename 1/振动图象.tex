\bta{振动图象}


\begin{enumerate}
	%\renewcommand{\labelenumi}{\arabic{enumi}.}
	% A(\Alph) a(\alph) I(\Roman) i(\roman) 1(\arabic)
	%设定全局标号series=example	%引用全局变量resume=example
	%[topsep=-0.3em,parsep=-0.3em,itemsep=-0.3em,partopsep=-0.3em]
	%可使用leftmargin调整列表环境左边的空白长度 [leftmargin=0em]
	\item
\exwhere{$ 2017 $ 年北京卷}
某弹簧振子沿 $ x $ 轴的简谐振动图像
如图所示,下列描述正确的是 \xzanswer{A} 
\begin{figure}[h!]
	\centering
	\includesvg[width=0.23\linewidth]{picture/svg/GZ-3-tiyou-1314}
\end{figure}



\item 
\exwhere{$ 2016 $ 年北京卷}
如图所示,弹簧振子在 $ M $、$ N $ 之间做简谐运动。以平衡位置 $ O $ 为原点,建立 $ Ox $
轴。向右为 $ x $ 的轴的正方向。若振子位于 $ N $ 点时开始计时,则
其振动图像为 \xzanswer{A} 
\begin{figure}[h!]
	\centering
	\includesvg[width=0.23\linewidth]{picture/svg/GZ-3-tiyou-1315}
\end{figure}


\pfourchoices
{\includesvg[width=4.3cm]{picture/svg/GZ-3-tiyou-1316}}
{\includesvg[width=4.3cm]{picture/svg/GZ-3-tiyou-1317}}
{\includesvg[width=4.3cm]{picture/svg/GZ-3-tiyou-1318}}
{\includesvg[width=4.3cm]{picture/svg/GZ-3-tiyou-1319}}

\item 
\exwhere{$ 2012 $ 年理综重庆卷}
装有砂粒的试管竖直静立于小面,如题 $ 14 $ 图所示,将管竖直提起少许,
然后由静止释放并开始计时,在一定时间内试管在竖直方向近似做简谐运
动。若取竖直向上为正方向,则以下描述试管振动的图象中可能正确的是 \xzanswer{D} 
\begin{figure}[h!]
	\centering
	\includesvg[width=0.23\linewidth]{picture/svg/GZ-3-tiyou-1325} 
	%\includesvg[width=0.23\linewidth]{picture/svg/GZ-3-tiyou-1320}
\end{figure}

\pfourchoices
{\includesvg[width=4.3cm]{picture/svg/GZ-3-tiyou-1321}}
{\includesvg[width=4.3cm]{picture/svg/GZ-3-tiyou-1322}}
{\includesvg[width=4.3cm]{picture/svg/GZ-3-tiyou-1323}}
{\includesvg[width=4.3cm]{picture/svg/GZ-3-tiyou-1324}}


\item 
\exwhere{$ 2012 $ 年理综北京卷}
一个弹簧振子沿 $ x $ 轴做简谐运动,取平衡位置 $ O $ 为 $ x $ 轴坐标原点。从某时刻开始计时,经过四
分之一的周期,振子具有沿 $ x $ 轴正方向的最大加速度。能正确反映振子位移 $ x $ 与时间,关系的图像
是 \xzanswer{A} 

\pfourchoices
{\includesvg[width=4.3cm]{picture/svg/GZ-3-tiyou-1326}}
{\includesvg[width=4.3cm]{picture/svg/GZ-3-tiyou-1327}}
{\includesvg[width=4.3cm]{picture/svg/GZ-3-tiyou-1329}}
{\includesvg[width=4.3cm]{picture/svg/GZ-3-tiyou-1330}}



\item 
\exwhere{$ 2014 $ 年物理上海卷}
质点做简谐运动,其 $ x-t $ 关系如图。以 $ x $ 轴正向为速度 $ v $ 的正方向,该
质点的 $ v-t $ 关系是 \xzanswer{B} 
\begin{figure}[h!]
	\centering
	\includesvg[width=0.23\linewidth]{picture/svg/GZ-3-tiyou-1331}
\end{figure}


\pfourchoices
{\includesvg[width=4.3cm]{picture/svg/GZ-3-tiyou-1332}}
{\includesvg[width=4.3cm]{picture/svg/GZ-3-tiyou-1333}}
{\includesvg[width=4.3cm]{picture/svg/GZ-3-tiyou-1334}}
{\includesvg[width=4.3cm]{picture/svg/GZ-3-tiyou-1335}}


\item 
\exwhere{$ 2014 $ 年理综浙江卷}
一位游客在千岛湖边欲乘游船,当日风浪很大,游船上下浮动。可把游艇浮动简化成竖直方向的
简谐运动,振幅为 $ 20 \ cm $,周期为 $ 3.0 \ s $。当船上升到最高点时,甲板刚好与码头地面平齐。地面与
甲板的高度差不超过 $ 10 \ cm $ 时,游客能舒服地登船。在一个周期内,游客能舒服地登船的时间是 \xzanswer{C} 

\fourchoices
{$ 0.5 \ s $}
{$ 0.75 \ s $}
{$ 1.0 \ s $}
{$ 1.5 \ s $}



\item 
\exwhere{$ 2018 $ 年天津卷}
一振子沿 $ x $ 轴做简谐运动,平衡位置在坐标原点。$ t=0 $ 时振子的位移为$ -0.1 \ m $,
$ t=1 \ s $ 时位移为 $ 0.1 \ m $,则 \xzanswer{AD} 


\fourchoices
{若振幅为 $ 0.1 \ m $,振子的周期可能为$ \frac{ 2 }{ 3 } \ s $}
{若振幅为 $ 0.1 \ m $,振子的周期可能为$ \frac{ 4 }{ 5 } \ s $}
{若振幅为 $ 0.2 \ m $,振子的周期可能为 $ 4 \ s $}
{若振幅为 $ 0.2 \ m $,振子的周期可能为 $ 6 \ s $}





	
	
	
\end{enumerate}

