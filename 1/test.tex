\bta{复习}

%电场

\begin{enumerate}[leftmargin=0em]
\renewcommand{\labelenumi}{\arabic{enumi}.}
% A(\Alph) a(\alph) I(\Roman) i(\roman) 1(\arabic)
%设定全局标号series=example	%引用全局变量resume=example
%[topsep=-0.3em,parsep=-0.3em,itemsep=-0.3em,partopsep=-0.3em]
%可使用leftmargin调整列表环境左边的空白长度 [leftmargin=0em]
\item
\exwhere{$ 2018 $年浙江卷($ 4 $月选考)}
真空中两个完全相同、带等量同种电荷的金属小球$ A $和$ B $(可视为点电荷),分别固定在两处,它们之间的静电力为$ F $。用一个不带电的同样金属球$ C $先后与$ A $、$ B $球接触,然后移开球$ C $,此时$ A $、$ B $球间的静电力为 \xzanswer{C} 
\fourchoices
{$ \frac{F}{3} $}
{$ \frac{F}{4} $}
{$ \frac{3F}{8} $}
{$ \frac{F}{2} $}

\item
\exwhere{$ 2014 $年物理海南卷}
如图,两根平行长直导线相距$ 2L $,通有大小相等、方向相同的恒定电流,$ a $、$ b $、$ c $是导线所在平面内的三点,左侧导线与它们的距离分别为$ \frac{l}{2} $、$ l $和$ 3l $ 。关于这三点处的磁感应强度,下列判断正确的是 \xzanswer{AD} 
\begin{figure}[h!]
	\centering
\includesvg[width=0.17\linewidth]{picture/svg/146}
\end{figure}




\fourchoices
{$ a $处的磁感应强度大小比$ c $处的大}
{$ b $、$ c $两处的磁感应强度大小相等}
{$ a $、$ c $两处的磁感应强度方向相同}
{$ b $处的磁感应强度为零}


\item
\exwhere{$ 2013 $年天津卷}
两个带等量正电的点电荷,固定在图中$ P $、$ Q $两点,$ MN $为$ PQ $连线的中垂线,交$ PQ $于$ O $点,$ A $点为$ MN $上的一点。一带负电的试探电荷$ q $,从$ A $点由静止释放,只在静电力作用下运动。取无限远处的电势为零,则 \xzanswer{BC} 
\begin{figure}[h!]
	\centering
	\includesvg[width=0.19\linewidth]{picture/svg/027}
\end{figure}


\fourchoices
{$ q $由$ A $向$ O $的运动是匀加速直线运动 }
{$ q $由$ A $向$ O $运动的过程电势能逐渐减小}
{$ q $运动到$ O $点时的动能最大 }
{$ q $运动到$ O $点时电势能为零}

\item 
\exwhere{$ 2019 $年物理全国卷\lmd{2}}
静电场中,一带电粒子仅在电场力的作用下自$ M $点由静止开始运动,$ N $为粒子运动轨迹上的另外一点,则 \xzanswer{AC} 



\fourchoices
{运动过程中,粒子的速度大小可能先增大后减小}
{在$ M $、$ N $两点间,粒子的轨迹一定与某条电场线重合}
{粒子在$ M $点的电势能不低于其在$ N $点的电势能}
{粒子在$ N $点所受电场力的方向一定与粒子轨迹在该点的切线平行}







\item
\exwhere{$ 2015 $年广东卷}
如图所示的水平匀强电场中,将两个带电小球$ M $和$ N $分别沿图示路径移动到同一水平线上的不同位置,释放后,$ M $、$ N $保持静止,不计重力,则 \xzanswer{BD} 
\begin{figure}[h!]
\centering
\includesvg[width=0.19\linewidth]{picture/svg/046}
\end{figure}


\fourchoices
{$ M $的带电量比$ N $的大 }
{$ M $带负电荷,$ N $带正电荷}
{静止时$ M $受到的合力比$ N $的大 }
{移动过程中匀强电场对$ M $做负功}

\item
\exwhere{$ 2011 $年理综山东卷}
如图所示,在两等量异种点电荷的电场中,$ MN $为两电荷连线的中垂线,$ a $、$ b $、$ c $三点所在直线平行于两电荷的连线,且$ a $与$ c $关于$ MN $对称,$ b $点位于$ MN $上,$ d $点位于两电荷的连线上。以下判断正确的是 \xzanswer{BC} 
\begin{figure}[h!]
\centering
\includesvg[width=0.19\linewidth]{picture/svg/064}
\end{figure}


\fourchoices
{$ b $点场强大于$ d $点场强 }
{$ b $点场强小于$ d $点场强}
{$ a $、$ b $两点间的电势差等于$ b $、$ c $两点间的电势差}
{试探电荷$ +q $在$ a $点的电势能小于在$ c $点的电势能}


\item
\exwhere{$ 2012 $年物理海南卷}
将平行板电容器两极板之间的距离、电压、电场强度大小和极板所带的电荷量分别用$ d $、$ U $、$ E $和$ Q $表示。下列说法正确的是 \xzanswer{AD} 


\fourchoices
{保持$ U $不变,将$ d $变为原来的两倍,则$ E $变为原来的一半}
{保持$ E $不变,将$ d $变为原来的一半,则$ U $变为原来的两倍}
{保持$ d $不变,将$ Q $变为原来的两倍,则$ U $变为原来的一半}
{保持$ d $不变,将$ Q $变为原来的一半,则$ E $变为原来的一半}



\item
\exwhere{$ 2015 $年海南卷}
如图,一充电后的平行板电容器的两极板相距$ l $,在正极板附近有一质量为$ M $、电荷量为$ q $($ q > 0 $)的粒子,在负极板附近有另一质量为$ m $、电荷量为$ -q $的粒子,在电场力的作用下,两粒子同时从静止开始运动。已知两粒子同时经过一平行于正极板且与其相距$ \frac{ 2 }{ 5 } l $的平面。若两粒子间相互作用力可忽略,不计重力,则$ M : m $为 \xzanswer{A} 
\begin{figure}[h!]
	\centering
	\includesvg[width=0.23\linewidth]{picture/svg/082}
\end{figure}

\fourchoices
{$ 3:2 $}
{$ 2:1 $}
{$ 5:2 $}
{$ 3:1 $}

\item
\exwhere{$ 2016 $年上海卷}
如图,一束电子沿$ z $轴正向流动,则在图中$ y $轴上$ A $点的磁场方向是 \xzanswer{A} 
\begin{figure}[h!]
\centering
\includesvg[width=0.23\linewidth]{picture/svg/138}
\end{figure}


\fourchoices
{$ +x $方向}
{$ -x $方向}
{$ +y $方向}
{$ -y $方向}	



\item
\exwhere{$ 2015 $年理综新课标\lmd{2}卷}
如图,一质量为$ m $、电荷量为$ q $($ q>0 $)的粒子在匀强电场中运动,$ A $、$ B $为其运动轨迹上的两点。已知该粒子在$ A $点的速度大小为$ v_{0} $,方向与电场方向的夹角为$ 60 ^{ \circ } $;它运动到$ B $点时速度方向与电场方向的夹角为$ 30 ^{ \circ } $。不计重力。求$ A $、$ B $两点间的电势差。
\begin{figure}[h!]
\flushright
\includesvg[width=0.23\linewidth]{picture/svg/100}
\end{figure}


\banswer{
$U _ { A B } = \frac { m v _ { 0 } ^ { 2 } } { q }$
}



\item
\exwhere{$ 2013 $年新课标\lmd{2}卷}
如图,匀强电场中有一半径为$ r $的光滑绝缘圆轨道,轨道平面与电场方向平行。$ a $、$ b $为轨道直径的两端,该直径与电场方向平行。一电荷量为$ q $($ q>0 $)的质点沿轨道内侧运动,经过$ a $点和$ b $点时对轨道压力的大小分别为$ N_a $和$ N_b $,不计重力,求电场强度的大小$ E $、质点经过$ a $点和$ b $点时的动能。
\begin{figure}[h!]
\flushright
\includesvg[width=0.25\linewidth]{picture/svg/101}
\end{figure}

\banswer{
$E = \frac { 1 } { 6 q } \left( N _ { b } - N _ { a } \right)$\\
$E _ { k a } = \frac { r } { 12 } \left( N _ { b } + 5 N _ { a } \right)$\\
$E _ { k b } = \frac { r } { 12 } \left( 5 N _ { b } + N _ { a } \right)$
}






%磁场与安培力



\newpage
\item
\exwhere{$ 2016 $年北京卷}
中国宋代科学家沈括在《梦溪笔谈》中最早记载了地磁偏角:“以磁石磨针锋,则能指南,然常微偏东,不全南也。”进一步研究表明,地球周围地磁场的磁感线分布示意如图。结合上述材料,下列说法不正确的是 \xzanswer{C} 
\begin{figure}[h!]
	\centering
	\includesvg[width=0.17\linewidth]{picture/svg/147}
\end{figure}

\fourchoices
{地理南、北极与地磁场的南、北极不重合}
{地球内部也存在磁场,地磁南极在地理北极附近}
{地球表面任意位置的地磁场方向都与地面平行}
{地磁场对射向地球赤道的带电宇宙射线粒子有力的作用}



\item
\exwhere{$ 2017 $年新课标\lmd{3}卷}
如图,在磁感应强度大小为$ B_{0} $的匀强磁场中,两长直导线$ P $和$ Q $垂直于纸面固定放置,两者之间的距离为$ l $。在两导线中均通有方向垂直于纸面向里的电流$ I $时,纸面内与两导线距离为$ l $的$ a $点处的磁感应强度为零。如果让$ P $中的电流反向、其他条件不变,则$ a $点处磁感应强度的大小为 \xzanswer{C} 
\begin{figure}[h!]
	\centering
	\includesvg[width=0.2\linewidth]{picture/svg/148}
\end{figure}
\fourchoices
{$ 0 $}
{$\frac { \sqrt { 3 } } { 3 } B _ { 0 }$}
{$\frac { 2 \sqrt { 3 } } { 3 } B _ { 0 }$}
{$2 B _ { 0 }$}





\item
\exwhere{$ 2018 $年全国\lmd{2}卷}
如图,纸面内有两条互相垂直的长直绝缘导线$ L_{1} $、$ L_{2} $,$ L_{1} $中的电流方向向左,$ L_{2} $中的电流方向向上; $ L_{1} $的正上方有$ a $、$ b $两点,它们相对于$ L_{2} $对称。整个系统处于匀强外磁场中,外磁场的磁感应强度大小为$ B_{0} $,方向垂直于纸面向外。已知$ a $、$ b $两点的磁感应强度大小分别为$ \frac{ 1 }{ 3 } B_{0} $和$ \frac{ 1 }{ 2 } B_{0} $,方向也垂直于纸面向外。则 \xzanswer{AC} 
\begin{figure}[h!]
	\centering
	\includesvg[width=0.2\linewidth]{picture/svg/149}
\end{figure}


\fourchoices
{流经$ L_{1} $的电流在$ b $点产生的磁感应强度大小为$\frac { 7 } { 12 } B _ { 0 }$ }
{流经$ L_{1} $的电流在$ a $点产生的磁感应强度大小为$\frac { 1 } { 12 } B _ { 0 }$ }
{流经$ L_{2} $的电流在$ b $点产生的磁感应强度大小为$\frac { 1 } { 12 } B _ { 0 }$ }
{流经$ L_{2} $的电流在$ a $点产生的磁感应强度大小为$\frac { 7 } { 12 } B _ { 0 }$}


\item
\exwhere{$ 2012 $年物理海南卷}
图中装置可演示磁场对通电导线的作用。电磁铁上下两磁极之间某一水平面内固定两条平行金属导轨,$ L $是置于导轨上并与导轨垂直的金属杆。当电磁铁线圈两端$ a $、$ b $,导轨两端$ e $、$ f $,分别接到两个不同的直流电源上时,$ L $便在导轨上滑动。下列说法正确的是 \xzanswer{BD} 
\begin{figure}[h!]
	\centering
	\includesvg[width=0.23\linewidth]{picture/svg/162}
\end{figure}


\fourchoices
{若$ a $接正极,$ b $接负极,$ e $接正极,$ f $接负极,则$ L $向右滑动}
{若$ a $接正极,$ b $接负极,$ e $接负极,$ f $接正极,则$ L $向右滑动}
{若$ a $接负极,$ b $接正极,$ e $接正极,$ f $接负极,则$ L $向左滑动}
{若$ a $接负极,$ b $接正极,$ e $接负极,$ f $接正极,则$ L $向左滑动}
	
\item
\exwhere{$ 2019 $年物理全国\lmd{1}卷}
如图,等边三角形线框$ LMN $由三根相同的导体棒连接而成,固定于匀强磁场中,线框平面与磁感应强度方向垂直,线框顶点$ M $、$ N $与直流电源两端相接,已如导体棒$ MN $受到的安培力大小为$ F $,则线框$ LMN $受到的安培力的大小为 \xzanswer{B} 
\begin{figure}[h!]
	\centering
	\includesvg[width=0.23\linewidth]{picture/svg/153}
\end{figure}

\fourchoices
{$ 2F $}
{$ 1.5F $}
{$ 0.5F $}
{$ 0 $}
	

	
\end{enumerate}





