\bta{探究摩擦力的实验}
\begin{enumerate}
\renewcommand{\labelenumi}{\arabic{enumi}.}
% A(\Alph) a(\alph) I(\Roman) i(\roman) 1(\arabic)
%设定全局标号series=example	%引用全局变量resume=example
%[topsep=-0.3em,parsep=-0.3em,itemsep=-0.3em,partopsep=-0.3em]
%可使用leftmargin调整列表环境左边的空白长度 [leftmargin=0em]
\item
\exwhere{$ 2019 $ 年物理全国\lmd{2}卷}
如图($ a $),某同学设计了测量铁块与木板间动摩擦因数的实验。所用器材
有:铁架台、长木板、铁块、米尺、电磁打点计时器、频率 $ 50 \ Hz $ 的交流电源,纸带等。回答下列
问题:
\begin{enumerate}
\renewcommand{\labelenumi}{\arabic{enumi}.}
% A(\Alph) a(\alph) I(\Roman) i(\roman) 1(\arabic)
%设定全局标号series=example	%引用全局变量resume=example
%[topsep=-0.3em,parsep=-0.3em,itemsep=-0.3em,partopsep=-0.3em]
%可使用leftmargin调整列表环境左边的空白长度 [leftmargin=0em]
\item
铁块与木板间动摩擦因数$ \mu = $ \tk{$\frac{g \sin \theta-a}{g \cos \theta}$} (用木板与水平面的夹角$ \theta $、重力加速度 $ g $ 和铁块下滑的加
速度 $ a $ 表示)。
\begin{figure}[h!]
\centering
\includesvg[width=0.43\linewidth]{picture/svg/GZ-3-tiyou-0496}
\end{figure}


\item 
某次实验时,调整木板与水平面的夹角$ \theta =30 ^{ \circ } $。接通电源。开启打点计时器,释放铁块,铁块
从静止开始沿木板滑下。多次重复后选择点迹清晰的一条纸带,如图($ b $)所示。图中的点为计数
点(每两个相邻的计数点间还有 $ 4 $ 个点未画出)。重力加速度为 $ 9.8 \ m/s^{2} $。可以计算出铁块与木板间
的动摩擦因数为 \tk{$ 0.35 $} (结果保留 $ 2 $ 位小数)。
\begin{figure}[h!]
\centering
\includesvg[width=0.83\linewidth]{picture/svg/GZ-3-tiyou-0497}
\end{figure}

\end{enumerate}



\newpage
\item 
\exwhere{$ 2015 $ 年理综新课标$ \lmd{2} $卷}
某同学用图($ a $)所示的实验装置测量物块与斜面的动摩擦因数。已知打点计时器所用电源的频率
为 $ 50 \ Hz $,物块下滑过程中所得到的纸带的一部分如图($ b $)所示,图中标出了 $ 5 $ 个连续点之间的距
离。
\begin{figure}[h!]
\centering
\includesvg[width=0.83\linewidth]{picture/svg/GZ-3-tiyou-0498}
\end{figure}


\begin{enumerate}
\renewcommand{\labelenumi}{\arabic{enumi}.}
% A(\Alph) a(\alph) I(\Roman) i(\roman) 1(\arabic)
%设定全局标号series=example	%引用全局变量resume=example
%[topsep=-0.3em,parsep=-0.3em,itemsep=-0.3em,partopsep=-0.3em]
%可使用leftmargin调整列表环境左边的空白长度 [leftmargin=0em]
\item
物块下滑时的加速度 $ a= $ \tk{$ 3.25 $} $ m/s^{2} $,打 $ C $ 点时物块的速度 $ v= $ \tk{$ 1.79 $} $ m/s $;
\item 
已知重力加速度大小为 $ g $,为求出动摩擦因数,还必须测量的物理量是
\tk{C} 
(填
正确答案标号)
\threechoices
{物块的质量}
{斜面的高度}
{斜面的倾角}

\end{enumerate}




\newpage
\item
\exwhere{$ 2013 $ 年新课标 \lmd{1} 卷}
图$ (a) $为测量物块与水平桌面之间动摩擦因数的实验装置示意图。实验步骤如下$ : $ 

①用天平测量物块
和遮光片的总质量 $ M $、重物的质量 $ m $;用游标卡尺测量遮光片的宽度 $ d $;用米尺测最两光电门之间
的距离 $ s $;

②调整轻滑轮,使细线水平$ : $


③让物块从光电门 $ A $ 的左侧由静止
释放,用数字毫秒计分别测出遮光片
经过光电门 $ A $ 和光电门 $ B $ 所用的时间
$ \triangle t_{A} $ 和$ \triangle t_{B} $,求出加速度 $ a $;

④多次重复步骤③,求 $ a $ 的平均 $ \bar{a} $;

⑤根据上述实验数据求出动擦因数$ \mu $。

\begin{figure}[h!]
\centering
\includesvg[width=0.43\linewidth]{picture/svg/GZ-3-tiyou-0499}
 \qquad 
 \includesvg[width=0.43\linewidth]{picture/svg/GZ-3-tiyou-0500} 
\end{figure}




回答下列问题:
\begin{enumerate}
\renewcommand{\labelenumi}{\arabic{enumi}.}
% A(\Alph) a(\alph) I(\Roman) i(\roman) 1(\arabic)
%设定全局标号series=example	%引用全局变量resume=example
%[topsep=-0.3em,parsep=-0.3em,itemsep=-0.3em,partopsep=-0.3em]
%可使用leftmargin调整列表环境左边的空白长度 [leftmargin=0em]
\item
测量 $ d $ 时,某次游标卡尺(主尺的最小分度为 $ 1 \ mm) $的示如图($ b $)所示。其读数为
\tk{$ 0.960 $} 
$ cm $。

\item 
物块的加速度 $ a $ 可用 $ d $、$ s $、$ \triangle t_{A} $,和$ \triangle t_{B} $,表示为 $ a =$ \tk{$\frac{v_{2}^{2}-v_{1}^{2}}{2 s}=\frac{1}{2 s}\left(\frac{d^{2}}{\Delta t_{B}^{2}}-\frac{d^{2}}{\Delta t_{A}^{2}}\right)$}。 



\item 
动摩擦因数$ \mu $可用 $ M $、$ m $、 $ \bar{a} $;和重力加速度 $ g $ 表示为$ \mu = $ \tk{$\frac{m_{1} g-\left(M+m_{1}\right) a}{M g}$} ;

\item 
如果细线没有调整到水平,由此引起的误差属于
\tk{系统误差} 
(填“偶然误差”或”系统误差” )。



\end{enumerate}




\newpage
\item
\exwhere{$ 2013 $ 年全国卷大纲卷}
测量小物块 $ Q $ 与平板 $ P $ 之间的动摩擦因数的实验装置如图所示。$ AB $ 是半径足够大的
光滑的四分之一圆弧轨道,与水平固定放置的 $ P $ 板的上表面 $ BC $ 在 $ B $ 点相切,$ C $ 点在水平地面的垂
直投影为 $ C ^{\prime} $。重力加速度大小为 $ g $。实验步骤如下:


①用天平称出物块 $ Q $ 的质量 $ m $;

②测量出轨道 $ AB $ 的半径 $ R $、$ BC $ 的长度 $ L $ 和 $ CC ^{\prime} $的长度 $ h $;


③将物块 $ Q $ 在 $ A $ 点从静止释放,在物块 $ Q $ 落地处标记其落点 $ D $;





④重复步骤③,共做 $ 10 $ 次;

⑤将 $ 10 $ 个落地点用一个尽量小的圆围住,用米尺测量圆心到 $ C ^{\prime} $ 的距离 $ s $。
\begin{figure}[h!]
\centering
\includesvg[width=0.38\linewidth]{picture/svg/GZ-3-tiyou-0501}
\end{figure}

\begin{enumerate}
\renewcommand{\labelenumi}{\arabic{enumi}.}
% A(\Alph) a(\alph) I(\Roman) i(\roman) 1(\arabic)
%设定全局标号series=example	%引用全局变量resume=example
%[topsep=-0.3em,parsep=-0.3em,itemsep=-0.3em,partopsep=-0.3em]
%可使用leftmargin调整列表环境左边的空白长度 [leftmargin=0em]
\item
用实验中的测量量表示:
\begin{enumerate}
\renewcommand{\labelenumiii}{\roman{enumiii}.}
% A(\Alph) a(\alph) I(\Roman) i(\roman) 1(\arabic)
%设定全局标号series=example	%引用全局变量resume=example
%[topsep=-0.3em,parsep=-0.3em,itemsep=-0.3em,partopsep=-0.3em]
%可使用leftmargin调整列表环境左边的空白长度 [leftmargin=0em]
\item
物块 $ Q $ 到达 $ B $ 点时的动能 $ E_{kB} = $
\tk{$ mgR $} 
;

\item 
物块 $ Q $ 到达 $ C $ 点时的动能 $ E_{kC} = $
\tk{$\frac{m g s^{2}}{4 h}$} 
;

\item 
在物块 $ Q $ 从 $ B $ 运动到 $ C $ 的过程中,物块 $ Q $ 克服摩擦力做的功 $ W_f= $ \tk{$m g R-\frac{m g s^{2}}{4 h}$} 。

\item 
物块 $ Q $ 与平板 $ P $ 之间的动摩擦因数$ \mu = $ \tk{$\frac{R}{L}-\frac{s^{2}}{4 h L}$}。 




\end{enumerate}



\item 
回答下列问题:

\begin{enumerate}
\renewcommand{\labelenumiii}{\roman{enumiii}.}
% A(\Alph) a(\alph) I(\Roman) i(\roman) 1(\arabic)
%设定全局标号series=example	%引用全局变量resume=example
%[topsep=-0.3em,parsep=-0.3em,itemsep=-0.3em,partopsep=-0.3em]
%可使用leftmargin调整列表环境左边的空白长度 [leftmargin=0em]
\item
实验步骤④⑤的目的是
\tk{减小实验偶然误差} 
。

\item 
已知实验测得的$ \mu $值比实际值偏大,其原因除了实验中测量量的误差之外,其它的可能是 \tk{圆弧轨道存在摩擦、接缝 $ B $ 处不平滑} 
。(写出一个可能的原因即可)


\end{enumerate}



\end{enumerate}



\newpage
\item 
\exwhere{$ 2012 $ 年物理江苏卷}
为测定木块与桌面之间的动摩擦因数,小亮设计了
如图所示的装置进行实验. 实验中,当木块 $ A $ 位于水平桌面
上的 $ O $ 点时,重物 $ B $ 刚好接触地面. 将 $ A $ 拉到 $ P $ 点,待 $ B $ 稳定
后静止释放,$ A $ 最终滑到 $ Q $ 点. 分别测量 $ OP $、$ OQ $ 的长度 $ h $
和 $ s $. 改变 $ h $,重复上述实验,分别记录几组实验数据.
\begin{figure}[h!]
\centering
\includesvg[width=0.83\linewidth]{picture/svg/GZ-3-tiyou-0503}
\end{figure}

\begin{enumerate}
\renewcommand{\labelenumi}{\arabic{enumi}.}
% A(\Alph) a(\alph) I(\Roman) i(\roman) 1(\arabic)
%设定全局标号series=example	%引用全局变量resume=example
%[topsep=-0.3em,parsep=-0.3em,itemsep=-0.3em,partopsep=-0.3em]
%可使用leftmargin调整列表环境左边的空白长度 [leftmargin=0em]
\item
实验开始时,发现 $ A $ 释放后会撞到滑轮. 请提出两个解决
方法.

\tk{减小 $ B $ 的质量,增加细线的长度,或增大 $ A $ 的质量,降低 $ B $ 的起始高度} 

\item 
请根据下表的实验数据作出 $ s-h $ 关系的图象.
\begin{table}[h!]
\centering 
\begin{tabular}{|c|c|c|c|c|c|}
\hline 
$ h(cm) $ & $ 20.0 $ & $ 30.0 $ & $ 40.0 $ & $ 50.0 $ & $ 60.0 $
 \\
\hline
$ s(cm) $ & $ 19.5 $ & $ 28.5 $ & $ 39.0 $ & $ 48.0 $ & $ 56.5 $\\ 
\hline 
\end{tabular}
\end{table} 

\banswer{
 \includesvg[width=0.23\linewidth]{picture/svg/GZ-3-tiyou-0504} 
}


\item 
实验测得 $ A $、$ B $ 的质量分别为 $ m=0.40 \ kg $、$ M=0.50 \ kg $. 根据 $ s-h $ 图象可计算出 $ A $ 木块与桌面间的
动摩擦因数$ \mu = $ \tk{$ 0.4 $} (结果保留一位有效数字)。

\item 
实验中,滑轮轴的摩擦会导致$ \mu $的测量结果 \tk{偏大} (选填“偏大”或“偏小”).

\end{enumerate}



\banswer{

}


\newpage
\item 
\exwhere{$ 2011 $ 年理综山东卷}
某探究小组设计了“用一把尺子测定动摩擦因数”的实验方案。如图所示,将一个小球和一个滑块用细绳
连接,跨在斜面上端。开始时小球和滑块均静止,剪断细绳后,小球自由下落,
滑块沿斜面下滑,可先后听到小球落地和滑块撞击挡板的声音。保持小球和滑
块释放的位置不变,调整挡板位置,重复以上操作,直到能同时听到小球落地
和滑块撞击挡板的声音。用刻度尺测出小球下落的高度 $ H $、滑块释放点与挡板
处的高度差 $ h $ 和沿斜面运动的位移 $ x $。
(空气阻力对本实验的影响可以忽略)
\begin{figure}[h!]
\centering 
\includesvg[width=0.23\linewidth]{picture/svg/GZ-3-tiyou-0505}
\end{figure}


①滑块沿斜面运动的加速度与重力加速度的比值为 \tk{$ \frac{x}{H} $} 。


②滑块与斜面间的动摩擦因数为 \tk{$\left(h-\frac{x^{2}}{H}\right) \frac{1}{\sqrt{x^{2}-h^{2}}}$} 。


③以下能引起实验误差的是 \tk{CD} 。
\fourchoices
{滑块的质量}
{当地重力加速度的大小}
{长度测量时的读数误差}
{小球落地和滑块撞击挡板不同时}




\newpage
\item
\exwhere{$ 2018 $ 年全国\lmd{2}卷}
某同学用图($ a $)所示的装置测量木块与木板之间的动摩擦
因数。跨过光滑定滑轮的细线两端分别与木块和弹簧秤相
连,滑轮和木块间的细线保持水平,在木块上方放置砝码。
缓慢向左拉动水平放置的木板,当木块和砝码相对桌面静止
且木板仍在继续滑动时,弹簧秤的示数即为木块受到的滑动
摩擦力的大小。某次实验所得数据在下表中给出,其中 $ f_{4} $ 的值
可从图($ b $)中弹簧秤的示数读出。
\begin{figure}[h!]
\centering
\includesvg[width=0.33\linewidth]{picture/svg/GZ-3-tiyou-0506}
\end{figure}

\begin{table}[h!]
\centering 
\begin{tabular}{|c|c|c|c|c|c|}
\hline 
砝码的质量$ m/kg $ & $ 0.05 $ & $ 0.10 $ & $ 0.15 $ & $ 0.20 $ & $ 0.25 $
 \\
\hline
滑动摩擦力$ f/N $ & $ 2.15 $ & $ 2.36 $ & $ 2.55 $ & $ f_{4} $ & $ 2.93 $\\ 
\hline 
\end{tabular}
\end{table} 
\begin{figure}[h!]
\centering
\includesvg[width=0.83\linewidth]{picture/svg/GZ-3-tiyou-0507}
\end{figure}




回答下列问题:
\begin{enumerate}
\renewcommand{\labelenumi}{\arabic{enumi}.}
% A(\Alph) a(\alph) I(\Roman) i(\roman) 1(\arabic)
%设定全局标号series=example	%引用全局变量resume=example
%[topsep=-0.3em,parsep=-0.3em,itemsep=-0.3em,partopsep=-0.3em]
%可使用leftmargin调整列表环境左边的空白长度 [leftmargin=0em]
\item
$ f_4= $ \tk{$ 2.75 $} $ N $;




\item 
在图($ c $)的坐标
纸上补齐未画出的数据
点并绘出 $ f-m $ 图线;
\banswer{
 \includesvg[width=0.23\linewidth]{picture/svg/GZ-3-tiyou-0508} 
}


\item 
$ f $ 与 $ m $、木块质量
$ M $、木板与木块之间的
动摩擦因数$ \mu $及重力加
速度大小 $ g $ 之间的关系式为 $ f= $
\tk{$ \mu(M+m)g $} 
, $ f-m $ 图线(直线)的斜率的表达式为 $ k= $ \tk{$ \mu g $}; 

\item 
取 $ g=9.80 \ m/s^{2} $,由绘出的 $ f-m $ 图线求得 $ \mu= $
\tk{$ 0.40 $} 。(保留 $ 2 $ 位有效数字)
\end{enumerate}




\newpage
\item 
\exwhere{$ 2011 $ 年理综重庆卷}
某同学设计了如图 $ 3 $ 所示的装置,利用米尺、秒表、轻绳、轻滑轮、轨道、滑块、托盘和
砝码等器材来测定滑块和轨道间的动摩擦因数$ \mu $。滑块
和托盘上分别放有若干砝码,滑块质量为 $ M $,滑块上
砝码总质量为 $ m ^{\prime} $,托盘和盘中砝码的总质量为 $ m $。实
验中,滑块在水平轨道上从 $ A $ 到 $ B $ 做初速为零的匀加
速直线运动,重力加速度 $ g $ 取 $ 10 \ m/s^{2} $。
\begin{figure}[h!]
\centering
\includesvg[width=0.63\linewidth]{picture/svg/GZ-3-tiyou-0509}
\end{figure}

①为测量滑块的加速度 $ a $,须测出它在 $ A $、$ B $ 间运动的
\tk{位移} 
与
\tk{时间} 
,计算 $ a $ 的运动学公式是
\tk{$a=\frac{2 s}{t^{2}}$} 
;


②根据牛顿运动定律得到 $ a $ 与 $ m $ 的关系为:
$a=\frac{(1+\mu) g}{M+\left(m^{\prime}+m\right)} m-\mu g$,他想通过多次改
变 $ m $,测出相应的 $ a $ 值,并利用上式来计算
$ \mu $。若要求 $ a $ 是 $ m $ 的一次函数,必须使上式
中的保持 \tk{$m+m^{\prime}$} 不变,实验中应将从托盘中
取出的砝码置于
\tk{滑块上} 
;

③实验得到 $ a $ 与 $ m $ 的关系如图 $ 4 $ 所示,由
此可知$ \mu = $ \tk{0.23(0.21~0.25)}。
\begin{figure}[h!]
\centering
\includesvg[width=0.53\linewidth]{picture/svg/GZ-3-tiyou-0510}
\end{figure}



\newpage
\item 
\exwhere{$ 2014 $ 年理综山东卷}
某实验小组利用弹簧秤和刻度尺,测量滑块在木板上运动的最大速度。实验步骤:

①用弹簧秤测量橡皮泥和滑块的总重力,记作 $ G $;

②将装有橡皮泥的滑块放在水平木板上,通过水平细绳和固定弹簧秤相连,如图甲所示。在 $ A $ 端向
右拉动木板,等弹簧秤示数稳定后,将读数记作 $ F $;

③改变滑块上橡皮泥的质量,重复步骤①②;实验数据如下表所示:
\begin{table}[h!]
\centering 
\begin{tabular}{|c|c|c|c|c|c|c|}
\hline 
$ G/N $ & $ 1.50 $ & $ 2.00 $ & $ 2.50 $ & $ 3.00 $ & $ 3.50 $ & $ 4.00 $
 \\
\hline
$ F/N $ & $ 0.59 $ & $ 0.83 $ & $ 0.99 $ & $ 1.22 $ & $ 1.37 $ & $ 1.61 $\\ 
\hline 
\end{tabular}
\end{table} 



④如图乙所示,将木板固定在水平桌面上,滑块置于木板上左端 $ C $ 处,细绳跨过定滑轮分别与滑块
和重物 $ P $ 连接,保持滑块静止,测量重物 $ P $ 离地面的高度 $ h $;


⑤滑块由静止释放后开始运动并最终停在木板上的 $ D $ 点(未与滑轮碰撞)
,测量 $ C $、$ D $ 间的距离 $ s $。

完成下列作图和填空:
\begin{enumerate}
\renewcommand{\labelenumi}{\arabic{enumi}.}
% A(\Alph) a(\alph) I(\Roman) i(\roman) 1(\arabic)
%设定全局标号series=example	%引用全局变量resume=example
%[topsep=-0.3em,parsep=-0.3em,itemsep=-0.3em,partopsep=-0.3em]
%可使用leftmargin调整列表环境左边的空白长度 [leftmargin=0em]
\item
根据表中数据在给定的坐标纸(见答题卡)上作出 $ F - G $ 图线。
\begin{figure}[h!]
\centering
\includesvg[width=0.83\linewidth]{picture/svg/GZ-3-tiyou-0511}
\end{figure}

\banswer{
 \includesvg[width=0.23\linewidth]{picture/svg/GZ-3-tiyou-0512} 
}


\item 
由图线求得滑块和木板间的动摩擦因数$ \mu = $ \tk{)$ 0.40 $($ 0.38 $、$ 0.39 $、$ 0.41 $、$ 0.42 $ 均正确)} (保留 $ 2 $ 位有效数字)。


\item 
滑块最大速度的大小 $ v= $ \tk{$v=\sqrt{2 \mu g(s-h)}$} (用 $ h $、$ s $、$ \mu $和重力加速度 $ g $ 表示。
)

\end{enumerate}


\banswer{

}





\end{enumerate}

