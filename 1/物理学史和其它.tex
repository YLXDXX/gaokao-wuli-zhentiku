\bta{物理学史和其它}

\begin{enumerate}
	%\renewcommand{\labelenumi}{\arabic{enumi}.}
	% A(\Alph) a(\alph) I(\Roman) i(\roman) 1(\arabic)
	%设定全局标号series=example	%引用全局变量resume=example
	%[topsep=-0.3em,parsep=-0.3em,itemsep=-0.3em,partopsep=-0.3em]
	%可使用leftmargin调整列表环境左边的空白长度 [leftmargin=0em]
	\item
\exwhere{$ 2019 $ 年 $ 4 $ 月浙江物理选考}
下列物理量属于基本量且单位属于国际单位制中基本单位的是 \xzanswer{B} 

\fourchoices
{功/焦耳}
{质量/千克}
{电荷量/库仑}
{力/牛顿}



\item 
\exwhere{$ 2019 $ 年 $ 4 $ 月浙江物理选考}
下列式子属于比值定义物理量的是 \xzanswer{C} 

\fourchoices
{$t=\frac{\Delta x}{v}$}
{$a=\frac{F}{m}$}
{$C=\frac{Q}{U}$}
{$I=\frac{U}{R}$}




\item 
\exwhere{$ 2019 $ 年物理北京卷}
国际单位制(缩写 $ SI $)定义了米($ m $)、秒($ s $)等 $ 7 $ 个基本单位,其他单位
均可由物理关系导出。例如,由 $ m $ 和 $ s $ 可以导出速度单位 $ m \cdot s^{-1} $。历史上,曾用“米原器”定义米,
用平均太阳日定义秒。但是,以实物或其运动来定义基本单位会受到环境和测量方式等因素的影响,
而采用物理常量来定义则可避免这种困扰。$ 1967 $ 年用铯$- 133 $ 原子基态的两个超精细能级间跃迁辐
射的频率$ \Delta\nu =9192631770 \ Hz $ 定义 $ s $;$ 1983 $ 年用真空中的光速 $ c=299792458 \ m \cdot s  ^{-1} $ 定义 $ m $。$ 2018 $ 年
第 $ 26 $ 届国际计量大会决定,$ 7 $ 个基本单位全部用基本物理常量来定义(对应关系如图,例如,$ s $ 对
应$ \Delta \nu $,$ m $ 对应 $ c $)。新 $ SI $ 自 $ 2019 $ 年 $ 5 $ 月 $ 20 $ 日(国际计量日)正式实施,这将对科学和技术发展产
生深远影响。下列选项不正确的是 \xzanswer{D} 
\begin{figure}[h!]
	\centering
	\includesvg[width=0.23\linewidth]{picture/svg/GZ-3-tiyou-1429}
\end{figure}


\fourchoices
{$ 7 $ 个基本单位全部用物理常量定义,保证了基本单位的稳定性}
{用真空中的光速 $ c $($ m \cdot s ^{-1} $)定义 $ m $,因为长度 $ l $ 与速度 $ v $ 存在 $ l=vt $,而 $ s $ 已定义}
{用基本电荷 $ e(C) $定义安培$ (A) $,因为电荷量与电流 $ I $ 存在 $ I=q/t $,而 $ s $ 已定义}
{因为普朗克常量 $ h $($ J \cdot s $)的单位中没有 $ kg $,所以无法用它来定义质量单位}


\item 
\exwhere{$ 2019 $ 年物理天津卷}
第 $ 26 $ 届国际计量大会决定,质量单位“千克”用普朗克常量 $ h $ 定义,“国际千
克原器”于 $ 2019 $ 年 $ 5 $ 月 $ 20 $ 日正式“退役” $ h $ 的数值为 $ 6.63 \times 10^{-34} $,根据能量子定义, $ h $ 的单位是 \underlinegap 
,该单位用国际单位制中的力学基本单位表示,则为 \underlinegap 。

 \tk{$ J \cdot s $ \quad $ kg \cdot m^{2}/s $} 

\item 
\exwhere{$ 2018 $年浙江卷($ 4 $月选考)}
通过理想斜面实验得出“力不是维持物体运动的原因”的科学家是 \xzanswer{B} 

\fourchoices
{亚里士多德}
{伽利略}
{笛卡尔}
{牛顿}



\item
\exwhere{$ 2018 $ 年浙江卷($ 4 $ 月选考)}
用国际单位制的基本单位表示能量的单位,下列正确的是 \xzanswer{A} 

\fourchoices
{$ kg \cdot m_{2}/s^{2} $}
{$ kg \cdot m/s^{2} $}
{$ N/m $}
{$ N \cdot m $}



\item
\exwhere{$ 2017 $ 年浙江选考卷}
下列描述正确的是 \xzanswer{A} 

\fourchoices
{开普勒提出所有行星绕太阳运动的轨道是椭圆}
{牛顿通过实验测出了万有引力常数}
{库伦通过扭秤实验测定了电子的电荷量}
{法拉第发现了电流的磁效应}



\item 
\exwhere{$ 2016 $ 年新课标 \lmd{3} 卷}
 关于行星运动的规律,下列说法符合史实的是 \xzanswer{B} 
 
\fourchoices
{开普勒在牛顿定律的基础上,导出了行星运动的规律}
{开普勒在天文观测数据的基础上,总结出了行星运动的规律}
{开普勒总结出了行星运动的规律,找出了行星按照这些规律运动的原因}
{开普勒总结出了行星运动的规律,发现了万有引力定律}



\item 
\exwhere{$ 2016 $ 年天津卷}
物理学家通过对实验的深入观察和研究,获得正确的科学认知,推动物理学的
发展,下列说法符合事实的是 \xzanswer{AC} 

\fourchoices
{赫兹通过一系列实验证实了麦克斯韦关于光的电磁理论}
{查德威克用$ \alpha $粒子轰击 $ ^{14}_{7}N $获得反冲核 $ ^{17}_{8}O $,发现了中子}
{贝克勒尔发现的天然放射性现象,说明原子核有复杂结构}
{卢瑟福通过对阴极射线的研究,提出了原子核式结构模型}



\item 
\exwhere{$ 2014 $ 年物理海南卷}
下列说法中,符合物理学史实的是 \xzanswer{ABD} 


\fourchoices
{亚里士多德认为,必须有力作用在物体上,物体才能运动;没有力的作用,物体就静止}
{牛顿认为,力是物体运动状态改变的原因,而不是物体运动的原因}
{麦克斯韦发现了电流的磁效应,即电流可以在其周围产生磁场}
{奥斯特发现导线通电时,导线附近的小磁针发生偏转}


\item 
\exwhere{$ 2013 $ 年新课标 \lmd{2} 卷}
在物理学发展过程中,观测、实验、假说和逻辑推理等方法都起到了重要作用。下列叙述符合
史实的是 \xzanswer{ABD} 

\fourchoices
{奥斯特在实验中观察到电流的磁效应,该效应揭示了电和磁之间存在联系}
{安培根据通电螺线管的磁场和条形磁铁的磁场的相似性,提出了分子电流假说}
{法拉第在实验中观察到,在通有恒定电流的静止导线附近的固定导线圈中,会出现感应电流}
{楞次在分析了许多实验事实后提出,感应电流应具有这样的方向,即感应电流的磁场总要阻碍引起感应电流的磁通量的变化}



\item 
\exwhere{$ 2013 $ 年海南卷}
科学家关于物体运动的研究对树立正确的自然观具有重要作用。下列说法符合历史事实的是 \xzanswer{BCD} 

\fourchoices
{亚里士多德认为,必须有力作用在物体上,物体的运动状态才会改变}
{伽利略通过“理想实验”得出结论:运动必具有一定速度,如果它不受力,它将以这一速度永远运动下去}
{笛卡儿指出:如果运动中的物体没有受到力的作用,它将继续以同一速度沿同一直线运动,既不停下来也不偏离原来的方向}
{牛顿认为,物体具有保持原来匀速直线运动状态或静止状态的性质}



\item 
\exwhere{$ 2012 $年理综山东卷}
以下叙述正确的是 \xzanswer{AD} 


\fourchoices
{法拉第发现了电磁感应现象}
{惯性是物体的固有属性,速度大的物体惯性一定大}
{牛顿最早通过理想斜面实验得出力不是维持物体运动的原因}
{感应电流遵从楞次定律所描述的方向,这是能量守恒定律的必然结果}


\item
\exwhere{$ 2011 $ 年海南卷}
自然界的电、热和磁等现象都是相互联系的,很多物理学家为寻找它们之间的联系做出了贡献。
下列说法正确的是 \xzanswer{ACD} 


\fourchoices
{奥斯特发现了电流的磁效应,揭示了电现象和磁现象之间的联系}
{欧姆发现了欧姆定律,说明了热现象和电现象之间存在联系}
{法拉第发现了电磁感应现象,揭示了磁现象和电现象之间的联系}
{焦耳发现了电流的热效应,定量给出了电能和热能之间的转换关系}


\item
\exwhere{$ 2011 $ 年理综山东卷}
了解物理规律的发现过程,学会像科学家那样观察和思考,往往比掌握知识本身更重要。以下符
合事实的是 \xzanswer{AB} 

\fourchoices
{焦耳发现了电流热效应的规律}
{库仑总结出了点电荷间相互作用的规律}
{楞次发现了电流的磁效应,拉开了研究电与磁相互关系的序幕}
{牛顿将斜面实验的结论合理外推,间接证明了自由落体运动是匀变速直线运动}



\item
\exwhere{$ 2015 $ 年理综天津卷}
物理学重视逻辑,崇尚理性,其理论总是建立在对事实观察的基础上。
下列说法正确的是 \xzanswer{A} 


\fourchoices
{天然放射现象说明原子核内部是有结构的}
{电子的发现使人认识到原子具有核式结构}
{$ \alpha $粒子散射实验的重要发现是电荷是量子化的}
{密立根油滴实验表明核外电子的轨道是不连续的}


\item 
\exwhere{$ 2016 $ 年上海卷}
国际单位制中,不是电场强度的单位是 \xzanswer{C} 

\fourchoices
{$ N/C $}
{$ V/m $}
{$ J/C $}
{$ T \cdot m/s $}


\item 
\exwhere{$ 2013 $ 年福建卷}
在国际单位制(简称 $ SI $)中,力学和电学的基本单位有:$ m $(米)
、$ kg $(千克)
、$ s $(秒)
、$ A $(安
培)
。导出单位 $ V $(伏特)用上述基本单位可表示为 \xzanswer{B} 


\fourchoices
{$m^{2} \cdot kg \cdot s^{-4} \cdot A^{-1}$}
{$m^{2} \cdot kg \cdot s^{-3} \cdot A^{-1}$}
{$m^{2} \cdot kg \cdot s^{-2} \cdot A^{-1}$}
{$m^{2} \cdot kg \cdot s^{-1} \cdot A^{-1}$}



\item 
\exwhere{$ 2011 $ 年理综北京卷}
物理关系式不仅反映了物理量之间的关系,也确定了单位间的关系。如关系式 $ U=IR $ 既反映了
电压、电流和电阻之间的关系,也确定了 $ V $(伏)与 $ A $(安)和$ \Omega $(欧)的乘积等效。现有物理量
单位:$ m $(米)、$ s $(秒)
、$ N $(牛)
、$ J $(焦)、$ W $(瓦)、$ C $(库)
、$ F $(法)、$ A $(安)、$ \Omega $(欧)和 $ T $(特),
由他们组合成的单位都与电压单位 $ V $(伏)等效的是 \xzanswer{B} 

\fourchoices
{$ J/C $ 和 $ N/C $}
{$ C/F $ 和$ T \cdot m^{2} /s $}
{$ W/A $ 和 $ C \cdot T \cdot m/s $}
{$W^{\frac{1}{2}} \cdot \Omega^{\frac{1}{2}}$ 和 $T \cdot A \cdot m$}



\item 
\exwhere{$ 2015 $ 年理综浙江卷}
下列说法正确的是 \xzanswer{C} 


\fourchoices
{电流通过导体的热功率与电流大小成正比}
{力对物体所做的功与力的作用时间成正比}
{电容器所带电荷量与两极板间的电势差成正比}
{弹性限度内,弹簧的劲度系数与弹簧伸长量成正比}


\item
\exwhere{$ 2017 $ 年浙江选考卷}
下面物理量及其对应的国际单位制单位符号,正确的是 \xzanswer{D} 

\fourchoices
{力,$ kg $}
{功率,$ J $}
{电场强度,$ C/N $}
{电压,$ V $}



\item 
\exwhere{$ 2017 $ 年浙江选考卷}
下列各组物理量中均为矢量的是 \xzanswer{B} 

\fourchoices
{路程和位移}
{速度和加速度}
{力和功}
{电场强度和电势}






	
	
	
\end{enumerate}

