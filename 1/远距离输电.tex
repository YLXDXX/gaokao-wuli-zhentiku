\bta{远距离输送电}

\begin{enumerate}
%\renewcommand{\labelenumi}{\arabic{enumi}.}
% A(\Alph) a(\alph) I(\Roman) i(\roman) 1(\arabic)
%设定全局标号series=example	%引用全局变量resume=example
%[topsep=-0.3em,parsep=-0.3em,itemsep=-0.3em,partopsep=-0.3em]
%可使用leftmargin调整列表环境左边的空白长度 [leftmargin=0em]
\item
\exwhere{$ 2014 $ 年理综四川卷}
如图所示,甲是远距离的输电示意图,乙是发电机输出电压随时间变化的图像,则 \xzanswer{D} 
\begin{figure}[h!]
\centering
\begin{subfigure}{0.4\linewidth}
\centering
\includesvg[width=0.7\linewidth]{picture/svg/GZ-3-tiyou-1197} 
\caption{}\label{}
\end{subfigure}
\begin{subfigure}{0.4\linewidth}
\centering
\includesvg[width=0.7\linewidth]{picture/svg/GZ-3-tiyou-1198} 
\caption{}\label{}
\end{subfigure}
\end{figure}



\fourchoices
{用户用电器上交流电的频率是 $ 100 \ Hz $}
{发电机输出交流电的电压有效值是 $ 500 \ V $}
{输电线的电流只由降压变压器原副线圈的匝数比决定}
{当用户用电器的总电阻增大时,输电线上损失功率减小}


\item 
\exwhere{$ 2014 $ 年物理江苏卷}
远距离输电的原理图如图所示, 升压变压器原、副线圈的匝数分别为 $ n_{1} $、 $ n_{2} $, 电压分别为
$ U_{1} $、$ U_{2} $,电流分别为 $ I_{1} $、$ I_{2} $,输电线上的电阻为 $ R $。变
压器为理想变压器,则下列关系式中正确的是 \xzanswer{D} 
\begin{figure}[h!]
\centering
\includesvg[width=0.23\linewidth]{picture/svg/GZ-3-tiyou-1199}
\end{figure}


\fourchoices
{$\frac{I_{1}}{I_{2}}=\frac{n_{1}}{n_{2}}$}
{$I_{2}=\frac{U_{2}}{R}$}
{$I_{1} U_{1}=I_{2}^{2} R$}
{$I_{1} U_{1}=I_{2} U_{2}$}


\item 
\exwhere{$ 2012 $ 年理综天津卷}
通过一理想变压器,经同一线路输送相同电功率 $ P $,原线圈的电压 $ U $ 保持不变,输电线路的总电
阻为 $ R $。当副线圈与原线圈的匝数比为 $ k $ 时,线路损耗的电功率为 $ P_{1} $,若将副线圈与原线圈的匝数
比提高到 $ nk $,线路损耗的电功率为 $ P_{2} $, 则 $ P_{1} $ 和 $ P_{2} / P_{1} $ 分别为 \xzanswer{D} 


\fourchoices
{$ PR/kU $, $ 1/n $}
{$ (P/kU)^{2}R $,$ 1/n $}
{$ PR/kU $,$ 1/ n^{2} $}
{$ (P/kU)^{2}R $,$ 1/ n^{2} $}


\item 
\exwhere{$ 2014 $ 年理综浙江卷}
如图所示为远距离交流输电的简化电路图。发电厂的输出电压是 $ U $,用等效总电阻是 $ r $ 的两条输
电线输电,输电线路中的电流是 $ I_{1} $,其末端间的电压为 $ U_{1} $。在输电线与用户间连有一想想变压器,
流入用户端的电流是 $ I_{2} $。则 \xzanswer{A} 
\begin{figure}[h!]
\centering
\includesvg[width=0.23\linewidth]{picture/svg/GZ-3-tiyou-1200}
\end{figure}

\fourchoices
{用户端的电压为 $ I_{1} U_{1} / I_{2} $}
{输电线上的电压降为 $ U $}
{理想变压器的输入功率为 $ I_{1} ^{2} r $}
{输电线路上损失的电功率为 $ I_{1} U $}


\item 
\exwhere{$ 2018 $ 年江苏卷}
采用 $ 220 \ kV $ 高压向远方的城市输电.当输送功率一定时,为使输电线上损耗的
功率减小为原来的$ \frac{ 1 }{ 4 } $ ,输电电压应变为 \xzanswer{C} 


\fourchoices
{$ 55 \ kV $}
{$ 110 \ kV $}
{$ 440 \ kV $}
{$ 880 \ kV $}


\item 
\exwhere{$ 2014 $ 年理综福建卷}
图为模拟远距离输电实验电路图,两理想变压器的匝数 $ n_{1} = n_{4} < n_{2} = n_{3} $,四根模拟输电线的电阻
$ R_{1} $、$ R_{2} $、$ R_{3} $、$ R_{4} $ 的阻值均为 $ R $,$ A_{1} $、$ A_{2} $ 为相同的理想交流电流表,$ L_{1} $、$ L_{2} $ 为相同的小灯泡,灯丝电
阻 $ R_{L} >2R $,忽略灯丝电阻随温度的变化。当 $ A $、$ B $
端接入低压交流电源时 \xzanswer{D} 
\begin{figure}[h!]
\centering
\includesvg[width=0.23\linewidth]{picture/svg/GZ-3-tiyou-1201}
\end{figure}


\fourchoices
{$ A_{1} $、$ A_{2} $ 两表的示数相同}
{$ L_{1} $、$ L_{2} $ 两灯泡的亮度相同}
{$ R_{1} $ 消耗的功率大于 $ R_{3} $ 消耗的功率}
{$ R_{2} $ 两端的电压小于 $ R_{4} $ 两端的电压}



\item 
\exwhere{$ 2015 $ 年理综福建卷}
图为远距离输电示意图,两变压器均为理想变压器,升压变压器 $ T $ 的原、
副线圈匝数分别为 $ n_{1} $、$ n_{2} $,在 $ T $ 的原线圈两端接入一电压 $ u= U_{m} \sin \omega t $ 的交流电源,若输送电功率为
$ P $,输电线的总电阻为 $ 2r $,不考虑其它因素的影响,则输电线上的损失电功率为 \xzanswer{C} 
\begin{figure}[h!]
\centering
\includesvg[width=0.23\linewidth]{picture/svg/GZ-3-tiyou-1202}
\end{figure}



\fourchoices
{$\left(\frac{n_{1}}{n_{2}}\right) \frac{U_{m}^{2}}{4 r}$}
{$\left(\frac{n_{2}}{n_{1}}\right) \frac{U_{m}^{2}}{4 r}$}
{$4\left(\frac{n_{1}}{n_{2}}\right)^{2}\left(\frac{P}{U_{m}}\right)^{2} \cdot r$}
{$4\left(\frac{n_{2}}{n_{1}}\right)^{2}\left(\frac{P}{U_{m}}\right)^{2} \cdot r$}









\end{enumerate}

