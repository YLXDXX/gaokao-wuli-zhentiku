\bta{磁场的基本概念}
\begin{enumerate}
\item
\exwhere{$ 2016 $年上海卷}
形象描述磁场分布的曲线叫做 \underlinegap ,通常 \underlinegap 的大小也叫做磁通量密度。

\tk{磁感线 \quad 磁感应强度} 

\item
\exwhere{$ 2016 $年上海卷}
如图,一束电子沿$ z $轴正向流动,则在图中$ y $轴上$ A $点的磁场方向是 \xzanswer{A} 
\begin{figure}[h!]
\centering
\includesvg[width=0.23\linewidth]{picture/svg/138}
\end{figure}


\fourchoices
{$ +x $方向}
{$ -x $方向}
{$ +y $方向}
{$ -y $方向}





\item
\exwhere{$ 2013 $年上海卷}
如图,足够长的直线$ ab $靠近通电螺线管,与螺线管平行。用磁传感器测量$ ab $上各点的磁感应强度$ B $,在计算机屏幕上显示的大致图像是 \xzanswer{C} 
\begin{figure}[h!]
\centering
\includesvg[width=0.23\linewidth]{picture/svg/139}
\end{figure}

\pfourchoices
{\includesvg[width=3cm]{picture/svg/GZ-3-tiyou-1204}}
{\includesvg[width=3cm]{picture/svg/GZ-3-tiyou-1206}}
{\includesvg[width=3cm]{picture/svg/GZ-3-tiyou-1207}}
{\includesvg[width=3cm]{picture/svg/GZ-3-tiyou-1208}}



\item
\exwhere{$ 2017 $年江苏卷}
如图所示,两个单匝线圈$ a $、$ b $的半径分别为$ r $和$ 2r $.圆形匀强磁场$ B $的边缘恰好与$ a $线圈重合,则穿过$ a $、$ b $两线圈的磁通量之比为 \xzanswer{A} 
\begin{figure}[h!]
\centering
\includesvg[width=0.2\linewidth]{picture/svg/141}
\end{figure}


\fourchoices
{$ 1:1 $}
{$ 1:2 $ }
{$ 1:4 $}
{$ 4:1 $}







\item
\exwhere{$ 2011 $年新课标版}
为了解释地球的磁性,$ 19 $世纪安培假设:地球的磁场是由绕过地心的轴的环形电流$ I $引起的。在下列四个图中,正确表示安培假设中环形电流方向的是 \xzanswer{B} 

\pfourchoices
{\includesvg[width=3cm]{picture/svg/GZ-3-tiyou-1209}}
{\includesvg[width=3cm]{picture/svg/GZ-3-tiyou-1210}}
{\includesvg[width=3cm]{picture/svg/GZ-3-tiyou-1211}}
{\includesvg[width=3cm]{picture/svg/GZ-3-tiyou-1212}}





\item
\exwhere{$ 2015 $年理综新课标\lmd{2}卷}
指南针是我国古代四大发明之一。关于指南针,下列说法正确的是 \xzanswer{BC} 

\fourchoices
{指南针可以仅具有一个磁极}
{指南针能够指向南北,说明地球具有磁场}
{指南针的指向会受到附近铁块的干扰}
{指南针正上方附近沿指针方向放置一直导线,导线通电时指南针不偏转}








\item
\exwhere{$ 2013 $年海南卷}
三条在同一平面(纸面)内的长直绝缘导线组成一等边三角形,在导线中通过的电流均为$ I $,方向如图所示。$ a $、$ b $和$ c $三点分别位于三角形的三个顶角的平分线上,且到相应顶点的距离相等。将$ a $、$ b $和$ c $处的磁感应强度大小分别记为$ B_{1} $、$ B_{2} $和$ B_{3} $,下列说法正确的是 \xzanswer{AC} 

\begin{figure}[h!]
\centering
\includesvg[width=0.23\linewidth]{picture/svg/143}
\end{figure}


\fourchoices
{$ B_1=B_2<B_3 $}
{$ B_1=B_2=B_3 $}
{$ a $和$ b $处磁场方向垂直于纸面向外,$ c $处磁场方向垂直于纸面向里}
{$ a $处磁场方向垂直于纸面向外,$ b $和$ c $处磁场方向垂直于纸面向里}





\item
\exwhere{$ 2012 $年理综全国卷}
如图,两根互相平行的长直导线过纸面上的$ M $、$ N $两点,且与纸面垂直,导线中通有大小相等、方向相反的电流。$ a $、$ O $、$ b $在$ M $、$ N $的连线上,$ O $为$ MN $的中点,$ c $、$ d $位于$ MN $的中垂线上,且$ a $、$ b $、$ c $、$ d $到$ O $点的距离均相等。关于以上几点处的磁场,下列说法正确的是 \xzanswer{C} 
\begin{figure}[h!]
\centering
\includesvg[width=0.23\linewidth]{picture/svg/144}
\end{figure}


\fourchoices
{$ O $点处的磁感应强度为零}
{$ a $、$ b $两点处的磁感应强度大小相等,方向相反}
{$ c $、$ d $两点处的磁感应强度大小相等,方向相同}
{$ a $、$ c $两点处磁感应强度的方向不同}




\item
\exwhere{$ 2011 $年理综全国卷}
如图,两根相互平行的长直导线分别通有方向相反的电流$ I_{1} $和$ I_{2} $,且$ I_1>I_2 $;$ a $、$ b $、$ c $、$ d $为导线某一横截面所在平面内的四点,且$ a $、$ b $、$ c $与两导线共面;$ b $点在两导线之间,$ b $、$ d $的连线与导线所在平面垂直。磁感应强度可能为零的点是 \xzanswer{C} 
\begin{figure}[h!]
\centering
\includesvg[width=0.23\linewidth]{picture/svg/145}
\end{figure}

\fourchoices
{$ a $点}
{$ b $点}
{$ c $点}
{$ d $点}





\item
\exwhere{$ 2014 $年物理海南卷}
如图,两根平行长直导线相距$ 2L $,通有大小相等、方向相同的恒定电流,$ a $、$ b $、$ c $是导线所在平面内的三点,左侧导线与它们的距离分别为$ \frac{l}{2} $、$ l $和$ 3l $ 。关于这三点处的磁感应强度,下列判断正确的是 \xzanswer{AD} 
\begin{figure}[h!]
\centering
\includesvg[width=0.16\linewidth]{picture/svg/146}
\end{figure}

\fourchoices
{$ a $处的磁感应强度大小比$ c $处的大}
{$ b $、$ c $两处的磁感应强度大小相等}
{$ a $、$ c $两处的磁感应强度方向相同}
{$ b $处的磁感应强度为零}







\item
\exwhere{$ 2016 $年北京卷}
中国宋代科学家沈括在《梦溪笔谈》中最早记载了地磁偏角:“以磁石磨针锋,则能指南,然常微偏东,不全南也。”进一步研究表明,地球周围地磁场的磁感线分布示意如图。结合上述材料,下列说法不正确的是 \xzanswer{C} 
\begin{figure}[h!]
\centering
\includesvg[width=0.17\linewidth]{picture/svg/147}
\end{figure}

\fourchoices
{地理南、北极与地磁场的南、北极不重合}
{地球内部也存在磁场,地磁南极在地理北极附近}
{地球表面任意位置的地磁场方向都与地面平行}
{地磁场对射向地球赤道的带电宇宙射线粒子有力的作用}



\item
\exwhere{$ 2017 $年新课标\lmd{3}卷}
如图,在磁感应强度大小为$ B_{0} $的匀强磁场中,两长直导线$ P $和$ Q $垂直于纸面固定放置,两者之间的距离为$ l $。在两导线中均通有方向垂直于纸面向里的电流$ I $时,纸面内与两导线距离为$ l $的$ a $点处的磁感应强度为零。如果让$ P $中的电流反向、其他条件不变,则$ a $点处磁感应强度的大小为 \xzanswer{C} 
\begin{figure}[h!]
\centering
\includesvg[width=0.23\linewidth]{picture/svg/148}
\end{figure}
\fourchoices
{$ 0 $}
{$\frac { \sqrt { 3 } } { 3 } B _ { 0 }$}
{$\frac { 2 \sqrt { 3 } } { 3 } B _ { 0 }$}
{$2 B _ { 0 }$}





\item
\exwhere{$ 2018 $年全国\lmd{2}卷}
如图,纸面内有两条互相垂直的长直绝缘导线$ L_{1} $、$ L_{2} $,$ L_{1} $中的电流方向向左,$ L_{2} $中的电流方向向上; $ L_{1} $的正上方有$ a $、$ b $两点,它们相对于$ L_{2} $对称。整个系统处于匀强外磁场中,外磁场的磁感应强度大小为$ B_{0} $,方向垂直于纸面向外。已知$ a $、$ b $两点的磁感应强度大小分别为$ \frac{ 1 }{ 3 } B_{0} $和$ \frac{ 1 }{ 2 } B_{0} $,方向也垂直于纸面向外。则 \xzanswer{AC} 
\begin{figure}[h!]
\centering
\includesvg[width=0.23\linewidth]{picture/svg/149}
\end{figure}


\fourchoices
{流经$ L_{1} $的电流在$ b $点产生的磁感应强度大小为$\frac { 7 } { 12 } B _ { 0 }$ }
{流经$ L_{1} $的电流在$ a $点产生的磁感应强度大小为$\frac { 1 } { 12 } B _ { 0 }$ }
{流经$ L_{2} $的电流在$ b $点产生的磁感应强度大小为$\frac { 1 } { 12 } B _ { 0 }$ }
{流经$ L_{2} $的电流在$ a $点产生的磁感应强度大小为$\frac { 7 } { 12 } B _ { 0 }$}






\item
\exwhere{$ 2018 $年浙江卷($ 4 $月选考)}
在城市建设施工中,经常需要确定地下金属管线的位置,如图所示。有一种探测方法是,首先给金属长直管线通上电流,再用可以测量磁场强弱、方向的仪器进行以下操作:①用测量仪在金属管线附近的水平地面上找到磁感应强度最强的某点,记为$ a $;②在$ a $点附近的地面上,找到与$ a $点磁感应强度相同的若干点,将这些点连成直线$ EF $;③在地面上过$ a $点垂直于$ EF $的直线上,找到磁场方向与地面夹角为$ 45 ^{\circ} $的$ b $、$ c $两点,测得$ b $、$ c $两点距离为$ L $。由此可确定金属管线 \xzanswer{A} 
\begin{figure}[h!]
\centering
\includesvg[width=0.18\linewidth]{picture/svg/150}
\end{figure}

\fourchoices
{平行于$ EF $,深度为$ \frac{L}{2} $}
{平行于$ EF $,深度为$ L $}
{垂直于$ EF $,深度为$ \frac{L}{2} $}
{平行于$ EF $,深度为$ L $}





\end{enumerate}








