\bta{波的图像}


\begin{enumerate}
	%\renewcommand{\labelenumi}{\arabic{enumi}.}
	% A(\Alph) a(\alph) I(\Roman) i(\roman) 1(\arabic)
	%设定全局标号series=example	%引用全局变量resume=example
	%[topsep=-0.3em,parsep=-0.3em,itemsep=-0.3em,partopsep=-0.3em]
	%可使用leftmargin调整列表环境左边的空白长度 [leftmargin=0em]
	\item
\exwhere{$ 2014 $ 年物理上海卷}
一列横波沿水平放置的弹性绳向右传播,绳上两质点 $ A $、$ B $ 的平衡位置相距 $ 3/4 $ 波长,$ B $ 位于 $ A $
右方。$ t $ 时刻 $ A $ 位于平衡位置上方且向上运动,再经过 $ 1/4 $ 周期,$ B $ 位于平衡位置 \xzanswer{D} 

\fourchoices
{上方且向上运动}
{上方且向下运动}
{下方且向上运动}
{下方且向下运动}



\item 
\exwhere{$ 2014 $ 年理综天津卷}
平衡位置处于坐标原点的波源 $ S $ 在 $ y $ 轴上振动,产生频率为 $ 50 \ Hz $ 的简谐横波向 $ x $ 正、负两个方
向传播,波速均为 $ 100 \ m /s $.平衡位置在 $ x $ 轴上的 $ P $、$ Q $ 两个质点随波源振动着,$ P $、$ Q $ 的 $ x $ 轴坐标
分别为 $ x^{P}=3.5 \ m $、$ x_Q=-3 \ m $ .当 $ S $ 位移为负且向 $ -y $ 方向运动时,$ P $、$ Q $ 两质点的 \xzanswer{D} 
\fourchoices
{位移方向相同、速度方向相反}
{位移方向相同、速度方向相同}
{位移方向相反、速度方向相反}
{位移方向相反、速度方向相同}



\item 
\exwhere{$ 2012 $ 年理综天津卷}
沿 $ x $ 轴正向传播的一列简谐横波在 $ t=0 $ 时刻的波形如图所示,$ M $ 为介质中的一个质点,该波的传
播速度为 $ 40 \ m /s $,则 $ t= $($ 1/40 $)$ s $ 时 \xzanswer{CD} 
\begin{figure}[h!]
	\centering
	\includesvg[width=0.23\linewidth]{picture/svg/GZ-3-tiyou-1348}
\end{figure}

\fourchoices
{质点 $ M $ 对平衡位置的位移一定为负值}
{质点 $ M $ 的速度方向与对平衡位置的位移方向相同}
{质点 $ M $ 的加速度方向与速度方向一定相同}
{质点 $ M $ 的加速度与对平衡位置的位移方向相反}

\item 
\exwhere{$ 2017 $ 年浙江选考卷}
一列向右传播的简谐横波,当波传到 $ x=2.0 \ m $ 处的 $ P $ 点时开始计时,该时刻
波形如图所示,$ t=0.9 \ s $ 时,观察到质点 $ P $ 第三次到达
波峰位置,下列说法正确的是 \xzanswer{BCD} 
\begin{figure}[h!]
	\centering
	\includesvg[width=0.23\linewidth]{picture/svg/GZ-3-tiyou-1349}
\end{figure}



\fourchoices
{波速为 $ 0.5 \ m /s $}
{经 $ 1.4 \ s $ 质点 $ P $ 运动的路程为 $ 70 \ cm $}
{$ t=1.6 \ s $ 时,$ x=4.5 \ m $ 处的质点 $ Q $ 第三次到达波谷}
{与该波发生干涉的另一列简谐横波的频率一定为 $ 2.5 \ HZ $}


\item 
\exwhere{$ 2012 $ 年理综安徽卷}
一列简谐波沿 $ x $ 轴正方向传播,在 $ t=0 $ 时波形如图 所示,已
知波速为 $ 10 \ m /s $。则 $ t=0.1 \ s $ 时正确的波形应是下图中的 \xzanswer{C} 
\begin{figure}[h!]
	\centering
	\includesvg[width=0.23\linewidth]{picture/svg/GZ-3-tiyou-1350}
\end{figure}


\pfourchoices
{\includesvg[width=4.3cm]{picture/svg/GZ-3-tiyou-1357}}
{\includesvg[width=4.3cm]{picture/svg/GZ-3-tiyou-1356}}
{\includesvg[width=4.3cm]{picture/svg/GZ-3-tiyou-1355}}
{\includesvg[width=4.3cm]{picture/svg/GZ-3-tiyou-1354}}


\item 
\exwhere{$ 2016 $ 年天津卷}
在均匀介质中坐标原点 $ O $ 处有一波源做简谐运动,其表达式为 $y=5 \sin \left(\frac{\pi}{2} t\right)$,
它在介质中形成的简谐横波沿 $ x $ 轴正方向传播,某时刻波刚好传播到 $ x=12 \ m $ 处,波形图像如图所示,
则 \xzanswer{AB} 
\begin{figure}[h!]
	\centering
	\includesvg[width=0.23\linewidth]{picture/svg/GZ-3-tiyou-1358}
\end{figure}

\fourchoices
{此后再经过 $ 6 \ s $ 该波传播到 $ x=24 \ m $ 处}
{$ M $ 点在此后第 $ 3 \ s $ 末的振动方向沿 $ y $ 轴正方向}
{波源开始振动时的运动方向沿 $ y $ 轴负方向}
{此后 $ M $ 点第一次到达 $ y=-3 \ m $ 处所需时间是 $ 2 \ s $}


\item 
\exwhere{$ 2015 $ 年上海卷}
一简谐横波沿水平绳向右传播,波速为 $ v $,周期为 $ T $,振幅为 $ A $。绳上两质点
$ M $、$ N $ 的平衡位置相距 $ 3/4 $ 波长,$ N $ 位于 $ M $ 右方。设向上为正,在 $ t=0 $ 时 $ M $ 位移为$ +A/2 $,且向上运
动;经时间 $ t $($ t<T $),$ M $ 位移仍为$ +A/2 $,但向下运动,则 \xzanswer{C} 


\fourchoices
{在 $ t $ 时刻,$ N $ 恰好在波谷位置}
{在 $ t $ 时刻,$ N $ 位移为负,速度向上}
{在 $ t $ 时刻,$ N $ 位移为负,速度向下}
{在 $ 2t $ 时刻,$ N $ 位移为$ -A/2 $,速度向下}


\item 
\exwhere{$ 2019 $ 年物理北京卷}
一列简谐横波某时刻的波形如图所示,比较介质中的三个质点 $ a $、$ b $、$ c $,
则 \xzanswer{C} 
\begin{figure}[h!]
	\centering
	\includesvg[width=0.23\linewidth]{picture/svg/GZ-3-tiyou-1359}
\end{figure}


\fourchoices
{此刻 $ a $ 的加速度最小}
{此刻 $ b $ 的速度最小}
{若波沿 $ x $ 轴正方向传播,此刻 $ b $ 向 $ y $ 轴正方向运动}
{若波沿 $ x $ 轴负方向传播,$ a $ 比 $ c $ 先回到平衡位置}






	
	
	
\end{enumerate}

