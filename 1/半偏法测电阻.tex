\bta{半偏法测电阻}



\begin{enumerate}[leftmargin=0em]
\renewcommand{\labelenumi}{\arabic{enumi}.}
% A(\Alph) a(\alph) I(\Roman) i(\roman) 1(\arabic)
%设定全局标号series=example	%引用全局变量resume=example
%[topsep=-0.3em,parsep=-0.3em,itemsep=-0.3em,partopsep=-0.3em]
%可使用leftmargin调整列表环境左边的空白长度 [leftmargin=0em]
\item
\exwhere{$ 2018 $年海南卷}
某同学利用图($ a $)中的电路测量电流表 \ammetermytikz 的内阻$ R_{A} $(约为)和直流电源的电动势$ E $(约为$ 10\ V $)。图中$ R_{1} $和$ R_{2} $为电阻箱,$ S_{1} $和$ S_{2} $为开关。已知电流表的量程为$ 100\ mA $,直流电源的内阻为$ r $。
\begin{figure}[h!]
\centering
\includesvg[width=0.83\linewidth]{picture/svg/640}
\end{figure}

\begin{enumerate}
\renewcommand{\labelenumi}{\arabic{enumi}.}
% A(\Alph) a(\alph) I(\Roman) i(\roman) 1(\arabic)
%设定全局标号series=example	%引用全局变量resume=example
%[topsep=-0.3em,parsep=-0.3em,itemsep=-0.3em,partopsep=-0.3em]
%可使用leftmargin调整列表环境左边的空白长度 [leftmargin=0em]
\item
断开$ S_{2} $,闭合$ S_{1} $,调节$ R_{1} $的阻值,使 \ammetermytikz 满偏;保持$ R_{1} $的阻值不变,闭合$ S_{2} $,调节$ S_{2} $,当$ R_{2} $的阻值为$ 4.8 \ \Omega $时的示数为$ 48.0\ mA $。忽略$ S_{2} $闭合后电路中总电阻的变化,经计算得$ R_{A}= $ \tk{5.2} $ \Omega $ ;(保留$ 2 $位有效数字)
\item 
保持$ S_{1} $闭合,断开$ S_{2} $,多次改变$ R_{1} $的阻值,并记录电流表的相应示数。若某次$ R_{1} $的示数如图($ b $)所示,则此次$ R_{1} $的阻值为 \tk{148.2} $ \Omega $ ;
\item 
利用记录的$ R_{1} $的阻值和相应的电流表示数 \lmd{1} ,作出$ I^{-1}- R_{1} $图线,如图($ c $)所示。用电池的电动势$ E $、内阻$ r $和电流表内阻$ R_{A} $表示$ I^{-1} $随$ R_{1} $变化的关系式为$ I^{-1}= $ \tk{$\frac { R _ { 1 } } { E } + \frac { r + R _ { \mathrm { A } } } { E }$} 。利用图($ c $)可求得 $ E= $ \tk{9.1} 
$ V $。(保留$ 2 $位有效数字)

\end{enumerate}

\item 
\exwhere{$ 2014 $年理综安徽卷}

某同学为了测量一个量程为$ 3V $的电压表的内阻,进行了如下实验:
\begin{figure}[h!]
\centering
\includesvg[width=0.23\linewidth]{picture/svg/641} \qquad \qquad 
\includesvg[width=0.23\linewidth]{picture/svg/642}
\end{figure}


\begin{enumerate}
\renewcommand{\labelenumi}{\arabic{enumi}.}
% A(\Alph) a(\alph) I(\Roman) i(\roman) 1(\arabic)
%设定全局标号series=example	%引用全局变量resume=example
%[topsep=-0.3em,parsep=-0.3em,itemsep=-0.3em,partopsep=-0.3em]
%可使用leftmargin调整列表环境左边的空白长度 [leftmargin=0em]
\item
他先用多用电表进行了正确的测量,测量时指针位置如图$ 1 $所示,得出电压表内阻为$ 3.00 \times 10^3 \ \Omega $,此时电压表的指针也偏转了。已知多用表欧姆档表盘中央刻度值为“$ 15 $”,表内电池电动势为$ 1.5V $,则电压表的示数应为 \tk{1.0} $ V $(结果保留两位有效数字)。

\item 
为了更准确地测量该电压表的内阻$ R_V $,该同学设计了图$ 2 $所示的电路图,实验步骤如下:

A.断开开关$ S $,按图$ 2 $连接好电路;

B.把滑动变阻器$ R $的滑片$ P $滑到$ b $端;

C.将电阻箱$ R_{0} $的阻值调到零;

D.闭合开关$ S $;

$ E $.移动滑动变阻器$ R $的滑片$ P $的位置,使
电压表的指针指到$ 3V $位置;

$ F $.保持滑动变阻器$ R $的滑片$ P $位置不变,调节电阻箱$ R_{0} $的阻值使电压表的指针指到$ 1.5V $位置,读出此时电阻箱$ R_{0} $的阻值,此值即为电压表内阻$ R_V $的测量值;

$ G $.断开开关$ S $。

实验中可供选择的实验器材有:

$ a $.待测电压表

$ b $.滑动变阻器:最大阻值$ 2000 \ \Omega $

$ c $.滑动变阻器:最大阻值$ 10 \ \Omega $

$ d $.电阻箱:最大阻值$ 9999.9 \ \Omega $,阻值最小该变量为$ 0.1 \ \Omega $

$ e $.电阻箱:最大阻值$ 999.9 \ \Omega $,阻值最小该变量为$ 0.1 \ \Omega $

$ f $.电池组:电动势约$ 6V $,内阻可忽略

$ g $.开关,导线若干

按照这位同学设计的实验方案,回答下列问题:

①要使测量更精确,除了选用电池组、导线、开关和待测电压表外,还应从提供的滑动变阻器中选用 \tk{c} (填“$ b $”或“$ c $”),电阻箱中选用 \tk{d} (填“$ d $”或“$ e $”)。

②电压表的内阻$ R_V $的测量值$ R_{ \text{测} } $和真实值$ R_{ \text{真} } $相比,$ R_{ \text{测} } $ \tk{$ > $} $ R_{ \text{真} } $(填“$ > $”或“$ < $”);若$ R_V $越大,则越 \tk{小} (填“大”或“小”)。




\end{enumerate}











\end{enumerate}

