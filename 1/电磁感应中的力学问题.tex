\bta{电磁感应中的力学问题}

\begin{enumerate}
\renewcommand{\labelenumi}{\arabic{enumi}.}
% A(\Alph) a(\alph) I(\Roman) i(\roman) 1(\arabic)
%设定全局标号series=example	%引用全局变量resume=example
%[topsep=-0.3em,parsep=-0.3em,itemsep=-0.3em,partopsep=-0.3em]
%可使用leftmargin调整列表环境左边的空白长度 [leftmargin=0em]
\begin{minipage}[h!]{0.7\linewidth}
\vspace{0.3em}
\item
\exwhere{$ 2015 $年上海卷}
如图,一无限长通电直导线固定在光滑水平面上,金属环质量为$ 0.02 \ kg $,在该平面上以$ v_0=2 \ m/s $、与导线成$ 60 ^{ \circ } $角的初速度运动,其最终的运动状态是\tk{匀速直线运动},环中最多能产生\tk{$ 0.03 $}$ J $的电能。
\vspace{0.3em}
\end{minipage}
\hfill
\begin{minipage}[h!]{0.3\linewidth}
\flushright
\vspace{0.3em}
\includesvg[width=0.6\linewidth]{picture/svg/350}
\vspace{0.3em}
\end{minipage}



\item 
\exwhere{$ 2012 $年物理上海卷}
正方形导体框处于匀强磁场中,磁场方向垂直框平面,磁感应强度随时间均匀增加,变化率为$ k $。导体框质量为$ m $、边长为$ L $,总电阻为$ R $,在恒定外力$ F $作用下由静止开始运动。导体框在磁场中的加速度大小为\tk{$ \frac{F}{m} $};导体框中感应电流做功的功率为\tk{$ \frac{k^{2}L^{4}}{R} $}。
\begin{figure}[h!]
\centering
\includesvg[width=0.23\linewidth]{picture/svg/351}
\end{figure}



\item 
\exwhere{$ 2014 $年理综大纲卷}
很多相同的绝缘铜圆环沿竖直方向叠放,形成一很长的竖直圆筒。一条形磁铁沿圆筒的中心轴竖直放置,其下端与圆筒上端开口平齐。让条形磁铁从静止开始下落。条形磁铁在圆筒中的运动速率 \xzanswer{C} 


\fourchoices
{均匀增大 }
{先增大,后减小}
{逐渐增大,趋于不变 }
{先增大,再减小,最后不变}



\item 
\exwhere{$ 2014 $年理综广东卷}
如图所示,上下开口、内壁光滑的铜管$ P $和塑料管$ Q $竖直放置,小磁块先后在两管中从相同高度处由静止释放,并落至底部,则小磁块 \xzanswer{C} 


\begin{figure}[h!]
\centering
\includesvg[width=0.23\linewidth]{picture/svg/352}
\end{figure}

\fourchoices
{在$ P $和$ Q $中都做自由落体运动}
{在两个下落过程中的机械能都守恒}
{在$ P $中的下落时间比在$ Q $中的长}
{落至底部时在$ P $中的速度比在$ Q $中的大}


\item 
\exwhere{$ 2013 $年天津卷}
如图所示,纸面内有一矩形导体闭合线框动$ abcd $。$ ab $边长大于$ bc $边长,置于垂直纸面向里、边界为$ MN $的匀强磁场外,线框两次匀速地完全进入磁场,两次速度大小相同,方向均垂直于$ MN $。第一次$ ab $边平行$ MN $进入磁场,线框上产生的热量为$ Q_{1} $,通过线框导体横截面的电荷量为$ q_{1} $;第二次$ bc $边平行$ MN $进入磁场.线框上产生的热量为$ Q_{2} $,通过线框导体横截面的电荷量为$ q_{2} $,则 \xzanswer{A} 
\begin{figure}[h!]
\centering
\includesvg[width=0.23\linewidth]{picture/svg/353}
\end{figure}


\fourchoices
{$ Q_{1} > Q_{2} \quad q_{1} = q_{2} $ }
{$ Q_{1} > Q_{2} \quad q_{1} > q_{2} $}
{$ Q_{1} = Q_{2} \quad q_{1} = q_{2} $ }
{$ Q_{1} = Q_{2} \quad q_{1} > q_{2} $}




\item 
\exwhere{$ 2013 $年安徽卷}
如图所示,足够长平行金属导轨倾斜放置,倾角为$ 37 ^{ \circ } $,宽度为$ 0.5m $,电阻忽略不计,其上端接一小灯泡,电阻为$ 1 \Omega $。一导体棒$ MN $垂直于导轨放置,质量为$ 0.2 \ kg $,接入电路的电阻为$ 1 \Omega $,两端与导轨接触良好,与导轨间的动摩擦因数为$ 0.5 $.在导轨间存在着垂直于导轨平面的匀强磁场,磁感应强度为$ 0.8 \ T $。将导体棒$ MN $由静止释放,运动一段时间后,小灯泡稳定发光,此后导体棒$ MN $的运动速度以及小灯泡消耗的电功率分别为(重力加速度$ g $取$ 10 \ m/s ^{2} $,$ \sin 37 ^{ \circ } =0.6 $) \xzanswer{B} 
\begin{figure}[h!]
\centering
\includesvg[width=0.23\linewidth]{picture/svg/354}
\end{figure}


\fourchoices
{$ 2.5 \ m/s \quad 1 \ W $ }
{$ 5 \ m/s \quad 1 \ W $}
{$ 7.5 \ m/s \quad 9 \ W $ }
{$ 15 \ m/s \quad 9 \ W $}

\item 
\exwhere{$ 2011 $年理综福建卷}
如图,足够长的$ U $型光滑金属导轨平面与水平面成$ \theta $角($ 0 < \theta <90 ^{ \circ } ) $,其中$ MN $与$ PQ $平行且间距为$ L $,导轨平面与磁感应强度为$ B $的匀强磁场垂直,导轨电阻不计。金属棒$ ab $由静止开始沿导轨下滑,并与两导轨始终保持垂直且良好接触,$ ab $棒接入电路的电阻为$ R $,当流过$ ab $棒某一横截面的电量为$ q $时,棒的速度大小为$ v $,则金属棒$ ab $在这一过程中 \xzanswer{B} 
\begin{figure}[h!]
\centering
\includesvg[width=0.23\linewidth]{picture/svg/355}
\end{figure}


\fourchoices
{运动的平均速度大小为$ \frac{ 1 }{ 2 } v $}
{下滑位移大小为$\frac { q R } { B L }$}
{产生的焦耳热为$ qBLv $}
{受到的最大安培力大小为$\frac { B ^ { 2 } L ^ { 2 } v } { R } \sin \theta$}


\item 
\exwhere{$ 2012 $年理综山东卷}
如图所示,相距为$ L $的两条足够长的光滑平行金属导轨与水平面的夹角为$ \theta $,上端接有定值电阻$ R $,匀强磁场垂直于导轨平面,磁感应强度为$ B $。将质量为$ m $的导体棒由静止释放,当速度达到$ v $时开始匀速运动,此时对导体棒施加一平行于导轨向下的拉力,并保持拉力的功率为$ P $,导体棒最终以$ 2v $的速度匀速运动。导体棒始终与导轨垂直且接触良好,不计导轨和导体棒的电阻,重力加速度为$ g $,下列选项正确的是 \xzanswer{AC} 
\begin{figure}[h!]
\centering
\includesvg[width=0.23\linewidth]{picture/svg/356}
\end{figure}


\fourchoices
{$ P=2mgv \sin \theta $}
{$ P=3mgv \sin \theta $}
{当导体棒速度达到$ \frac{v}{2} $时加速度大小为$\frac { g } { 2 } \sin \theta$}
{在速度达到$ 2v $ 以后匀速运动的过程中,$ R $上产生的焦耳热等于拉力所做的功}




\item 
\exwhere{$ 2017 $年海南卷}
如图,空间中存在一匀强磁场区域,磁场方向与竖直面(纸面)垂直,磁场的上、下边界(虚线)均为水平面;纸面内磁场上方有一个正方形导线框$ abcd $,其上、下两边均为磁场边界平行,边长小于磁场上、下边界的间距。若线框自由下落,从$ ab $边进入磁场时开始,直至$ ab $边到达磁场下边界为止,线框下落的速度大小可能 \xzanswer{CD} 


\begin{minipage}[h!]{0.7\linewidth}
\vspace{0.3em}
\fourchoices
{始终减小}
{始终不变}
{始终增加}
{先减小后增加}

\vspace{0.3em}
\end{minipage}
\hfill
\begin{minipage}[h!]{0.3\linewidth}
\flushright
\vspace{0.3em}
\includesvg[width=0.5\linewidth]{picture/svg/359}
\vspace{0.3em}
\end{minipage}
\item 
\exwhere{$ 2013 $年江苏卷}
如图所示,匀强磁场中有一矩形闭合线圈$ abcd $,线圈平面与磁场垂直. 已知线圈的匝数$ N $ $ =100 $,边长$ ab $ $ =1.0 $ $ m $、$ bc=0.5 $ $ m $,电阻$ r $ $ =2 \ \Omega .$ 磁感应强度$ B $ 在$ 0 \sim 1 s $ 内从零均匀变化到$ 0.2 $ $ T $. 在$ 1 \sim 5 $ $ s $ 内从$ 0. 2\ T $ 均匀变化到$ -0.2 $ $ T $,取垂直纸面向里为磁场的正方向. 求:
\begin{enumerate}
\renewcommand{\labelenumii}{(\arabic{enumii})}

\item 
$ 0. 5 s $时线圈内感应电动势的大小$ E $和感应电流的方向;$ $

\item 
在$ 1\sim 5 \ s $内通过线圈的电荷量$ q $;

\item 
在$ 0 \sim 5 \ s $内线圈产生的焦耳热$ Q. $

\end{enumerate}
\begin{figure}[h!]
\flushright 
\includesvg[width=0.17\linewidth]{picture/svg/357}
\end{figure}

\banswer{
\begin{enumerate}
\renewcommand{\labelenumi}{\arabic{enumi}.}
% A(\Alph) a(\alph) I(\Roman) i(\roman) 1(\arabic)
%设定全局标号series=example	%引用全局变量resume=example
%[topsep=-0.3em,parsep=-0.3em,itemsep=-0.3em,partopsep=-0.3em]
%可使用leftmargin调整列表环境左边的空白长度 [leftmargin=0em]
\item
$10 \mathrm { V } \quad a \rightarrow d \rightarrow c \rightarrow b \rightarrow a$
\item 
$ 10\ C $
\item 
$ 100\ J $


\end{enumerate}


}

\item 
\exwhere{$ 2017 $年江苏卷}
如图所示,两条相距$ d $的平行金属导轨位于同一水平面内,其右端接一阻值为$ R $的电阻.质量为$ m $的金属杆静置在导轨上,其左侧的矩形匀强磁场区域$ MNPQ $的磁感应强度大小为$ B $、方向竖直向下.当该磁场区域以速度$ v_{0} $匀速地向右扫过金属杆后,金属杆的速度变为$ v. $导轨和金属杆的电阻不计,导轨光滑且足够长,杆在运动过程中始终与导轨垂直且两端与导轨保持良好接触.求:
\begin{enumerate}
\renewcommand{\labelenumi}{\arabic{enumi}.}
% A(\Alph) a(\alph) I(\Roman) i(\roman) 1(\arabic)
%设定全局标号series=example	%引用全局变量resume=example
%[topsep=-0.3em,parsep=-0.3em,itemsep=-0.3em,partopsep=-0.3em]
%可使用leftmargin调整列表环境左边的空白长度 [leftmargin=0em]
\item
$ MN $刚扫过金属杆时,杆中感应电流的大小$ I $;
\item 
$ MN $刚扫过金属杆时,杆的加速度大小$ a $;
\item 
$ PQ $刚要离开金属杆时,感应电流的功率$ P $.

\end{enumerate}
\begin{figure}[h!]
\flushright
\includesvg[width=0.35\linewidth]{picture/svg/358}
\end{figure}

\banswer{
\begin{enumerate}
\renewcommand{\labelenumi}{\arabic{enumi}.}
% A(\Alph) a(\alph) I(\Roman) i(\roman) 1(\arabic)
%设定全局标号series=example	%引用全局变量resume=example
%[topsep=-0.3em,parsep=-0.3em,itemsep=-0.3em,partopsep=-0.3em]
%可使用leftmargin调整列表环境左边的空白长度 [leftmargin=0em]
\item
$I = \frac { E } { R } = \frac { B d v _ { 0 } } { R }$
\item 
$a = \frac { B ^ { 2 } d ^ { 2 } v _ { 0 } } { m R }$
\item 
$P = \frac { B ^ { 2 } d ^ { 2 } \left( v _ { 0 } - v \right) ^ { 2 } } { R }$



\end{enumerate}	
}





\item 
\exwhere{$ 2017 $年海南卷}
如图,两光滑平行金属导轨置于水平面(纸面)内,轨间距为$ l $,左端连有阻值为$ R $的电阻。一金属杆置于导轨上,金属杆右侧存在一磁感应强度大小为$ B $、方向竖直向下的匀强磁场区域。已知金属杆以速度$ v_{0} $向右进入磁场区域,做匀变速直线运动,到达磁场区域右边界(图中虚线位置)时速度恰好为零。金属杆与导轨始终保持垂直且接触良好。除左端所连电阻外,其他电阻忽略不计。求金属杆运动到磁场区域正中间时所受安培力的大小及此时电流的功率。
\begin{figure}[h!]
\flushright 
\includesvg[width=0.23\linewidth]{picture/svg/360}
\end{figure}

\banswer{
$F = B I l = \frac { \sqrt { 2 } B ^ { 2 } l ^ { 2 } v _ { 0 } } { 2 R }$ \qquad $P = I ^ { 2 } R = \frac { B ^ { 2 } l ^ { 2 } v _ { 0 } ^ { 2 } } { 2 R }$
}


\newpage
\item 
\exwhere{$ 2014 $年物理江苏卷}
如图所示,在匀强磁场中有一倾斜的平行金属导轨,导轨间距为 $ L $,长为 $ 3 d $,导轨平面与水平面的夹角为 $ \theta $,在导轨的中部刷有一段长为 $ d $ 的薄绝缘涂层. 匀强磁场的磁感应强度大小为 $ B $,方向与导轨平面垂直. 质量为$ m $ 的导体棒从导轨的顶端由静止释放, 在滑上涂层之前已经做匀速运动, 并一直匀速滑到导轨底端. 导体棒始终与导轨垂直,且仅与涂层间有摩擦,接在两导轨间的电阻为 $ R $,其他部分的电阻均不计,重力加速度为 $ g $. 求: 

\begin{enumerate}
\renewcommand{\labelenumi}{\arabic{enumi}.}
% A(\Alph) a(\alph) I(\Roman) i(\roman) 1(\arabic)
%设定全局标号series=example	%引用全局变量resume=example
%[topsep=-0.3em,parsep=-0.3em,itemsep=-0.3em,partopsep=-0.3em]
%可使用leftmargin调整列表环境左边的空白长度 [leftmargin=0em]
\item
导体棒与涂层间的动摩擦因数$ \mu $; 
\item 
导体棒匀速运动的速度大小 $ v $;
\item 
整个运动过程中,电阻产生的焦耳热 $ Q $.



\end{enumerate}
\begin{figure}[h!]
\flushright
\includesvg[width=0.4\linewidth]{picture/svg/361}
\end{figure}

\banswer{
\begin{enumerate}
\renewcommand{\labelenumi}{\arabic{enumi}.}
% A(\Alph) a(\alph) I(\Roman) i(\roman) 1(\arabic)
%设定全局标号series=example	%引用全局变量resume=example
%[topsep=-0.3em,parsep=-0.3em,itemsep=-0.3em,partopsep=-0.3em]
%可使用leftmargin调整列表环境左边的空白长度 [leftmargin=0em]
\item
$\tan \theta$
\item 
$\frac { m g R \sin \theta } { B ^ { 2 } L ^ { 2 } }$
\item 
$2 m g d \sin \theta - \frac { m ^ { 3 } g ^ { 2 } R ^ { 2 } \sin ^ { 2 } \theta } { 2 B ^ { 4 } L ^ { 4 } }$



\end{enumerate}


}



\item
\exwhere{$ 2013 $年上海卷}
如图,两根相距$ l=0.4m $、电阻不计的平行光滑金属导轨水平放置,一端与阻值$ R=0.15 \Omega $的电阻相连。导轨间$ x > 0 $一侧存在沿$ x $方向均匀增大的稳恒磁场,其方向与导轨平面垂直,变化率$ k=0.5 \ T/m $,$ x=0 $处磁场的磁感应强度$ B_0=0.5 \ T $。一根质量$ m=0.1 \ kg $、电阻$ r=0.05\ \Omega $的金属棒置于导轨上,并与导轨垂直。棒在外力作用下从$ x=0 $处以初速度$ v_0=2 \ m/s $沿导轨向右运动,运动过程中电阻上消耗的功率不变。求:
\begin{enumerate}
\renewcommand{\labelenumi}{\arabic{enumi}.}
% A(\Alph) a(\alph) I(\Roman) i(\roman) 1(\arabic)
%设定全局标号series=example	%引用全局变量resume=example
%[topsep=-0.3em,parsep=-0.3em,itemsep=-0.3em,partopsep=-0.3em]
%可使用leftmargin调整列表环境左边的空白长度 [leftmargin=0em]
\item
电路中的电流;
\item 
金属棒在$ x=2\ m $处的速度;
\item 
金属棒从$ x=0 $运动到$ x=2\ m $过程中安培力做功的大小;
\item 
金属棒从$ x=0 $运动到$ x=2\ m $过程中外力的平均功率。

\end{enumerate}
\begin{figure}[h!]
\flushright
\includesvg[width=0.25\linewidth]{picture/svg/362}
\end{figure}


\banswer{
\begin{enumerate}
\renewcommand{\labelenumi}{\arabic{enumi}.}
% A(\Alph) a(\alph) I(\Roman) i(\roman) 1(\arabic)
%设定全局标号series=example	%引用全局变量resume=example
%[topsep=-0.3em,parsep=-0.3em,itemsep=-0.3em,partopsep=-0.3em]
%可使用leftmargin调整列表环境左边的空白长度 [leftmargin=0em]
\item
$ 2\ A $
\item 
$v = \frac { B _ { 0 } v _ { 0 } } { B _ { 0 } + k x } = \frac { 0.5 \times 2 } { 0.5 ( 1 + 2 ) } = \frac { 2 } { 3 } \mathrm { m } / \mathrm { s }$
\item 
$W _ { F m } = \frac { 1 } { 2 } \left[ B _ { 0 } + \left( B _ { 0 } + k x \right) \right] I l x=1.6\ J$
\item 
$\bar { P } = \frac { W _ { F } } { t } = 0.7 \mathrm { W }$


\end{enumerate}


}



\newpage
\item 
\exwhere{$ 2012 $年物理上海卷}
如图,质量为$ M $的足够长金属导轨$ abcd $放在光滑的绝缘水平面上。一电阻不计,质量为$ m $的导体棒$ PQ $放置在导轨上,始终与导轨接触良好,$ PQbc $构成矩形。棒与导轨间动摩擦因数为$ \mu $,棒左侧有两个固定于水平面的立柱。导轨$ bc $段长为$ L $,开始时$ PQ $左侧导轨的总电阻为$ R $,右侧导轨单位长度的电阻为$ R_{0} $。以$ ef $为界,其左侧匀强磁场方向竖直向上,右侧匀强磁场水平向左,磁感应强度大小均为$ B $。在$ t=0 $时,一水平向左的拉力$ F $垂直作用在导轨的$ bc $边上,使导轨由静止开始做匀加速直线运动,加速度为$ a $。
\begin{enumerate}
\renewcommand{\labelenumi}{\arabic{enumi}.}
% A(\Alph) a(\alph) I(\Roman) i(\roman) 1(\arabic)
%设定全局标号series=example	%引用全局变量resume=example
%[topsep=-0.3em,parsep=-0.3em,itemsep=-0.3em,partopsep=-0.3em]
%可使用leftmargin调整列表环境左边的空白长度 [leftmargin=0em]
\item
求回路中感应电动势及感应电流随时间变化的表达式;
\item 
经过多长时间拉力$ F $达到最大值,拉力$ F $的最大值为多少?
\item 
某一过程中回路产生的焦耳热为$ Q $,导轨克服摩擦力做功为$ W $,求导轨动能的增加量。



\end{enumerate}
\begin{figure}[h!]
\flushright
\includesvg[width=0.35\linewidth]{picture/svg/363}
\end{figure}

\banswer{
\begin{enumerate}
\renewcommand{\labelenumi}{\arabic{enumi}.}
% A(\Alph) a(\alph) I(\Roman) i(\roman) 1(\arabic)
%设定全局标号series=example	%引用全局变量resume=example
%[topsep=-0.3em,parsep=-0.3em,itemsep=-0.3em,partopsep=-0.3em]
%可使用leftmargin调整列表环境左边的空白长度 [leftmargin=0em]
\item
$E = B L a t$ \qquad $I = \frac { B L v } { R _ { \text{总} }} = \frac { B L a t } { R + 2 R _ { 0 } \left( \frac { 1 } { 2 } a t ^ { 2 } \right) } = \frac { B L a t } { R + R _ { 0 } a t ^ { 2 } }$
\item 
即$t = \sqrt { \frac { a } { R R _ { 0 } } }$时外力$ F $取最大值,$F _ { \max } = M a + \mu m g + \frac { 1 } { 2 } ( 1 + \mu ) B ^ { 2 } L ^ { 2 } \sqrt { \frac { a } { R R _ { 0 } } }$
\item 
$\Delta E _ { \mathrm { k } } = M a s = \frac { M a } { \mu m g } ( W - \mu Q )$



\end{enumerate}


}


\item 
\exwhere{$ 2011 $年海南卷}
如图,$ ab $和$ cd $是两条竖直放置的长直光滑金属导轨,$ MN $和$M ^ { \prime } N ^ { \prime }$是两根用细线连接的金属杆,其质量分别为$ m $和$ 2 \ m $。竖直向上的外力$ F $作用在杆$ MN $上,使两杆水平静止,并刚好与导轨接触;两杆的总电阻为$ R $,导轨间距为$ l $。整个装置处在磁感应强度为$ B $的匀强磁场中,磁场方向与导轨所在平面垂直。导轨电阻可忽略,重力加速度为$ g $。在$ t=0 $时刻将细线烧断,保持$ F $不变,金属杆和导轨始终接触良好。求:
\begin{enumerate}
\renewcommand{\labelenumi}{\arabic{enumi}.}
% A(\Alph) a(\alph) I(\Roman) i(\roman) 1(\arabic)
%设定全局标号series=example	%引用全局变量resume=example
%[topsep=-0.3em,parsep=-0.3em,itemsep=-0.3em,partopsep=-0.3em]
%可使用leftmargin调整列表环境左边的空白长度 [leftmargin=0em]
\item
细线烧断后,任意时刻两杆运动的速度之比;
\item 
两杆分别达到的最大速度。



\end{enumerate}
\begin{figure}[h!]
\flushright
\includesvg[width=0.15\linewidth]{picture/svg/364}
\end{figure}

\banswer{
\begin{enumerate}
\renewcommand{\labelenumi}{\arabic{enumi}.}
% A(\Alph) a(\alph) I(\Roman) i(\roman) 1(\arabic)
%设定全局标号series=example	%引用全局变量resume=example
%[topsep=-0.3em,parsep=-0.3em,itemsep=-0.3em,partopsep=-0.3em]
%可使用leftmargin调整列表环境左边的空白长度 [leftmargin=0em]
\item
$\frac { v _ { \text{上} } } { v _ { \text{下} } } = \frac{ 2 }{ 1 } $
\item 
$v _ { \text{上} max } = \frac { 4 m g R } { 3 B ^ { 2 } l ^ { 2 } } , \quad v _ { \text{下} max } = \frac { 2m g R } { 3 B ^ { 2 } l ^ { 2 } }$



\end{enumerate}

}



\newpage
\item 
\exwhere{$ 2011 $年理综四川卷}
24.($ 19 $分)如图所示,间距$ l=0.3m $的平行金属导轨$a _ { 1 } b _ { 1 } c _ { 1 }$和$ a_2b_2c_2 $分别固定在两个竖直面内,在水平面$ a_1b_1b_2a_2 $区域内和倾角$ \theta =37 ^{\circ} $的斜面$ c_1b_1b_2c_2 $区域内分别有磁感应强度$ B_1=0.4 \ T $ (方向竖直向上)和$ B_2=1 \ T $ (方向垂直于斜面向上的匀强磁场)。电阻$ R=0.3 \Omega $、质量$ m_{1}=0.1 \ kg $、长为$ l $ 的相同导体杆$ K $、$ S $、$ Q $分别放置在导轨上,$ S $杆的两端固定在$ b_{1} $、$ b_{2} $点,$ K $、$ Q $杆可沿导轨无摩擦滑动且始终接触良好。一端系于$ K $杆中点的轻绳平行于导轨绕过轻质滑轮自然下垂,绳上穿有质量$ m_{2}=0.05 \ kg $的小环。已知小环以$ a=6 \ m/s ^{2} $的加速度沿绳下滑,$ K $杆保持静止,$ Q $杆在垂直于杆且沿斜面向下的拉力$ F $作用下匀速运动。不计导轨电阻和滑轮摩擦,绳不可伸长。取$ g=10 \ m/s ^{2} $,$ \sin 37 ^{\circ} =0.6 $,$ \cos 37 ^{\circ} =0.8 $。求:
\begin{enumerate}
\renewcommand{\labelenumi}{\arabic{enumi}.}
% A(\Alph) a(\alph) I(\Roman) i(\roman) 1(\arabic)
%设定全局标号series=example	%引用全局变量resume=example
%[topsep=-0.3em,parsep=-0.3em,itemsep=-0.3em,partopsep=-0.3em]
%可使用leftmargin调整列表环境左边的空白长度 [leftmargin=0em]
\item
小环所受摩擦力的大小;
\item 
$ Q $杆所受拉力的瞬时功率。



\end{enumerate}
\begin{figure}[h!]
\flushright
\includesvg[width=0.36\linewidth]{picture/svg/365}
\end{figure}

\banswer{
\begin{enumerate}
\renewcommand{\labelenumi}{\arabic{enumi}.}
% A(\Alph) a(\alph) I(\Roman) i(\roman) 1(\arabic)
%设定全局标号series=example	%引用全局变量resume=example
%[topsep=-0.3em,parsep=-0.3em,itemsep=-0.3em,partopsep=-0.3em]
%可使用leftmargin调整列表环境左边的空白长度 [leftmargin=0em]
\item
$ f=0.2\ N $
\item 
$ P=2\ W $


\end{enumerate}


}



\item 
\exwhere{$ 2011 $年理综天津卷}
如图所示,两根足够长的光滑平行金属导轨$ MN $、$ PQ $间距为$ l=0.5m $,其电阻不计,两导轨及其构成的平面均与水平面成$ 30 $º角。完全相同的两金属棒$ ab $、$ cd $分别垂直导轨放置,每棒两端都与导轨始终有良好接触,已知两棒质量均为$ m=0.02 \ kg $,电阻均为$ R=0.1\ \Omega $,整个装置处在垂直于导轨平面向上的匀强磁场中,磁感应强度$ B=0.2 \ T $,棒$ ab $在平行于导轨向上的力$ F $作用下,沿导轨向上匀速运动,而棒$ cd $恰好能够保持静止。取$ g=10 \ m/s ^{2} $,问:
\begin{enumerate}
\renewcommand{\labelenumi}{\arabic{enumi}.}
% A(\Alph) a(\alph) I(\Roman) i(\roman) 1(\arabic)
%设定全局标号series=example	%引用全局变量resume=example
%[topsep=-0.3em,parsep=-0.3em,itemsep=-0.3em,partopsep=-0.3em]
%可使用leftmargin调整列表环境左边的空白长度 [leftmargin=0em]
\item
通过棒$ cd $的电流$ I $是多少,方向如何?
\item 
棒$ ab $受到的力$ F $多大?
\item 
棒$ cd $每产生$ Q=0.1\ J $的热量,力$ F $做的功$ W $是多少?



\end{enumerate}

\begin{figure}[h!]
\flushright
\includesvg[width=0.29\linewidth]{picture/svg/366}
\end{figure}

\banswer{
\begin{enumerate}
\renewcommand{\labelenumi}{\arabic{enumi}.}
% A(\Alph) a(\alph) I(\Roman) i(\roman) 1(\arabic)
%设定全局标号series=example	%引用全局变量resume=example
%[topsep=-0.3em,parsep=-0.3em,itemsep=-0.3em,partopsep=-0.3em]
%可使用leftmargin调整列表环境左边的空白长度 [leftmargin=0em]
\item
$ I=1\ A $
\item 
$ F=0.2\ N $
\item 
$ W=0.4\ J $

\end{enumerate}


}




\newpage
\item 
\exwhere{$ 2011 $年理综浙江卷}
如图甲所示,在水平面上固定有长为$ L=2m $、宽为$ d=1m $的金属“$ U $”型导轨,在“$ U $”型导轨右侧$ l=0.5m $范围内存在垂直纸面向里的匀强磁场,且磁感应强度随时间变化规律如图乙所示。在$ t=0 $时刻,质量为$ m=0.1 \ kg $的导体棒以$ v_0=1 \ m/s $的初速度从导轨的左端开始向右运动,导体棒与导轨之间的动摩擦因数为$ \mu =0.1 $,导轨与导体棒单位长度的电阻均为$ \lambda=0.1\ \Omega /m $,不计导体棒与导轨之间的接触电阻及地球磁场的影响(取$ g=10 \ m/s ^{2} $)。
\begin{enumerate}
\renewcommand{\labelenumi}{\arabic{enumi}.}
% A(\Alph) a(\alph) I(\Roman) i(\roman) 1(\arabic)
%设定全局标号series=example	%引用全局变量resume=example
%[topsep=-0.3em,parsep=-0.3em,itemsep=-0.3em,partopsep=-0.3em]
%可使用leftmargin调整列表环境左边的空白长度 [leftmargin=0em]
\item
通过计算分析$ 4 \ s $内导体棒的运动情况;
\item 
计算$ 4 \ s $内回路中电流的大小,并判断电流方向;
\item 
计算$ 4 \ s $内回路产生的焦耳热。



\end{enumerate}
\begin{figure}[h!]
\flushright 
\includesvg[width=0.43\linewidth]{picture/svg/367}
\end{figure}

\banswer{
\begin{enumerate}
\renewcommand{\labelenumi}{\arabic{enumi}.}
% A(\Alph) a(\alph) I(\Roman) i(\roman) 1(\arabic)
%设定全局标号series=example	%引用全局变量resume=example
%[topsep=-0.3em,parsep=-0.3em,itemsep=-0.3em,partopsep=-0.3em]
%可使用leftmargin调整列表环境左边的空白长度 [leftmargin=0em]
\item
导体棒在$ 1 \ s $前做匀减速运动,在$ 1 \ s $后以后一直保持静止。
\item 
$ 0.2 \ A $,电流方向是顺时针方向。
\item 
$ 0.04\ J $

\end{enumerate}


}


\item 
\exwhere{$ 2015 $年海南卷}
如图,两平行金属导轨位于同一水平面上,相距,左端与一电阻$ R $相连;整个系统置于匀强磁场中,磁感应强度大小为$ B $,方向竖直向下。一质量为$ m $的导体棒置于导轨上,在水平外力作用下沿导轨以速度匀速向右滑动,滑动过程中始终保持与导轨垂直并接触良好。已知导体棒与导轨间的动摩擦因数为$ \mu $,重力加速度大小为$ g $,导轨和导体棒的电阻均可忽略。求:
\begin{enumerate}
\renewcommand{\labelenumi}{\arabic{enumi}.}
% A(\Alph) a(\alph) I(\Roman) i(\roman) 1(\arabic)
%设定全局标号series=example	%引用全局变量resume=example
%[topsep=-0.3em,parsep=-0.3em,itemsep=-0.3em,partopsep=-0.3em]
%可使用leftmargin调整列表环境左边的空白长度 [leftmargin=0em]
\item
电阻$ R $消耗的功率;
\item 
水平外力的大小。



\end{enumerate}
\begin{figure}[h!]
\flushright
\includesvg[width=0.25\linewidth]{picture/svg/368}
\end{figure}

\banswer{
\begin{enumerate}
\renewcommand{\labelenumi}{\arabic{enumi}.}
% A(\Alph) a(\alph) I(\Roman) i(\roman) 1(\arabic)
%设定全局标号series=example	%引用全局变量resume=example
%[topsep=-0.3em,parsep=-0.3em,itemsep=-0.3em,partopsep=-0.3em]
%可使用leftmargin调整列表环境左边的空白长度 [leftmargin=0em]
\item
$P = \frac { B ^ { 2 } L ^ { 2 } v ^ { 2 } } { R }$
\item 
$F = \frac { B ^ { 2 } l ^ { 2 } v } { R } + \mu m g$



\end{enumerate}


}



\newpage
\item 
\exwhere{$ 2015 $年上海卷}
如图($ a $),两相距$ L=0.5m $的平行金属导轨固定于水平面上,导轨左端与阻值$ R=2 \Omega $的电阻连接,导轨间虚线右侧存在垂直导轨平面的匀强磁场。质量$ m=0.2 \ kg $的金属杆垂直置于导轨上,与导轨接触良好,导轨与金属杆的电阻可忽略。杆在水平向右的恒定拉力作用下由静止开始运动,并始终与导轨垂直,其$ v - t $图像如图($ b $)所示。在$ 15 \ s $时撤去拉力,同时使磁场随时间变化,从而保持杆中电流为$ 0 $。求:
\begin{enumerate}
\renewcommand{\labelenumi}{\arabic{enumi}.}
% A(\Alph) a(\alph) I(\Roman) i(\roman) 1(\arabic)
%设定全局标号series=example	%引用全局变量resume=example
%[topsep=-0.3em,parsep=-0.3em,itemsep=-0.3em,partopsep=-0.3em]
%可使用leftmargin调整列表环境左边的空白长度 [leftmargin=0em]
\item
金属杆所受拉力的大小$ F $;
\item 
$ 0 \sim15 \ s $内匀强磁场的磁感应强度大小$ B_{0} $;
\item 
$ 15 \sim 20 \ s $内磁感应强度随时间的变化规律。

\end{enumerate}
\begin{figure}[h!]
\flushright
\includesvg[width=0.4\linewidth]{picture/svg/369}
\end{figure}

\banswer{
\begin{enumerate}
\renewcommand{\labelenumi}{\arabic{enumi}.}
% A(\Alph) a(\alph) I(\Roman) i(\roman) 1(\arabic)
%设定全局标号series=example	%引用全局变量resume=example
%[topsep=-0.3em,parsep=-0.3em,itemsep=-0.3em,partopsep=-0.3em]
%可使用leftmargin调整列表环境左边的空白长度 [leftmargin=0em]
\item
$ 0.24 \ N $
\item 
$ 0.4\ T $
\item 
$B ( t ) = \frac { 20 } { 50 + 10 t - t^{2} } T$

\end{enumerate}


}



\item 
\exwhere{$ 2016 $年新课标 \lmd{1} 卷}
如图,两固定的绝缘斜面倾角均为$ \theta $,上沿相连。两细金属棒$ ab $(仅标出$ a $端)和$ cd $(仅标出$ c $端)长度均为$ L $,质量分别为$ 2 \ m $和$ m $;用两根不可伸长的柔软轻导线将它们连成闭合回路$ abdca $,并通过固定在斜面上沿的两光滑绝缘小定滑轮跨放在斜面上,使两金属棒水平。右斜面上存在匀强磁场,磁感应强度大小为$ B $,方向垂直于斜面向上。已知两根导线刚好不在磁场中,回路电阻为$ R $,两金属棒与斜面间的动摩擦因数均为$ \mu $,重力加速度大小为$ g $。已知金属棒$ ab $匀速下滑。求
\begin{enumerate}
\renewcommand{\labelenumii}{(\arabic{enumii})}

\item 
作用在金属棒$ ab $上的安培力的大小;

\item 
金属棒运动速度的大小。

\end{enumerate}
\begin{figure}[h!]
\flushright
\includesvg[width=0.26\linewidth]{picture/svg/372}
\end{figure}

\banswer{
\begin{enumerate}
\renewcommand{\labelenumi}{\arabic{enumi}.}
% A(\Alph) a(\alph) I(\Roman) i(\roman) 1(\arabic)
%设定全局标号series=example	%引用全局变量resume=example
%[topsep=-0.3em,parsep=-0.3em,itemsep=-0.3em,partopsep=-0.3em]
%可使用leftmargin调整列表环境左边的空白长度 [leftmargin=0em]
\item
$F_{ \text{安} }= m g \sin \theta - 3 \mu m g \cos \theta$
\item 
$v = \frac { m g R ( \sin \theta - 3 \mu \cos \theta ) } { B ^ { 2 } L ^ { 2 } }$


\end{enumerate}


}


\newpage
\item 
\exwhere{$ 2016 $年浙江卷}
小明设计的电磁健身器的简化装置如图所示, 两根平行金属导轨相距$ l=0.50\ m $, 倾角$ \theta =53 ^{ \circ } $,导轨上端串接一个$ 0.05 $ $ \Omega $的电阻。在导轨间长$ d=0.56\ m $的区域内,存在方向垂直导轨平面向下的匀强磁场,磁感应强度$ B=2.0 $ $ T $。质量$ m=4.0 \ kg $的金属棒$ CD $水平置于导轨上,用绝缘绳索通过定滑轮与拉杆$ GH $相连。$ CD $棒的初始位置与磁场区域的下边界相距$ s=0.24\ m $。一位健身者用恒力$ F=80 \ N $拉动$ GH $杆,$ CD $棒由静止开始运动,上升过程中$ CD $棒始终保持与导轨垂直。当$ CD $棒到达磁场上边界时健身者松手,触发恢复装置使$ CD $棒回到初始位置(重力加速度$ g=10 \ m/s^{2} $,$ \sin 53 ^{ \circ } =0.8 $,不计其他电阻、摩擦力以及拉杆和绳索的质量)。求:
\begin{enumerate}
\renewcommand{\labelenumi}{\arabic{enumi}.}
% A(\Alph) a(\alph) I(\Roman) i(\roman) 1(\arabic)
%设定全局标号series=example	%引用全局变量resume=example
%[topsep=-0.3em,parsep=-0.3em,itemsep=-0.3em,partopsep=-0.3em]
%可使用leftmargin调整列表环境左边的空白长度 [leftmargin=0em]
\item
$ CD $棒进入磁场时速度$ v $的大小;
\item 
$ CD $棒进入磁场时所受的安培力$ F_A $的大小;
\item 
在拉升$ CD $棒的过程中,健身者所做的功$ W $和电阻产生的焦耳热$ Q $。


\end{enumerate}
\begin{figure}[h!]
\flushright 
\includesvg[width=0.25\linewidth]{picture/svg/371}
\end{figure}


\banswer{
\begin{enumerate}
\renewcommand{\labelenumi}{\arabic{enumi}.}
% A(\Alph) a(\alph) I(\Roman) i(\roman) 1(\arabic)
%设定全局标号series=example	%引用全局变量resume=example
%[topsep=-0.3em,parsep=-0.3em,itemsep=-0.3em,partopsep=-0.3em]
%可使用leftmargin调整列表环境左边的空白长度 [leftmargin=0em]
\item
$v = \sqrt { 2 a x } = 2.4 \mathrm { m } / \mathrm { s }$
\item 
$F _ { \mathrm { A } } = \frac { ( B l ) ^ { 2 } v } { R } = 48 \mathrm { N }$
\item 
$Q = I ^ { 2 } R t = 26.88 \mathrm { J }$



\end{enumerate}


}



\item 
\exwhere{$ 2019 $年物理天津卷}
如图所示,固定在水平面上间距为$ l $的两条平行光滑金属导轨,垂直于导轨放置的两根金属棒$ MN $和$ PQ $长度也为$ l $、电阻均为$ R $,两棒与导轨始终接触良好。$ MN $两端通过开关$ S $与电阻为$ R $的单匝金属线圈相连,线圈内存在竖直向下均匀增加的磁场,磁通量变化率为常量$ k $。图中虚线右侧有垂直于导轨平面向下的匀强磁场,磁感应强度大小为$ B $。$ PQ $的质量为$ m $,金属导轨足够长,电阻忽略不计。
\begin{enumerate}
\renewcommand{\labelenumi}{\arabic{enumi}.}
% A(\Alph) a(\alph) I(\Roman) i(\roman) 1(\arabic)
%设定全局标号series=example	%引用全局变量resume=example
%[topsep=-0.3em,parsep=-0.3em,itemsep=-0.3em,partopsep=-0.3em]
%可使用leftmargin调整列表环境左边的空白长度 [leftmargin=0em]
\item
闭合$ S $,若使$ PQ $保持静止,需在其上加多大水平恒力$ F $,并指出其方向;
\item 
断开$ S $,$ PQ $在上述恒力作用下,由静止开始到速度大小为$ v $的加速过程中流过$ PQ $的电荷量为$ q $,求该过程安培力做的功$ W $。


\end{enumerate}

\begin{figure}[h!]
\flushright 
\includesvg[width=0.3\linewidth]{picture/svg/373}
\end{figure}



\banswer{
\begin{enumerate}
\renewcommand{\labelenumi}{\arabic{enumi}.}
% A(\Alph) a(\alph) I(\Roman) i(\roman) 1(\arabic)
%设定全局标号series=example	%引用全局变量resume=example
%[topsep=-0.3em,parsep=-0.3em,itemsep=-0.3em,partopsep=-0.3em]
%可使用leftmargin调整列表环境左边的空白长度 [leftmargin=0em]
\item
$F = \frac { B k l } { 3 R }$,方向水平向右;
\item 
$W = \frac { 1 } { 2 } m v ^ { 2 } - \frac { 2 } { 3 } k q$



\end{enumerate}


}




\newpage
\item 
\exwhere{$ 2019 $年$ 4 $月浙江物理选考【加试题】}
如图所示,倾角$ \theta =37 ^{\circ} $、间距$ l = 0.1\ m $的足够长金属导轨底端接有阻值$ R=0.1\ \Omega $的电阻,质量$ m=0.1 \ kg $的金属棒$ ab $垂直导轨放置,与导轨间的动摩擦因数$ \mu =0.45 $。建立原点位于底端、方向沿导轨向上的坐标轴$ x $。在$ 0.2\ m \leq x \leq 0.8\ m $区间有垂直导轨平面向上的匀强磁场。从$ t=0 $时刻起,棒$ ab $在沿$ x $轴正方向的外力$ F $作用下从$ x=0 $处由静止开始沿斜面向上运动,其速度与位移$ x $满足$ v=kx $(可导出$ a=kv $)$ k=5 \ s^{-1} $。当棒$ ab $运动至$ x_1=0.2\ m $处时,电阻$ R $消耗的电功率$ P=0.12 \ W $,运动至$ x_2=0.8\ m $处时撤去外力$ F $,此后棒$ ab $将继续运动,最终返回至$ x=0 $处。棒$ ab $始终保持与导轨垂直,不计其它电阻,求:(提示:可以用$ F - x $图象下的“面积”代表力$ F $做的功。
\begin{enumerate}
\renewcommand{\labelenumi}{\arabic{enumi}.}
% A(\Alph) a(\alph) I(\Roman) i(\roman) 1(\arabic)
%设定全局标号series=example	%引用全局变量resume=example
%[topsep=-0.3em,parsep=-0.3em,itemsep=-0.3em,partopsep=-0.3em]
%可使用leftmargin调整列表环境左边的空白长度 [leftmargin=0em]
\item
磁感应强度$ B $的大小;
\item 
外力$ F $随位移$ x $变化的关系式;
\item 
在棒$ ab $整个运动过程中,电阻$ R $产生的焦耳热$ Q $。



\end{enumerate}
\begin{figure}[h!]
\flushright 
\includesvg[width=0.43\linewidth]{picture/svg/374}
\end{figure}


\banswer{
\begin{enumerate}
\renewcommand{\labelenumi}{\arabic{enumi}.}
% A(\Alph) a(\alph) I(\Roman) i(\roman) 1(\arabic)
%设定全局标号series=example	%引用全局变量resume=example
%[topsep=-0.3em,parsep=-0.3em,itemsep=-0.3em,partopsep=-0.3em]
%可使用leftmargin调整列表环境左边的空白长度 [leftmargin=0em]
\item
$\frac { \sqrt { 30 } } { 5 } T$
\item 
无磁场区间:$F = 0.96 + 2.5 x$;有磁场区间:$F = 0.96 + 3.1 x$;
\item 
$ 0.324\ J $

\end{enumerate}


}







\end{enumerate}	




