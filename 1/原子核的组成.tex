\bta{原子核的组成}


\begin{enumerate}
	%\renewcommand{\labelenumi}{\arabic{enumi}.}
	% A(\Alph) a(\alph) I(\Roman) i(\roman) 1(\arabic)
	%设定全局标号series=example	%引用全局变量resume=example
	%[topsep=-0.3em,parsep=-0.3em,itemsep=-0.3em,partopsep=-0.3em]
	%可使用leftmargin调整列表环境左边的空白长度 [leftmargin=0em]
	\item
\exwhere{$ 2016 $ 年上海卷}
卢瑟福通过对$ \alpha $粒子散射实验结果的分析,提出了原子内部存在 \xzanswer{D} 

\fourchoices
{电子}
{中子}
{质子}
{原子核}





\item 
\exwhere{$ 2015 $ 年上海卷}
在$ \alpha $粒子散射实验中,电子对$ \alpha $粒子运动的影响可以忽略。这是因为与$ \alpha $粒子相
比,电子的 \xzanswer{D} 

\fourchoices
{电量太小}
{速度太小}
{体积太小}
{质量太小}

\item 
\exwhere{$ 2012 $ 年物理上海卷}
某种元素具有多种同位素,反映这些同位素的质量数 $ A $ 与中子数 $ N $ 关系的是图 \xzanswer{B} 
\pfourchoices
{\includesvg[width=3cm]{picture/svg/GZ-3-tiyou-1299}}
{\includesvg[width=3cm]{picture/svg/GZ-3-tiyou-1300}}
{\includesvg[width=3cm]{picture/svg/GZ-3-tiyou-1301}}
{\includesvg[width=3cm]{picture/svg/GZ-3-tiyou-1302}}


\item 
\exwhere{$ 2011 $年上海卷}
卢瑟福利用$ \alpha $粒子轰击金箔的实验研究原子结构,正确反映实验结果的示意图是 \xzanswer{D} 
\pfourchoices
{\includesvg[width=4.3cm]{picture/svg/GZ-3-tiyou-1303}}
{\includesvg[width=4.3cm]{picture/svg/GZ-3-tiyou-1304}}
{\includesvg[width=4.3cm]{picture/svg/GZ-3-tiyou-1305}}
{\includesvg[width=4.3cm]{picture/svg/GZ-3-tiyou-1306}}


\item 
\exwhere{$ 2014 $ 年物理上海卷}
不能用卢瑟福原子核式结构模型得出的结论是 \xzanswer{B} 


\fourchoices
{原子中心有一个很小的原子核}
{原子核是由质子和中子组成的}
{原子质量几乎全部集中在原子核内}
{原子的正电荷全部集中在原子核内}




	
	
	
\end{enumerate}

