\bta{整体法和隔离法}

\begin{enumerate}[leftmargin=0em]
\renewcommand{\labelenumi}{\arabic{enumi}.}
% A(\Alph) a(\alph) I(\Roman) i(\roman) 1(\arabic)
%设定全局标号series=example	%引用全局变量resume=example
%[topsep=-0.3em,parsep=-0.3em,itemsep=-0.3em,partopsep=-0.3em]
%可使用leftmargin调整列表环境左边的空白长度 [leftmargin=0em]
\item
\exwhere{$ 2013 $年重庆卷}
如图所示,某人静躺在椅子上,椅子的靠背与水平面之间有固定倾斜角$ \theta $。若此人所受重力为$ G $,则椅子各部分对他的作用力的合力大小为 \xzanswer{A}
\begin{figure}[h!]
\centering
\includesvg[width=0.23\linewidth]{picture/svg/486}
\end{figure}

\fourchoices
{$ G $ }
{$ G \sin \theta $}
{$ G \cos \theta $ }
{$ G \tan \theta $}


\item
\exwhere{$ 2013 $年山东卷}
如图所示,用完全相同的轻弹簧$ A $、$ B $、$ C $将两个相同的小球连接并悬挂,小球处于静止状态,弹簧$ A $与竖直方向夹角为$ 30 ^{ \circ } $,弹簧$ C $水平,则弹簧$ A $、$ C $的伸长量之比为 \xzanswer{D}
\begin{figure}[h!]
\centering
\includesvg[width=0.23\linewidth]{picture/svg/487}
\end{figure}

\fourchoices
{$ \sqrt { 3 }: 4 $}
{$ 4: \sqrt { 3 } $}
{$ 1: 2 $}
{$ 2: 1 $}



\item
\exwhere{$ 2012 $年物理江苏卷}
如图所示,一夹子夹住木块,在力$ F $作用下向上提升. 夹子和木块的质量分别为$ m $、$ M $,夹子与木块两侧间的最大静摩擦力均为$ f $. 若木块不滑动,力$ F $的最大值是 \xzanswer{A}
\begin{figure}[h!]
\centering
\includesvg[width=0.15\linewidth]{picture/svg/488}
\end{figure}




\fourchoices
{$\frac { 2 f ( m + M ) } { M }$}
{$\frac { 2 f ( m + M ) } { m }$}
{$\frac { 2 f ( m + M ) } { M } - ( m + M ) g$}
{$\frac { 2 f ( m + M ) } { m } + ( m + M ) g$}


\newpage
\item 
\exwhere{$ 2012 $年物理上海卷}
如图,光滑斜面固定于水平面,滑块$ A $、$ B $叠放后一起冲上斜面,且始终保持相对静止,$ A $上表面水平。则在斜面上运动时,$ B $受力的示意图为 \xzanswer{A}
\begin{figure}[h!]
\centering
\includesvg[width=0.75\linewidth]{picture/svg/489}
\end{figure}

\item 
\exwhere{$ 2011 $年理综天津卷}
如图所示,$ A $、$ B $两物块叠放在一起,在粗糙的水平面上保持相对静止地向右做匀减速直线运动,运动过程中$ B $受到的摩擦力 \xzanswer{A}
\begin{figure}[h!]
\centering
\includesvg[width=0.23\linewidth]{picture/svg/490}
\end{figure}

\fourchoices
{方向向左,大小不变 }
{方向向左,逐渐减小}
{方向向右,大小不变 }
{方向向右,逐渐减小}



\item 
\exwhere{$ 2016 $年上海卷}
如图,顶端固定着小球的直杆固定在小车上,当小车向右做匀加速运动时,球所受合外力的方向沿图中的 \xzanswer{D}
\begin{figure}[h!]
\centering
\includesvg[width=0.23\linewidth]{picture/svg/491}
\end{figure}


\fourchoices
{$ OA $方向}
{$ OB $方向}
{$ OC $方向}
{$ OD $方向}





\item 
\exwhere{$ 2013 $年北京卷}
倾角为$ \alpha $、质量为$ M $的斜面体静止在水平桌面上,质量为$ m $的木块静止在斜面体上。下列结论正确的是 \xzanswer{D}
\begin{figure}[h!]
\centering
\includesvg[width=0.23\linewidth]{picture/svg/492}
\end{figure}

\fourchoices
{木块受到的摩擦力大小是$ mg \cos \alpha $}
{木块对斜两体的压力大小是$ mg \sin \alpha $}
{桌面对斜面体的摩擦力大小是$ mg \sin \alpha \cos \alpha $}
{桌面对斜面体的支持力大小是$ (M+m)g $}

\newpage
\item 
\exwhere{$ 2011 $年上海卷}
如图,在水平面上的箱子内,带异种电荷的小球$ a $、$ b $用绝缘细线分别系于上、下两边,处于静止状态。地面受到的压力为$ N $,球$ b $所受细线的拉力为$ F $。剪断连接球$ b $的细线后,在球$ b $上升过程中地面受到的压力 \xzanswer{D}
\begin{figure}[h!]
\centering
\includesvg[width=0.15\linewidth]{picture/svg/493}
\end{figure}

\fourchoices
{小于$ N $ }
{等于$ N $}
{等于$ N+F $ }
{大于$ N+F $}


\item
\exwhere{$ 2015 $年理综新课标$ \lmd{2} $卷}
在一东西向的水平直铁轨上,停放着一列已用挂钩连接好的车厢。当机车在东边拉着这列车厢以大小为$ a $的加速度向东行驶时,连接某两相邻车厢的挂钩$ P $和$ Q $间的拉力大小为$ F $;当机车在西边拉着这列车厢以大小为$ \frac{ 2 }{ 3 } a $的加速度向西行驶时, $ P $和$ Q $间的拉力大小仍为$ F $。不计车厢与铁轨间的摩擦,每节车厢质量相同,则这列车厢的节数可能为 \xzanswer{BC}

\fourchoices
{$ 8 $}
{$ 10 $}
{$ 15 $}
{$ 18 $}


\item
\exwhere{$ 2015 $年理综山东卷}
如图,滑块$ A $置于水平地面上,滑块$ B $在一水平力作用下紧靠滑块$ A $($ A $、$ B $接触面竖直),此时$ A $恰好不滑动,$ B $刚好不下滑。已知$ A $与$ B $间的动摩擦因数为$ \mu _1 $,$ A $与地面间的动摩擦因数为
$ \mu _2 $,最大静摩擦力等于滑动摩擦力。$ A $与$ B $的质量之比为 \xzanswer{B}
\begin{figure}[h!]
\centering
\includesvg[width=0.23\linewidth]{picture/svg/494}
\end{figure}

\fourchoices
{$ \frac { 1 } { \mu _ { 1 } \mu _ { 2 } } $}
{$ \frac { 1 - \mu _ { 1 } \mu _ { 2 } } { \mu _ { 1 } \mu _ { 2 } } $}
{$ \frac { 1 + \mu _ { 1 } \mu _ { 2 } } { \mu _ { 1 } \mu _ { 2 } } $}
{$ \frac { 2 + \mu _ { 1 } \mu _ { 2 } } { \mu _ { 1 } \mu _ { 2 } } $}



\item
\exwhere{$ 2017 $年海南卷}
如图,水平地面上有三个靠在一起的物块$ P $、$ Q $和$ R $,质量分别为$ m $、$ 2 \ m $和$ 3 \ m $,物块与地面间的动摩擦因数都为$ \mu $。用大小为$ F $的水平外力推动物块$ P $,设$ R $和$ Q $之间相互作用力与$ Q $与$ P $之间相互作用力大小之比为$ k $。下列判断正确的是 \xzanswer{BD}
\begin{figure}[h!]
\centering
\includesvg[width=0.3\linewidth]{picture/svg/495}
\end{figure}

\fourchoices
{若$ \mu \neq 0 $,则$ k= \frac{ 5 }{ 6 } $ }
{若$ \mu \neq 0 $ ,则$ k= \frac{ 3 }{ 5 } $}
{若$ \mu =0 $,则 $ k= \frac{ 1 }{ 2 } $}
{若$ \mu =0 $,则$ k= \frac{ 3 }{ 5 } $}








\end{enumerate}



