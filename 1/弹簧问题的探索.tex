\bta{弹簧问题的探索}
\begin{enumerate}
\renewcommand{\labelenumi}{\arabic{enumi}.}
% A(\Alph) a(\alph) I(\Roman) i(\roman) 1(\arabic)
%设定全局标号series=example	%引用全局变量resume=example
%[topsep=-0.3em,parsep=-0.3em,itemsep=-0.3em,partopsep=-0.3em]
%可使用leftmargin调整列表环境左边的空白长度 [leftmargin=0em]
\item
\exwhere{$ 2011 $ 年理综安徽卷}
为了测量某一弹簧的劲度系数,将该弹簧竖直悬
挂起来,在自由端挂上不同质量的砝码。实验测出了砝码
的质量 $ m $ 与弹簧长度 $ l $ 的相应数据,其对应点已在图上标
出。
($ g=9.8 \ m/s^{2}) $
\begin{figure}[h!]
\centering
\includesvg[width=0.43\linewidth]{picture/svg/GZ-3-tiyou-0513}
\end{figure}

\begin{enumerate}
\renewcommand{\labelenumi}{\arabic{enumi}.}
% A(\Alph) a(\alph) I(\Roman) i(\roman) 1(\arabic)
%设定全局标号series=example	%引用全局变量resume=example
%[topsep=-0.3em,parsep=-0.3em,itemsep=-0.3em,partopsep=-0.3em]
%可使用leftmargin调整列表环境左边的空白长度 [leftmargin=0em]
\item
作出 $ m-l $ 的关系图线;
\banswer{
 \includesvg[width=0.23\linewidth]{picture/svg/GZ-3-tiyou-0514} 
}

\item 
弹簧的劲度系数为 \tk{$ 0.248\sim 0.262 $} $ N/m $。



\end{enumerate}



\banswer{

}


\newpage
\item
\exwhere{$ 2015 $ 年理综福建卷}
某同学做“探究弹力和弹簧伸长量的关系”的实验。

①图甲是不挂钩码时弹簧下端指针所指的标尺刻度,其示数为 $ 7.73 \ cm $;图乙是在弹簧下端悬挂钩码
后指针所指的标尺刻度,此时弹簧的伸长量$ \Delta l $
为 \tk{6.93} $ cm $;
\begin{figure}[h!]
\centering
\includesvg[width=0.63\linewidth]{picture/svg/GZ-3-tiyou-0515}
\end{figure}


②本实验通过在弹簧下端悬挂钩码的方法来
改变弹簧的弹力,关于此操作,下列选项中规
范的做法是 \tk{A} ;(填选项前的字母)


A.逐一增挂钩码,记下每增加一只钩码后指针所指的标尺刻度和对应的钩码总重

B.随意增减钩码,记下增减钩码后指针所指的标尺刻度和对应的钩码总重


③图丙是该同学所描绘的弹簧的伸长量$ \Delta l $ 与弹力 $ F $ 的关系图线,图线的 $ AB $ 段明显偏离直线 $ OA $,
造成这种现象的主要原因是 \tk{超过弹簧的弹性限度} 。

\banswer{

}

\item 
\exwhere{$ 2015 $ 年理综四川卷}
某同学在”探究弹力和弹簧伸长的关系”时,安装好实验装置,让刻度尺零刻度与弹簧上端平齐,在
弹簧下端挂 $ 1 $ 个钩码,静止时弹簧长度为 $ l_{1} $。如图 $ 1 $ 所示,图 $ 2 $ 是此时固定在弹簧挂钩上的指针在
刻度尺(最小分度是 $ 1 $ 毫米)上位置的放大图,示数 $ l_{1} = $ \tk{25.85} 
$ cm $.。在弹簧下端分别挂 $ 2 $ 个、$ 3 $ 个、$ 4 $ 个、$ 5 $ 个相同钩码,静止
时弹簧长度分别是 $ l_{2} $、$ l_{3} $、$ l_{4} $、$ l_{5} $。已知每个钩码质量是 $ 50 \ g $,挂 $ 2 $
个钩码时,弹簧弹力 $ F_{2} = $
\tk{$ 0.98 $} 
$ N $(当地重力加速度 $ g=9.8 \ m/s^{2} $),
要得到弹簧伸长量 $ x $,还需要测量的是
\tk{弹簧的原长 $ l_{0} $} 
。作出 $ F-x $ 曲线,
得到弹力与弹簧伸长量的关系。
\begin{figure}[h!]
\centering
\includesvg[width=0.43\linewidth]{picture/svg/GZ-3-tiyou-0516}
\end{figure}




\newpage
\item 
\exwhere{$ 2014 $ 年理综浙江卷}
在“探究弹力和弹簧伸长的关系”时,某同学把两根弹簧如图$ 1 $连接起来进行探究.


\begin{enumerate}
\renewcommand{\labelenumi}{\arabic{enumi}.}
% A(\Alph) a(\alph) I(\Roman) i(\roman) 1(\arabic)
%设定全局标号series=example	%引用全局变量resume=example
%[topsep=-0.3em,parsep=-0.3em,itemsep=-0.3em,partopsep=-0.3em]
%可使用leftmargin调整列表环境左边的空白长度 [leftmargin=0em]
\item
某次测量如图 $ 2 $ 所示,指针示数为 \tk{$ (15.95 \sim 16.05)cm $,有效数字位数正确} $ cm $.




\item 
在弹性限度内,将 $ 50 \ g $ 的钩码逐个挂在弹簧下端,得到指针 $ A $、$ B $ 的示数
$ L_A $ 和 $ L_B $ 如表 $ 1 $.用表 $ 1 $ 数据计算弹簧$ \lmd{1} $的劲度系数为 \tk{$ (12.2 \sim 12.8) \ N/m $} $ N/m $(重力加速度 $ g $
取 $ 10 \ m/s^{2}) $.由表 $ 1 $ 数据 \tk{能} (选填“能”或“不能”)计算出弹簧$ \lmd{2} $的劲度系数.
\begin{table}[h!]
\centering 
\begin{tabular}{|c|c|c|c|c|}
\hline 
钩码数 & $ 1 $ & $ 2 $ & $ 3 $ & $ 4 $
 \\
\hline
$ L_A/cm $ & $ 15.71 $ & $ 19.71 $ & $ 23.66 $ & $ 27.76 $
 \\
\hline
$ L_B/cm $ & $ 29.96 $ & $ 35.76 $ & $ 41.51 $ & $ 47.36 $\\ 
\hline 
\end{tabular}
\end{table} 


\end{enumerate}

\begin{figure}[h!]
\centering
\includesvg[width=0.1\linewidth]{picture/svg/GZ-3-tiyou-0517} \qquad 
\includesvg[width=0.38\linewidth]{picture/svg/GZ-3-tiyou-0518} 
\end{figure}


\newpage
\item 
\exwhere{$ 2018 $ 年全国\lmd{1}卷}
如图($ a $)
,一弹簧上端固定在支架顶端,下端悬挂一托
盘;一标尺由游标和主尺构成,主尺竖直固定在弹簧左
边;托盘上方固定有一能与游标刻度线准确对齐的装
置,简化为图中的指针。

现要测量图($ a $)中弹簧的劲度系数。当托盘内没有砝
码时,移动游标,使其零刻度线对准指针,此时标尺
读数为 $ 1.950 \ cm $;当托盘内放有质量为 $ 0.100 \ kg $ 的砝码时,移动游标,再次使其零刻度线对准指针,
标尺示数如图($ b $)所示,其读数为
\tk{$ 3.775 $} 
$ cm $。当地的重力加速度大小为 $ 9.80 \ m/s^{2} $,此弹簧的劲度
系数为
\tk{$ 53.7 $} 
$ N/m $(保留 $ 3 $ 位有效数字)。
\begin{figure}[h!]
\centering
\includesvg[width=0.43\linewidth]{picture/svg/GZ-3-tiyou-0519}
\end{figure}




\newpage
\item
\exwhere{$ 2012 $ 年理综广东卷}
某同学探究弹力与弹簧伸长量的关系。

①将弹簧悬挂在铁架台上,将刻度尺固定在弹簧一侧,弹簧轴线和刻度尺都应在方向 \tk{竖直} (填“水
平”或“竖直”)

②弹簧自然悬挂,待弹簧 \tk{稳定} 时,长度记为$ L_{0} $,弹簧下端挂上砝码盘时,长度记为$ L_x $;在砝码盘
中每次增加$ 10 \ g $砝码,弹簧长度依次记为$ L_{1} $至$ L_{6} $,数据如下表表:
\begin{table}[h!]
\centering 
\begin{tabular}{|c|c|c|c|c|c|c|c|c|}
\hline 
代表符号 & $ L_{0} $ & $ Lx $ & $ L_{1} $ & $ L_{2} $ & $ L_{3} $ & $ L_{4} $ & $ L_{5} $ & $ L_{6} $
 \\
\hline
数值($ cm $) & $ 25.35 $ & $ 27.35 $ & $ 29.35 $ & $ 31.30 $ & $ 33.4 $ & $ 35.35 $ & $ 37.40 $ & $ 39.30 $\\ 
\hline 
\end{tabular}
\end{table} 



表中有一个数值记录不规范,代表符号 \tk{$ L_{3} $} 。由表可知所用刻度尺
的最小长度为 \tk{$ 1 \ mm $}。

③图$ 16 $是该同学根据表中数据作的图,纵轴是砝码的质量,横轴是弹簧
长度与 \tk{$ L_{0} $} 的差值(填“$ L_{0} $或$ L_{1} $”)。
\begin{figure}[h!]
\centering
\includesvg[width=0.33\linewidth]{picture/svg/GZ-3-tiyou-0520}
\end{figure}

④由图可知弹簧的劲度系数为 \tk{$ 4.9 $} $ N/m $;通过图和表可知砝码盘的
质量为 \tk{$ 10 $} $ g $(结果保留两位有效数字,重力加速度取$ 9.8 \ m/s^{2} $)。



\newpage
\item 
\exwhere{$ 2012 $ 年物理海南卷}
一水平放置的轻弹簧,一端固定,另一端与一小滑块接触,但不粘连;初始时滑块静止于水平气
垫导轨上的 $ O $ 点,如图($ a $)所示。现利用此装置探究弹簧的弹性势能 $ E_{p} $ 与其被压缩时长度的改变
量 $ x $ 的关系。先推动小滑块压缩弹簧,用米尺测出 $ x $ 的数值;然后将小滑块从静止释放。用计时器
测出小滑块从 $ O $ 点运动至气垫导轨上另一固定点 $ A $ 所用
时间 $ t $。多次改变 $ x $,测得的 $ x $ 值及其对应的 $ t $ 值如下表
所示。(表中的 $ 1/t $ 值是根据 $ t $ 值计算得出的)
\begin{figure}[h!]
\centering
\includesvg[width=0.4\linewidth]{picture/svg/GZ-3-tiyou-0521}
\end{figure}


\begin{table}[h!]
\centering 
\begin{tabular}{|c|c|c|c|c|c|}
\hline 
$ x(cm) $ & $ 1.00 $ & $ 1.50 $ & $ 2.00 $ & $ 2.50 $ & $ 3.00 $
 \\
\hline
$ t(s) $ & $ 3.33 $ & $ 2.20 $ & $ 1.60 $ & $ 1.32 $ & $ 1.08 $
 \\
\hline
$ 1/t(s^{-1}) $ & $ 0.300 $ & $ 0.455 $ & $ 0.625 $ & $ 0.758 $ & $ 0.926 $\\ 
\hline 
\end{tabular}
\end{table} 



\begin{enumerate}
\renewcommand{\labelenumi}{\arabic{enumi}.}
% A(\Alph) a(\alph) I(\Roman) i(\roman) 1(\arabic)
%设定全局标号series=example	%引用全局变量resume=example
%[topsep=-0.3em,parsep=-0.3em,itemsep=-0.3em,partopsep=-0.3em]
%可使用leftmargin调整列表环境左边的空白长度 [leftmargin=0em]
\item
根据表中数据,在图($ b $)中
的方格纸上作出 $ \frac{1}{t}-x $ 图线。
\begin{figure}[h!]
\centering
\includesvg[width=0.43\linewidth]{picture/svg/GZ-3-tiyou-0522}
\end{figure}

\banswer{
 \includesvg[width=0.23\linewidth]{picture/svg/GZ-3-tiyou-0523} 
}



\item 
回答下列问题:(不要求写出计算
或推导过程)




①已知点($ 0 $,$ 0 $)在 $ \frac{1}{t}-x $ 图线上,
从 $ \frac{1}{t}-x $ 图线看, $ \frac{1}{t} $与$ x $ 是什么关系?

\tk{$\frac{1}{t}$ 与 $x$ 成正比关系} 


②从理论上分析,
小滑块刚脱离弹簧时的动能 $ E_{k} $ 与$ \frac{1}{t} $
是什么关系(不考虑摩擦力)?

\tk{$E_{k}$ 与 $\left(\frac{1}{t}\right)^{2}$ 成正比} 












③当弹簧长度改变量为 $ x $ 时,弹性势能 $ E_{p} $ 与相应的
$ E_k $是什么关系?

\tk{$E_{p}=E_{k}$} 


④综合考虑以上分析, $ E_{p} $ 与 $ x $ 是什么关系?


$E_{k}$ 与 $x^{2}$ 成正比


\end{enumerate}




\newpage
\item 
\exwhere{$ 2014 $ 年理综新课标$ \lmd{2} $卷}
某实验小组探究弹簧的劲度系数 $ k $ 与其长度(圈
数)的关系;实验装置如图($ a $)所示:一均匀长弹簧竖直悬挂,$ 7 $ 个指针 $ P_{0} $、$ P_{1} $、$ P_{2} $、$ P_{3} $、$ P_{4} $、$ P_{5} $、
$ P_{6} $ 分别固定在弹簧上距悬点 $ 0 $、$ 10 $、$ 20 $、$ 30 $、$ 40 $、$ 50 $、$ 60 $ 圈处;通过旁边竖直放置的刻度尺,可以
读出指针的位置,$ P_{0} $ 指向 $ 0 $ 刻度;设弹簧下端未挂重物时,各指针的位置记为 $ x_{0} $;挂有质量为 $ 0.100 \ kg $
砝码时,各指针的位置记为 $ x $;测量结果及部分计算结果如下表所示($ n $ 为弹簧的圈数,取重力加
速度为 $ 9.80 \ m/s^{2} $).已知实验所用弹簧的总圈数为 $ 60 $,整个弹簧的自由长度为 $ 11.88 \ cm $.
\begin{minipage}[h!]{0.6\linewidth}
\vspace{0.5em}
\begin{tabular}{|c|c|c|c|c|c|c|}
\hline 
& $ P_{1} $ & $ P_{2} $ & $ P_{3} $ & $ P_{4} $ & $ P_{5} $ & $ P_{6} $
 \\
\hline
$ x_0(cm) $ & $ 2.04 $ & $ 4.06 $ & $ 6.06 $ & $ 8.05 $ & $ 10.03 $ & $ 12.01 $
 \\
\hline
$ x(cm) $ & $ 2.64 $ & $ 5.26 $ & $ 7.81 $ & $ 10.30 $ & $ 12.93 $ & $ 15.41 $
 \\
\hline
$ n $ & $ 10 $ & $ 20 $ & $ 30 $ & $ 40 $ & $ 50 $ & $ 60 $
 \\
\hline
$ k(N/m) $ & $ 163 $ & ① & $ 56.0 $ & $ 43.6 $ & $ 33.8 $ & $ 28.8 $
 \\
\hline
$ 1/k(m/N) $ & $ 0.0061 $ & ② & $ 0.0179 $ & $ 0.0229 $ & $ 0.0296 $ & $ 0.0347 $\\ 
\hline 
\end{tabular}
\vspace{0.5em}
\end{minipage}
\hfill
\begin{minipage}[h!]{0.3\linewidth}
\flushright
\vspace{0.5em}
\includesvg[width=0.7\linewidth]{picture/svg/GZ-3-tiyou-0524}
\vspace{0.5em}
\end{minipage}




\begin{enumerate}
\renewcommand{\labelenumi}{\arabic{enumi}.}
% A(\Alph) a(\alph) I(\Roman) i(\roman) 1(\arabic)
%设定全局标号series=example	%引用全局变量resume=example
%[topsep=-0.3em,parsep=-0.3em,itemsep=-0.3em,partopsep=-0.3em]
%可使用leftmargin调整列表环境左边的空白长度 [leftmargin=0em]
\item
将表中数据补充完整: ① \tk{$ 81.7 $} , ② \tk{$ 0.0122 $} ;


\item 
以 $ n $ 为横坐标,$ 1/k $ 为纵坐标,在图($ b $)给出的坐标纸上画出 $ 1/k \sim n $ 图像;
\begin{figure}[h!]
\centering
\includesvg[width=0.53\linewidth]{picture/svg/GZ-3-tiyou-0525}
\end{figure}


\banswer{
 \includesvg[width=0.23\linewidth]{picture/svg/GZ-3-tiyou-0526} 
}


\item 
图($ b $)中画出的直线可以近似认为通
过原点;若从实验中所用的弹簧截取圈数为
$ n $ 的一段弹簧,该弹簧的劲度系数 $ k $ 与其圈
数 $ n $ 的关系的表达式为 $ k= $
\tk{$\frac{1.71 \times 10^{3}}{n}$$\left(\text {在} \frac{1.67 \times 10^{3}}{n}-\frac{1.83 \times 10^{3}}{n} \text { 之间均可 }\right)$} 
$ N/m $;该弹
簧的劲度系数 $ k $ 与其自由长度 $ l_{0} $(单位为 $ m) $
的表达式为 $ k= $
\tk{$k=\frac{3.38}{l_{0}},\left(\text { 在 } \frac{3.31}{l_{0}}-\frac{3.62}{l_{0}} \text { 之间均可 }\right)$} 
$ N/m $.

\end{enumerate}






\newpage
\item 
\exwhere{$ 2014 $ 年理综广东卷}
某同学根据机械能守恒定律,设计实验探究弹簧的弹性势能与压缩量的关系.
\begin{figure}[h!]
\centering
\includesvg[width=0.83\linewidth]{picture/svg/GZ-3-tiyou-0527}
\end{figure}

①如图 $ 23(a) $,将轻质弹簧下端固定于铁架台,在上端的托盘中依次增加砝码,测得相应的弹簧长
度,部分数据如下表,有数据算得劲度系数 $ k= $
\tk{$ 50 $} 
$ N/m $,($ g $ 取 $ 9.8 \ m/s^{2} $)

\begin{table}[h!]
\centering 
\begin{tabular}{|c|c|c|c|}
\hline 
砝码质量($ g $) & $ 50 $ & $ 100 $ & $ 150 $
 \\
\hline
弹簧长度($ cm $) & $ 8.62 $ & $ 7.63 $ & $ 6.66 $\\ 
\hline 
\end{tabular}
\end{table} 



②取下弹簧,将其一端固定于气垫导轨左侧,如图 $ 23(b) $所示;调整导轨,使滑块自由滑动时,通
过两个光电门的速度大小 \tk{相等} .


③用滑块压缩弹簧,记录弹簧的压缩量 $ x $;释放滑块,记录滑块脱离弹簧后的速度 $ v $,释放滑块过
程中,弹簧的弹性势能转化为 \tk{动能} .

④重复③中的操作,得到 $ v $ 与 $ x $ 的关系如图 $ 23 $($ c $)。有图可知,$ v $ 与 $ x $ 成 \tk{正比} 关系,由 上述实
验可得结论:对同一根弹簧,弹性势能与弹簧的
\tk{压缩量的平方} 
成正比.





\newpage
\item 
\exwhere{$ 2013 $ 年新课标 \lmd{2} 卷}
某同学利用下述装置对轻质弹簧的弹性势能进行探究,一轻质弹簧放置在光滑水平桌面上,
弹簧左端固定,右端与一小球接触而不固连;弹簧处于原长时,小球恰好在桌面边缘,如图$ (a) $所示。
向左推小球,使弹黄压缩一段距离后由静止释放;小球离开桌面后落到水平地面。通过测量和计算,
可求得弹簧被压缩后的弹性势能。
\begin{figure}[h!]
\centering
\includesvg[width=0.23\linewidth]{picture/svg/GZ-3-tiyou-0528}
\end{figure}



回答下列问题:
\begin{enumerate}
\renewcommand{\labelenumi}{\arabic{enumi}.}
% A(\Alph) a(\alph) I(\Roman) i(\roman) 1(\arabic)
%设定全局标号series=example	%引用全局变量resume=example
%[topsep=-0.3em,parsep=-0.3em,itemsep=-0.3em,partopsep=-0.3em]
%可使用leftmargin调整列表环境左边的空白长度 [leftmargin=0em]
\item
本实验中可认为,弹簧被压缩后的弹性势能 $ E_{p} $ 与小球抛出
时的动能 $ E_{k} $ 相等。已知重力加速度大小为 $ g $。为求得 $ E_{k} $,至
少需要测量下列物理量中的 \tk{ABC} (填正确答案标号)。
\fivechoices
{小球的质量 $ m $}
{小球抛出点到落地点的水平距离 $ s $}
{桌面到地面的高度 $ h $}
{弹簧的压缩量$ \triangle x $}
{弹簧原长 $ l_{0} $}


\item 
用所选取的测量量和已知量表示 $ E_{k} $,得 $ E_{k} = $ \tk{$\frac{m g s^{2}}{4 h}$} 。



\item 
图$ (b) $中的直线是实验测量得到的 $ s- \Delta x $ 图线。从理论上可推出,如果 $ h $ 不变,$ m $ 增加,$ s- \Delta x $ 图线
的斜率会 \tk{减小} (填“增大”、“减小”或“不变”);如果 $ m $ 不变,$ h $ 增加,$ s- \Delta x $ 图线的斜率会 \tk{增大} (填
“增大”、“减小”或“不变”)。由图$ (b) $ 中给出的直线关系和 $ E_{k} $ 的表达式可知,$ E_{p} $ 与$ \triangle x $ 的 \tk{$ 2 $} 次
方成正比。
\begin{figure}[h!]
\centering
\includesvg[width=0.23\linewidth]{picture/svg/GZ-3-tiyou-0529}
\end{figure}




\end{enumerate}







\end{enumerate}

