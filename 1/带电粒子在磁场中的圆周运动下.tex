\bta{带电粒子在磁场中的圆周运动(下)}
\begin{enumerate}[leftmargin=0em]
\renewcommand{\labelenumi}{\arabic{enumi}.}
% A(\Alph) a(\alph) I(\Roman) i(\roman) 1(\arabic)
%设定全局标号series=example	%引用全局变量resume=example
%[topsep=-0.3em,parsep=-0.3em,itemsep=-0.3em,partopsep=-0.3em]
%可使用leftmargin调整列表环境左边的空白长度 [leftmargin=0em]
\item 
\exwhere{$ 2017 $年新课标\lmd{2}卷}
如图,虚线所示的圆形区域内存在一垂直于纸面的匀强磁场,$ P $为磁场边界上的一点,大量相同的带电粒子以相同的速率经过$ P $点,在纸面内沿不同的方向射入磁场,若粒子射入的速度为$ v_{1} $,这些粒子在磁场边界的出射点分布在六分之一圆周上;若粒子射入速度为$ v_{2} $,相应的出射点分布在三分之一圆周上,不计重力及带电粒子之间的相互作用,则$v _ { 2 }: v _ { 1 }$为 \xzanswer{C} 


\begin{minipage}[h!]{0.7\linewidth}
\vspace{0.3em}
\fourchoices
{$\sqrt { 3 }: 2 \quad$}
{$\sqrt { 2 }: 1 \quad$}
{$\sqrt { 3 }: 1 \quad$}
{$3: \sqrt { 2 }$}

\vspace{0.3em}
\end{minipage}
\hfill
\begin{minipage}[h!]{0.3\linewidth}
\flushright
\vspace{0.3em}
\includesvg[width=0.6\linewidth]{picture/svg/220}
\vspace{0.3em}
\end{minipage}


\item
\exwhere{$ 2017 $年新课标\lmd{3}卷}
如图,空间存在方向垂直于纸面($ xOy $平面)向里的磁场。在$ x \geq 0 $区域,磁感应强度的大小为$ B_{0} $;$ x $<$ 0 $区域,磁感应强度的大小为$ \lambda B_{0} $(常数$ \lambda> 1 $)。一质量为$ m $、电荷量为$ q $($ q > 0 $)的带电粒子以速度$ v_{0} $从坐标原点$ O $沿$ x $轴正向射入磁场,此时开始计时,当粒子的速度方向再次沿$ x $轴正向时,求(不计重力)
\begin{enumerate}
\renewcommand{\labelenumi}{\arabic{enumi}.}
% A(\Alph) a(\alph) I(\Roman) i(\roman) 1(\arabic)
%设定全局标号series=example	%引用全局变量resume=example
%[topsep=-0.3em,parsep=-0.3em,itemsep=-0.3em,partopsep=-0.3em]
%可使用leftmargin调整列表环境左边的空白长度 [leftmargin=0em]
\item
粒子运动的时间;
\item 
粒子与$ O $点间的距离。



\end{enumerate}
\begin{figure}[h!]
\flushright
\includesvg[width=0.35\linewidth]{picture/svg/221}
\end{figure}

\banswer{
\begin{enumerate}
\renewcommand{\labelenumi}{\arabic{enumi}.}
% A(\Alph) a(\alph) I(\Roman) i(\roman) 1(\arabic)
%设定全局标号series=example	%引用全局变量resume=example
%[topsep=-0.3em,parsep=-0.3em,itemsep=-0.3em,partopsep=-0.3em]
%可使用leftmargin调整列表环境左边的空白长度 [leftmargin=0em]
\item
$\frac { ( \lambda + 1 ) \pi n } { \lambda q B _ { 0 } }$
\item 
$d = \frac { 2 ( \lambda - 1 ) m v _ { 0 } } { \lambda q B _ { 0 } }$



\end{enumerate}
}



\item
\exwhere{$ 2017 $年浙江选考卷}
如图所示,在$ xOy $平面内,有一电子源持续不断地沿$ x $正方向每秒发射出$ N $个速率均为$ v $的电子,形成宽为$ 2b $,在$ y $轴方向均匀分布且关于$ x $轴对称的电子流。电子流沿$ x $方向射入一个半径为$ R $,中心位于原点$ O $的圆形匀强磁场区域,磁场方向垂直$ xOy $平面向里,电子经过磁场偏转后均从$ P $点射出,在磁场区域的正下方有一对平行于$ x $轴的金属平行板$ K $和$ A $,其中$ K $板与$ P $点的距离为$ d $,中间开有宽度为$ 2l $且关于轴对称的小孔。$ K $板接地,$ A $与$ K $两板间加有正负、大小均可调的电压$ UAK $,穿过$ K $板小孔到达$ A $板的所有电子被收集且导出,从而形成电流。已知$b = \frac { \sqrt { 3 } } { 2 } R$,电子质量为$ m $,电荷量为$ e $,忽略电子间相互作用。



\begin{enumerate}
\renewcommand{\labelenumi}{\arabic{enumi}.}
% A(\Alph) a(\alph) I(\Roman) i(\roman) 1(\arabic)
%设定全局标号series=example	%引用全局变量resume=example
%[topsep=-0.3em,parsep=-0.3em,itemsep=-0.3em,partopsep=-0.3em]
%可使用leftmargin调整列表环境左边的空白长度 [leftmargin=0em]
\item
求磁感应强度$ B $的大小;
\item 
求电子从$ P $点射出时与负轴方向的夹角$ \theta $的范围;
\item 
当$ U_{AK}=0 $时,每秒经过极板$ K $上的小孔到达极板$ A $的电子数;
\item 
画出电流$ i $随$ U_{AK} $变化的关系曲线(在答题纸上的方格纸上)。 

\end{enumerate}
\begin{figure}[h!]
\flushright
\includesvg[width=0.35\linewidth]{picture/svg/222}
\end{figure}

\banswer{
\begin{enumerate}
\renewcommand{\labelenumi}{\arabic{enumi}.}
% A(\Alph) a(\alph) I(\Roman) i(\roman) 1(\arabic)
%设定全局标号series=example	%引用全局变量resume=example
%[topsep=-0.3em,parsep=-0.3em,itemsep=-0.3em,partopsep=-0.3em]
%可使用leftmargin调整列表环境左边的空白长度 [leftmargin=0em]
\item
$B = \frac { m v } { e R }$
\item 
$60 ^ { \circ }$
\item 
$n = \frac { \sqrt { 6 } } { 3 } N = 0.82 N$
\item 
$i _ { \max } = 0.82 N e$
\item 
电流$ i $随$ U_{AK} $变化的关系曲线如右图示(略)
%\includesvg[width=0.23\linewidth]{picture/svg/223}


\end{enumerate}
}


\newpage
\item
\exwhere{$ 2013 $年全国卷大纲卷}
如图,虚线$ OL $与$ y $轴的夹角$ \theta =60 ^{ \circ } $,在此角范围内有垂直于$ xOy $平面向外的匀强磁场,磁感应强度大小为$ B $。一质量为$ m $、电荷量为$ q $($ q>0 $)的粒子从左侧平行于$ x $轴射入磁场,入射点为$ M $。粒子在磁场中运动的轨道半径为$ R $。粒子离开磁场后的运动轨迹与$ x $轴交于$ P $点(图中未画出)且$\overline { O P } = R$。不计重力。求$ M $点到$ O $点的距离和粒子在磁场中运动的时间。
\begin{figure}[h!]
\flushright
\includesvg[width=0.29\linewidth]{picture/svg/224}
\end{figure}


\banswer{
根据题意,带电粒子进入磁场后做圆周运动,运动轨迹交虚线$ OL $于$ A $点,圆心为$ y $轴上的$ C $点,$ AC $与$ y $轴的夹角为$ \alpha $;粒子从$ A $点射出后,运动轨迹交$ x $轴于$ P $点,与$ x $轴的夹角为$ \beta $,可得$\alpha = \beta$和$\sin \alpha + \frac { 1 } { \sqrt { 3 } } \cos \alpha = 1$,解得$\alpha = 30 ^ { \circ }$或$\alpha = 90 ^ { \circ }$.\\
当$\alpha = 30 ^ { \circ }$时,粒子在磁场中运动的时间为$t = \frac { T } { 12 } = \frac { \pi m } { 6 q B }$.\\
当$\alpha = 90 ^ { \circ }$时,粒子在磁场中运动的时间为$t = \frac { T } { 4 } = \frac { \pi m } { 2 q B }$.
}


\newpage
\item
\exwhere{$ 2012 $年理综山东卷}
如图甲所示,相隔一定距离的竖直边界两侧为相同的匀强磁场区,磁场方向垂直纸面向里,在边界上固定两长为$ L $的平行金属极板$ MN $和$ PQ $,两极板中心各有一小孔$ S_{1} $、$ S_{2} $,两极板间电压的变化规律如图乙所示,正反向电压的大小均为$ U_{0} $,周期为$ T_{0} $。在$ t=0 $时刻将一个质量为$ m $、电量为$ -q $($ q>0 $)的粒子由$ S_{1} $静止释放,粒子在电场力的作用下向右运动,在$t = \frac { T _ { 0 } } { 2 }$时刻通过$ S_{2} $垂直于边界进入右侧磁场区。(不计粒子重力,不考虑极板外的电场)
\begin{enumerate}
\renewcommand{\labelenumi}{\arabic{enumi}.}
% A(\Alph) a(\alph) I(\Roman) i(\roman) 1(\arabic)
%设定全局标号series=example	%引用全局变量resume=example
%[topsep=-0.3em,parsep=-0.3em,itemsep=-0.3em,partopsep=-0.3em]
%可使用leftmargin调整列表环境左边的空白长度 [leftmargin=0em]
\item
求粒子到达$ S_{2} $时的速度大小$ v $和极板间距$ d $;
\item 
为使粒子不与极板相撞,求磁感应强度的大小应满足的条件;
\item 
已保证了粒子未与极板相撞,为使粒子在$ t=3 \ T_0 $时刻再次到达$ S_{2} $,且速度恰好为零,求该过程中粒子在磁场内运动的时间和磁感强度的大小.

\end{enumerate}
\begin{figure}[h!]
\flushright
\includesvg[width=0.59\linewidth]{picture/svg/225}
\end{figure}


\banswer{
\begin{enumerate}
\renewcommand{\labelenumi}{\arabic{enumi}.}
% A(\Alph) a(\alph) I(\Roman) i(\roman) 1(\arabic)
%设定全局标号series=example	%引用全局变量resume=example
%[topsep=-0.3em,parsep=-0.3em,itemsep=-0.3em,partopsep=-0.3em]
%可使用leftmargin调整列表环境左边的空白长度 [leftmargin=0em]
\item
$v = \sqrt { \frac { 2 q U _ { 0 } } { m } }$;$d = \frac { T _ { 0 } } { 4 } \sqrt { \frac { 2 q U _ { 0 } } { m } }$
\item 
$B < \frac { 4 } { L } \sqrt { \frac { 2 m U _ { 0 } } { q } }$
\item 
$t = \frac { 7 T _ { 0 } } { 4 }$;$B = \frac { 8 \pi n } { 7 q T _ { 0 } }$.



\end{enumerate}
}




\newpage
\item
\exwhere{$ 2011 $年理综福建卷}
如图甲,在$ x $ $ >0 $的空间中存在沿$ y $轴负方向的匀强电场和垂直于$ xOy $平面向里的匀强磁场,电场强度大小为$ E $,磁感应强度大小为$ B $。一质量为$ m $,带电量为$ q $($ q>0 $)的粒子从坐标原点$ O $处,以初速度$ v_{0} $沿$ x $轴正方向射入,粒子的运动轨迹见图甲,不计粒子的重力。
\begin{enumerate}
\renewcommand{\labelenumi}{\arabic{enumi}.}
% A(\Alph) a(\alph) I(\Roman) i(\roman) 1(\arabic)
%设定全局标号series=example	%引用全局变量resume=example
%[topsep=-0.3em,parsep=-0.3em,itemsep=-0.3em,partopsep=-0.3em]
%可使用leftmargin调整列表环境左边的空白长度 [leftmargin=0em]
\item
求该粒子运动到$ y=h $时的速度大小$ v $;
\item 
现只改变入射粒子初速度的大小,发现初速度大小不同的粒子虽然运动轨迹($ y-x $曲线)不同,但具有相同的空间周期性,如图乙所示;同时,这些粒子在$ y $轴方向上的运动($ y-t $关系)是简谐运动,且都有相同的周期$T = \frac { 2 \pi m } { q B }$。
\begin{enumerate}
\renewcommand{\labelenumiii}{\roman{enumiii}}
% A(\Alph) a(\alph) I(\Roman) i(\roman) 1(\arabic)
%设定全局标号series=example	%引用全局变量resume=example
%[topsep=-0.3em,parsep=-0.3em,itemsep=-0.3em,partopsep=-0.3em]
%可使用leftmargin调整列表环境左边的空白长度 [leftmargin=0em]
\item
求粒子在一个周期$ T $内,沿$ x $轴方向前进的距离$ s $;
\item 
当入射粒子的初速度大小为$ v_{0} $时,其$ y-t $图像如图丙所示,求该粒子在$ y $轴方向上做简谐运动的振幅$ A_y $,并写出$ y-t $的函数表达式。


\end{enumerate}


\end{enumerate}
\begin{figure}[h!]
\flushright
\includesvg[width=0.85\linewidth]{picture/svg/226}
\end{figure}

\banswer{
\begin{enumerate}
\renewcommand{\labelenumi}{\arabic{enumi}.}
% A(\Alph) a(\alph) I(\Roman) i(\roman) 1(\arabic)
%设定全局标号series=example	%引用全局变量resume=example
%[topsep=-0.3em,parsep=-0.3em,itemsep=-0.3em,partopsep=-0.3em]
%可使用leftmargin调整列表环境左边的空白长度 [leftmargin=0em]
\item
$v = \sqrt { v _ { 0 } ^ { 2 } - \frac { 2 q E h } { m } }$
\item 
\begin{enumerate}
\renewcommand{\labelenumiii}{\roman{enumiii}}
% A(\Alph) a(\alph) I(\Roman) i(\roman) 1(\arabic)
%设定全局标号series=example	%引用全局变量resume=example
%[topsep=-0.3em,parsep=-0.3em,itemsep=-0.3em,partopsep=-0.3em]
%可使用leftmargin调整列表环境左边的空白长度 [leftmargin=0em]
\item
$s = \frac { 2 \pi m E } { q B ^ { 2 } }$
\item 
$y = \frac { m } { q B } \left( v _ { 0 } - \frac { E } { B } \right) \left( 1 - \cos \frac { q B } { m } t \right)$



\end{enumerate}



\end{enumerate}
}


\newpage
\item
\exwhere{$ 2011 $年理综广东卷}
如图$ 19(a) $所示,在以$ O $为圆心,内外半径分别为$ R_{1} $和$ R_{2} $的圆环区域内,存在辐射状电场和垂直纸面的匀强磁场,内外圆间的电势差$ U $为常量,$ R_{1}=R_{0} $,$ R_{2}=3R_{0} $,一电荷量为$ +q $,质量为$ m $的粒子从内圆上的$ A $点进入该区域,不计重力。
\begin{enumerate}
\renewcommand{\labelenumi}{\arabic{enumi}.}
% A(\Alph) a(\alph) I(\Roman) i(\roman) 1(\arabic)
%设定全局标号series=example	%引用全局变量resume=example
%[topsep=-0.3em,parsep=-0.3em,itemsep=-0.3em,partopsep=-0.3em]
%可使用leftmargin调整列表环境左边的空白长度 [leftmargin=0em]
\item
已知粒子从外圆上以速度$ v_{1} $射出,求粒子在$ A $点的初速度$ v_{0} $的大小。
\item 
若撤去电场,如图$ (b) $,已知粒子从$ OA $延长线与外圆的交点$ C $以速度$ v_{2} $射出,方向与$ OA $延长线成$ 45 ^{ \circ } $角,求磁感应强度的大小及粒子在磁场中运动的时间。
\item 
在图$ (b) $中,若粒子从$ A $点进入磁场,速度大小为$ v_{3} $,方向不确定,要使粒子一定能够从外圆射出,磁感应强度应小于多少?



\end{enumerate}
\begin{figure}[h!]
\flushright
\includesvg[width=0.47\linewidth]{picture/svg/227}
\end{figure}


\banswer{
\begin{enumerate}
\renewcommand{\labelenumi}{\arabic{enumi}.}
% A(\Alph) a(\alph) I(\Roman) i(\roman) 1(\arabic)
%设定全局标号series=example	%引用全局变量resume=example
%[topsep=-0.3em,parsep=-0.3em,itemsep=-0.3em,partopsep=-0.3em]
%可使用leftmargin调整列表环境左边的空白长度 [leftmargin=0em]
\item
$v _ { 0 } = \sqrt { v _ { 1 } ^ { 2 } - \frac { 2 q U } { m } }$
\item 
$t = \frac { T } { 4 } = \frac { 2 \pi m } { 4 q B } = \frac { 2 \pi m } { \frac { 4 m v _ { 2 } } { \sqrt { 2 } R _ { 0 } } } = \frac { \sqrt { 2 } \pi R _ { 0 } } { 2 v _ { 2 } }$
\item 
考虑两种临界情况,综合可得:$B ^ { \prime } < \frac { m v _ { 3 } } { 2 q R _ { 0 } }$。



\end{enumerate}
}


\newpage
\item
\exwhere{$ 2011 $年理综山东卷}
扭摆器是同步辐射装置中的插入件,能使粒子的运动轨迹发生扭摆。其简化模型如图:\lmd{1}、\lmd{2}两处的条形匀强磁场区边界竖直,相距为$ L $,磁场方向相反且垂直纸面。一质量为$ m $、电量为$ -q $、重力不计的粒子,从靠近平行板电容器$ MN $板处由静止释放,极板间电压为$ U $,粒子经电场加速后平行于纸面射入Ⅰ区,射入时速度与水平方向夹角$ \theta =30 $º.
\begin{enumerate}
\renewcommand{\labelenumi}{\arabic{enumi}.}
% A(\Alph) a(\alph) I(\Roman) i(\roman) 1(\arabic)
%设定全局标号series=example	%引用全局变量resume=example
%[topsep=-0.3em,parsep=-0.3em,itemsep=-0.3em,partopsep=-0.3em]
%可使用leftmargin调整列表环境左边的空白长度 [leftmargin=0em]
\item
当 \lmd{1} 区宽度$ L_1=L $、磁感应强度大小$ B_1=B_0 $时,粒子从 \lmd{1} 区右边界射出时速度与水平方向夹角也为$ 30 ^{\circ} $,求$ B_{0} $及粒子在Ⅰ区运动的时间$ t $。
\item 
若 \lmd{2} 区宽度$ L_2=L_1=L $、磁感应强度大小$ B_2=B_1=B_0 $,求粒子在 \lmd{1} 区的最高点与 \lmd{2} 区的最低点之间的高度差$ h $。
\item 
若$ L_2=L_1=L $、$ B_1=B_0 $,为使粒子能返回 \lmd{1} 区,求$ B_{2} $应满足的条件。
\item 
若$ B_1 \neq B_2 $,$ L_1 \neq L_2 $,且已保证了粒子能从 \lmd{2} 区右边界射出。为使粒子从 \lmd{2} 区右边界射出的方向与从 \lmd{1} 区左边界射出的方向总相同,求$ B_{1} $、$ B_{2} $、$ L_{1} $、$ L_{2} $之间应满足的关系式。



\end{enumerate}
\begin{figure}[h!]
\flushright
\includesvg[width=0.5\linewidth]{picture/svg/228}
\end{figure}

\banswer{
\begin{enumerate}
\renewcommand{\labelenumi}{\arabic{enumi}.}
% A(\Alph) a(\alph) I(\Roman) i(\roman) 1(\arabic)
%设定全局标号series=example	%引用全局变量resume=example
%[topsep=-0.3em,parsep=-0.3em,itemsep=-0.3em,partopsep=-0.3em]
%可使用leftmargin调整列表环境左边的空白长度 [leftmargin=0em]
\item
$B _ { 0 } = \frac { 1 } { L } \sqrt { \frac { 2 m U } { q } }$;$t = \frac { \pi L } { 3 } \sqrt { \frac { m } { 2 q U } }$
\item 
$h = \left( 2 - \frac { 2 \sqrt { 3 } } { 3 } \right) L$
\item 
$B _ { 2 } \geq \frac { 3 } { L } \sqrt { \frac { m U } { 2 q } }$
\item 
$B _ { 1 } L _ { 1 } = B _ { 2 } L _ { 2 }$


\end{enumerate}
}



\newpage
\item
\exwhere{$ 2011 $年理综四川卷}
如图所示,正方形绝缘光滑水平台面$ WXYZ $边长$ l=1.8\ m $,距地面$ h=0.8\ m $。平行板电容器的极板$ CD $间距$ d=0.1\ m $且垂直放置于台面。$ C $板位于边界$ WX $上,$ D $板与边界$ WZ $相交处有一小孔。电容器外的台面区域内有磁感应强度$ B=1 \ T $,方向竖直向上的匀强磁场。电荷量$ q=5 \times 10^{-13}\ C $的微粒静止于$ W $处,在$ CD $间加上恒定电压$ U=2.5 \ V $,板间微粒经电场加速后由$ D $板所开小孔进入磁场(微粒始终不与极板接触),然后由$ XY $边界离开台面。在微粒离开台面瞬时,静止于$ X $正下方水平地面上$ A $点的滑块获得一水平速度,在微粒落地时恰好与之相遇。假定微粒在真空中运动、极板间电场视为匀强电场,滑块视为质点。滑块与地面间的动摩擦因数$ \mu =0.2 $,取$ g=10 \ m/s ^{2} $。
\begin{enumerate}
\renewcommand{\labelenumi}{\arabic{enumi}.}
% A(\Alph) a(\alph) I(\Roman) i(\roman) 1(\arabic)
%设定全局标号series=example	%引用全局变量resume=example
%[topsep=-0.3em,parsep=-0.3em,itemsep=-0.3em,partopsep=-0.3em]
%可使用leftmargin调整列表环境左边的空白长度 [leftmargin=0em]
\item
求微粒在极板间所受电场力的大小并说明两板的极性;
\item 
求由$ XY $边界离开台面的微粒的质量范围;
\item 
若微粒质量$ m_0=1 \times 10^{-13} \ kg $,求滑块开始运动所获得的速度。



\end{enumerate}
\begin{figure}[h!]
\flushright
\includesvg[width=0.4\linewidth]{picture/svg/229}
\end{figure}

\banswer{
\begin{enumerate}
\renewcommand{\labelenumi}{\arabic{enumi}.}
% A(\Alph) a(\alph) I(\Roman) i(\roman) 1(\arabic)
%设定全局标号series=example	%引用全局变量resume=example
%[topsep=-0.3em,parsep=-0.3em,itemsep=-0.3em,partopsep=-0.3em]
%可使用leftmargin调整列表环境左边的空白长度 [leftmargin=0em]
\item
故$ C $板为正极,$ D $板为负极。
\item 
$8.1 \times 10 ^ { - 14 } \mathrm { kg } < m \leq 2.89 \times 10 ^ { - 13 } \mathrm { kg }$
\item 
$v _ { 0 } = 4.15 \mathrm { m } / \mathrm { s }$(方向略)


\end{enumerate}
}



\newpage
\item
\exwhere{$ 2011 $年理综重庆卷}
某仪器用电场和磁场来控制电子在材料表面上方的运动。如图所示,材料表面上方矩形区域$ PP ^{\prime} N ^{\prime} N $充满竖直向下的匀强电场,宽为$ d $;矩形区域$ NN ^{\prime} M ^{\prime} M $充满垂直纸面向里的匀强磁场,磁感应强度为$ B $,长为$ 3 \ s $,宽为$ s $;$ NN ^{\prime} $为磁场与电场之间的薄隔离层。一个电荷量为$ e $、质量为$ m $、初速为零的电子,从$ P $点开始被电场加速经隔离层垂直进入磁场,电子每次穿越隔离层,运动方向不变,其动能损失是每次穿越前动能的$ 10 \% $,最后电子仅能从磁场边界$ M ^{\prime} N ^{\prime} $飞出。不计电子所受重力。
\begin{enumerate}
\renewcommand{\labelenumi}{\arabic{enumi}.}
% A(\Alph) a(\alph) I(\Roman) i(\roman) 1(\arabic)
%设定全局标号series=example	%引用全局变量resume=example
%[topsep=-0.3em,parsep=-0.3em,itemsep=-0.3em,partopsep=-0.3em]
%可使用leftmargin调整列表环境左边的空白长度 [leftmargin=0em]
\item
求电子第二次与第一次圆周运动半径之比;
\item 
求电场强度的取值范围;
\item 
$ A $是$ M ^{\prime} N ^{\prime} $的中点,若要使电子在$ A $、$ M ^{\prime} $间垂直于$ AM ^{\prime} $飞出,求电子在磁场区域中运动的时间。



\end{enumerate}
\begin{figure}[h!]
\flushright
\includesvg[width=0.45\linewidth]{picture/svg/230}
\end{figure}


\banswer{
\begin{enumerate}
\renewcommand{\labelenumi}{\arabic{enumi}.}
% A(\Alph) a(\alph) I(\Roman) i(\roman) 1(\arabic)
%设定全局标号series=example	%引用全局变量resume=example
%[topsep=-0.3em,parsep=-0.3em,itemsep=-0.3em,partopsep=-0.3em]
%可使用leftmargin调整列表环境左边的空白长度 [leftmargin=0em]
\item
$\frac { R _ { 2 } } { R _ { 1 } } = 0.9$
\item 
$\frac { B ^ { 2 } e s ^ { 2 } } { 80 m d } < E \leq \frac { 5 B ^ { 2 } e s ^ { 2 } } { 9 m d }$
\item 
$t = \frac { 5 \pi m } { 2 e B }$



\end{enumerate}
}



\newpage
\item
\exwhere{$ 2011 $年江苏卷}
某种加速器的理想模型如图$ 1 $所示:两块相距很近的平行小极板中间各开有一小孔$ a $、$ b $,两极板间电压$ uab $的变化图象如图$ 2 $所示,电压的最大值为$ U_{0} $、周期为$ T_{0} $,在两极板外有垂直纸面向里的匀强磁场。若将一质量为$ m_{0} $、电荷量为$ q $的带正电的粒子从板内$ a $孔处静止释放,经电场加速后进入磁场,在磁场中运动时间$ T_{0} $后恰能再次从$ a $ 孔进入电场加速。现该粒子的质量增加了$ \frac{1}{100}m_{0} $。(粒子在两极板间的运动时间不计,两极板外无电场,不考虑粒子所受的重力)
\begin{enumerate}
\renewcommand{\labelenumi}{\arabic{enumi}.}
% A(\Alph) a(\alph) I(\Roman) i(\roman) 1(\arabic)
%设定全局标号series=example	%引用全局变量resume=example
%[topsep=-0.3em,parsep=-0.3em,itemsep=-0.3em,partopsep=-0.3em]
%可使用leftmargin调整列表环境左边的空白长度 [leftmargin=0em]
\item
若在$ t=0 $时刻将该粒子从板内$ a $孔处静止释放,求其第二次加速后从$ b $孔射出时的动能;
\item 
现要利用一根长为$ L $的磁屏蔽管(磁屏蔽管置于磁场中时管内无磁场,忽略其对管外磁场的影响),使图$ 1 $中实线轨迹(圆心为$ O $)上运动的粒子从$ a $孔正下方相距$ L $处的$ c $孔水平射出,请在答题卡图上的相应位置处画出磁屏蔽管;
\item 
若将电压$ u_{ab} $的频率提高为原来的$ 2 $倍,该粒子应何时由板内$ a $孔处静止开始加速,才能经多次加速后获得最大动能?最大动能是多少?



\end{enumerate}
\begin{figure}[h!]
\flushright
\includesvg[width=0.55\linewidth]{picture/svg/231}
\end{figure}

\banswer{
\begin{enumerate}
\renewcommand{\labelenumi}{\arabic{enumi}.}
% A(\Alph) a(\alph) I(\Roman) i(\roman) 1(\arabic)
%设定全局标号series=example	%引用全局变量resume=example
%[topsep=-0.3em,parsep=-0.3em,itemsep=-0.3em,partopsep=-0.3em]
%可使用leftmargin调整列表环境左边的空白长度 [leftmargin=0em]
\item
$\frac { 49 } { 25 } q U _ { 0 }$ 
\item 
如图
\item 
$\frac { 313 } { 25 } q U _ { 0 }$



\end{enumerate}
}



\newpage
\item
\exwhere{$ 2014 $年物理江苏卷}
某装置用磁场控制带电粒子的运动,工作原理如图所示。 装置的长为 $ L $,上下两个相同的矩形区域内存在匀强磁场,磁感应强度大小均为 $ B $、方向与纸面垂直且相反,两磁场的间距为 $ d $. 装置右端有一收集板,$ M $、$ N $、$ P $ 为板上的三点,$ M $ 位于轴线 $ OO ^{\prime} $上,$ N $、$ P $ 分别位于下方磁场的上、下边界上。 在纸面内,质量为 $ m $、电荷量为$ -q $ 的粒子以某一速度从装置左端的中点射入,方向与轴线成 $ 30 $ $ ^{ \circ } $ 角,经过上方的磁场区域一次,恰好到达 $ P $ 点. 改变粒子入射速度的大小,可以控制粒子到达收集板上的位置。 不计粒子的重力.
\begin{enumerate}
\renewcommand{\labelenumi}{\arabic{enumi}.}
% A(\Alph) a(\alph) I(\Roman) i(\roman) 1(\arabic)
%设定全局标号series=example	%引用全局变量resume=example
%[topsep=-0.3em,parsep=-0.3em,itemsep=-0.3em,partopsep=-0.3em]
%可使用leftmargin调整列表环境左边的空白长度 [leftmargin=0em]
\item
求磁场区域的宽度 $ h $; 
\item 
欲使粒子到达收集板的位置从 $ P $ 点移到 $ N $ 点,求粒子入射速度的最小变化量$ \Delta v $;
\item 
欲使粒子到达 $ M $ 点,求粒子入射速度大小的可能值.



\end{enumerate}
\begin{figure}[h!]
\flushright
\includesvg[width=0.45\linewidth]{picture/svg/232}
\end{figure}

\banswer{
\begin{enumerate}
\renewcommand{\labelenumi}{\arabic{enumi}.}
% A(\Alph) a(\alph) I(\Roman) i(\roman) 1(\arabic)
%设定全局标号series=example	%引用全局变量resume=example
%[topsep=-0.3em,parsep=-0.3em,itemsep=-0.3em,partopsep=-0.3em]
%可使用leftmargin调整列表环境左边的空白长度 [leftmargin=0em]
\item
$\left( \frac { 2 } { 3 } L - \sqrt { 3 } d \right) \left( 1 - \frac { \sqrt { 3 } } { 2 } \right)$
\item 
$\frac { q B } { m } \left( \frac { L } { 6 } - \frac { \sqrt { 3 } } { 4 } d \right)$
\item 
$v _ { n } = \frac { q B } { m } \left( \frac { L } { n + 1 } - \sqrt { 3 } d \right) \quad \left( 1 \leq n < \frac { \sqrt { 3 } L } { 3 d } - 1 , n\text{取整数} \right)$



\end{enumerate}
}


\newpage
\item
\exwhere{$ 2014 $年理综山东卷}
如图甲所示,间距为$ d $垂直于纸面的两平行板$ P $、$ Q $间存在匀强磁场。取垂直于纸面向里为磁场的正方向,磁感应强度随时间的变化规律如图乙所示。$ t=0 $时刻,一质量为$ m $、带电量为$ +q $的粒子(不计重力),以初速度$ v_{0} $由$ Q $板左端靠近板面的位置,沿垂直于磁场且平行于板面的方向射入磁场区。当$ B_{0} $和$ TB $取某些特定值时,可使$ t=0 $时刻入射的粒子经时间恰能垂直打在$ P $板上(不考虑粒子反弹)。上述$ m $、$ q $、$ d $、$ v_{0} $为已知量。
\begin{enumerate}
\renewcommand{\labelenumi}{\arabic{enumi}.}
% A(\Alph) a(\alph) I(\Roman) i(\roman) 1(\arabic)
%设定全局标号series=example	%引用全局变量resume=example
%[topsep=-0.3em,parsep=-0.3em,itemsep=-0.3em,partopsep=-0.3em]
%可使用leftmargin调整列表环境左边的空白长度 [leftmargin=0em]
\item
若$\Delta t = \frac { 1 } { 2 } T _ { B }$,求$ B_{0} $;
\item 
若$\Delta t = \frac { 3 } { 2 } T _ { B }$,求粒子在磁场中运动时加速度的大小;
\item 
若$B _ { 0 } = \frac { 4 m v _ { 0 } } { q d }$,为使粒子仍能垂直打在$ P $板上,求$ T_B $。



\end{enumerate}
\begin{figure}[h!]
\flushright
\includesvg[width=0.5\linewidth]{picture/svg/233}
\end{figure}


\banswer{
\begin{enumerate}
\renewcommand{\labelenumi}{\arabic{enumi}.}
% A(\Alph) a(\alph) I(\Roman) i(\roman) 1(\arabic)
%设定全局标号series=example	%引用全局变量resume=example
%[topsep=-0.3em,parsep=-0.3em,itemsep=-0.3em,partopsep=-0.3em]
%可使用leftmargin调整列表环境左边的空白长度 [leftmargin=0em]
\item
$B _ { 0 } = \frac { m v _ { 0 } } { q d }$
\item 
$a = \frac { 3 v _ { 0 } ^ { 2 } } { d }$
\item 
设经历整个完整$ T_B $的个数为$ n $(n=0、1、2、3……)\\
当n=0时,无解.\\
当n=1时,$ T _ { B } = \left(\frac { \pi } { 2 } + \arcsin \frac { 1 } { 4 } \right) \frac { d } { 2 v _ { 0 } }$\\
当$n \geq 2$时,不满足$0 < \theta < \frac { \pi } { 2 }$的要求.



\end{enumerate}
}

\newpage
\item
\exwhere{$ 2014 $年理综广东卷}
如图所示,足够大的平行挡板$ A_{1} $、$ A_{2} $竖直放置,间距$ 6L $.两板间存在两个方向相反的匀强磁场区域 \lmd{1} 和 \lmd{2} ,以水平面$ MN $为理想分界面, \lmd{1} 区的磁感应强度为$ B_{0} $,方向垂直纸面向外 $. A1 $、$ A_{2} $上各有位置正对的小孔$ S_{1} $、$ S_{2} $,两孔与分界面$ MN $的距离均为$ L $.质量为$ m $、电量为$ +q $的粒子经宽度为$ d $的匀强电场由静止加速后,沿水平方向从$ S_{1} $进入 \lmd{1} 区,并直接偏转到$ MN $上的$ P $点,再进入 \lmd{2} 区,$ P $点与$ A_{1} $板的距离是$ L $的$ k $倍,不计重力,碰到挡板的粒子不予考虑.
\begin{enumerate}
\renewcommand{\labelenumii}{(\arabic{enumii})}

\item 
若$ k=1 $,求匀强电场的电场强度$ E $;

\item 
若$ 2<k<3 $,且粒子沿水平方向从$ S_{2} $射出,求出粒子在磁场中的速度大小$ v $与$ k $的关系式和 \lmd{2} 区的磁感应强度$ B $与$ k $的关系式.




\end{enumerate}
\begin{figure}[h!]
\flushright
\includesvg[width=0.39\linewidth]{picture/svg/234}
\end{figure}

\banswer{
\begin{enumerate}
\renewcommand{\labelenumi}{\arabic{enumi}.}
% A(\Alph) a(\alph) I(\Roman) i(\roman) 1(\arabic)
%设定全局标号series=example	%引用全局变量resume=example
%[topsep=-0.3em,parsep=-0.3em,itemsep=-0.3em,partopsep=-0.3em]
%可使用leftmargin调整列表环境左边的空白长度 [leftmargin=0em]
\item
$E = \frac { q B _ { 0 } ^ { 2 } L ^ { 2 } } { 2 m d }$
\item 
$v = \frac { \left( k ^ { 2 } + 1 \right) } { 2 m } q B _ { 0 } L$;粒子在 \lmd{2} 区洛伦兹力提供向心力,得$B = \frac { k } { 3 - k } B _ { 0 }$.



\end{enumerate}
}


\newpage
\item
\exwhere{$ 2014 $年理综浙江卷}
离子推进器是太空飞行器常用的动力系统,某种推进器设计的简化原理如图$ 1 $所示,截面半径为$ R $的圆柱腔分为两个工作区。\lmd{1}为电离区,将氙气电离获得$ 1 $价正离子,\lmd{2}为加速区,长度为$ L $,两端加有电压,形成轴向的匀强电场。\lmd{1}区产生的正离子以接近$ 0 $的初速度进入\lmd{2}区,被加速后以速度$ v_{M} $从右侧喷出。 
\lmd{1}区内有轴向的匀强磁场,磁感应强度大小为$ B $,在离轴线$ R/2 $处的$ C $点持续射出一定速率范围的电子。假设射出的电子仅在垂直于轴线的截面上运动,截面如图$ 2 $所示(从左向右看)。电子的初速度方向与中心$ O $点和$ C $点的连线成$ \alpha $角($ 0 < \alpha \leq 90 $◦)。推进器工作时,向\lmd{1}区注入稀薄的氙气。电子使氙气电离的最小速度为$ v_{0} $,电子在\lmd{1}区内不与器壁相碰且能到达的区域越大,电离效果越好。已知离子质量为$ M $;电子质量为$ m $,电量为$ e $。(电子碰到器壁即被吸收,不考虑电子间的碰撞)。
\begin{enumerate}
\renewcommand{\labelenumi}{\arabic{enumi}.}
% A(\Alph) a(\alph) I(\Roman) i(\roman) 1(\arabic)
%设定全局标号series=example	%引用全局变量resume=example
%[topsep=-0.3em,parsep=-0.3em,itemsep=-0.3em,partopsep=-0.3em]
%可使用leftmargin调整列表环境左边的空白长度 [leftmargin=0em]
\item
求\lmd{2}区的加速电压及离子的加速度大小;
\item 
为取得好的电离效果,请判断求\lmd{1}区的加速电压及离子的加速度大小;区中的磁场方向(按图$ 2 $说明是“垂直纸面向里”或“垂直纸面向外”);
\item 
$ \alpha $为$ 90 ^{\circ} $时,要取得好的电离效果,求射出的电子速率$ v $的范围;
\item 
要取得好的电离效果,求射出的电子最大速率$ v_{max} $与$ \alpha $的关系。



\end{enumerate}
\begin{figure}[h!]
\flushright
\includesvg[width=0.45\linewidth]{picture/svg/235}
\end{figure}

\banswer{
\begin{enumerate}
\renewcommand{\labelenumi}{\arabic{enumi}.}
% A(\Alph) a(\alph) I(\Roman) i(\roman) 1(\arabic)
%设定全局标号series=example	%引用全局变量resume=example
%[topsep=-0.3em,parsep=-0.3em,itemsep=-0.3em,partopsep=-0.3em]
%可使用leftmargin调整列表环境左边的空白长度 [leftmargin=0em]
\item
$a = \frac { e E } { M } = e \frac { U } { M L } = \frac { v _ { M } ^ { 2 } } { 2 L }$
\item 
垂直纸面向外
\item 
$v _ { 0 } \leq v \leq \frac { 3 e B R } { 4 m }$,其中$B > \frac { 4 m v _ { 0 } } { 3 e R }$。
\item 
$v _ { \max } = \frac { 3 e B R } { 4 m ( 2 - \sin \alpha ) }$



\end{enumerate}
}




\newpage
\item
\exwhere{$ 2015 $年江苏卷}
一台质谱仪的工作原理如图所示,电荷量均为$ +q $、质量不同的离子飘入电压为 $ U_{0} $的加速电场,其初速度几乎为零。 这些离子经加速后通过狭缝 $ O $ 沿着与磁场垂直的方向进入磁感应强度为 $ B $ 的匀强磁场,最后打在底片上。 已知放置底片的区域 $ MN $ $ = $ $ L $,且 $ OM $ $ = $ $ L $。某次测量发现 $ MN $ 中左侧$ \frac{ 2 }{ 3 } $区域 $ MQ $ 损坏,检测不到离子,但右侧$ \frac{ 1 }{ 3 } $区域 $ QN $ 仍能正常检测到离子. 在适当调节加速电压后,原本打在 $ MQ $ 的离子即可在 $ QN $ 检测到。
\begin{enumerate}
\renewcommand{\labelenumii}{(\arabic{enumii})}

\item 
求原本打在 $ MN $ 中点$ P $的离子质量 $ m $;

\item 
为使原本打在 $ P $ 的离子能打在 $ QN $ 区域,求加速电压 $ U $ 的调节范围;

\item 
为了在 $ QN $ 区域将原本打在 $ MQ $ 区域的所有离子检测完整,求需要调节 $ U $ 的最少次数 $. $(取 $\lg 2 = 0.301 , \quad \lg 3 = 0.477 , \quad \lg 5 = 0.699$ )

\end{enumerate}
\begin{figure}[h!]
\flushright
\includesvg[width=0.29\linewidth]{picture/svg/236}
\end{figure}


\banswer{
\begin{enumerate}
\renewcommand{\labelenumi}{\arabic{enumi}.}
% A(\Alph) a(\alph) I(\Roman) i(\roman) 1(\arabic)
%设定全局标号series=example	%引用全局变量resume=example
%[topsep=-0.3em,parsep=-0.3em,itemsep=-0.3em,partopsep=-0.3em]
%可使用leftmargin调整列表环境左边的空白长度 [leftmargin=0em]
\item
$m = \frac { 9 q B ^ { 2 } L ^ { 2 } } { 32 U _ { 0 } }$
\item 
$\frac { 100 U _ { 0 } } { 81 } \leq U \leq \frac { 16 U _ { 0 } } { 9 }$
\item 
$ 3 $次



\end{enumerate}
}

\newpage
\item
\exwhere{$ 2015 $年理综山东卷}
如图所示,直径分别为$ D $和$ 2D $的同心圆处于同一竖直面内,$ O $为圆心,$ GH $为大圆的水平直径。两圆之间的环形区域( \lmd{1} 区)和小圆内部( \lmd{2} 区)均存在垂直圆面向里的匀强磁场。间距为$ d $的两平行金属极板间有一匀强电场,上极板开有一小孔。一质量为$ m $电量为$ +q $的粒子由小孔下方$ d/2 $ 处静止释放,加速后粒子以竖直向上的速度$ v $射出电场,由$ H $点紧靠大圆内侧射入磁场。不计粒子的重力。
\begin{enumerate}
\renewcommand{\labelenumi}{\arabic{enumi}.}
% A(\Alph) a(\alph) I(\Roman) i(\roman) 1(\arabic)
%设定全局标号series=example	%引用全局变量resume=example
%[topsep=-0.3em,parsep=-0.3em,itemsep=-0.3em,partopsep=-0.3em]
%可使用leftmargin调整列表环境左边的空白长度 [leftmargin=0em]
\item
求极板间电场强度的大小;
\item 
若粒子运动轨迹与小圆相切,求 \lmd{1} 区磁感应强度的大小;
\item 
若 \lmd{1} 区、 \lmd{2} 区磁感应强度的大小分别为$\frac { 2 m v } { q D }$、$\frac { 4 m v } { q D }$,粒子运动一段时间后再次经过$ H $点,求这段时间粒子运动的路程。



\end{enumerate}
\begin{figure}[h!]
\flushright
\includesvg[width=0.29\linewidth]{picture/svg/237}
\end{figure}


\banswer{
\begin{enumerate}
\renewcommand{\labelenumi}{\arabic{enumi}.}
% A(\Alph) a(\alph) I(\Roman) i(\roman) 1(\arabic)
%设定全局标号series=example	%引用全局变量resume=example
%[topsep=-0.3em,parsep=-0.3em,itemsep=-0.3em,partopsep=-0.3em]
%可使用leftmargin调整列表环境左边的空白长度 [leftmargin=0em]
\item
$\frac { m v ^ { 2 } } { q d }$
\item 
$\frac { 4 m v } { q D }$或$\frac { 4 m v } { 3 q D }$
\item 
$5.5 \pi D$



\end{enumerate}
}





\end{enumerate}



