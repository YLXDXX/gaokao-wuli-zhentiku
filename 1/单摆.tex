\bta{单摆}


\begin{enumerate}
	%\renewcommand{\labelenumi}{\arabic{enumi}.}
	% A(\Alph) a(\alph) I(\Roman) i(\roman) 1(\arabic)
	%设定全局标号series=example	%引用全局变量resume=example
	%[topsep=-0.3em,parsep=-0.3em,itemsep=-0.3em,partopsep=-0.3em]
	%可使用leftmargin调整列表环境左边的空白长度 [leftmargin=0em]
	\item
\exwhere{$ 2013 $ 年上海卷}
如图,在半径为 $ 2.5 \ m $ 的光滑圆环上切下一小段圆弧,放置于竖直平面内,两端点距最低点高度
差 $ H $ 为 $ 1 \ cm $。将小环置于圆弧端点并从静止释放,小环运动到最低点所需的最短时间为 \underlinegap $ s $,在
最低点处的加速度为 \underlinegap 
$ m/s^{2} $。(取 $ g=10 \ m/s^{2} ) $
\begin{figure}[h!]
	\centering
	\includesvg[width=0.23\linewidth]{picture/svg/GZ-3-tiyou-1313}
\end{figure}

 \tk{$ 0.785 $ \quad $ 0.08 $} 

\item 
\exwhere{$ 2014 $ 年理综安徽卷}
在科学研究中,科学家常将未知现象同已知现象进行比较,找出其共同点,进一步推测未知现
象的特性和规律。法国物理学家库仑在研究异种电荷的吸引问题时,曾将扭秤的振动周期与电荷间
距离的关系类比单摆的振动周期与摆球到地心距离的关系。已知单摆摆长为 $ l $,引力常量为 $ G $。地
球的质量为 $ M $。摆球到地心的距离为 $ r $,则单摆振动周期 $ T $ 与距离 $ r $ 的关系式为 \xzanswer{B} 

\fourchoices
{$T=2 \pi r \sqrt{\frac{G M}{l}}$}
{$T=2 \pi r \sqrt{\frac{l}{G M}}$}
{$T=\frac{2 \pi}{r} \sqrt{\frac{G M}{l}}$}
{$T=2 \pi l \sqrt{\frac{l}{G M}}$}



\item 
\exwhere{$ 2011 $年上海卷}
两个相同的单摆静止于平衡位置,使摆球分别以水平初速$ v_{1} $、$ v_{2} $($ v_{1} > v_{2} $)在竖直平面内做小角度
摆动,它们的频率与振幅分别为$ f_{1} $,$ f_{2} $和$ A_{1} $,$ A_{2} $,则 \xzanswer{C} 

\fourchoices
{$ f_{1} > f_{2} $,$ A_{1} = A_{2} $}
{$ f_{1} < f_{2} $,$ A_{1} = A_{2} $}
{$ f_{1} = f_{2} $,$ A_{1} > A_{2} $}
{$ f_{1} = f_{2} $,$ A_{1} < A_{2} $}



\item 
\exwhere{$ 2012 $ 年理综全国卷}
一单摆在地面处的摆动周期与在某矿井底部摆动周期的比值为 $ k $。设地球的半径为 $ R $。假定地球的
密度均匀。已知质量均匀分布的球壳对壳内物体的引力为零,求矿井的深度 $ d $。

\banswer{
	$d=R-k^{2} R$
}





	
	
	
\end{enumerate}

