\bta{振动图象和波动图像}


\begin{enumerate}
	%\renewcommand{\labelenumi}{\arabic{enumi}.}
	% A(\Alph) a(\alph) I(\Roman) i(\roman) 1(\arabic)
	%设定全局标号series=example	%引用全局变量resume=example
	%[topsep=-0.3em,parsep=-0.3em,itemsep=-0.3em,partopsep=-0.3em]
	%可使用leftmargin调整列表环境左边的空白长度 [leftmargin=0em]
	\item
\exwhere{$ 2019 $ 年 $ 4 $ 月浙江物理选考}
图 \subref{2019浙江振动a} 为一列简请横波在 $ t=0 $ 时刻的波形图,$ P $、$ Q $ 为介质中的两个
质点,图 \subref{2019浙江振动b} 为质点 $ P $ 的振动图象,则 \xzanswer{CD} 

\begin{figure}[h!]
	\centering
\begin{subfigure}{0.4\linewidth}
	\centering
	\includesvg[width=0.7\linewidth]{picture/svg/GZ-3-tiyou-1360} 
	\caption{}\label{2019浙江振动a}
\end{subfigure}
\begin{subfigure}{0.4\linewidth}
	\centering
	\includesvg[width=0.7\linewidth]{picture/svg/GZ-3-tiyou-1361} 
	\caption{}\label{2019浙江振动b}
\end{subfigure}
\end{figure}

\fourchoices
{$ t=0.2 \ s $ 时,质点 $ Q $ 沿 $ y $ 轴负方向运动}
{$ 0 \sim 0.3 \ s $ 内,质点 $ Q $ 运动的路程为 $ 0.3 \ m $}
{$ t=0.5 \ s $ 时,质点 $ Q $ 的加速度小于质点 $ P $ 的加速度}
{$ t=0.7 \ s $ 时,质点 $ Q $ 距平衡位置的距离小于质点 $ P $ 距平衡位置的距离}



\item 
\exwhere{$ 2013 $ 年四川卷}
图 \subref{2013四川05a} 是一列简谐横波在 $ t=1.25 \ s $ 时的波形图,已知 $ c $ 位置的质点比 $ a $ 位置的晚 $ 0.5 \ s $ 起振。则图 \subref{2013四川05b}
所示振动图像对应的质点可能位于 \xzanswer{D} 
\begin{figure}[h!]
	\centering
\begin{subfigure}{0.4\linewidth}
	\centering
	\includesvg[width=0.7\linewidth]{picture/svg/GZ-3-tiyou-1362} 
	\caption{}\label{2013四川05a}
\end{subfigure}
\begin{subfigure}{0.4\linewidth}
	\centering
	\includesvg[width=0.7\linewidth]{picture/svg/GZ-3-tiyou-1363} 
	\caption{}\label{2013四川05b}
\end{subfigure}
\end{figure}

\fourchoices
{$ a<x<b $}
{$ b<x<c $}
{$ c<x<d $}
{$ d<x<e $}


\item 
\exwhere{$ 2014 $ 年理综安徽卷}
一简谐横波沿 $ x $ 轴正向传播,图 \subref{2014安徽16a} 是 $ t=0 $ 时刻的波形图,图 \subref{2014安徽16b} 是介质中某点的振动图象,则该
质点的 $ x $ 坐标值合理的是 \xzanswer{C} 
\begin{figure}[h!]
	\centering
\begin{subfigure}{0.4\linewidth}
	\centering
	\includesvg[width=0.7\linewidth]{picture/svg/GZ-3-tiyou-1364} 
	\caption{}\label{2014安徽16a}
\end{subfigure}
\begin{subfigure}{0.4\linewidth}
	\centering
	\includesvg[width=0.7\linewidth]{picture/svg/GZ-3-tiyou-1365} 
	\caption{}\label{2014安徽16b}
\end{subfigure}
\end{figure}

\fourchoices
{$ 0.5 \ m $}
{$ 1.5 \ m $}
{$ 2.5 \ m $}
{$ 3.5 \ m $}


\item 
\exwhere{$ 2014 $ 年理综北京卷}
一简谐机械横波沿 $ x $ 轴正方向传播,波长为 $ \lambda $,周期为 $ T $,$ t=0 $ 时刻的波形如图 \subref{2014北京17a} 所示,$ a $、$ b $ 是
波上的两个质点。图 \subref{2014北京17b} 是波上某一质点的振动图像。下列说法正确的是 \xzanswer{D} 
\begin{figure}[h!]
	\centering
\begin{subfigure}{0.4\linewidth}
	\centering
	\includesvg[width=0.7\linewidth]{picture/svg/GZ-3-tiyou-1366} 
	\caption{}\label{2014北京17a}
\end{subfigure}
\begin{subfigure}{0.4\linewidth}
	\centering
	\includesvg[width=0.7\linewidth]{picture/svg/GZ-3-tiyou-1367} 
	\caption{}\label{2014北京17b}
\end{subfigure}
\end{figure}

\fourchoices
{$ t=0 $ 时质点 $ a $ 的速度比质点 $ b $ 的大}
{$ t=0 $ 时质点 $ a $ 的加速度比质点 $ b $ 的小}
{图 $ 2 $ 可以表示质点 $ a $ 的振动}
{图 $ 2 $ 可以表示质点 $ b $ 的振动}

\item 
\exwhere{$ 2012 $ 年理综福建卷}
一列简谐波沿 $ x $ 轴传播,$ t=0 $ 时刻的波形如图 \subref{2012福建13a} 所示,此时质点 $ P $ 正沿 $ y $ 轴负方向运动,其振动
图像如图 \subref{2012福建13b} 所示,则该波的传播方向和波速分别是 \xzanswer{A} 
\begin{figure}[h!]
	\centering
\begin{subfigure}{0.4\linewidth}
	\centering
	\includesvg[width=0.7\linewidth]{picture/svg/GZ-3-tiyou-1368} 
	\caption{}\label{2012福建13a}
\end{subfigure}
\begin{subfigure}{0.4\linewidth}
	\centering
	\includesvg[width=0.7\linewidth]{picture/svg/GZ-3-tiyou-1369} 
	\caption{}\label{2012福建13b}
\end{subfigure}
\end{figure}


\fourchoices
{沿 $ x $ 轴负方向,$ 60 \ m /s $}
{沿 $ x $ 轴正方向,$ 60 \ m /s $}
{沿 $ x $ 轴负方向,$ 30 \ m /s $}
{沿 $ x $ 轴正方向,$ 30 \ m /s $}


\item 
\exwhere{$ 2012 $ 年理综全国卷}
一列简谐横波沿 $ x $ 轴正方向传播,图 \subref{2012全国20a} 是 $ t=0 $ 时刻的波形图,图 \subref{2012全国20c} 和图 \subref{2012全国20c} 分别是 $ x $ 轴
上某两处质点的振动图像。由此可知,这两质点平衡位置之间的距离可能是 \xzanswer{BD} 
\begin{figure}[h!]
	\centering
\begin{subfigure}{0.4\linewidth}
	\centering
	\includesvg[width=0.7\linewidth]{picture/svg/GZ-3-tiyou-1370} 
	\caption{}\label{2012全国20a}
\end{subfigure}
\begin{subfigure}{0.4\linewidth}
	\centering
	\includesvg[width=0.7\linewidth]{picture/svg/GZ-3-tiyou-1371} 
	\caption{}\label{2012全国20b}
\end{subfigure}
\begin{subfigure}{0.4\linewidth}
	\centering
	\includesvg[width=0.7\linewidth]{picture/svg/GZ-3-tiyou-1372} 
	\caption{}\label{2012全国20c}
\end{subfigure}
\end{figure}





\fourchoices
{$\frac{1}{3} \ m$}
{$\frac{2}{3} \ m$}
{$ 1 \ m $}
{$\frac{4}{3} \ m$}





\item 
\exwhere{$ 2016 $ 年四川卷}
简谐横波在均匀介质中沿直线传播,$ P $、$ Q $ 是传播方向上相距 $ 10 \ m $ 的两质点,
波先传到 $ P $,当波传到 $ Q $ 开始计时,$ P $、$ Q $ 两质点的振动图像如图所示。则 \xzanswer{AD} 
\begin{figure}[h!]
	\centering
	\includesvg[width=0.23\linewidth]{picture/svg/GZ-3-tiyou-1373}
\end{figure}


\fourchoices
{质点 $ Q $ 开始振动的方向沿 $ y $ 轴正方向}
{该波从 $ P $ 传到 $ Q $ 的时间可能为 $ 7 \ s $}
{该波的传播速度可能为 $ 2 \ m /s $}
{该波的波长可能为 $ 6 \ m $}



\item 
\exwhere{$ 2011 $ 年理综重庆卷}
介质中坐标原点 $ O $ 处的波源在 $ t=0 $ 时刻开始振动,产生的
简谐波沿 $ x $ 轴正向传播,$ t_{0} $ 时刻传到 $ L $ 处,波形如图所示。下列
能描述 $ x_{0} $ 处质点振动的图象是 \xzanswer{C} 
\begin{figure}[h!]
	\centering
	\includesvg[width=0.23\linewidth]{picture/svg/GZ-3-tiyou-1374}
\end{figure}
\pfourchoices
{\includesvg[width=4.3cm]{picture/svg/GZ-3-tiyou-1375}}
{\includesvg[width=4.3cm]{picture/svg/GZ-3-tiyou-1376}}
{\includesvg[width=4.3cm]{picture/svg/GZ-3-tiyou-1377}}
{\includesvg[width=4.3cm]{picture/svg/GZ-3-tiyou-1378}}

\item 
\exwhere{$ 2014 $ 年理综四川卷}
如图所示, \subref{2014四川05a} 为 $ t=1 \ s $ 时某横波的波形图像, \subref{2014四川05b} 为该波传播方向上某一质点的振动图像,距该
质点$ \triangle x=0.5 \ m $ 处质点的振动图像可能是 \xzanswer{A} 
\begin{figure}[h!]
	\centering
\begin{subfigure}{0.4\linewidth}
	\centering
	\includesvg[width=0.7\linewidth]{picture/svg/GZ-3-tiyou-1379} 
	\caption{}\label{2014四川05a}
\end{subfigure}
\begin{subfigure}{0.4\linewidth}
	\centering
	\includesvg[width=0.7\linewidth]{picture/svg/GZ-3-tiyou-1380} 
	\caption{}\label{2014四川05b}
\end{subfigure}
\end{figure}

\pfourchoices
{\includesvg[width=4.3cm]{picture/svg/GZ-3-tiyou-1381}}
{\includesvg[width=4.3cm]{picture/svg/GZ-3-tiyou-1382}}
{\includesvg[width=4.3cm]{picture/svg/GZ-3-tiyou-1383}}
{\includesvg[width=4.3cm]{picture/svg/GZ-3-tiyou-1384}}


\item 
\exwhere{$ 2014 $ 年理综福建卷}
在均匀介质中,一列沿 $ x $ 轴正向传播的横波,其波源 $ O $ 在第
一个周期内的振动图像,如右图所示,则该波在第一个周期末的
波形图是 \xzanswer{D} 
\begin{figure}[h!]
	\centering
	\includesvg[width=0.23\linewidth]{picture/svg/GZ-3-tiyou-1385}
\end{figure}

\pfourchoices
{\includesvg[width=4.3cm]{picture/svg/GZ-3-tiyou-1386}}
{\includesvg[width=4.3cm]{picture/svg/GZ-3-tiyou-1387}}
{\includesvg[width=4.3cm]{picture/svg/GZ-3-tiyou-1388}}
{\includesvg[width=4.3cm]{picture/svg/GZ-3-tiyou-1389}}


\item 
\exwhere{$ 2015 $ 年理综天津卷}
图 \subref{2015天津11a} 为一列简谐横波在某一时刻的波形图,$ a $、$ b $ 两质点的横坐标分别
为 $ x_a=2 \ m $ 和 $ x_b=6 \ m $,图 \subref{2015天津11b} 为质点 $ b $ 从该
时刻开始计时的振动图象,下列说法正
确的是 \xzanswer{D} 
\begin{figure}[h!]
	\centering
\begin{subfigure}{0.4\linewidth}
	\centering
	\includesvg[width=0.7\linewidth]{picture/svg/GZ-3-tiyou-1390} 
	\caption{}\label{2015天津11a}
\end{subfigure}
\begin{subfigure}{0.4\linewidth}
	\centering
	\includesvg[width=0.7\linewidth]{picture/svg/GZ-3-tiyou-1391} 
	\caption{}\label{2015天津11b}
\end{subfigure}
\end{figure}


\fourchoices
{该波沿$ +x $ 方向传播,波速为 $ 1 \ m /s $}
{质点 $ a $ 经 $ 4 \ s $ 振动的路程为 $ 4 \ m $}
{此时刻质点 $ a $ 的速度沿$ +y $ 方向}
{质点 $ a $ 在 $ t=2 \ s $ 时速度为零}


	
	
\end{enumerate}

