\bta{波形图中的多解问题}


\begin{enumerate}
	%\renewcommand{\labelenumi}{\arabic{enumi}.}
	% A(\Alph) a(\alph) I(\Roman) i(\roman) 1(\arabic)
	%设定全局标号series=example	%引用全局变量resume=example
	%[topsep=-0.3em,parsep=-0.3em,itemsep=-0.3em,partopsep=-0.3em]
	%可使用leftmargin调整列表环境左边的空白长度 [leftmargin=0em]
	\item
\exwhere{$ 2018 $ 年北京卷}
如图所示,一列简谐横波向右传播,$ P $、$ Q $ 两质点平衡位置相距 $ 0.15 \ m $。当
$ P $ 运动到上方最大位移处时,$ Q $ 刚好运动到下方最大位移处,
则这列波的波长可能是 \xzanswer{B} 
\begin{figure}[h!]
	\centering
	\includesvg[width=0.23\linewidth]{picture/svg/GZ-3-tiyou-1392}
\end{figure}


\fourchoices
{$ 0.60 \ m $}
{$ 0.30 \ m $}
{$ 0.20 \ m $}
{$ 0.15 \ m $}


\item 
\exwhere{$ 2013 $ 年天津卷}
一列简谐横波沿直线传播,该直线上平衡位置相距 $ 9 \ m $ 的
$ a $、$ b $ 两质点的振动图象如右图所示。下列描述该波的图象可
能正确的是 \xzanswer{AC} 
\begin{figure}[h!]
	\centering
	\includesvg[width=0.23\linewidth]{picture/svg/GZ-3-tiyou-1393}
\end{figure}

\pfourchoices
{\includesvg[width=4.3cm]{picture/svg/GZ-3-tiyou-1394}}
{\includesvg[width=4.3cm]{picture/svg/GZ-3-tiyou-1395}}
{\includesvg[width=4.3cm]{picture/svg/GZ-3-tiyou-1396}}
{\includesvg[width=4.3cm]{picture/svg/GZ-3-tiyou-1397}}


\item 
\exwhere{$ 2012 $ 年理综四川卷}
在 $ xOy $ 平面内有一列沿 $ x $ 轴正方向传播的简谐横波,波速为 $ 2 \ m /s $,振幅为 $ A $。$ M $、$ N $ 是平衡位
置相距 $ 2 \ m $ 的两个质点,如图所示。在 $ t=0 $ 时,$ M $ 通过其平衡位置沿 $ y $ 轴正方向运动,$ N $ 位于其平
衡位置上方最大位移处。已知该波的周期大于 $ 1 \ s $。则 \xzanswer{D} 
\begin{figure}[h!]
	\centering
	\includesvg[width=0.23\linewidth]{picture/svg/GZ-3-tiyou-1398}
\end{figure}

\fourchoices
{该波的周期为$ \frac{ 5 }{ 3 } \ s $}
{在 $ t= \frac{ 1 }{ 3 } \ s $ 时,$ N $ 的速度一定为 $ 2 \ m /s $}
{从 $ t=0 $ 到 $ t=1 \ s $,$ M $ 向右移动了 $ 2 \ m $}
{从 $ t= \frac{ 1 }{ 3 } \ s $ 到 $ t= \frac{ 2 }{ 3 } \ s $,$ M $ 的动能逐渐增大}

\item 
\exwhere{$ 2012 $ 年物理上海卷}
如图,简谐横波在 $ t $ 时刻的波形如实线所示,经过
$ \Delta t=3 \ s $,其波形如虚线所示。己知图中 $ x_{1} $ 与 $ x_{2} $ 相距 $ 1 \ m $,
波的周期为 $ T $,且 $ 2  T < \Delta t<4 T $。则可能的最小波速为
 \underlinegap 
$ m/s $,最小周期为 \underlinegap $ s $。
\begin{figure}[h!]
	\centering
	\includesvg[width=0.23\linewidth]{picture/svg/GZ-3-tiyou-1399}
\end{figure}

 \tk{$ 5 $ \quad $ 7/9 $} 

\item 
\exwhere{$ 2015 $ 年理综福建卷}
简谐横波在同一均匀介质中沿 $ x $ 轴正方向传播,波速为 $ v $。若某时刻在波
的传播方向上,位于平衡位置的两质点 $ a $、$ b $ 相距为 $ s $,$ a $、$ b $ 之间只存在一个波谷,则从该时刻起,
下列四幅波形图中质点 $ a $ 最早到达波谷的是 \xzanswer{D} 
\pfourchoices
{\includesvg[width=4.3cm]{picture/svg/GZ-3-tiyou-1400}}
{\includesvg[width=4.3cm]{picture/svg/GZ-3-tiyou-1401}}
{\includesvg[width=4.3cm]{picture/svg/GZ-3-tiyou-1402}}
{\includesvg[width=4.3cm]{picture/svg/GZ-3-tiyou-1403}}



	
	
	
\end{enumerate}

