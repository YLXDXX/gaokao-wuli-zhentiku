
\bta{电磁感应中的电路问题}


\begin{enumerate}
%\renewcommand{\labelenumi}{\arabic{enumi}.}
% A(\Alph) a(\alph) I(\Roman) i(\roman) 1(\arabic)
%设定全局标号series=example	%引用全局变量resume=example
%[topsep=-0.3em,parsep=-0.3em,itemsep=-0.3em,partopsep=-0.3em]
%可使用leftmargin调整列表环境左边的空白长度 [leftmargin=0em]
\item
\exwhere{$ 2016 $ 年新课标$ \l m_{d} {2} $卷}
如图,水平面(纸面)内间距为 $ l $
的平行金属导轨间接一电阻,质量为 $ m $、长度为 $ l $ 的金属杆置于导轨上。
$ t=0 $ 时,金属杆在水平向右、大小为 $ F $ 的恒定拉力作用下由静止开始运
动,$ t_{0} $ 时刻,金属杆进入磁感应强度大小为 $ B $、方向垂直于纸面向里的
匀强磁场区域,且在磁场中恰好能保持匀速运动。杆与导轨的电阻均忽略不计,两者始终保持垂直
且接触良好,两者之间的动摩擦因数为$ \mu $。重力加速度大小为 $ g $。求:
\begin{enumerate}
%\renewcommand{\labelenumi}{\arabic{enumi}.}
% A(\Alph) a(\alph) I(\Roman) i(\roman) 1(\arabic)
%设定全局标号series=example	%引用全局变量resume=example
%[topsep=-0.3em,parsep=-0.3em,itemsep=-0.3em,partopsep=-0.3em]
%可使用leftmargin调整列表环境左边的空白长度 [leftmargin=0em]
\item
金属杆在磁场中运动时产生的电动势的大小;
\item 
电阻的阻值。
\end{enumerate}
\begin{figure}[h!]
\flushright
\includesvg[width=0.25\linewidth]{picture/svg/GZ-3-tiyou-0830}
\end{figure}

\banswer{
\begin{enumerate}
%\renewcommand{\labelenumi}{\arabic{enumi}.}
% A(\Alph) a(\alph) I(\Roman) i(\roman) 1(\arabic)
%设定全局标号series=example	%引用全局变量resume=example
%[topsep=-0.3em,parsep=-0.3em,itemsep=-0.3em,partopsep=-0.3em]
%可使用leftmargin调整列表环境左边的空白长度 [leftmargin=0em]
\item
$\frac{B l t_{0}(F-\mu m g)}{m}$
\item 
$\frac{B^{2} l^{2} t_{0}}{m}$
\end{enumerate}
}



\item 
\exwhere{$ 2015 $ 年理综北京卷}
如图所示,足够长的平行光滑金属导轨水平放置,宽度 $ L=0.4 \ m $,
一端连接 $ R=1 \ \Omega $的电阻。导轨所在空间存在竖直向下的匀强磁场,磁感应强度 $ B=1 \ T $。导体棒 $ MN $ 放
在导轨上,其长度恰好等于导轨间距,与导轨接触良好。导轨和导体棒的电阻均可忽略不计。在平
行于导轨的拉力 $ F $ 作用下,导体棒沿导轨向右匀速运动,速度 $ v=5 \ m /s $。
求:
\begin{enumerate}
%\renewcommand{\labelenumi}{\arabic{enumi}.}
% A(\Alph) a(\alph) I(\Roman) i(\roman) 1(\arabic)
%设定全局标号series=example	%引用全局变量resume=example
%[topsep=-0.3em,parsep=-0.3em,itemsep=-0.3em,partopsep=-0.3em]
%可使用leftmargin调整列表环境左边的空白长度 [leftmargin=0em]
\item
感应电动势 $ E $ 和感应电流 $ I $;
\item 
在 $ 0.1 \ s $ 时间内,拉力的冲量 $ I_{F} $ 的大小;



\item 
若将 $ MN $ 换为电阻 $ r=1 \ \Omega $的导体棒,其它条件不变,求导体棒两端的电压 $ U $。

\end{enumerate}
\begin{figure}[h!]
\flushright
\includesvg[width=0.25\linewidth]{picture/svg/GZ-3-tiyou-0837}
\end{figure}

\banswer{
\begin{enumerate}
%\renewcommand{\labelenumi}{\arabic{enumi}.}
% A(\Alph) a(\alph) I(\Roman) i(\roman) 1(\arabic)
%设定全局标号series=example	%引用全局变量resume=example
%[topsep=-0.3em,parsep=-0.3em,itemsep=-0.3em,partopsep=-0.3em]
%可使用leftmargin调整列表环境左边的空白长度 [leftmargin=0em]
\item
$ E=2 \ V \quad I=2 \ A $
\item 
$I_{F}=0.08 N \cdot s$
\item 
$ U=1 \ V $
\end{enumerate}
}


\item 
\exwhere{$ 2015 $ 年理综福建卷}
$ 18 $. 如图,由某种粗细均匀的总电阻为 $ 3R $ 的金属条制成的矩形线框 $ abcd $,固
定在水平面内且处于方向竖直向下的匀强磁场 $ B $ 中。一接入电路电阻为 $ R $ 的导体棒 $ PQ $,在水平拉
力作用下沿 $ ab $、$ dc $ 以速度 $ v $ 匀速滑动,滑动过程 $ PQ $ 始终与 $ ab $ 垂直,且与线框接触良好,不计摩
擦。在 $ PQ $ 从靠近 $ ad $ 处向 $ bc $ 滑动的过程中 \xzanswer{C} 
\begin{figure}[h!]
\centering
\includesvg[width=0.23\linewidth]{picture/svg/GZ-3-tiyou-0832}
\end{figure}


\fourchoices
{$ PQ $ 中电流先增大后减小}
{$ PQ $ 两端电压先减小后增大}
{$ PQ $ 上拉力的功率先减小后增大}
{线框消耗的电功率先减小后增大}






\item
\exwhere{$ 2013 $ 年四川卷}
如图所示,边长为 $ L $、不可形变的正方形导体框内有半径为 $ r $ 的圆形磁场区域,
其磁感应强度 $ B $ 随时间 $ t $ 的变化关系为 $ B=kt $(常量 $ k > 0 $)
。回路中滑动变阻器 $ R $ 的最大阻值为 $ R_{0} $,
滑动片 $ P $ 位于滑动变阻器中央,定值电阻 $ R_{1} = R_{0} $、$ R_{2} = R_{0} /2 $。闭合开关 $ S $,电压表的示数为 $ U $,不考
虑虚线 $ MN $ 右侧导体的感应电动势。则 \xzanswer{AC} 
\begin{figure}[h!]
\centering
\includesvg[width=0.23\linewidth]{picture/svg/GZ-3-tiyou-0833}
\end{figure}



\fourchoices
{$ R_{2} $ 两端的电压为 $ U/7 $}
{电容器的 $ a $ 极板带正电}
{滑动变阻器 $ R $ 的热功率为电阻 $ R_{2} $ 的 $ 5 $ 倍}
{正方形导线框中的感应电动势为 $ k L^{2} $}



\item
\exwhere{$ 2013 $ 年海南卷}
如图,水平桌面上固定有一半径为 $ R $ 的金属细圆环,环面水平,圆环每单位长度的电阻为 $ r $,空
间有一匀强磁场,磁感应强度大小为 $ B $,方向竖直向下;一长度为 $ 2R $、电阻可忽略的导体棒置于圆
环左侧并与环相切,切点为棒的中点。棒在拉力的作用下以恒定加速度 $ a $ 从静止开始向右运动,运
动过程中棒与圆环接触良好。下列说法正确的是 \xzanswer{D} 
\begin{figure}[h!]
\centering
\includesvg[width=0.23\linewidth]{picture/svg/GZ-3-tiyou-0834}
\end{figure}

\fourchoices
{拉力的大小在运动过程中保持不变}
{棒通过整个圆环所用的时间为 $\sqrt{2 R / a}$}
{棒经过环心时流过棒的电流为 $B \sqrt{2 a R} / \pi r$}
{棒经过环心时所受安培力的大小为 $8 B^{2} R \sqrt{2 a R} / \pi r$}



\item 
\exwhere{$ 2011 $ 年理综全国卷}
如图所示,两根足够长的金属导轨 $ ab $、$ cd $ 竖直放置,导轨
间距离为 $ L $,电阻不计。在导轨上端并接两个额定功率均为 $ P $、电阻均
为 $ R $ 的小灯泡。整个系统置于匀强磁场中,磁感应强度方向与导轨所在
平面垂直。现将一质量为 $ m $、电阻可以忽略的金属棒 $ MN $ 从图示位置由
静止开始释放。金属棒下落过程中保持水平,且与导轨接触良好。已知
某时刻后两灯泡保持正常发光。重力加速度为 $ g $。求:
\begin{enumerate}
%\renewcommand{\labelenumi}{\arabic{enumi}.}
% A(\Alph) a(\alph) I(\Roman) i(\roman) 1(\arabic)
%设定全局标号series=example	%引用全局变量resume=example
%[topsep=-0.3em,parsep=-0.3em,itemsep=-0.3em,partopsep=-0.3em]
%可使用leftmargin调整列表环境左边的空白长度 [leftmargin=0em]
\item
磁感应强度的大小;
\item 
灯泡正常发光时导体棒的运动速率。


\end{enumerate}
\begin{figure}[h!]
\flushright
\includesvg[width=0.25\linewidth]{picture/svg/GZ-3-tiyou-0838}
\end{figure}


\banswer{
\begin{enumerate}
%\renewcommand{\labelenumi}{\arabic{enumi}.}
% A(\Alph) a(\alph) I(\Roman) i(\roman) 1(\arabic)
%设定全局标号series=example	%引用全局变量resume=example
%[topsep=-0.3em,parsep=-0.3em,itemsep=-0.3em,partopsep=-0.3em]
%可使用leftmargin调整列表环境左边的空白长度 [leftmargin=0em]
\item
$B=\frac{m g}{2 L} \sqrt{\frac{R}{P}}$
\item 
$v=\frac{2 P}{m g}$
\end{enumerate}
}



\item 
\exwhere{$ 2012 $ 年理综广东卷}
如图所示,质量为 $ M $ 的导体棒 $ ab $,垂直放
在相距为 $ l $ 的平行光滑金属轨道上。导轨平面与水
平面的夹角为$ \theta $,并处于磁感应强度大小为 $ B $、方
向垂直与导轨平面向上的匀强磁场中,左侧是水平
放置、间距为 $ d $ 的平行金属板,$ R $ 和 $ R_{x} $ 分别表示定
值电阻和滑动变阻器的阻值,不计其他电阻。
\begin{enumerate}
%\renewcommand{\labelenumi}{\arabic{enumi}.}
% A(\Alph) a(\alph) I(\Roman) i(\roman) 1(\arabic)
%设定全局标号series=example	%引用全局变量resume=example
%[topsep=-0.3em,parsep=-0.3em,itemsep=-0.3em,partopsep=-0.3em]
%可使用leftmargin调整列表环境左边的空白长度 [leftmargin=0em]
\item
调节 $ R_{x} =R $,释放导体棒,当棒沿导轨匀速下滑时,求通过棒的电流 $ I $ 及棒的速率 $ v $。
\item 
改变 $ R_{x} $,待棒沿导轨再次匀速下滑后,将质量为 $ m $、带电量为$ +q $ 的微粒水平射入金属板间,
若它能匀速通过,求此时的 $ R_{x} $。


\end{enumerate}
\begin{figure}[h!]
\flushright
\includesvg[width=0.25\linewidth]{picture/svg/GZ-3-tiyou-0836}
\end{figure}

\banswer{
\begin{enumerate}
%\renewcommand{\labelenumi}{\arabic{enumi}.}
% A(\Alph) a(\alph) I(\Roman) i(\roman) 1(\arabic)
%设定全局标号series=example	%引用全局变量resume=example
%[topsep=-0.3em,parsep=-0.3em,itemsep=-0.3em,partopsep=-0.3em]
%可使用leftmargin调整列表环境左边的空白长度 [leftmargin=0em]
\item
$I=\frac{M g \sin \theta}{B l}$ \quad $v=\frac{2 M g R \sin \theta}{B^{2} l^{2}}$
\item 
$R_{x}=\frac{m l d B}{M q \sin \theta}$
\end{enumerate}	
}






\item
\exwhere{$ 2012 $ 年理综浙江卷}
为了提高自行车夜间行驶的安全性,小明同学设计了一种“闪烁”装置。如图所示,自
行车后轮由半径 $ r_{1} =5.0 \times 10^{-2} \ m $ 的金属内圈、半径 $ r_{2} =0.40 \ m $ 的金属外圈和绝缘辐条构成。后轮的内、
外圈之间等间隔地接有 $ 4 $ 根金属条,每根金属条的中间均串联有一电阻值为 $ R $ 的小灯泡。在支架上
装有磁铁,形成了磁感应强度 $ B=0.10 \ T $、方向垂直纸面向外的“扇形”
匀强磁场,其内半径为 $ r_{1} $,外半径为 $ r_{2} $、张角$ \theta =\frac{\pi}{6} $,后轮以角速度$ \omega =2 \pi \ rad/s $ 相对于转轴转动。若不计其它电阻,忽略磁场的边缘效应。
\begin{enumerate}
%\renewcommand{\labelenumi}{\arabic{enumi}.}
% A(\Alph) a(\alph) I(\Roman) i(\roman) 1(\arabic)
%设定全局标号series=example	%引用全局变量resume=example
%[topsep=-0.3em,parsep=-0.3em,itemsep=-0.3em,partopsep=-0.3em]
%可使用leftmargin调整列表环境左边的空白长度 [leftmargin=0em]
\item
当金属条 $ ab $ 进入“扇形”磁场时,求感应电动势 $ E $,并指出 $ ab $ 上的
电流方向;

\item 
当金属条 $ ab $ 进入“扇形”磁场时,画出“闪烁”装置的电路图;

\item 
从金属条 $ ab $ 进入“扇形”磁场时开始,经计算画出轮子转一圈过程
中,内圈与外圈之间电势差 $ U_{ab} $ 随时间 $ t $ 变化的 $ U_{ab}-t $ 图象;

\item 
若选择的是“$ 1.5 \ V $、$ 0.3 \ A $”的小灯泡,该“闪烁”装置能否正常工作?有同学提出,通过改变磁感应
强度 $ B $、后轮外圈半径 $ r_{2} $、角速度$ \omega $ 和张角$ \theta $等物理量的大小,优化前同学的设计方案,请给出你
的评价。

\end{enumerate}
\begin{figure}[h!]
\flushright
\includesvg[width=0.25\linewidth]{picture/svg/GZ-3-tiyou-0839}
\end{figure}

\banswer{
\begin{enumerate}
%\renewcommand{\labelenumi}{\arabic{enumi}.}
% A(\Alph) a(\alph) I(\Roman) i(\roman) 1(\arabic)
%设定全局标号series=example	%引用全局变量resume=example
%[topsep=-0.3em,parsep=-0.3em,itemsep=-0.3em,partopsep=-0.3em]
%可使用leftmargin调整列表环境左边的空白长度 [leftmargin=0em]
\item
$E=4.9 \times 10^{-2} \ V$\\
电流方向由$ b $到$ a $。
\item 	
将$ ab $条可看做电源,并且有内阻,其它三条看做外电路,如图所示
\begin{center}
 \includesvg[width=0.23\linewidth]{picture/svg/GZ-3-tiyou-0840} 
\end{center}
\item 	
如图所示
\begin{center}
 \includesvg[width=0.23\linewidth]{picture/svg/GZ-3-tiyou-0842} 
\end{center}
\item 
小灯泡不能正常工作,因为感应电动势为 $E=4.9 \times 10^{-2} V$ 远小于灯泡的额定电压,因此闪乐装置不可
能工作。\\
$B$ 增大, $E$ 增大,但有限度; $r$ 增大, $E$ 增大,但有限度;
$\omega$ 增大, $E$ 增大,但有限度; $\theta$ 增大, $E$ 不增大。
\end{enumerate}
}



\item
\exwhere{$ 2015 $ 年广东卷}
如图 $ (a) $所示,平行长直金属导轨水平放置,间距 $ L=0.4 \ m $,导轨右
端接有阻值 $ R=1 \ \Omega $的电阻,导体棒垂直放置在导轨上,且接触良好,导体棒及导轨的电阻均不计,
导轨间正方形区域 $ abcd $ 内有方向竖直向下的匀强磁场,$ bd $ 连线与导轨垂直,长度也为 $ L $,从 $ 0 $ 时
刻开始,磁感应强度 $ B $ 的大小随时间 $ t $ 变化,规律如图 $ (b) $所示;同一时刻,棒从导轨左端开始向
右匀速运动,$ 1 \ s $ 后刚
好进入磁场,若使棒在
导轨上始终以速度 $ v $
$ =1 \ m /s $ 做直线运动,
求:
\begin{enumerate}
%\renewcommand{\labelenumi}{\arabic{enumi}.}
% A(\Alph) a(\alph) I(\Roman) i(\roman) 1(\arabic)
%设定全局标号series=example	%引用全局变量resume=example
%[topsep=-0.3em,parsep=-0.3em,itemsep=-0.3em,partopsep=-0.3em]
%可使用leftmargin调整列表环境左边的空白长度 [leftmargin=0em]
\item
棒进入磁场前,回
路中的电动势 $ E $;




\item 
棒在运动过程中受到的最大安培力 $ F $,以及棒通过三角形 $ abd $ 区域时电流 $ i $ 与时间 $ t $ 的关系式。


\end{enumerate}
\begin{figure}[h!]
\centering
\begin{subfigure}{0.4\linewidth}
\centering
\includesvg[width=0.7\linewidth]{picture/svg/GZ-3-tiyou-0843} 
\caption{}\label{}
\end{subfigure}
\begin{subfigure}{0.4\linewidth}
\centering
\includesvg[width=0.7\linewidth]{picture/svg/GZ-3-tiyou-0844} 
\caption{}\label{}
\end{subfigure}
\end{figure}


\banswer{
\begin{enumerate}
%\renewcommand{\labelenumi}{\arabic{enumi}.}
% A(\Alph) a(\alph) I(\Roman) i(\roman) 1(\arabic)
%设定全局标号series=example	%引用全局变量resume=example
%[topsep=-0.3em,parsep=-0.3em,itemsep=-0.3em,partopsep=-0.3em]
%可使用leftmargin调整列表环境左边的空白长度 [leftmargin=0em]
\item
$ E=0.04 \ V $
\item 
$ F_{m}=0.04 \ N $,$ i=(t-1) \ A $(其中,$ 1 \ s \leq t \leq 1.2 \ s ) $
\end{enumerate}	
}





\item
\exwhere{$ 2019 $ 年物理北京卷}
如图所示,垂直于纸面的匀强磁场磁感应强度为 $ B $。纸面内有一正方形
均匀金属线框 $ abcd $,其边长为 $ L $,总电阻为 $ R $,$ ad $ 边与磁场边界平行。从 $ ad $ 边刚进入磁场直至 $ bc $
边刚要进入的过程中,线框在向左的拉力作用下以速度 $ v $ 匀速运动,求:
\begin{enumerate}
%\renewcommand{\labelenumi}{\arabic{enumi}.}
% A(\Alph) a(\alph) I(\Roman) i(\roman) 1(\arabic)
%设定全局标号series=example	%引用全局变量resume=example
%[topsep=-0.3em,parsep=-0.3em,itemsep=-0.3em,partopsep=-0.3em]
%可使用leftmargin调整列表环境左边的空白长度 [leftmargin=0em]
\item
感应电动势的大小 $ E $;
\item 
拉力做功的功率 $ P $;
\item 
$ ab $ 边产生的焦耳热 $ Q $。

\end{enumerate}
\begin{figure}[h!]
\flushright
\includesvg[width=0.25\linewidth]{picture/svg/GZ-3-tiyou-0845}
\end{figure}

\banswer{
\begin{enumerate}
%\renewcommand{\labelenumi}{\arabic{enumi}.}
% A(\Alph) a(\alph) I(\Roman) i(\roman) 1(\arabic)
%设定全局标号series=example	%引用全局变量resume=example
%[topsep=-0.3em,parsep=-0.3em,itemsep=-0.3em,partopsep=-0.3em]
%可使用leftmargin调整列表环境左边的空白长度 [leftmargin=0em]
\item
$E=B L v$
\item 
$P=\frac{B^{2} L^{2} v^{2}}{R}$
\item 
$Q=\frac{B^{2} L^{3} v}{4}$
\end{enumerate}
}


\item 
\exwhere{$ 2019 $ 年物理江苏卷}
如图所示,匀强磁场中有一个用软导线制成的单匝闭合线圈,线圈平面
与磁场垂直.已知线圈的面积 $ S=0.3 \ m^{2} $、电阻 $ R=0.6 \ \Omega $,磁场的磁感应强度 $ B=0.2 \ T $.现同时向两侧拉
动线圈,线圈的两边在$ \Delta t=0.5 \ s $ 时间内合到一起.求线圈在上述过程中:
\begin{enumerate}
%\renewcommand{\labelenumi}{\arabic{enumi}.}
% A(\Alph) a(\alph) I(\Roman) i(\roman) 1(\arabic)
%设定全局标号series=example	%引用全局变量resume=example
%[topsep=-0.3em,parsep=-0.3em,itemsep=-0.3em,partopsep=-0.3em]
%可使用leftmargin调整列表环境左边的空白长度 [leftmargin=0em]
\item
感应电动势的平均值 $ E $;
\item 
感应电流的平均值 $ I $,并在图中标出电流方向;
\item 
通过导线横截面的电荷量 $ q $.


\end{enumerate}
\begin{figure}[h!]
\flushright
\includesvg[width=0.25\linewidth]{picture/svg/GZ-3-tiyou-0846}
\end{figure}



\banswer{
\begin{enumerate}
%\renewcommand{\labelenumi}{\arabic{enumi}.}
% A(\Alph) a(\alph) I(\Roman) i(\roman) 1(\arabic)
%设定全局标号series=example	%引用全局变量resume=example
%[topsep=-0.3em,parsep=-0.3em,itemsep=-0.3em,partopsep=-0.3em]
%可使用leftmargin调整列表环境左边的空白长度 [leftmargin=0em]
\item
$ E=0.12 \ V $
\item 
$ I=0.2 \ A $(电流方向见图)
\begin{center}
 \includesvg[width=0.23\linewidth]{picture/svg/GZ-3-tiyou-0847} 
\end{center}
\item 
$ q=0.1 \ C $
\end{enumerate}
}







\end{enumerate}

