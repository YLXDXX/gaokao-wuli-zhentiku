\bta{第七讲$ \quad $电容和电容器}


\begin{enumerate}[leftmargin=0em]
\renewcommand{\labelenumi}{\arabic{enumi}.}
% A(\Alph) a(\alph) I(\Roman) i(\roman) 1(\arabic)
%设定全局标号series=example	%引用全局变量resume=example
%[topsep=-0.3em,parsep=-0.3em,itemsep=-0.3em,partopsep=-0.3em]
%可使用leftmargin调整列表环境左边的空白长度 
\item
\exwhere{$ 2012 $年物理江苏卷}
一充电后的平行板电容器保持两极板的正对面积、间距和电荷量不变,在两极板间插入一电介质,其电容$ C $和两极板间的电势差$ U $的变化情况是 \xzanswer{B} 


\fourchoices
{$C $和$ U $均增大}
{$C $增大$ ,U $减小}
{$C $减小$ , U $增大}
{$C $和$ U $均减小}





\item
\exwhere{$ 2012 $年理综浙江卷}
为了测量储罐中不导电液体的高度,将与储罐外壳绝缘的两块平行金属板构成的电容器$ C $置于储罐中,电容器可通过开关$ S $与线圈$ L $或电源相连,如图所示。当开关从$ a $拨到$ b $时,由$ L $与$ C $构成的回路中产生周期$T = 2 \pi \sqrt { L C }$的振荡电流。当罐中的液面上升时 \xzanswer{BC} 
\begin{figure}[h!]
\centering
\includesvg[width=0.19\linewidth]{picture/svg/072}
\end{figure}


\fourchoices
{电容器的电容减小 }
{电容器的电容增大}
{$ LC $回路的振荡频率减小 }
{$ LC $回路的振荡频率增大}




\item
\exwhere{$ 2012 $年物理海南卷}
将平行板电容器两极板之间的距离、电压、电场强度大小和极板所带的电荷量分别用$ d $、$ U $、$ E $和$ Q $表示。下列说法正确的是 \xzanswer{AD} 


\fourchoices
{保持$ U $不变,将$ d $变为原来的两倍,则$ E $变为原来的一半}
{保持$ E $不变,将$ d $变为原来的一半,则$ U $变为原来的两倍}
{保持$ d $不变,将$ Q $变为原来的两倍,则$ U $变为原来的一半}
{保持$ d $不变,将$ Q $变为原来的一半,则$ E $变为原来的一半}





\item
\exwhere{$ 2014 $年物理海南卷}
如图,一平行板电容器的两极板与一电压恒定的电源相连,极板水平放置,极板间距为$ d $,在下极板上叠放一厚度为$ l $的金属板,其上部空间有一带电粒子$ P $静止在电容器中,当把金属板从电容器中快速抽出后,粒子$ P $开始运动,重力加速度为$ g $。粒子运动加速度为 \xzanswer{A} 
\begin{figure}[h!]
\centering
\includesvg[width=0.19\linewidth]{picture/svg/073}
\end{figure}
\fourchoices
{$\frac { l } { d } g$}
{$\frac { d - l } { d } g$}
{$\frac { l } { d - l } g$}
{$\frac { d } { d - l } g$}





\item
\exwhere{$ 2016 $年新课标\lmd{1}卷}
一平行板电容器两极板之间充满云母介质,接在恒压直流电源上。若将云母介质移出,则电容器 \xzanswer{D} 


\fourchoices
{极板上的电荷量变大,极板间电场强度变大}
{极板上的电荷量变小,极板间电场强度变大}
{极板上的电荷量变大,极板间电场强度不变}
{极板上的电荷量变小,极板间电场强度不变}






\item
\exwhere{$ 2016 $年浙江卷}
以下说法正确的是 \xzanswer{A} 


\fourchoices
{在静电场中,沿着电场线方向电势逐渐降低}
{外力对物体所做的功越多,对应的功率越大}
{电容器电容$ C $与电容器所带电荷量$ Q $成正比}
{在超重和失重现象中,地球对物体的实际作用力发生了变化}




\item
\exwhere{$ 2016 $年天津卷}
如图所示,平行板电容器带有等量异种电荷,与静电计相连,静电计金属外壳和电容器下极板都接地。在两极板间有一个固定在$ P $点的点电荷,以$ E $表示两板间的电场强度,$ E_{p} $表示点电荷在$ P $点的电势能,$ \theta $表示静电计指针的偏角。若保持下极板不动,将上极板向下移动一小段距离至图中虚线位置,则 \xzanswer{D} 
\begin{figure}[h!]
\centering
\includesvg[width=0.25\linewidth]{picture/svg/074}
\end{figure}


\fourchoices
{$ \theta $增大,$ E $增大}
{$ \theta $增大,$ Ep $不变 }
{$ \theta $减小,$ Ep $增大}
{$ \theta $减小,$ E $不变}



\item
\exwhere{$ 2011 $年理综天津卷}
板间距为$ d $的平行板电容器所带电荷量为$ Q $时,两极板间电势差为$ U_{1} $,板间场强为$ E_{1} $。现将电容器所带电荷量变为$ 2Q $,板间距变为$ d/2 $,其他条件不变,这时两极板间电势差为$ U_{2} $,板间场强为$ E_{2} $,下列说法正确的是 \xzanswer{C} 


\fourchoices
{$ U_{2} = U_{1} $,$ E_{2} = E_{1} $ }
{$ U_{2} =2 U_{1} $,$ E_{2} =4 E_{1} $}
{$ U_{2} = U_{1} $,$ E_{2} =2 E_{1} $}
{$ U_{2} =2 U_{1} $,$ E_{2} =2 E_{1} $}






\item
\exwhere{$ 2012 $年理综全国卷}
如图,一平行板电容器的两个极板竖直放置,在两极板间有一带电小球,小球用一绝缘轻线悬挂于$ O $点。先给电容器缓慢充电,使两极板所带电荷量分别为$ + Q $和$ -Q $,此时悬线与竖直方向的夹角为$ \pi /6 $。再给电容器缓慢充电,直到悬线和竖直方向的夹角增加到$ \pi /3 $,且小球与两极板不接触。求第二次充电使电容器正极板增加的电荷量。


\begin{figure}[h!]
\flushright
\includesvg[width=0.25\linewidth]{picture/svg/075}
\end{figure}
\banswer{
第二次充电使电容器正极板增加的电荷量$\Delta Q = \frac { 2 \sqrt { 3 } m g C d } { 3 q } = 2 Q$
}



\item
\exwhere{$ 2014 $年理综安徽卷}
如图所示,充电后的平行板电容器水平放置,电容为$ C $,极板间距离为$ d $,上极板正中有一小孔。质量为$ m $、电荷量为$ +q $的小球从小孔正上方高$ h $处由静止开始下落,穿过小孔到达下极板处速度恰为零(空气阻力忽略不计,极板间电场可视为匀强电场,重力加速度为$ g $)。求:

\begin{minipage}[h!]{0.7\linewidth}
\vspace{0.3em}
\begin{enumerate}
\renewcommand{\labelenumi}{\arabic{enumi}.}
% A(\Alph) a(\alph) I(\Roman) i(\roman) 1(\arabic)
%设定全局标号series=example	%引用全局变量resume=example
%[topsep=-0.3em,parsep=-0.3em,itemsep=-0.3em,partopsep=-0.3em]
%可使用leftmargin调整列表环境左边的空白长度 [leftmargin=0em]
\item
小球到达小孔处的速度;
\item 极板间电场强度大小和电容器所带电荷量;
\item 小球从开始下落运动到下极板的时间。



\end{enumerate}
\vspace{0.3em}
\end{minipage}
\hfill
\begin{minipage}[h!]{0.3\linewidth}
\flushright
\vspace{0.3em}
\includesvg[width=0.8\linewidth]{picture/svg/076}
\vspace{0.3em}
\end{minipage}

\vspace{0.7cm}

\banswer{
\begin{enumerate}
\renewcommand{\labelenumi}{\arabic{enumi}.}
% A(\Alph) a(\alph) I(\Roman) i(\roman) 1(\arabic)
%设定全局标号series=example	%引用全局变量resume=example
%[topsep=-0.3em,parsep=-0.3em,itemsep=-0.3em,partopsep=-0.3em]
%可使用leftmargin调整列表环境左边的空白长度 [leftmargin=0em]
\item
$ v_{\text{孔}}\sqrt { 2 g h }$

\item 
$E=\frac { m g ( h + d ) } { q d }$,$Q=C \frac { m g ( h + d ) } { q }$
\item 
$t=\frac { h + d } { h } \sqrt { \frac { 2 h } { g } }$


\end{enumerate}
}



\item
\exwhere{$ 2018 $年江苏卷}
如图所示,电源$ E $对电容器$ C $充电,当$ C $两端电压达到$ 80 $ $ V $时,闪光灯瞬间导通并发光,$ C $放电.放电后,闪光灯断开并熄灭,电源再次对$ C $充电。这样不断地充电和放电,闪光灯就周期性地发光。该电路 \xzanswer{BCD} 
\begin{figure}[h!]
\centering
\includesvg[width=0.19\linewidth]{picture/svg/077}
\end{figure}


\fourchoices
{充电时,通过$ R $的电流不变}
{若$ R $增大,则充电时间变长}
{若$ C $增大,则闪光灯闪光一次通过的电荷量增大}
{若$ E $减小为$ 85 $ $ V $,闪光灯闪光一次通过的电荷量不变}





\item
\exwhere{$ 2018 $年北京卷}
研究与平行板电容器电容有关因素的实验装置如图所示。下列说法正确的是 \xzanswer{A} 
\begin{figure}[h!]
\centering
\includesvg[width=0.19\linewidth]{picture/svg/078}
\end{figure}


\fourchoices
{实验前,只用带电玻璃棒与电容器$ a $板接触,能使电容器带电}
{实验中,只将电容器$ b $板向上平移,静电计指针的张角变小}
{实验中,只在极板间插入有机玻璃板,静电计指针的张角变大}
{实验中,只增加极板带电量,静电计指针的张角变大,表明电容增大}







\item 
\exwhere{$ 2019 $年物理北京卷}
电容器作为储能器件,在生产生活中有广泛的应用。对给定电容值为$ C $的电容器充电,无论采用何种充电方式,其两极间的电势差$ u $随电荷量$ q $的变化图像都相同。

\begin{enumerate}
\renewcommand{\labelenumi}{\arabic{enumi}.}
% A(\Alph) a(\alph) I(\Roman) i(\roman) 1(\arabic)
%设定全局标号series=example	%引用全局变量resume=example
%[topsep=-0.3em,parsep=-0.3em,itemsep=-0.3em,partopsep=-0.3em]
%可使用leftmargin调整列表环境左边的空白长度 [leftmargin=0em]
\item
请在图$ 1 $中画出上述$ u - q $图像。类比直线运动中由$ v - t $图像求位移的方法,求两极间电压为$ U $时电容器所储存的电能$ E_{p}= $\tk{$\frac { C U ^ { 2 } } { 2 }$}。
\begin{figure}[h!]
\centering
\includesvg[width=0.19\linewidth]{picture/svg/079}
\end{figure}



\item 
在如图$ 2 $所示充电电路中,$ R $表示电阻,$ E $表示电源(忽略内阻)。通过改变电路中元件的参数对同一电容器进行两次充电,对应的$ q - t $曲线如图$ 3 $中的①②所示。
\begin{figure}[h!]
\centering
\includesvg[width=0.49\linewidth]{picture/svg/080}
\end{figure}

\begin{enumerate}
\renewcommand{\labelenumiii}{\arabic{enumiii}.}
% A(\Alph) a(\alph) I(\Roman) i(\roman) 1(\arabic)
%设定全局标号series=example	%引用全局变量resume=example
%[topsep=-0.3em,parsep=-0.3em,itemsep=-0.3em,partopsep=-0.3em]
%可使用leftmargin调整列表环境左边的空白长度 [leftmargin=0em]
\item
①②两条曲线不同是\tk{R}(选填$ E $或$ R $)的改变造成的;
\item 
电容器有时需要快速充电,有时需要均匀充电。依据$ a $中的结论,说明实现这两种充电方式的途径。\\
\tk{快速度充电,减小图2中的电阻R;均匀充电,适当增大图2中的电阻R}


\end{enumerate}


\item 
设想使用理想的“恒流源”替换($ 2 $)中电源对电容器充电,可实现电容器电荷量随时间均匀增加。请思考使用“恒流源”和($ 2 $)中电源对电容器的充电过程,填写下表(选填“增大”、“减小”或“不变”)。
\begin{table}[h!]
\centering 
\begin{tabular}{|c|c|c|}
\hline 
& “恒流源” & ($ 2 $)中电源 \\
\hline
\tabincell{c}{
电源\\两端电压
} & \tk{增大} & \tk{不变} \\
\hline
\tabincell{c}{
通过电源\\的电流
} & \tk{不变} & \tk{减小}\\ 
\hline 
\end{tabular}
\end{table} 





\end{enumerate}



\end{enumerate}






