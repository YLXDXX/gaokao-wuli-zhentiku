\bta{原子物理综合}

\begin{enumerate}
	%\renewcommand{\labelenumi}{\arabic{enumi}.}
	% A(\Alph) a(\alph) I(\Roman) i(\roman) 1(\arabic)
	%设定全局标号series=example	%引用全局变量resume=example
	%[topsep=-0.3em,parsep=-0.3em,itemsep=-0.3em,partopsep=-0.3em]
	%可使用leftmargin调整列表环境左边的空白长度 [leftmargin=0em]
	\item
\exwhere{$ 2013 $ 年重庆卷}
如图所示,高速运动的$ \alpha $粒子被位于 $ O $ 点的重原子核散射,实线表示$ \alpha $粒子运动的轨迹,$ M $、$ N $
和 $ Q $ 为轨迹上的三点,$ N $ 点离核最近,$ Q $ 点比 $ M $ 点离核更远,则 \xzanswer{B} 
\begin{figure}[h!]
	\centering
	\includesvg[width=0.23\linewidth]{picture/svg/GZ-3-tiyou-1307}
\end{figure}


\fourchoices
{$ \alpha $粒子在 $ M $ 点的速率比在 $ Q $ 点的大}
{三点中,$ \alpha $粒子在 $ N $ 点的电势能最大}
{在重核产生的电场中,$ M $ 点的电势比 $ Q $ 点的低}
{$ \alpha $ 粒子从 $ M $ 点运动到 $ Q $ 点,电场力对它做的总功为负功}



\item 
\exwhere{$ 2012 $ 年理综天津卷}
下列说法正确的是 \xzanswer{B} 

\fourchoices
{采用物理或化学方法可以有效地改变放射性元素的半衰期}
{由玻尔理论知道氢原子从激发态跃迁到基态时会放出光子}
{从高空对地面进行遥感摄影是利用紫外线良好的穿透能力}
{原子核所含核子单独存在时的总质量小于该原子核的质量}



\item
\exwhere{$ 2011 $ 年理综天津卷}
下列能揭示原子具有核式结构的实验是 \xzanswer{C} 

\fourchoices
{光电效应实验}
{伦琴射线的发现}
{$ \alpha $粒子散射实验}
{氢原子光谱的发现}


\item 
\exwhere{$ 2012 $ 年理综北京卷}
“约瑟夫森结”由超导体和绝缘体制成。若在结两端加恒定电压 $ U $,则它会辐射频率为 $ \nu $ 的电磁波,
且 $ \nu $ 与 $ U $ 成正比,即 $ \nu=kU $。已知比例系数 $ k $ 仅与元电荷 $ e $ 的 $ 2 $ 倍和普朗克常量 $ h $ 有关。你可能不
了解此现象为机理,但仍可运用物理学中常用的方法。在下列选项中,推理判断比例系数 $ k $ 的值可
能为 \xzanswer{B} 
\fourchoices
{$\frac{h}{2 e}$}
{$\frac{2 e}{h}$}
{$ 2 he $}
{$\frac{1}{2 h e}$}



\item 
\exwhere{$ 2011 $ 年理综重庆卷}
核电站核泄漏的污染物中含有碘 $ 131 $ 和铯 $ 137 $。碘 $ 131 $ 的半衰期约为 $ 8 $ 天,会释放$ \beta $射线;铯
$ 137 $ 是铯 $ 133 $ 的同位素,半衰期约为 $ 30 $ 年,发生衰变期时会辐射$ \gamma $射线。下列说法正确的是 \xzanswer{D} 

\fourchoices
{碘 $ 131 $ 释放的$ \beta $射线由氦核组成}
{铯 $ 137 $ 衰变时辐射出的$ \gamma $光子能量小于可见光光子能量}
{与铯 $ 137 $ 相比,碘 $ 131 $ 衰变更慢}
{铯 $ 133 $ 和铯 $ 137 $ 含有相同的质子数}


\item
\exwhere{$ 2013 $ 年天津卷}
下列说法正确的是 \xzanswer{C} 

\fourchoices
{原子核发生衰变时要遵守电荷守恒和质量守恒的规律}
{$ \alpha $射线、$ \beta $射线、$ \gamma $射线都是高速运动的带电粒子流}
{氢原子从激发态向基态跃迁只能辐射特定频率的光子}
{发生光电效应时光电子的动能只与入射光的强度有关}


\item 
\exwhere{$ 2014 $ 年理综天津卷}
下列说法正确的是 \xzanswer{BD} 

\fourchoices
{玻尔对氢原子光谱的研究导致原子的核式结构模型的建立}
{可利用某些物质在紫外线照射下发出荧光来设计防伪措施}
{天然放射现象中产生的射线都能在电场或磁场中发生偏转}
{观察者与波源互相远离时接收到波的频率与波源频率不同}





	
	
	
\end{enumerate}

