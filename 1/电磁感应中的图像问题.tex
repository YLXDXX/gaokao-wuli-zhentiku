\bta{电磁感应中的图像问题}



\begin{enumerate}
%\renewcommand{\labelenumi}{\arabic{enumi}.}
% A(\Alph) a(\alph) I(\Roman) i(\roman) 1(\arabic)
%设定全局标号series=example	%引用全局变量resume=example
%[topsep=-0.3em,parsep=-0.3em,itemsep=-0.3em,partopsep=-0.3em]
%可使用leftmargin调整列表环境左边的空白长度 [leftmargin=0em]
\item
\exwhere{$ 2019 $年物理全国\lmd{1}卷}
空间存在一方向与直面垂直、大小随时间变化的匀强磁场,其边界如图($ a $)
中虚线$ MN $所示,一硬质细导线的电阻率为$ \rho $、横截面积为$ S $,将该导线做成半径为$ r $的圆环固定在纸
面内,圆心$ O $在$ MN $上。$ t=0 $时磁感应强度的方向如图($ a $)所示:磁感应强度$ B $随时间$ t $的变化关系如
图($ b $)所示,则在$ t=0 $到$ t= t_{1} $的时间间隔内 \xzanswer{BC} 
\begin{figure}[h!]
\centering
\begin{subfigure}{0.4\linewidth}
\centering
\includesvg[width=0.7\linewidth]{picture/svg/GZ-3-tiyou-0848} 
\caption{}\label{}
\end{subfigure}
\begin{subfigure}{0.4\linewidth}
\centering
\includesvg[width=0.7\linewidth]{picture/svg/GZ-3-tiyou-0849} 
\caption{}\label{}
\end{subfigure}
\end{figure}


\fourchoices
{圆环所受安培力的方向始终不变}
{圆环中的感应电流始终沿顺时针方向}
{圆环中的感应电流大小为$\frac{B_{0} r S}{4 t_{0} \rho}$}
{圆环中的感应电动势大小为$\frac{B_{0} \pi r^{2}}{4 t_{0}}$}


\item 
\exwhere{$ 2019 $ 年物理全国\lmd{2}卷}
如图,两条光滑平行金属导轨固定,所在平面与水平面夹角为$ \theta $,导轨电
阻忽略不计。虚线 $ ab $、$ cd $ 均与导轨垂直,在 $ ab $ 与 $ cd $ 之间的区域存在垂直于导轨所在平面的匀强
磁场。将两根相同的导体棒 $ PQ $、$ MN $ 先后自导轨上同一位置由静止释放,两者始终与导轨垂直且
接触良好。已知 $ PQ $ 进入磁场开始计时,到 $ MN $ 离开磁场区域为止,流过 $ PQ $ 的电流随时间变化的
图像可能正确的是 \xzanswer{AD} 
\begin{figure}[h!]
\centering
\includesvg[width=0.23\linewidth]{picture/svg/GZ-3-tiyou-0850}
\end{figure}


\pfourchoices
{\includesvg[width=4.3cm]{picture/svg/GZ-3-tiyou-0851}}
{\includesvg[width=4.3cm]{picture/svg/GZ-3-tiyou-0852}}
{\includesvg[width=4.3cm]{picture/svg/GZ-3-tiyou-0853}}
{\includesvg[width=4.3cm]{picture/svg/GZ-3-tiyou-0854}}




\item 
\exwhere{$ 2019 $ 年物理全国\lmd{3}卷}
如图,方向竖直向下的匀强磁场中有两根位于同一水平面内的足够长的
平行金属导轨,两相同的光滑导体棒 $ ab $、$ cd $ 静止在导轨上。$ t=0 $ 时,棒 $ ab $ 以初速度 $ v_{0} $ 向右滑动。
运动过程中,$ ab $、$ cd $ 始终与导轨垂直并接触良好,两者速度分别用 $ v_{1} $、$ v_{2} $ 表示,回路中的电流用 $ I $
表示。下列图像中可能正确的是 \xzanswer{AC} 
\begin{figure}[h!]
\centering
\includesvg[width=0.23\linewidth]{picture/svg/GZ-3-tiyou-0855}
\end{figure}


\pfourchoices
{\includesvg[width=4.3cm]{picture/svg/GZ-3-tiyou-0856}}
{\includesvg[width=4.3cm]{picture/svg/GZ-3-tiyou-0857}}
{\includesvg[width=4.3cm]{picture/svg/GZ-3-tiyou-0858}}
{\includesvg[width=4.3cm]{picture/svg/GZ-3-tiyou-0859}}





\item 
\exwhere{$ 2013 $ 年新课标\lmd{2}卷}
如图,在光滑水平桌面上有一边长为 $ L $、电阻为 $ R $ 的正方形导线框;在导线框右侧有一宽度为
$ d ( d>L) $ 的条形匀强磁场区域,磁场的边界与导线框的一边
平行,磁场方向竖直向下。导线框以某一初速度向右运动,$ t=0 $
是导线框的的右边恰与磁场的左边界重合,随后导线框进入并
通过磁场区域。下列 $ v-t $ 图象中,可能正确描述上述过程的是 \xzanswer{D} 
\begin{figure}[h!]
\centering
\includesvg[width=0.23\linewidth]{picture/svg/GZ-3-tiyou-0860}
\end{figure}


\pfourchoices
{\includesvg[width=4.3cm]{picture/svg/GZ-3-tiyou-0861}}
{\includesvg[width=4.3cm]{picture/svg/GZ-3-tiyou-0862}}
{\includesvg[width=4.3cm]{picture/svg/GZ-3-tiyou-0863}}
{\includesvg[width=4.3cm]{picture/svg/GZ-3-tiyou-0864}}




\item
\exwhere{$ 2013 $ 年全国大纲卷}
纸面内两个半径均为 $ R $ 的圆相切于 $ O $ 点,两圆形区域内分别存在垂直于纸面的匀强磁场,磁
感应强度大小相等、方向相反,且不随时间变化。一长为 $ 2R $ 的导体
杆 $ OA $ 绕过 $ O $ 点且垂直于纸面的轴顺时针转动,角速度为 $ \omega $。$ t=0 $
时,$ OA $ 恰好位于两圆的公切线上,如图所示,若选取从 $ O $ 指向 $ A $ 的电动势为正,下列描述导体杆中
感应电动势随时间变化的图像可能正确的是 \xzanswer{C} 
\begin{figure}[h!]
\centering
\includesvg[width=0.23\linewidth]{picture/svg/GZ-3-tiyou-0865}
\end{figure}

\pfourchoices
{\includesvg[width=4.3cm]{picture/svg/GZ-3-tiyou-0867}}
{\includesvg[width=4.3cm]{picture/svg/GZ-3-tiyou-0868}}
{\includesvg[width=4.3cm]{picture/svg/GZ-3-tiyou-0869}}
{\includesvg[width=4.3cm]{picture/svg/GZ-3-tiyou-0870}}



\item 
\exwhere{$ 2013 $ 年山东卷}
将一段导线绕成图甲所示的闭合回路,并固定在水平面(纸面)内。回路的 $ ab $ 边置于垂直纸
面向里的匀强磁场 \lmd{1} 中。回路的圆环区域内有
垂直纸面的磁场 \lmd{2} 。以向里为磁场 \lmd{2} 的正方
向,其磁感应强度 $ B $ 随时间 $ t $ 变化的图像如图
乙所示。用 $ F $ 表示 $ ab $ 边受到的安培力,以水
平向右为 $ F $ 的正方向,能正确反映 $ F $ 随时间 $ t $
变化的图像是 \xzanswer{B} 
\begin{figure}[h!]
\centering
\begin{subfigure}{0.4\linewidth}
\centering
\includesvg[width=0.7\linewidth]{picture/svg/GZ-3-tiyou-0871} 
\caption{}\label{}
\end{subfigure}
\begin{subfigure}{0.4\linewidth}
\centering
\includesvg[width=0.7\linewidth]{picture/svg/GZ-3-tiyou-0872} 
\caption{}\label{}
\end{subfigure}
\end{figure}


\pfourchoices
{\includesvg[width=4.3cm]{picture/svg/GZ-3-tiyou-0874}}
{\includesvg[width=4.3cm]{picture/svg/GZ-3-tiyou-0875}}
{\includesvg[width=4.3cm]{picture/svg/GZ-3-tiyou-0876}}
{\includesvg[width=4.3cm]{picture/svg/GZ-3-tiyou-0877}}


\item 
\exwhere{$ 2013 $ 年浙江卷}
磁卡的磁条中有用于存储信息的磁极方向不同的磁化区,刷卡器中有检测线圈。当以速度 $ v_{0} $
刷卡时,在线圈中产生感应电动势。其 $ E-t $ 关系如右图所示。如果只将刷卡速度改为 $ v_{0}/2 $,线圈中
的 $ E-t $ 关系可能是 \xzanswer{D} 
\begin{figure}[h!]
\centering
\includesvg[width=0.23\linewidth]{picture/svg/GZ-3-tiyou-0878}
\end{figure}

\pfourchoices
{\includesvg[width=4.3cm]{picture/svg/GZ-3-tiyou-0879}}
{\includesvg[width=4.3cm]{picture/svg/GZ-3-tiyou-0880}}
{\includesvg[width=4.3cm]{picture/svg/GZ-3-tiyou-0881}}
{\includesvg[width=4.3cm]{picture/svg/GZ-3-tiyou-0882}}



\item
\exwhere{$ 2013 $ 年福建卷}
如图,矩形闭合线框在匀强磁场上方,由不同高度静止释放,用 $ t_{1} $、$ t_{2} $
分别表示线框 $ ab $ 边和 $ cd $ 边刚进入磁场的时刻。线框下落过程形状不变,
$ ab $ 边始终保持与磁场水平边界 $ OO ^{\prime} $ 平行,线框平面与磁场方向垂直。设
$ OO ^{\prime} $ 下方磁场磁场区域足够大,不计空气影响,则下列哪一个图像不可能
反映线框下落过程中速度 $ v $ 随时间 $ t $ 变化的规律 \xzanswer{A} 
\begin{figure}[h!]
\centering
\includesvg[width=0.23\linewidth]{picture/svg/GZ-3-tiyou-0883}
\end{figure}

\pfourchoices
{\includesvg[width=4.3cm]{picture/svg/GZ-3-tiyou-0884}}
{\includesvg[width=4.3cm]{picture/svg/GZ-3-tiyou-0885}}
{\includesvg[width=4.3cm]{picture/svg/GZ-3-tiyou-0886}}
{\includesvg[width=4.3cm]{picture/svg/GZ-3-tiyou-0887}}






\item
\exwhere{$ 2012 $ 年理综新课标卷}
如图,一载流长直导线和一矩形导线框固定在同一平面内,线框在长直导线右侧,且其长边与长
直导线平行。已知在 $ t=0 $ 到 $ t= t_{1} $ 的时间间隔内,直导线中电流 $ i $ 发生某种变化,而线框中感应电流
总是沿顺时针方向;线框受到的安培力的合力先水平向左、后水平向右。设电流 $ i $ 正方向与图中箭
头方向相同,则 $ i $ 随时间 $ t $ 变化的图线可能是 \xzanswer{A} 
\begin{figure}[h!]
\centering
\includesvg[width=0.23\linewidth]{picture/svg/GZ-3-tiyou-0888}
\end{figure}


\pfourchoices
{\includesvg[width=4.3cm]{picture/svg/GZ-3-tiyou-0889}}
{\includesvg[width=4.3cm]{picture/svg/GZ-3-tiyou-0890}}
{\includesvg[width=4.3cm]{picture/svg/GZ-3-tiyou-0891}}
{\includesvg[width=4.3cm]{picture/svg/GZ-3-tiyou-0892}}


\item 
\exwhere{$ 2012 $ 年理综重庆卷}
如图所示,正方形区域 $ MNPQ $ 垂直纸面向里的匀强磁场。在外力
作用下,一正方形闭合刚性导线框沿 $ QN $ 方向匀速运动,$ t=0 $ 时刻,其四个
顶点 $ M^{\prime} $、 $ N^{\prime} $、 $ P^{\prime} $、 $ Q^{\prime} $恰好在磁场边界中点。下列图象中能反映线框所
受安培力 $ f $ 的大小随时间 $ t $ 变化规律的是 \xzanswer{B} 
\begin{figure}[h!]
\centering
\includesvg[width=0.23\linewidth]{picture/svg/GZ-3-tiyou-0893}
\end{figure}


\pfourchoices
{\includesvg[width=4.3cm]{picture/svg/GZ-3-tiyou-0894}}
{\includesvg[width=4.3cm]{picture/svg/GZ-3-tiyou-0895}}
{\includesvg[width=4.3cm]{picture/svg/GZ-3-tiyou-0896}}
{\includesvg[width=4.3cm]{picture/svg/GZ-3-tiyou-0897}}


\item 
\exwhere{$ 2012 $ 年理综福建卷}
如图,一圆形闭合铜环由高处从静止开始下落,穿过一根竖直悬挂的条形磁
铁,铜环的中心轴线与条形磁铁的中轴线始终保持重合。若取磁铁中心 $ O $ 为坐标原
点,建立竖直向下正方向的 $ x $ 轴,则图乙中最能正确反映环中感应电流 $ i $ 随环心位
置坐标 $ x $ 变化的关系图像是 \xzanswer{B} 
\begin{figure}[h!]
\centering
\includesvg[width=0.23\linewidth]{picture/svg/GZ-3-tiyou-0898}
\end{figure}

\pfourchoices
{\includesvg[width=4.3cm]{picture/svg/GZ-3-tiyou-0899}}
{\includesvg[width=4.3cm]{picture/svg/GZ-3-tiyou-0900}}
{\includesvg[width=4.3cm]{picture/svg/GZ-3-tiyou-0901}}
{\includesvg[width=4.3cm]{picture/svg/GZ-3-tiyou-0902}}

\item 
\exwhere{$ 2011 $ 年海南卷}
如图,$ EOF $ 和 $ E ^{\prime} OF ^{\prime} $为空间一匀强磁场的边界,其中 $ EO // E ^{\prime} O ^{\prime} $,$ OF // F ^{\prime} O ^{\prime} $,且 $ EO \perp OF $;$ OO ^{\prime} $
为$ \angle EOF $ 的角平分线,$ OO ^{\prime} $ 间的距离为 $ l $;磁场方向垂直于
纸面向里。一边长为 $ l $ 的正方形导线框沿 $ O ^{\prime} O $ 方向匀速通过
磁场,$ t=0 $ 时刻恰好位于图示位置。规定导线框中感应电流沿
逆时针方向时为正,则感应电流 $ i $ 与时间 $ t $ 的关系图线可能正
确的是 \xzanswer{B} 
\begin{figure}[h!]
\centering
\includesvg[width=0.23\linewidth]{picture/svg/GZ-3-tiyou-0903}
\end{figure}

\pfourchoices
{\includesvg[width=4.3cm]{picture/svg/GZ-3-tiyou-0904}}
{\includesvg[width=4.3cm]{picture/svg/GZ-3-tiyou-0905}}
{\includesvg[width=4.3cm]{picture/svg/GZ-3-tiyou-0906}}
{\includesvg[width=4.3cm]{picture/svg/GZ-3-tiyou-0907}}



\item
\exwhere{$ 2016 $ 年四川卷}
如图所示,电阻不计、间距为 $ l $ 的光滑平行金属导轨水平放置于磁感应强度为 $ B $、
方向竖直向下的匀强磁场中,导轨左端接一定值电阻 $ R $。质量为 $ m $、电阻为 $ r $ 的金属棒 $ MN $ 置于导
轨上,受到垂直于金属棒的水平外力 $ F $ 的作用由静止开始运动,外力 $ F $ 与金属棒速度 $ v $ 的关系是
$ F=F_{0}+kv $($ F $、$ k $ 是常量),金属棒与导轨始终垂直且接触良好。金属棒中感应电流为 $ i $,受到的安培
力大小为 $ F_{ \text{安} } $ ,电阻 $ R $ 两端的电压为 $ U_{R} $,感应电流的功率为 $ P $,它们随时间 $ t $ 变化图像可能正确的
有 \xzanswer{BC} 
\begin{figure}[h!]
\centering
\includesvg[width=0.23\linewidth]{picture/svg/GZ-3-tiyou-0908}
\end{figure}

\pfourchoices
{\includesvg[width=4.3cm]{picture/svg/GZ-3-tiyou-0909}}
{\includesvg[width=4.3cm]{picture/svg/GZ-3-tiyou-0910}}
{\includesvg[width=4.3cm]{picture/svg/GZ-3-tiyou-0911}}
{\includesvg[width=4.3cm]{picture/svg/GZ-3-tiyou-0912}}



\item 
\exwhere{$ 2018 $ 年全国\lmd{2}卷}
如图,在同一水平面内有两根平行长导轨,导轨间存在依次相邻的矩
形匀强磁场区域,区域宽度均为 $ l $,磁感应强度大小相等、方向交替向上向下。一边长为 $ \frac{ 3 }{ 2 } l $ 的正
方形金属线框在导轨上向左匀速运动。线框中感应电流 $ i $ 随时间 $ t $ 变化的正确图线可能是 \xzanswer{D} 
\begin{figure}[h!]
\centering
\includesvg[width=0.23\linewidth]{picture/svg/GZ-3-tiyou-0913}
\end{figure}

\pfourchoices
{\includesvg[width=4.3cm]{picture/svg/GZ-3-tiyou-0914}}
{\includesvg[width=4.3cm]{picture/svg/GZ-3-tiyou-0915}}
{\includesvg[width=4.3cm]{picture/svg/GZ-3-tiyou-0916}}
{\includesvg[width=4.3cm]{picture/svg/GZ-3-tiyou-0917}}


\item 
\exwhere{$ 2015 $ 年理综山东卷}
如图甲,$ R_{0} $ 为定值电阻,两金属圆环固定在同一绝缘平面内。左端连接
在一周期为 $ T_{0} $ 的正弦交流电源上,经
二极管整流后,通过 $ R_{0} $ 的电流 $ i $ 始终向
左,其大小按图乙所示规律变化。规定
内圆环 $ a $ 端电势高于 $ b $ 端时,$ a $、$ b $ 间的
电压 $ u_{ab} $ 为正,下列 $ u_{ab}-t $ 图像可能正确
的是 \xzanswer{C} 
\begin{figure}[h!]
\centering
\begin{subfigure}{0.4\linewidth}
\centering
\includesvg[width=0.7\linewidth]{picture/svg/GZ-3-tiyou-0923} 
\caption{}\label{}
\end{subfigure}
\begin{subfigure}{0.4\linewidth}
\centering
\includesvg[width=0.7\linewidth]{picture/svg/GZ-3-tiyou-0924} 
\caption{}\label{}
\end{subfigure}
\end{figure}

\pfourchoices
{\includesvg[width=4.3cm]{picture/svg/GZ-3-tiyou-0925}}
{\includesvg[width=4.3cm]{picture/svg/GZ-3-tiyou-0926}}
{\includesvg[width=4.3cm]{picture/svg/GZ-3-tiyou-0927}}
{\includesvg[width=4.3cm]{picture/svg/GZ-3-tiyou-0928}}






\item 
\exwhere{$ 2011 $ 年理综山东卷}
如图甲所示,两固定的竖直光滑金属导轨足够长且电阻不计。两
质量、长度均相同的导体棒 $ c $、$ d $,置于边界水平的匀强磁场上方同一
高度 $ h $ 处。磁场宽为 $ 3 h $,方向与导轨平面垂直。先由静止释放 $ c $,$ c $
刚进入磁场即匀速运动,此时再由静止释放 $ d $,两导体棒与导轨始终
保持良好接触。用 $ a_c $ 表示 $ c $ 的加速度,$ E_{kd} $ 表示 $ d $ 的动能,$ x_c $、$ x_d $ 分别
表示 $ c $、$ d $ 相对释放点的位移。图乙中正确的是 \xzanswer{BD} 
\begin{figure}[h!]
\centering
\includesvg[width=0.23\linewidth]{picture/svg/GZ-3-tiyou-0918}
\end{figure}


\pfourchoices
{\includesvg[width=4.3cm]{picture/svg/GZ-3-tiyou-0919}}
{\includesvg[width=4.3cm]{picture/svg/GZ-3-tiyou-0920}}
{\includesvg[width=4.3cm]{picture/svg/GZ-3-tiyou-0921}}
{\includesvg[width=4.3cm]{picture/svg/GZ-3-tiyou-0922}}


\item 
\exwhere{$ 2013 $ 年广东卷}
如图($ a $)所示,在垂直于匀强磁场 $ B $ 的平面内,半径为 $ r $ 的金属圆盘绕过圆心 $ O $ 的轴转动,圆
心 $ O $ 和边缘 $ K $ 通过电刷与一个电路连接。电路中的 $ P $ 是加上一定正向电压才能导通的电子元件。
流过电流表的电流 $ I $ 与圆盘角速度$ \omega $的关系如图($ b $)所示,图中 $ ab $ 段和 $ bc $ 段均为直线,且 $ ab $
段过坐标原点。$ \omega >0 $ 代表圆盘逆时针转动。已知:$ R=3.0 \ \Omega $,$ B=1.0 \ T $,$ r=0.2 \ m $。忽略圆盘、电流表和
导线的电阻。
\begin{enumerate}
%\renewcommand{\labelenumi}{\arabic{enumi}.}
% A(\Alph) a(\alph) I(\Roman) i(\roman) 1(\arabic)
%设定全局标号series=example	%引用全局变量resume=example
%[topsep=-0.3em,parsep=-0.3em,itemsep=-0.3em,partopsep=-0.3em]
%可使用leftmargin调整列表环境左边的空白长度 [leftmargin=0em]
\item
根据图($ b $)
写出 $ ab $、$ bc $ 段对应 $ I $
与$ \omega $的关系式;


\item 
求出图($ b $)
中 $ b $、$ c $ 两点对应的 $ P $
两端的电压 $ U_{b} $、$ U_{c} $;


\item 
分别求出 $ ab $、$ bc $ 段流过 $ P $ 的电流 $ IP $ 与其两端电压 $ U_P $ 的关系式。

\end{enumerate}
\begin{figure}[h!]
\centering
\begin{subfigure}{0.4\linewidth}
\centering
\includesvg[width=0.7\linewidth]{picture/svg/GZ-3-tiyou-0929} 
\caption{}\label{}
\end{subfigure}
\begin{subfigure}{0.4\linewidth}
\centering
\includesvg[width=0.7\linewidth]{picture/svg/GZ-3-tiyou-0930} 
\caption{}\label{}
\end{subfigure}
\end{figure}



\banswer{
\begin{enumerate}
%\renewcommand{\labelenumi}{\arabic{enumi}.}
% A(\Alph) a(\alph) I(\Roman) i(\roman) 1(\arabic)
%设定全局标号series=example	%引用全局变量resume=example
%[topsep=-0.3em,parsep=-0.3em,itemsep=-0.3em,partopsep=-0.3em]
%可使用leftmargin调整列表环境左边的空白长度 [leftmargin=0em]
\item
$I=\left\{\begin{array}{l}\frac{\omega}{150}, \quad-45 \leq \omega \leq 15 \\ \frac{\omega}{100}-0.05, \quad 15 \leq \omega \leq 45\end{array}\right.$	
\item 	
$U_{b}=0.30 \ V$ \quad 
$U_{c}=0.90 \ V$	
\item 
$a b$ 段流过 $P$ 的电流 $I_{P}$ 与其两端电压 $U_{P}$ 的关系式为$U_{P}=\frac{1}{50} \omega-3 I_{P} \quad(0 \leq \omega \leq 15 rad / s)$或 $U_{P}=3 I_{P^{-}} \frac{1}{50} \omega \quad(-45 rad / s \leq \omega \leq 0)$\\
$b c$ 段流过 $P$ 的电流 $I_{P}$ 与其两端电压 $U_{P}$ 的关系式$U_{P}=-3 I_{P^{+}} \frac{3}{100} \omega-0.15 \quad(15 rad / s \leq \omega \leq 45 rad / s)$
\end{enumerate}
}








\end{enumerate}

