\bta{力与受力分析}


\begin{enumerate}[leftmargin=0em]
\renewcommand{\labelenumi}{\arabic{enumi}.}
% A(\Alph) a(\alph) I(\Roman) i(\roman) 1(\arabic)
%设定全局标号series=example	%引用全局变量resume=example
%[topsep=-0.3em,parsep=-0.3em,itemsep=-0.3em,partopsep=-0.3em]
%可使用leftmargin调整列表环境左边的空白长度 [leftmargin=0em]
\item
\exwhere{$ 2019 $年$ 4 $月浙江物理选考}
如图所示,$ A $、$ B $、$ C $为三个实心小球,$ A $为铁球,$ B $、$ C $为木球。$ A $、$ B $两球分别连在两根弹簧上,$ C $球连接在细线一端,弹簧和细线的下端固定在装水的杯子底部,该水杯置于用绳子悬挂的静止吊篮内。若将挂吊篮的绳子剪断,则剪断的瞬间相对于杯底(不计空气阻力,$ \rho_{ \text{木} } < \rho_{ \text{铁} } < \rho_{ \text{铁} } $) \xzanswer{D} 
\begin{figure}[h!]
\centering
\includesvg[width=0.15\linewidth]{picture/svg/407}
\end{figure}



\fourchoices
{$ A $球将向上运动,$ B $、$ C $球将向下运动}
{$ A $、$ B $球将向上运动,$ C $球不动}
{$ A $球将向下运动,$ B $球将向上运动,$ C $球不动}
{$ A $球将向上运动,$ B $球将向下运动,$ C $球不动}


\item 
\exwhere{$ 2013 $年上海卷}
秋千的吊绳有些磨损。在摆动过程中,吊绳最容易断裂的时候是秋千 \xzanswer{D} 

\fourchoices
{在下摆过程中}
{在上摆过程中}
{摆到最高点时}
{摆到最低点时}

\item
\exwhere{$ 2013 $年上海卷}
如图,质量$ m_A > m_B $的两物体$ A $、$ B $叠放在一起,靠着竖直墙面。让它们由静止释放,在沿粗糙墙面下落过程中,物体$ B $的受力示意图是 \xzanswer{A} 
\begin{figure}[h!]
\centering
\includesvg[width=0.05\linewidth]{picture/svg/408}\\
\includesvg[width=0.83\linewidth]{picture/svg/409}
\end{figure}


\item 
\exwhere{$ 2013 $年上海卷}
两个共点力$ F_{1} $、$ F_{2} $大小不同,它们的合力大小为$ F $,则 \xzanswer{AD} 


\fourchoices
{$ F_{1} $、$ F_{2} $同时增大一倍,$ F $也增大一倍 }
{$ F_{1} $、$ F_{2} $同时增加$ 10 \ N $,$ F $也增加$ 10 \ N $}
{$ F_{1} $增加$ 10 \ N $,$ F_{2} $减少$ 10 \ N $,$ F $一定不变}
{若$ F_{1} $、$ F_{2} $中的一个增大,$ F $不一定增大}

\item 
\exwhere{$ 2012 $年理综新课标卷}
伽利略根据小球在斜面上运动的实验和理想实验,提出了惯性的概念,从而奠定了牛顿力学的基础。早期物理学家关于惯性有下列说法,其中正确的是 \xzanswer{AD} 

\fourchoices
{物体抵抗运动状态变化的性质是惯性}
{没有力作用,物体只能处于静止状态}
{行星在圆周轨道上保持匀速率运动的性质是惯性}
{运动物体如果没有受到力的作用,将继续以同一速度沿同一直线运动}

\item 
\exwhere{$ 2012 $年上海卷}
已知两个共点力的合力为$ 50 \ N $,分力$ F_{1} $的方向与合力$ F $的方向成$ 30 ^{ \circ } $角,分力$ F_{2} $的大小为$ 30 \ N $。则 \xzanswer{C} 

\fourchoices
{$ F_{1} $的大小是唯一的}
{$ F_{2} $的力向是唯一的}
{$ F_{2} $有两个可能的方向}
{$ F_{2} $可取任意方向}



\item 
\exwhere{$ 2012 $年物理海南卷}
根据牛顿第二定律,下列叙述正确的是 \xzanswer{D} 

\fourchoices
{物体加速度的大小跟它的质量和速度大小的乘积成反比}
{.物体所受合外力必须达到一定值时,才能使物体产生加速度}
{物体加速度的大小跟它的所受作用力中的任一个的大小成正比}
{当物体质量改变但其所受合力的水平分力不变时,物体水平加速度大小与其质量成反比}

\item 
\exwhere{$ 2012 $年物理海南卷}
下列关于摩擦力的说法,正确的是 \xzanswer{CD} 

\fourchoices
{作用在物体上的滑动摩擦力只能使物体减速,不可能使物体加速}
{作用在物体上的静摩擦力只能使物体加速,不可能使物体减速}
{作用在物体上的滑动摩擦力既可能使物体减速,也可能使物体加速}
{作用在物体上的静摩擦力既可能使物体加速,也可能使物体减速}

\item 
\exwhere{$ 2011 $年理综广东卷}
如图所示的水平面上,橡皮绳一端固定,另一端连接两根弹簧,
连接点$ P $在$ F_{1} $、$ F_{2} $和$ F_{3} $三力作用下保持静止。下列判断正确的是 \xzanswer{B} 
\begin{figure}[h!]
\centering
\includesvg[width=0.23\linewidth]{picture/svg/410}
\end{figure}

\fourchoices
{$ F _ { 1 } > F _ { 2 } > F _ { 3 } $}
{$ F _ { 3 } > F _ { 1 } > F _ { 2 } $}
{$ F _ { 2 } > F _ { 3 } > F _ { 1 } $}
{$ F _ { 3 } > F _ { 2 } > F _ { 1 } $}


\item 
\exwhere{$ 2011 $年海南卷}
如图,粗糙的水平地面上有一斜劈,斜劈上一物块正在沿斜面以速度$ v_{0} $匀速下滑,斜劈保持静止,则地面对斜劈的摩擦力 \xzanswer{A} 
\begin{figure}[h!]
\centering
\includesvg[width=0.23\linewidth]{picture/svg/411}
\end{figure}

\fourchoices
{等于零}
{不为零,方向向右}
{不为零,方向向左}
{不为零,$ v_{0} $较大时方向向左,$ v_{0} $较小时方向向右}



\item 
\exwhere{$ 2015 $年上海卷}
如图,鸟沿虚线斜向上加速飞行,空气对其作用力可能是 \xzanswer{B} 
\begin{figure}[h!]
\centering
\includesvg[width=0.23\linewidth]{picture/svg/412}
\end{figure}

\fourchoices
{$ F_{1} $}
{$ F_{2} $}
{$ F_{3} $}
{$ F_{4} $}


\item 
\exwhere{$ 2016 $年江苏卷}
一轻质弹簧原长为$ 8 $ $ cm $,在$ 4 $ $ N $的拉力作用下伸长了$ 2 $ $ cm $,弹簧未超出弹性限度,则该弹簧的劲度系数为 \xzanswer{D} 


\fourchoices
{$ 40 $ $ m/N $ }
{$ 40 $ $ N/m $}
{$ 200 $ $ m/N $ }
{$ 200 $ $ N/m $}


\item 
\exwhere{$ 2017 $年浙江选考卷}
重力为$ G $的体操运动员在进行自由体操比赛时,运动员竖直倒立保持静止状态,两手臂对称支撑,夹角为$ \theta $,则 \xzanswer{A} 


\fourchoices
{当时,运动员单手对地面的正压力大小为}
{当时,运动员单手对地面的正压力大小为}
{当$ \theta $不同时,运动员受到的合力不同}
{当$ \theta $不同时,运动员与地面之间的相互作用力不相等}


\item 
\exwhere{$ 2018 $年天津卷}
明朝谢肇淛《五杂组》中记载:“明姑苏虎丘寺庙倾侧,议欲正之,非万缗不可。一游僧见之,曰:无烦也,我能正之。”游僧每天将木楔从塔身倾斜一侧的砖缝间敲进去,经月余扶正了塔身。假设所用的木楔为等腰三角形,木楔的顶角为$ \theta $,现在木楔背上加一力$ F $,方向如图所示,木楔两侧产生推力$ F_{N} $,则 \xzanswer{BC} 
\begin{figure}[h!]
\centering
\includesvg[width=0.23\linewidth]{picture/svg/413}
\end{figure}

\fourchoices
{若$ F $一定,$ \theta $大时$ F_{N} $大}
{若$ F $一定,$ \theta $小时$ F_{N} $大}
{若$ \theta $一定,$ F $大时$ F_{N} $大}
{若$ \theta $一定,$ F $小时$ F_{N} $大}

\item 
\exwhere{$ 2017 $年浙江选考卷}
重型自卸车利用液压装置使车厢缓慢倾斜到一定角度,车厢上的石块就会自动滑下,以下说法正确的是 \xzanswer{C} 
\begin{figure}[h!]
\centering
\includesvg[width=0.23\linewidth]{picture/svg/414}
\end{figure}


\fourchoices
{在石块下滑前后,自卸车与石块整体的重心位置不变}
{自卸车车厢倾角越大,石块与车厢的动摩擦因数越小}
{自卸车车厢倾角变大,车厢与石块间的正压力减小}
{石块开始下滑时,受到的摩擦力大于重力沿斜面方向的分力}


\item 
\exwhere{$ 2016 $年海南卷}
如图,在水平桌面上放置一斜面体$ P $,两长方体物块$ a $和$ b $叠放在$ P $的斜面上,整个系统处于静止状态。若将$ a $和$ b $、$ b $与$ P $、$ P $与桌面之间摩擦力的大小分别用$ f_{1} $、$ f_{2} $和$ f_{3} $表示。则 \xzanswer{C} 
\begin{figure}[h!]
\centering
\includesvg[width=0.23\linewidth]{picture/svg/415}
\end{figure}


\fourchoices
{$f _ { 1 } = 0 , f _ { 2 } \neq 0 , f _ { 3 } \neq 0$}
{$f _ { 1 } \neq 0 , f _ { 2 } = 0 , f _ { 3 } = 0$}
{$f _ { 1 } \neq 0 , f _ { 2 } \neq 0 , f _ { 3 } = 0$}
{$f _ { 1 } \neq 0 , f _ { 2 } \neq 0 , f _ { 3 } \neq 0$}


\item 
\exwhere{$ 2011 $年理综福建卷}
如图所示,绷紧的水平传送带始终以恒定速率$ v_{1} $运行。初速度大小为$ v_{2} $的小物块从与传送带等高的光滑水平地面上的$ A $处滑上传送带。若从小物块滑上传送带开始计时,小物块在传送带上运动的$ v-t $图象(以地面为参考系)如图乙所示。已知$ v_2>v_1 $,则 \xzanswer{B} 
\begin{figure}[h!]
\centering
\includesvg[width=0.23\linewidth]{picture/svg/416}
\end{figure}


\fourchoices
{$ t_{2} $时刻,小物块离$ A $处的距离达到最大}
{$ t_{2} $时刻,小物块相对传送带滑动的距离达到最大}
{$ 0 \sim t2 $时间内,小物块受到的摩擦力方向先向右后向左}
{$ 0 \sim t3 $时间内,小物块始终受到大小不变的摩擦力作用}


\item 
\exwhere{$ 2011 $年理综山东卷}
如图所示,将两相同的木块$ a $、$ b $置于粗糙的水平地面上,中间用一轻弹簧连接,两侧用细绳固定于墙壁。开始时$ a $、$ b $均静止。弹簧处于伸长状态,两细绳均有拉力,$ a $所受摩擦力$ F_{fa} \neq 0 $,$ b $所受摩擦力$ F_{fb}=0 $,现将右侧细绳剪断,则剪断瞬间 \xzanswer{AD} 
\begin{figure}[h!]
\centering
\includesvg[width=0.23\linewidth]{picture/svg/417}
\end{figure}


\fourchoices
{$ F_{fa} $大小不变 }
{$ F_{fa} $方向改变}
{$ F_{fb} $仍然为零 }
{$ F_{fb} $方向向右}

\item 
\exwhere{$ 2011 $年理综浙江卷}
如图所示,甲、已两人在冰面上“拔河”。两人中间位置处有一分界线,约定先使对方过分界线者为赢。若绳子质量不计,冰面可看成光滑,则下列说法正确的是 \xzanswer{C} 
\begin{figure}[h!]
\centering
\includesvg[width=0.23\linewidth]{picture/svg/418}
\end{figure}


\fourchoices
{甲对绳的拉力与绳对甲的拉力是一对平衡力}
{甲对绳的拉力与乙对绳的拉力是作用力与反作用力}
{若甲的质量比乙大,则甲能赢得“拔河”比赛的胜利}
{若乙收绳的速度比甲快,则乙能赢得“拔河”比赛的胜利}

\item 
\exwhere{$ 2011 $年江苏卷}
如图所示,倾角为$ \alpha $的等腰三角形斜面固定在水平面上,一足够长的轻质绸带跨过斜面的顶端铺放在斜面的两侧,绸带与斜面间无摩擦。现将质量分别为$ M $、$ m $($ M>m $)的小物块同时轻放在斜面两侧的绸带上。两物块与绸带间的动摩擦因数相等,且最大静摩擦力与滑动摩擦力大小相等。在$ \alpha $角取不同值的情况下,下列说法正确的有 \xzanswer{AC} 
\begin{figure}[h!]
\centering
\includesvg[width=0.23\linewidth]{picture/svg/419}
\end{figure}


\fourchoices
{两物块所受摩擦力的大小总是相等}
{两物块不可能同时相对绸带静止}
{$ M $不可能相对绸带发生滑动}
{$ m $不可能相对斜面向上滑动}

\item 
\exwhere{$ 2013 $年重庆卷}
图$ 1 $为伽利略研究自由落体运动实验的示意图,让小球由倾角为$ \theta $的光滑斜面滑下,然后在不同的$ \theta $角条件下进行多次实验,最后推理出自由落体运动是一种匀加速直线运动。分析该实验可知,小球对斜面的压力、小球运动的加速度和重力加速度与各自最大值的比值$ y $随$ \theta $变化的图像分别对应图$ 2 $中的 \xzanswer{B} 
\begin{figure}[h!]
\centering
\includesvg[width=0.33\linewidth]{picture/svg/421}
\end{figure}


\fourchoices
{①、②和③ }
{③、②和①}
{②、③和① }
{③、①和②}

\item 
\exwhere{$ 2013 $年山东卷}
伽利略开创了实验研究和逻辑推理相结合探索自然规律的科学方法,利用这种方法伽利略发现的规律有 \xzanswer{AC} 


\fourchoices
{力不是维持物体运动的原因}
{物体之间普遍存在相互吸引力}
{忽略空气阻力,重物与轻物下落得同样快}
{物体间的相互作用力总是大小相等,方向相反}


\item 
\exwhere{$ 2013 $年广东卷}
如图,物体$ P $静止于固定的斜面上,$ P $的上表面水平。现把物体$ Q $轻轻地叠放在$ P $上,则 \xzanswer{BD} 
\begin{figure}[h!]
\centering
\includesvg[width=0.23\linewidth]{picture/svg/422}
\end{figure}


\fourchoices
{$ P $向下滑动}
{$ P $静止不动}
{$ P $所受的合外力增大}
{$ P $与斜面间的静摩擦力增大}


\item 
\exwhere{$ 2013 $年浙江卷}
如图所示,总质量为$ 460 \ kg $的热气球,从地面刚开始竖直上升时的加速度为$ 0.5 \ m/s ^{2} $,当热气球上升到$ 180 \ m $时,以$ 5 \ m/s $的速度向上匀速运动。若离开地面后热气球所受浮力保持不变,上升过程中热气球总质量不变,重力加速度$ g=10 \ \ m/s ^{2} $。关于热气球,下列说法正确的是 \xzanswer{AD} 
\begin{figure}[h!]
\centering
\includesvg[width=0.23\linewidth]{picture/svg/423}
\end{figure}

\fourchoices
{所受浮力大小为$ 4830 \ N $}
{加速上升过程中所受空气阻力保持不变}
{从地面开始上升$ 10 \ s $后的速度大小为$ 5 \ m/s $}
{以$ 5 \ m/s $匀速上升时所受空气阻力大小为$ 230 \ N $}


\item 
\exwhere{$ 2014 $年理综北京卷}
伽利略创造的把实验、假设和逻辑推理相结合的科学方法,有力地促进了人类科学认识的发展。利用如图所示的装置做如下实验$ : $小球从左侧斜面上的$ O $点由静止释放后沿斜面向下运动,并沿右侧斜面上升。斜面上先后铺垫三种粗糙程度逐渐减低的材料时,小球沿右侧斜面上升到的最高位置依次为$ 1 $、$ 2 $、$ 3 $。根据三次实验结果的对比,可以得到的最直接的结论是 \xzanswer{A} 
\begin{figure}[h!]
\centering
\includesvg[width=0.23\linewidth]{picture/svg/424}
\end{figure}


\fourchoices
{如果斜面光滑,小球将上升到与$ O $点等高的位置}
{如果小球不受力,它将一直保持匀速运动或静止状态}
{如果小球受到力的作用,它的运动状态将发生改变}
{小球受到的力一定时,质量越大,它的加速度越小}



\item 
\exwhere{$ 2014 $年物理江苏卷}
如图所示,$ A $、$ B $ 两物块的质量分别为 $ 2 $ $ m $ 和 $ m $, 静止叠放在水平地面上。 $ A $、$ B $ 间的动摩擦因数为$ \mu $,$ B $ 与地面间的动摩擦因数为$ \frac{ 1 }{ 2 } μ $, 最大静摩擦力等于滑动摩擦力,重力加速度为 $ g $. 现对 $ A $ 施加一水平拉力 $ F $,则 \xzanswer{BCD} 
\begin{figure}[h!]
\centering
\includesvg[width=0.23\linewidth]{picture/svg/425}
\end{figure}


\fourchoices
{当 $ F <2 \mu mg $ 时,,$ A $、$ B $ 都相对地面静止}
{当 $ F = \frac{ 5 }{ 2 } \mu mg $ 时, $ A $ 的加速度为$ \frac{ 1 }{ 3 } \mu g $}
{当 $ F >3 \mu mg $ 时,$ A $ 相对 $ B $ 滑动}
{无论 $ F $ 为何值,$ B $ 的加速度不会超过$ \frac{ 1 }{ 2 } \mu g $}


\item 
\exwhere{$ 2014 $年理综山东卷}
如图,用两根等长轻绳将木板悬挂在竖直木桩上等高的两点,制成一简易秋千。某次维修时将两轻绳各剪去一小段,但仍保持等长且悬挂点不变。木板静止时,$ F_{1} $表示木板所受合力的大小,$ F_{2} $表示单根轻绳对木板拉力的大小,则维修后 \xzanswer{A} 
\begin{figure}[h!]
\centering
\includesvg[width=0.23\linewidth]{picture/svg/426}
\end{figure}


\fourchoices
{$ F_{1} $不变,$ F_{2} $变大 }
{$ F_{1} $不变,$ F_{2} $变小}
{$ F_{1} $变大,$ F_{2} $变大}
{$ F_{1} $变小,$ F_{2} $变小}



\item 
\exwhere{$ 2014 $年理综山东卷}
一质点在外力作用下做直线运动,其速度$ v $随时间$ t $变化的图像如图。在图中标出的时刻中,质点所受合外力的方向与速度方向相同的有 \xzanswer{AC} 
\begin{figure}[h!]
\centering
\includesvg[width=0.23\linewidth]{picture/svg/427}
\end{figure}


\fourchoices
{$ t_{1} $}
{$ t_{2} $}
{$ t_{3} $}
{$ t_{4} $}




\item 
\exwhere{$ 2014 $年理综广东卷}
如图所示,水平地面上堆放着原木,关于原木$ P $在支撑点$ M $、$ N $处受力的方向,下列说法正确的是 \xzanswer{A} 
\begin{figure}[h!]
\centering
\includesvg[width=0.23\linewidth]{picture/svg/428}
\end{figure}


\fourchoices
{$ M $处受到的支持力竖直向上}
{$ N $处受到的支持力竖直向上}
{$ M $处受到的摩擦力沿$ MN $方向}
{$ N $处受到的摩擦力沿水平方向}

\item 
\exwhere{$ 2015 $年广东卷}
如图所示,三条绳子的一端都系在细直杆顶端,另一端都固定在水平面上,将杆竖直紧压在地面上,若三条绳长度不同,下列说法正确的有 \xzanswer{BC} 
\begin{figure}[h!]
\centering
\includesvg[width=0.23\linewidth]{picture/svg/429}
\end{figure}

\fourchoices
{三条绳中的张力都相等}
{杆对地面的压力大于自身重力}
{绳子对杆的拉力在水平方向的合力为零}
{绳子拉力的合力与杆的重力是一对平衡力}








\end{enumerate}



