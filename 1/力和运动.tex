\bta{力和运动}

\begin{enumerate}[leftmargin=0em]
\renewcommand{\labelenumi}{\arabic{enumi}.}
% A(\Alph) a(\alph) I(\Roman) i(\roman) 1(\arabic)
%设定全局标号series=example	%引用全局变量resume=example
%[topsep=-0.3em,parsep=-0.3em,itemsep=-0.3em,partopsep=-0.3em]
%可使用leftmargin调整列表环境左边的空白长度 [leftmargin=0em]


\item 
\exwhere{$ 2016 $年江苏卷}
如图所示,一只猫在桌边猛地将桌布从鱼缸下拉出,鱼缸最终没有滑出桌面.若鱼缸、桌布、桌面两两之间的动摩擦因数均相等,则在上述过程中 \xzanswer{BD}
\begin{figure}[h!]
\centering
\includesvg[width=0.23\linewidth]{picture/svg/506}
\end{figure}

\fourchoices
{桌布对鱼缸摩擦力的方向向左}
{鱼缸在桌布上的滑动时间和在桌面上的相等}
{若猫增大拉力,鱼缸受到的摩擦力将增大}
{若猫减小拉力,鱼缸有可能滑出桌面}

\item 
\exwhere{$ 2019 $年物理全国卷}
如图($ a $),物块和木板叠放在实验台上,物块用一不可伸长的细绳与固定在实验台上的力传感器相连,细绳水平。$ t=0 $时,木板开始受到水平外力$ F $的作用,在$ t=4\ s $时撤去外力。细绳对物块的拉力$ f $随时间$ t $变化的关系如图($ b $)所示,木板的速度$ v $与时间$ t $的关系如图($ c $)所示。木板与实验台之间的摩擦可以忽略。重力加速度取$ g=10 \ m/s^{2} $。由题给数据可以得出 \xzanswer{AB}
\begin{figure}[h!]
\centering
\includesvg[width=0.63\linewidth]{picture/svg/496}
\end{figure}


\fourchoices
{木板的质量为$ 1 \ kg $}
{$ 2\ s \sim 4\ s $内,力$ F $的大小为$ 0.4 \ N $}
{$ 0 \sim 2\ s $内,力$ F $的大小保持不变}
{物块与木板之间的动摩擦因数为$ 0.2 $}

\item 
\exwhere{$ 2013 $年海南卷}
一质点受多个力的作用,处于静止状态,现使其中一个力的大小逐渐减小到零,再沿原方向逐渐恢复到原来的大小。在此过程中,其它力保持不变,则质点的加速度大小$ a $和速度大小$ v $的变化情况是 \xzanswer{C}

\fourchoices
{$ a $和$ v $都始终增大}
{$ a $和$ v $都先增大后减小}
{$ a $先增大后减小,$ v $始终增大}
{$ a $和$ v $都先减小后增大}


\newpage	

\item
\exwhere{$ 2019 $年$ 4 $月浙江物理选考}
小明以初速度$ v_0=10 \ m/s $竖直向上抛出一个质量$ m=0.1 \ kg $的小皮球,最后在抛出点接住。假设小皮球在空气中所受阻力大小为重力的$ 0.1 $倍。求小皮球:
\begin{enumerate}
\renewcommand{\labelenumi}{\arabic{enumi}.}
% A(\Alph) a(\alph) I(\Roman) i(\roman) 1(\arabic)
%设定全局标号series=example	%引用全局变量resume=example
%[topsep=-0.3em,parsep=-0.3em,itemsep=-0.3em,partopsep=-0.3em]
%可使用leftmargin调整列表环境左边的空白长度 [leftmargin=0em]
\item
上升的最大高度;
\item 
从抛出到接住的过程中重力和空气阻力所做的功;
\item 
上升和下降的时间。

\end{enumerate}

\banswer{
\begin{enumerate}
\renewcommand{\labelenumi}{\arabic{enumi}.}
% A(\Alph) a(\alph) I(\Roman) i(\roman) 1(\arabic)
%设定全局标号series=example	%引用全局变量resume=example
%[topsep=-0.3em,parsep=-0.3em,itemsep=-0.3em,partopsep=-0.3em]
%可使用leftmargin调整列表环境左边的空白长度 [leftmargin=0em]
\item
$\frac { 50 } { 11 } m$
\item 
$0 ; - \frac { 10 } { 11 };$
\item 
$\frac { 10 } { 11 } s , \frac { 10 \sqrt { 11 } } { 33 } s$

\end{enumerate}


}

\item 
\exwhere{$ 2013 $年新课标 \lmd{2} 卷}
一长木板在水平地面上运动,在$ t=0 $时刻将一相对于地面静止的物块轻放到木板上,以后木板运动的速度-时间图像如图所示。己知物块与木板的质量相等,物块与木板间及木板与地面间均有摩擦。物块与木板间的最大静摩擦力等于滑动摩擦力,且物块始终在木板上。取重力加速度的大小$ g = 10 \ m/s^{2} $求:
\begin{enumerate}
\renewcommand{\labelenumii}{(\arabic{enumii})}

\item 
物块与木板间、木板与地面间的动摩擦因数;

\item 
从$ t=0 $时刻到物块与木板均停止运动时,物块相对于木板的位移的大小。

\end{enumerate}
\begin{figure}[h!]
\flushright 
\includesvg[width=0.17\linewidth]{picture/svg/505}
\end{figure}

\banswer{
\begin{enumerate}
\renewcommand{\labelenumi}{\arabic{enumi}.}
% A(\Alph) a(\alph) I(\Roman) i(\roman) 1(\arabic)
%设定全局标号series=example	%引用全局变量resume=example
%[topsep=-0.3em,parsep=-0.3em,itemsep=-0.3em,partopsep=-0.3em]
%可使用leftmargin调整列表环境左边的空白长度 [leftmargin=0em]
\item
物块与木板$ \mu_{1}=0.20 $ ,木板与地面$ \mu_{2}=0.30 $
\item 
$ s=1.125\ m $


\end{enumerate}


}

\newpage

\item 
\exwhere{$ 2019 $年物理江苏卷}
如图所示,质量相等的物块$ A $和$ B $叠放在水平地面上,左边缘对齐.$ A $与$ B $、$ B $与地面间的动摩擦因数均为$ \mu $。先敲击$ A $,$ A $立即获得水平向右的初速度,在$ B $上滑动距离$ L $后停下。接着敲击$ B $,$ B $立即获得水平向右的初速度,$ A $、$ B $都向右运动,左边缘再次对齐时恰好相对静止,此后两者一起运动至停下.最大静摩擦力等于滑动摩擦力,重力加速度为$ g $.求:
\begin{enumerate}
\renewcommand{\labelenumi}{\arabic{enumi}.}
% A(\Alph) a(\alph) I(\Roman) i(\roman) 1(\arabic)
%设定全局标号series=example	%引用全局变量resume=example
%[topsep=-0.3em,parsep=-0.3em,itemsep=-0.3em,partopsep=-0.3em]
%可使用leftmargin调整列表环境左边的空白长度 [leftmargin=0em]
\item
$ A $被敲击后获得的初速度大小$ v_A $;
\item 
在左边缘再次对齐的前、后,$ B $运动加速度的大小$ a_B $、$ a_B $;
\item 
$ B $被敲击后获得的初速度大小$ v_B $.
\end{enumerate}
\begin{figure}[h!]
\flushright
\includesvg[width=0.3\linewidth]{picture/svg/497}
\end{figure}

\banswer{
\begin{enumerate}
\renewcommand{\labelenumi}{\arabic{enumi}.}
% A(\Alph) a(\alph) I(\Roman) i(\roman) 1(\arabic)
%设定全局标号series=example	%引用全局变量resume=example
%[topsep=-0.3em,parsep=-0.3em,itemsep=-0.3em,partopsep=-0.3em]
%可使用leftmargin调整列表环境左边的空白长度 [leftmargin=0em]
\item
$v _ { A } = \sqrt { 2 \mu g L }$
\item 
$a _ { B } = 3 \mu g , \quad a _ { B } ^ { \prime } = \mu g$
\item 
$v _ { B } = 2 \sqrt { 2 \mu g L }$


\end{enumerate}


}





\item 
\exwhere{$ 2012 $年理综浙江卷}
为了研究鱼所受水的阻力与其形状的关系,小明同学用石腊做成两条质量均为$ m $、形
状不同的“$ A $鱼”和“$ B $鱼”,如图所示。在高出水面$ H $处分别静止释放“$ A $鱼”和“$ B $鱼”,“$ A $鱼”竖直下潜$ h_A $后速度减为零,“$ B $鱼”竖直下潜$ h_B $后速度减为零。“鱼”在水中运动时,除受重力外,还受浮力和水的阻力。已知“鱼”在水中所受浮力是其重力的$ \frac{10}{9} $倍,重力加速度为$ g $,“鱼”运动的位移值远大于“鱼”的长度。假设“鱼”运动时所受水的阻力恒定,空气阻力不计。求:
\begin{enumerate}
\renewcommand{\labelenumii}{(\arabic{enumii})}
\item 

“$ A $鱼”入水瞬间的速度$ v_{A1} $;


\item 
“$ A $鱼”在水中运动时所受阻力$ f_A $;


\item 
“$ A $鱼”与“$ B $鱼”在水中运动时所受阻力之比$ f_A : f_B $。

\end{enumerate}
\begin{figure}[h!]
\flushright
\includesvg[width=0.2\linewidth]{picture/svg/498}
\end{figure}

\banswer{
\begin{enumerate}
\renewcommand{\labelenumi}{\arabic{enumi}.}
% A(\Alph) a(\alph) I(\Roman) i(\roman) 1(\arabic)
%设定全局标号series=example	%引用全局变量resume=example
%[topsep=-0.3em,parsep=-0.3em,itemsep=-0.3em,partopsep=-0.3em]
%可使用leftmargin调整列表环境左边的空白长度 [leftmargin=0em]
\item
$v _ { A 1 } = \sqrt { 2 g H }$
\item 
$f _ { A } = m g \left( \frac { H } { h _ { A } } - \frac { 1 } { 9 } \right)$
\item 
$f _ { A }: f _ { B } = \frac { \left( 9 H - h _ { A } \right) h _ { B } } { \left( 9 H - h _ { B } \right) h _ { A } }$

\end{enumerate}


}


\newpage	
\item 
\exwhere{$ 2012 $年理综重庆卷}
某校举行托乒乓球跑步比赛,赛道为水平直道,比赛距离为$ S $,比赛时,某同学将球置于球拍中心,以大小为$ a $的加速度从静止开始做匀加速直线运动,当速度达到$ v_{0} $时,再以$ v_0 $做匀速直线运动跑至终点。整个过程中球一直保持在球拍中心不动。比赛中,该同学在匀速直线运动阶段保持球拍的倾角为$ \theta _0 $ ,如图所示。设球在运动过程中受到的空气阻力大小与其速度大小成正比,方向与运动方向相反,不计球与球拍之间的摩擦,球的质量为$ m $,重力加速度为$ g $.
\begin{enumerate}
\renewcommand{\labelenumi}{\arabic{enumi}.}
% A(\Alph) a(\alph) I(\Roman) i(\roman) 1(\arabic)
%设定全局标号series=example	%引用全局变量resume=example
%[topsep=-0.3em,parsep=-0.3em,itemsep=-0.3em,partopsep=-0.3em]
%可使用leftmargin调整列表环境左边的空白长度 [leftmargin=0em]
\item
空气阻力大小与球速大小的比例系数$ k $;
\item 
求在加速跑阶段球拍倾角$ \theta $随球速$ v $变化的关系式;
\item 
整个匀速跑阶段,若该同学速率仍为$ v_{0} $ ,而球拍的倾角比$ \theta _0 $大了$ \beta $并保持不变,不计球在球拍上的移动引起的空气阻力的变化,为保证到达终点前球不从球拍上距离中心为$ r $的下边沿掉落,求$ \beta $应满足的条件。

\end{enumerate}
\begin{figure}[h!]
\flushright
\includesvg[width=0.25\linewidth]{picture/svg/499}
\end{figure}


\banswer{
\begin{enumerate}
\renewcommand{\labelenumi}{\arabic{enumi}.}
% A(\Alph) a(\alph) I(\Roman) i(\roman) 1(\arabic)
%设定全局标号series=example	%引用全局变量resume=example
%[topsep=-0.3em,parsep=-0.3em,itemsep=-0.3em,partopsep=-0.3em]
%可使用leftmargin调整列表环境左边的空白长度 [leftmargin=0em]
\item
$k = m g \tan \theta _ { 0 } / v _ { 0 }$
\item 
$\tan \theta = a / g + v \tan \theta _ { 0 } / v _ { 0 }$
\item 
$\sin \beta \leq \frac { 2 r \cos \theta _ { 0 } } { g \left( \frac { s } { v _ { 0 } } - \frac { v _ { 0 } } { 2 a } \right) ^ { 2 } }$



\end{enumerate}


}



\item 
\exwhere{$ 2011 $年上海卷}
如图,质量$ m=2 \ kg $的物体静止于水平地面的$ A $处,$ A $、$ B $间距$ L=20\ m $。用大小为$ 30 \ N $,沿水平方向的外力拉此物体,经$ t_0=2\ s $拉至$ B $处。(已知$ \cos 37 ^{\circ} =0.8 $,$ \sin 37 ^{\circ} =0.6 $。取$ g=10 \ m/s^{2} $).
\begin{enumerate}
\renewcommand{\labelenumi}{\arabic{enumi}.}
% A(\Alph) a(\alph) I(\Roman) i(\roman) 1(\arabic)
%设定全局标号series=example	%引用全局变量resume=example
%[topsep=-0.3em,parsep=-0.3em,itemsep=-0.3em,partopsep=-0.3em]
%可使用leftmargin调整列表环境左边的空白长度 [leftmargin=0em]
\item
求物体与地面间的动摩擦因数$ \mu $;
\item 
用大小为$ 30 \ N $,与水平方向成$ 37 ^{ \circ } $的力斜向上拉此物体,使物体从$ A $处由静止开始运动并能到达$ B $处,求该力作用的最短时间$ t $。


\end{enumerate}
\begin{figure}[h!]
\flushright
\includesvg[width=0.4\linewidth]{picture/svg/500}
\end{figure}

\banswer{
\begin{enumerate}
\renewcommand{\labelenumi}{\arabic{enumi}.}
% A(\Alph) a(\alph) I(\Roman) i(\roman) 1(\arabic)
%设定全局标号series=example	%引用全局变量resume=example
%[topsep=-0.3em,parsep=-0.3em,itemsep=-0.3em,partopsep=-0.3em]
%可使用leftmargin调整列表环境左边的空白长度 [leftmargin=0em]
\item
$\mu = \frac { f } { m g } = \frac { 10 } { 2 \times 10 } = 0.5$
\item 
$t = \sqrt { \frac { 2 s } { a } } = \sqrt { \frac { 2 \times 6.06 } { 11.5 } } = 1.03 ( s )$

\end{enumerate}


}


\newpage
\item 
\exwhere{$ 2014 $年理综山东卷}
研究表明,一般人的刹车反应时间(即图甲中“反应过程”所用时间)$ t_{0}=0.4\ s $,但饮酒会导致反应时间延长,在某次试验中,志愿者少量饮酒后驾车以$ v_0=72 \ km/h $的速度在试验场的水平路面上匀速行驶,从发现情况到汽车停止,行驶距离$ L=39 \ m $。减速过程中汽车位移$ s $与速度$ v $的关系曲线如图乙所示,此过程可视为匀变速直线运动。取重力加速度的大小$ g=10 \ m/s^{2} $。求:
\begin{enumerate}
\renewcommand{\labelenumi}{\arabic{enumi}.}
% A(\Alph) a(\alph) I(\Roman) i(\roman) 1(\arabic)
%设定全局标号series=example	%引用全局变量resume=example
%[topsep=-0.3em,parsep=-0.3em,itemsep=-0.3em,partopsep=-0.3em]
%可使用leftmargin调整列表环境左边的空白长度 [leftmargin=0em]
\item
减速过程汽车加速度的大小及所用时间;
\item 
饮酒使志愿者的反应时间比一般人增加了多少?
\item 
减速过程汽车对志愿者作用力的大小与志愿者重力大小的比值。

\end{enumerate}
\begin{figure}[h!]
\flushright
\includesvg[width=0.7\linewidth]{picture/svg/501}
\end{figure}


\banswer{
\begin{enumerate}
\renewcommand{\labelenumi}{\arabic{enumi}.}
% A(\Alph) a(\alph) I(\Roman) i(\roman) 1(\arabic)
%设定全局标号series=example	%引用全局变量resume=example
%[topsep=-0.3em,parsep=-0.3em,itemsep=-0.3em,partopsep=-0.3em]
%可使用leftmargin调整列表环境左边的空白长度 [leftmargin=0em]
\item
$ a=8\ m/s^{2} $
\item 
$ \Delta t=0.3\ s $
\item 
$\frac { F _ { 0 } } { m g } = \frac { \sqrt { 41 } } { 5 }$

\end{enumerate}


}



\item 
\exwhere{$ 2013 $年江苏卷}
如图所示,将小砝码置于桌面上的薄纸板上,用水平向右的拉力将纸板迅速抽出, 砝码的移动很小,几乎观察不到,这就是大家熟悉的惯性演示实验. 若砝码和纸板的质量分别为$ m_{1} $ 和$ m_{2} $,各接触面间的动摩擦因数均为$ \mu $. 重力加速度为$ g $.
\begin{enumerate}
\renewcommand{\labelenumi}{\arabic{enumi}.}
% A(\Alph) a(\alph) I(\Roman) i(\roman) 1(\arabic)
%设定全局标号series=example	%引用全局变量resume=example
%[topsep=-0.3em,parsep=-0.3em,itemsep=-0.3em,partopsep=-0.3em]
%可使用leftmargin调整列表环境左边的空白长度 [leftmargin=0em]
\item
当纸板相对砝码运动时,求纸板所受摩擦力大小;
\item 
要使纸板相对砝码运动,求所需拉力的大小;
\item 
本实验中,$ m_{1} =0.5 \ kg $, $ m_{2} =0.1 \ kg $, $ \mu =0.2 $,砝码与纸板左端的距离$ d =0.1 $ $ m $,取$ g =10 \ m/s^{2} $. 若砝码移动的距离超过$ l=0.002 $ $ m $,人眼就能感知.为确保实验成功,纸板所需的拉力至少多大? 

\end{enumerate}
\begin{figure}[h!]
\flushright
\includesvg[width=0.35\linewidth]{picture/svg/502}
\end{figure}

\banswer{
\begin{enumerate}
\renewcommand{\labelenumi}{\arabic{enumi}.}
% A(\Alph) a(\alph) I(\Roman) i(\roman) 1(\arabic)
%设定全局标号series=example	%引用全局变量resume=example
%[topsep=-0.3em,parsep=-0.3em,itemsep=-0.3em,partopsep=-0.3em]
%可使用leftmargin调整列表环境左边的空白长度 [leftmargin=0em]
\item
$f = \mu \left( 2 m _ { 1 } + m _ { 2 } \right) g$
\item 
$F > 2 \mu \left( m _ { 1 } + m _ { 2 } \right) g$
\item 
$F = 22.4\ N$

\end{enumerate}


}


\newpage
\item 
\exwhere{$ 2013 $年山东卷}
如图所示,一质量$ m=0.4 \ kg $的小物块,以$ v_0=2 \ m/s $的初速度,在与斜面成某一角度的拉力$ F $作用下,沿斜面向上做匀加速运动,经$ t=2\ s $的时间物块由$ A $点运动到$ B $点,$ A $、$ B $之间的距离$ L $=$ 10 \ m $。已知斜面倾角,物块与斜面之间的动摩擦因数$\mu = \frac { \sqrt { 3 } } { 3 }$,重力加速度$ g $取$ 10 \ m/s^{2} $.
\begin{enumerate}
\renewcommand{\labelenumi}{\arabic{enumi}.}
% A(\Alph) a(\alph) I(\Roman) i(\roman) 1(\arabic)
%设定全局标号series=example	%引用全局变量resume=example
%[topsep=-0.3em,parsep=-0.3em,itemsep=-0.3em,partopsep=-0.3em]
%可使用leftmargin调整列表环境左边的空白长度 [leftmargin=0em]
\item
求物块加速度的大小及到达$ B $点时速度的大小;
\item 
拉力$ F $与斜面夹角多大时,拉力$ F $最小?拉力$ F $的最小值是多少?

\end{enumerate}
\begin{figure}[h!]
\flushright
\includesvg[width=0.25\linewidth]{picture/svg/503}
\end{figure}

\banswer{
\begin{enumerate}
\renewcommand{\labelenumi}{\arabic{enumi}.}
% A(\Alph) a(\alph) I(\Roman) i(\roman) 1(\arabic)
%设定全局标号series=example	%引用全局变量resume=example
%[topsep=-0.3em,parsep=-0.3em,itemsep=-0.3em,partopsep=-0.3em]
%可使用leftmargin调整列表环境左边的空白长度 [leftmargin=0em]
\item
$ a=3\ m/s^{2} \qquad v=8\ m/s $
\item 
$F _ { \min } = \frac { 13 \sqrt { 3 } } { 5 } N$

\end{enumerate}


}


\item 
\exwhere{$ 2013 $年四川卷}
近来,我国多个城市开始重点治理“中国式过马路”行为。每年全国由于行人不遵守交通规则而引发的交通事故上万起,死亡上千人。只有科学设置交通管制,人人遵守交通规则,才能保证行人的生命安全。 如右图所示,停车线$ AB $与前方斑马线边界$ CD $间的距离为$ 23 \ m $。质量$ 8\ t $、车长$ 7 \ m $的卡车以$ 54 \ km/h $的速度向北匀速行驶,当车前端刚驶过停车线$ AB $,该车前方的机动车交通信号灯由绿灯变黄灯。
\begin{enumerate}
\renewcommand{\labelenumi}{\arabic{enumi}.}
% A(\Alph) a(\alph) I(\Roman) i(\roman) 1(\arabic)
%设定全局标号series=example	%引用全局变量resume=example
%[topsep=-0.3em,parsep=-0.3em,itemsep=-0.3em,partopsep=-0.3em]
%可使用leftmargin调整列表环境左边的空白长度 [leftmargin=0em]
\item
若此时前方$ C $处人行横道路边等待的行人就抢先过马路,卡车司机发现行人,立即制动,卡车受到的阻力为$ 3 \times 10^4 \ N $,求卡车的制动距离;
\item 
若人人遵守交通规则,该车将不受影响地驶过前方斑马线边界$ CD $。为确保行人安全,$ D $处人行横道信号灯应该在南北向机动车信号灯变黄灯后至少多久变为绿灯?

\end{enumerate}
\begin{figure}[h!]
\flushright
\includesvg[width=0.4\linewidth]{picture/svg/504}
\end{figure}
\banswer{
\begin{enumerate}
\renewcommand{\labelenumi}{\arabic{enumi}.}
% A(\Alph) a(\alph) I(\Roman) i(\roman) 1(\arabic)
%设定全局标号series=example	%引用全局变量resume=example
%[topsep=-0.3em,parsep=-0.3em,itemsep=-0.3em,partopsep=-0.3em]
%可使用leftmargin调整列表环境左边的空白长度 [leftmargin=0em]
\item
$ S_{1}=30 \ m $
\item 
$ \Delta t =2\ s $

\end{enumerate}


}


\newpage





\item 
\exwhere{$ 2016 $年四川卷}
避险车道是避免恶性交通事故的重要设施,由制动坡床和防撞设施等组成,如图竖直平面内,制动坡床视为与水平面夹角为$ \theta $的斜面。一辆长$ 12 $ $ m $的载有货物的货车因刹车失灵从干道驶入制动坡床,当车速为$ 23 $ $ m/s $时,车尾位于制动坡床的底端,货物开始在车厢内向车头滑动,当货物在车厢内滑动了$ 4 $ $ m $时,车头距制动坡床顶端$ 38 $ $ m $,再过一段时间,货车停止。已知货车质量是货物质量的$ 4 $倍,货物与车厢间的动摩擦因数为$ 0.4 $;货车在制动坡床上运动受到的坡床阻力大小为货车和货物总重的$ 0.44 $倍。货物与货车分别视为小滑块和平板,取。求:
\begin{enumerate}
\renewcommand{\labelenumi}{\arabic{enumi}.}
% A(\Alph) a(\alph) I(\Roman) i(\roman) 1(\arabic)
%设定全局标号series=example	%引用全局变量resume=example
%[topsep=-0.3em,parsep=-0.3em,itemsep=-0.3em,partopsep=-0.3em]
%可使用leftmargin调整列表环境左边的空白长度 [leftmargin=0em]
\item
货物在车厢内滑动时加速度的大小和方向;
\item 
制动坡床的长度。

\end{enumerate}		
\begin{figure}[h!]
\flushright
\includesvg[width=0.5\linewidth]{picture/svg/507}
\end{figure}


\banswer{
\begin{enumerate}
\renewcommand{\labelenumi}{\arabic{enumi}.}
% A(\Alph) a(\alph) I(\Roman) i(\roman) 1(\arabic)
%设定全局标号series=example	%引用全局变量resume=example
%[topsep=-0.3em,parsep=-0.3em,itemsep=-0.3em,partopsep=-0.3em]
%可使用leftmargin调整列表环境左边的空白长度 [leftmargin=0em]
\item
$ 5 \ m/s^{2} $,方向沿斜面向下 
\item 
$ 98 \ m $

\end{enumerate}


}


\item 
\exwhere{$ 2017 $年新课标$ \lmd{3} $卷}
如图,两个滑块$ A $和$ B $的质量分别为$ m_A=1 $ $ kg $和$ m_B=5 $ $ kg $,放在静止于水平地面上的木板的两端,两者与木板间的动摩擦因数均为$ \mu_1=0.5 $;木板的质量为$ m=4 $ $ kg $,与地面间的动摩擦因数为$ \mu_2=0.1 $。某时刻$ A $、$ B $两滑块开始相向滑动,初速度大小均为$ v_0=3 $ $ m/s $。$ A $、$ B $相遇时,$ A $与木板恰好相对静止。设最大静摩擦力等于滑动摩擦力,取重力加速度大小$ g=10 $ $ m/s^{2} $。求:
\begin{enumerate}
\renewcommand{\labelenumi}{\arabic{enumi}.}
% A(\Alph) a(\alph) I(\Roman) i(\roman) 1(\arabic)
%设定全局标号series=example	%引用全局变量resume=example
%[topsep=-0.3em,parsep=-0.3em,itemsep=-0.3em,partopsep=-0.3em]
%可使用leftmargin调整列表环境左边的空白长度 [leftmargin=0em]
\item
$ B $与木板相对静止时,木板的速度;
\item 
$ A $、$ B $开始运动时,两者之间的距离。

\end{enumerate}
\begin{figure}[h!]
\flushright
\includesvg[width=0.35\linewidth]{picture/svg/508}
\end{figure}


\banswer{
\begin{enumerate}
\renewcommand{\labelenumi}{\arabic{enumi}.}
% A(\Alph) a(\alph) I(\Roman) i(\roman) 1(\arabic)
%设定全局标号series=example	%引用全局变量resume=example
%[topsep=-0.3em,parsep=-0.3em,itemsep=-0.3em,partopsep=-0.3em]
%可使用leftmargin调整列表环境左边的空白长度 [leftmargin=0em]
\item
$ 1\ m/s $
\item 
$ 1.9\ m $

\end{enumerate}


}










\end{enumerate}

