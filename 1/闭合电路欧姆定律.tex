\bta{闭合电路欧姆定律}
\begin{enumerate}
%\renewcommand{\labelenumi}{\arabic{enumi}.}
% A(\Alph) a(\alph) I(\Roman) i(\roman) 1(\arabic)
%设定全局标号series=example	%引用全局变量resume=example
%[topsep=-0.3em,parsep=-0.3em,itemsep=-0.3em,partopsep=-0.3em]
%可使用leftmargin调整列表环境左边的空白长度 [leftmargin=0em]
\item
\exwhere{$ 2019 $ 年物理江苏卷}
如图所示的电路中,电阻 $ R=2 \ \Omega $.断开 $ S $ 后,电压表的读数为 $ 3 \ V $;闭合 $ S $
后,电压表的读数为 $ 2 \ V $,则电源的内阻 $ r $ 为 \xzanswer{A} 
\begin{figure}[h!]
\centering
\includesvg[width=0.23\linewidth]{picture/svg/GZ-3-tiyou-1106}
\end{figure}


\fourchoices
{$ 1 \ \Omega $}
{$ 2 \ \Omega $}
{$ 3 \ \Omega $}
{$ 4 \ \Omega $}



\item 
\exwhere{$ 2016 $ 年上海卷}
电源电动势反映了电源把其它形式的能量转化为电能的能力,因此 \xzanswer{C} 

\fourchoices
{电动势是一种非静电力}
{电动势越大,表明电源储存的电能越多}
{电动势的大小是非静电力做功能力的反映}
{电动势就是闭合电路中电源两端的电压}


\item 
\exwhere{$ 2012 $ 年物理上海卷}
直流电路如图所示,在滑动变阻器的滑片 $ P $ 向右移动时,电源的 \xzanswer{ABC} 
\begin{figure}[h!]
\centering
\includesvg[width=0.23\linewidth]{picture/svg/GZ-3-tiyou-1107}
\end{figure}


\fourchoices
{总功率一定减小}
{效率一定增大}
{内部损耗功率一定减小}
{输出功率一定先增大后减小}




\item 
\exwhere{$ 2013 $ 年上海卷}
如图,电路中三个电阻 $ Rl $、$ R_{2} $ 和 $ R_{3} $ 的阻值分别为 $ R $、$ 2R $ 和
$ 4R $。当电键 $ S_{1} $ 断开、$ S_{2} $ 闭合时,电源输出功率为 $ P_{0} $;当 $ S_{1} $ 闭合、
$ S_{2} $ 断开时,电源输出功率也为 $ P_{0} $。则电源电动势为 \underlinegap ;当
$ S_{1} $、$ S_{2} $ 都断开时,电源的总功率为 \underlinegap 。
\begin{figure}[h!]
\centering
\includesvg[width=0.23\linewidth]{picture/svg/GZ-3-tiyou-1108}
\end{figure}


\tk{$ E=3 \sqrt{P_{0} R} $ \quad $ P_{0} $} 



\item
\exwhere{$ 2014 $ 年物理上海卷}
将阻值随温度升高而减小的热敏电阻 \lmd{1} 和 \lmd{2} 串联,接在不计内阻的稳压电源两端。开始时 \lmd{1} 和 \lmd{2} 
阻值相等,保持 \lmd{1} 温度不变,冷却或加热 \lmd{2} ,则 \lmd{2} 的电功率在 \xzanswer{C} 

\fourchoices
{加热时变大,冷却时变小}
{加热时变小,冷却时变大}
{加热或冷却时都变小}
{加热或冷却时都变大}



\item 
\exwhere{$ 2016 $ 年江苏卷}
如图所示的电路中,电源电动势为 $ 12 \ V $,内阻为 $ 2 \ \Omega $,四个电阻的阻值已在图
中标出。闭合开关 $ S $,下列说法正确的有 \xzanswer{AC} 
\begin{figure}[h!]
\centering
\includesvg[width=0.23\linewidth]{picture/svg/GZ-3-tiyou-1109}
\end{figure}

\fourchoices
{路端电压为 $ 10 \ V $}
{电源的总功率为 $ 10 \ W $}
{$ a $、$ b $ 间电压的大小为 $ 5 \ V $}
{$ a $、$ b $ 间用导线连接后,电路的总电流为 $ 1 \ A $}


\item 
\exwhere{$ 2018 $ 年海南卷}
如图,三个电阻 $ R_{1} $、$ R_{2} $、$ R_{3} $ 的阻值均为 $ R $,电源的内阻 $ r<R $,$ c $ 为滑动变阻
器的中点。闭合开关后,将滑动变阻器的滑片由 $ c $ 点向 $ a $ 端滑动,
下列说法正确的是 \xzanswer{CD} 
\begin{figure}[h!]
\centering
\includesvg[width=0.23\linewidth]{picture/svg/GZ-3-tiyou-1110}
\end{figure}

\fourchoices
{$ R_{2} $ 消耗的功率变小}
{$ R_{3} $ 消耗的功率变大}
{电源输出的功率变大}
{电源内阻消耗的功率变大}



\item 
\exwhere{$ 2018 $ 年北京卷}
如图 \subref{2018BeiJin01} 所示,用电动势为 $ E $、内阻为 $ r $ 的电源,向滑动变阻器 $ R $ 供电。改变变阻器 $ R $ 的阻值,路端
电压 $ U $ 与电流 $ I $ 均随之变化。
\begin{figure}[h!]
\centering
\begin{subfigure}{0.4\linewidth}
\centering
\includesvg[width=0.7\linewidth]{picture/svg/GZ-3-tiyou-1111} 
\caption{}\label{2018BeiJin01}
\end{subfigure}
\begin{subfigure}{0.4\linewidth}
\centering
\includesvg[width=0.7\linewidth]{picture/svg/GZ-3-tiyou-1112} 
\caption{}\label{2018BeiJin02}
\end{subfigure}
\end{figure}

\begin{enumerate}
%\renewcommand{\labelenumi}{\arabic{enumi}.}
% A(\Alph) a(\alph) I(\Roman) i(\roman) 1(\arabic)
%设定全局标号series=example	%引用全局变量resume=example
%[topsep=-0.3em,parsep=-0.3em,itemsep=-0.3em,partopsep=-0.3em]
%可使用leftmargin调整列表环境左边的空白长度 [leftmargin=0em]
\item
以 $ U $ 为纵坐标,$ I $ 为横坐标,在图 \subref{2018BeiJin02}
中画出变阻器阻值 $ R $ 变化过程中 $ U-I $ 图
像的示意图,并说明 $ U-I $ 图像与两坐标
轴交点的物理意义。




\item 
\begin{enumerate}
%\renewcommand{\labelenumi}{\arabic{enumi}.}
% A(\Alph) a(\alph) I(\Roman) i(\roman) 1(\arabic)
%设定全局标号series=example	%引用全局变量resume=example
%[topsep=-0.3em,parsep=-0.3em,itemsep=-0.3em,partopsep=-0.3em]
%可使用leftmargin调整列表环境左边的空白长度 [leftmargin=0em]
\item
请在图 \subref{2018BeiJin02} 画好的 $ U-I $ 关系图线上任取一点,画出带网格的图形,以其面积表示此时电源
的输出功率;

\item 
请推导该电源对外电路能够输出的最大电功率及条件。


\end{enumerate}


\item 
请写出电源电动势定义式,并结合能量守恒定律证明:电源电动势在数值上等于内、外电路
电势降落之和。



\end{enumerate}

\tk{
\begin{enumerate}
%\renewcommand{\labelenumi}{\arabic{enumi}.}
% A(\Alph) a(\alph) I(\Roman) i(\roman) 1(\arabic)
%设定全局标号series=example	%引用全局变量resume=example
%[topsep=-0.3em,parsep=-0.3em,itemsep=-0.3em,partopsep=-0.3em]
%可使用leftmargin调整列表环境左边的空白长度 [leftmargin=0em]
\item
$ U-I $ 图像如图所示。
\begin{center}
\includesvg[width=0.23\linewidth]{picture/svg/GZ-3-tiyou-1113} 
\end{center}
图像与纵轴交点的坐标值为电源电动势,与横轴交点的坐标值为
短路电流。
\item 
\begin{enumerate}
%\renewcommand{\labelenumi}{\arabic{enumi}.}
% A(\Alph) a(\alph) I(\Roman) i(\roman) 1(\arabic)
%设定全局标号series=example	%引用全局变量resume=example
%[topsep=-0.3em,parsep=-0.3em,itemsep=-0.3em,partopsep=-0.3em]
%可使用leftmargin调整列表环境左边的空白长度 [leftmargin=0em]
\item
如上图所示
\item 
电源输出的电功率
\[ P=I^{2} R=\left(\frac{E}{R+r}\right)^{2} R=\frac{E^{2}}{R+2 r+\frac{r^{2}}{R}} \]
当外电路电阻 $ R=r $ 时,电源输出的电功率最大,为$P_{\max }=\frac{E^{2}}{4 r}$。
\end{enumerate}
\item 
电动势定义式 $\quad E=\frac{W}{q}$,根据能量守恒,在图$ 1 $所示电路中,非静电力做功$ W $产生的电能等于在外电路和内电路产生的电热,即:
$ W=I^{2} r t+I^{2} R t=I r q+I R q $
有$E=I r+I R=U_{\text {内 }}+U_{\text {外 }}$。
\end{enumerate}
} 





\end{enumerate}

