\bta{气体的等温变化}


\begin{enumerate}[leftmargin=0em]
\renewcommand{\labelenumi}{\arabic{enumi}.}
% A(\Alph) a(\alph) I(\Roman) i(\roman) 1(\arabic)
%设定全局标号series=example	%引用全局变量resume=example
%[topsep=-0.3em,parsep=-0.3em,itemsep=-0.3em,partopsep=-0.3em]
%可使用leftmargin调整列表环境左边的空白长度 [leftmargin=0em]
\item
\exwhere{$ 2013 $年上海卷}
已知湖水深度为$ 20 \ m $,湖底水温为$ 4\ ^{ \circ } C $,水面温度为$ 17\ ^{ \circ } C $,大气压强为$ 1.0 \times 10^5\ Pa $。当一气泡从湖底缓慢升到水面时,其体积约为原来的(取$ g=10 \ m/s ^{2} $,$ \rho =1.0 \times 10^3 \ kg/m^{3} ) $ \xzanswer{C} 


\fourchoices
{$ 12.8 $倍}
{$ 8.5 $倍}
{$ 3.1 $倍}
{$ 2.1 $倍}

\item 
\exwhere{$ 2014 $年物理上海卷}
如图,竖直放置、开口向下的试管内用水银封闭一段气体,若试管自由下落,管内气体 \xzanswer{B} 


\begin{minipage}[h!]{0.7\linewidth}
\vspace{0.3em}
\fourchoices
{压强增大,体积增大}
{压强增大,体积减小}
{压强减小,体积增大}
{压强减小,体积减小}

\vspace{0.3em}
\end{minipage}
\hfill
\begin{minipage}[h!]{0.3\linewidth}
\flushright
\vspace{0.3em}
\includesvg[width=0.07\linewidth]{picture/svg/253}
\vspace{0.3em}
\end{minipage}

\item 
\exwhere{$ 2012 $年物理上海卷}
如图,长$ L=100 \ cm $,粗细均匀的玻璃管一端封闭。水平放置时,长$ L_0=50 \ cm $的空气柱被水银封住,水银柱长$ h=30 \ cm $。将玻璃管缓慢地转到开口向下的竖直位置,然后竖直插入水银槽,插入后有 $ \Delta h=15 \ cm $的水银柱进入玻璃管。设整个过程中温度始终保持不变,大气压强$ p_0=75 \ cmHg $。求:
\begin{enumerate}
\renewcommand{\labelenumi}{\arabic{enumi}.}
% A(\Alph) a(\alph) I(\Roman) i(\roman) 1(\arabic)
%设定全局标号series=example	%引用全局变量resume=example
%[topsep=-0.3em,parsep=-0.3em,itemsep=-0.3em,partopsep=-0.3em]
%可使用leftmargin调整列表环境左边的空白长度 [leftmargin=0em]
\item
插入水银槽后管内气体的压强$ p $;
\item 
管口距水银槽液面的距离$ H $。



\end{enumerate}
\begin{figure}[h!]
\flushright
\includesvg[width=0.25\linewidth]{picture/svg/254}
\end{figure}


\banswer{
\begin{enumerate}
\renewcommand{\labelenumi}{\arabic{enumi}.}
% A(\Alph) a(\alph) I(\Roman) i(\roman) 1(\arabic)
%设定全局标号series=example	%引用全局变量resume=example
%[topsep=-0.3em,parsep=-0.3em,itemsep=-0.3em,partopsep=-0.3em]
%可使用leftmargin调整列表环境左边的空白长度 [leftmargin=0em]
\item
插入后压强$p = p _ { 0 } L _ { 0 } / L ^ { \prime } = 62.5 \mathrm { cm } \mathrm { Hg }$.
\item 
管口距槽内水银面距离距离$H = L - L ^ { \prime } - h ^ { \prime } = 27.5 \mathrm { cm }$.



\end{enumerate}
}


\newpage

\item
\exwhere{$ 2015 $年上海卷}
如图,长为$ h $的水银柱将上端封闭的玻璃管内气体分隔成两部分,$ A $处管内外水银面相平。将玻璃管缓慢向上提升$ H $高度(管下端未离开水银面),上下两部分气体的压强变化分别为$ \triangle p_1 $和$ \triangle p_2 $,体积变化分别为$ \triangle V_1 $和$ \triangle V_2 $。已知水银密度为$ \rho $,玻璃管截面积为$ S $,则 \xzanswer{A} 

\begin{minipage}[h!]{0.7\linewidth}
\vspace{0.3em}
\fourchoices
{$ \triangle p_2 $一定等于$ \triangle p_1 $}
{$ \triangle V_2 $一定等于$ \triangle V_1 $}
{$ \triangle p_2 $与$ \triangle p_1 $之差为$ \rho gh $	}
{$ \triangle V_2 $与$ \triangle V_1 $之和为$ HS $}

\vspace{0.3em}
\end{minipage}
\hfill
\begin{minipage}[h!]{0.3\linewidth}
\flushright
\vspace{0.3em}
\includesvg[width=0.26\linewidth]{picture/svg/255}
\vspace{0.3em}
\end{minipage}




\item 
\exwhere{$ 2016 $年上海卷}
如图,粗细均匀的玻璃管$ A $和$ B $由一橡皮管连接,一定质量的空气被水银柱封闭在$ A $管内,初始时两管水银面等高,$ B $管上方与大气相通。若固定$ A $管,将$ B $管沿竖直方向缓慢下移一小段距离$ H $,$ A $管内的水银面高度相应变化$ h $,则 \xzanswer{B} 


\begin{minipage}[h!]{0.7\linewidth}
\vspace{0.3em}
\fourchoices
{$ h=H $}
{$h < \frac { H } { 2 }$}
{$h = \frac { H } { 2 }$}
{$\frac { H } { 2 } < h < H$}

\vspace{0.3em}
\end{minipage}
\hfill
\begin{minipage}[h!]{0.3\linewidth}
\flushright
\vspace{0.3em}
\includesvg[width=0.4\linewidth]{picture/svg/256}
\vspace{0.3em}
\end{minipage}



\item 
\exwhere{$ 2019 $年物理江苏卷}
如图所示,一定质量理想气体经历$ A \rightarrow B $的等压过程,$ B \rightarrow C $的绝热过程(气体与外界无热量交换),其中$ B \rightarrow C $过程中内能减少$ 900\ J $.求$ A \rightarrow B \rightarrow C $过程中气体对外界做的总功.
\begin{figure}[h!]
\flushright
\includesvg[width=0.25\linewidth]{picture/svg/257}
\end{figure}


\banswer{
$ W=1500\ J $
}








\end{enumerate}


