\bta{光电效应}


\begin{enumerate}
	%\renewcommand{\labelenumi}{\arabic{enumi}.}
	% A(\Alph) a(\alph) I(\Roman) i(\roman) 1(\arabic)
	%设定全局标号series=example	%引用全局变量resume=example
	%[topsep=-0.3em,parsep=-0.3em,itemsep=-0.3em,partopsep=-0.3em]
	%可使用leftmargin调整列表环境左边的空白长度 [leftmargin=0em]
	\item
\exwhere{$ 2019 $ 年物理北京卷}
光电管是一种利用光照射产生电流的装置,当入射光照在管中金属板上时,
可能形成光电流。表中给出了 $ 6 $ 次实验的结果。
%%%
%表格%
%%%


\input{table/physics-3-009} 




由表中数据得出的论断中不正确的是 \xzanswer{B} 

\fourchoices
{两组实验采用了不同频率的入射光}
{两组实验所用的金属板材质不同}
{若入射光子的能量为 $ 5.0 \ eV $,逸出光电子的最大动能为 $ 1.9 \ eV $}
{若入射光子的能量为 $ 5.0 \ eV $,相对光强越强,光电流越大}


\item 
\exwhere{$ 2019 $ 年物理江苏卷}
在“焊接”视网膜的眼科手术中,所用激光的波长$ \lambda =6.4 \times 10^{7} \ m $,每个激光脉冲
的能量 $ E=1.5 \times 10^{-2} \ J $.求每个脉冲中的光子数目.
(已知普朗克常量 $ h=6.63 \times 10 ^{-34} \ J \cdot s $,光速 $ c=3 \times 10^{8} \ m /s $.计
算结果保留一位有效数字)

\banswer{
	$n=5 \times 10^{16}$
}


\item 
\exwhere{$ 2013 $ 年上海卷}
当用一束紫外线照射锌板时,产生了光电效应,这时 \xzanswer{C} 

\fourchoices
{锌板带负电}
{有正离子从锌板逸出}
{有电子从锌板逸出}
{锌板会吸附空气中的正离子}



\item
\exwhere{$ 2014 $ 年物理上海卷}
在光电效应的实验结果中,与光的波动理论不矛盾的是 \xzanswer{C} 


\fourchoices
{光电效应是瞬时发生的}
{所有金属都存在极限颇率}
{光电流随着入射光增强而变大}
{入射光频率越大,光电子最大初动能越大}


\item 
\exwhere{$ 2014 $ 年理综广东卷}
在光电效应实验中,用频率为$ \nu $的光照射光电管阴极,发生了光电效应,下列说法正确的是 \xzanswer{AD} 


\fourchoices
{增大入射光的强度,光电流增大}
{减小入射光的强度,光电效应现象消失}
{改变频率小于$ \nu $的光照射,一定不发生光电效应}
{改变频率大于$ \nu $的光照射,光电子的最大初动能变大}



\item 
\exwhere{$ 2017 $ 年新课标 \lmd{3} 卷}
在光电效应试验中,分别用频率为 $ \nu_{a} $, $ \nu_{b} $ 的单色光 $ a $、$ b $ 照射到同种金属
上,测得相应的遏止电压分别为 $ U_{a} $ 和 $ U_{b} $、光电子的最大初动能分别为 $ E_{ka} $ 和 $ E_{kb} $。 $ h $ 为普朗克常
量。下列说法正确的是 \xzanswer{BC} 

\fourchoices
{若 $ \nu_{a} > \nu_{b} $,则一定有 $ U_{a} < U_{b} $}
{若 $ \nu_{a} > \nu_{b} $,则一定有 $ E_{ka} > E_{kb} $}
{若 $ U_{a} < U_{b} $,则一定有 $ E_{ka} < E_{kb} $}
{若 $ \nu_{a} > \nu_{b} $,则一定有 $ h \nu_{a} - E_{ka} >h \nu_{b} - E_{kb} $}



\item 
\exwhere{$ 2017 $ 年北京卷}
$ 2017 $ 年年初,我国研制的“大连光源”——极紫外自由电子激光装置,发出了
波长在 $ 100 \ nm $($ 1 \ nm =10^{-9} \ m $)附近连续可调的世界上个最强的极紫外激光脉冲,大连光源因其光子
的能量大、密度高,可在能源利用、光刻技术、雾霾治理等领域的研究中发挥重要作用。
一个处于极紫外波段的光子所具有的能量可以电离一个分子,但又不会把分子打碎。据此判断,能
够电离一个分子的能量约为(取普朗克常量 $h=6.6 \times 10^{-34} \ J \cdot s$,真空光速 $c=3 \times 10^{8} \ m / s$ ) \xzanswer{B} 

\fourchoices
{$ 10^{-21} \ J $}
{$ 10^{-18} \ J $}
{$ 10^{-15} \ J $}
{$ 10^{-12} \ J $}


\item 
\exwhere{$ 2017 $ 年海南卷}
三束单色光 $ 1 $、$ 2 $ 和 $ 3 $ 的波长分别为$ \lambda_{1} $、$ \lambda_{2} $ 和$ \lambda_{3} $($\lambda_{1}>\lambda_{2}>\lambda_{3}$)。分别用这三束光照
射同一种金属。已知用光束 $ 2 $ 照射时,恰能产生光电子。下列说法正确的是 \xzanswer{AC} 

\fourchoices
{用光束 $ 1 $ 照射时,不能产生光电子}
{用光束 $ 3 $ 照射时,不能产生光电子}
{用光束 $ 2 $ 照射时,光越强,单位时间内产生的光电子数目越多}
{用光束 $ 2 $ 照射时,光越强,产生的光电子的最大初动能越大}


\item
\exwhere{$ 2013 $ 年上海卷}
某半导体激光器发射波长为 $ 1.5 \times 10^{-6} \ m $,功率为 $ 5.0 \times 10^{-3} \ W $ 的连续激光。已知可见光波长的数量
级为 $ 10^{-7} \ m $,普朗克常量 $ h=6.63 \times 10^{-34} \ J \cdot s $,该激光器发出的 \xzanswer{BD} 

\fourchoices
{是紫外线}
{是红外线}
{光子能量约为 $ 1.3 \times 10^{-18} \ J $}
{光子数约为每秒 $ 3.8 \times 10^{16} $ 个}



\item 
\exwhere{$ 2012 $ 年上海卷}
在光电效应实验中,用单色光照时某种金属表面,有光电子逸出,则光电子的最大初动能取决
于入射光的 \xzanswer{A} 

\fourchoices
{频率}
{强度}
{照射时间}
{光子数目}




\item 
\exwhere{$ 2011 $ 年理综广东卷}
光电效应实验中,下列表述正确的是 \xzanswer{CD} 

\fourchoices
{光照时间越长光电流越大}
{入射光足够强就可以有光电流}
{遏止电压与入射光的频率有关}
{入射光频率大于极限频率才能产生光电子}



\item 
\exwhere{$ 2011 $年上海卷}
用一束紫外线照射某金属时不能产生光电效应,可能使该金属产生光电效应的措施是 \xzanswer{B} 

\fourchoices
{改用频率更小的紫外线照射}
{改用$ X $射线照射}
{改用强度更大的原紫外线照射}
{延长原紫外线的照射时间}


\item 
\exwhere{$ 2013 $ 年北京卷}
以往我们认识的光电效应是单光子光电效应,即一个电子在极短时间内只能吸收到一个光子而
从金属表面逸出。强激光的出现丰富了人们对于光电效应的认识,用强激光照射金属,由于其光子
密度极大,一个电子在极短时间内吸收多个光子成为可能,从而形成多光子电效应,这已被实验证
实。
光电效应实验装置示意如图。用频率为$ \nu $的普通光源照射阴极 $ K $,没有发生光电效应。换用同样频
率为$ \nu $的强激光照射阴极 $ K $,则发生了光电效应;此时,若加上反向电压 $ U $,即将阴极 $ K $ 接电源正
极,阳极 $ A $ 接电源负极,在 $ KA $ 之间就形成了使光电子减速的电场,
逐渐增大 $ U $,光电流会逐渐减小;当光电流恰好减小到零时,所加
反向电压 $ U $ 可能是下列的(其中 $ W $ 为逸出功,$ h $ 为普朗克常量,$ e $
为电子电量) \xzanswer{B} 
\begin{figure}[h!]
	\centering
	\includesvg[width=0.23\linewidth]{picture/svg/GZ-3-tiyou-1290}
\end{figure}

\fourchoices
{$U=\frac{h \nu}{e}-\frac{W}{e} $}
{$\quad U=\frac{2 h \nu}{e}-\frac{W}{e}$}
{$U=2 h \nu-W $}
{$ U=\frac{5 h \nu}{2 e}-\frac{W}{e}$}

\item 
\exwhere{$ 2012 $ 年上海卷}
根据爱因斯坦的“光子说”可知 \xzanswer{B} 

\fourchoices
{“光子说”本质就是牛顿的“微粒说”}
{光的波长越大,光子的能量越小}
{一束单色光的能量可以连续变化}
{只有光子数很多时,光才具有粒子性}


\item 
\exwhere{$ 2015 $ 年上海卷}
某光源发出的光由不同波长的光组成,不同波长的光的强度如图所示。表中
给出了一些材料的极限波长,用该光源发出的光照射表
中材料 \xzanswer{D} 
\begin{figure}[h!]
	\centering
\begin{subfigure}{0.4\linewidth}
	\centering
	\includesvg[width=0.7\linewidth]{picture/svg/GZ-3-tiyou-1291} 
	\caption{}\label{}
\end{subfigure}
\begin{subfigure}{0.4\linewidth}
	\centering
 \begin{tabular}{|c|c|c|c|}
	\hline 
	材  料 & 钠 & 铜 & 铂
 \\
	\hline
	极限波长(nm) & 541 & 268 & 196\\ 
	\hline 
\end{tabular}
	\caption{}\label{}
\end{subfigure}
\end{figure}


\fourchoices
{仅钠能产生光电子}
{仅钠、铜能产生光电子}
{仅铜、铂能产生光电子}
{都能产生光电子}




\item 
\exwhere{$ 2018 $ 年全国\lmd{2}卷}
用波长为 $ 300 \ nm $ 的光照射锌板,电子逸出锌板表面的最大初动能为


$ 1.28 \times 10^{-19} \ J $。已知普朗克常量为 $ 6.63 \times 10^{-34} \ J \cdot s $,真空中的光速为 $ 3.00 \times 10^{8} \ m \cdot s^{-1} $。能使锌产生光
电效应的单色光的最低频率约为 \xzanswer{B} 

\fourchoices
{$ 1 \times 10^{14} \ Hz $}
{$ 8 \times 10^{14} \ Hz $}
{$ 2 \times 10^{15} \ Hz $}
{$ 8 \times 10^{15} \ Hz $}








	
	
	
\end{enumerate}


