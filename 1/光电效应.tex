\bta{光电效应}


\begin{enumerate}
	%\renewcommand{\labelenumi}{\arabic{enumi}.}
	% A(\Alph) a(\alph) I(\Roman) i(\roman) 1(\arabic)
	%设定全局标号series=example	%引用全局变量resume=example
	%[topsep=-0.3em,parsep=-0.3em,itemsep=-0.3em,partopsep=-0.3em]
	%可使用leftmargin调整列表环境左边的空白长度 [leftmargin=0em]
	\item
\exwhere{$ 2019 $ 年物理北京卷}
光电管是一种利用光照射产生电流的装置,当入射光照在管中金属板上时,
可能形成光电流。表中给出了 $ 6 $ 次实验的结果。
%%%
%表格%
%%%


%%%%%%%%%%%%%%%%%%%%%%%%%%%%%%%%%%%%%%%%%%%%%%%%%%%%%%%%%%%%%%%%%%%%%%
%%                                                                  %%
%%  This is the header of a LaTeX2e file exported from Gnumeric.    %%
%%                                                                  %%
%%  This file can be compiled as it stands or included in another   %%
%%  LaTeX document. The table is based on the longtable package so  %%
%%  the longtable options (headers, footers...) can be set in the   %%
%%  preamble section below (see PRAMBLE).                           %%
%%                                                                  %%
%%  To include the file in another, the following two lines must be %%
%%  in the including file:                                          %%
%%        \def\inputGnumericTable{}                                 %%
%%  at the beginning of the file and:                               %%
%%        \input{name-of-this-file.tex}                             %%
%%  where the table is to be placed. Note also that the including   %%
%%  file must use the following packages for the table to be        %%
%%  rendered correctly:                                             %%
%%    \usepackage{ucs}                                              %%
%%    \usepackage[utf8x]{inputenc}                                  %%
%%    \usepackage[T2A]{fontenc}    % if cyrillic is used            %%
%%    \usepackage{color}                                            %%
%%    \usepackage{array}                                            %%
%%    \usepackage{longtable}                                        %%
%%    \usepackage{calc}                                             %%
%%    \usepackage{multirow}                                         %%
%%    \usepackage{hhline}                                           %%
%%    \usepackage{ifthen}                                           %%
%%  optionally (for landscape tables embedded in another document): %%
%%    \usepackage{lscape}                                           %%
%%                                                                  %%
%%%%%%%%%%%%%%%%%%%%%%%%%%%%%%%%%%%%%%%%%%%%%%%%%%%%%%%%%%%%%%%%%%%%%%



%%  This section checks if we are begin input into another file or  %%
%%  the file will be compiled alone. First use a macro taken from   %%
%%  the TeXbook ex 7.7 (suggestion of Han-Wen Nienhuys).            %%
\def\ifundefined#1{\expandafter\ifx\csname#1\endcsname\relax}


%%  Check for the \def token for inputed files. If it is not        %%
%%  defined, the file will be processed as a standalone and the     %%
%%  preamble will be used.                                          %%
\ifundefined{inputGnumericTable}

%%  We must be able to close or not the document at the end.        %%
	\def\gnumericTableEnd{\end{document}}


%%%%%%%%%%%%%%%%%%%%%%%%%%%%%%%%%%%%%%%%%%%%%%%%%%%%%%%%%%%%%%%%%%%%%%
%%                                                                  %%
%%  This is the PREAMBLE. Change these values to get the right      %%
%%  paper size and other niceties.                                  %%
%%                                                                  %%
%%%%%%%%%%%%%%%%%%%%%%%%%%%%%%%%%%%%%%%%%%%%%%%%%%%%%%%%%%%%%%%%%%%%%%

	\documentclass[12pt%
			  %,landscape%
                    ]{report}
       \usepackage{ucs}
       \usepackage[utf8x]{inputenc}
       \usepackage{fullpage}
       \usepackage{color}
       \usepackage{array}
       \usepackage{longtable}
       \usepackage{calc}
       \usepackage{multirow}
       \usepackage{hhline}
       \usepackage{ifthen}

	\begin{document}


%%  End of the preamble for the standalone. The next section is for %%
%%  documents which are included into other LaTeX2e files.          %%
\else

%%  We are not a stand alone document. For a regular table, we will %%
%%  have no preamble and only define the closing to mean nothing.   %%
    \def\gnumericTableEnd{}

%%  If we want landscape mode in an embedded document, comment out  %%
%%  the line above and uncomment the two below. The table will      %%
%%  begin on a new page and run in landscape mode.                  %%
%       \def\gnumericTableEnd{\end{landscape}}
%       \begin{landscape}


%%  End of the else clause for this file being \input.              %%
\fi

%%%%%%%%%%%%%%%%%%%%%%%%%%%%%%%%%%%%%%%%%%%%%%%%%%%%%%%%%%%%%%%%%%%%%%
%%                                                                  %%
%%  The rest is the gnumeric table, except for the closing          %%
%%  statement. Changes below will alter the table's appearance.     %%
%%                                                                  %%
%%%%%%%%%%%%%%%%%%%%%%%%%%%%%%%%%%%%%%%%%%%%%%%%%%%%%%%%%%%%%%%%%%%%%%

\providecommand{\gnumericmathit}[1]{#1} 
%%  Uncomment the next line if you would like your numbers to be in %%
%%  italics if they are italizised in the gnumeric table.           %%
%\renewcommand{\gnumericmathit}[1]{\mathit{#1}}
\providecommand{\gnumericPB}[1]%
{\let\gnumericTemp=\\#1\let\\=\gnumericTemp\hspace{0pt}}
 \ifundefined{gnumericTableWidthDefined}
        \newlength{\gnumericTableWidth}
        \newlength{\gnumericTableWidthComplete}
        \newlength{\gnumericMultiRowLength}
        \global\def\gnumericTableWidthDefined{}
 \fi
%% The following setting protects this code from babel shorthands.  %%
 \ifthenelse{\isundefined{\languageshorthands}}{}{\languageshorthands{english}}
%%  The default table format retains the relative column widths of  %%
%%  gnumeric. They can easily be changed to c, r or l. In that case %%
%%  you may want to comment out the next line and uncomment the one %%
%%  thereafter                                                      %%
\providecommand\gnumbox{\makebox[0pt]}
%%\providecommand\gnumbox[1][]{\makebox}

%% to adjust positions in multirow situations                       %%
\setlength{\bigstrutjot}{\jot}
\setlength{\extrarowheight}{\doublerulesep}

%%  The \setlongtables command keeps column widths the same across  %%
%%  pages. Simply comment out next line for varying column widths.  %%
\setlongtables

\setlength\gnumericTableWidth{%
	87pt+%
	67pt+%
	117pt+%
	117pt+%
	117pt+%
	117pt+%
0pt}
\def\gumericNumCols{6}
\setlength\gnumericTableWidthComplete{\gnumericTableWidth+%
         \tabcolsep*\gumericNumCols*2+\arrayrulewidth*\gumericNumCols}
\ifthenelse{\lengthtest{\gnumericTableWidthComplete > \linewidth}}%
         {\def\gnumericScale{10/10*\ratio{\linewidth-%
                        \tabcolsep*\gumericNumCols*2-%
                        \arrayrulewidth*\gumericNumCols}%
{\gnumericTableWidth}}}%
{\def\gnumericScale{1}}

%%%%%%%%%%%%%%%%%%%%%%%%%%%%%%%%%%%%%%%%%%%%%%%%%%%%%%%%%%%%%%%%%%%%%%
%%                                                                  %%
%% The following are the widths of the various columns. We are      %%
%% defining them here because then they are easier to change.       %%
%% Depending on the cell formats we may use them more than once.    %%
%%                                                                  %%
%%%%%%%%%%%%%%%%%%%%%%%%%%%%%%%%%%%%%%%%%%%%%%%%%%%%%%%%%%%%%%%%%%%%%%

\ifthenelse{\isundefined{\gnumericColA}}{\newlength{\gnumericColA}}{}\settowidth{\gnumericColA}{\begin{tabular}{@{}m{87pt*\gnumericScale}@{}}x\end{tabular}}
\ifthenelse{\isundefined{\gnumericColB}}{\newlength{\gnumericColB}}{}\settowidth{\gnumericColB}{\begin{tabular}{@{}m{67pt*\gnumericScale}@{}}x\end{tabular}}
\ifthenelse{\isundefined{\gnumericColC}}{\newlength{\gnumericColC}}{}\settowidth{\gnumericColC}{\begin{tabular}{@{}m{117pt*\gnumericScale}@{}}x\end{tabular}}
\ifthenelse{\isundefined{\gnumericColD}}{\newlength{\gnumericColD}}{}\settowidth{\gnumericColD}{\begin{tabular}{@{}m{117pt*\gnumericScale}@{}}x\end{tabular}}
\ifthenelse{\isundefined{\gnumericColE}}{\newlength{\gnumericColE}}{}\settowidth{\gnumericColE}{\begin{tabular}{@{}m{117pt*\gnumericScale}@{}}x\end{tabular}}
\ifthenelse{\isundefined{\gnumericColF}}{\newlength{\gnumericColF}}{}\settowidth{\gnumericColF}{\begin{tabular}{@{}m{117pt*\gnumericScale}@{}}x\end{tabular}}

\begin{longtable}[c]{%
	b{\gnumericColA}%
	b{\gnumericColB}%
	b{\gnumericColC}%
	b{\gnumericColD}%
	b{\gnumericColE}%
	b{\gnumericColF}%
	}

%%%%%%%%%%%%%%%%%%%%%%%%%%%%%%%%%%%%%%%%%%%%%%%%%%%%%%%%%%%%%%%%%%%%%%
%%  The longtable options. (Caption, headers... see Goosens, p.124) %%
%	\caption{The Table Caption.}             \\	%
% \hline	% Across the top of the table.
%%  The rest of these options are table rows which are placed on    %%
%%  the first, last or every page. Use \multicolumn if you want.    %%

%%  Header for the first page.                                      %%
%	\multicolumn{6}{c}{The First Header} \\ \hline 
%	\multicolumn{1}{c}{colTag}	%Column 1
%	&\multicolumn{1}{c}{colTag}	%Column 2
%	&\multicolumn{1}{c}{colTag}	%Column 3
%	&\multicolumn{1}{c}{colTag}	%Column 4
%	&\multicolumn{1}{c}{colTag}	%Column 5
%	&\multicolumn{1}{c}{colTag}	\\ \hline %Last column
%	\endfirsthead

%%  The running header definition.                                  %%
%	\hline
%	\multicolumn{6}{l}{\ldots\small\slshape continued} \\ \hline
%	\multicolumn{1}{c}{colTag}	%Column 1
%	&\multicolumn{1}{c}{colTag}	%Column 2
%	&\multicolumn{1}{c}{colTag}	%Column 3
%	&\multicolumn{1}{c}{colTag}	%Column 4
%	&\multicolumn{1}{c}{colTag}	%Column 5
%	&\multicolumn{1}{c}{colTag}	\\ \hline %Last column
%	\endhead

%%  The running footer definition.                                  %%
%	\hline
%	\multicolumn{6}{r}{\small\slshape continued\ldots} \\
%	\endfoot

%%  The ending footer definition.                                   %%
%	\multicolumn{6}{c}{That's all folks} \\ \hline 
%	\endlastfoot
%%%%%%%%%%%%%%%%%%%%%%%%%%%%%%%%%%%%%%%%%%%%%%%%%%%%%%%%%%%%%%%%%%%%%%

\hhline{|-|-|-|-|-|-}
	 \multicolumn{1}{|m{\gnumericColA}|}%
	{\gnumericPB{\centering}组}
	&\multicolumn{1}{m{\gnumericColB}|}%
	{\gnumericPB{\centering}次}
	&\multicolumn{1}{m{\gnumericColC}|}%
	{\gnumericPB{\centering}入射光子的能量$  /eV $}
	&\multicolumn{1}{m{\gnumericColD}|}%
	{\gnumericPB{\centering}相对光强}
	&\multicolumn{1}{m{\gnumericColE}|}%
	{\gnumericPB{\centering}光电流大小$  /mA $}
	&\multicolumn{1}{m{\gnumericColF}|}%
	{\gnumericPB{\centering}逸出光电子的最大动能$  /eV $}
\\
\hhline{|------|}
	 \multicolumn{1}{|m{\gnumericColA}|}%
	{\setlength{\gnumericMultiRowLength}{0pt}%
	 \addtolength{\gnumericMultiRowLength}{\gnumericColA}%
	 \multirow{3}[1]{\gnumericMultiRowLength}{\parbox{\gnumericMultiRowLength}{%
	 \gnumericPB{\centering}第 一 组}}}
	&\multicolumn{1}{m{\gnumericColB}|}%
	{\gnumericPB{\centering}1}
	&\multicolumn{1}{m{\gnumericColC}|}%
	{\gnumericPB{\centering}4.0}
	&\multicolumn{1}{m{\gnumericColD}|}%
	{\gnumericPB{\centering}弱}
	&\multicolumn{1}{m{\gnumericColE}|}%
	{\gnumericPB{\centering}29}
	&\multicolumn{1}{m{\gnumericColF}|}%
	{\gnumericPB{\centering}0.9}
\\
\hhline{~|-----|}
	 \multicolumn{1}{|m{\gnumericColA}|}%
	{}
	&\multicolumn{1}{m{\gnumericColB}|}%
	{\gnumericPB{\centering}2}
	&\multicolumn{1}{m{\gnumericColC}|}%
	{\gnumericPB{\centering}4.0}
	&\multicolumn{1}{m{\gnumericColD}|}%
	{\gnumericPB{\centering}中}
	&\multicolumn{1}{m{\gnumericColE}|}%
	{\gnumericPB{\centering}43}
	&\multicolumn{1}{m{\gnumericColF}|}%
	{\gnumericPB{\centering}0.9}
\\
\hhline{~|-----|}
	 \multicolumn{1}{|m{\gnumericColA}|}%
	{}
	&\multicolumn{1}{m{\gnumericColB}|}%
	{\gnumericPB{\centering}3}
	&\multicolumn{1}{m{\gnumericColC}|}%
	{\gnumericPB{\centering}4.0}
	&\multicolumn{1}{m{\gnumericColD}|}%
	{\gnumericPB{\centering}强}
	&\multicolumn{1}{m{\gnumericColE}|}%
	{\gnumericPB{\centering}60}
	&\multicolumn{1}{m{\gnumericColF}|}%
	{\gnumericPB{\centering}0.9}
\\
\hhline{|------|}
	 \multicolumn{1}{|m{\gnumericColA}|}%
	{\setlength{\gnumericMultiRowLength}{0pt}%
	 \addtolength{\gnumericMultiRowLength}{\gnumericColA}%
	 \multirow{3}[1]{\gnumericMultiRowLength}{\parbox{\gnumericMultiRowLength}{%
	 \gnumericPB{\centering}第 二 组}}}
	&\multicolumn{1}{m{\gnumericColB}|}%
	{\gnumericPB{\centering}4}
	&\multicolumn{1}{m{\gnumericColC}|}%
	{\gnumericPB{\centering}6.0}
	&\multicolumn{1}{m{\gnumericColD}|}%
	{\gnumericPB{\centering}弱}
	&\multicolumn{1}{m{\gnumericColE}|}%
	{\gnumericPB{\centering}27}
	&\multicolumn{1}{m{\gnumericColF}|}%
	{\gnumericPB{\centering}2.9}
\\
\hhline{~|-----|}
	 \multicolumn{1}{|m{\gnumericColA}|}%
	{}
	&\multicolumn{1}{m{\gnumericColB}|}%
	{\gnumericPB{\centering}5}
	&\multicolumn{1}{m{\gnumericColC}|}%
	{\gnumericPB{\centering}6.0}
	&\multicolumn{1}{m{\gnumericColD}|}%
	{\gnumericPB{\centering}中}
	&\multicolumn{1}{m{\gnumericColE}|}%
	{\gnumericPB{\centering}40}
	&\multicolumn{1}{m{\gnumericColF}|}%
	{\gnumericPB{\centering}2.9}
\\
\hhline{~|-----|}
	 \multicolumn{1}{|m{\gnumericColA}|}%
	{}
	&\multicolumn{1}{m{\gnumericColB}|}%
	{\gnumericPB{\centering}6}
	&\multicolumn{1}{m{\gnumericColC}|}%
	{\gnumericPB{\centering}6.0}
	&\multicolumn{1}{m{\gnumericColD}|}%
	{\gnumericPB{\centering}强}
	&\multicolumn{1}{m{\gnumericColE}|}%
	{\gnumericPB{\centering}55}
	&\multicolumn{1}{m{\gnumericColF}|}%
	{\gnumericPB{\centering}2.9}
\\
\hhline{|-|-|-|-|-|-|}
	 \gnumericPB{\raggedright}\gnumbox[l]{}
	&\gnumericPB{\raggedright}\gnumbox[l]{}
	&\gnumericPB{\raggedright}\gnumbox[l]{}
	&\gnumericPB{\raggedright}\gnumbox[l]{}
	&\gnumericPB{\raggedright}\gnumbox[l]{}
	&\gnumericPB{\raggedright}\gnumbox[l]{}
\\
\end{longtable}

\ifthenelse{\isundefined{\languageshorthands}}{}{\languageshorthands{\languagename}}
\gnumericTableEnd
 




由表中数据得出的论断中不正确的是 \xzanswer{B} 

\fourchoices
{两组实验采用了不同频率的入射光}
{两组实验所用的金属板材质不同}
{若入射光子的能量为 $ 5.0 \ eV $,逸出光电子的最大动能为 $ 1.9 \ eV $}
{若入射光子的能量为 $ 5.0 \ eV $,相对光强越强,光电流越大}


\item 
\exwhere{$ 2019 $ 年物理江苏卷}
在“焊接”视网膜的眼科手术中,所用激光的波长$ \lambda =6.4 \times 10^{7} \ m $,每个激光脉冲
的能量 $ E=1.5 \times 10^{-2} \ J $.求每个脉冲中的光子数目.
(已知普朗克常量 $ h=6.63 \times 10 ^{-34} \ J \cdot s $,光速 $ c=3 \times 10^{8} \ m /s $.计
算结果保留一位有效数字)

\banswer{
	$n=5 \times 10^{16}$
}


\item 
\exwhere{$ 2013 $ 年上海卷}
当用一束紫外线照射锌板时,产生了光电效应,这时 \xzanswer{C} 

\fourchoices
{锌板带负电}
{有正离子从锌板逸出}
{有电子从锌板逸出}
{锌板会吸附空气中的正离子}



\item
\exwhere{$ 2014 $ 年物理上海卷}
在光电效应的实验结果中,与光的波动理论不矛盾的是 \xzanswer{C} 


\fourchoices
{光电效应是瞬时发生的}
{所有金属都存在极限颇率}
{光电流随着入射光增强而变大}
{入射光频率越大,光电子最大初动能越大}


\item 
\exwhere{$ 2014 $ 年理综广东卷}
在光电效应实验中,用频率为$ \nu $的光照射光电管阴极,发生了光电效应,下列说法正确的是 \xzanswer{AD} 


\fourchoices
{增大入射光的强度,光电流增大}
{减小入射光的强度,光电效应现象消失}
{改变频率小于$ \nu $的光照射,一定不发生光电效应}
{改变频率大于$ \nu $的光照射,光电子的最大初动能变大}



\item 
\exwhere{$ 2017 $ 年新课标 \lmd{3} 卷}
在光电效应试验中,分别用频率为 $ \nu_{a} $, $ \nu_{b} $ 的单色光 $ a $、$ b $ 照射到同种金属
上,测得相应的遏止电压分别为 $ U_{a} $ 和 $ U_{b} $、光电子的最大初动能分别为 $ E_{ka} $ 和 $ E_{kb} $。 $ h $ 为普朗克常
量。下列说法正确的是 \xzanswer{BC} 

\fourchoices
{若 $ \nu_{a} > \nu_{b} $,则一定有 $ U_{a} < U_{b} $}
{若 $ \nu_{a} > \nu_{b} $,则一定有 $ E_{ka} > E_{kb} $}
{若 $ U_{a} < U_{b} $,则一定有 $ E_{ka} < E_{kb} $}
{若 $ \nu_{a} > \nu_{b} $,则一定有 $ h \nu_{a} - E_{ka} >h \nu_{b} - E_{kb} $}



\item 
\exwhere{$ 2017 $ 年北京卷}
$ 2017 $ 年年初,我国研制的“大连光源”——极紫外自由电子激光装置,发出了
波长在 $ 100 \ nm $($ 1 \ nm =10^{-9} \ m $)附近连续可调的世界上个最强的极紫外激光脉冲,大连光源因其光子
的能量大、密度高,可在能源利用、光刻技术、雾霾治理等领域的研究中发挥重要作用。
一个处于极紫外波段的光子所具有的能量可以电离一个分子,但又不会把分子打碎。据此判断,能
够电离一个分子的能量约为(取普朗克常量 $h=6.6 \times 10^{-34} \ J \cdot s$,真空光速 $c=3 \times 10^{8} \ m / s$ ) \xzanswer{B} 

\fourchoices
{$ 10^{-21} \ J $}
{$ 10^{-18} \ J $}
{$ 10^{-15} \ J $}
{$ 10^{-12} \ J $}


\item 
\exwhere{$ 2017 $ 年海南卷}
三束单色光 $ 1 $、$ 2 $ 和 $ 3 $ 的波长分别为$ \lambda_{1} $、$ \lambda_{2} $ 和$ \lambda_{3} $($\lambda_{1}>\lambda_{2}>\lambda_{3}$)。分别用这三束光照
射同一种金属。已知用光束 $ 2 $ 照射时,恰能产生光电子。下列说法正确的是 \xzanswer{AC} 

\fourchoices
{用光束 $ 1 $ 照射时,不能产生光电子}
{用光束 $ 3 $ 照射时,不能产生光电子}
{用光束 $ 2 $ 照射时,光越强,单位时间内产生的光电子数目越多}
{用光束 $ 2 $ 照射时,光越强,产生的光电子的最大初动能越大}


\item
\exwhere{$ 2013 $ 年上海卷}
某半导体激光器发射波长为 $ 1.5 \times 10^{-6} \ m $,功率为 $ 5.0 \times 10^{-3} \ W $ 的连续激光。已知可见光波长的数量
级为 $ 10^{-7} \ m $,普朗克常量 $ h=6.63 \times 10^{-34} \ J \cdot s $,该激光器发出的 \xzanswer{BD} 

\fourchoices
{是紫外线}
{是红外线}
{光子能量约为 $ 1.3 \times 10^{-18} \ J $}
{光子数约为每秒 $ 3.8 \times 10^{16} $ 个}



\item 
\exwhere{$ 2012 $ 年上海卷}
在光电效应实验中,用单色光照时某种金属表面,有光电子逸出,则光电子的最大初动能取决
于入射光的 \xzanswer{A} 

\fourchoices
{频率}
{强度}
{照射时间}
{光子数目}




\item 
\exwhere{$ 2011 $ 年理综广东卷}
光电效应实验中,下列表述正确的是 \xzanswer{CD} 

\fourchoices
{光照时间越长光电流越大}
{入射光足够强就可以有光电流}
{遏止电压与入射光的频率有关}
{入射光频率大于极限频率才能产生光电子}



\item 
\exwhere{$ 2011 $年上海卷}
用一束紫外线照射某金属时不能产生光电效应,可能使该金属产生光电效应的措施是 \xzanswer{B} 

\fourchoices
{改用频率更小的紫外线照射}
{改用$ X $射线照射}
{改用强度更大的原紫外线照射}
{延长原紫外线的照射时间}


\item 
\exwhere{$ 2013 $ 年北京卷}
以往我们认识的光电效应是单光子光电效应,即一个电子在极短时间内只能吸收到一个光子而
从金属表面逸出。强激光的出现丰富了人们对于光电效应的认识,用强激光照射金属,由于其光子
密度极大,一个电子在极短时间内吸收多个光子成为可能,从而形成多光子电效应,这已被实验证
实。
光电效应实验装置示意如图。用频率为$ \nu $的普通光源照射阴极 $ K $,没有发生光电效应。换用同样频
率为$ \nu $的强激光照射阴极 $ K $,则发生了光电效应;此时,若加上反向电压 $ U $,即将阴极 $ K $ 接电源正
极,阳极 $ A $ 接电源负极,在 $ KA $ 之间就形成了使光电子减速的电场,
逐渐增大 $ U $,光电流会逐渐减小;当光电流恰好减小到零时,所加
反向电压 $ U $ 可能是下列的(其中 $ W $ 为逸出功,$ h $ 为普朗克常量,$ e $
为电子电量) \xzanswer{B} 
\begin{figure}[h!]
	\centering
	\includesvg[width=0.23\linewidth]{picture/svg/GZ-3-tiyou-1290}
\end{figure}

\fourchoices
{$U=\frac{h \nu}{e}-\frac{W}{e} $}
{$\quad U=\frac{2 h \nu}{e}-\frac{W}{e}$}
{$U=2 h \nu-W $}
{$ U=\frac{5 h \nu}{2 e}-\frac{W}{e}$}

\item 
\exwhere{$ 2012 $ 年上海卷}
根据爱因斯坦的“光子说”可知 \xzanswer{B} 

\fourchoices
{“光子说”本质就是牛顿的“微粒说”}
{光的波长越大,光子的能量越小}
{一束单色光的能量可以连续变化}
{只有光子数很多时,光才具有粒子性}


\item 
\exwhere{$ 2015 $ 年上海卷}
某光源发出的光由不同波长的光组成,不同波长的光的强度如图所示。表中
给出了一些材料的极限波长,用该光源发出的光照射表
中材料 \xzanswer{D} 
\begin{figure}[h!]
	\centering
\begin{subfigure}{0.4\linewidth}
	\centering
	\includesvg[width=0.7\linewidth]{picture/svg/GZ-3-tiyou-1291} 
	\caption{}\label{}
\end{subfigure}
\begin{subfigure}{0.4\linewidth}
	\centering
 \begin{tabular}{|c|c|c|c|}
	\hline 
	材  料 & 钠 & 铜 & 铂
 \\
	\hline
	极限波长(nm) & 541 & 268 & 196\\ 
	\hline 
\end{tabular}
	\caption{}\label{}
\end{subfigure}
\end{figure}


\fourchoices
{仅钠能产生光电子}
{仅钠、铜能产生光电子}
{仅铜、铂能产生光电子}
{都能产生光电子}




\item 
\exwhere{$ 2018 $ 年全国\lmd{2}卷}
用波长为 $ 300 \ nm $ 的光照射锌板,电子逸出锌板表面的最大初动能为


$ 1.28 \times 10^{-19} \ J $。已知普朗克常量为 $ 6.63 \times 10^{-34} \ J \cdot s $,真空中的光速为 $ 3.00 \times 10^{8} \ m \cdot s^{-1} $。能使锌产生光
电效应的单色光的最低频率约为 \xzanswer{B} 

\fourchoices
{$ 1 \times 10^{14} \ Hz $}
{$ 8 \times 10^{14} \ Hz $}
{$ 2 \times 10^{15} \ Hz $}
{$ 8 \times 10^{15} \ Hz $}








	
	
	
\end{enumerate}


