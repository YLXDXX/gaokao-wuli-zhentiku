\bta{用双缝干涉测光的波长}


\begin{enumerate}
	%\renewcommand{\labelenumi}{\arabic{enumi}.}
	% A(\Alph) a(\alph) I(\Roman) i(\roman) 1(\arabic)
	%设定全局标号series=example	%引用全局变量resume=example
	%[topsep=-0.3em,parsep=-0.3em,itemsep=-0.3em,partopsep=-0.3em]
	%可使用leftmargin调整列表环境左边的空白长度 [leftmargin=0em]
	\item
\exwhere{$ 2011 $ 年理综北京卷}
如图所示的双缝干涉实验,用绿光照射单缝 $ S $ 时,在光屏 $ P $ 上
观察到干涉条纹。要得到相邻条纹间距更大的干涉图样,可以 \xzanswer{C} 
\begin{figure}[h!]
	\centering
	\includesvg[width=0.23\linewidth]{picture/svg/GZ-3-tiyou-1479}
\end{figure}

\fourchoices
{增大 $ S_{1} $ 与 $ S_{2} $ 的间距}
{减小双缝屏到光屏的距离}
{将绿光换为红光}
{将绿光换为紫光}



\item 
\exwhere{$ 2012 $ 年理综福建卷}
在“用双缝干涉测光的波长”实验中(实验装置如下图)
:
\begin{enumerate}
	%\renewcommand{\labelenumi}{\arabic{enumi}.}
	% A(\Alph) a(\alph) I(\Roman) i(\roman) 1(\arabic)
	%设定全局标号series=example	%引用全局变量resume=example
	%[topsep=-0.3em,parsep=-0.3em,itemsep=-0.3em,partopsep=-0.3em]
	%可使用leftmargin调整列表环境左边的空白长度 [leftmargin=0em]
	\item
下列说法哪一个是错误 \underlinegap 的。(填选项前
的字母)


\threechoices
{调节光源高度使光束沿遮光筒轴线照在屏中心时,应放上单缝和双缝}
{测量某条干涉亮纹位置时,应使测微目镜分划中心刻线与该亮纹的中心对齐}
{为了减少测量误差,可用测微目镜测出 $ n $ 条亮纹间的距离 $ a $,求出相邻两条亮纹间距$ \Delta x=a/(n-1) $}

\item 
测量某亮纹位置时,手轮上的示数如右图,其示数为 \underlinegap $ mm $。

	
\end{enumerate}
\begin{figure}[h!]
	\centering
\begin{subfigure}{0.4\linewidth}
	\centering
	\includesvg[width=0.7\linewidth]{picture/svg/GZ-3-tiyou-1480} 
	\caption{}\label{}
\end{subfigure}
\begin{subfigure}{0.4\linewidth}
	\centering
	\includesvg[width=0.7\linewidth]{picture/svg/GZ-3-tiyou-1482} 
	\caption{}\label{}
\end{subfigure}

\end{figure}


 \tk{
\begin{enumerate}
	%\renewcommand{\labelenumi}{\arabic{enumi}.}
	% A(\Alph) a(\alph) I(\Roman) i(\roman) 1(\arabic)
	%设定全局标号series=example	%引用全局变量resume=example
	%[topsep=-0.3em,parsep=-0.3em,itemsep=-0.3em,partopsep=-0.3em]
	%可使用leftmargin调整列表环境左边的空白长度 [leftmargin=0em]
	\item
	A
	\item 
	$ 1.970 \ mm $
\end{enumerate}
} 


\item 
\exwhere{$ 2016 $ 年上海卷}
在双缝干涉实验中,屏上出现了明暗相间的条纹,则 \xzanswer{D} 

\fourchoices
{中间条纹间距较两侧更宽}
{不同色光形成的条纹完全重合}
{双缝间距离越大条纹间距离也越大}
{遮住一条缝后屏上仍有明暗相间的条纹}


	
	
	
\end{enumerate}

