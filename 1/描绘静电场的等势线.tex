\bta{描绘静电场的等势线}


\begin{enumerate}[leftmargin=0em]
\renewcommand{\labelenumi}{\arabic{enumi}.}
% A(\Alph) a(\alph) I(\Roman) i(\roman) 1(\arabic)
%设定全局标号series=example	%引用全局变量resume=example
%[topsep=-0.3em,parsep=-0.3em,itemsep=-0.3em,partopsep=-0.3em]
%可使用leftmargin调整列表环境左边的空白长度 [leftmargin=0em]
\item
\exwhere{$ 2013 $年全国卷大纲卷}
如图,$ E $为直流电源,$ G $为灵敏电流计,$ A $、$ B $为两个圆柱形电极,$ P $是木板,$ C $、$ D $为两个探针,$ S $为开关。现用上述实验器材进行“用描迹法画出电场中平面上的等势线”的实验。
\begin{enumerate}
\renewcommand{\labelenumi}{\arabic{enumi}.}
% A(\Alph) a(\alph) I(\Roman) i(\roman) 1(\arabic)
%设定全局标号series=example	%引用全局变量resume=example
%[topsep=-0.3em,parsep=-0.3em,itemsep=-0.3em,partopsep=-0.3em]
%可使用leftmargin调整列表环境左边的空白长度 [leftmargin=0em]
\item
木板$ P $上有白纸、导电纸和复写纸,最上面的应该是 \tk{导电} 纸;
\item 
用实线代表导线将实验器材正确连接。

\end{enumerate}
\begin{figure}[h!]
\centering
\includesvg[width=0.53\linewidth]{picture/svg/605}
\end{figure}

\banswer{
连线如图所示($ 4 $分。探针和灵敏电流计部分$ 2 $分,有任何错误不给这$ 2 $分;其余部分$ 2 $分,有任何错误也不给这$ 2 $分)	
\begin{figure}[h!]
\centering
\includesvg[width=0.23\linewidth]{picture/svg/606}
\end{figure}
}



\item 
\exwhere{$ 2016 $年上海卷}
“用$ DIS $描绘电场的等势线”的实验装置示意图如图所示。
\begin{enumerate}
\renewcommand{\labelenumi}{\arabic{enumi}.}
% A(\Alph) a(\alph) I(\Roman) i(\roman) 1(\arabic)
%设定全局标号series=example	%引用全局变量resume=example
%[topsep=-0.3em,parsep=-0.3em,itemsep=-0.3em,partopsep=-0.3em]
%可使用leftmargin调整列表环境左边的空白长度 [leftmargin=0em]
\item
(单选题)该实验描绘的是 \xzanswer{B} 

\begin{minipage}[h!]{0.7\linewidth}
\vspace{0.3em}
\fourchoices
{两个等量同种电荷周围的等势线}
{两个等量异种电荷周围的等势线}
{两个不等量同种电荷周围的等势线}
{两个不等量异种电荷周围的等势线}
\vspace{0.3em}
\end{minipage}
\hfill
\begin{minipage}[h!]{0.3\linewidth}
\flushright
\vspace{0.3em}
\includesvg[width=0.6\linewidth]{picture/svg/607}
\vspace{0.3em}
\end{minipage}





\item 
(单选题)实验操作时,需在平整的木板上依次铺放 \xzanswer{D} 
\fourchoices
{导电纸、复写纸、白纸}
{白纸、导电纸、复写纸}
{导电纸、白纸、复写纸}
{白纸、复写纸、导电纸}


\item 
若电压传感器的红、黑探针分别接触图中$ d $、$ f $两点($ f $、$ d $连线与$ A $、$ B $连线垂直)时,示数小于零。为使示数为零,应保持红色探针与$ d $点接触,而将黑色探针 \tk{向右} (选填:“向左”或“向右”)移动。




\end{enumerate}








\end{enumerate}

