\bta{动量和能量}

\begin{enumerate}[leftmargin=0em]
\renewcommand{\labelenumi}{\arabic{enumi}.}
% A(\Alph) a(\alph) I(\Roman) i(\roman) 1(\arabic)
%设定全局标号series=example	%引用全局变量resume=example
%[topsep=-0.3em,parsep=-0.3em,itemsep=-0.3em,partopsep=-0.3em]
%可使用leftmargin调整列表环境左边的空白长度 [leftmargin=0em]
\item
\exwhere{$ 2019 $年物理全国\lmd{3}卷}
静止在水平地面上的两小物块$ A $、$ B $,质量分别为$ m_{A} =l.0 \ kg $,$ m_{B} =4.0 \ kg $;两者之间有一被压缩的微型弹簧,$ A $与其右侧的竖直墙壁距离$ l=1.0m $,如图所示。某时刻,将压缩的微型弹簧释放,使$ A $、$ B $瞬间分离,两物块获得的动能之和为$ E_k=10.0J $。释放后,$ A $沿着与墙壁垂直的方向向右运动。$ A $、$ B $与地面之间的动摩擦因数均为$ u=0.20 $。重力加速度取$ g=10 \ m/s^{2} $。$ A $、$ B $运动过程中所涉及的碰撞均为弹性碰撞且碰撞时间极短。
\begin{enumerate}
\renewcommand{\labelenumi}{\arabic{enumi}.}
% A(\Alph) a(\alph) I(\Roman) i(\roman) 1(\arabic)
%设定全局标号series=example	%引用全局变量resume=example
%[topsep=-0.3em,parsep=-0.3em,itemsep=-0.3em,partopsep=-0.3em]
%可使用leftmargin调整列表环境左边的空白长度 [leftmargin=0em]
\item
求弹簧释放后瞬间$ A $、$ B $速度的大小;
\item 
物块$ A $、$ B $中的哪一个先停止?该物块刚停止时$ A $与$ B $之间的距离是多少?
\item 
$ A $和$ B $都停止后,$ A $与$ B $之间的距离是多少?

\end{enumerate}
\begin{figure}[h!]
\flushright
\includesvg[width=0.4\linewidth]{picture/svg/532}
\end{figure}


\banswer{
\begin{enumerate}
\renewcommand{\labelenumi}{\arabic{enumi}.}
% A(\Alph) a(\alph) I(\Roman) i(\roman) 1(\arabic)
%设定全局标号series=example	%引用全局变量resume=example
%[topsep=-0.3em,parsep=-0.3em,itemsep=-0.3em,partopsep=-0.3em]
%可使用leftmargin调整列表环境左边的空白长度 [leftmargin=0em]
\item
$v _ { A } = 4.0 \mathrm { m } / \mathrm { s } , \quad v _ { B } = 1.0 \mathrm { m } / \mathrm { s }$
\item 
A先停止; 0.50 m;
\item 
0.91 m;


\end{enumerate}


}




\item 
\exwhere{$ 2018 $年天津卷}
质量为$ 0.45 $ $ kg $的木块静止在光滑水平面上,一质量为$ 0.05 $ $ kg $的子弹以$ 200 $ $ m/s $的水平速度击中木块,并留在其中,整个木块沿子弹原方向运动,则木块最终速度的大小是\tk{$ 20 $ }$ m/s $。若子弹在木块中运动时受到的平均阻力为$ 4.5 \times 10^3 $ $ N $,则子弹射入木块的深度为\tk{$ 0.2m $}。

\item 
\exwhere{$ 2018 $年全国\lmd{1} 卷}
高铁列车在启动阶段的运动可看作初速度为零的均加速直线运动,在启动阶段列车的动能 \xzanswer{B} 
\fourchoices
{与它所经历的时间成正比}
{与它的位移成正比}
{与它的速度成正比 }
{与它的动量成正比}

\newpage
\item
\exwhere{$ 2018 $年海南卷}
如图,用长为$ l $的轻绳悬挂一质量为$ M $的沙箱,沙箱静止。一质量为$ m $的弹丸以速度$ v $水平射入沙箱并留在其中,随后与沙箱共同摆动一小角度。不计空气阻力。对子弹射向沙箱到与其共同摆过一小角度的过程 \xzanswer{C} 
\begin{figure}[h!]
\centering
\includesvg[width=0.15\linewidth]{picture/svg/533}
\end{figure}

\fourchoices
{若保持$ m $、$ v $、$ l $不变,$ M $变大,则系统损失的机械能变小}
{若保持$ M $、$ v $、$ l $不变,$ m $变大,则系统损失的机械能变小}
{若保持$ M $、$ m $、$ l $不变,$ v $变大,则系统损失的机械能变大}
{若保持$ M $、$ m $、$ v $不变,$ l $变大,则系统损失的机械能变大}



\item
\exwhere{$ 2018 $年全国\lmd{1}卷}
一质量为$ m $的烟花弹获得动能$ E $后,从地面竖直升空。当烟花弹上升的速度为零时,弹中火药爆炸将烟花弹炸为质量相等的两部分,两部分获得的动能之和也为$ E $,且均沿竖直方向运动。爆炸时间极短,重力加速度大小为$ g $,不计空气阻力和火药的质量。求:
\begin{enumerate}
\renewcommand{\labelenumi}{\arabic{enumi}.}
% A(\Alph) a(\alph) I(\Roman) i(\roman) 1(\arabic)
%设定全局标号series=example	%引用全局变量resume=example
%[topsep=-0.3em,parsep=-0.3em,itemsep=-0.3em,partopsep=-0.3em]
%可使用leftmargin调整列表环境左边的空白长度 [leftmargin=0em]
\item
烟花弹从地面开始上升到弹中火药爆炸所经过的时间;
\item 
爆炸后烟花弹向上运动的部分距地面的最大高度。

\end{enumerate}


\banswer{
\begin{enumerate}
\renewcommand{\labelenumi}{\arabic{enumi}.}
% A(\Alph) a(\alph) I(\Roman) i(\roman) 1(\arabic)
%设定全局标号series=example	%引用全局变量resume=example
%[topsep=-0.3em,parsep=-0.3em,itemsep=-0.3em,partopsep=-0.3em]
%可使用leftmargin调整列表环境左边的空白长度 [leftmargin=0em]
\item
$t = \frac { 1 } { g } \sqrt { \frac { 2 E } { m } }$
\item 
$h = h _ { 1 } + h _ { 2 } = \frac { 2 E } { m g }$



\end{enumerate}


}


\newpage
\item 
\exwhere{$ 2018 $年全国\lmd{2}卷}
汽车$ A $在水平冰雪路面上行驶。驾驶员发现其正前方停有汽车$ B $,立即采取制动措施,但仍然撞上了汽车$ B $。两车碰撞时和两车都完全停止后的位置如图所示,碰撞后$ B $车向前滑动了,$ A $车向前滑动了。已知$ A $和$ B $的质量分别为$ 2.0 \times 10^{3}\ kg $和$ 1.5 \times 10^{3}\ kg $,两车与该冰雪路面间的动摩擦因数均为$ 0.10 $,两车碰撞时间极短,在碰撞后车轮均没有滚动,重力加速度大小$ g=10\ m/s^{2} $。求:
\begin{enumerate}
\renewcommand{\labelenumi}{\arabic{enumi}.}
% A(\Alph) a(\alph) I(\Roman) i(\roman) 1(\arabic)
%设定全局标号series=example	%引用全局变量resume=example
%[topsep=-0.3em,parsep=-0.3em,itemsep=-0.3em,partopsep=-0.3em]
%可使用leftmargin调整列表环境左边的空白长度 [leftmargin=0em]
\item
碰撞后的瞬间$ B $车速度的大小;
\item 
碰撞前的瞬间$ A $车速度的大小。


\end{enumerate}
\begin{figure}[h!]
\flushright
\includesvg[width=0.4\linewidth]{picture/svg/534}
\end{figure}



\banswer{
\begin{enumerate}
\renewcommand{\labelenumi}{\arabic{enumi}.}
% A(\Alph) a(\alph) I(\Roman) i(\roman) 1(\arabic)
%设定全局标号series=example	%引用全局变量resume=example
%[topsep=-0.3em,parsep=-0.3em,itemsep=-0.3em,partopsep=-0.3em]
%可使用leftmargin调整列表环境左边的空白长度 [leftmargin=0em]
\item
$v _ { \mathrm { B } } ^ { \prime } = 3.0 \mathrm { m } / \mathrm { s }$
\item 
$v _ { \mathrm { A } } = 4.3 \mathrm { m } / \mathrm { s }$



\end{enumerate}


}








\end{enumerate}







