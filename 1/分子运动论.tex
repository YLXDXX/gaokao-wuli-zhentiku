\bta{分子运动论}


\begin{enumerate}[leftmargin=0em]
\renewcommand{\labelenumi}{\arabic{enumi}.}
% A(\Alph) a(\alph) I(\Roman) i(\roman) 1(\arabic)
%设定全局标号series=example	%引用全局变量resume=example
%[topsep=-0.3em,parsep=-0.3em,itemsep=-0.3em,partopsep=-0.3em]
%可使用leftmargin调整列表环境左边的空白长度 [leftmargin=0em]
\item
\exwhere{$ 2017 $年北京卷}
以下关于热运动的说法正确的是 \xzanswer{C} 


\fourchoices
{水流速度越大,水分子的热运动越剧烈}
{水凝结成冰后,水分子的热运动停止}
{水的温度越高,水分子的热运动越剧烈}
{水的温度升高,每一个水分子的运动速率都会增大}



\item
\exwhere{$ 2012 $年理综全国卷}
下列关于布朗运动的说法,正确的是 \xzanswer{BD} 


\fourchoices
{布朗运动是液体分子的无规则运动}
{液体温度越高,悬浮粒子越小,布朗运动越剧烈}
{布朗运动是由于液体各个部分的温度不同而引起的}
{布朗运动是由液体分子从各个方向对悬浮粒子撞击作用的不平衡引起的}

\item 
\exwhere{$ 2012 $年理综广东卷}
清晨,草叶上的露珠是由空气中的水汽凝结成水珠,这一物理过程中,水分子间的 \xzanswer{D} 


\fourchoices
{引力消失,斥力增大}
{斥力消失,引力增大}
{引力、斥力都减小}
{引力、斥力都增大}


\item
\exwhere{$ 2011 $年理综广东卷}
如图所示,两个接触面平滑的铅柱压紧后悬挂起来,下面的铅柱不脱落,主要原因是 \xzanswer{D} 
\begin{figure}[h!]
\centering
\includesvg[width=0.23\linewidth]{picture/svg/251}
\end{figure}

\fourchoices
{铅分子做无规则热运动}
{铅柱受到大气压力作用}
{铅柱间存在万有引力作用}
{铅柱间存在分子引力作用}


\item
\exwhere{$ 2011 $年理综四川卷}
气体能够充满密闭容器,说明气体分子除相互碰撞的短暂时间外 \xzanswer{C} 


\fourchoices
{气体分子可以做布朗运动}
{气体分子的动能都一样大}
{相互作用力十分微弱,气体分子可以自由运动}
{相互作用力十分微弱,气体分子间的距离都一样大}


\item
\exwhere{$ 2014 $年理综北京卷}
下列说法中正确的是 \xzanswer{B} 


\fourchoices
{物体温度降低,其分子热运动的平均动能增大}
{物体温度升高,其分子热运动的平均动能增大}
{物体温度降低,其内能一定增大}
{物体温度不变,其内能一定不变}


\item
\exwhere{$ 2014 $年物理上海卷}
分子间同时存在着引力和斥力,当分子间距增加时,分子间的 \xzanswer{C} 


\fourchoices
{引力增加,斥力减小}
{引力增加,斥力增加}
{引力减小,斥力减小}
{引力减小.斥力增加}

\item
\exwhere{$ 2015 $年上海卷}
一定质量的理想气体在升温过程中 \xzanswer{C} 


\fourchoices
{分子平均势能减小}
{每个分子速率都增大}
{分子平均动能增大}
{分子间作用力先增大后减小}




\item
\exwhere{$ 2016 $年北京卷}
雾霾天气是对大气中各种悬浮颗粒物含量超标的笼统表述,是特定气候条件与人类活动相互作用的结果。雾霾中,各种悬浮颗粒物形状不规则,但可视为密度相同、直径不同的球体,并用$ PM10 $、$ PM2.5 $分别表示直径小于或等于$ 10 \mu m $、$ 2.5 \mu m $的颗粒物($ PM $是颗粒物的英文缩写)。

某科研机构对北京地区的检测结果表明,在静稳的雾霾天气中,近地面高度百米的范围内,$ PM10 $的浓度随高度的增加略有减小,大于$ PM10 $的大悬浮颗粒物的浓度随高度的增加明显减小,且两种浓度分布基本不随时间变化。

据此材料,以下叙述正确的是 \xzanswer{C} 


\fourchoices
{$ PM10 $表示直径小于或等于$ 1.0 \times 10^{-6} $ $ m $的悬浮颗粒物}
{$ PM10 $受到的空气分子作用力的合力始终大于其受到的重力}
{$ PM10 $和大悬浮颗粒物都在做布朗运动}
{$ PM2.5 $浓度随高度的增加逐渐增大}


\item
\exwhere{$ 2016 $年上海卷}
某气体的摩尔质量为$ M $,分子质量为$ m $ 。若$ 1 $摩尔该气体的体积为$ V_m $,密度为$ \rho $,则该气体单位体积分子数为(阿伏伽德罗常数为$ N_A $) \xzanswer{ABC} 


\fourchoices
{$ \frac { N _ { \mathrm { A } } } { V _ { \mathrm { m } } } $}
{$ \frac { M } { m V _ { \mathrm { m } } } $}
{$ \frac { \rho N _ { \mathrm { A } } } { M } $}
{$ \frac { \rho N _ { \mathrm { A } } } { m } $}




\item
\exwhere{$ 2018 $年北京卷}
关于分子动理论,下列说法正确的是 \xzanswer{C} 


\fourchoices
{气体扩散的快慢与温度无关}
{布朗运动是液体分子的无规则运动}
{分子间同时存在着引力和斥力}
{分子间的引力总是随分子间距增大而增大}


\item
\exwhere{$ 2019 $年物理江苏卷}
在没有外界影响的情况下,密闭容器内的理想气体静置足够长时间后,该气体 \xzanswer{CD} 


\fourchoices
{分子的无规则运动停息下来}
{每个分子的速度大小均相等}
{分子的平均动能保持不变}
{分子的密集程度保持不变}


\item
\exwhere{$ 2019 $年物理江苏卷}
由于水的表面张力,荷叶上的小水滴总是球形的.在小水滴表面层中,水分子之间的相互作用总体上表现为\tk{引力}(选填“引力”或“斥力”). 分子势能$ E_p $和分子间距离$ r $的关系图象如图所示,能总体上反映小水滴表面层中水分子$ E_p $的是图中\tk{C}(选填“$ A $”“$ B $”或“$ C $”)的位置.
\begin{figure}[h!]
\centering
\includesvg[width=0.23\linewidth]{picture/svg/252}
\end{figure}



\item
\exwhere{$ 2019 $年物理北京卷}
下列说法正确的是 \xzanswer{A} 


\fourchoices
{温度标志着物体内大量分子热运动的剧烈程度}
{内能是物体中所有分子热运动所具有的动能的总和}
{气体压强仅与气体分子的平均动能有关}
{气体膨胀对外做功且温度降低,分子平均动能可能不变}







\end{enumerate}


