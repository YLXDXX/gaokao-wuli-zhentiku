\bta{比较法和替代法测电阻}


\begin{enumerate}[leftmargin=0em]
\renewcommand{\labelenumi}{\arabic{enumi}.}
% A(\Alph) a(\alph) I(\Roman) i(\roman) 1(\arabic)
%设定全局标号series=example	%引用全局变量resume=example
%[topsep=-0.3em,parsep=-0.3em,itemsep=-0.3em,partopsep=-0.3em]
%可使用leftmargin调整列表环境左边的空白长度 [leftmargin=0em]
\item
\exwhere{$ 2011 $年新课标卷}	
为了测量一微安表头$ A $的内阻,某同学设计了如图所示的电路。图中,$ A_{0} $是标准电流表,$ R_{0} $和$ R_N $分别是滑动变阻器和电阻箱,$ S $和$ S_{1} $分别是单刀双掷开关和单刀开关,$ E $是电池。完成下列实验步骤中的填空:
\begin{figure}[h!]
\centering
\includesvg[width=0.43\linewidth]{picture/svg/633}
\end{figure}

\begin{enumerate}
\renewcommand{\labelenumi}{\arabic{enumi}.}
% A(\Alph) a(\alph) I(\Roman) i(\roman) 1(\arabic)
%设定全局标号series=example	%引用全局变量resume=example
%[topsep=-0.3em,parsep=-0.3em,itemsep=-0.3em,partopsep=-0.3em]
%可使用leftmargin调整列表环境左边的空白长度 [leftmargin=0em]
\item
将$ S $拨向接点$ 1 $,接通$ S_{1} $,调节\tk{$ R_{0} $},使待测表头指针偏转到适当位置,记下此时\tk{标准电流表$ A_{0} $}的读数 I ;
\item 
然后将$ S $拨向接点$ 2 $,调节\tk{$ R_N $},使\tk{标准电流表$ A_{0} $的示数仍为 I },记下此时$ R_N $的读数;
\item 
多次重复上述过程,计算$ R_N $读数的\tk{平均值},此即为待测微安表头内阻的测量值。



\end{enumerate}


\item 
\exwhere{$ 2012 $年海南卷}
图示电路可用来测量电阻的阻值。其中$ E $为电源,$ R $为已知电阻,$ R_{x} $为待测电阻, $ V $可视为理想电压表,$ S_{0} $为单刀单掷开关,$ S_{1} $、$ S_{2} $为单刀双掷开关。
\begin{figure}[h!]
\centering
\includesvg[width=0.33\linewidth]{picture/svg/634}
\end{figure}

\begin{enumerate}
\renewcommand{\labelenumi}{\arabic{enumi}.}
% A(\Alph) a(\alph) I(\Roman) i(\roman) 1(\arabic)
%设定全局标号series=example	%引用全局变量resume=example
%[topsep=-0.3em,parsep=-0.3em,itemsep=-0.3em,partopsep=-0.3em]
%可使用leftmargin调整列表环境左边的空白长度 [leftmargin=0em]
\item
当$ S_{0} $闭合时,若$ S_{1} $、$ S_{2} $均向左闭合,电压表读数为$ U_{1} $;若$ S_{1} $、$ S_{2} $均向右闭合,电压表读数为$ U_{2} $。由此可求出$ R_{x} = $\tk{$R \frac { U _ { 1 } } { U _ { 2 } }$}。
\item 
若电源电动势$ E=1.5V $,内阻可忽略;电压表量程为$ 1V $,$ R=100 \ \Omega $。此电路可测量的$ R_{x} $的最大值为\tk{200} $ \Omega $。



\end{enumerate}


\item 
\exwhere{$ 2018 $年全国\lmd{3}卷}
一课外实验小组用如图所示的电路测量某待测电阻的阻值,图中$ R_{0} $为标准定值电阻($R _ { 0 } = 20.0 \ \Omega$);可视为理想电压表;$ S_{1} $为单刀开关,$ S_{2} $为单刀双掷开关;为电源;$ R $为滑动变阻器。采用如下步骤完成实验:
\begin{figure}[h!]
\centering
\includesvg[width=0.63\linewidth]{picture/svg/635}
\end{figure}


\begin{enumerate}
\renewcommand{\labelenumi}{\arabic{enumi}.}
% A(\Alph) a(\alph) I(\Roman) i(\roman) 1(\arabic)
%设定全局标号series=example	%引用全局变量resume=example
%[topsep=-0.3em,parsep=-0.3em,itemsep=-0.3em,partopsep=-0.3em]
%可使用leftmargin调整列表环境左边的空白长度 [leftmargin=0em]
\item
按照实验原理线路图($ a $),将图($ b $)中实物连线;
\item 
将滑动变阻器滑动端置于适当的位置,闭合$ S_{1} $;
\item 
将开关$ S_{2} $掷于$ 1 $端,改变滑动变阻器滑动端的位置,记下此时电压表的示数$ U_{1} $;然后将$ S_{2} $掷于$ 2 $端,记下此时电压表的示数$ U_{2} $;
\item 
待测电阻阻值的表达式$ R_{x}= $\tk{$\left( \frac { U _ { 2 } } { U _ { 1 } } - 1 \right) R _ { 0 }$}
(用$ R_{0} $、$ U_{1} $、$ U_{2} $表示);
\item 
重复步骤($ 3 $),得到如下数据:

\begin{table}[h!]
\centering 
\begin{tabular}{|c|c|c|c|c|c|}
\hline 
& $ 1 $ & $ 2 $ & $ 3 $ & $ 4 $ & $ 5 $
 \\
\hline
$ U_{1} /V $ & $ 0.25 $ & $ 0.30 $ & $ 0.36 $ & $ 0.40 $ & $ 0.44 $
 \\
\hline
$ U_{2} /V $ & $ 0.86 $ & $ 1.03 $ & $ 1.22 $ & $ 1.36 $ & $ 1.49 $
 \\
\hline
$ U_{2} / U_{1} $ & $ 3.44 $ & $ 3.43 $ & $ 3.39 $ & $ 3.40 $ & $ 3.39 $\\ 
\hline 
\end{tabular}
\end{table} 


\item 
利用上述$ 5 $次测量所得$\frac { U _ { 2 } } { U _ { 1 } }$的平均值,求得 $ R_{x}= $ \tk{48.2} $ \Omega $。(保留$ 1 $位小数)



\end{enumerate}

\banswer{
实物连线如图示:\\
\begin{figure}[h!]
\includesvg[width=0.23\linewidth]{picture/svg/636}
\end{figure}
}


\newpage

\item 
\exwhere{$ 2014 $年理综四川卷}
下图是测量阻值约几十欧的未知电阻$ R_{x} $的原理图,图中$ R_{0} $是保护电阻($ 10 \ \Omega $),$ R_{1} $ 是电阻箱($ 0 \sim 99.9 \ \Omega $),$ R $是滑动变阻器,$ A_{1} $和$ A_{2} $是电流表,$ E $是电源(电动势$ 100V $,内阻很小)。
\begin{figure}[h!]
\centering
\includesvg[width=0.4\linewidth]{picture/svg/637} \qquad 
\includesvg[width=0.35\linewidth]{picture/svg/638}
\end{figure}


在保证安全和满足要求的情况下,使测量范围尽可能大。实验具体步骤如下:
\begin{enumerate}
\renewcommand{\labelenumii}{(\roman{enumii})}
% A(\Alph) a(\alph) I(\Roman) i(\roman) 1(\arabic)
%设定全局标号series=example	%引用全局变量resume=example
%[topsep=-0.3em,parsep=-0.3em,itemsep=-0.3em,partopsep=-0.3em]
%可使用leftmargin调整列表环境左边的空白长度 [leftmargin=0em]
\item
连接好电路,将滑动变阻器$ R $调到最大;
\item 
闭合$ S $,从最大值开始调节电阻箱$ R_{1} $,先调$ R_{1} $为适当值,再调滑动变阻器$ R $,使$ A_{1} $示数$ I_{1} $ $ = $ $ 0.15A $,记下此时电阻箱的阻值$ R_{1} $和$ A_{2} $示数$ I_{2} $。
\item 
重复步骤($ ii $),再测量$ 6 $组$ R_{1} $和$ I_{2} $;
\item 
将实验测得的$ 7 $组数据在坐标纸上描点。

\end{enumerate}



根据实验回答以下问题:

① 现有四只供选用的电流表:

A.电流表($ 0 \sim 3mA $,内阻为$ 2.0 \ \Omega $)

B.电流表($ 0 \sim 3mA $,内阻未知)

C.电流表($ 0 \sim 0.3A $,内阻为$ 5.0 \ \Omega $

D.电流表($ 0 \sim 0.3A $,内阻未知)

$ A_{1} $应选用 \tk{D} ,$ A_{2} $应选用 \tk{C} 。

② 测得一组$ R_{1} $和$ I_{2} $值后,调整电阻箱$ R_{1} $,使其阻值变小,要使$ A_{1} $示数$ I_{1} $ $ = $ $ 0.15\ A $,应让滑动变阻器$ R $接入电路的阻值 \tk{变大} (选填“不变”、“变大”或“变小”)。

③ 在坐标纸上画出$ R_{1} $与$ I_{2} $的关系图。


④ 根据以上实验得出$ R_{x} = $ \tk{$ 31.3 $} $ \Omega $。


\banswer{
③ 关系图线如图:\\
\includesvg[width=0.23\linewidth]{picture/svg/639}

}







\end{enumerate}

