\gaokaoheader{2020}{全国\lmd{3}卷}


\gaokaoxz
\begin{enumerate}
%\renewcommand{\labelenumi}{\arabic{enumi}.}
% A(\Alph) a(\alph) I(\Roman) i(\roman) 1(\arabic)
%设定全局标号series=example	%引用全局变量resume=example
%[topsep=-0.3em,parsep=-0.3em,itemsep=-0.3em,partopsep=-0.3em]
%可使用leftmargin调整列表环境左边的空白长度 [leftmargin=0em]
\item
如图,水平放置的圆柱形光滑玻璃棒左边绕有一线圈,右边套有一金属圆环。圆环初始时静止。将图
中开关 $ S $ 由断开状态拨至连接状态,电路接通的瞬间,可观察到 \xzanswer{B} 
\begin{figure}[h!]
\centering
\includesvg[width=0.23\linewidth]{picture/svg/GZ-3-tiyou-0634}
\end{figure}


\fourchoices
{拨至 $ M $ 端或 $ N $ 端,圆环都向左运动}
{拨至 $ M $ 端或 $ N $ 端,圆环都向右运动}
{拨至 $ M $ 端时圆环向左运动,拨至 $ N $ 端时向右运动}
{拨至 $ M $ 端时圆环向右运动,拨至 $ N $ 端时向左运动}

%题目类型:选择
%题目难度:9
%题目区域:电磁感应:楞次定律
%思想方法:
%题目特征:
%题目备注:

\item
甲、乙两个物块在光滑水平桌面上沿同一直线运动,甲追上乙,并与乙发生碰撞,碰撞前后甲、乙的
速度随时间的变化如图中实线所示。已知甲的质量为 $ 1 \ kg $,则碰撞过程两物块损失的机械能为 \xzanswer{A} 
\begin{figure}[h!]
\centering
\includesvg[width=0.33\linewidth]{picture/svg/GZ-3-tiyou-0635}
\end{figure}


\fourchoices
{$ 3 \ J $}
{$ 4 \ J $}
{$ 5 \ J $}
{$ 6 \ J $}

%题目类型:选择
%题目难度:8.5
%题目区域:动量:动量守恒
%思想方法:
%题目特征:
%题目备注:

\item
“嫦娥四号”探测器于 $ 2019 $ 年 $ 1 $ 月在月球背面成功着陆,着陆前曾绕月球飞行,某段时间可认为绕月
做匀速圆周运动,圆周半径为月球半径的 $ K $ 倍。已知地球半径 $ R $ 是月球半径的 $ P $ 倍,地球质量是月球
质量的 $ Q $ 倍,地球表面重力加速度大小为 $ g $.则“嫦娥四号”绕月球做圆周运动的速率为 \xzanswer{D} 

\fourchoices
{$\sqrt{\frac{R K g}{Q P}}$}
{$\sqrt{\frac{R P K g}{Q}}$}
{$ \sqrt{\frac{R Q g}{K P}}$}
{$\sqrt{\frac{R P g}{Q K}}$}


%题目类型:选择
%题目难度:8
%题目区域:万有引力
%思想方法:比例
%题目特征:材料分析
%题目备注:






\item
如图,悬挂甲物体的细线拴牢在一不可伸长的轻质细绳上 $ O $ 点处;绳的一端固定在墙上,另一端通过
光滑定滑轮与物体乙相连。甲、乙两物体质量相等。系统平衡时,$ O $ 点两侧绳与竖直方向的夹角分别
为$ \alpha $和$ \beta $。若$ \alpha =70 ^{ \circ } $,则$ \beta $等于 \xzanswer{B} 
\begin{figure}[h!]
\centering
\includesvg[width=0.24\linewidth]{picture/svg/GZ-3-tiyou-0636}
\end{figure}


\fourchoices
{$ 45 ^{ \circ } $}
{$ 55 ^{ \circ } $}
{$ 60 ^{ \circ } $}
{$ 70 ^{ \circ } $}


%题目类型:选择
%题目难度:5
%题目区域:相互作用
%思想方法:几何:相似
%题目特征:
%题目备注:




\item
真空中有一匀强磁场,磁场边界为两个半径分别为 $ a $ 和 $ 3a $ 的同轴圆柱面,磁场的方向与圆柱轴线平行,
其横截面如图所示。一速率为 $ v $ 的电子从圆心沿半径方向进入磁场。已知电子质量为 $ m $,电荷量为 $ e $,
忽略重力。为使该电子的运动被限制在图中实线圆围成的区域内,磁场的磁感应强度最小为 \xzanswer{C} 
\begin{figure}[h!]
\centering
\includesvg[width=0.23\linewidth]{picture/svg/GZ-3-tiyou-0637}
\end{figure}


\fourchoices
{$\frac{3 m v}{2 a e}$}
{$\frac{m v}{a e}$}
{$\frac{3 m v}{4 a e}$}
{$\frac{3 m v}{5 a e}$}


%题目类型:选择
%题目难度:5
%题目区域:磁场
%思想方法:几何
%题目特征:
%题目备注:





\item
$ 1934 $ 年,约里奥—居里夫妇用$ \alpha $粒子轰击铝箔,首次产生了人工放射性同位素 \ce{X} ,反应方程为:
${ }_{2}^{4} \mathrm{He}+{ }_{13}^{27} \mathrm{Al} \rightarrow \mathrm{X}+{ }_{0}^{1} \mathrm{n}$。 \ce{X} 会衰变成原子核 \ce{Y} ,衰变方程为 $\mathrm{X} \rightarrow \mathrm{Y}+{ }_{1}^{0} \mathrm{e}$,则 \xzanswer{AC} 


\fourchoices
{\ce{X} 的质量数与 \ce{Y} 的质量数相等}
{\ce{X} 的电荷数比 \ce{Y} 的电荷数少$ 1 $}
{\ce{X} 的电荷数比${ }_{13}^{27} \mathrm{Al}$的电荷数多$ 2 $}
{\ce{X} 的质量数与${ }_{13}^{27} \mathrm{Al}$的质量数相等}

%题目类型:选择
%题目难度:8
%题目区域:原子物理
%思想方法:
%题目特征:
%题目备注:



\item
在图 \subref{2020:全国3:7a} 所示的交流电路中,电源电压的有效值为 $ 220 \ V $,理想变压器原、副线圈的匝数比为 $ 10:1 $,
$ R_{1} $、$ R_{2} $、$ R_{3} $ 均为固定电阻,$ R_{2} =10 \ \Omega $,$ R_{3} =20 \ \Omega $,各电表均为理想电表。已知电阻 $ R_{2} $ 中电流 $ i_{2} $ 随时间 $ t $变化的正弦曲线如图 \subref{2020:全国3:7b} 所示。下列说法正确的是 \xzanswer{AD} 
\begin{figure}[h!]
\centering
\begin{subfigure}{0.4\linewidth}
\centering
\includesvg[width=0.7\linewidth]{picture/svg/GZ-3-tiyou-0638} 
\caption{}\label{2020:全国3:7a}
\end{subfigure}
\hfil
\begin{subfigure}{0.4\linewidth}
\centering
\includesvg[width=0.9\linewidth]{picture/svg/GZ-3-tiyou-0639} 
\caption{}\label{2020:全国3:7b}
\end{subfigure}
\end{figure}


\fourchoices
{所用交流电的频率为$ 50 \ Hz $}
{电压表的示数为$ 100 \ V $}
{电流表的示数为$ 1.0 \ A $}
{变压器传输的电功率为$ 15.0 \ W $}

%题目类型:选择
%题目难度:6.5
%题目区域:电磁感应:变压器与高压输电
%思想方法:等效
%题目特征:
%题目备注:变压器有三大功能:变电阻、变电流、变电压。



\item
如图,$ \angle M $是锐角三角形$ PMN $最大的内角,电荷量为$ q $($ q>0 $)的点电荷固定在$ P $点。下列说法正确的是 \xzanswer{BC} 
\begin{figure}[h!]
\centering
\includesvg[width=0.18\linewidth]{picture/svg/GZ-3-tiyou-0640}
\end{figure}


\fourchoices
{沿$ MN $边,从$ M $点到$ N $点,电场强度的大小逐渐增大}
{沿$ MN $边,从$ M $点到$ N $点,电势先增大后减小}
{正电荷在$ M $点的电势能比其在$ N $点的电势能大}
{将正电荷从$ M $点移动到$ N $点,电场力所做的总功为负}


%题目类型:选择
%题目难度:5
%题目区域:电场:电场强度:电势能
%思想方法:几何
%题目特征:
%题目备注:



\gaokaosy

\item
某同学利用图 \subref{2020:全国3:9a} 所示装置验证动能定理。调整木板的倾角平衡摩擦阻力后,挂上钩码,钩码下落,
带动小车运动并打出纸带。某次实验得到的纸带及相关数据如图 \subref{2020:全国3:9b} 所示。
\begin{figure}[h!]
\centering
\begin{subfigure}{0.4\linewidth}
\centering
\includesvg[width=0.9\linewidth]{picture/svg/GZ-3-tiyou-0641} 
\caption{}\label{2020:全国3:9a}
\end{subfigure}
\hfil
\begin{subfigure}{0.53\linewidth}
\centering
\includesvg[width=0.99\linewidth]{picture/svg/GZ-3-tiyou-0642} 
\caption{}\label{2020:全国3:9b}
\end{subfigure}
\end{figure}

已知打出图 \subref{2020:全国3:9b} 中相邻两点的时间间隔为 $ 0.02 \ s $,从图 \subref{2020:全国3:9b} 给出的数据中可以得到,打出 $ B $ 点时小
车的速度大小 $ v_{B} =$ \underlinegap $m/s $,打出 $ P $ 点时小车的速度大小 $ v_{P}=$ \underlinegap $m/s $。(结果均保留 $ 2 $ 位小数)

若要验证动能定理,除了需测量钩码的质量和小车的质量外,还需要从图 \subref{2020:全国3:9b} 给出的数据中求得的物
理量为 \underlinegap 。


\tk{ $ 0.36 \quad 1.80$ \quad $ B $、$ P $之间的距离} 


%题目类型:实验
%题目难度:9
%题目区域:直线运动:能量守恒:动能定理
%思想方法:
%题目特征:
%题目备注:




\newpage
\item 
已知一热敏电阻当温度从 $ 10 \ \celsius $升至 $ 60 \ \celsius $时阻值从几千欧姆降至几百欧姆,某同学利用伏安法测量其
阻值随温度的变化关系。所用器材:电源 $ E $、开关 $ S $、滑动变阻器 $ R $(最大阻值为 $ 20 \ \Omega $)、电压表(可视
为理想电表)和毫安表(内阻约为 $ 100 \ \Omega $)。
\begin{enumerate}
%\renewcommand{\labelenumi}{\arabic{enumi}.}
% A(\Alph) a(\alph) I(\Roman) i(\roman) 1(\arabic)
%设定全局标号series=example	%引用全局变量resume=example
%[topsep=-0.3em,parsep=-0.3em,itemsep=-0.3em,partopsep=-0.3em]
%可使用leftmargin调整列表环境左边的空白长度 [leftmargin=0em]
\item
在答题卡上所给的器材符号之间画出连线,组成测量电路图。
\begin{figure}[h!]
\centering
\includesvg[width=0.43\linewidth]{picture/svg/GZ-3-tiyou-0643}
\end{figure}

\item 
实验时,将热敏电阻置于温度控制室中,记录不同温度下电压表和亳安表的示数,计算出相应的
热敏电阻阻值。若某次测量中电压表和毫安表的示数分别为 $ 5.5 \ V $ 和 $ 3.0 \ mA $,则此时热敏电阻的阻值为 \underlinegap 
$ k \Omega $(保留 $ 2 $ 位有效数字)。实验中得到的该热敏电阻阻值 $ R $ 随温度 $ t $ 变化的曲线如图 \subref{2020:全国3:10:2a} 所示。
\begin{figure}[h!]
\centering
\begin{subfigure}{0.5\linewidth}
\centering
\includesvg[width=0.96\linewidth]{picture/svg/GZ-3-tiyou-0644} 
\caption{}\label{2020:全国3:10:2a}
\end{subfigure}
\hfil
\begin{subfigure}{0.4\linewidth}
\centering
\includesvg[width=0.7\linewidth]{picture/svg/GZ-3-tiyou-0645} 
\caption{}\label{2020:全国3:10:2b}
\end{subfigure}
\end{figure}

\item 
将热敏电阻从温控室取出置于室温下,测得达到热平衡后热敏电阻的阻值为 $ 2.2 \ k\Omega $。由图 \subref{2020:全国3:10:2a} 求
得,此时室温为 \underlinegap $ \celsius $(保留 $ 3 $ 位有效数字)。
\item 
利用实验中的热敏电阻可以制作温控报警器,其电路的一部分如图 \subref{2020:全国3:10:2b} 所示。图中,$ E $ 为直流电
源(电动势为 $ 10 \ V $,内阻可忽略);当图中的输出电压达到或超过 $ 6.0 \ V $ 时,便触发报警器(图中未画出)
报警。若要求开始报警时环境温度为 $ 50 \ \celsius $,则图中 \underlinegap (填
“$ R_{1} $”或“$ R_{2} $”)应使用热敏电阻,另一固定电阻的阻值应为 \underlinegap $ k \Omega $(保留 $ 2 $ 位有效数字)。

\end{enumerate}

\tk{
\begin{enumerate}
%\renewcommand{\labelenumi}{\arabic{enumi}.}
% A(\Alph) a(\alph) I(\Roman) i(\roman) 1(\arabic)
%设定全局标号series=example	%引用全局变量resume=example
%[topsep=-0.3em,parsep=-0.3em,itemsep=-0.3em,partopsep=-0.3em]
%可使用leftmargin调整列表环境左边的空白长度 [leftmargin=0em]
\item
如图所示。 
\begin{center}
\includesvg[width=0.23\linewidth]{picture/svg/GZ-3-tiyou-0652} 	
\end{center}

\item 
$ 1.8 $ 
\item 
$ 26.0 $,官方答案为:$ 25.5 $ 
\item 
$ R_{1} \quad 1.2 $	
\end{enumerate}
} 


%题目类型:实验
%题目难度:6
%题目区域:电路
%思想方法:
%题目特征:
%题目备注:




\newpage
\item
如图,一边长为 $ l_{0} $ 的正方形金属框 $ abcd $ 固定在水平面内,空间存在方向垂直于水平面、磁感应强度大
小为 $ B $ 的匀强磁场。一长度大于 $ \sqrt{2}l_{0} $ 的均匀导体棒以速率 $ v $ 自左向右在金属框上匀速滑过,滑动过程中导
体棒始终与 $ ac $ 垂直且中点位于 $ ac $ 上,导体棒与金属框接触良好。已知导体棒单位长度的电阻为 $ r $,金属框
电阻可忽略。将导体棒与 $ a $ 点之间的距离记为 $ x $,求导体棒所受安培力的大小随 $ x $( $ 0 \leq x \leq \sqrt{2}l_{0} $ )变化的关
系式。
\begin{figure}[h!]
\flushright
\includesvg[width=0.25\linewidth]{picture/svg/GZ-3-tiyou-0646}
\end{figure}

\banswer{
$f=\left\{\begin{array}{l}\frac{2 B^{2} v}{r} x, 0 \leq x \leq \frac{\sqrt{2}}{2} l_{0} \\ \frac{2 B^{2} v}{r}\left(\sqrt{2} l_{0}-x\right), \frac{\sqrt{2}}{2} l_{0}<x \leq \sqrt{2} l_{0}\end{array}\right.$
}


%题目类型:计算
%题目难度:6
%题目区域:磁场:安培力:电磁感应:电磁感应定律
%思想方法:函数
%题目特征:
%题目备注:




\item
如图,相距 $ L=11.5 \ m $ 的两平台位于同一水平面内,二者之间用传送带相接。传送带向右匀速运动,其
速度的大小 $ v $ 可以由驱动系统根据需要设定。质量 $ m=10 \ kg $ 的载物箱(可视为质点),以初速度 $ v_{0} =5.0 \ m/s $ 自左侧平台滑上传送带。载物箱与传送带间的动摩擦因数$ \mu =0.10 $,重力加速度取 $ g=10 \ m/s^{2} $。
\begin{enumerate}
%\renewcommand{\labelenumi}{\arabic{enumi}.}
% A(\Alph) a(\alph) I(\Roman) i(\roman) 1(\arabic)
%设定全局标号series=example	%引用全局变量resume=example
%[topsep=-0.3em,parsep=-0.3em,itemsep=-0.3em,partopsep=-0.3em]
%可使用leftmargin调整列表环境左边的空白长度 [leftmargin=0em]
\item
若 $ v=4.0 \ m/s $,求载物箱通过传送带所需的时间;
\item 
求载物箱到达右侧平台时所能达到的最大速度和最小速度;
\item 
若 $ v=6.0 \ m/s $,载物箱滑上传送带$\Delta t=\frac{13}{12} \ s $ 后,传送带速度突然变为零。求载物箱从左侧平台向右
侧平台运动的过程中,传送带对它的冲量。

\end{enumerate}
\begin{figure}[h!]
\flushright
\includesvg[width=0.33\linewidth]{picture/svg/GZ-3-tiyou-0647}
\end{figure}

\banswer{
\begin{enumerate}
%\renewcommand{\labelenumi}{\arabic{enumi}.}
% A(\Alph) a(\alph) I(\Roman) i(\roman) 1(\arabic)
%设定全局标号series=example	%引用全局变量resume=example
%[topsep=-0.3em,parsep=-0.3em,itemsep=-0.3em,partopsep=-0.3em]
%可使用leftmargin调整列表环境左边的空白长度 [leftmargin=0em]
\item
$t_{1}=2.75 \ s$	
\item 
$v_{min}=\sqrt{2} \ m / s, \quad v_{max}=4 \sqrt{3} \ m / s$
\item 
$I=0$	
\end{enumerate}
}


%题目类型:计算
%题目难度:5.5
%题目区域:运动定律:动量:动量定理
%思想方法:
%题目特征:计算练习
%题目备注:



\newpage

\gaokaoxx{$ 3 - 3 $}



\item 
%选修$ 3 - 3 $
\begin{enumerate}
%\renewcommand{\labelenumi}{\arabic{enumi}.}
% A(\Alph) a(\alph) I(\Roman) i(\roman) 1(\arabic)
%设定全局标号series=example	%引用全局变量resume=example
%[topsep=-0.3em,parsep=-0.3em,itemsep=-0.3em,partopsep=-0.3em]
%可使用leftmargin调整列表环境左边的空白长度 [leftmargin=0em]
\item
如图,一开口向上的导热气缸内。用活塞封闭了一定质量的理想气体,活塞与气缸壁间
无摩擦。现用外力作用在活塞上。使其缓慢下降。环境温度保持不变,系统始终处于平衡状态。在活塞下
降过程中 \underlinegap 。(填正确答案标号。选对 $ 1 $ 个得 $ 2 $ 分。选对 $ 2 $ 个得 $ 4 $ 分,选对 $ 3 $ 个得 $ 5 $ 分;每选错$ 1 $ 个扣 $ 3 $ 分,最低得分为 $ 0 $ 分)
\begin{figure}[h!]
\centering
\includesvg[width=0.2\linewidth]{picture/svg/GZ-3-tiyou-0648}
\end{figure}

\fivechoices
{气体体积逐渐减小,内能增加}
{气体压强逐渐增大,内能不变}
{气体压强逐渐增大,放出热量}
{外界对气体做功,气体内能不变}
{外界对气体做功,气体吸收热量}

\tk{BCD} 

%题目类型:选择
%题目难度:7
%题目区域:热学:热力学第一定律
%思想方法:
%题目特征:
%题目备注:




\item 
如图,两侧粗细均匀、横截面积相等、高度均为 $ H=18 \ cm $ 的 $ U $ 型管,左管上端封闭,
右管上端开口。右管中有高 $ h_{0} =4 \ cm $ 的水银柱,水银柱上表面离管口的距离 $ l=12 \ cm $。管底水平段的体积可
忽略。环境温度为 $ T_{1} =283 \ K $。大气压强 $ P_0=76 \ cmHg $。
\begin{enumerate}
%\renewcommand{\labelenumi}{\arabic{enumi}.}
% A(\Alph) a(\alph) I(\Roman) i(\roman) 1(\arabic)
%设定全局标号series=example	%引用全局变量resume=example
%[topsep=-0.3em,parsep=-0.3em,itemsep=-0.3em,partopsep=-0.3em]
%可使用leftmargin调整列表环境左边的空白长度 [leftmargin=0em]
\item
现从右侧端口缓慢注入水银(与原水银柱之间无气隙),恰好使水银柱下端到达右管底部。此时
水银柱的高度为多少?
\item 
再将左管中密封气体缓慢加热,使水银柱上表面恰与右管口平齐,此时密封气体的温度为多少?
\end{enumerate}
\begin{figure}[h!]
\flushright
\includesvg[width=0.25\linewidth]{picture/svg/GZ-3-tiyou-0649}
\end{figure}


\banswer{
\begin{enumerate}
%\renewcommand{\labelenumi}{\arabic{enumi}.}
% A(\Alph) a(\alph) I(\Roman) i(\roman) 1(\arabic)
%设定全局标号series=example	%引用全局变量resume=example
%[topsep=-0.3em,parsep=-0.3em,itemsep=-0.3em,partopsep=-0.3em]
%可使用leftmargin调整列表环境左边的空白长度 [leftmargin=0em]
\item
$ h=12.9 \ cm $	
\item 		
$ T_{2} =363 \ K $	
\end{enumerate}
}


%题目类型:计算
%题目难度:7
%题目区域:热学:理想气体状态方程
%思想方法:
%题目特征:计算练习
%题目备注:




\end{enumerate}


\newpage

\gaokaoxx{$ 3 - 4 $}



\item 
%选修 $ 3 - 4 $
\begin{enumerate}
%\renewcommand{\labelenumi}{\arabic{enumi}.}
% A(\Alph) a(\alph) I(\Roman) i(\roman) 1(\arabic)
%设定全局标号series=example	%引用全局变量resume=example
%[topsep=-0.3em,parsep=-0.3em,itemsep=-0.3em,partopsep=-0.3em]
%可使用leftmargin调整列表环境左边的空白长度 [leftmargin=0em]
\item
如图,一列简谐横波平行于 $ x $ 轴传播,图中的实线和虚线分别为 $ t=0 $ 和 $ t=0.1 \ s $ 时的波形
图。已知平衡位置在 $ x=6 \ m $ 处的质点,在 $ 0 $ 到 $ 0.1 \ s $ 时间内运动方向不变。这列简谐波的周期为 \underlinegap $ s $,
波速为 \underlinegap $ m/s $,传播方向沿 $ x $ 轴 \underlinegap (填“正方向”或“负方向”)。
\begin{figure}[h!]
\centering
\includesvg[width=0.53\linewidth]{picture/svg/GZ-3-tiyou-0650}
\end{figure}


\tk{0.4 \quad 10 \quad 负方向} 

%题目类型:填空
%题目难度:7.5
%题目区域:机械波:波的描述
%思想方法:
%题目特征:图像分析
%题目备注:




\item 
如图,一折射率为 $ \sqrt{3} $ 的材料制作的三棱镜,其横截面为直角三角形 $ ABC $,$ \angle A=90 ^{ \circ } $,
$ \angle B=30 ^{ \circ } $。一束平行光平行于 $ BC $ 边从 $ AB $ 边射入棱镜,不计光线在棱镜内的多次反射,求 $ AC $ 边与 $ BC $ 边上
有光出射区域的长度的比值。
\begin{figure}[h!]
\flushright
\includesvg[width=0.45\linewidth]{picture/svg/GZ-3-tiyou-0651}
\end{figure}

\banswer{
如图($ a $)所示,设从$ D $点入射的光线经折射后恰好射向$ C $点,光在$ AB $边上的入射角为$ \theta 1 $,折射角为$ \theta 2 $。设从$ AD $范围入射的光折射后在$ AC $边上的入射角为$ \theta ^{\prime\prime} $,如图($ b $)所示。
\begin{figure}[h!]
\centering
\begin{subfigure}{1\linewidth}
\centering
\includesvg[width=0.85\linewidth]{picture/svg/GZ-3-tiyou-0653} 
\caption{}\label{}
\end{subfigure}
\\
\begin{subfigure}{1\linewidth}
\centering
\includesvg[width=0.85\linewidth]{picture/svg/GZ-3-tiyou-0654} 
\caption{}\label{}
\end{subfigure}
\end{figure}
易得:$\frac{A C}{C F}=2$
}

%题目类型:计算
%题目难度:6.5
%题目区域:光学
%思想方法:几何
%题目特征:
%题目备注:




\end{enumerate}





\end{enumerate}


