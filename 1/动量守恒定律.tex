\bta{动量守恒定律}



\begin{enumerate}[leftmargin=0em]
\renewcommand{\labelenumi}{\arabic{enumi}.}
% A(\Alph) a(\alph) I(\Roman) i(\roman) 1(\arabic)
%设定全局标号series=example	%引用全局变量resume=example
%[topsep=-0.3em,parsep=-0.3em,itemsep=-0.3em,partopsep=-0.3em]
%可使用leftmargin调整列表环境左边的空白长度 [leftmargin=0em]
\item
\exwhere{$ 2012 $年物理上海卷}
$ A $、$ B $两物体在光滑水平地面上沿一直线相向而行,$ A $质量为$ 5 \ kg $,速度大小为$ 10 \ m/s $,$ B $质量为$ 2 \ kg $,速度大小为$ 5 \ m/s $,它们的总动量大小为 \tk{40} $ kg \cdot m/s $;两者碰撞后,$ A $沿原方向运动,速度大小为$ 4 \ m/s $,则$ B $的速度大小为 \tk{10} $ m/s $。

\item 
\exwhere{$ 2011 $年上海卷}
光滑水平面上两小球$ a $、$ b $用不可伸长的松弛细绳相连。开始时$ a $球静止,$ b $球以一定速度运动直至绳被拉紧,然后两球一起运动,在此过程中两球的总动量 \tk{守恒} (填“守恒”或“不守恒”);机械能 \tk{不守恒} (填“守恒”或“不守恒”)。

\item 
\exwhere{$ 2013 $年上海卷}
质量为$ M $的物块静止在光滑水平桌面上,质量为$ m $的子弹以水平速度$ v_{0} $射入物块后,以水平速度$ \frac{ 2 }{ 3 } v_{0} $射出。则物块的速度为 \tk{$\frac { m v _ { 0 } } { 3 M }$} ,此过程中损失的机械能为 \tk{$\frac { ( 5 M - m ) m v _ { 0 } ^ { 2 } } { 18 M }$} 。


\item
\exwhere{$ 2015 $年上海卷}
两小孩在冰面上乘坐“碰碰车”相向运动。$ A $车总质量为$ 50 \ kg $,以$ 2 \ m/s $的速度向右运动;$ B $车总质量为$ 70 \ kg $,以$ 3 \ m/s $的速度向左运动。碰撞后,$ A $以$ 1.5 \ m/s $的速度向左运动,则$ B $的速度大小为\tk{0.5}$ m/s $,方向 \tk{左}(选填“左”或“右”)。

\item 
\exwhere{$ 2017 $年新课标 \lmd{1} 卷}
将质量为$ 1.00 \ kg $的模型火箭点火升空,$ 50g $燃烧的燃气以大小为$ 600 $ $ m/s $的速度从火箭喷口在很短时间内喷出。在燃气喷出后的瞬间,火箭的动量大小为(喷出过程中重力和空气阻力可忽略) \xzanswer{A} 

\fourchoices
{$ 30 \ kg \cdot m/s $ }
{$ 5.7 \times 102 $ $ kg \cdot m/s $}
{$ 6.0 \times 102 $ $ kg m/s $ }
{$ 6.3 \times 102 $ $ kg \cdot m/s $}

\item 
\exwhere{$ 2016 $年上海卷}
如图,粗糙水平面上,两物体$ A $、$ B $以轻绳相连,在恒力$ F $作用下做匀速运动。某时刻轻绳断开,$ A $在$ F $牵引下继续前进,$ B $最后静止。则在$ B $静止前,$ A $和$ B $组成的系统动量 \tk{守恒}(选填:“守恒”或“不守恒“)。在$ B $静止后,$ A $和$ B $组成的系统动量 \tk{不守恒} 。(选填:“守恒”或“不守恒“)
\begin{figure}[h!]
\centering
\includesvg[width=0.23\linewidth]{picture/svg/516}
\end{figure}



\item 
\exwhere{$ 2016 $年天津卷}
如图所示,方盒$ A $静止在光滑的水平面上,盒内有一个小滑块$ B $,盒的质量是滑块质量的$ 2 $倍,滑块与盒内水平面间的动摩擦因数为$ \mu $;若滑块以速度$ v $开始向左运动,与盒的左右壁发生无机械能损失的碰撞,滑块在盒中来回运动多次,最终相对于盒静止,则此时盒的速度大小为 \tk{$ \frac{v}{3} $} ;滑块相对于盒运动的路程为 \tk{$\frac { v ^ { 2 } } { 3 \mu g }$} 。
\begin{figure}[h!]
\centering
\includesvg[width=0.23\linewidth]{picture/svg/517}
\end{figure}


\item 
\exwhere{$ 2014 $年理综重庆卷}
一弹丸在飞行到距离地面$ 5 \ m $高时仅有水平速度$ v=2 \ m/s $,爆炸成为甲、乙两块水平飞出,甲、乙的质量比为$ 3:1 $。不计质量损失,取重力加速度$ g=10 \ m/s^{2} $。则下列图中两块弹片飞行的轨迹可能正确的是 \xzanswer{B} 
\begin{figure}[h!]
\centering
\includesvg[width=0.83\linewidth]{picture/svg/518}
\end{figure}



\item 
\exwhere{$ 2012 $年理综重庆卷}
质量为$ m $的人站在质量为$ 2 \ m $的平板小车上,以共同的速度在水平地面上沿直线前行,车受地面阻力的大小与车对地面压力的大小成正比。当车速为$ v_{0} $时,人从车上以相对于地面大小为$ v_{0} $的速度水平向后跳下。跳离瞬间地面阻力的冲量忽略不计,则能正确表示车运动的$ v - t $图象为 \xzanswer{B} 
\begin{figure}[h!]
\centering
\includesvg[width=0.83\linewidth]{picture/svg/519}
\end{figure}



\item
\exwhere{$ 2012 $年理综天津卷}
$ 10 $($ 16 $分)、如图所示,水平地面上固定有高为$ h $的平台,台面上固定有光滑坡道,坡道顶端距台面高也为$ h $,坡道底端与台面相切。小球$ A $从坡道顶端由静止开始滑下,到达水平光滑的台面后与静止在台面上的小球$ B $发生碰撞,并粘连在一起,共同沿台面滑行并从台面边缘飞出,落地点与飞出点的水平距离恰好为台高的一半。两球均可视为质点,忽略空气阻力,重力加速度为$ g $。求:
\begin{enumerate}
\renewcommand{\labelenumi}{\arabic{enumi}.}
% A(\Alph) a(\alph) I(\Roman) i(\roman) 1(\arabic)
%设定全局标号series=example	%引用全局变量resume=example
%[topsep=-0.3em,parsep=-0.3em,itemsep=-0.3em,partopsep=-0.3em]
%可使用leftmargin调整列表环境左边的空白长度 [leftmargin=0em]
\item
小球$ A $刚滑至水平台面的速度$ v_{A} $;
\item 
$ A $、$ B $两球的质量之比$ m_{A} : m_{B} $.

\end{enumerate}
\begin{figure}[h!]
\flushright
\includesvg[width=0.25\linewidth]{picture/svg/520}
\end{figure}

\banswer{
\begin{enumerate}
\renewcommand{\labelenumi}{\arabic{enumi}.}
% A(\Alph) a(\alph) I(\Roman) i(\roman) 1(\arabic)
%设定全局标号series=example	%引用全局变量resume=example
%[topsep=-0.3em,parsep=-0.3em,itemsep=-0.3em,partopsep=-0.3em]
%可使用leftmargin调整列表环境左边的空白长度 [leftmargin=0em]
\item
$v _ { A } = \sqrt { 2 g h }$
\item 
$m _ { A }: m _ { B } = 1: 3$


\end{enumerate}


}


\item 
\exwhere{$ 2014 $年理综北京卷}
如图所示,竖直平面内的四分之一圆弧轨道下端与水平桌面相切,小滑块$ A $和$ B $分别静止在圆弧轨道的最高点和最低点。现将$ A $无初速度释放,$ A $与$ B $碰撞后结合为一个整体,并沿桌面滑动。已知圆弧轨道光滑,半径$ R=0.2\ m;A $和$ B $的质量相等;$ A $和$ B $整体与桌面之间的动摩擦因数$ \mu =0.2 $。取重力加速度$ g=10 \ m/s^{2} $。求$ : $
\begin{enumerate}
\renewcommand{\labelenumi}{\arabic{enumi}.}
% A(\Alph) a(\alph) I(\Roman) i(\roman) 1(\arabic)
%设定全局标号series=example	%引用全局变量resume=example
%[topsep=-0.3em,parsep=-0.3em,itemsep=-0.3em,partopsep=-0.3em]
%可使用leftmargin调整列表环境左边的空白长度 [leftmargin=0em]
\item
碰撞前瞬间$ A $的速率$ v $;
\item 
碰撞后瞬间$ A $和$ B $整体的速率$ v ^{\prime} $;
\item 
$ A $和$ B $整体在桌面上滑动的距离$ l $.
\end{enumerate}
\begin{figure}[h!]
\flushright
\includesvg[width=0.35\linewidth]{picture/svg/521}
\end{figure}



\banswer{
\begin{enumerate}
\renewcommand{\labelenumi}{\arabic{enumi}.}
% A(\Alph) a(\alph) I(\Roman) i(\roman) 1(\arabic)
%设定全局标号series=example	%引用全局变量resume=example
%[topsep=-0.3em,parsep=-0.3em,itemsep=-0.3em,partopsep=-0.3em]
%可使用leftmargin调整列表环境左边的空白长度 [leftmargin=0em]
\item
$ 2\ m/s $
\item 
$ 2\ m/s $
\item 
$ 0.25\ m $
\end{enumerate}


}


\item 
\exwhere{$ 2011 $年理综四川卷}
随着机动车数量的增加,交通安全问题日益凸显。分析交通违法事例,将警示我们遵守交通法规,珍惜生命。一货车严重超载后的总质量为$ 49\ t $,以$ 54 \ km/h $的速率匀速行驶。发现红灯时司机刹车,货车即做匀减速直线运动,加速度的大小为$ 2.5 \ m/s^{2} $(不超载时则为$ 5 \ m/s^{2} $)。
\begin{enumerate}
\renewcommand{\labelenumi}{\arabic{enumi}.}
% A(\Alph) a(\alph) I(\Roman) i(\roman) 1(\arabic)
%设定全局标号series=example	%引用全局变量resume=example
%[topsep=-0.3em,parsep=-0.3em,itemsep=-0.3em,partopsep=-0.3em]
%可使用leftmargin调整列表环境左边的空白长度 [leftmargin=0em]
\item
若前方无阻挡,问从刹车到停下来此货车在超载及不超载时分别前进多远?
\item 
若超载货车刹车时正前方$ 25 \ m $处停着总质量为$ 1t $的轿车,两车将发生碰撞,设相互作用$ 0.1s $后获得相同速度,问货车对轿车的平均冲力多大?

\end{enumerate}


\banswer{
\begin{enumerate}
\renewcommand{\labelenumi}{\arabic{enumi}.}
% A(\Alph) a(\alph) I(\Roman) i(\roman) 1(\arabic)
%设定全局标号series=example	%引用全局变量resume=example
%[topsep=-0.3em,parsep=-0.3em,itemsep=-0.3em,partopsep=-0.3em]
%可使用leftmargin调整列表环境左边的空白长度 [leftmargin=0em]
\item
超载$ 45\ m $;不超载$ 25\ m $。
\item 
$f = 9.8 \times 10 ^ { 4 } \mathrm { N }$

\end{enumerate}


}


\newpage
\item 
\exwhere{$ 2014 $年理综天津卷}
如图所示,水平地面上静止放置一辆小车$ A $,质量$ m_{A} =4 \ kg $,上表面光滑,小车与地面间的摩擦力极小,可以忽略不计.可视为质点的物块$ B $置于$ A $的最右端,$ B $的质量$ m_{B} =2 \ kg $.现对$ A $施加一个水平向右的恒力$ F=10 \ N $,$ A $运动一段时间后,小车左端固定的挡板$ B $发生碰撞,碰撞时间极短,碰后$ A $、$ B $粘合在一起,共同在$ F $的作用下继续运动,碰撞后经时间$ t=0.6\ s $,二者的速度达到$ v_t=2 \ m/s $.求:
\begin{enumerate}
\renewcommand{\labelenumi}{\arabic{enumi}.}
% A(\Alph) a(\alph) I(\Roman) i(\roman) 1(\arabic)
%设定全局标号series=example	%引用全局变量resume=example
%[topsep=-0.3em,parsep=-0.3em,itemsep=-0.3em,partopsep=-0.3em]
%可使用leftmargin调整列表环境左边的空白长度 [leftmargin=0em]
\item
$ A $开始运动时加速度$ a $的大小;
\item 
$ A $、$ B $碰撞后瞬间的共同速度$ v $的大小;
\item 
$ A $的上表面长度$ l $.

\end{enumerate}
\begin{figure}[h!]
\flushright
\includesvg[width=0.25\linewidth]{picture/svg/522}
\end{figure}


\banswer{
\begin{enumerate}
\renewcommand{\labelenumi}{\arabic{enumi}.}
% A(\Alph) a(\alph) I(\Roman) i(\roman) 1(\arabic)
%设定全局标号series=example	%引用全局变量resume=example
%[topsep=-0.3em,parsep=-0.3em,itemsep=-0.3em,partopsep=-0.3em]
%可使用leftmargin调整列表环境左边的空白长度 [leftmargin=0em]
\item
$ 2.5\ m/s^{2} $
\item 
$ 1\ m/s $
\item 
$ 0.45\ m $

\end{enumerate}


}



\item 
\exwhere{$ 2017 $年天津卷}
如图所示,物块$ A $和$ B $通过一根轻质不可伸长的细绳连接,跨放在质量不计的光滑定滑轮两侧,质量分别为$ m_{A} =2 $ $ kg $、$ m_{B} =1 $ $ kg $。初始时$ A $静止与水平地面上,$ B $悬于空中。先将$ B $竖直向上再举高$ h=1.8 $ $ m $(未触及滑轮)然后由静止释放。一段时间后细绳绷直,$ A $、$ B $以大小相等的速度一起运动,之后$ B $恰好可以和地面接触。取$ g=10 $ $ m/s^{2} $。
\begin{enumerate}
\renewcommand{\labelenumi}{\arabic{enumi}.}
% A(\Alph) a(\alph) I(\Roman) i(\roman) 1(\arabic)
%设定全局标号series=example	%引用全局变量resume=example
%[topsep=-0.3em,parsep=-0.3em,itemsep=-0.3em,partopsep=-0.3em]
%可使用leftmargin调整列表环境左边的空白长度 [leftmargin=0em]
\item
$ B $从释放到细绳绷直时的运动时间$ t $;
\item 
$ A $的最大速度$ v $的大小;
\item 
初始时$ B $离地面的高度$ H $。

\end{enumerate}
\begin{figure}[h!]
\flushright
\includesvg[width=0.1\linewidth]{picture/svg/523}
\end{figure}


\banswer{
\begin{enumerate}
\renewcommand{\labelenumi}{\arabic{enumi}.}
% A(\Alph) a(\alph) I(\Roman) i(\roman) 1(\arabic)
%设定全局标号series=example	%引用全局变量resume=example
%[topsep=-0.3em,parsep=-0.3em,itemsep=-0.3em,partopsep=-0.3em]
%可使用leftmargin调整列表环境左边的空白长度 [leftmargin=0em]
\item
$ t=0.6\ s $
\item 
$ v=2\ m/s $
\item 
$ H=0.6\ m $

\end{enumerate}


}


\item 
\exwhere{$ 2019 $年物理江苏卷}
质量为$ M $的小孩站在质量为$ m $的滑板上,小孩和滑板均处于静止状态,忽略滑板与地面间的摩擦.小孩沿水平方向跃离滑板,离开滑板时的速度大小为$ v $,此时滑板的速度大小为 \xzanswer{B} 

\fourchoices
{$ \frac { m } { M } v $}
{$ \frac { M } { m } v $}
{$ \frac { m } { m + M } v $}
{$ \frac { M } { m + M } v $}








\end{enumerate}



