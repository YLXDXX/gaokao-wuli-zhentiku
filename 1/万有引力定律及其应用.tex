\bta{万有引力定律及其应用}

\begin{enumerate}[leftmargin=0em]
\renewcommand{\labelenumi}{\arabic{enumi}.}
% A(\Alph) a(\alph) I(\Roman) i(\roman) 1(\arabic)
%设定全局标号series=example	%引用全局变量resume=example
%[topsep=-0.3em,parsep=-0.3em,itemsep=-0.3em,partopsep=-0.3em]
%可使用leftmargin调整列表环境左边的空白长度 [leftmargin=0em]
\item
\exwhere{$ 2014 $年物理上海卷}
动能相等的两人造地球卫星$ A $、$ B $的轨道半径之比$ R_A : R_B= 1 : 2 $,它们的角速度之比$ \omega _A : \omega _B= $ \tk{$ 2\sqrt{2}:1 $} ,质量之比$ m_{A} : m_{B} = $ \tk{$ 1:2 $} .



\item 
\exwhere{$ 2015 $年上海卷}
两靠得较近的天体组成的系统称为双星,它们以两者连线上某点为圆心做匀速圆周运动,因而不至于由于引力作用而吸引在一起。设两天体的质量分别为$ m_{1} $和$ m_{2} $,则它们的轨道半径之比$ R _{m1} : R _{m2} = $\tk{$ m_{2} : m_{1} $};速度之比$ v _{m1} : v _{m2} = $\tk{$ m_{2} : m_{1} $}。

\item 
\exwhere{$ 2011 $年理综福建卷}
“嫦娥二号”是我国月球探测第二期工程的先导星。若测得“嫦娥二号”在月球(可视为密度均匀的球体)表面附近圆形轨道运行的周期$ T $,已知引力常数$ G $,半径为$ R $的球体体积公式$V = \frac { 4 } { 3 } \pi R ^ { 3 }$,则可估算月球的 \xzanswer{A} 

\fourchoices
{密度 }
{质量}
{半径 }
{自转周期}



\item 
\exwhere{$ 2012 $年理综山东卷}
$ 2011 $年$ 11 $月$ 3 $日,“神舟八号”飞船与“天宫一号”目标飞行器成功实施了首次交会对接。任务完成后“天宫一号”经变轨升到更高的轨道,等待与“神舟九号”交会对接。变轨前和变轨完成后“天宫一号”的运行轨道均可视为圆轨道,对应的轨道半径分别为$ R_{1} $、$ R_{2} $,线速度大小分别为$ v_{1} $、$ v_{2} $。则$\frac { v _ { 1 } } { v _ { 2 } }$等于 \xzanswer{B} 

\fourchoices
{$\sqrt { \frac { R _ { 1 } ^ { 3 } } { R _ { 2 } ^ { 3 } } } \quad$}
{$\sqrt { \frac { R _ { 2 } } { R _ { 1 } } } \quad$}
{$\frac { R _ { 2 } ^ { 2 } } { R _ { 1 } ^ { 2 } } \quad$}
{$\frac { R _ { 2 } } { R _ { 1 } }$}



\item 
\exwhere{$ 2012 $年理综福建卷}
一卫星绕某一行星表面附近做匀速圆周运动,其线速度大小为$ v_{0} $假设宇航员在该行星表面上用弹簧测力计测量一质量为$ m $的物体重力,物体静止时,弹簧测力计的示数为$ N_{0} $,已知引力常量为$ G $,则这颗行星的质量为 \xzanswer{B} 

\fourchoices
{$ \frac { m v ^ { 2 } } { G N } $}
{$ \frac { m v ^ { 4 } } { G N } $}
{$ \frac { N v ^ { 2 } } { G m } $}
{$ \frac { N v ^ { 4 } } { G m } $}


\item 
\exwhere{$ 2014 $年理综天津卷}
研究表明,地球自转在逐渐变慢,$ 3 $亿年前地球自转的周期约为$ 22 $小时.假设这种趋势会持续下去,地球的其他条件都不变,未来人类发射的地球同步卫星与现在的相比 \xzanswer{A} 

\fourchoices
{距地面的高度变大}
{向心加速度变大}
{线速度变大}
{角速度变大}



\item 
\exwhere{$ 2014 $年理综广东卷}
如图所示,飞行器$ P $ 绕某星球做匀速圆周运动,星球相对飞行器的角度为$ \theta $,下列说法正确的是 \xzanswer{AC} 
\begin{figure}[h!]
\centering
\includesvg[width=0.23\linewidth]{picture/svg/587}
\end{figure}

\fourchoices
{轨道半径越大,周期越长}
{轨道半径越大,速度越大}
{若测得周期和张角,可得到星球的平均密度}
{若测得周期和轨道半径,可得到星球的平均密度}




\item 
\exwhere{$ 2015 $年理综北京卷}
假设地球和火星都绕太阳做匀速圆周运动,已知地球到太阳的距离小于火星到太阳的距离,那么 \xzanswer{D} 

\fourchoices
{地球公转周期大于火星的公转周期}
{地球公转的线速度小于火星公转的线速度}
{地球公转的加速度小于火星公转的加速度}
{地球公转的角速度大于火星公转的角速度}

\item 
\exwhere{$ 2015 $年理综重庆卷}
宇航员王亚平在“天宫$ 1 $号”飞船内进行了我国首次太空授课,演示了一些完全失重状态下的物理现象。若飞船质量为$ m $,距地面高度为$ h $,地球质量为$ M $,半径为$ R $,引力常量为$ G $,则飞船所在处的重力加速度大小为 \xzanswer{B} 

\fourchoices
{$ 0 $}
{$ \frac { G M } { ( R + h ) ^ { 2 } } $}
{$ \frac { G M m } { ( R + h ) ^ { 2 } } $}
{$ \frac { G M } { h ^ { 2 } } $}


\item
$ 2015 $年江苏卷$ 3 $. 过去几千年来,人类对行星的认识与研究仅限于太阳系内,行星“$ 51 $ $ peg $ $ b $”的发现拉开了研究太阳系外行星的序幕. “$ 51peg $ $ b $”绕其中心恒星做匀速圆周运动,周期约为 $ 4 $ 天,轨道半径约为地球绕太阳运动半径的 $ 1/20 $. 该中心恒星与太阳的质量比约为 \xzanswer{B} 


\fourchoices
{$ 1/10 $}
{$ 1 $}
{$ 5 $}
{$ 10 $}




\item 
\exwhere{$ 2015 $年海南卷}
若在某行星和地球上相对于各自的水平地面附近相同的高度处、以相同的速率平抛一物体,它们在水平方向运动的距离之比为$ 2:\sqrt{7} $。已知该行星质量约为地球的$ 7 $倍,地球的半径为$ R $,由此可知,该行星的半径为 \xzanswer{C} 

\fourchoices
{$ \frac { 1 } { 2 } R $}
{$ \frac { 7 } { 2 } R $}
{$ 2 \mathrm { R } $}
{$ \frac { \sqrt { 7 } } { 2 } R $}

\item 
\exwhere{$ 2014 $年物理海南卷}
设地球自转周期为$ T $,质量为$ M $,引力常量为$ G $,假设地球可视为质量均匀分布的球体,半径为$ R $。同一物体在南极和赤道水平面上静止时所受到的支持力之比为 \xzanswer{A} 

\fourchoices
{$ \frac { G M T ^ { 2 } } { G M T ^ { 2 } - 4 \pi ^ { 2 } R ^ { 3 } } $}
{$ \frac { G M T ^ { 2 } } { G M T ^ { 2 } + 4 \pi ^ { 2 } R ^ { 3 } } $}
{$ \frac { G M T ^ { 2 } - 4 \pi ^ { 2 } R ^ { 3 } } { G M T ^ { 2 } } $}
{$ \frac { G M T ^ { 2 } + 4 \pi ^ { 2 } R ^ { 3 } } { G M T ^ { 2 } } $}




\item 
\exwhere{$ 2013 $年浙江卷}
如图所示,三颗质量均为$ m $的地球同步卫星等间隔分布在半径为$ r $的圆轨道上,设地球质量为$ M $,半径为$ R $。下列说法正确的是 \xzanswer{BC} 
\begin{figure}[h!]
\centering
\includesvg[width=0.2\linewidth]{picture/svg/588}
\end{figure}

\fourchoices
{地球对一颗卫星的引力大小为$\frac { G M m } { ( r - R ) ^ { 2 } }$}
{一颗卫星对地球的引力大小为$\frac { G M m } { r ^ { 2 } }$}
{两颗卫星之间的引力大小为$\frac { G m ^ { 2 } } { 3 r ^ { 2 } }$}
{三颗卫星对地球引力的合力大小为$\frac {3 G m ^ { 2 } } { r ^ { 2 } }$}


\item
\exwhere{$ 2013 $年全国卷大纲卷}
“嫦娥一号”是我国首次发射的探月卫星,它在距月球表面高度为$ 200 \ km $的圆形轨道上运行,运行周期为$ 127 $分钟。已知引力常量$G = 6.67 \times 10 ^ { - 11 } \mathrm { N } \cdot \mathrm { m } ^ { 2 } / \mathrm { kg } ^ { 2 }$,月球半径约为$1.74 \times 10 ^ { 3 }\ \mathrm { km }$,利用以上数据估算月球的质量约为 \xzanswer{D} 

\fourchoices
{$ 8.1 \times 10 ^ { 10 }\ \mathrm { kg } $}
{$ 7.4 \times 10 ^ { 13 }\ \mathrm { kg } $}
{$ 5.4 \times 10 ^ { 19 }\ \mathrm { kg } $}
{$ 7.4 \times 10 ^ { 22 }\ \mathrm { kg } $}

\item 
\exwhere{$ 2011 $年新课标卷}
卫星电话信号需要通过地球同步卫星传送.如果你与同学在地面上用卫星电话通话.则从你发出信号至对方接收到信号所需最短时间最接近于(可能用到的数据:月球绕地球运动的轨道半径约为$ 3.8 \times 10^5 \ km $,运行周期约为$ 27 $天,地球半径约为$ 6400 \ km $,无线电信号的传播速度为$ 3 \times 10^8 \ m/s $.) \xzanswer{B} 

\fourchoices
{$ 0.1\ s $ }
{$ 0.25\ s $}
{$ 0.5\ s $ }
{$ 1\ s $}



\item 
\exwhere{$ 2018 $年浙江卷($ 4 $月选考)}
土星最大的卫星叫“泰坦”,每$ 16 $天绕土星一周,其公转轨道半径为$ 1.2 \times 10^6 $ $ km $。已知引力常量$G = 6.67 \times 10 ^ { - 11 } \mathrm { N } \cdot \mathrm { m } ^ { 2 } / \mathrm { kg } ^ { 2 }$,则土星的质量约为 \xzanswer{B} 

\fourchoices
{$ 5 \times 10^{17} $ $ kg $}
{$ 5 \times 10^{26} $ $ kg $}
{$ 7 \times 10^{33} $ $ kg $}
{$ 4 \times 10^{36} $ $ kg $}



\item 
\exwhere{$ 2018 $年全国\lmd{1}卷}
$ 2017 $年,人类第一次直接探测到来自双中子星合并的引力波。根据科学家们复原的过程,在两颗中子星合并前约$ 100\ s $时,它们相距约$ 400\ km $,绕二者连线上的某点每秒转动$ 12 $圈。将两颗中子星都看作是质量均匀分布的球体,由这些数据、万有引力常量并利用牛顿力学知识,可以估算出这一时刻两颗中子星 \xzanswer{BC} 

\fourchoices
{质量之积}
{质量之和}
{速率之和 }
{各自的自转角速度}



\item 
\exwhere{$ 2018 $年全国\lmd{2}卷}
$ 2018 $年$ 2 $月,我国$ 500 $ $ m $口径射电望远镜(天眼)发现毫秒脉冲星“$ J0318+0253 $”,其自转周期$ T=5.19 $ $ ms $,假设星体为质量均匀分布的球体,已知万有引力常量为$6.67 \times 10 ^ { - 11 } \mathrm { N } \cdot \mathrm { m } ^ { 2 } / \mathrm { kg } ^ { 2 }$。以周期$ T $稳定自转的星体的密度最小值约为 \xzanswer{C} 
\fourchoices
{$5 \times 10 ^ { 9 } \ \mathrm { kg } / \mathrm { m } ^ { 3 }$}
{$5 \times 10 ^ { 12 } \ \mathrm { kg } / \mathrm { m } ^ { 3 }$}
{$5 \times 10 ^ { 15 } \ \mathrm { kg } / \mathrm { m } ^ { 3 }$}
{$5 \times 10 ^ { 18 } \ \mathrm { kg } / \mathrm { m } ^ { 3 }$}



\item 
\exwhere{$ 2018 $年北京卷}
若想检验“使月球绕地球运动的力”与“使苹果落地的力”遵循同样的规律,在已知月地距离约为地球半径$ 60 $倍的情况下,需要验证 \xzanswer{B} 

\fourchoices
{地球吸引月球的力约为地球吸引苹果的力的$ 1/60^2 $}
{月球公转的加速度约为苹果落向地面加速度的$ 1/60^2 $}
{自由落体在月球表面的加速度约为地球表面的$ 1/6 $}
{苹果在月球表面受到的引力约为在地球表面的$ 1/60 $}


\item 
\exwhere{$ 2012 $年理综新课标卷}
假设地球是一半径为$ R $、质量分布均匀的球体。一矿井深度为$ d $。已知质量分布均匀的球壳对壳内物体的引力为零。矿井底部和地面处的重力加速度大小之比为 \xzanswer{A} 
\fourchoices
{$ 1 - \frac { d } { R } $}
{$ 1 + \frac { d } { R } $}
{$ \left( \frac { R - d } { R } \right) ^ { 2 } $}
{$ \left( \frac { R } { R - d } \right) ^ { 2 } $}
\item 
\exwhere{$ 2012 $年理综重庆卷}
冥王星与其附近的星体卡戎可视为双星系统,质量比约为$ 7 $:$ 1 $,同时绕它们连线上某点$ O $做匀速圆周运动。由此可知冥王星绕$ O $点运动的 \xzanswer{A} 

\fourchoices
{轨道半径约为卡戎的$ 1/7 $}
{角速度大小约为卡戎的$ 1/7 $}
{线速度大小约为卡戎的$ 7 $倍}
{向心力小约为卡戎的$ 7 $倍}

\item 
\exwhere{$ 2014 $年理综新课标$ \lmd{2} $卷}
假设地球可视为质量均匀分布的球体,已知地球表面的重力加速度在两极的大小为$ g_{0} $,在赤道的大小为$ g $;地球自转的周期为$ T $,引力常数为$ G $,则地球的密度为 \xzanswer{B} 

\fourchoices
{$\frac { 3 \pi } { G T ^ { 2 } } \cdot \frac { g _ { 0 } - g } { g _ { 0 } } \quad$}
{$\frac { 3 \pi } { G T ^ { 2 } } \cdot \frac { g _ { 0 } } { g _ { 0 } - g } \quad$}
{$\frac { 3 \pi } { G T ^ { 2 } } \quad$}
{$\frac { 3 \pi } { G T ^ { 2 } } \cdot \frac { g _ { 0 } } { g }$}


\item 
\exwhere{$ 2017 $年北京卷}
利用引力常量$ G $和下列某一组数据,不能计算出地球质量的是 \xzanswer{D} 

\fourchoices
{地球的半径及重力加速度(不考虑地球自转)}
{人造卫星在地面附近绕地球做圆周运动的速度及周期}
{月球绕地球做圆周运动的周期及月球与地球间的距离}
{地球绕太阳做圆周运动的周期及地球与太阳间的距离}


\item 
\exwhere{$ 2017 $年天津卷}
我国自主研制的首艘货运飞船“天舟一号”发射升空后,与已经在轨运行的“天宫二号”成功对接形成组合体。假设组合体在距地面高度为$ h $的圆形轨道上绕地球做匀速圆周运动,已知地球半径为$ R $,地球表面重力加速度为$ g $,且不考虑地球自转的影响。则组合体运动的线速度大小为\tk{$R \sqrt { \frac { g } { R + h } }$},向心加速度大小为\tk{$\frac { R ^ { 2 } } { ( R + h ) ^ { 2 } } g$}。



\item 
\exwhere{$ 2017 $年海南卷}
已知地球质量为月球质量的$ 81 $倍,地球半径约为月球半径的$ 4 $倍。若在月球和地球表面同样高度处,以相同的初速度水平抛出物体, 抛出点与落地点间的水平距离分别为$ s_{ \text{月} } $和$ s_{ \text{地} } $, 则$ s_{ \text{月} }:s_{ \text{地} } $约为 \xzanswer{A} 

\fourchoices
{$ 9:4 $ }
{$ 6:1 $}
{$ 3:2 $ }
{$ 1:1 $}



\item 
\exwhere{$ 2016 $年海南卷}
通过观察冥王星的卫星,可以推算出冥王星的质量。假设卫星绕冥王星做匀速圆周运动,除了引力常量外,至少还需要两个物理量才能计算出冥王星的质量。这两个物理量可以是 \xzanswer{AD} 

\fourchoices
{卫星的速度和角速度}
{卫星的质量和轨道半径}
{卫星的质量和角速度}
{卫星的运行周期和轨道半径}







\newpage
\item 
\exwhere{$ 2014 $年理综北京卷}
万有引力定律揭示了天体运动规律与地上物体运动规律具有内在的一致性。
\begin{enumerate}
\renewcommand{\labelenumi}{\arabic{enumi}.}
% A(\Alph) a(\alph) I(\Roman) i(\roman) 1(\arabic)
%设定全局标号series=example	%引用全局变量resume=example
%[topsep=-0.3em,parsep=-0.3em,itemsep=-0.3em,partopsep=-0.3em]
%可使用leftmargin调整列表环境左边的空白长度 [leftmargin=0em]
\item
用弹簧秤称量一个相对于地球静止的小物体的重量,随称量位置的变化可能会有不同的结果。已知地球质量为$ M $,自转周期为$ T $,万有引力常量为$ G $。将地球视为半径为$ R $、质量均匀分布的球体,不考虑空气的影响。设在地球北极地面称量时,弹簧秤的读数是$ F_{0} $.

\begin{enumerate}
\renewcommand{\labelenumiii}{\alph{enumiii}.}
% A(\Alph) a(\alph) I(\Roman) i(\roman) 1(\arabic)
%设定全局标号series=example	%引用全局变量resume=example
%[topsep=-0.3em,parsep=-0.3em,itemsep=-0.3em,partopsep=-0.3em]
%可使用leftmargin调整列表环境左边的空白长度 [leftmargin=0em]
\item
若在北极上空高出地面$ h $处称量,弹簧秤读数为$ F_{1} $,求比值 $\frac { F _ { 1 } } { F _ { 0 } }$ 的表达式,并就$ h=1.0 \% R $的情形算出具体数值(计算结果保留两位有效数字);
\item 
若在赤道地面称量,弹簧秤读数为$ F_{2} $,求比值 $\frac { F _ { 2 } } { F _ { 0 } }$ 的表达式。



\end{enumerate}


\item 
设想地球绕太阳公转的圆周轨道半径为$ r $、太阳的半径为$ Rs $和地球的半径$ R $三者均减小为现在的$ 1.0 \% $,而太阳和地球的密度均匀且不变。仅考虑太阳和地球之间的相互作用,以现实地球的$ 1 $年为标准,计算“设想地球”的一年将变为多长?



\end{enumerate}


\banswer{
\begin{enumerate}
\renewcommand{\labelenumi}{\arabic{enumi}.}
% A(\Alph) a(\alph) I(\Roman) i(\roman) 1(\arabic)
%设定全局标号series=example	%引用全局变量resume=example
%[topsep=-0.3em,parsep=-0.3em,itemsep=-0.3em,partopsep=-0.3em]
%可使用leftmargin调整列表环境左边的空白长度 [leftmargin=0em]
\item
a. $\frac { F _ { 1 } } { F _ { 0 } } = 0.98 \quad$ b. $\frac { F _ { 2 } } { F _ { 0 } } = 1 - \frac { 4 \pi ^ { 2 } R ^ { 3 } } { T ^ { 2 } G M }$
\item 
不变

\end{enumerate}


}



\item 
\exwhere{$ 2015 $年广东卷}
在星球表面发射探测器,当发射速度为$ v $时,探测器可绕星球表面做匀速圆周运动;当发射速度达到$ \sqrt{2} v $时,可摆脱星球引力束缚脱离该星球,已知地球、火星两星球的质量比约为$ 10 : 1 $,半径比约为$ 2 : 1 $,下列说法正确的有 \xzanswer{BD} 

\fourchoices
{探测器的质量越大,脱离星球所需的发射速度越大}
{探测器在地球表面受到的引力比在火星表面的大}
{探测器分别脱离两星球所需要的发射速度相等}
{探测器脱离星球的过程中,势能逐渐增大}

\newpage
\item 
\exwhere{$ 2015 $年理综四川卷}
登上火星是人类的梦想,“嫦娥之父”欧阳自远透露:中国计划于$ 2020 $年登陆火星。地球和火星是公转视为匀速圆周运动。忽略行星自转影响:根据下表,火星和地球相比 \xzanswer{B} 
\begin{table}[h!]
\centering 
\begin{tabular}{|c|c|c|c|}
\hline 
行星 & 半径$ /m $ & 质量$ /kg $ & 轨道半径$ /m $
 \\
\hline
地球 & $ 6.4 \times 10^6 $ & $ 6.0 \times 10^{24} $ & $ 1.5 \times 10^{11} $
 \\
\hline
火星 & $ 3.4 \times 1066 $ & $ 6.4 \times 10^{23} $ & $ 2.3 \times 10^{11} $\\ 
\hline 
\end{tabular}
\end{table} 

\fourchoices
{火星的公转周期较小}
{火星做圆周运动的加速度较小}
{火星表面的重力加速度较大}
{火星的第一宇宙速度较大}
\item 
\exwhere{$ 2019 $年物理全国\lmd{2}卷}
$ 2019 $年$ 1 $月,我国嫦娥四号探测器成功在月球背面软着陆,在探测器“奔向”月球的过程中,用$ h $表示探测器与地球表面的距离,$ F $表示它所受的地球引力,能够描$ F $随$ h $变化关系的图像是 \xzanswer{D} 
\begin{figure}[h!]
\centering
\includesvg[width=0.83\linewidth]{picture/svg/589}
\end{figure}

\item 
\exwhere{$ 2019 $年物理全国\lmd{1}卷}
在星球$ M $上将一轻弹簧竖直固定在水平桌面上,把物体$ P $轻放在弹簧上端,$ P $由静止向下运动,物体的加速度$ a $与弹簧的压缩量$ x $间的关系如图中实线所示。在另一星球$ N $上用完全相同的弹簧,改用物体$ Q $完成同样的过程,其$ a - x $关系如图中虚线所示,假设两星球均为质量均匀分布的球体。已知星球$ M $的半径是星球$ N $的$ 3 $倍,则 \xzanswer{AC} 
\begin{figure}[h!]
\centering
\includesvg[width=0.23\linewidth]{picture/svg/590}
\end{figure}


\fourchoices
{$ M $与$ N $的密度相等}
{$ Q $的质量是$ P $的$ 3 $倍}
{$ Q $下落过程中的最大动能是$ P $的$ 4 $倍}
{$ Q $下落过程中弹簧的最大压缩量是$ P $的$ 4 $倍}


\newpage
\item 
\exwhere{$ 2015 $年理综安徽卷}
由三颗星体构成的系统,忽略其它星体对它们的作用,存在着一种运动形式:三颗星体在相互之间的万有引力作用下,分别位于等边三角形的三个顶点上,绕某一共同的圆心$ O $在三角形所在的平面内做相同角速度的圆周运动(图示为$ A $、$ B $、$ C $三颗星体质量不相同时的一般情况)。若$ A $星体质量为$ 2 \ m $,$ B $、$ C $两星体的质量均为$ m $,三角形的边长为$ a $,求:
\begin{enumerate}
\renewcommand{\labelenumi}{\arabic{enumi}.}
% A(\Alph) a(\alph) I(\Roman) i(\roman) 1(\arabic)
%设定全局标号series=example	%引用全局变量resume=example
%[topsep=-0.3em,parsep=-0.3em,itemsep=-0.3em,partopsep=-0.3em]
%可使用leftmargin调整列表环境左边的空白长度 [leftmargin=0em]
\item
$ A $星体所受合力大小$ F_A $;
\item 
$ B $星体所受合力大小$ F_B $;
\item 
$ C $星体的轨道半径$ R_C $;
\item 
三星体做圆周运动的周期$ T $。
\end{enumerate}
\begin{figure}[h!]
\flushright
\includesvg[width=0.25\linewidth]{picture/svg/591}
\end{figure}

\banswer{
\begin{enumerate}
\renewcommand{\labelenumi}{\arabic{enumi}.}
% A(\Alph) a(\alph) I(\Roman) i(\roman) 1(\arabic)
%设定全局标号series=example	%引用全局变量resume=example
%[topsep=-0.3em,parsep=-0.3em,itemsep=-0.3em,partopsep=-0.3em]
%可使用leftmargin调整列表环境左边的空白长度 [leftmargin=0em]
\item
$F _ { A } = \frac { 2 \sqrt { 3 } G m ^ { 2 } } { a ^ { 2 } }$
\item 
$F _ { B } = \frac { \sqrt { 7 } G m ^ { 2 } } { a ^ { 2 } }$
\item 
$R _ { C } = \frac { \sqrt { 7 } } { 4 } a$
\item 
$T = \pi \sqrt { \frac { a ^ { 3 } } { G m } }$



\end{enumerate}


}










\end{enumerate}

