\bta{选修模块 $ 3-4 $(下)}


\begin{enumerate}
	%\renewcommand{\labelenumi}{\arabic{enumi}.}
	% A(\Alph) a(\alph) I(\Roman) i(\roman) 1(\arabic)
	%设定全局标号series=example	%引用全局变量resume=example
	%[topsep=-0.3em,parsep=-0.3em,itemsep=-0.3em,partopsep=-0.3em]
	%可使用leftmargin调整列表环境左边的空白长度 [leftmargin=0em]
	\item
\exwhere{$ 2014 $ 年理综新课标\lmd{1}卷}
\begin{enumerate}
	%\renewcommand{\labelenumi}{\arabic{enumi}.}
	% A(\Alph) a(\alph) I(\Roman) i(\roman) 1(\arabic)
	%设定全局标号series=example	%引用全局变量resume=example
	%[topsep=-0.3em,parsep=-0.3em,itemsep=-0.3em,partopsep=-0.3em]
	%可使用leftmargin调整列表环境左边的空白长度 [leftmargin=0em]
	\item
图 \subref{2014全国134a} 为一列波在 $ t  =2 \ s $ 时的波形图,图 \subref{2014全国134b} 为媒质是平衡位置在 $ x=1.5 \ m $ 处的质
点的振动图象,$ P $ 是平衡位置为 $ x=2 \ m $ 的质点,下列说法正确的是 \underlinegap 。
(填正确答案标号,选对一个得 $ 3 $ 分,选对 $ 2 $ 个得 $ 4 $ 分,选对 $ 3 $ 个得 $ 6 $ 分。每选错 $ 1 $ 个扣 $ 3 $ 分,最
低得 $ 0 $ 分)
\begin{figure}[h!]
	\centering
\begin{subfigure}{0.4\linewidth}
	\centering
	\includesvg[width=0.7\linewidth]{picture/svg/GZ-3-tiyou-1562} 
	\caption{}\label{2014全国134a}
\end{subfigure}
\begin{subfigure}{0.4\linewidth}
	\centering
	\includesvg[width=0.7\linewidth]{picture/svg/GZ-3-tiyou-1563} 
	\caption{}\label{2014全国134b}
\end{subfigure}
\end{figure}

\fivechoices
{波速为 $ 0.5 \ m /s $}
{波的传播方向向右}
{$ 0 \sim 2 \ s $ 时间内,$ P $ 运动的路程为 $ 8 \ cm $}
{$ 0 \sim 2 \ s $ 时间内,$ P $ 向 $ y $ 轴正方向运动}
{当 $ t =7 \ s $ 时,$ P $ 恰好回到平衡位置}

 \tk{ACE} 


\item 
一个半圆形玻璃砖,某横截面半径为 $ R $ 的半圆,$ AB $ 为半圆的直径。
$ O $ 为圆心,如图所示,玻璃的折射率为 $ n=\sqrt{2} $。
\begin{enumerate}
	%\renewcommand{\labelenumi}{\arabic{enumi}.}
	% A(\Alph) a(\alph) I(\Roman) i(\roman) 1(\arabic)
	%设定全局标号series=example	%引用全局变量resume=example
	%[topsep=-0.3em,parsep=-0.3em,itemsep=-0.3em,partopsep=-0.3em]
	%可使用leftmargin调整列表环境左边的空白长度 [leftmargin=0em]
	\item
一束平行光垂直射向玻璃砖的下表面,若光线到达上表面后,都能从该表面射出,则入射光
束在 $ AB $ 上的最大宽度为多少?
\item 
一细束光线在 $ O $ 点左侧与 $ O $ 相距$ \frac{\sqrt{3}}{2} R $ 处垂直于 $ AB $ 从下方入射,求此光线从玻璃砖射出点
的位置。

\end{enumerate}
\begin{figure}[h!]
	\flushright
	\includesvg[width=0.25\linewidth]{picture/svg/GZ-3-tiyou-1564}
\end{figure}


\banswer{
	\begin{enumerate}
		%\renewcommand{\labelenumi}{\arabic{enumi}.}
		% A(\Alph) a(\alph) I(\Roman) i(\roman) 1(\arabic)
		%设定全局标号series=example	%引用全局变量resume=example
		%[topsep=-0.3em,parsep=-0.3em,itemsep=-0.3em,partopsep=-0.3em]
		%可使用leftmargin调整列表环境左边的空白长度 [leftmargin=0em]
		\item
		$d=\sqrt{2} R$
		\item 
		右侧与 $O$ 相距 $\frac{\sqrt{3}}{2} R$
	\end{enumerate}
}



\end{enumerate}

\item 
\exwhere{$ 2014 $ 年理综新课标 \lmd{2} 卷}
\begin{enumerate}
	%\renewcommand{\labelenumi}{\arabic{enumi}.}
	% A(\Alph) a(\alph) I(\Roman) i(\roman) 1(\arabic)
	%设定全局标号series=example	%引用全局变量resume=example
	%[topsep=-0.3em,parsep=-0.3em,itemsep=-0.3em,partopsep=-0.3em]
	%可使用leftmargin调整列表环境左边的空白长度 [leftmargin=0em]
	\item
图 \subref{2014全国234a} 为一列简谐横波在 $ t=0.10 \ s $ 时刻的波形图,$ P $ 是平衡位置在 $ x=1.0 \ m $ 处的质点,$ Q $ 是平
衡位置在 $ x=4.0 \ m $ 处的质点;图 \subref{2014全国234b} 为质点 $ Q $ 的振动图像,下列说法正确的是
 \underlinegap 
。(填正确答案标
号,选对 $ 1 $ 个给 $ 2 $ 分,选对 $ 2 $ 个得 $ 4 $ 分,选对 $ 3 $ 个得 $ 5 $ 分,每选错 $ 1 $ 个扣 $ 3 $ 分,最低得分 $ 0 $ 分)
\begin{figure}[h!]
	\centering
\begin{subfigure}{0.4\linewidth}
	\centering
	\includesvg[width=0.7\linewidth]{picture/svg/GZ-3-tiyou-1565} 
	\caption{}\label{2014全国234a}
\end{subfigure}
\begin{subfigure}{0.4\linewidth}
	\centering
	\includesvg[width=0.7\linewidth]{picture/svg/GZ-3-tiyou-1566} 
	\caption{}\label{2014全国234b}
\end{subfigure}

\end{figure}


\fivechoices
{在 $ t=0.10 \ s $ 时,质点 $ Q $ 向 $ y $ 轴正方向运动}
{在 $ t=0.25 \ s $ 时,质点 $ P $ 的加速度方向与 $ y $ 轴正方向相同}
{从 $ t=0.10 \ s $ 到 $ t=0.25 \ s $,该波沿 $ x $ 轴负方向传播了 $ 6 \ m $}
{从 $ t=0.10 \ s $ 到 $ t=0.25 \ s $,质点 $ P $ 通过的路程为 $ 30 \ cm $}
{质点 $ Q $ 简谐运动的表达式为 $ y=0.10 \sin 10 \pi t $ (国际单位)}


 \tk{BCE} 

\item 
一厚度为 $ h $ 的大平板玻璃水平放置,其下表面贴有一半径为 $ r $ 的圆形发光面。在玻
璃板上表面放置一半径为 $ R $ 的圆纸片,圆纸片与圆形发光面的中心在同一竖直线上。已知圆纸片
恰好能完全遮挡住从圆形发光面发出的光线(不考虑反射),求平板玻璃的折射率。

\banswer{
	$\left.n=\sqrt{1+\left(\frac{h}{R-r}\right.}\right)^{2}$
}


\end{enumerate}


\item 
\exwhere{$ 2014 $ 年理综重庆卷}
\begin{enumerate}
	%\renewcommand{\labelenumi}{\arabic{enumi}.}
	% A(\Alph) a(\alph) I(\Roman) i(\roman) 1(\arabic)
	%设定全局标号series=example	%引用全局变量resume=example
	%[topsep=-0.3em,parsep=-0.3em,itemsep=-0.3em,partopsep=-0.3em]
	%可使用leftmargin调整列表环境左边的空白长度 [leftmargin=0em]
	\item
打磨某剖面如图所示的宝石时,必须将 $ OP $、$ OQ $ 边与轴线的夹角$ \theta $切割在
$ \theta _{1} < \theta < \theta _{2} $ 的范围内,才能使从 $ MN $ 边垂直入射的光线,在 $ OP $ 边和 $ OQ $ 边都发生全反射(仅考虑如
图所示的光线第一次射到 $ OP $ 边并反射到 $ OQ $ 边后射向 $ MN $ 边的情
况),则下列判断正确的是 \xzanswer{D} 
\begin{figure}[h!]
	\centering
	\includesvg[width=0.23\linewidth]{picture/svg/GZ-3-tiyou-1567}
\end{figure}

\fourchoices
{若$ \theta > \theta _{2} $,光线一定在 $ OP $ 边发生全反射}
{若$ \theta > \theta _{2} $,光线会从 $ OQ $ 边射出}
{若$ \theta < \theta _{1} $,光线会从 $ OQ $ 边射出}
{若$ \theta < \theta _{1} $,光线会在 $ OP $ 边发生全反射}



\item 
一竖直悬挂的弹簧振子,下端装有一记录笔,在竖直面内放置有一记录纸。当振子上
下振动时,以速率水平向左拉动记录纸,记录笔在纸上
留下如图所示的图象。$ y_{1} $、$ y_{2} $、$ x_{0} $、$ 2 x_{0} $ 为纸上印
迹的位置坐标。由此求振动的周期和振幅。
\begin{figure}[h!]
	\flushright
	\includesvg[width=0.25\linewidth]{picture/svg/GZ-3-tiyou-1568}
\end{figure}


\banswer{
	$T=\frac{2 x_{0}}{v} \quad A=\frac{y_{1}-y_{2}}{2}$
}


\end{enumerate}


\item 
\exwhere{$ 2014 $ 年物理江苏卷}
\begin{enumerate}
	%\renewcommand{\labelenumi}{\arabic{enumi}.}
	% A(\Alph) a(\alph) I(\Roman) i(\roman) 1(\arabic)
	%设定全局标号series=example	%引用全局变量resume=example
	%[topsep=-0.3em,parsep=-0.3em,itemsep=-0.3em,partopsep=-0.3em]
	%可使用leftmargin调整列表环境左边的空白长度 [leftmargin=0em]
	\item
某同学用单色光进行双缝干涉实验,在屏上观察到图 \subref{2014江苏12Ba} 所示的条纹,仅改变一
个实验条件后,观察到的条纹如图 \subref{2014江苏12Bb} 所示。他改变的
实验条件可能是 \underlinegap 。
\begin{figure}[h!]
	\centering
	\begin{subfigure}{0.4\linewidth}
		\centering
		\includesvg[width=0.7\linewidth]{picture/svg/GZ-3-tiyou-1569} 
		\caption{}\label{2014江苏12Ba}
	\end{subfigure}
	\begin{subfigure}{0.4\linewidth}
		\centering
		\includesvg[width=0.7\linewidth]{picture/svg/GZ-3-tiyou-1570} 
		\caption{}\label{2014江苏12Bb}
	\end{subfigure}
	
\end{figure}

\fourchoices
{减小光源到单缝的距离}
{减小双缝之间的距离}
{减小双缝到光屏之间的距离}
{换用频率更高的单色光源}

 \tk{B} 


\item 
在“ 探究单摆的周期与摆长的关系” 实验中,某同学准备好相关实验器材后,把单摆从平衡
位置拉开一个很小的角度后释放,同时按下秒表开始计时,当单摆再次回到释放位置时停止计
时,将记录的这段时间作为单摆的周期. 以上操作中有不妥之处,请对其中两处加以改正.


 \hfullline 


 \tk{
①应从单摆运动到最低点开始计时时,此位置容易判断,计时误差较小 \\
②为了减小偶然
误差,可以多次测量多次全振动的时间,然后取平均值求周期。 
} 
	
\item 
$ Morpho $ 蝴蝶的翅膀在阳光的照射下呈现出闪亮耀眼的蓝色光芒, 这是因为光照射到翅膀的
鳞片上发生了干涉。电子显微镜下鳞片结构的示意
图见题图. 一束光以入射角 $ i $ 从 $ a $ 点入射,
经过折射和反射后从 $ b $ 点出射。 设鳞片的折射率
为 $ n $, 厚度为 $ d $, 两片之间空气层厚度为 $ h $. 取光
在空气中的速度为 $ c $,求光从 $ a $ 到 $ b $ 所需的时间 $ t $.
\begin{figure}[h!]
	\flushright
	\includesvg[width=0.25\linewidth]{picture/svg/GZ-3-tiyou-1571}
\end{figure}


\banswer{
	在鳞片中传播的时间$t_{1}=\frac{2 n^{2} d}{c \sqrt{n^{2}-\sin ^{2} i}}$\\
	在空气中传播的时间 $t_{2}=\frac{2 h}{c \cos i}$\\
	$t=t_{1}+t_{2}=\frac{2 n^{2} d}{c \sqrt{n^{2}-\sin ^{2} i}}+\frac{2 h}{c \cos i}$
}



\end{enumerate}


\item 
\exwhere{$ 2014 $ 年理综山东卷}
\begin{enumerate}
	%\renewcommand{\labelenumi}{\arabic{enumi}.}
	% A(\Alph) a(\alph) I(\Roman) i(\roman) 1(\arabic)
	%设定全局标号series=example	%引用全局变量resume=example
	%[topsep=-0.3em,parsep=-0.3em,itemsep=-0.3em,partopsep=-0.3em]
	%可使用leftmargin调整列表环境左边的空白长度 [leftmargin=0em]
	\item
一列简谐横波沿直线传播。以波源 $ O $ 由平衡位置开始振
动为计时零点,质点 $ A $ 的振动图像如图所示,已知 $ O $、$ A $ 的平
衡位置相距 $ 0.9 \ m $.以下判断正确的是 \underlinegap 。(双选,填
正确答案标号)
\begin{figure}[h!]
	\centering
	\includesvg[width=0.23\linewidth]{picture/svg/GZ-3-tiyou-1572}
\end{figure}


\fourchoices
{波长为 $ 1.2 \ m $}
{波源起振方向沿 $ y $ 轴正方向}
{波速大小为 $ 0.4 \ m /s $}
{质点 $ A $ 的动能在 $ t=4 \ s $ 时最大}


 \tk{AB} 

\item 
如图,三角形 $ ABC $ 为某透明介质的横截面,$ O $ 为 $ BC $ 边的中点,位于截面所在平面内的一束
光 线 自 $ O $ 以 角 $ i $ 入 射 , 第 一 次 到 达 $ AB $ 边 恰 好 发 生 全 反 射 。 已 知
$ q=15  \degree  $ ,$ BC $ 边长为 $ 2L $,该介质的折射率为 $ \sqrt{2} $。求:
\begin{enumerate}
	%\renewcommand{\labelenumi}{\arabic{enumi}.}
	% A(\Alph) a(\alph) I(\Roman) i(\roman) 1(\arabic)
	%设定全局标号series=example	%引用全局变量resume=example
	%[topsep=-0.3em,parsep=-0.3em,itemsep=-0.3em,partopsep=-0.3em]
	%可使用leftmargin调整列表环境左边的空白长度 [leftmargin=0em]
	\item
入射角 $ i $;
\item 
从入射到发生第一次全反射所用的时间(设光在真空中的速度为
$ c $,可能用到 $\sin 75^{\circ}=\frac{\sqrt{6}+\sqrt{2}}{4}$ 或 $\tan 15^{\circ}=2-\sqrt{3}$)
\end{enumerate}
\begin{figure}[h!]
	\flushright
	\includesvg[width=0.25\linewidth]{picture/svg/GZ-3-tiyou-1573}
\end{figure}


\banswer{
	\begin{enumerate}
		%\renewcommand{\labelenumi}{\arabic{enumi}.}
		% A(\Alph) a(\alph) I(\Roman) i(\roman) 1(\arabic)
		%设定全局标号series=example	%引用全局变量resume=example
		%[topsep=-0.3em,parsep=-0.3em,itemsep=-0.3em,partopsep=-0.3em]
		%可使用leftmargin调整列表环境左边的空白长度 [leftmargin=0em]
		\item
		$ i=45 \degree  $
		\item 
		$t=\frac{\sqrt{6}+\sqrt{2}}{2 c} L$
	\end{enumerate}
}


	
\end{enumerate}


\item 
\exwhere{$ 2014 $ 年物理海南卷}
\begin{enumerate}
	%\renewcommand{\labelenumi}{\arabic{enumi}.}
	% A(\Alph) a(\alph) I(\Roman) i(\roman) 1(\arabic)
	%设定全局标号series=example	%引用全局变量resume=example
	%[topsep=-0.3em,parsep=-0.3em,itemsep=-0.3em,partopsep=-0.3em]
	%可使用leftmargin调整列表环境左边的空白长度 [leftmargin=0em]
	\item
一列简谐横波沿 $ x $ 轴传播,$ a $、$ b $ 为 $ x $ 轴上的两质
点,平衡位置分别为 $ x=0 $,$ x= x_{b} $($ x_{b} >0 $)。$ a $ 点的振动规律如图
所示,已知波速为 $ v=10 \ m /s $,在 $ t=0.1 \ s $ 时,$ b $ 的位移为 $ 0.05 \ m $,
则下列判断可能正确的是 \xzanswer{BC} 

\fourchoices
{波沿 $ x $ 轴正向传播,$ x_{b} =0.5 \ m $}
{波沿 $ x $ 轴正向传播,$ x_{b} =1.5 \ m $}
{波沿 $ x $ 轴负向传播,$ x_{b} =2.5 \ m $}
{波沿 $ x $ 轴负向传播,$ x_{b} =3.5 \ m $}


\item 
如图,矩形 $ ABCD $ 为一水平放置的玻璃砖的截面,在截面所在平面有一细束激光照射
玻璃砖,入射点距底面的高度为 $ h $,反射光线和折射光线与
底面所在平面的交点到 $ AB $ 的距离分别 $ l_{1} $ 和 $ l_{2} $,在截面所在平
面内,改变激光束在 $ AB $ 面上入射点的高度与入射角的大小,
当折射光线与底面的交点到 $ AB $ 的距离为 $ l_{3} $ 时,光线恰好不
能从底面射出,求此时入射点距底面的高度 $ H $。
\begin{figure}[h!]
	\flushright
	\includesvg[width=0.25\linewidth]{picture/svg/GZ-3-tiyou-1574}
\end{figure}

\banswer{
	$H=\sqrt{\frac{l_{2}^{2}-l_{1}^{2}}{l_{1}^{2}+h^{2}}} l_{3}$
}



\end{enumerate}


\item 
\exwhere{$ 2013 $ 年新课标  \lmd{1} 卷}
\begin{enumerate}
	%\renewcommand{\labelenumi}{\arabic{enumi}.}
	% A(\Alph) a(\alph) I(\Roman) i(\roman) 1(\arabic)
	%设定全局标号series=example	%引用全局变量resume=example
	%[topsep=-0.3em,parsep=-0.3em,itemsep=-0.3em,partopsep=-0.3em]
	%可使用leftmargin调整列表环境左边的空白长度 [leftmargin=0em]
	\item
如图,$ a.b,c.d $ 是均匀媒质中 $ x $ 轴上的四个质点,相邻两点的间距依次为 $ 2 \ m $、$ 4 \ m $ 和 $ 6 \ m $。
一列简谐横波以 $ 2 \ m /s $ 的波速沿 $ x $ 轴正向传播,在 $ t=0 $ 时刻到达质点 $ a $ 处,质点 $ a $ 由平衡位置开始
竖直向下运动,$ t=3 \ s $ 时 $ a $ 第一次到达最高点。下列说法正确的是
 \underlinegap 
(填正确答案标号。选对 $ 1 $ 个得
$ 3 $分,选对 $ 2 $ 个得 $ 4 $ 分,选对 $ 3 $ 个得 $ 6 $ 分。每选错 $ 1 $ 个扣 $ 3 $ 分,最低得分为 $ 0 $ 分)
\begin{figure}[h!]
	\centering
	\includesvg[width=0.23\linewidth]{picture/svg/GZ-3-tiyou-1575}
\end{figure}

\fivechoices
{在 $ t=6 \ s $ 时刻波恰好传到质点 $ d $ 处}
{在 $ t=5 \ s $ 时刻质点 $ c $ 恰好到达最高点}
{质点 $ b $ 开始振动后,其振动周期为 $ 4 \ s $}
{在 $ 4 \ s <t<6 \ s $ 的时间间隔内质点 $ c $ 向上运动}
{当质点 $ d $ 向下运动时,质点 $ b $ 一定向上运动}

 \tk{ACD} 


\item 
图示为一光导纤维(可简化为一长玻璃丝)的示意图,玻璃丝长为 $ L $,折射率为 $ n $,$ AB $ 代表
端面。已知光在真空中的传播速度为 $ c $.
\begin{enumerate}
	%\renewcommand{\labelenumi}{\arabic{enumi}.}
	% A(\Alph) a(\alph) I(\Roman) i(\roman) 1(\arabic)
	%设定全局标号series=example	%引用全局变量resume=example
	%[topsep=-0.3em,parsep=-0.3em,itemsep=-0.3em,partopsep=-0.3em]
	%可使用leftmargin调整列表环境左边的空白长度 [leftmargin=0em]
	\item
为使光线能从玻璃丝的 $ AB $ 端面传播到另一端面,求光线在端面 $ AB $ 上的入射角应满足的条
件;
\item 
求光线从玻璃丝的 $ AB $ 端面传播到另一端面所需的最长时间。
\end{enumerate}
\begin{figure}[h!]
	\flushright
	\includesvg[width=0.25\linewidth]{picture/svg/GZ-3-tiyou-1576}
\end{figure}

\banswer{
	\begin{enumerate}
		%\renewcommand{\labelenumi}{\arabic{enumi}.}
		% A(\Alph) a(\alph) I(\Roman) i(\roman) 1(\arabic)
		%设定全局标号series=example	%引用全局变量resume=example
		%[topsep=-0.3em,parsep=-0.3em,itemsep=-0.3em,partopsep=-0.3em]
		%可使用leftmargin调整列表环境左边的空白长度 [leftmargin=0em]
		\item
		$\sin i \leq \sqrt{n^{2}-1}$
		\item 
		$t_{\max }=\frac{L n^{2}}{c}$
	\end{enumerate}
}



\end{enumerate}


\item 
\exwhere{$ 2013 $ 年新课标  \lmd{2}  卷}
\begin{enumerate}
	%\renewcommand{\labelenumi}{\arabic{enumi}.}
	% A(\Alph) a(\alph) I(\Roman) i(\roman) 1(\arabic)
	%设定全局标号series=example	%引用全局变量resume=example
	%[topsep=-0.3em,parsep=-0.3em,itemsep=-0.3em,partopsep=-0.3em]
	%可使用leftmargin调整列表环境左边的空白长度 [leftmargin=0em]
	\item
如图,一轻弹簧一端固定,另一端连接一物块构成弹簧振子,该物块是由 $ a $、$ b $ 两个
小物块粘在一起组成的。物块在光滑水平面上左右振动,振幅为 $ A_{0} $,周期为 $ T_{0} $。当物块向右通过
平衡位置时,$ a $、$ b $ 之间的粘胶脱开;以后小物块 $ a $ 振动的振幅和周期分别为 $ A $ 和 $ T $,则 $ A $ \underlinegap $ A_{0} $
(填“$ > $”“$ < $”“$ = $”),$ T $ \underlinegap $ T_{0} $(填“$ > $”“$ < $”“$ = $”)。
\begin{figure}[h!]
	\centering
	\includesvg[width=0.23\linewidth]{picture/svg/GZ-3-tiyou-1577}
\end{figure}

 \tk{$ < $ \quad $ < $} 


\item 
如图,三棱镜的横截面为直角三角形 $ ABC $,$ \angle A=30 \degree $,$ \angle B=60 \degree $。一束平行于 $ AC $ 边的光
线自 $ AB $ 边的 $ P $ 点射入三棱镜,在 $ AC $ 边发生反射后从 $ BC $ 边的 $ M $
点射出,若光线在 $ P $ 点的入射角和在 $ M $ 点的折射角相等。
\begin{enumerate}
	%\renewcommand{\labelenumi}{\arabic{enumi}.}
	% A(\Alph) a(\alph) I(\Roman) i(\roman) 1(\arabic)
	%设定全局标号series=example	%引用全局变量resume=example
	%[topsep=-0.3em,parsep=-0.3em,itemsep=-0.3em,partopsep=-0.3em]
	%可使用leftmargin调整列表环境左边的空白长度 [leftmargin=0em]
	\item
求三棱镜的折射率
\item 
在三棱镜的 $ AC $ 边是否有光线透出,写出分析过程。
(不考
虑多次反射)

	
\end{enumerate}
\begin{figure}[h!]
	\flushright
	\includesvg[width=0.25\linewidth]{picture/svg/GZ-3-tiyou-1578}
\end{figure}


\banswer{
	\begin{enumerate}
		%\renewcommand{\labelenumi}{\arabic{enumi}.}
		% A(\Alph) a(\alph) I(\Roman) i(\roman) 1(\arabic)
		%设定全局标号series=example	%引用全局变量resume=example
		%[topsep=-0.3em,parsep=-0.3em,itemsep=-0.3em,partopsep=-0.3em]
		%可使用leftmargin调整列表环境左边的空白长度 [leftmargin=0em]
		\item
		$ n=\sqrt{3} $
		\item 
		$ AC $ 边没有光线透出,分析略
	\end{enumerate}
}


\end{enumerate}


\item 
\exwhere{$ 2013 $ 年重庆卷}
\begin{enumerate}
	%\renewcommand{\labelenumi}{\arabic{enumi}.}
	% A(\Alph) a(\alph) I(\Roman) i(\roman) 1(\arabic)
	%设定全局标号series=example	%引用全局变量resume=example
	%[topsep=-0.3em,parsep=-0.3em,itemsep=-0.3em,partopsep=-0.3em]
	%可使用leftmargin调整列表环境左边的空白长度 [leftmargin=0em]
	\item
一列简谐波沿直线传播,某时刻该列波上正好经过平衡位置的两质点相距 $ 6 \ m $,且这
两质点之间的波峰只有一个,则该简谐波可能的波长为 \xzanswer{C} 

\fourchoices
{$ 4 \ m $、$ 6 \ m $ 和 $ 8 \ m $}
{$ 6 \ m $、$ 8 \ m $ 和 $ 12 \ m $}
{$ 4 \ m $、$ 6 \ m $ 和 $ 12 \ m $}
{$ 4 \ m $、$ 8 \ m $ 和 $ 12 \ m $}



\item 
利用半圆柱形玻璃,可减小激光束的发散
程度。在如图所示的光路中,$ A $ 为激光的出射点,$ O $ 为
半圆柱形玻璃横截面的圆心,$ AO $ 过半圆顶点。若某条从
$ A $点发出的与 $ AO $ 成$ \alpha $角的光线,以入射角 $ i $ 入射到半圆
弧上,出射光线平行于 $ AO $,求此玻璃的折射率。
\begin{figure}[h!]
	\flushright
	\includesvg[width=0.25\linewidth]{picture/svg/GZ-3-tiyou-1579}
\end{figure}
\banswer{
	$n=\frac{\sin i}{\sin (i-\alpha)}$
}

\end{enumerate}


\item 
\exwhere{$ 2013 $年江苏卷}
\begin{enumerate}
	%\renewcommand{\labelenumi}{\arabic{enumi}.}
	% A(\Alph) a(\alph) I(\Roman) i(\roman) 1(\arabic)
	%设定全局标号series=example	%引用全局变量resume=example
	%[topsep=-0.3em,parsep=-0.3em,itemsep=-0.3em,partopsep=-0.3em]
	%可使用leftmargin调整列表环境左边的空白长度 [leftmargin=0em]
	\item
如图所示的装置,弹簧振子的固有频率是$ 4 \ Hz $. 现匀速转动把
手,给弹簧振子以周期性的驱动力,测得弹簧振子振动达到稳定时的频
率为$ 1 \ Hz $,则把手转动的频率为 \underlinegap 。
\begin{figure}[h!]
	\centering
	\includesvg[width=0.23\linewidth]{picture/svg/GZ-3-tiyou-1580}
\end{figure}

\fourchoices
{$1 \ Hz $}
{$3 \ Hz $}
{$4 \ Hz $}
{$5 \ Hz $}

 \tk{A} 

\item 
如图所示,两艘飞船$ A $、$ B $ 沿同一直线
同向飞行,相对地面的速度均为$ v(v $ 接近光速$ c) $. 地面
上测得它们相距为$ L $,则$ A $ 测得两飞船间的距离
 \underlinegap  (选填“大于”、“等于”或“小于”)$ L $. 当$ B $ 向$ A $ 发出一光信号,$ A $ 测得该信号的速度为 \underlinegap 。
\begin{figure}[h!]
	\centering
	\includesvg[width=0.23\linewidth]{picture/svg/GZ-3-tiyou-1581}
\end{figure}


 \tk{大于	 \quad $ c $(或光速)} 

\item 
下图为单反照相机取景器的示意图,$ ABCDE $为五棱镜的一个
截面,$ AB \perp BC $. 光线垂直$ AB $ 射入,分别在$ CD $ 和$ E_{A} $ 上发生反射,且两次
反射的入射角相等,最后光线垂直$ BC $ 射出.若两次反射都为全反射,则
该五棱镜折射率的最小值是多少?(计算结果可用三角函数表示)
\begin{figure}[h!]
	\flushright
	\includesvg[width=0.25\linewidth]{picture/svg/GZ-3-tiyou-1582}
\end{figure}

\banswer{
	$n=\frac{1}{\sin 22.5^{\circ}}$
}


	
\end{enumerate}



\item 
\exwhere{$ 2013 $ 年山东卷}
\begin{enumerate}
	%\renewcommand{\labelenumi}{\arabic{enumi}.}
	% A(\Alph) a(\alph) I(\Roman) i(\roman) 1(\arabic)
	%设定全局标号series=example	%引用全局变量resume=example
	%[topsep=-0.3em,parsep=-0.3em,itemsep=-0.3em,partopsep=-0.3em]
	%可使用leftmargin调整列表环境左边的空白长度 [leftmargin=0em]
	\item
如图所示,在某一均匀介质中,$ A $、$ B $ 是振动情况完全相同的两个波源,其简谐运动表达式
均为 $ x=0.1 \sin (20 \pi t) \ m $,介质中 $ P $ 点与 $ A $、$ B $ 两波源间的距离分别为 $ 4 \ m $ 和
$ 5 \ m $,两波源形成的简谐横波分别沿 $ AP $、$ BP $ 方向传播,波速都是 $ 10 \ m /s $。
\begin{enumerate}
	%\renewcommand{\labelenumi}{\arabic{enumi}.}
	% A(\Alph) a(\alph) I(\Roman) i(\roman) 1(\arabic)
	%设定全局标号series=example	%引用全局变量resume=example
	%[topsep=-0.3em,parsep=-0.3em,itemsep=-0.3em,partopsep=-0.3em]
	%可使用leftmargin调整列表环境左边的空白长度 [leftmargin=0em]
	\item
求简谐横波的波长。
\item 
$ P $ 点的振动 \underlinegap (填“加强”或“减弱”)。
\end{enumerate}
\begin{figure}[h!]
	\flushright
	\includesvg[width=0.25\linewidth]{picture/svg/GZ-3-tiyou-1583}
\end{figure}

\banswer{
	\begin{enumerate}
		%\renewcommand{\labelenumi}{\arabic{enumi}.}
		% A(\Alph) a(\alph) I(\Roman) i(\roman) 1(\arabic)
		%设定全局标号series=example	%引用全局变量resume=example
		%[topsep=-0.3em,parsep=-0.3em,itemsep=-0.3em,partopsep=-0.3em]
		%可使用leftmargin调整列表环境左边的空白长度 [leftmargin=0em]
		\item
		$ 1 \ m $
		\item 
		 加强
	\end{enumerate}
}

\item 
如图所示, $A B C D$ 是一直角梯形棱镜的横截面,位于截面所在
平面内的一束光线由 $O$ 点垂直 $A D$ 边射入。已知棱镜的折射率
$n=\sqrt{2}, A B=B C=8 \ cm, O A=2 \ cm,  \angle O A B=60^{\circ}$。
\begin{enumerate}
	%\renewcommand{\labelenumi}{\arabic{enumi}.}
	% A(\Alph) a(\alph) I(\Roman) i(\roman) 1(\arabic)
	%设定全局标号series=example	%引用全局变量resume=example
	%[topsep=-0.3em,parsep=-0.3em,itemsep=-0.3em,partopsep=-0.3em]
	%可使用leftmargin调整列表环境左边的空白长度 [leftmargin=0em]
	\item
	求光线第一次射出棱镜时,出射光线的方向。
	\item 
	第一次的出射点距$ C $ \underlinegap $ cm $。
	
	
	
\end{enumerate}
\begin{figure}[h!]
	\flushright
	\includesvg[width=0.25\linewidth]{picture/svg/GZ-3-tiyou-1584}
\end{figure}

\banswer{
	\begin{enumerate}
		%\renewcommand{\labelenumi}{\arabic{enumi}.}
		% A(\Alph) a(\alph) I(\Roman) i(\roman) 1(\arabic)
		%设定全局标号series=example	%引用全局变量resume=example
		%[topsep=-0.3em,parsep=-0.3em,itemsep=-0.3em,partopsep=-0.3em]
		%可使用leftmargin调整列表环境左边的空白长度 [leftmargin=0em]
		\item
		$ 45 \degree  $
		\item 
		$\frac{4}{3} \sqrt{3} $
	\end{enumerate}
}







\end{enumerate}


\item 
\exwhere{$ 2013 $ 年海南卷}
\begin{enumerate}
	%\renewcommand{\labelenumi}{\arabic{enumi}.}
	% A(\Alph) a(\alph) I(\Roman) i(\roman) 1(\arabic)
	%设定全局标号series=example	%引用全局变量resume=example
	%[topsep=-0.3em,parsep=-0.3em,itemsep=-0.3em,partopsep=-0.3em]
	%可使用leftmargin调整列表环境左边的空白长度 [leftmargin=0em]
	\item
下列选项与多普勒效应有关的是
 \underlinegap 
(填正确答案标号。选对 $ 1 $ 个得 $ 2 $ 分,选对 $ 2 $ 个
得 $ 3 $ 分.选对 $ 3 $ 个得 $ 4 $ 分;每选错 $ I $ 个扣 $ 2 $ 分,最低得分为 $ 0 $ 分)
\fivechoices
{科学家用激光测量月球与地球间的距离}
{医生利用超声波探测病人血管中血液的流速}
{技术人员用超声波探测金属、陶瓷、混凝土中是否有气泡}
{交通警察向车辆发射超声波并通过测量反射波的频率确定车辆行进的速度}
{科学家通过比较星球与地球上同种元素发出光的频率来计算星球远离地球的速度}


 \tk{BDE} 


\item 
如图,三棱镜的横截面为直角三角形 $ ABC $,$ \angle A=30 \degree  $,
$ AC $ 平行于光屏 $ MN $,与光屏的距离为 $ L $,棱镜对红光的折射率为
$ n_{1} $,对紫光的折射率为 $ n_{2} $。一束很细的白光由棱镜的侧面 $ AB $ 垂直射
入,直接到达 $ AC $ 面并射出。画出光路示意图,并标出红光和紫光射
在光屏上的位置,求红光和紫光在光屏上的位置之间的距离。
\begin{figure}[h!]
	\flushright
	\includesvg[width=0.25\linewidth]{picture/svg/GZ-3-tiyou-1585}
\end{figure}


\banswer{
	光路图见图:
	\begin{center}
 \includesvg[width=0.23\linewidth]{picture/svg/GZ-3-tiyou-1586} 
	\end{center}
	$d_{2}-d_{1}=L\left(\frac{n_{2}}{\sqrt{4-n_{2}^{2}}}-\frac{n_{1}}{\sqrt{4-n_{1}^{2}}}\right)$
}


\end{enumerate}


\item 
\exwhere{$ 2012 $ 年理综新课标卷}
\begin{enumerate}
	%\renewcommand{\labelenumi}{\arabic{enumi}.}
	% A(\Alph) a(\alph) I(\Roman) i(\roman) 1(\arabic)
	%设定全局标号series=example	%引用全局变量resume=example
	%[topsep=-0.3em,parsep=-0.3em,itemsep=-0.3em,partopsep=-0.3em]
	%可使用leftmargin调整列表环境左边的空白长度 [leftmargin=0em]
	\item
一简谐横波沿 $ x $ 轴正向传播,$ t=0 $ 时刻的波形如图 \subref{2012新课标3401a} 所示,$ x=0.30 \ m $ 处的质点的振
动图线如图 \subref{2012新课标3401b} 所示,该质点在 $ t=0 $ 时刻的运动方向沿 $ y $ 轴 \underlinegap (填“正向”或“负向”)。已知
该波的波长大于 $ 0.30 \ m $,则该波的波长为 \underlinegap $ m $。
\begin{figure}[h!]
	\centering
\begin{subfigure}{0.4\linewidth}
	\centering
	\includesvg[width=0.7\linewidth]{picture/svg/GZ-3-tiyou-1587} 
	\caption{}\label{2012新课标3401a}
\end{subfigure}
\begin{subfigure}{0.4\linewidth}
	\centering
	\includesvg[width=0.7\linewidth]{picture/svg/GZ-3-tiyou-1588} 
	\caption{}\label{2012新课标3401b}
\end{subfigure}
\end{figure}


 \tk{正向 \quad $ 0.8 \ m $} 


\item 
一玻璃立方体中心有一点状光源。今在立方体的部分表面镀上不透明薄膜,以致从光源
发出的光线只经过一次折射不能透出立方体。已知该玻璃的折射率为 $ \sqrt{2} $,求镀膜的面积与立方体
表面积之比的最小值。


\banswer{
	$\frac{S^{\prime}}{S}=\frac{\pi}{4}$
}


\end{enumerate}


\item 
\exwhere{$ 2012 $年理综山东卷}
\begin{enumerate}
	%\renewcommand{\labelenumi}{\arabic{enumi}.}
	% A(\Alph) a(\alph) I(\Roman) i(\roman) 1(\arabic)
	%设定全局标号series=example	%引用全局变量resume=example
	%[topsep=-0.3em,parsep=-0.3em,itemsep=-0.3em,partopsep=-0.3em]
	%可使用leftmargin调整列表环境左边的空白长度 [leftmargin=0em]
	\item
一列简谐横波沿$ x $轴正方向传播,$ t=0 $时刻的波形如图所示,介质中质点$ P $、$ Q $分别位于$ x=2 \ m $、
$ x=4 \ m $处。从$ t=0 $时刻开始计时,当$ t=15 \ s $时质点$ Q $刚好第$ 4 $次
到达波峰。
\begin{enumerate}
	%\renewcommand{\labelenumi}{\arabic{enumi}.}
	% A(\Alph) a(\alph) I(\Roman) i(\roman) 1(\arabic)
	%设定全局标号series=example	%引用全局变量resume=example
	%[topsep=-0.3em,parsep=-0.3em,itemsep=-0.3em,partopsep=-0.3em]
	%可使用leftmargin调整列表环境左边的空白长度 [leftmargin=0em]
	\item
求波速。

\item 
写出质点$ P $做简谐运动的表达式(不要求推导过程) 。

	
\end{enumerate}
\begin{figure}[h!]
	\flushright
	\includesvg[width=0.25\linewidth]{picture/svg/GZ-3-tiyou-1589}
\end{figure}


\banswer{
	\begin{enumerate}
		%\renewcommand{\labelenumi}{\arabic{enumi}.}
		% A(\Alph) a(\alph) I(\Roman) i(\roman) 1(\arabic)
		%设定全局标号series=example	%引用全局变量resume=example
		%[topsep=-0.3em,parsep=-0.3em,itemsep=-0.3em,partopsep=-0.3em]
		%可使用leftmargin调整列表环境左边的空白长度 [leftmargin=0em]
		\item
		$ v=1 \ m/s  $
		\item 
		$y=0.2 \sin (0.5 \pi t) \ m$
	\end{enumerate}
}



\item 
如图所示,一玻璃球体的半径为$ R $,$ O $为球心,$ AB $为直径。来自$ B $ 点的光线$ BM $在$ M $点射出。出
射光线平行于$ AB $,另一光线$ BN $恰好在$ N $点发生全反射。已知
$ \angle ABM=300 $,求:
\begin{enumerate}
	%\renewcommand{\labelenumi}{\arabic{enumi}.}
	% A(\Alph) a(\alph) I(\Roman) i(\roman) 1(\arabic)
	%设定全局标号series=example	%引用全局变量resume=example
	%[topsep=-0.3em,parsep=-0.3em,itemsep=-0.3em,partopsep=-0.3em]
	%可使用leftmargin调整列表环境左边的空白长度 [leftmargin=0em]
	\item
玻璃的折射率。

\item 
球心$ O $到$ BN $的距离 。
\end{enumerate}
\begin{figure}[h!]
	\flushright
	\includesvg[width=0.25\linewidth]{picture/svg/GZ-3-tiyou-1590}
\end{figure}


\banswer{
	\begin{enumerate}
		%\renewcommand{\labelenumi}{\arabic{enumi}.}
		% A(\Alph) a(\alph) I(\Roman) i(\roman) 1(\arabic)
		%设定全局标号series=example	%引用全局变量resume=example
		%[topsep=-0.3em,parsep=-0.3em,itemsep=-0.3em,partopsep=-0.3em]
		%可使用leftmargin调整列表环境左边的空白长度 [leftmargin=0em]
		\item
		$ n=\sqrt{3} $
		\item 
		$d=\frac{\sqrt{3}}{3} R$
	\end{enumerate}
}



\end{enumerate}


\item 
\exwhere{$ 2012 $ 年物理江苏卷}
\begin{enumerate}
	%\renewcommand{\labelenumi}{\arabic{enumi}.}
	% A(\Alph) a(\alph) I(\Roman) i(\roman) 1(\arabic)
	%设定全局标号series=example	%引用全局变量resume=example
	%[topsep=-0.3em,parsep=-0.3em,itemsep=-0.3em,partopsep=-0.3em]
	%可使用leftmargin调整列表环境左边的空白长度 [leftmargin=0em]
	\item
如图所示,白炽灯的右侧依次平行放置偏振片 $ P $ 和 $ Q $,$ A $ 点位于 $ P $、$ Q $ 之间,$ B $ 点位于
$ Q $ 右侧. 旋转偏振片 $ P $,$ A $、$ B $ 两点光的强度变化情况是 \underlinegap .
\begin{figure}[h!]
	\centering
	\includesvg[width=0.23\linewidth]{picture/svg/GZ-3-tiyou-1591}
\end{figure}

\fourchoices
{$A $、$ B $ 均不变}
{$A $、$ B $ 均有变化}
{$A $ 不变,$ B $ 有变化}
{$A $ 有变化,$ B $ 不变}

 \tk{C} 


\item 
“测定玻璃的折射率”实验中,在玻璃砖的一侧竖直插两个大头
针$ A $、$ B $,在另一侧再竖直插两个大头针$ C $、D. 在插入第四个大
头针$ D $时,要使它 \underline{\hbox to 30mm{}}  .下图是在白纸上留下
的实验痕迹,其中直线$ a $、$ a ^{\prime} $ 是描在纸上的玻璃砖的两个边. 根
据该图可算得玻璃的折射率$ n= $ \underlinegap 。 (计算结果保留两
位有效数字)
\begin{figure}[h!]
	\centering
	\includesvg[width=0.23\linewidth]{picture/svg/GZ-3-tiyou-1592}
\end{figure}


 \tk{挡住 $ C $ 及 $ A $、$ B $ 的像;$ 1.8 $($ 1.6 \sim 1.9 $ 都算对)} 


\item 
地震时,震源会同时产生两种波,一种是传播速度约为 $ 3.5 \ km /s $ 的 $ S $ 波,另一种是传播速度约为
$ 7.0 \ km /s $ 的 $ P $ 波. 一次地震发生时,某地震监测点记录到首次到达的 $ P $ 波比首次到达的 $ S $ 波早 $ 3 \ min $。
假定地震波沿直线传播,震源的振动周期为 $ 1.2 \ s $,求震源与监测点之间的距离 $ x $ 和 $ S $ 波的波长$ \lambda $.

\banswer{
	$ x=1260 \ km $ \quad $  \lambda =4.2 \ km $
}



\end{enumerate}

\item 
\exwhere{$ 2012 $ 年物理海南卷}
\begin{enumerate}
	%\renewcommand{\labelenumi}{\arabic{enumi}.}
	% A(\Alph) a(\alph) I(\Roman) i(\roman) 1(\arabic)
	%设定全局标号series=example	%引用全局变量resume=example
	%[topsep=-0.3em,parsep=-0.3em,itemsep=-0.3em,partopsep=-0.3em]
	%可使用leftmargin调整列表环境左边的空白长度 [leftmargin=0em]
	\item
某波源 $ S $ 发出一列简谐横波,波源 $ S $ 的振动
图像如图所示。在波的传播方向上有 $ A $、$ B $ 两点,它们到
$ S $ 的距离分别为 $ 45 \ m $ 和 $ 55 \ m $。测得 $ A $、$ B $ 两点开始振动的
时间间隔为 $ 1.0 \ s $。由此可知:
\begin{figure}[h!]
	\centering
	\includesvg[width=0.23\linewidth]{picture/svg/GZ-3-tiyou-1593}
\end{figure}

\begin{enumerate}
	%\renewcommand{\labelenumi}{\arabic{enumi}.}
	% A(\Alph) a(\alph) I(\Roman) i(\roman) 1(\arabic)
	%设定全局标号series=example	%引用全局变量resume=example
	%[topsep=-0.3em,parsep=-0.3em,itemsep=-0.3em,partopsep=-0.3em]
	%可使用leftmargin调整列表环境左边的空白长度 [leftmargin=0em]
	\item
波长$ \lambda = $ \underlinegap $ m $;
\item 
当 $ B $ 点离开平衡位置的位移为$ +6 \ cm $ 时,$ A $ 点离开平衡位置的位移是 \underlinegap $ cm $。

\end{enumerate}


 \tk{
\begin{enumerate}
	%\renewcommand{\labelenumi}{\arabic{enumi}.}
	% A(\Alph) a(\alph) I(\Roman) i(\roman) 1(\arabic)
	%设定全局标号series=example	%引用全局变量resume=example
	%[topsep=-0.3em,parsep=-0.3em,itemsep=-0.3em,partopsep=-0.3em]
	%可使用leftmargin调整列表环境左边的空白长度 [leftmargin=0em]
	\item
	$ 20 $
	\item 
	$ -6 $
\end{enumerate}
} 

\item 
一玻璃三棱镜,其横截面为等腰三角形,顶角$ \theta $为锐角,折射率
为 $ \sqrt{2} $。现在横截面内有一光线从其左侧面上半部射入棱镜。不考虑棱镜内
部的反射。若保持入射线在过入射点的法线的下方一侧(如图),且要求入
射角为任何值的光线都会从棱镜的右侧面射出,则顶角$ \theta $可在什么范围内取
值?
\begin{figure}[h!]
	\flushright
	\includesvg[width=0.25\linewidth]{picture/svg/GZ-3-tiyou-1594}
\end{figure}

\banswer{
	$0<\theta<45^{0}$
}



\end{enumerate}


\item 
\exwhere{$ 2011 $ 年新课标卷}
\begin{enumerate}
	%\renewcommand{\labelenumi}{\arabic{enumi}.}
	% A(\Alph) a(\alph) I(\Roman) i(\roman) 1(\arabic)
	%设定全局标号series=example	%引用全局变量resume=example
	%[topsep=-0.3em,parsep=-0.3em,itemsep=-0.3em,partopsep=-0.3em]
	%可使用leftmargin调整列表环境左边的空白长度 [leftmargin=0em]
	\item
一振动周期为 $ T $,振幅为 $ A $,位于 $ x=0 $ 点的波源从平衡位置沿 $ y $ 轴正向开始做简谐振
动,该波源产生的一维简谐横波沿 $ x $ 轴正向传播,波速为 $ v $,传播过程中无能量损失,一段时间
后,该振动传播至某质点 $ P $,关于质点 $ P $ 振动的说法正确的是 \underlinegap 。
\fivechoices
{振幅一定为 $ A $}
{周期一定为 $ T $}
{速度的最大值一定为 $ v $}
{开始振动的方向沿 $ y $ 轴向上或向下取决于它离波源的距离}
{若 $ P $ 点与波源距离 $ s=vT $,则质点 $ P $ 的位移与波源的相同}


 \tk{ABE} 

\item 
一半圆柱形透明物体横截面如图所示,底面 $ AOB $
镀银(图中粗线),$ O $ 表示半圆截面的圆心。一束光线在横截面内从 $ M $ 点的入射角为 $ 30 \degree $,
$ \angle MOA=60 \degree $,$ \angle NOB=30 \degree $。求:
\begin{enumerate}
	%\renewcommand{\labelenumi}{\arabic{enumi}.}
	% A(\Alph) a(\alph) I(\Roman) i(\roman) 1(\arabic)
	%设定全局标号series=example	%引用全局变量resume=example
	%[topsep=-0.3em,parsep=-0.3em,itemsep=-0.3em,partopsep=-0.3em]
	%可使用leftmargin调整列表环境左边的空白长度 [leftmargin=0em]
	\item
光线在 $ M $ 点的折射角;
\item 
透明物体的折射率。
	
\end{enumerate}
\begin{figure}[h!]
	\flushright
	\includesvg[width=0.25\linewidth]{picture/svg/GZ-3-tiyou-1595}
\end{figure}


\banswer{
	\begin{enumerate}
		%\renewcommand{\labelenumi}{\arabic{enumi}.}
		% A(\Alph) a(\alph) I(\Roman) i(\roman) 1(\arabic)
		%设定全局标号series=example	%引用全局变量resume=example
		%[topsep=-0.3em,parsep=-0.3em,itemsep=-0.3em,partopsep=-0.3em]
		%可使用leftmargin调整列表环境左边的空白长度 [leftmargin=0em]
		\item
		$ 15 \degree  $
		\item 
		$n=\frac{\sqrt{6}+\sqrt{2}}{2}$
	\end{enumerate}
}




\end{enumerate}


\item
\exwhere{$ 2011 $ 年海南卷}
\begin{enumerate}
	%\renewcommand{\labelenumi}{\arabic{enumi}.}
	% A(\Alph) a(\alph) I(\Roman) i(\roman) 1(\arabic)
	%设定全局标号series=example	%引用全局变量resume=example
	%[topsep=-0.3em,parsep=-0.3em,itemsep=-0.3em,partopsep=-0.3em]
	%可使用leftmargin调整列表环境左边的空白长度 [leftmargin=0em]
	\item
一列简谐横波在 $ t=0 $ 时的波形图如图所示。介质
中 $ x=2 \ m $ 处的质点 $ P $ 沿 $ y $ 轴方向做简谐运动的表达式为 $ y=10 \sin (5 \pi t ) \ cm $。关于这列简谐波,下列说法正确的是 \underlinegap (填
入正确选项前的字母。选对 $ 1 $ 个给 $ 2 $ 分,选对 $ 2 $ 个给 $ 4 $ 分;选
错 $ 1 $ 个扣 $ 2 $ 分,最低得 $ 0 $ 分)。
\begin{figure}[h!]
	\centering
	\includesvg[width=0.23\linewidth]{picture/svg/GZ-3-tiyou-1596}
\end{figure}

\fourchoices
{周期为 $ 4.0 \ s $}
{振幅为 $ 20 \ cm $}
{传播方向沿 $ x $ 轴正向}
{传播速度为 $ 10 \ m /s $}

 \tk{CD} 

\item 
一赛艇停在平静的水面上,赛艇前端有一标记 $ P $ 离水面的高度为 $ h_{1} =0.6 \ m $,尾部下端
$ Q $ 略高于水面;赛艇正前方离赛艇前端 $ s_{1} =0.8 \ m $ 处有一浮标,示意如图。一潜水员在浮标前方
$ s_{2} =3.0 \ m $ 处下潜到深度为 $ h_{2} =4.0 \ m $ 时,看到标记刚好被浮标挡住,此处看不到船尾端 $ Q $;继续下潜
$ \triangle h=4.0 \ m $,恰好能看见 $ Q $。求:
\begin{enumerate}
	%\renewcommand{\labelenumi}{\arabic{enumi}.}
	% A(\Alph) a(\alph) I(\Roman) i(\roman) 1(\arabic)
	%设定全局标号series=example	%引用全局变量resume=example
	%[topsep=-0.3em,parsep=-0.3em,itemsep=-0.3em,partopsep=-0.3em]
	%可使用leftmargin调整列表环境左边的空白长度 [leftmargin=0em]
	\item
水的折射率 $ n $;
\item 
赛艇的长度 $ l $。(可用根式表示)
\end{enumerate}
\begin{figure}[h!]
	\flushright
	\includesvg[width=0.25\linewidth]{picture/svg/GZ-3-tiyou-1597}
\end{figure}


\banswer{
	\begin{enumerate}
		%\renewcommand{\labelenumi}{\arabic{enumi}.}
		% A(\Alph) a(\alph) I(\Roman) i(\roman) 1(\arabic)
		%设定全局标号series=example	%引用全局变量resume=example
		%[topsep=-0.3em,parsep=-0.3em,itemsep=-0.3em,partopsep=-0.3em]
		%可使用leftmargin调整列表环境左边的空白长度 [leftmargin=0em]
		\item
		$  \frac{ 4 }{ 3 }  $
		\item 
		$ 3.3 \ m $
	\end{enumerate}
}



	
\end{enumerate}



\item 
\exwhere{$ 2011 $ 年理综山东卷}
\begin{enumerate}
	%\renewcommand{\labelenumi}{\arabic{enumi}.}
	% A(\Alph) a(\alph) I(\Roman) i(\roman) 1(\arabic)
	%设定全局标号series=example	%引用全局变量resume=example
	%[topsep=-0.3em,parsep=-0.3em,itemsep=-0.3em,partopsep=-0.3em]
	%可使用leftmargin调整列表环境左边的空白长度 [leftmargin=0em]
	\item
如图所示,一列简谐波沿 $ x $ 轴传播,实线为 $ t_{1} =0 $ 时
的波形图,此时 $ P $ 质点向 $ y $ 轴负方向运动,虚线为
$ t_{2} =0.01 \ s $ 时的波形图。已知周期 $ T>0.01 \ s $。
\begin{enumerate}
	%\renewcommand{\labelenumi}{\arabic{enumi}.}
	% A(\Alph) a(\alph) I(\Roman) i(\roman) 1(\arabic)
	%设定全局标号series=example	%引用全局变量resume=example
	%[topsep=-0.3em,parsep=-0.3em,itemsep=-0.3em,partopsep=-0.3em]
	%可使用leftmargin调整列表环境左边的空白长度 [leftmargin=0em]
	\item
波沿 $ x $ 轴 \underlinegap (填“正”或“负”)方向传播。
\item 
求波速。
\end{enumerate}
\begin{figure}[h!]
	\flushright
	\includesvg[width=0.25\linewidth]{picture/svg/GZ-3-tiyou-1598}
\end{figure}


\banswer{
	\begin{enumerate}
		%\renewcommand{\labelenumi}{\arabic{enumi}.}
		% A(\Alph) a(\alph) I(\Roman) i(\roman) 1(\arabic)
		%设定全局标号series=example	%引用全局变量resume=example
		%[topsep=-0.3em,parsep=-0.3em,itemsep=-0.3em,partopsep=-0.3em]
		%可使用leftmargin调整列表环境左边的空白长度 [leftmargin=0em]
		\item
		正
		\item 
		$ 100 \ m/s  $
	\end{enumerate}
}



\item 
如图所示,扇形 $ AOB $ 为透明柱状介质的横截面,圆心角
$ \angle AOB=60 \degree $。一束平行于角平分线 $ OM $ 的单色光由 $ OA $ 射入介质,
经 $ OA $ 折射的光线恰平行于 $ OB $。
\begin{enumerate}
	%\renewcommand{\labelenumi}{\arabic{enumi}.}
	% A(\Alph) a(\alph) I(\Roman) i(\roman) 1(\arabic)
	%设定全局标号series=example	%引用全局变量resume=example
	%[topsep=-0.3em,parsep=-0.3em,itemsep=-0.3em,partopsep=-0.3em]
	%可使用leftmargin调整列表环境左边的空白长度 [leftmargin=0em]
	\item
求介质的折射率。
\item 
折射光线中恰好射到 $ M $ 点的光线 \underlinegap (填“能”或“不能”)

\end{enumerate}
\begin{figure}[h!]
	\flushright
	\includesvg[width=0.25\linewidth]{picture/svg/GZ-3-tiyou-1599}
\end{figure}



\banswer{
	\begin{enumerate}
		%\renewcommand{\labelenumi}{\arabic{enumi}.}
		% A(\Alph) a(\alph) I(\Roman) i(\roman) 1(\arabic)
		%设定全局标号series=example	%引用全局变量resume=example
		%[topsep=-0.3em,parsep=-0.3em,itemsep=-0.3em,partopsep=-0.3em]
		%可使用leftmargin调整列表环境左边的空白长度 [leftmargin=0em]
		\item
		$ \sqrt{3} $
		\item 
		不能
	\end{enumerate}
}



	
\end{enumerate}



\item
\exwhere{$ 2011 $ 年物理江苏卷}
\begin{enumerate}
	%\renewcommand{\labelenumi}{\arabic{enumi}.}
	% A(\Alph) a(\alph) I(\Roman) i(\roman) 1(\arabic)
	%设定全局标号series=example	%引用全局变量resume=example
	%[topsep=-0.3em,parsep=-0.3em,itemsep=-0.3em,partopsep=-0.3em]
	%可使用leftmargin调整列表环境左边的空白长度 [leftmargin=0em]
	\item
如图所示,沿平直铁路线有间距相等的三座铁塔 $ A $、$ B $ 和 $ C $。假想有一列车沿 $ AC $ 方向以接近光
速行驶,当铁塔 $ B $ 发出一个闪光,列车上的观测者测得 $ A $、$ C $ 两铁塔被照亮的顺序是 \xzanswer{C} 
\begin{figure}[h!]
	\centering
	\includesvg[width=0.23\linewidth]{picture/svg/GZ-3-tiyou-1600}
\end{figure}

\fourchoices
{同时被照亮}
{$ A $ 先被照亮}
{$ C $ 先被照亮}
{无法判断}



\item 
一束光从空气射向折射率为 $ \sqrt{3} $ 的某种介质,若反向光线与折射光线垂直,则入射角为 \underlinegap 。
真空中的光速为 $ c $,则光在该介质中的传播速度为 \underlinegap 。

 \tk{$ 60 \degree $ \quad $ \frac{\sqrt{3}}{3}c $} 

\item 
将一劲度系数为 $ k $ 的轻质弹簧竖直悬挂,下端系上质量为 $ m $ 的物块。将物块向下拉离平衡位置
后松开,物块上下做简谐运动,其振动周期恰好等于以物块平衡时弹簧的伸长量为摆长的单摆周
期。请由单摆周期公式推算出物块做简谐运动的周期 $ T $。

 \tk{$T=2 \pi \sqrt{\frac{m}{k}}$} 

\end{enumerate}


	
	
	
\end{enumerate}

