\bta{法拉第电磁感应定律}



\begin{enumerate}
\renewcommand{\labelenumi}{\arabic{enumi}.}
% A(\Alph) a(\alph) I(\Roman) i(\roman) 1(\arabic)
%设定全局标号series=example	%引用全局变量resume=example
%[topsep=-0.3em,parsep=-0.3em,itemsep=-0.3em,partopsep=-0.3em]
%可使用leftmargin调整列表环境左边的空白长度 [leftmargin=0em]
\item
\exwhere{$ 2014 $年物理江苏卷}
如图所示,一正方形线圈的匝数为 $ n $,边长为 $ a $,线圈平面与匀强磁场垂直,且一半处在磁场中. 在 $ \Delta t $ 时间内,磁感应强度的方向不变,大小由 $ B $ 均匀地增大到 $ 2 B $.在此过程中,线圈中产生的感应电动势为 \xzanswer{B} 

\begin{minipage}[h!]{0.7\linewidth}
\vspace{0.3em}
\fourchoices
{$ \frac { B a ^ { 2 } } { 2 \Delta t } $}
{$ \frac { n B a ^ { 2 } } { 2 \Delta t } $}
{$ \frac { n B a ^ { 2 } } { \Delta t } $}
{$ \frac { 2 n B a ^ { 2 } } { \Delta t } $}

\vspace{0.3em}
\end{minipage}
\hfill
\begin{minipage}[h!]{0.3\linewidth}
\flushright
\vspace{0.3em}
\includesvg[width=0.5\linewidth]{picture/svg/307}
\vspace{0.3em}
\end{minipage}






\item 
\exwhere{$ 2014 $年理综安徽卷}
英国物理学家麦克斯韦认为,磁场变化时会在空间激发感生电场。如图所示,一个半径为$ r $的绝缘细圆环水平放置,环内存在竖直向上的匀强磁场$ B $,环上套一带电量为$ +q $的小球。已知磁感应强度$ B $随时间均匀增加,其变化率为$ k $,若小球在环上运动一周,则感生电场对小球的作用力所做功的大小是 \xzanswer{D} 

\begin{figure}[h!]
\centering
\includesvg[width=0.23\linewidth]{picture/svg/308}
\end{figure}


\fourchoices
{$ 0 $}
{$ \frac { 1 } { 2 } r ^ { 2 } q k $}
{$ 2 \pi r ^ { 2 } q k $}
{$ \pi r ^ { 2 } q k $}








\item 
\exwhere{$ 2013 $年北京卷}
如图,在磁感应强度为$ B $、方向垂直纸面向里的匀强磁场中,金属杆$ MN $在平行金属导轨上以速度$ v $向右匀速滑动,$ MN $中产生的感应电动势为$ E_1 $;若磁感应强度增为$ 2B $,其他条件不变,$ MN $中产生的感应电动势变为$ E_{2} $。则通过电阻$ R $的电流方向及$ E_{1} $与$ E_{2} $之比分别为 \xzanswer{C} 
\begin{figure}[h!]
\centering
\includesvg[width=0.23\linewidth]{picture/svg/311}
\end{figure}

\fourchoices
{$ c \rightarrow a , 2: 1 $}
{$ a \rightarrow c , 2: 1 $}
{$ a \rightarrow c , 1: 2 $}
{$ c \rightarrow a , 1: 2 $}

\item 
\exwhere{$ 2013 $年新课标 \lmd{1} 卷}
如图.在水平面(纸面)内有三根相同的均匀金属棒$ ab $、$ ac $和$ MN $,其中$ ab $、$ ac $在$ a $点接触,构成“$ V $”字型导轨。空间存在垂直于纸面的均匀磁场。用力使$ MN $向右匀速运动,从图示位置开始计时,运动中$ MN $始终与$ \angle bac $的平分线垂直且和导轨保持良好接触。下列关于回路中电流$ i $与时间$ t $的关系图线.可能正确的是 \xzanswer{A} 
\begin{figure}[h!]
\centering
\includesvg[width=0.23\linewidth]{picture/svg/309}\\
\includesvg[width=0.83\linewidth]{picture/svg/310}
\end{figure}


\item 
\exwhere{$ 2017 $年新课标 \lmd{1} 卷}
扫描隧道显微镜($ STM $)可用来探测样品表面原子尺寸上的形貌,为了有效隔离外界震动对$ STM $的扰动,在圆底盘周边沿其径向对称地安装若干对紫铜薄板,并施加磁场来快速衰减其微小震动,如图所示,无扰动时,按下列四种方案对紫铜薄板施加恒磁场;出现扰动后,对于紫铜薄板上下及其左右震动的衰减最有效的方案是 \xzanswer{A} 
\begin{figure}[h!]
\centering
\includesvg[width=0.23\linewidth]{picture/svg/312}\\
\includesvg[width=0.83\linewidth]{picture/svg/313}
\end{figure}




\item 
\exwhere{$ 2012 $年理综北京卷}
物理课上,老师做了一个奇妙的“跳环实验”。如图,她把一个带铁芯的线圈$ L $、开关$ S $和电源用导终连接起来后.将一金属套环置于线圈$ L $上,且使铁芯穿过套环。闭合开关$ S $的瞬间,套环立刻跳起。某同学另找来器材再探究此实验。他连接好电路,经重复试验,线圈上的套环均末动。对比老师演示的实验,下列四个选项中.导致套环未动的原因可能是 \xzanswer{D} 
\begin{figure}[h!]
\centering
\includesvg[width=0.23\linewidth]{picture/svg/314}
\end{figure}


\fourchoices
{线圈接在了直流电源上.}
{电源电压过高.}
{所选线圈的匝数过多,}
{所用套环的材料与老师的不同}

\item 
\exwhere{$ 2012 $年理综新课标卷}
如图,均匀磁场中有一由半圆弧及其直径构成的导线框,半圆直径与磁场边缘重合;磁场方向垂直于半圆面(纸面)向里,磁感应强度大小为$ B_{0} $.使该线框从静止开始绕过圆心$ O $、垂直于半圆面的轴以角速度$\omega$匀速转动半周,在线框中产生感应电流。现使线框保持图中所示位置,磁感应强度大小随时间线性变化。为了产生与线框转动半周过程中同样大小的电流,磁感应强度随时间的变化率$\frac { \Delta B } { \Delta t }$的大小应为 \xzanswer{C} 
\begin{figure}[h!]
\centering
\includesvg[width=0.23\linewidth]{picture/svg/315}
\end{figure}

\fourchoices
{$ \frac { 4 \omega B _ { 0 } } { \pi } $}
{$ \frac { 2 \omega B _ { 0 } } { \pi } $}
{$ \frac { \omega B _ { 0 } } { \pi } $}
{$ \frac { \omega B _ { 0 } } { 2 \pi } $}

\item 
\exwhere{$ 2014 $年理综山东卷}
如图,一端接有定值电阻的平行金属轨道固定在水平面内,通有恒定电流的长直绝缘导线垂直并紧靠轨道固定,导体棒与轨道垂直且接触良好。在向右匀速通过$ M $、$ N $两区的过程中,导体棒所受安培力分别用$ F_M $、$ F_{N} $表示。不计轨道电阻。以下叙述正确的是 \xzanswer{BCD} 
\begin{figure}[h!]
\centering
\includesvg[width=0.23\linewidth]{picture/svg/316}
\end{figure}




\fourchoices
{$ F_M $向右}
{$ F_{N} $向左 }
{$ F_M $逐渐增大}
{$ F_N $逐渐减小}




\item 
\exwhere{$ 2012 $年理综四川卷}
半径为$ a $右端开小口的导体圆环和长为$ 2a $的导体直杆,单位长度电阻均为$ R_{0} $。圆环水平固定放置,整个内部区域分布着竖直向下的匀强磁场,磁感应强度为$ B $。杆在圆环上以速度$ v $平行于直径$ CD $向右做匀速直线运动,杆始终有两点与圆环良好接触,从圆环中心$ O $开始,杆的位置由$ \theta $确定,如图所示。则 \xzanswer{AD} 
\begin{figure}[h!]
\centering
\includesvg[width=0.23\linewidth]{picture/svg/317}
\end{figure}



\fourchoices
{$ \theta =0 $时,杆产生的电动势为$ 2Bav $}
{$ \theta =\frac{\pi}{3} $时,杆产生的电动势为$\sqrt { 3 } B a v$}
{$ \theta =0 $时,杆受的安培力大小为$\frac { 2 B ^ { 2 } a v } { ( \pi + 2 ) R _ { 0 } }$}
{$ \theta =\frac{\pi}{3} $时,杆受的安培力大小为$\frac { 3 B ^ { 2 } a v } { ( 5 \pi + 3 ) R _ { 0 } }$}




\item 
\exwhere{$ 2011 $年理综广东卷}
将闭合多匝线圈置于仅随时间变化的磁场中,线圈平面与磁场方向垂直,关于线圈中产生的感应电动势和感应电流,下列表述正确的是 \xzanswer{C} 


\fourchoices
{感应电动势的大小与线圈的匝数无关}
{穿过线圈的磁通量越大,感应电动势越大}
{穿过线圈的磁通量变化越快,感应电动势越大}
{感应电流产生的磁场方向与原磁场方向始终相同}





\item 
\exwhere{$ 2014 $年理综新课标\lmd{1} 卷}
在法拉第时代,下列验证“由磁产生电”设想的实验中,能观察到感应电流的是 \xzanswer{D} 


\fourchoices
{将绕在磁铁上的线圈与电流表组合成一闭合回路,然后观察电流表的变化}
{在一通电线圈旁放置一连有电流表的闭合线圈,然后观察电流表的变化}
{将一房间内的线圈两端与相邻房间的电流表连接,往线圈中插入条形磁铁后,再到相邻房间去观察电流表的变化}
{绕在同一铁环上的两个线圈,分别接电源和电流表,在给线圈通电或断电的瞬间,观察电流表的变化}




\item 
\exwhere{$ 2018 $年海南卷}
如图,在磁感应强度大小为$ B $的匀强磁场中,有一面积为$ S $的矩形单匝闭合导线框$ abcd $,$ ab $边与磁场方向垂直,线框的电阻为$ R $。使线框以恒定角速度$ \omega $绕过$ ad $、$ bc $中点的轴旋转。下列说法正确的是 \xzanswer{AC} 


\begin{minipage}[h!]{0.7\linewidth}
\vspace{0.3em}
\fourchoices
{线框$ abcd $中感应电动势的最大值是 $B S \omega$}
{线框$ abcd $中感应电动势的有效值是 $B S \omega$}
{线框平面与磁场方向平行时,流经线框的电流最大}
{线框平面与磁场方向垂直时,流经线框的电流最大}

\vspace{0.3em}
\end{minipage}
\hfill
\begin{minipage}[h!]{0.3\linewidth}
\flushright
\vspace{0.3em}
\includesvg[width=0.5\linewidth]{picture/svg/370}
\vspace{0.3em}
\end{minipage}	


\item 
\exwhere{$ 2018 $年全国\lmd{1}卷}
如图,导体轨道$ OPQS $固定,其中$ PQS $是半圆弧,$ Q $为半圆弧的中点,$ O $为圆心。轨道的电阻忽略不计。$ OM $是有一定电阻、可绕$ O $转动的金属杆,$ M $端位于$ PQS $上,$ OM $与轨道接触良好。空间存在与半圆所在平面垂直的匀强磁场,磁感应强度的大小为$ B $。现使$ OM $从$ OQ $位置以恒定的角速度逆时针转到$ OS $位置并固定(过程$ I $);再使磁感应强度的大小以一定的变化率从$ B $增加到$ B ^{\prime} $(过程$ II $)。在过程$ I $、$ II $中,流过$ OM $的电荷量相等,则$\frac { B ^ { \prime } } { B }$等于 \xzanswer{B} 

\begin{minipage}[h!]{0.7\linewidth}
\vspace{0.3em}
\fourchoices
{$ \frac { 5 } { 4 } $}
{$ \frac { 3 } { 2 } $}
{$ \frac { 7 } { 4 } $}
{$ 2 $}
\vspace{0.3em}
\end{minipage}
\hfill
\begin{minipage}[h!]{0.3\linewidth}
\flushright
\vspace{0.3em}
\includesvg[width=0.6\linewidth]{picture/svg/318}
\vspace{0.3em}
\end{minipage}

\item 
\exwhere{$ 2018 $年全国\lmd{3}卷}
如图($ a $),在同一平面内固定有一长直导线$ PQ $和一导线框$ R $,$ R $在$ PQ $的右侧。导线$ PQ $中通有正弦交流电,电流的变化如图($ b $)所示,规定从$ Q $到$ P $为电流正方向。导线框$ R $中的感应电动势 \xzanswer{AC} 
\begin{figure}[h!]
\centering
\includesvg[width=0.23\linewidth]{picture/svg/319}
\end{figure}


\fourchoices
{在$t = \frac { T } { 4 }$时为零}
{在$t = \frac { T } { 2 }$时改变方向}
{在$t = \frac { T } { 2 }$时最大,且沿顺时针方向}
{在$ t=T $时最大,且沿顺时针方向}


\item 
\exwhere{$ 2014 $年理综新课标$ \lmd{1} $卷}
如图($ a $),线圈$ ab $、$ cd $绕在同一软铁芯上,在$ ab $线圈中通以变化的电流,测得$ cd $间的的电压如图($ b $)所示,已知线圈内部的磁场与流经的电流成正比, 则下列描述线圈$ ab $中电流随时间变化关系的图中,可能正确的是: \xzanswer{C} 
\begin{figure}[h!]
\centering
\includesvg[width=0.23\linewidth]{picture/svg/320}\\
\includesvg[width=0.83\linewidth]{picture/svg/321}
\end{figure}








\item 
\exwhere{$ 2014 $年物理江苏卷}
如图所示,在线圈上端放置一盛有冷水的金属杯,现接通交流电源,过了几分钟,杯内的水沸腾起来. 若要缩短上述加热时间,下列措施可行的有 \xzanswer{AB} 
\begin{figure}[h!]
\centering
\includesvg[width=0.23\linewidth]{picture/svg/322}
\end{figure}


\fourchoices
{增加线圈的匝数}
{提高交流电源的频率}
{将金属杯换为瓷杯}
{取走线圈中的铁芯}






\item 
\exwhere{$ 2015 $年理综新课标 \lmd{2} 卷}
如图,直角三角形金属框放置在匀强磁场中,磁感应强度大小为$ B $,方向平行于$ ab $边向上。当金属框绕$ ab $边以角速度ω逆时针转动时,$ a $、$ b $、$ c $三点的电势分别为$ U_a $、$ U_b $、$ U_c $.已知$ bc $边的长度为$ l $。下列判断正确的是 \xzanswer{C} 


\begin{minipage}[h!]{0.7\linewidth}
\vspace{0.3em}
\fourchoices
{$ U_a> U_c $,金属框中无电流}
{$ U_b >U_c $,金属框中电流方向沿$ a-b-c-a $}
{$U _ { b c } = - \frac { 1 } { 2 } B l ^ { 2 } \omega$,金属框中无电流}
{$U _ { b c } = \frac { 1 } { 2 } B l ^ { 2 } \omega$,金属框中电流方向沿$ a-c-b-a $ }
\vspace{0.3em}
\end{minipage}
\hfill
\begin{minipage}[h!]{0.3\linewidth}
\flushright
\vspace{0.3em}
\includesvg[width=0.5\linewidth]{picture/svg/323}
\vspace{0.3em}
\end{minipage}






\item 
\exwhere{$ 2015 $年海南卷}
如图,空间有一匀强磁场,一直金属棒与磁感应强度方向垂直,当它以速度$ v $沿与棒和磁感应强度都垂直的方向运动时,棒两端的感应电动势大小为$ \varepsilon $;将此棒弯成两段长度相等且相互垂直的折弯,置于与磁感应强度相垂直的平面内,当它沿两段折线夹角平分线的方向以速度$ v $运动时,棒两端的感应电动势大小为$\varepsilon ^ { \prime }$。则$\frac { \varepsilon ^ { \prime } } { \varepsilon }$等于 \xzanswer{B} 

\begin{minipage}[h!]{0.7\linewidth}
\vspace{0.3em}
\fourchoices
{$ \frac { 1 } { 2 } $}
{$ \frac { \sqrt { 2 } } { 2 } $}
{$ 1 $}
{$ \sqrt { 2 } $}
\vspace{0.3em}
\end{minipage}
\hfill
\begin{minipage}[h!]{0.3\linewidth}
\flushright
\vspace{0.3em}
\includesvg[width=0.5\linewidth]{picture/svg/325}
\vspace{0.3em}
\end{minipage}



\item 
\exwhere{$ 2014 $年理综四川卷}
如图所示,不计电阻的光滑$ U $形金属框水平放置,光滑竖直玻璃挡板$ H $、$ P $固定在框上,$ H $、$ P $的间距很小。质量为$ 0.2 \ kg $的细金属杆$ CD $恰好无挤压地放在两挡板之间,与金属框接触良好并围成边长为$ 1 \ m $的正方形,其有效电阻为$ 0.1 \Omega $。此时在整个空间加方向与水平面成$ 30 ^{\circ} $角且与金属杆垂直的匀强磁场,磁感应强度随时间变化规律是$ B = (0.4 -0.2t ) T $,图示磁场方向为正方向。框、挡板和杆不计形变。则: \xzanswer{AC} 
\begin{figure}[h!]
\centering
\includesvg[width=0.23\linewidth]{picture/svg/326}
\end{figure}



\fourchoices
{$ t = 1 \ s $时,金属杆中感应电流方向从$ C $至$ D $}
{$ t = 3 \ s $时,金属杆中感应电流方向从$ D $至$ C $}
{$ t = 1 \ s $时,金属杆对挡板$ P $的压力大小为$ 0.1 \ N $}
{$ t = 3 \ s $时,金属杆对挡板$ H $的压力大小为$ 1.2 \ N $}

\item 
\exwhere{$ 2016 $年新课标 \lmd{2} 卷}
法拉第圆盘发电机的示意图如图所示。铜圆盘安装在竖直的铜轴上,两铜片$ P $、$ Q $分别与圆盘的边缘和铜轴接触。圆盘处于方向竖直向上的匀强磁场$ B $中。圆盘旋转时,关于流过电阻$ R $的电流,下列说法正确的是 \xzanswer{AB} 
\begin{figure}[h!]
\centering
\includesvg[width=0.23\linewidth]{picture/svg/327}
\end{figure}


\fourchoices
{若圆盘转动的角速度恒定,则电流大小恒定}
{若从上向下看,圆盘顺时针转动,则电流沿$ a $到$ b $的方向流动}
{若圆盘转动方向不变,角速度大小发生变化,则电流方向可能发生变化}
{若圆盘转动的角速度变为原来的$ 2 $倍,则电流在$ R $上的热功率也变为原来的$ 2 $倍}


\newpage	

\item 
\exwhere{$ 2015 $年江苏卷}
做磁共振$ (MRI) $检查时,对人体施加的磁场发生变化时会在肌肉组织中产生感应电流。 某同学为了估算该感应电流对肌肉组织的影响,将包裹在骨骼上的一圈肌肉组织等效成单匝线圈,线圈的半径 $ r = 5. 0 \ cm $,线圈导线的截面积 $ A = 0.80 \ cm^{2} $,电阻率$ \rho $$ = $ $ 1.5 \Omega \cdot m $. 如图所示,匀强磁场方向与线圈平面垂直,若磁感应强度 $ B $ 在 $ 0 .3 $ $ s $ 内从$ 1.5 $ $ T $ 均匀地减为零,求:
(计算结果保留一位有效数字)
\begin{enumerate}
\renewcommand{\labelenumii}{(\arabic{enumii})}

\item 
该圈肌肉组织的电阻 $ R $;

\item 
该圈肌肉组织中的感应电动势 $ E $; 

\item 
$ 0 . 3 $ $ s $ 内该圈肌肉组织中产生的热量 $ Q $.
\end{enumerate}	
\begin{figure}[h!]
\flushright
\includesvg[width=0.25\linewidth]{picture/svg/324}
\end{figure}

\banswer{
\begin{enumerate}
\renewcommand{\labelenumi}{\arabic{enumi}.}
% A(\Alph) a(\alph) I(\Roman) i(\roman) 1(\arabic)
%设定全局标号series=example	%引用全局变量resume=example
%[topsep=-0.3em,parsep=-0.3em,itemsep=-0.3em,partopsep=-0.3em]
%可使用leftmargin调整列表环境左边的空白长度 [leftmargin=0em]
\item
$R = 6 \times 10 ^ { 3 }\ \Omega$
\item 
$E = 4 \times 10 ^ { - 2 } \mathrm { V }$
\item 
$Q = 4 \times 10 ^ { - 8 } \mathrm { J }$


\end{enumerate}
}




\item 
\exwhere{$ 2014 $年理综浙江卷}
其同学设计一个发电测速装置,工作原理如图所示。一个半径为$ R=0.1m $的圆形金属导轨固定在竖直平面上,一根长为$ R $的金属棒$ OA $,$ A $端与导轨接触良好,$ O $端固定在圆心处的转轴上。转轴的左端有一个半径为$ r=R/3 $的圆盘,圆盘和金属棒能随转轴一起转动。圆盘上绕有不可伸长的细线,下端挂着一个质量为$ m=0.5 \ kg $的铝块。在金属导轨区域内存在垂直于导轨平面向右的匀强磁场,磁感应强度$ B=0.5 \ T $。$ a $点与导轨相连,$ b $点通过电刷与$ O $端相连。测量$ a $、$ b $两点间的电势差$ U $可算得铝块速度。铝块由静止释放,下落$ h=0.3\ m $时,测得$ U=0.15 \ V $。(细线与圆盘间没有滑动国,金属棒、导轨、导线及电刷的电阻均不计,重力加速度$ g=10 \ \ m/s ^{2} $).
\begin{enumerate}
\renewcommand{\labelenumi}{\arabic{enumi}.}
% A(\Alph) a(\alph) I(\Roman) i(\roman) 1(\arabic)
%设定全局标号series=example	%引用全局变量resume=example
%[topsep=-0.3em,parsep=-0.3em,itemsep=-0.3em,partopsep=-0.3em]
%可使用leftmargin调整列表环境左边的空白长度 [leftmargin=0em]
\item
测$ U $时,$ a $点相接的是电压表的“正极”还是“负极”?
\item 
求此时铝块的速度大小;
\item 
求此下落过程中铝块机械能的损失。



\end{enumerate}
\begin{figure}[h!]
\flushright 
\includesvg[width=0.3\linewidth]{picture/svg/328}
\end{figure}


\banswer{
\begin{enumerate}
\renewcommand{\labelenumi}{\arabic{enumi}.}
% A(\Alph) a(\alph) I(\Roman) i(\roman) 1(\arabic)
%设定全局标号series=example	%引用全局变量resume=example
%[topsep=-0.3em,parsep=-0.3em,itemsep=-0.3em,partopsep=-0.3em]
%可使用leftmargin调整列表环境左边的空白长度 [leftmargin=0em]
\item
根据右手定则可判断$ a $点为电源的正极,故$ a $点相接的是电压表的正极。
\item 
$v = 2\ \mathrm { m } / \mathrm { s }$
\item 
$\Delta E = m g h - \frac { 1 } { 2 } m v ^ { 2 } = 0.5 \mathrm { J }$


\end{enumerate}


}











\end{enumerate}




