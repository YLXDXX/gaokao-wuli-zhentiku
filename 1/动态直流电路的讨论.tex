\bta{动态直流电路的讨论}

\begin{enumerate}
%\renewcommand{\labelenumi}{\arabic{enumi}.}
% A(\Alph) a(\alph) I(\Roman) i(\roman) 1(\arabic)
%设定全局标号series=example	%引用全局变量resume=example
%[topsep=-0.3em,parsep=-0.3em,itemsep=-0.3em,partopsep=-0.3em]
%可使用leftmargin调整列表环境左边的空白长度 [leftmargin=0em]
\item
\exwhere{$ 2011 $ 年理综北京卷}
如图所示电路,电源内阻不可忽略。开关 $ S $ 闭合后,在变阻器 $ R_{0} $ 的滑动端向下滑动的过程中 \xzanswer{A} 
\begin{figure}[h!]
\centering
\includesvg[width=0.23\linewidth]{picture/svg/GZ-3-tiyou-1114}
\end{figure}

\fourchoices
{电压表与电流表的示数都减小}
{电压表与电流表的示数都增大}
{电压表的示数增大,电流表的示数减小}
{电压表的示数减小,电流表的示数增大}


\item 
\exwhere{$ 2014 $ 年理综天津卷}
如图所示,电路中 $ R_{1} $、$ R_{2} $ 均为可变电阻,电源内阻不能忽略,平行板电容器 $ C $ 的极板水平放置.闭
合电键 $ S $,电路达到稳定时,带电油滴悬浮在两板之间静止不动.如果仅改变下列某一个条件,油
滴仍能静止不动的是 \xzanswer{B} 
\begin{figure}[h!]
\centering
\includesvg[width=0.23\linewidth]{picture/svg/GZ-3-tiyou-1115}
\end{figure}

\fourchoices
{增大 $ R_{1} $ 的阻值}
{增大 $ R_{2} $ 的阻值}
{增大两板间的距离}
{断开电键 $ S $}


\item 
\exwhere{$ 2014 $ 年物理上海卷}
如图,电路中定值电阻阻值 $ R $ 大于电源内阻阻值 $ r $。将滑动变阻器滑片向下滑动,理想电压表
$ V_{1} $、 $ V_{2} $、 $ V_{3} $ 示数变化量的绝对值分别为$ \triangle V_{1} $、$ \triangle V_{2} $、$ \triangle V_{3} $,理想电流表 $ A $ 示数变化量的绝对值为
$ \triangle I $,则 \xzanswer{ACD} 
\begin{figure}[h!]
\centering
\includesvg[width=0.23\linewidth]{picture/svg/GZ-3-tiyou-1116}
\end{figure}

\fourchoices
{$ A $ 的示数增大}
{$ V_{2} $ 的示数增大}
{$ \triangle V_{3} $ 与$ \triangle I $ 的比值大于 $ r $}
{$ \triangle V_{1} $ 大于$ \triangle V_{2} $}


\item 
\exwhere{$ 2013 $年江苏卷}
在输液时,药液有时会从针口流出体外,为了及时发现,设计了一种报警装置,电路如图所示。 $ M $
是贴在针口处的传感器,接触到药液时其电阻$ R_M $发生变化,导致$ S $ 两端电压$ U $ 增大,装置发出警
报,此时 \xzanswer{C} 
\begin{figure}[h!]
\centering
\includesvg[width=0.23\linewidth]{picture/svg/GZ-3-tiyou-1117}
\end{figure}

\fourchoices
{$ RM $变大,且$ R $ 越大,$ U $ 增大越明显}
{$ RM $变大,且$ R $ 越小,$ U $ 增大越明显}
{$ RM $变小,且$ R $ 越大,$ U $ 增大越明显}
{$ RM $ 变小,且 $ R $ 越小,$ U $ 增大越明显}


\item 
\exwhere{$ 2011 $ 年海南卷}
如图,$ E $ 为内阻不能忽略的电池,$ R_{1} $、$ R_{2} $、$ R_{3} $ 为定值电阻,
$ S_{0} $、$ S $ 为开关,$ V $ 与 $ A $ 分别为电压表与电流表。初始时 $ S_{0} $ 与 $ S $
均闭合,现将 $ S $ 断开,则 \xzanswer{B} 
\begin{figure}[h!]
\centering
\includesvg[width=0.23\linewidth]{picture/svg/GZ-3-tiyou-1118}
\end{figure}





\fourchoices
{$ V $ 的读数变大,$ A $ 的读数变小}
{$ V $ 的读数变大,$ A $ 的读数变大}
{$ V $ 的读数变小,$ A $ 的读数变小}
{$ V $ 的读数变小,$ A $ 的读数变大}



\item 
\exwhere{$ 2011 $年上海卷}
如图所示电路中,闭合电键$ S $,当滑动变阻器的滑动触头$ P $从最高端向下滑动时 \xzanswer{A} 
\begin{figure}[h!]
\centering
\includesvg[width=0.23\linewidth]{picture/svg/GZ-3-tiyou-1119}
\end{figure}

\fourchoices
{电压表$ V $读数先变大后变小,电流表$ A $读数变大}
{电压表$ V $读数先变小后变大,电流表$ A $读数变小}
{电压表$ V $读数先变大后变小,电流表$ A $读数先变小后变大}
{电压表$ V $读数先变小后变大,电流表$ A $读数先变大后变小}








\end{enumerate}

