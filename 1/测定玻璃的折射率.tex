\bta{测定玻璃的折射率}

\begin{enumerate}
	%\renewcommand{\labelenumi}{\arabic{enumi}.}
	% A(\Alph) a(\alph) I(\Roman) i(\roman) 1(\arabic)
	%设定全局标号series=example	%引用全局变量resume=example
	%[topsep=-0.3em,parsep=-0.3em,itemsep=-0.3em,partopsep=-0.3em]
	%可使用leftmargin调整列表环境左边的空白长度 [leftmargin=0em]
	\item
\exwhere{$ 2016 $ 年四川卷}
某同学通过实验测定半圆形玻璃砖的折射率 $ n $。如图 \subref{2016四川05a} 所示,$ O $ 是圆心,$ MN $ 是
法线,$ AO $、$ BO $ 分别表示某次测量时光线在空气和玻璃砖中的传播路径。该同学测得多组入射角 $ i $
和折射角 $ r $,作出 $ \sin i- \sin r $ 图像如图 \subref{2016四川05b} 所示。则 \xzanswer{B} 
\begin{figure}[h!]
	\centering
\begin{subfigure}{0.4\linewidth}
	\centering
	\includesvg[width=0.7\linewidth]{picture/svg/GZ-3-tiyou-1463} 
	\caption{}\label{2016四川05a}
\end{subfigure}
\begin{subfigure}{0.4\linewidth}
	\centering
	\includesvg[width=0.7\linewidth]{picture/svg/GZ-3-tiyou-1464} 
	\caption{}\label{2016四川05b}
\end{subfigure}
\end{figure}

\fourchoices
{光由 $ A $ 经 $ O $ 到 $ B $,$ n=1.5 $}
{光由 $ B $ 经 $ O $ 到 $ A $,$ n=1.5 $}
{光由 $ A $ 经 $ O $ 到 $ B $,$ n=0.67 $}
{光由 $ B $ 经 $ O $ 到 $ A $,$ n=0.67 $}


\item 
\exwhere{$ 2015 $ 年理综北京卷}
“测定玻璃的折射率”的实验中,在白纸上放好玻璃砖,$ aa ^{\prime} $ 和 $ bb ^{\prime} $ 分别是玻璃砖与空气的两个
界面,如图所示,在玻璃砖的一侧插上两枚大头针 $ P_{1} $ 和 $ P_{2} $,用“$ + $”表示大头针的位置,然后在另
一侧透过玻璃砖观察,并依次插上大头针 $ P_{3} $ 和 $ P_{4} $。在
插 $ P_{3} $ 和 $ P_{4} $ 时,应使 \underlinegap 。(选填选项前的字母)
\begin{figure}[h!]
	\centering
	\includesvg[width=0.23\linewidth]{picture/svg/GZ-3-tiyou-1465}
\end{figure}

\threechoices
{$ P_{3} $ 只挡住 $ P_{1} $ 的像}
{$ P_{4} $ 只挡住 $ P_{2} $ 的像}
{$ P_{3} $ 同时挡住 $ P_{1} $、$ P_{2} $ 的像}

 \tk{C} 

\item 
\exwhere{$ 2011 $ 年理综天津卷}
某同学用大头针、三角板、量角器等器材测半圆形玻璃砖的折射率。
开始玻璃砖的位置如图中实线所示,使大头针 $ P_{1} $、$ P_{2} $ 与圆心 $ O $ 在同一直线
上,该直线垂直于玻璃砖的直径边,然后使玻璃砖绕圆心 $ O $ 缓慢转动,同
时在玻璃砖的直径边一侧观察 $ P_{1} $、$ P_{2} $ 的像,且 $ P_{2} $ 的像挡住 $ P_{1} $ 的像,如此
观察,当玻璃砖转到图中虚线位置时,上述现象恰好消失。此时只须测量
出 \underlinegap ,即可计算出玻璃砖的折射率,请用你的测量量表示出
折射率 $ n= $ \underlinegap 。
\begin{figure}[h!]
	\centering
	\includesvg[width=0.23\linewidth]{picture/svg/GZ-3-tiyou-1466}
\end{figure}


 \tk{玻璃砖直角边绕$ O $点旋转过的角度$ \theta $ \quad $ \frac{1}{\sin \theta } $} 



\item 
\exwhere{$ 2012 $ 年理综浙江卷}
在“测定玻璃的折射率”实验中,某同学经正确操作插好了 $ 4 $ 枚大头针,如图 \subref{2012浙江21a} 所示。
\begin{figure}[h!]
	\centering
\begin{subfigure}{0.4\linewidth}
	\centering
	\includesvg[width=0.7\linewidth]{picture/svg/GZ-3-tiyou-1467} 
	\caption{}\label{2012浙江21a}
\end{subfigure}
\begin{subfigure}{0.4\linewidth}
	\centering
	\includesvg[width=0.7\linewidth]{picture/svg/GZ-3-tiyou-1468} 
	\caption{}\label{2012浙江21b}
\end{subfigure}
\end{figure}

\begin{enumerate}
	%\renewcommand{\labelenumi}{\arabic{enumi}.}
	% A(\Alph) a(\alph) I(\Roman) i(\roman) 1(\arabic)
	%设定全局标号series=example	%引用全局变量resume=example
	%[topsep=-0.3em,parsep=-0.3em,itemsep=-0.3em,partopsep=-0.3em]
	%可使用leftmargin调整列表环境左边的空白长度 [leftmargin=0em]
	\item
在答题纸上相应的图中画出完整的光路图;
\begin{figure}[h!]
	\centering
	\includesvg[width=0.23\linewidth]{picture/svg/GZ-3-tiyou-1471}
\end{figure}

\item 
对你画出的光路图进行测量和计算,求得该玻璃砖的折射率 $ n= $
 \underlinegap 
(保留 $ 3 $ 位有效数字);


\item 
为了观测光在玻璃砖不同表面的折射现象,某同学做了两次实验,经正确操作插好了 $ 8 $ 枚大头针,
如图 \subref{2012浙江21b} 所示。图中 $ P_{1} $ 和 $ P_{2} $ 是同一入射光线上的 $ 2 $ 枚大头针,其对应出射光线上的 $ 2 $ 枚大头针是 $ P_{3} $
和
 \underlinegap 
(填“$ A $”或“$ B $”)。

\end{enumerate}




 \tk{
\begin{enumerate}
	%\renewcommand{\labelenumi}{\arabic{enumi}.}
	% A(\Alph) a(\alph) I(\Roman) i(\roman) 1(\arabic)
	%设定全局标号series=example	%引用全局变量resume=example
	%[topsep=-0.3em,parsep=-0.3em,itemsep=-0.3em,partopsep=-0.3em]
	%可使用leftmargin调整列表环境左边的空白长度 [leftmargin=0em]
	\item
解析如图
\begin{center}
 \includesvg[width=0.83\linewidth]{picture/svg/GZ-3-tiyou-1470} 
\end{center}
\item 
$ n=1.51 $
\item 
$ A $
\end{enumerate}
} 


\item 
\exwhere{$ 2012 $ 年理综重庆卷}
\begin{enumerate}
	%\renewcommand{\labelenumi}{\arabic{enumi}.}
	% A(\Alph) a(\alph) I(\Roman) i(\roman) 1(\arabic)
	%设定全局标号series=example	%引用全局变量resume=example
	%[topsep=-0.3em,parsep=-0.3em,itemsep=-0.3em,partopsep=-0.3em]
	%可使用leftmargin调整列表环境左边的空白长度 [leftmargin=0em]
	\item
图 \subref{2012重庆22a} 所示为光学实验用的长方体玻璃砖,它的
 \underlinegap 
面不能用手直接接触。
在用插针法测定玻璃砖折射率的实验中,两
位同学绘出的玻璃砖和三个针孔 $ a $、$ b $、$ c $ 的
位置相同,且插在 $ c $ 位置的针正好挡住插在
$ a $、$ b $ 位置的针的像,但最后一个针孔的位置
不同,分别为 $ d $、$ e $ 两点,如图 \subref{2012重庆22b} 所示。
计算折射率时,用
 \underlinegap 
(填“$ d $”或“$ e $”)
 \underlinegap 
点得到的值较小,用
 \underlinegap 
(填“$ d $”或
“$ e $”)点得到的值误差较小。

	
\end{enumerate}
\begin{figure}[h!]
	\centering
\begin{subfigure}{0.4\linewidth}
	\centering
	\includesvg[width=0.7\linewidth]{picture/svg/GZ-3-tiyou-1472} 
	\caption{}\label{2012重庆22a}
\end{subfigure}
\begin{subfigure}{0.4\linewidth}
	\centering
	\includesvg[width=0.7\linewidth]{picture/svg/GZ-3-tiyou-1473} 
	\caption{}\label{2012重庆22b}
\end{subfigure}
\end{figure}

 \tk{光学 \quad $ d $ \quad $ e $} 


\item 
\exwhere{$ 2019 $ 年物理天津卷}
某小组做测定玻璃的折射率实验,所用器材有:玻璃砖,大头针,刻度
尺,圆规,笔,白纸。
\begin{enumerate}
	%\renewcommand{\labelenumi}{\arabic{enumi}.}
	% A(\Alph) a(\alph) I(\Roman) i(\roman) 1(\arabic)
	%设定全局标号series=example	%引用全局变量resume=example
	%[topsep=-0.3em,parsep=-0.3em,itemsep=-0.3em,partopsep=-0.3em]
	%可使用leftmargin调整列表环境左边的空白长度 [leftmargin=0em]
	\item
下列哪些措施能够提高实验准确程度 \underlinegap 。

\fourchoices
{选用两光学表面间距大的玻璃砖}
{选用两光学表面平行的玻璃砖}
{选用粗的大头针完成实验}
{插在玻璃砖同侧的两枚大头针间的距离尽量大些}



\item 
该小组用同一套器材完成了四次实验,记录的玻璃砖界线和四个大头针扎下的孔洞如下图所示,
其中实验操作正确的是 \underlinegap 。
\pfourchoices
{\includesvg[width=2.3cm]{picture/svg/GZ-3-tiyou-1474}}
{\includesvg[width=2.3cm]{picture/svg/GZ-3-tiyou-1475}}
{\includesvg[width=2.3cm]{picture/svg/GZ-3-tiyou-1476}}
{\includesvg[width=2.3cm]{picture/svg/GZ-3-tiyou-1477}}


\item 
该小组选取了操作正确的实验记录,在白纸上画出光线的径迹,以入射点 $ O $ 为圆心作圆,与入射
光线、折射光线分别交于 $ A $、 $ B $ 点,再过 $ A $、 $ B $ 点作法线 $ NN^{\prime} $的垂线,垂足分别为 $ C $、 $ D $ 点,如
图所示,则玻璃的折射率 $ n= $ \underlinegap 。(用图中线段的字母表示)
\begin{figure}[h!]
	\centering
	\includesvg[width=0.23\linewidth]{picture/svg/GZ-3-tiyou-1478}
\end{figure}

\end{enumerate}



 \tk{
\begin{enumerate}
	%\renewcommand{\labelenumi}{\arabic{enumi}.}
	% A(\Alph) a(\alph) I(\Roman) i(\roman) 1(\arabic)
	%设定全局标号series=example	%引用全局变量resume=example
	%[topsep=-0.3em,parsep=-0.3em,itemsep=-0.3em,partopsep=-0.3em]
	%可使用leftmargin调整列表环境左边的空白长度 [leftmargin=0em]
	\item
	AD
	\item 
	D
	\item 
	$ \frac{AC}{BD} $
\end{enumerate}
} 


	
\end{enumerate}

