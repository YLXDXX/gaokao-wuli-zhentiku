\bta{简单逻辑电路}

\begin{enumerate}
%\renewcommand{\labelenumi}{\arabic{enumi}.}
% A(\Alph) a(\alph) I(\Roman) i(\roman) 1(\arabic)
%设定全局标号series=example	%引用全局变量resume=example
%[topsep=-0.3em,parsep=-0.3em,itemsep=-0.3em,partopsep=-0.3em]
%可使用leftmargin调整列表环境左边的空白长度 [leftmargin=0em]
\item
\exwhere{$ 2012 $ 年物理上海卷}
如图,低电位报警器由两个基本的门电路与蜂鸣器组成,该报警
器只有当输入电压过低时蜂鸣器才会发出警报。其中 \xzanswer{B} 

\begin{table}[h!]
\centering 
\begin{tabular}{|c|c|c|}
\hline 
\multicolumn{2}{|c|}{输入} & 输出
 \\
\hline
A & B &Z
 \\
\hline
0 & 0 & 0
 \\
\hline
0 & 1 &1
 \\
\hline
1 & 0 &X
 \\
\hline
1 & 1 &1\\ 
\hline 
\end{tabular}
\hfil
\begin{tabular}{c} 
\includesvg[width=0.29\linewidth]{picture/svg/GZ-3-tiyou-1120} \\ 
\end{tabular}
\end{table} 


\fourchoices
{甲是“与门”,乙是“非门”}
{甲是“或门”,乙是“非门”}
{甲是“与门”,乙是“或门”}
{甲是“或门”,乙是“与门”}




\item 
\exwhere{$ 2011 $年上海卷}
右表是某逻辑电路的真值表,该电路是 \xzanswer{D} 
\begin{table}[h!]
\centering 
\begin{tabular}{|c|c|c|}
\hline 
\multicolumn{2}{|c|}{输入} & 输出
 \\
\hline
0 & 0 & 1
 \\
\hline
0 & 1 & 1
 \\
\hline
1 & 0 & 1
 \\
\hline
1 & 1 & 0\\ 
\hline 
\end{tabular}
\end{table} 

\pfourchoices
{\includesvg[width=3cm]{picture/svg/GZ-3-tiyou-1121}}
{\includesvg[width=3cm]{picture/svg/GZ-3-tiyou-1122}}
{\includesvg[width=3cm]{picture/svg/GZ-3-tiyou-1123}}
{\includesvg[width=3cm]{picture/svg/GZ-3-tiyou-1124}}



\item
\exwhere{$ 2013 $ 年上海卷}
在车门报警电路中,两个按钮开关分别装在汽车的两扇门上,只要有开关处于断开状态,
报警灯就发光。能实现此功能的电路是 \xzanswer{B} 


\pfourchoices
{\includesvg[width=4.3cm]{picture/svg/GZ-3-tiyou-1125}}
{\includesvg[width=4.3cm]{picture/svg/GZ-3-tiyou-1126}}
{\includesvg[width=4.3cm]{picture/svg/GZ-3-tiyou-1127}}
{\includesvg[width=4.3cm]{picture/svg/GZ-3-tiyou-1128}}


\item 
\exwhere{$ 2015 $ 年上海卷}
监控系统控制电路如图所示,电键 $ S $ 闭合时,系统白天和晚上都工作;
电键 $ S $ 断开时,系统仅晚上工作。在电路中虚框处分别接
入光敏电阻(受光照时阻值减小)和定值电阻,则电路中 \xzanswer{D} 
\begin{figure}[h!]
\centering
\includesvg[width=0.23\linewidth]{picture/svg/GZ-3-tiyou-1129}
\end{figure}



\fourchoices
{$ C $ 是“与门”,$ A $ 是光敏电阻}
{$ C $ 是“与门”,$ B $ 是光敏电阻}
{$ C $ 是“或门”,$ A $ 是光敏电阻}
{$ C $ 是“或门”,$ B $ 是光敏电阻}




\end{enumerate}

