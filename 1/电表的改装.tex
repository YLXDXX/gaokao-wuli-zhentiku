\bta{电表的改装}

\begin{enumerate}[leftmargin=0em]
\renewcommand{\labelenumi}{\arabic{enumi}.}
% A(\Alph) a(\alph) I(\Roman) i(\roman) 1(\arabic)
%设定全局标号series=example	%引用全局变量resume=example
%[topsep=-0.3em,parsep=-0.3em,itemsep=-0.3em,partopsep=-0.3em]
%可使用leftmargin调整列表环境左边的空白长度 [leftmargin=0em]
\item
\exwhere{$ 2019 $年物理全国\lmd{1}卷}
某同学要将一量程为$ 250 \ \mu A $的微安表改装为量程为$ 20 $ $ m_{A} $的电流表。该同学测得微安表内阻为$ 1 200 $ $ \Omega $,经计算后将一阻值为$ R $的电阻与微安表连接,进行改装。然后利用一标准毫安表,根据图($ a $)所示电路对改装后的电表进行检测(虚线框内是改装后的电表)。
\begin{figure}[h!]
\centering
\includesvg[width=0.73\linewidth]{picture/svg/651}
\end{figure}

\begin{enumerate}
\renewcommand{\labelenumi}{\arabic{enumi}.}
% A(\Alph) a(\alph) I(\Roman) i(\roman) 1(\arabic)
%设定全局标号series=example	%引用全局变量resume=example
%[topsep=-0.3em,parsep=-0.3em,itemsep=-0.3em,partopsep=-0.3em]
%可使用leftmargin调整列表环境左边的空白长度 [leftmargin=0em]
\item
根据图($ a $)和题给条件,将($ b $)中的实物连接。
\item 
当标准毫安表的示数为$ 16.0 $ $ m_{A} $时,微安表的指针位置如图($ c $)所示,由此可以推测出改装的电表量程不是预期值,而是 \xzanswer{C} 
。(填正确答案标号)
\begin{figure}[h!]
\centering
\includesvg[width=0.23\linewidth]{picture/svg/652}
\end{figure}

\fourchoices
{$ 18 $ $ m_{A} $ }
{$ 21 $ $ m_{A} $}
{$ 25 \ mA $ }
{$ 28 $ $ m_{A} $}


\item 
产生上述问题的原因可能是 \xzanswer{AC} 
。(填正确答案标号)
\fourchoices
{微安表内阻测量错误,实际内阻大于$ 1 200 $ $ \Omega $}
{微安表内阻测量错误,实际内阻小于$ 1 200 $ $ \Omega $}
{$ R $值计算错误,接入的电阻偏小}
{$ R $值计算错误,接入的电阻偏大}

\item 
要达到预期目的,无论测得的内阻值是都正确,都不必重新测量,只需要将阻值为$ R $的电阻换为一个阻值为$ kR $的电阻即可,其中$ k= $ \tk{$ \frac{99}{79} $} 。



\end{enumerate}

\banswer{
($ 1 $)电表改装时,微安表应与定值电阻$ R $并联接入虚线框内,则实物电路连接如下图所示:

\includesvg[width=0.23\linewidth]{picture/svg/653}


}


\newpage


\item 
\exwhere{$ 2011 $年海南卷}
图$ 1 $是改装并校准电流表的电路图,已知表头 \mammetermytikz 的量程为$ I_{g} =600 \ \mu A $,内阻为$ R_{g} $, \mammetermytikz 是标准电流表,要求改装后的电流表量程为$ I=60 \ mA $。完成下列填空。


\begin{enumerate}
\renewcommand{\labelenumi}{\arabic{enumi}.}
% A(\Alph) a(\alph) I(\Roman) i(\roman) 1(\arabic)
%设定全局标号series=example	%引用全局变量resume=example
%[topsep=-0.3em,parsep=-0.3em,itemsep=-0.3em,partopsep=-0.3em]
%可使用leftmargin调整列表环境左边的空白长度 [leftmargin=0em]
\item
图$ 1 $中分流电阻$ R_p $的阻值应为 \tk{$\frac { 1 } { 99 } R _ { g }$} (用$ I_{g} $、$ R_{g} $和 \lmd{1} 表示)。
\begin{figure}[h!]
\centering
\includesvg[width=0.53\linewidth]{picture/svg/656}
\end{figure}

\item 
在电表改装完成后的某次校准测量中, \mammetermytikz 表的示数如图$ 2 $所示,由此读出流过 \mammetermytikz 电流表的电流为\tk{49.5}$ m_{A} $。此时流过分流电阻$ R_P $的电流为\tk{49.0}$ m_{A} $(保留$ 1 $位小数)



\end{enumerate}


\item
\exwhere{$ 2013 $年海南卷}
某同学将量程为$ 200 \ \mu A $、内阻为$ 500 \ \Omega $的表头$ \mu A $改装成量程为$ 1 \ mA $和$ 10 \ mA $的双量程电流表,设计电路如图($ a $)所示。定值电阻$ R_{1} =500 \ \Omega $,$ R_{2} $和$ R_{3} $的值待定,$ S $为单刀双掷开关,$ A $、$ B $为接线柱。回答下列问题:
\begin{figure}[h!]
\centering
\includesvg[width=0.83\linewidth]{picture/svg/660}
\end{figure}

\begin{enumerate}
\renewcommand{\labelenumi}{\arabic{enumi}.}
% A(\Alph) a(\alph) I(\Roman) i(\roman) 1(\arabic)
%设定全局标号series=example	%引用全局变量resume=example
%[topsep=-0.3em,parsep=-0.3em,itemsep=-0.3em,partopsep=-0.3em]
%可使用leftmargin调整列表环境左边的空白长度 [leftmargin=0em]
\item
按图($ a $)在图($ b $)中将实物连线;
\item 
表笔$ a $的颜色为 \tk{黑} 色(填“红”或“黑”)
\item 
将开关$ S $置于“$ 1 $”挡时,量程为 \tk{10} $ m{A} $;
\item 
定值电阻的阻值$ R_{2} = $ \tk{225} $ \Omega $,$ R_3= $ \tk{25.0} $ \Omega $。(结果取$ 3 $位有效数字)
\item 
利用改装的电流表进行某次测量时,$ S $置于“$ 2 $”挡,表头指示如图($ c $)所示,则所测量电流的值为 \tk{$ 0.780(0.78 $同样给分)} $ m{A} $。



\end{enumerate}

\banswer{
$ (1) $如图:
\includesvg[width=0.23\linewidth]{picture/svg/661}

}





\newpage
\item 
\exwhere{$ 2016 $年海南卷}
某同学改装和校准电压表的电路图如图所示,图中虚线框内是电压表的改装电路。

\begin{enumerate}
\renewcommand{\labelenumi}{\arabic{enumi}.}
% A(\Alph) a(\alph) I(\Roman) i(\roman) 1(\arabic)
%设定全局标号series=example	%引用全局变量resume=example
%[topsep=-0.3em,parsep=-0.3em,itemsep=-0.3em,partopsep=-0.3em]
%可使用leftmargin调整列表环境左边的空白长度 [leftmargin=0em]
\item
已知表头$ G $满偏电流为$ 100 \ \mu A $,表头上标记的内阻值为$ 900 \ \Omega $。$ R_{1} $、$ R_{2} $和$ R_{3} $是定值电阻。利用$ R_{1} $和表头构成$ 1 $ $ m_{A} $的电流表,然后再将其改装为两个量程的电压表。若使用$ a $、$ b $两个接线柱,电压表的量程为$ 1 $ $ V $;若使用$ a $、$ c $两个接线柱,电压表的量程为$ 3 $ $ V $。则根据题给条件,定值电阻的阻值应选$ R_{1} = $\tk{100}$\Omega $,$ R_{2} =$\tk{910}$\Omega $,$ R_3= $\tk{2000}$\Omega $。
\begin{figure}[h!]
\centering
\includesvg[width=0.23\linewidth]{picture/svg/657}
\end{figure}

\item 
用量程为$ 3V $,内阻为$ 2500 \ \Omega $的标准电压表 \voltmetermytikz 对改装表$ 3V $挡的不同刻度进行校准。所用电池的电动势$ E $为$ 5V $;滑动变阻器$ R $有两种规格,最大阻值分别为$ 50 \ \Omega $和$ 5 \ k\Omega $。为了方便实验中调节电压,图中$ R $应选用最大阻值为\tk{50}$ \Omega $的滑动变阻器。
\item 
校准时,在闭合开关$ S $前,滑动变阻器的滑动端$ P $应靠近\tk{M}(填“$ M $”或“$ N $”)端。
\item 
若由于表头$ G $上标记的内阻值不准,造成改装后电压表的读数比标准电压表的读数偏小,则表头$ G $内阻的真实值\tk{大于}(填“大于”或“小于”)$ 900 \ \Omega $。



\end{enumerate}





\item 
\exwhere{$ 2013 $年新课标 \lmd{2} 卷}
某同学用量程为$ 1 $ $ mA $、内阻为$ 120 \ \Omega $ 的表头按图$ (a) $所示电路改装成量程分别为$ 1\ V $和$ 1\ A $的多用电表。图中$ R_{1} $和$ R_{2} $为定值电阻,$ S $为开关。回答下列问题$ : $
\begin{enumerate}
\renewcommand{\labelenumii}{(\arabic{enumii})}
\item 
根据图$ (a) $所示的电路,在图$ (b) $所示的实物图上连线。
\begin{figure}[h!]
\centering
\includesvg[width=0.43\linewidth]{picture/svg/658}
\end{figure}


\item 
开关$ S $闭合时,多用电表用于测量\tk{电流}(填“电流”、“电压,或“电阻”);
开关$ S $断开时,多用电表用于测量\tk{电压}(填“电流”、“电压”或“电阻”)。


\item 
表笔$ A $应为\tk{黑}色(填“红”或“黑”)。


\item 
定值电阻的阻值$ R_{1} = $ \tk{1.00} $\Omega $,
$ R_{2} = $\tk{880}$\Omega $。(结果取$ 3 $位有效数字)


\end{enumerate}

\banswer{
根据图(a)连线:电流表与$ R_2 $串联、开关与$ R_1 $串联,然后两支路并联分别接表笔A、B。
\includesvg[width=0.23\linewidth]{picture/svg/659}

}


\newpage

\item 
\exwhere{$ 2018 $年全国\lmd{2}卷}
某同学组装一个多用电表。可选用的器材有:微安表头(量程$ 100\ \mu A $,内阻$ 900\ \Omega $);电阻箱$ R_{1} $(阻值范围$ 0 \sim 999.9 \ \Omega $);电阻箱$ R_{1} $(阻值范围$ 0 \sim 99999.9 \ \Omega $);导线若干。

要求利用所给器材先组装一个量程为$ 1\ mA $的直流电流表,在此基础上再将它改装成量程为$ 3\ V $的直流电压表。组装好的多用电表有电流$ 1\ mA $和电压$ 3\ V $两挡。

回答下列问题:

\begin{enumerate}
\renewcommand{\labelenumi}{\arabic{enumi}.}
% A(\Alph) a(\alph) I(\Roman) i(\roman) 1(\arabic)
%设定全局标号series=example	%引用全局变量resume=example
%[topsep=-0.3em,parsep=-0.3em,itemsep=-0.3em,partopsep=-0.3em]
%可使用leftmargin调整列表环境左边的空白长度 [leftmargin=0em]
\item
在虚线框内画出电路图并标出$ R_{1} $和$ R_{2} $,其中 $ \star $ 为公共接线柱,$ a $和$ b $分别是电流挡和电压挡的接线柱。
\begin{figure}[h!]
\centering
\includesvg[width=0.23\linewidth]{picture/svg/654}
\end{figure}

\item 
电阻箱的阻值应取$ R_{1} = $ \tk{100} $ \Omega $,$ R_{2} = $ \tk{2910} $ \Omega $。(保留到个位)



\end{enumerate}


\banswer{
($ 1 $)如图所示 :
\includesvg[width=0.23\linewidth]{picture/svg/655}

}



\item 
\exwhere{$ 2012 $年理综天津卷}
某同学在进行扩大电流表量程的实验时,需要知道电流表的满偏电流和内阻。他设计了一个用标准电流表$ G_{1} $来校对待测电流表$ G_{2} $的满偏电流和测定$ G_{2} $内阻的电路,如图所示。已知$ G_{1} $的量程略大于$ G_{2} $的量程,图中$ R_{1} $为滑动变阻器。$ R_{2} $为电阻箱。该同学顺利完成了这个实验。
\begin{figure}[h!]
\centering
\includesvg[width=0.23\linewidth]{picture/svg/662}
\end{figure}

\begin{enumerate}
\renewcommand{\labelenumi}{\arabic{enumi}.}
% A(\Alph) a(\alph) I(\Roman) i(\roman) 1(\arabic)
%设定全局标号series=example	%引用全局变量resume=example
%[topsep=-0.3em,parsep=-0.3em,itemsep=-0.3em,partopsep=-0.3em]
%可使用leftmargin调整列表环境左边的空白长度 [leftmargin=0em]
\item
实验过程包含以下步骤,其合理的顺序依次为\tk{$ B $、$ E $、$ F $、$ A $、$ D $、$ C $}

A.合上开关$ S_{2} $

B.分别将$ R_{1} $和$ R_{2} $的阻值调至最大

C.记下$ R_{2} $的最终读数

D.反复调节$ R_{1} $和$ R_{2} $的阻值,使$ G_{1} $的示数仍为$ I_{1} $,使$ G_{2} $的指针偏转到满刻度的一半,此时$ R_{2} $的最终读数为$ r $

$ E $.合上开关$ S_{1} $

$ F $.调节$ R_{1} $使$ G_{2} $的指针偏转到满刻度,此时$ G_{1} $的示数为$ I_{1} $,记下此时$ G_{1} $的示数

\item 
仅从实验设计原理上看,用上述方法得到的$ G_{2} $内阻的测量值与其真实值相比\tk{相等}(填偏大、偏小或相等)
\item 
若要将$ G_{2} $的量程扩大为 \lmd{1} ,并结合前述实验过程中测量的结果,写出须在$ G_{2} $上并联的分流电阻$ R_S $的表达式,$ R_S= $ \tk{$\frac { I _ { 1 } } { I - I _ { 1 } } r$} .



\end{enumerate}


\newpage
\item 
\exwhere{$ 2015 $年理综新课标 \lmd{1} 卷}
图$ (a) $为某同学改装和校准毫安表的电路图,其中虚线框内是毫安表的改装电路。
\begin{figure}[h!]
\centering
\includesvg[width=0.63\linewidth]{picture/svg/663}
\end{figure}

\begin{enumerate}
\renewcommand{\labelenumii}{(\arabic{enumii})}
\item 
已知毫安表表头的内阻为$ 100 \ \Omega $,满偏电流为$ 1 \ mA $;$ R_{1} $和$ R_{2} $为阻值固定的电阻。若使用$ a $和$ b $两个接线柱,电表量程为$ 3 \ mA $;若使用$ a $和$ c $两个接线柱,电表量程为$ 10 \ mA $。由题给条件和数据,可求出$ R_{1} = $ \tk{15} $ \Omega $,$ R_{2} = $ \tk{35} $ \Omega $。


\item 
现用—量程为$ 3 \ mA $、内阻为$ 150 \ \Omega $的标准电流表$ A $对改装电表的$ 3 \ mA $挡进行校准,校准时需选取的刻度为$ 0.5 $、$ 1.0 $、$ 1.5 $、$ 2.0 $、$ 2.5 $、$ 3.0 \ mA $。电池的电动势为$ 1.5V $,内阻忽略不计;定值电阻$ R_{0} $有两种规格,阻值分别为$ 300 \ \Omega $和$ 1000 \ \Omega $;滑动变阻器$ R $有两种规格,最大阻值分别为$ 750 \ \Omega $和$ 3000 \ \Omega $。则$ R_{0} $应选用阻值为 \tk{300} $ \Omega $的电阻,$ R $应选用最大阻值为 \tk{3000} $ \Omega $的滑动变阻器。


\item 
若电阻$ R_{1} $和$ R_{2} $中有一个因损坏而阻值变为无穷大,利用图($ b) $的电路可以判断出损坏的电阻。图($ b) $中的$ R ^{\prime} $ 为保护电阻,虚线框内未画出的电路即为图$ (a) $ 虚线框内的电路。则图中的$ d $点应和接线柱 \tk{C} (填”$ b $”或”$ c $”)相连。判断依据是: \tk{闭合开关时,若电表指针偏转,则损坏的电阻是$ R_{1} $;若电表指针不动,则损坏的电阻是$ R_{2} $} 。

\end{enumerate}








\end{enumerate}

