\bta{牛顿运动定律及其应用}
\begin{enumerate}[leftmargin=0em]
\renewcommand{\labelenumi}{\arabic{enumi}.}
% A(\Alph) a(\alph) I(\Roman) i(\roman) 1(\arabic)
%设定全局标号series=example	%引用全局变量resume=example
%[topsep=-0.3em,parsep=-0.3em,itemsep=-0.3em,partopsep=-0.3em]
%可使用leftmargin调整列表环境左边的空白长度 [leftmargin=0em]
\item
\exwhere{$ 2014 $年物理上海卷}
牛顿第一定律表明,力是物体 \tk{运动状态} 发生变化的原因;该定律引出的一个重要概念是 \tk{惯性} 。


\item 
\exwhere{$ 2012 $年物理江苏卷}
将一只皮球竖直向上抛出,皮球运动时受到空气阻力的大小与速度的大小成正比. 下列描绘皮球在上升过程中加速度大小$ a $与时间$ t $ 关系的图象,可能正确的是 \xzanswer{C} 
\begin{figure}[h!]
\centering
\includesvg[width=0.83\linewidth]{picture/svg/462}
\end{figure}


\item 
\exwhere{$ 2012 $年理综安徽卷}
如图所示,放在固定斜面上的物块以加速度$ a $沿斜面匀加速下滑,若在物块上再施加一竖直向下的恒力$ F $,则 \xzanswer{C} 
\begin{figure}[h!]
\centering
\includesvg[width=0.23\linewidth]{picture/svg/463}
\end{figure}


\fourchoices
{物块可能匀速下滑}
{物块仍以加速度$ a $匀速下滑}
{物块将以大于的加速度$ a $匀加速下滑}
{物块将以小于的加速度$ a $匀加速下滑}

\item 
\exwhere{$ 2013 $年新课标 \lmd{2} 卷}
一物块静止在粗糙的水平桌面上。从某时刻开始,物块受到一方向不变的水平拉力作用。假设物块与桌面间的最大静摩擦力等于滑动摩擦力。以$ a $表示物块的加速度大小,$ F $表示水平拉力的大小。能正确描述$ F $与$ a $之间的关系的图像是 \xzanswer{C} 
\begin{figure}[h!]
\centering
\includesvg[width=0.83\linewidth]{picture/svg/464}
\end{figure}

\item 
\exwhere{$ 2013 $年新课标 \lmd{2} 卷}
如图,在固定斜面上的一物块受到一外力$ F $的作用,$ F $平行于斜面上。若要物块在斜面上保持静止,$ F $的取值应有一定范围,已知其最大值和最小值分别为$ F_{1} $和$ F_{2} $($ F_{2} $>$ 0 $)。由此可求出 \xzanswer{C} 
\begin{figure}[h!]
\centering
\includesvg[width=0.18\linewidth]{picture/svg/465}
\end{figure}


\fourchoices
{物块的质量 }
{斜面的倾角}
{物块与斜面间的最大静摩擦力}
{物块对斜面的正压力}


\item 
\exwhere{$ 2013 $年安徽卷}
如图所示,细线的一端系一质量为$ m $的小球,另一端固定在倾角为$ \theta $的光滑斜面体顶端,细线与斜面平行。在斜面体以加速度$ a $水平向右做匀加速直线运动的过程中,小球始终静止在斜面上,小球受到细线的拉力$ T $和斜面的支持力为$ F_n $分别为(重力加速度为$ g $) \xzanswer{A} 
\begin{figure}[h!]
\centering
\includesvg[width=0.23\linewidth]{picture/svg/466}
\end{figure}


\fourchoices
{$ T=m(g \sin \theta + a \cos \theta ) F_n= m(g \cos \theta - $ $ a \sin \theta ) $}
{$ T=m(g \sin \theta + a \cos \theta ) F_n= m(g \sin \theta - $ $ a \cos \theta ) $}
{$ T=m(a \cos \theta - g \sin \theta ) F_n= m(g \cos \theta + $ $ a \sin \theta ) $}
{$ T=m(a \sin \theta - g \cos \theta ) F_n= m(g \sin \theta + $ $ a \cos \theta ) $}

\item 
\exwhere{$ 2011 $年理综北京卷}
“蹦极””就是跳跃者把一端固定的长弹性绳绑在踝关节处,从几十米高处跳下的一种极限运动。某人做蹦极运动,所受绳子拉力$ F $的大小随时间$ t $变化的情况如图所示。将蹦极过程近似为在竖直方向的运动,重力加速度为$ g $。据图可知,此人在蹦极过程中最大加速度约为 \xzanswer{B} 
\begin{figure}[h!]
\centering
\includesvg[width=0.43\linewidth]{picture/svg/467}
\end{figure}


\fourchoices
{$ g $ }
{$ 2g $}
{$ 3g $ }
{$ 4g $}

\item 
\exwhere{$ 2012 $年物理上海卷}
如图,将质量$ m=0.1 \ kg $的圆环套在固定的水平直杆上。环的直径略大于杆的截面直径。环与杆间动摩擦因数$ \mu =0.8 $。对环施加一位于竖直平面内斜向上,与杆夹角$ \theta =53 ^{ \circ } $的拉力$ F $,使圆环以$ a=4.4 \ m/s ^{2} $的加速度沿杆运动,求$ F $的大小。
\begin{figure}[h!]
\flushright 
\includesvg[width=0.2\linewidth]{picture/svg/470}
\end{figure}

\banswer{
$ F=9\ N $
}



\newpage


\item 
\exwhere{$ 2018 $年全国 \lmd{1} 卷}
如图,轻弹簧的下端固定在水平桌面上,上端放有物块$ P $,系统处于静止状态。现用一竖直向上的力$ F $作用在$ P $上,使其向上做匀加速直线运动。以$ x $表示$ P $离开静止位置的位移,在弹簧恢复原长前,下列表示$ F $和$ x $之间关系的图像可能正确的是 \xzanswer{A} 
\begin{figure}[h!]
\centering
\includesvg[width=0.14\linewidth]{picture/svg/468} \quad 
\includesvg[width=0.8\linewidth]{picture/svg/469}
\end{figure}



\item 
\exwhere{$ 2014 $年理综重庆卷}
以不同初速度将两个物体同时竖直向上抛出并开始计时,一个物体所受空气阻力可忽略,另一个物体所受空气阻力大小与物体速率成正比。下列用虚线和实线描述两物体运动的$ v-t $图象可能正确的是 \xzanswer{D} 
\begin{figure}[h!]
\centering
\includesvg[width=0.83\linewidth]{picture/svg/471}
\end{figure}









\item 
\exwhere{$ 2015 $年海南卷}
如图,物块$ a $、$ b $和$ c $的质量相同,$ a $和$ b $、$ b $和$ c $之间用完全相同的轻弹簧$ S_{1} $和$ S_{2} $相连,通过系在$ a $上的细线悬挂于固定点$ O $。整个系统处于静止状态。现将细绳剪断,将物块$ a $的加速度记为$ a_{1} $,$ S_{1} $和$ S_{2} $相对原长的伸长分别为$ \triangle l_1 $和$ \triangle l_2 $,重力加速度大小为$ g $,在剪断的瞬间 $ ( \ \answer{ 	AC } \ ) $


\begin{minipage}[h!]{0.7\linewidth}
\vspace{0.3em}
\fourchoices
{$ a_1=3g $ }
{$ a_1=0 $}
{$ \triangle l_1=2 \triangle l_2 $ }
{$ \triangle l_1= \triangle l_2 $}
\vspace{0.3em}
\end{minipage}
\hfill
\begin{minipage}[h!]{0.3\linewidth}
\flushright
\vspace{0.3em}
\includesvg[width=0.25\linewidth]{picture/svg/472}
\vspace{0.3em}
\end{minipage}

\item 
\exwhere{$ 2012 $年理综安徽卷}
质量为$ 0.1 $ $ kg $ 的弹性球从空中某高度由静止开始下落,该下落过程对应的$ v-t $图象如图所示。球与水平地面相碰后离开地面时的速度大小为碰撞前的$ 3/4 $。该球受到的空气阻力大小恒为$ f $,取$ =10 $ $ m/s^{2} $, 求:
\begin{enumerate}
\renewcommand{\labelenumi}{\arabic{enumi}.}
% A(\Alph) a(\alph) I(\Roman) i(\roman) 1(\arabic)
%设定全局标号series=example	%引用全局变量resume=example
%[topsep=-0.3em,parsep=-0.3em,itemsep=-0.3em,partopsep=-0.3em]
%可使用leftmargin调整列表环境左边的空白长度 [leftmargin=0em]
\item
弹性球受到的空气阻力$ f $的大小;

\item 
弹性球第一次碰撞后反弹的高度$ h $。

\end{enumerate}
\begin{figure}[h!]
\flushright
\includesvg[width=0.25\linewidth]{picture/svg/473}
\end{figure}

\banswer{
\begin{enumerate}
\renewcommand{\labelenumi}{\arabic{enumi}.}
% A(\Alph) a(\alph) I(\Roman) i(\roman) 1(\arabic)
%设定全局标号series=example	%引用全局变量resume=example
%[topsep=-0.3em,parsep=-0.3em,itemsep=-0.3em,partopsep=-0.3em]
%可使用leftmargin调整列表环境左边的空白长度 [leftmargin=0em]
\item
$ 0.2\ N $
\item 
$ 0.375\ m $


\end{enumerate}


}


\item 
\exwhere{$ 2011 $年新课标版}	
如图,在光滑水平面上有一质量为$ m_{1} $的足够长的木板,其上叠放一质量为$ m_{2} $的木块。假定木块和木板之间的最大静摩擦力和滑动摩擦力相等。现给木块施加一随时间$ t $增大的水平力$ F=kt $($ k $是常数),木板和木块加速度的大小分别为$ a_{1} $和$ a_{2} $,下列反映$ a_{1} $和$ a_{2} $变化的图线中正确的是 \xzanswer{A} 
\begin{figure}[h!]
\centering
\includesvg[width=0.3\linewidth]{picture/svg/475}\\
\includesvg[width=0.83\linewidth]{picture/svg/474}
\end{figure}





\item 
\exwhere{$ 2016 $年新课标 \lmd{1} 卷}
一质点做匀速直线运动。现对其施加一恒力,且原来作用在质点上的力不发生改变,则 \xzanswer{BC} 


\fourchoices
{质点速度的方向总是与该恒力的方向相同}
{质点速度的方向不可能总是与该恒力的方向垂直}
{质点加速度的方向总是与该恒力的方向相同}
{质点单位时间内速率的变化量总是不变}


\item 
\exwhere{$ 2016 $年新课标 \lmd{2} 卷}
两实心小球甲和乙由同一种材质制成,甲球质量大于乙球质量。两球在空气中由静止下落,假设它们运动时受到的阻力与球的半径成正比,与球的速率无关。若它们下落相同的距离,则 \xzanswer{BD} 


\fourchoices
{甲球用的时间比乙球长}
{甲球末速度的大小大于乙球末速度的大小}
{甲球加速度的大小小于乙球加速度的大小}
{甲球克服阻力做的功大于乙球克服阻力做的功}




\item 
\exwhere{$ 2011 $年上海卷}
受水平外力$ F $作用的物体,在粗糙水平面上作直线运动,其$ v-t $ 图线如图所示,则 \xzanswer{CD} 


\begin{minipage}[h!]{0.7\linewidth}
\vspace{0.3em}
\fourchoices
{在$ 0 \sim t_1 $秒内,外力$ F $大小不断增大}
{在$ t_{1} $时刻,外力$ F $为零}
{在$ t_1 \sim t_2 $秒内,外力$ F $大小可能不断减小}
{在$ t_1 \sim t_2 $秒内,外力$ F $大小可能先减小后增大}

\vspace{0.3em}
\end{minipage}
\hfill
\begin{minipage}[h!]{0.3\linewidth}
\flushright
\vspace{0.3em}
\includesvg[width=0.7\linewidth]{picture/svg/476}
\vspace{0.3em}
\end{minipage}


\newpage
\item 
\exwhere{$ 2014 $年物理上海卷}
如图,水平地面上的矩形箱子内有一倾角为$ \theta $的固定斜面,斜面上放一质量为$ m $的光滑球。静止时,箱子顶部与球接触但无压力。箱子由静止开始向右做匀加速运动,然后改做加速度大小为$ a $的匀减速运动直至静止,经过的总路程为$ s $,运动过程中的最大速度为$ v $。
\begin{enumerate}
\renewcommand{\labelenumii}{(\arabic{enumii})}

\item 
求箱子加速阶段的加速度大小$ a ^{\prime} $。

\item 
若$ a>g \tan \theta $,求减速阶段球受到箱子左壁和顶部的作用力。

\end{enumerate}
\begin{figure}[h!]
\flushright
\includesvg[width=0.25\linewidth]{picture/svg/477}
\end{figure}

\banswer{
\begin{enumerate}
\renewcommand{\labelenumi}{\arabic{enumi}.}
% A(\Alph) a(\alph) I(\Roman) i(\roman) 1(\arabic)
%设定全局标号series=example	%引用全局变量resume=example
%[topsep=-0.3em,parsep=-0.3em,itemsep=-0.3em,partopsep=-0.3em]
%可使用leftmargin调整列表环境左边的空白长度 [leftmargin=0em]
\item
$\frac { a v ^ { 2 } } { 2 a s - v ^ { 2 } }$
\item 
$0 \qquad m ( a \cot \theta - g )$


\end{enumerate}


}



\item 
\exwhere{$ 2011 $年理综福建卷}
如图,一不可伸长的轻质细绳跨过滑轮后,两端分别悬挂质量为$ m_{1} $和$ m_{2} $的物体$ A $和$ B $。若滑轮有一定大小,质量为$ m $且分布均匀,滑轮转动时与绳之间无相对滑动,不计滑轮与轴之间的摩擦。设细绳对$ A $和$ B $的拉力大小分别为$ T_{1} $和$ T_{2} $。已知下列四个关于$ T_{1} $的表达式中有一个是正确的,请你根据所学的物理知识,通过一定的分析,判断正确的表达式是 \xzanswer{C} 


\begin{minipage}[h!]{0.7\linewidth}
\vspace{0.3em}
\fourchoices
{$ T _ { 1 } = \frac { \left( m + 2 m _ { 2 } \right) m _ { 1 } g } { m + 2 \left( m _ { 1 } + m _ { 2 } \right) } $}
{$ T _ { 1 } = \frac { \left( m + 2 m _ { 1 } \right) m _ { 2 } g } { m + 4 \left( m _ { 1 } + m _ { 2 } \right) } $}
{$ T _ { 1 } = \frac { \left( m + 4 m _ { 2 } \right) m _ { 1 } g } { m + 2 \left( m _ { 1 } + m _ { 2 } \right) } $}
{$ T _ { 1 } = \frac { \left( m + 4 m _ { 1 } \right) m _ { 2 } g } { m + 4 \left( m _ { 1 } + m _ { 2 } \right) } $}
\vspace{0.3em}
\end{minipage}
\hfill
\begin{minipage}[h!]{0.3\linewidth}
\flushright
\vspace{0.3em}
\includesvg[width=0.4\linewidth]{picture/svg/478}
\vspace{0.3em}
\end{minipage}



\item 
\exwhere{$ 2019 $年$ 4 $月浙江物理选考}
如图所示,小明撑杆使船离岸,则下列说法正确的是
\begin{figure}[h!]
\centering
\includesvg[width=0.23\linewidth]{picture/svg/479}
\end{figure}


\fourchoices
{小明与船之间存在摩擦力}
{杆的弯曲是由于受到杆对小明的力}
{杆对岸的力大于岸对杆的力}
{小明对杆的力和岸对杆的力是一对相互作用力}




\item 
\exwhere{$ 2019 $年物理全国\lmd{2}卷}
物块在轻绳的拉动下沿倾角为$ 30 ^{ \circ } $的固定斜面向上匀速运动,轻绳与斜面平行。已知物块与斜面之间的动摩擦因数为$ \frac{\sqrt{3}}{3} $,重力加速度取$ 10 \ m/s ^{2} $。若轻绳能承受的最大张力为$ 1500 $ $ N $,则物块的质量最大为 \xzanswer{A} 


\fourchoices
{$ 150 \ kg $}
{$ 100\sqrt{3} \ kg $	}
{$ 200 $ $ kg $}
{$ 200\sqrt{3} \ kg $}



\item 
\exwhere{$ 2017 $年海南卷}
一轻弹簧的一端固定在倾角为$ \theta $的固定光滑斜面的底部,另一端和质量为$ m $的小物块$ a $相连,如图所示。质量为$ \frac{ 3 }{ 5 } m $的小物块$ b $紧靠$ a $静止在斜面上,此时弹簧的压缩量为$ x_{0} $,从$ t=0 $时开始,对$ b $施加沿斜面向上的外力,使$ b $始终做匀加速直线运动。经过一段时间后,物块$ a $、$ b $分离;再经过同样长的时间,$ b $距其出发点的距离恰好也为$ x_{0} $。弹簧的形变始终在弹性限度内,重力加速度大小为$ g $。求:
\begin{enumerate}
\renewcommand{\labelenumi}{\arabic{enumi}.}
% A(\Alph) a(\alph) I(\Roman) i(\roman) 1(\arabic)
%设定全局标号series=example	%引用全局变量resume=example
%[topsep=-0.3em,parsep=-0.3em,itemsep=-0.3em,partopsep=-0.3em]
%可使用leftmargin调整列表环境左边的空白长度 [leftmargin=0em]
\item
弹簧的劲度系数;
\item 
物块$ b $加速度的大小;
\item 
在物块$ a $、$ b $分离前,外力大小随时间变化的关系式。


\end{enumerate}
\begin{figure}[h!]
\flushright
\includesvg[width=0.25\linewidth]{picture/svg/480}
\end{figure}


\banswer{
\begin{enumerate}
\renewcommand{\labelenumi}{\arabic{enumi}.}
% A(\Alph) a(\alph) I(\Roman) i(\roman) 1(\arabic)
%设定全局标号series=example	%引用全局变量resume=example
%[topsep=-0.3em,parsep=-0.3em,itemsep=-0.3em,partopsep=-0.3em]
%可使用leftmargin调整列表环境左边的空白长度 [leftmargin=0em]
\item
$k = \frac { F _ { 0 } } { x _ { 0 } } = \frac { 8 m g \sin \theta } { 5 x _ { 0 } }$
\item 
$a = \frac { 1 } { 5 } g \sin \theta$
\item 
$F = \frac { 8 } { 25 } m g \sin \theta + k x = \frac { 8 } { 25 } m g \sin \theta + \frac { 1 } { 2 } k a t ^ { 2 } = \frac { 4 } { 25 } m g \sin \theta \left( 2 + \frac { g \sin \theta } { x _ { 0 } } \cdot t ^ { 2 } \right)$


\end{enumerate}


}








\end{enumerate}



