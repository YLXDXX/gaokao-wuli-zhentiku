\bta{碰撞}


\begin{enumerate}[leftmargin=0em]
\renewcommand{\labelenumi}{\arabic{enumi}.}
% A(\Alph) a(\alph) I(\Roman) i(\roman) 1(\arabic)
%设定全局标号series=example	%引用全局变量resume=example
%[topsep=-0.3em,parsep=-0.3em,itemsep=-0.3em,partopsep=-0.3em]
%可使用leftmargin调整列表环境左边的空白长度 [leftmargin=0em]
\item
\exwhere{$ 2015 $年理综天津卷}
如图所示,在光滑水平面的左侧固定一竖直挡板,$ A $球在水平面上静止放置,$ B $球向左运动与$ A $球发生正碰,$ B $球碰撞前、后的速率之比为$ 3 $∶$ 1 $,$ A $球垂直撞向挡板,碰后原速率返回,两球刚好不发生碰撞,$ A $、$ B $两球的质量之比为\tk{$ 4:1 $},$ A $、$ B $碰撞前、后两球总动能之比为\tk{$ 9:5 $}.


\item
\exwhere{$ 2013 $年江苏卷}
水平面上,一白球与一静止的灰球碰撞,两球质量相等. 碰撞过程的频闪照片如图所示,据此可推断,碰撞过程中系统损失的动能约占碰撞前动能的 \xzanswer{A} 
\begin{figure}[h!]
\centering
\includesvg[width=0.16\linewidth]{picture/svg/524}
\end{figure}



 
\fourchoices
{$ 30 \% $}
{$ 50 \% $}
{$ 70 \% $}
{$ 90 \% $}




\item
\exwhere{$ 2012 $年理综全国卷}
如图,大小相同的摆球$ a $和$ b $的质量分别为$ m $和$ 3 \ m $,摆长相同,并排悬挂,平衡时两球刚好接触,现将摆球$ a $向左边拉开一小角度后释放,若两球的碰撞是弹性的,下列判断正确的是 \xzanswer{AD} 


\begin{minipage}[h!]{0.7\linewidth}
\vspace{0.3em}
\fourchoices
{第一次碰撞后的瞬间,两球的速度大小相等}
{第一次碰撞后的瞬间,两球的动量大小相等}
{第一次碰撞后,两球的最大摆角不相同}
{发生第二次碰撞时,两球在各自的平衡位置}
\vspace{0.3em}
\end{minipage}
\hfill
\begin{minipage}[h!]{0.3\linewidth}
\flushright
\vspace{0.3em}
\includesvg[width=0.2\linewidth]{picture/svg/525}
\vspace{0.3em}
\end{minipage}




\item 
\exwhere{$ 2011 $年理综全国卷}
质量为$ M $、内壁间距为$ L $的箱子静止于光滑的水平面上,箱子中间有一质量为$ m $的小物块,小物块与箱子底板间的动摩擦因数为$ \mu $。初始时小物块停在箱子正中间,如图所示。现给小物块一水平向右的初速度$ v $,小物块与箱壁碰撞$ N $次后恰又回到箱子正中间,井与箱子保持相对静止。设碰撞都是弹性的,则整个过程中,系统损失的动能为 \xzanswer{BD} 
\begin{figure}[h!]
\centering
\includesvg[width=0.2\linewidth]{picture/svg/526}
\end{figure}

\fourchoices
{$ \frac { 1 } { 2 } m v ^ { 2 } $}
{$ \frac { 1 } { m + M } v ^ { 2 } $}
{$ \frac { 1 } { 2 } N \mu m g L $}
{$ N \mu m g L $}


\item
\exwhere{$ 2014 $年理综大纲卷}
一中子与一质量数为$ A $($ A $>$ 1 $)的原子核发生弹性正碰。若碰前原子核静止,则碰撞前与碰撞后中子的速率之比为 \xzanswer{A} 

\fourchoices
{$ \frac { A + 1 } { A - 1 } $}
{$ \frac { A - 1 } { A + 1 } $}
{$ \frac { 4 A } { ( A + 1 ) ^ { 2 } } $}
{$ \frac { ( A + 1 ) ^ { 2 } } { ( A - 1 ) ^ { 2 } } $}



\item 
\exwhere{$ 2014 $年理综大纲卷}
冰球运动员甲的质量为$ 80.0 \ kg $。当他以$ 5.0 \ m/s $的速度向前运动时,与另一质量为$ 100 \ kg $、速度为$ 3.0 \ m/s $的迎面而来的运动员乙相撞。碰后甲恰好静止。假设碰撞时间极短,求:
\begin{enumerate}
\renewcommand{\labelenumi}{\arabic{enumi}.}
% A(\Alph) a(\alph) I(\Roman) i(\roman) 1(\arabic)
%设定全局标号series=example	%引用全局变量resume=example
%[topsep=-0.3em,parsep=-0.3em,itemsep=-0.3em,partopsep=-0.3em]
%可使用leftmargin调整列表环境左边的空白长度 [leftmargin=0em]
\item
碰后乙的速度的大小;
\item 
碰撞中总机械能的损失。

\end{enumerate}


\banswer{
\begin{enumerate}
\renewcommand{\labelenumi}{\arabic{enumi}.}
% A(\Alph) a(\alph) I(\Roman) i(\roman) 1(\arabic)
%设定全局标号series=example	%引用全局变量resume=example
%[topsep=-0.3em,parsep=-0.3em,itemsep=-0.3em,partopsep=-0.3em]
%可使用leftmargin调整列表环境左边的空白长度 [leftmargin=0em]
\item
$ 1.0\ m/s $
\item 
$ 1400\ J $

\end{enumerate}


}





\item 
\exwhere{$ 2014 $年理综广东卷}
下图的水平轨道中,$ AC $段的中点$ B $的正上方有一探测器,$ C $处有一竖直挡板,物体$ P_{1} $沿轨道向右以速度$ v_{1} $与静止在$ A $点的物体$ P_{2} $碰撞,并接合成复合体$ P $,以此碰撞时刻为计时零点,探测器只在$ t_{1} =2\ s $至$ t_{2} =4\ s $内工作,已知$ P_{1} $、$ P_{2} $的质量都为$ m=1 \ kg $,$ P $与$ AC $间的动摩擦因数为$ \mu =0.1 $,$ AB $段长$ L=4m $,$ g $取$ 10 \ m/s^{2} $,$ P_{1} $、$ P_{2} $和$ P $均视为质点,$ P $与挡板的碰撞为弹性碰撞.
\begin{enumerate}
\renewcommand{\labelenumi}{\arabic{enumi}.}
% A(\Alph) a(\alph) I(\Roman) i(\roman) 1(\arabic)
%设定全局标号series=example	%引用全局变量resume=example
%[topsep=-0.3em,parsep=-0.3em,itemsep=-0.3em,partopsep=-0.3em]
%可使用leftmargin调整列表环境左边的空白长度 [leftmargin=0em]
\item
若$ v_{1} =6 \ m/s $,求$ P_{1} $、$ P_{2} $碰后瞬间的速度大小$ v $和碰撞损失的动能$ \Delta E $;
\item 
若$ P $与挡板碰后,能在探测器的工作时间内通过$ B $点,求$ v_{1} $的取值范围和$ P $向左经过$ A $点时的最大动能$ E $.

\end{enumerate}
\begin{figure}[h!]
\flushright
\includesvg[width=0.4\linewidth]{picture/svg/527}
\end{figure}


\banswer{
\begin{enumerate}
\renewcommand{\labelenumi}{\arabic{enumi}.}
% A(\Alph) a(\alph) I(\Roman) i(\roman) 1(\arabic)
%设定全局标号series=example	%引用全局变量resume=example
%[topsep=-0.3em,parsep=-0.3em,itemsep=-0.3em,partopsep=-0.3em]
%可使用leftmargin调整列表环境左边的空白长度 [leftmargin=0em]
\item
$\Delta E=9\ J$
\item 
$ E_{max}=17\ J $

\end{enumerate}


}



\newpage	
\item 
\exwhere{$ 2018 $年海南卷}
如图,光滑轨道$ PQO $的水平段$ QO=\frac{h}{2} $,轨道在点与水平地面平滑连接。一质量为的小物块$ A $从高处由静止开始沿轨道下滑,在点与质量为的静止小物块$ B $发生碰撞。$ A $、$ B $与地面间的动摩擦因数均为$ \mu=0.5 $,重力加速度大小为$ g $。假设$ A $、$ B $间的碰撞为完全弹性碰撞,碰撞时间极短。求:
\begin{enumerate}
\renewcommand{\labelenumi}{\arabic{enumi}.}
% A(\Alph) a(\alph) I(\Roman) i(\roman) 1(\arabic)
%设定全局标号series=example	%引用全局变量resume=example
%[topsep=-0.3em,parsep=-0.3em,itemsep=-0.3em,partopsep=-0.3em]
%可使用leftmargin调整列表环境左边的空白长度 [leftmargin=0em]
\item
第一次碰撞后瞬间$ A $和$ B $速度的大小;
\item 
$ A $、$ B $均停止运动后,二者之间的距离。

\end{enumerate}
\begin{figure}[h!]
\flushright
\includesvg[width=0.35\linewidth]{picture/svg/528}
\end{figure}


\banswer{
\begin{enumerate}
\renewcommand{\labelenumi}{\arabic{enumi}.}
% A(\Alph) a(\alph) I(\Roman) i(\roman) 1(\arabic)
%设定全局标号series=example	%引用全局变量resume=example
%[topsep=-0.3em,parsep=-0.3em,itemsep=-0.3em,partopsep=-0.3em]
%可使用leftmargin调整列表环境左边的空白长度 [leftmargin=0em]
\item
$v_{A}=\frac { 3 } { 5 } \sqrt { 2 g h }$ \qquad $v_{B}\frac { 2 } { 5 } \sqrt { 2 g h }$
\item 
$s _ { \mathrm { A } } + s _ { \mathrm { B } } = \frac { 26 } { 125 } h$

\end{enumerate}


}






\item 
\exwhere{$ 2013 $年广东卷}
如图,两块相同平板$ P_{1} $、$ P_{2} $置于光滑水平面上,质量均为$ m $。$ P_{2} $的右端固定一轻质弹簧,左端$ A $与弹簧的自由端$ B $相距$ L $。物体$ P $置于$ P_{1} $的最右端,质量为$ 2 \ m $且可以看作质点。$ P_{1} $与$ P $以共同速度$ v_{0} $向右运动,与静止的$ P_{2} $发生碰撞,碰撞时间极短,碰撞后$ P_{1} $与$ P_{2} $粘连在一起,$ P $压缩弹簧后被弹回并停在$ A $点(弹簧始终在弹性限度内)。$ P $与$ P_{2} $之间的动摩擦因数为$ \mu $,求:
\begin{enumerate}
\renewcommand{\labelenumi}{\arabic{enumi}.}
% A(\Alph) a(\alph) I(\Roman) i(\roman) 1(\arabic)
%设定全局标号series=example	%引用全局变量resume=example
%[topsep=-0.3em,parsep=-0.3em,itemsep=-0.3em,partopsep=-0.3em]
%可使用leftmargin调整列表环境左边的空白长度 [leftmargin=0em]
\item
$ P_{1} $、$ P_{2} $刚碰完时的共同速度$ v_{1} $和$ P $的最终速度$ v_{2} $;
\item 
此过程中弹簧最大压缩量$ x $和相应的弹性势能$ E_p $.


\end{enumerate}
\begin{figure}[h!]
\flushright
\includesvg[width=0.4\linewidth]{picture/svg/529}
\end{figure}

\banswer{
\begin{enumerate}
\renewcommand{\labelenumi}{\arabic{enumi}.}
% A(\Alph) a(\alph) I(\Roman) i(\roman) 1(\arabic)
%设定全局标号series=example	%引用全局变量resume=example
%[topsep=-0.3em,parsep=-0.3em,itemsep=-0.3em,partopsep=-0.3em]
%可使用leftmargin调整列表环境左边的空白长度 [leftmargin=0em]
\item
$v _ { 1 } = \frac { 1 } { 2 } v _ { 0 }$ \qquad $v _ { 2 } = \frac { 3 } { 4 } v _ { 0 }$
\item 
$x = \frac { v _ { 0 } ^ { 2 } } { 32 \mu g } - L$ \qquad $E _ { p } = \frac { m v _ { 0 } ^ { 2 } } { 16 }$

\end{enumerate}


}


\newpage	
\item 
\exwhere{$ 2011 $年理综重庆卷}
如图所示,静置于水平地面的三辆手推车沿一直线排列,质量均为$ m $,人在极短时间内给第一辆车一水平冲量使其运动,当车运动了距离$ L $时与第二辆车相碰,两车以共同速度继续运动了距离$ L $时与第三车相碰,三车以共同速度又运动了距离$ L $时停止。车运动时受到的摩擦阻力恒为车所受重力的$ k $倍,重力加速度为$ g $,若车与车之间仅在碰撞时发生相互作用,碰撞时间很短,忽略空气阻力,求:
\begin{enumerate}
\renewcommand{\labelenumi}{\arabic{enumi}.}
% A(\Alph) a(\alph) I(\Roman) i(\roman) 1(\arabic)
%设定全局标号series=example	%引用全局变量resume=example
%[topsep=-0.3em,parsep=-0.3em,itemsep=-0.3em,partopsep=-0.3em]
%可使用leftmargin调整列表环境左边的空白长度 [leftmargin=0em]
\item
整个过程中摩擦阻力所做的总功;
\item 
人给第一辆车水平冲量的大小;
\item 
第一次与第二次碰撞系统功能损失之比。
\end{enumerate}
\begin{figure}[h!]
\flushright
\includesvg[width=0.4\linewidth]{picture/svg/530}
\end{figure}


\banswer{
\begin{enumerate}
\renewcommand{\labelenumi}{\arabic{enumi}.}
% A(\Alph) a(\alph) I(\Roman) i(\roman) 1(\arabic)
%设定全局标号series=example	%引用全局变量resume=example
%[topsep=-0.3em,parsep=-0.3em,itemsep=-0.3em,partopsep=-0.3em]
%可使用leftmargin调整列表环境左边的空白长度 [leftmargin=0em]
\item
$W = - k m g L - 2 k m g L - 3 k m g L = - 6 k m g L$
\item 
$I = m u _ { 2 } = 2 m \sqrt { 7 k g L }$
\item 
$\frac { \Delta E _ { k 1 } } { \Delta E _ { k 2 } } = \frac { 13 } { 3 }$

\end{enumerate}


}



\item 
\exwhere{$ 2015 $年广东卷}
如图所示,一条带有圆轨道的长轨道水平固定,圆轨道竖直,底端分别与两侧的直轨道相切,半径$ R= 0.5\ m $,物块$ A $以$ V_{0} $=$ 6 \ m/s $的速度滑入圆轨道,滑过最高点$ Q $,再沿圆轨道滑出后,与直轨上$ P $处静止的物块$ B $碰撞,碰后粘在一起运动,$ P $点左侧轨道光滑,右侧轨道呈粗糙段、光滑段交替排列,每段长度都为$ L $=$ 0.1\ m $,物块与各粗糙段间的动摩擦因数都为$ \mu =0.1 $,$ A $、$ B $的质量均为$ m =1 \ kg $(重力加速度$ g $取$ 10 \ m/s^{2} $;$ A $、$ B $视为质点,碰撞时间极短)。
\begin{enumerate}
\renewcommand{\labelenumii}{(\arabic{enumii})}
\item 

求$ A $滑过$ Q $点时的速度大小$ V $和受到的弹力大小$ F $;


\item 
若碰后$ AB $最终停止在第$ k $个粗糙段上,求$ k $的数值;


\item 
求碰后$ AB $滑至第$ n $个$ (n < k) $光滑段上的速度$ V_n $与$ n $的关系式。


\end{enumerate}
\begin{figure}[h!]
\flushright
\includesvg[width=0.5\linewidth]{picture/svg/531}
\end{figure}

\banswer{
\begin{enumerate}
\renewcommand{\labelenumi}{\arabic{enumi}.}
% A(\Alph) a(\alph) I(\Roman) i(\roman) 1(\arabic)
%设定全局标号series=example	%引用全局变量resume=example
%[topsep=-0.3em,parsep=-0.3em,itemsep=-0.3em,partopsep=-0.3em]
%可使用leftmargin调整列表环境左边的空白长度 [leftmargin=0em]
\item
$ F=22\ N $
\item 
$ k=45 $
\item 
$V _ { n } = \sqrt { 9 - 0.2 n }\ \mathrm { m } / \mathrm { s } , \quad ( n < k )$
\end{enumerate}


}








\end{enumerate}


