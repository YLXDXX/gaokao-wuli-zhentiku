\bta{选修模块3—3(上)}


\begin{enumerate}[leftmargin=0em]
\renewcommand{\labelenumi}{\arabic{enumi}.}
% A(\Alph) a(\alph) I(\Roman) i(\roman) 1(\arabic)
%设定全局标号series=example	%引用全局变量resume=example
%[topsep=-0.3em,parsep=-0.3em,itemsep=-0.3em,partopsep=-0.3em]
%可使用leftmargin调整列表环境左边的空白长度 [leftmargin=0em]
\item
\exwhere{$ 2019 $年物理全国\lmd{3}卷}
\begin{enumerate}
\renewcommand{\labelenumi}{\arabic{enumi}.}
% A(\Alph) a(\alph) I(\Roman) i(\roman) 1(\arabic)
%设定全局标号series=example	%引用全局变量resume=example
%[topsep=-0.3em,parsep=-0.3em,itemsep=-0.3em,partopsep=-0.3em]
%可使用leftmargin调整列表环境左边的空白长度 [leftmargin=0em]
\item
用油膜法估算分子大小的实验中,首先需将纯油酸稀释成一定浓度的油酸酒精溶液,稀释的目的是\tk{使油酸在浅盘的水面上容易形成一块单分子层油膜}. 实验中为了测量出一滴已知浓度的油酸酒精溶液中纯油酸的体积,可以\tk{把油酸酒精溶液一滴一滴地滴入小量筒中,测出$ 1mL $油酸酒精溶液的滴数,得到一滴溶液中纯油酸的体积}。为得到油酸分子的直径,还需测量的物理量是\tk{油膜稳定后的表面积$ S $}. 
\item 
如图,一粗细均匀的细管开口向上竖直放置,管内有一段高度为$ 2.0 \ cm $的水银柱,水银柱下密封了一定量的理想气体,水银柱上表面到管口的距离为$ 2.0 \ cm $。若将细管倒置,水银柱下表面恰好位于管口处,且无水银滴落,管内气体温度与环境温度相同。已知大气压强为$ 76 \ cmHg $,环境温度为$ 296\ K $。
\begin{enumerate}
\renewcommand{\labelenumi}{\arabic{enumi}.}
% A(\Alph) a(\alph) I(\Roman) i(\roman) 1(\arabic)
%设定全局标号series=example	%引用全局变量resume=example
%[topsep=-0.3em,parsep=-0.3em,itemsep=-0.3em,partopsep=-0.3em]
%可使用leftmargin调整列表环境左边的空白长度 [leftmargin=0em]
\item
求细管的长度;
\item 
若在倒置前,缓慢加热管内被密封的气体,直到水银柱的上表面恰好与管口平齐为止,求此时密封气体的温度。


\end{enumerate}
\begin{figure}[h!]
\flushright
\includesvg[width=0.18\linewidth]{picture/svg/278}
\end{figure}



\banswer{
\begin{enumerate}
\renewcommand{\labelenumi}{\arabic{enumi}.}
% A(\Alph) a(\alph) I(\Roman) i(\roman) 1(\arabic)
%设定全局标号series=example	%引用全局变量resume=example
%[topsep=-0.3em,parsep=-0.3em,itemsep=-0.3em,partopsep=-0.3em]
%可使用leftmargin调整列表环境左边的空白长度 [leftmargin=0em]
\item
$ 41 \ cm $;
\item 
$ 312K $



\end{enumerate}


}





\end{enumerate}



\newpage
\item 
\exwhere{$ 2019 $年物理全国\lmd{1}卷}
\begin{enumerate}
\renewcommand{\labelenumi}{\arabic{enumi}.}
% A(\Alph) a(\alph) I(\Roman) i(\roman) 1(\arabic)
%设定全局标号series=example	%引用全局变量resume=example
%[topsep=-0.3em,parsep=-0.3em,itemsep=-0.3em,partopsep=-0.3em]
%可使用leftmargin调整列表环境左边的空白长度 [leftmargin=0em]
\item
某容器中的空气被光滑活塞封住,容器和活塞绝热性能良好,空气可视为理想气体。初始时容器中空气的温度与外界相同,压强大于外界。现使活塞缓慢移动,直至容器中的空气压强与外界相同。此时,容器中空气的温度\tk{低于}(填“高于”“低于”或“等于”)外界温度,容器中空气的密度\tk{大于}(填“大于”“小于”或“等于”)外界空气的密度。



\item 
热等静压设备广泛用于材料加工中。该设备工作时,先在室温下把惰性气体用压缩机压入到一个预抽真空的炉腔中,然后炉腔升温,利用高温高气压环境对放入炉腔中的材料加工处理,改变其性能。一台热等静压设备的炉腔中某次放入固体材料后剩余的容积为$ 013 $ $ m^{3} $,炉腔抽真空后,在室温下用压缩机将$ 10 $瓶氩气压入到炉腔中。已知每瓶氩气的容积为$ 3.2 \times 10^{-2} $ $ m^{3} $,使用前瓶中气体压强为$ 1.5 \times 10^7 $ $ Pa $,使用后瓶中剩余气体压强为$ 2.0 \times 10^{6} $ $ Pa $;室温温度为$ 27 $ $ ^{ \circ } C $。氩气可视为理想气体。
\begin{enumerate}
\renewcommand{\labelenumi}{\arabic{enumi}.}
% A(\Alph) a(\alph) I(\Roman) i(\roman) 1(\arabic)
%设定全局标号series=example	%引用全局变量resume=example
%[topsep=-0.3em,parsep=-0.3em,itemsep=-0.3em,partopsep=-0.3em]
%可使用leftmargin调整列表环境左边的空白长度 [leftmargin=0em]
\item
求压入氩气后炉腔中气体在室温下的压强;
\item 
将压入氩气后的炉腔加热到$ 1227 $ $ ^{ \circ } C $,求此时炉腔中气体的压强。



\end{enumerate}





\end{enumerate}


\banswer{
\begin{enumerate}
\renewcommand{\labelenumi}{\arabic{enumi}.}
% A(\Alph) a(\alph) I(\Roman) i(\roman) 1(\arabic)
%设定全局标号series=example	%引用全局变量resume=example
%[topsep=-0.3em,parsep=-0.3em,itemsep=-0.3em,partopsep=-0.3em]
%可使用leftmargin调整列表环境左边的空白长度 [leftmargin=0em]
\item
$3.2 \times 10 ^ { 7 } \mathrm { Pa }$
\item 
$1.6 \times 10 ^ { 8 } \mathrm { Pa }$



\end{enumerate}


}


\item 
\exwhere{$ 2019 $年物理全国\lmd{2}卷}
\begin{enumerate}
\renewcommand{\labelenumi}{\arabic{enumi}.}
% A(\Alph) a(\alph) I(\Roman) i(\roman) 1(\arabic)
%设定全局标号series=example	%引用全局变量resume=example
%[topsep=-0.3em,parsep=-0.3em,itemsep=-0.3em,partopsep=-0.3em]
%可使用leftmargin调整列表环境左边的空白长度 [leftmargin=0em]
\item
如$ P-V $图所示,$ 1 $、$ 2 $、$ 3 $三个点代表某容器中一定量理想气体的三个不同状态,对应的温度分别是$ T_{1} $、$ T_{2} $、$ T_{3} $。用$ N_{1} $、$ N_{2} $、$ N_{3} $分别表示这三个状态下气体分子在单位时间内撞击容器壁上单位面积的次数,则$ N_1 $ \tk{大于} $ N_2 $,$ T_1 $ \tk{等于} $ T_{3} $,$ N_2 $ \tk{大于} $ N_3 $。(填“大于”“小于”或“等于”)
\begin{figure}[h!]
\centering
\includesvg[width=0.23\linewidth]{picture/svg/279}
\end{figure}




\item 
如图,一容器由横截面积分别为$ 2S $和$ S $的两个汽缸连通而成,容器平放在地面上,汽缸内壁光滑。整个容器被通过刚性杆连接的两活塞分隔成三部分,分别充有氢气、空气和氮气。平衡时,氮气的压强和体积分别为$ p_{0} $和$ V_{0} $,氢气的体积为$ 2 V_0 $,空气的压强为$ p $。现缓慢地将中部的空气全部抽出,抽气过程中氢气和氮气的温度保持不变,活塞没有到达两汽缸的连接处,求:
\begin{enumerate}
\renewcommand{\labelenumi}{\arabic{enumi}.}
% A(\Alph) a(\alph) I(\Roman) i(\roman) 1(\arabic)
%设定全局标号series=example	%引用全局变量resume=example
%[topsep=-0.3em,parsep=-0.3em,itemsep=-0.3em,partopsep=-0.3em]
%可使用leftmargin调整列表环境左边的空白长度 [leftmargin=0em]
\item
抽气前氢气的压强;
\item 
抽气后氢气的压强和体积。


\end{enumerate}
\begin{figure}[h!]
\flushright
\includesvg[width=0.4\linewidth]{picture/svg/280}
\end{figure}

\banswer{
\begin{enumerate}
\renewcommand{\labelenumi}{\arabic{enumi}.}
% A(\Alph) a(\alph) I(\Roman) i(\roman) 1(\arabic)
%设定全局标号series=example	%引用全局变量resume=example
%[topsep=-0.3em,parsep=-0.3em,itemsep=-0.3em,partopsep=-0.3em]
%可使用leftmargin调整列表环境左边的空白长度 [leftmargin=0em]
\item
$\frac { 1 } { 2 } \left( p _ { 0 } + p \right)$
\item 
$\frac { 1 } { 2 } p _ { 0 } + \frac { 1 } { 4 } p ; \quad \frac { 4 \left( p _ { 0 } + p \right) V _ { 0 } } { 2 p _ { 0 } + p }$



\end{enumerate}


}




\end{enumerate}



\item 
\exwhere{$ 2018 $年江苏卷}
\begin{enumerate}
\renewcommand{\labelenumi}{\arabic{enumi}.}
% A(\Alph) a(\alph) I(\Roman) i(\roman) 1(\arabic)
%设定全局标号series=example	%引用全局变量resume=example
%[topsep=-0.3em,parsep=-0.3em,itemsep=-0.3em,partopsep=-0.3em]
%可使用leftmargin调整列表环境左边的空白长度 [leftmargin=0em]
\item
如题图所示,一支温度计的玻璃泡外包着纱布,纱布的下端浸在水中。纱布中的水在蒸发时带走热量,使温度计示数低于周围空气温度。当空气温度不变,若一段时间后发现该温度计示数减小,则 \xzanswer{A} 
\begin{figure}[h!]
\centering
\includesvg[width=0.23\linewidth]{picture/svg/281}
\end{figure}

\fourchoices
{空气的相对湿度减小}
{空气中水蒸汽的压强增大}
{空气中水的饱和气压减小}
{空气中水的饱和气压增大}



\item 
一定量的氧气贮存在密封容器中,在$ T_{1} $和$ T_{2} $温度下其分子速率分布的情况见右表。则$ T_{1} $ \tk{大于}(选填“大于”“小于”或“等于”)$ T_{2} $。若约$ 10 \% $的氧气从容器中泄漏,泄漏前后容器内温度均为$ T_{1} $,则在泄漏后的容器中,速率处于$ 400 \sim 500 $ $ \ m/s $区间的氧气分子数占总分子数的百分比 \tk{等于} (选填“大于”“小于”或“等于”)$ 18.6 \% $。
\begin{table}[h!]
\centering 
\begin{tabular}{|c|c|c|}
\hline 
速率区间 & \multicolumn{2}{c|}{各速率区间的分子数占总分子数的百分比/\%}
\\
\hline
$ (m \cdot s ^{-1}) $ & 温度$ T_1 $ & 温度$ T_2 $
\\
\hline
100以下 & 0.7 & 1.4
\\
\hline
100~200 & 5.4 & 8.1
\\
\hline
200~300 & 11.9 & 17.0
\\
\hline
300~400 & 17.4 & 21.4
\\
\hline
400~500 & 18.6 & 20.4
\\
\hline
500~600 & 16.7 & 15.1
\\
\hline
600~700 & 12.9 & 9.2
\\
\hline
700~800 & 7.9 & 4.5
\\
\hline
800~900 & 4.6 & 2.0
\\
\hline
900以上 & 3.9 & 0.9\\ 
\hline 
\end{tabular}
\end{table} 



\item 
如题图所示,一定质量的理想气体在状态$ A $时压强为$ 2.0 \times 10^5 $ $ Pa $,经历$ A \rightarrow B \rightarrow C \rightarrow A $的过程,整个过程中对外界放出$ 61.4 $ $ J $热量。求该气体在$ A \rightarrow B $过程中对外界所做的功。
\begin{figure}[h!]
\flushright
\includesvg[width=0.25\linewidth]{picture/svg/282}
\end{figure}

\banswer{
$ 138.6\ J $
}



\end{enumerate}



\newpage
\item
\exwhere{$ 2018 $年海南卷}
\begin{enumerate}
\renewcommand{\labelenumi}{\arabic{enumi}.}
% A(\Alph) a(\alph) I(\Roman) i(\roman) 1(\arabic)
%设定全局标号series=example	%引用全局变量resume=example
%[topsep=-0.3em,parsep=-0.3em,itemsep=-0.3em,partopsep=-0.3em]
%可使用leftmargin调整列表环境左边的空白长度 [leftmargin=0em]
\item
如图,一定量的理想气体,由状态$ a $等压变化到状态$ b $,再从$ b $等容变化到状态$ c $。$ a $、$ c $两状态温度相等。下列说法正确的是 \xzanswer{BD} 
。(填入正确答案标号。选对$ 1 $个得$ 2 $分,选对$ 2 $个得$ 4 $分;有选错的得$ 0 $分)
\begin{figure}[h!]
\centering
\includesvg[width=0.23\linewidth]{picture/svg/283}
\end{figure}


\fourchoices
{从状态$ b $到状态$ c $的过程中气体吸热}
{气体在状态$ a $的内能等于在状态$ c $的内能}
{气体在状态$ b $的温度小于在状态$ a $的温度}
{从状态$ a $到状态$ b $的过程中气体对外做正功}


\item 
一储存氮气的容器被一绝热轻活塞分隔成两个气室$ A $和$ B $,活塞可无摩擦地滑动。开始时用销钉固定活塞,$ A $中气体体积为$2.5 \times 10 ^ { - 4 } \mathrm { m } ^ { 3 }$,温度为$ 27\ C^{\circ} $,压强为$6.0 \times 10 ^ { 4 } \mathrm { Pa }$;$ B $中气体体积为$4.0 \times 10 ^ { - 4 } \mathrm { m } ^ { 3 }$,温度为$ -17\ C^{\circ} $,压强为$2.0 \times 10 ^ { 4 } \mathrm { Pa }$。现将$ A $中气体的温度降至$ -17\ C^{\circ} $,然后拔掉销钉,并保持$ A $、$ B $中气体温度不变,求稳定后$ A $和$ B $中气体的压强。

\banswer{
$p = 3.2 \times 10 ^ { 4 } \mathrm { Pa }$
}





\end{enumerate}



\newpage
\item 
\exwhere{$ 2018 $年全国\lmd{1}卷}
\begin{enumerate}
\renewcommand{\labelenumi}{\arabic{enumi}.}
% A(\Alph) a(\alph) I(\Roman) i(\roman) 1(\arabic)
%设定全局标号series=example	%引用全局变量resume=example
%[topsep=-0.3em,parsep=-0.3em,itemsep=-0.3em,partopsep=-0.3em]
%可使用leftmargin调整列表环境左边的空白长度 [leftmargin=0em]
\item
如图,一定质量的理想气体从状态$ a $开始,经历过程①、②、③、④到达状态$ e $。对此气体,下列说法正确的是 \xzanswer{BDE} 
(选对$ 1 $个得$ 2 $分,选对$ 2 $个得$ 4 $分,选对$ 3 $个得$ 5 $分;每选错$ 1 $个扣$ 3 $分,最低得分为$ 0 $分)。
\begin{figure}[h!]
\centering
\includesvg[width=0.25\linewidth]{picture/svg/284}
\end{figure}

\fivechoices
{过程①中气体的压强逐渐减小}
{过程②中气体对外界做正功}
{过程④中气体从外界吸收了热量}
{状态$ c $、$ d $的内能相等}
{状态$ d $的压强比状态$ b $的压强小}

\item 
如图,容积为$ V $的汽缸由导热材料制成,面积为$ S $的活塞将汽缸分成容积相等的上下两部分,汽缸上部通过细管与装有某种液体的容器相连,细管上有一阀门$ K $。开始时,$ K $关闭,汽缸内上下两部分气体的压强均为$ P_{0} $。现将$ K $打开,容器内的液体缓慢地流入汽缸,当流入的液体体积为$ \frac{V}{8} $时,将$ K $关闭,活塞平衡时其下方气体的体积减小了$ \frac{V}{6} $。不计活塞的质量和体积,外界温度保持不变,重力加速度大小为$ g $。求流入汽缸内液体的质量。

\begin{figure}[h!]
\flushright
\includesvg[width=0.25\linewidth]{picture/svg/285}
\end{figure}



\banswer{
$m = \frac { 15 p _ { 0 } S } { 26 g }$
}





\end{enumerate}


\newpage
\item 
\exwhere{$ 2018 $年全国\lmd{2}卷}
\begin{enumerate}
\renewcommand{\labelenumi}{\arabic{enumi}.}
% A(\Alph) a(\alph) I(\Roman) i(\roman) 1(\arabic)
%设定全局标号series=example	%引用全局变量resume=example
%[topsep=-0.3em,parsep=-0.3em,itemsep=-0.3em,partopsep=-0.3em]
%可使用leftmargin调整列表环境左边的空白长度 [leftmargin=0em]
\item
对于实际的气体,下列说法正确的是 \xzanswer{BDE} 
。(填正确答案标号。选对$ 1 $个得$ 2 $分,选对$ 2 $个得$ 4 $分,选对$ 3 $个得$ 5 $分。每选错$ 1 $个扣$ 3 $分,最低得分为$ 0 $分)
\fivechoices
{气体的内能包括气体分子的重力势能}
{气体的内能包括气体分子之间相互作用的势能}
{气体的内能包括气体整体运动的动能}
{气体的体积变化时,其内能可能不变}
{气体的内能包括气体分子热运动的动能}

\item 
如图,一竖直放置的汽缸上端开口,汽缸壁内有卡口$ a $和$ b $,$ a $、$ b $间距为$ h $,$ a $距缸底的高度为$ H $;活塞只能在$ a $、$ b $间移动,其下方密封有一定质量的理想气体。已知活塞质量为$ m $,面积为$ S $,厚度可忽略;活塞和汽缸壁均绝热,不计它们之间的摩擦。开始时活塞处于静止状态,上、下方气体压强均为$ p_{0} $,温度均为$ T_{0} $。现用电热丝缓慢加热汽缸中的气体,直至活塞刚好到达$ b $处。求此时汽缸内气体的温度以及在此过程中气体对外所做的功。重力加速度大小为$ g $。
\begin{figure}[h!]
\flushright
\includesvg[width=0.25\linewidth]{picture/svg/286}
\end{figure}

\banswer{

$T_{2}=\left(1+\frac{h}{H}\right)\left(1+\frac{m g}{p_{0} S}\right) T_{0}$\\
$W = \left( p _ { 0 } S + m g \right) h$
}




\end{enumerate}


\newpage
\item 
\exwhere{$ 2018 $年全国\lmd{3}卷}
\begin{enumerate}
\renewcommand{\labelenumi}{\arabic{enumi}.}
% A(\Alph) a(\alph) I(\Roman) i(\roman) 1(\arabic)
%设定全局标号series=example	%引用全局变量resume=example
%[topsep=-0.3em,parsep=-0.3em,itemsep=-0.3em,partopsep=-0.3em]
%可使用leftmargin调整列表环境左边的空白长度 [leftmargin=0em]
\item
如图,一定量的理想气体从状态$ a $变化到状态$ b $,其过程如$ p-V $图中从$ a $到$ b $的直线所示。在此过程中 \xzanswer{BCD} 
。(填正确答案标号。选对$ 1 $个得$ 2 $分,选对$ 2 $个得$ 4 $分,选对$ 3 $个得$ 5 $分。每选错$ 1 $个扣$ 3 $分,最低得分为$ 0 $分)
\begin{figure}[h!]
\centering
\includesvg[width=0.23\linewidth]{picture/svg/287}
\end{figure}

\fivechoices
{气体温度一直降低}
{气体内能一直增加}
{气体一直对外做功}
{气体一直从外界吸热}
{气体吸收的热量一直全部用于对外做功}

\item 
在两端封闭、粗细均匀的$ U $形细玻璃管内有一段水银柱,水银柱的两端各封闭有一段空气。当$ U $形管两端竖直朝上时,左、右两边空气柱的长度分别为$l _ { 1 } = 18.0 \mathrm { cm }$和$l _ { 2 } = 12.0 \mathrm { cm }$,左边气体的压强为$12.0\ \mathrm { cmHg }$。现将$ U $形管缓慢平放在水平桌面上,没有气体从管的一边通过水银逸入另一边。求$ U $形管平放时两边空气柱的长度。在整个过程中,气体温度不变。
\begin{figure}[h!]
\flushright
\includesvg[width=0.23\linewidth]{picture/svg/288}
\end{figure}

\banswer{
两边气柱长度的变化量大小相等$l _ { 1 } ^ { \prime } - l _ { 1 } = l _ { 2 } - l _ { 2 } ^ { \prime }$,$l _ { 1 } ^ { \prime } = 22.5 \mathrm { cm } \quad l _ { 2 } ^ { \prime } = 7.5 \mathrm { cm }$
}





\end{enumerate}


\newpage
\item 
\exwhere{$ 2017 $年新课标 \lmd{1} 卷}
\begin{enumerate}
\renewcommand{\labelenumi}{\arabic{enumi}.}
% A(\Alph) a(\alph) I(\Roman) i(\roman) 1(\arabic)
%设定全局标号series=example	%引用全局变量resume=example
%[topsep=-0.3em,parsep=-0.3em,itemsep=-0.3em,partopsep=-0.3em]
%可使用leftmargin调整列表环境左边的空白长度 [leftmargin=0em]
\item
氧气分子在$ 0 $ $ ^{ \circ } C $和$ 100 $ $ ^{ \circ } C $温度下单位速率间隔的分子数占总分子数的百分比随气体分子速率的变化分别如图中两条曲线所示。下列说法正确的是 \xzanswer{ABC} 。(填正确答案标号。选对$ 1 $个得$ 2 $分,选对$ 2 $个得$ 4 $分,选对$ 3 $个得$ 5 $分。每选错$ 1 $个扣$ 3 $分,最低得分为$ 0 $分)
\begin{figure}[h!]
\centering
\includesvg[width=0.27\linewidth]{picture/svg/289}
\end{figure}


\fivechoices
{图中两条曲线下面积相等}
{图中虚线对应于氧气分子平均动能较小的情形}
{图中实线对应于氧气分子在$ 100 $ $ ^{ \circ } C $时的情形}
{图中曲线给出了任意速率区间的氧气分子数目}
{与$ 0 $ $ ^{ \circ } C $时相比,$ 100 $ $ ^{ \circ } C $时氧气分子速率出现在$ 0 \sim 400 $ $ \ m/s $区间内的分子数占总分子数的百分比较大}



\item 
如图,容积均为$ V $的汽缸$ A $、$ B $下端有细管(容积可忽略)连通,阀门$ K_{2} $位于细管的中部,$ A $、$ B $的顶部各有一阀门$ K_{1} $、$ K_{3} $,$ B $中有一可自由滑动的活塞(质量、体积均可忽略)。初始时,三个阀门均打开,活塞在$ B $的底部;关闭$ K_{2} $、$ K_{3} $,通过$ K_{1} $给汽缸充气,使$ A $中气体的压强达到大气压$ p_{0} $的$ 3 $倍后关闭$ K_{1} $。已知室温为$ 27 $ $ ^{ \circ } C $,汽缸导热。
\begin{enumerate}
\renewcommand{\labelenumi}{\arabic{enumi}.}
% A(\Alph) a(\alph) I(\Roman) i(\roman) 1(\arabic)
%设定全局标号series=example	%引用全局变量resume=example
%[topsep=-0.3em,parsep=-0.3em,itemsep=-0.3em,partopsep=-0.3em]
%可使用leftmargin调整列表环境左边的空白长度 [leftmargin=0em]
\item
打开$ K_{2} $,求稳定时活塞上方气体的体积和压强;
\item 
接着打开$ K_{3} $,求稳定时活塞的位置;
\item 
再缓慢加热汽缸内气体使其温度升高$ 20 ^{ \circ } C $,求此时活塞下方气体的压强。

\end{enumerate}

\begin{minipage}[h!]{0.7\linewidth}
\vspace{0.3em}
\banswer{
\begin{enumerate}
\renewcommand{\labelenumi}{\arabic{enumi}.}
% A(\Alph) a(\alph) I(\Roman) i(\roman) 1(\arabic)
%设定全局标号series=example	%引用全局变量resume=example
%[topsep=-0.3em,parsep=-0.3em,itemsep=-0.3em,partopsep=-0.3em]
%可使用leftmargin调整列表环境左边的空白长度 [leftmargin=0em]
\item
$V_{1}=\frac{V}{2}$ \qquad 
$p_{1}=2 p_{0}$
\item 
活塞上升直到B的顶部
\item 
$p_{3}=1.6 p_{0}$


\end{enumerate}


}

\vspace{0.3em}
\end{minipage}
\hfill
\begin{minipage}[h!]{0.3\linewidth}
\flushright
\vspace{0.3em}
\includesvg[width=0.8\linewidth]{picture/svg/290}
\vspace{0.3em}
\end{minipage}




\end{enumerate}



\newpage	
\item 
\exwhere{$ 2017 $年新课标\lmd{2}卷}
\begin{enumerate}
\renewcommand{\labelenumi}{\arabic{enumi}.}
% A(\Alph) a(\alph) I(\Roman) i(\roman) 1(\arabic)
%设定全局标号series=example	%引用全局变量resume=example
%[topsep=-0.3em,parsep=-0.3em,itemsep=-0.3em,partopsep=-0.3em]
%可使用leftmargin调整列表环境左边的空白长度 [leftmargin=0em]
\item
如图,用隔板将一绝热汽缸分成两部分,隔板左侧充有理想气体,隔板右侧与绝热活塞之间是真空。现将隔板抽开,气体会自发扩散至整个汽缸。待气体达到稳定后,缓慢推压活塞,将气体压回到原来的体积。假设整个系统不漏气。下列说法正确的是 \xzanswer{ABD} 
(选对$ 1 $个得$ 2 $分,选对$ 2 $个得$ 4 $分,选对$ 3 $个得$ 5 $分;每选错$ 1 $个扣$ 3 $分,最低得分为$ 0 $分)。
\begin{figure}[h!]
\centering
\includesvg[width=0.35\linewidth]{picture/svg/291}
\end{figure}

\fivechoices
{气体自发扩散前后内能相同}
{气体在被压缩的过程中内能增大}
{在自发扩散过程中,气体对外界做功}
{气体在被压缩的过程中,外界对气体做功}
{气体在被压缩的过程中,气体分子的平均动能不变}



\item 
一热气球体积为$ V $,内部充有温度为$ T_a $的热空气,气球外冷空气的温度为$ T_b $。已知空气在$ 1 $个大气压、温度为$ T_{0} $时的密度为$ \rho _0 $,该气球内、外的气压始终都为$ 1 $个大气压,重力加速度大小为$ g $。
\begin{enumerate}
\renewcommand{\labelenumi}{\arabic{enumi}.}
% A(\Alph) a(\alph) I(\Roman) i(\roman) 1(\arabic)
%设定全局标号series=example	%引用全局变量resume=example
%[topsep=-0.3em,parsep=-0.3em,itemsep=-0.3em,partopsep=-0.3em]
%可使用leftmargin调整列表环境左边的空白长度 [leftmargin=0em]
\item
求该热气球所受浮力的大小;
\item 
求该热气球内空气所受的重力;
\item 
设充气前热气球的质量为$ m_{0} $,求充气后它还能托起的最大质量。



\end{enumerate}

\banswer{
\begin{enumerate}
\renewcommand{\labelenumi}{\arabic{enumi}.}
% A(\Alph) a(\alph) I(\Roman) i(\roman) 1(\arabic)
%设定全局标号series=example	%引用全局变量resume=example
%[topsep=-0.3em,parsep=-0.3em,itemsep=-0.3em,partopsep=-0.3em]
%可使用leftmargin调整列表环境左边的空白长度 [leftmargin=0em]
\item
$\frac { \rho _ { 0 } g V T _ { 0 } } { T _ { b } }$
\item 
$\frac { \rho _ { 0 } g V T _ { 0 } } { T _ { a } }$
\item 
$\frac { \rho _ { 0 } V T _ { 0 } } { T _ { b } } - \frac { \rho _ { 0 } V T _ { 0 } } { T _ { a } } - m _ { 0 }$



\end{enumerate}


}




\end{enumerate}



\newpage
\item 
\exwhere{$ 2017 $年新课标 \lmd{3} 卷}
\begin{enumerate}
\renewcommand{\labelenumi}{\arabic{enumi}.}
% A(\Alph) a(\alph) I(\Roman) i(\roman) 1(\arabic)
%设定全局标号series=example	%引用全局变量resume=example
%[topsep=-0.3em,parsep=-0.3em,itemsep=-0.3em,partopsep=-0.3em]
%可使用leftmargin调整列表环境左边的空白长度 [leftmargin=0em]
\item
如图,一定质量的理想气体从状态$ a $出发,经过等容过程$ ab $到达状态$ b $,再经过等温过程$ bc $到达状态$ c $,最后经等压过程$ ca $回到状态$ a $。下列说法正确的是 \xzanswer{ABD} 
(填正确答案标号。选对$ 1 $个得$ 2 $分,选对$ 2 $个得$ 4 $分,选对$ 3 $个得$ 5 $分。每选错$ 1 $个扣$ 3 $分,最低得分为$ 0 $分)。
\begin{figure}[h!]
\centering
\includesvg[width=0.23\linewidth]{picture/svg/292}
\end{figure}

\fivechoices
{在过程$ ab $中气体的内能增加}
{在过程$ ca $中外界对气体做功}
{在过程$ ab $中气体对外界做功}
{在过程$ bc $中气体从外界吸收热量}
{在过程$ ca $中气体从外界吸收热量}

\item 
一种测量稀薄气体压强的仪器如图($ a $)所示,玻璃泡$ M $的上端和下端分别连通两竖直玻璃细管$ K_{1} $和$ K_{2} $。$ K_{1} $长为$ l $,顶端封闭,$ K_{2} $上端与待测气体连通;$ M $下端经橡皮软管与充有水银的容器$ R $连通。开始测量时,$ M $与$ K_{2} $相通;逐渐提升$ R $,直到$ K_{2} $中水银面与$ K_{1} $顶端等高,此时水银已进入$ K_{1} $,且$ K_{1} $中水银面比顶端低$ h $,如图($ b $)所示。设测量过程中温度、与$ K_{2} $相通的待测气体的压强均保持不变。已知$ K_{1} $和$ K_{2} $的内径均为$ d $,$ M $的容积为$ V_{0} $,水银的密度为$ \rho $,重力加速度大小为$ g $。求:
\begin{enumerate}
\renewcommand{\labelenumi}{\arabic{enumi}.}
% A(\Alph) a(\alph) I(\Roman) i(\roman) 1(\arabic)
%设定全局标号series=example	%引用全局变量resume=example
%[topsep=-0.3em,parsep=-0.3em,itemsep=-0.3em,partopsep=-0.3em]
%可使用leftmargin调整列表环境左边的空白长度 [leftmargin=0em]
\item
待测气体的压强;
\item 
该仪器能够测量的最大压强。



\end{enumerate}


\begin{minipage}[h!]{0.5\linewidth}
\vspace{0.3em}
\banswer{
\begin{enumerate}
\renewcommand{\labelenumi}{\arabic{enumi}.}
% A(\Alph) a(\alph) I(\Roman) i(\roman) 1(\arabic)
%设定全局标号series=example	%引用全局变量resume=example
%[topsep=-0.3em,parsep=-0.3em,itemsep=-0.3em,partopsep=-0.3em]
%可使用leftmargin调整列表环境左边的空白长度 [leftmargin=0em]
\item
$p _ { x } = \frac { \pi \rho g d ^ { 2 } h ^ { 2 } } { 4 V _ { 0 } + \pi d ^ { 2 } ( l - h ) }$
\item 
$p _ { m } = \frac { \pi \rho g d ^ { 2 } l ^ { 2 } } { 4 V _ { 0 } }$



\end{enumerate}


}

\vspace{0.3em}
\end{minipage}
\hfill
\begin{minipage}[h!]{0.5\linewidth}
\flushright
\vspace{0.3em}
\includesvg[width=0.7\linewidth]{picture/svg/293}
\vspace{0.3em}
\end{minipage}




\end{enumerate}




\newpage
\item 
\exwhere{$ 2017 $年江苏卷}
\begin{enumerate}
\renewcommand{\labelenumi}{\arabic{enumi}.}
% A(\Alph) a(\alph) I(\Roman) i(\roman) 1(\arabic)
%设定全局标号series=example	%引用全局变量resume=example
%[topsep=-0.3em,parsep=-0.3em,itemsep=-0.3em,partopsep=-0.3em]
%可使用leftmargin调整列表环境左边的空白长度 [leftmargin=0em]
\item
一定质量的理想气体从状态$ A $经过状态$ B $变化到状态$ C $,其$ V - T $图象如图图所示。下列说法正确的有 \xzanswer{BC} 
\begin{figure}[h!]
\centering
\includesvg[width=0.23\linewidth]{picture/svg/294}
\end{figure}


\fourchoices
{$ A \rightarrow B $的过程中,气体对外界做功}
{$ A \rightarrow B $的过程中,气体放出热量}
{$ B \rightarrow C $的过程中,气体压强不变}
{$ A \rightarrow B \rightarrow C $的过程中,气体内能增加}


\item 
图中(甲)和(乙)是某同学从资料中查到的两张记录水中炭粒运动位置连线的图片,记录炭粒位置的时间间隔均为$ 30 $ $ s $,两方格纸每格表示的长度相同。比较两张图片可知:若水温相同,\tk{甲}(选填“甲”或“乙”)中炭粒的颗粒较大;若炭粒大小相同,\tk{乙}(选填“甲”或“乙”)中水分子的热运动较剧烈。
\begin{figure}[h!]
\centering
\includesvg[width=0.38\linewidth]{picture/svg/295}
\end{figure}

\item 
科学家可以运用无规则运动的规律来研究生物蛋白分子。资料显示,某种蛋白的摩尔质量为$ 66 $ $ kg/mol $,其分子可视为半径为$ 3 \times 10 ^{-9}\ m $的球,已知阿伏伽德罗常数为$ 6.0 \times 10^{23} \ mol^{-1} $。请估算该蛋白的密度。(计算结果保留一位有效数字)

\banswer{
摩尔体积$V = \frac { 4 } { 3 } \pi r ^ { 3 } N _ { A }$(或$V = ( 2 r ) ^ { 3 } N _ { A }$),得
$\rho = \frac { 3 M } { 4 \pi r ^ { 3 } N _ { A } }$(或$\rho = \frac { M } { 8 r ^ { 3 } N _ { A } }$),$\rho = 1 \times 10 ^ { 3 } \mathrm { kg } / \mathrm { m } ^ { 3 }$(或$\rho = 0.5 \times 10 ^ { 3 } \mathrm { kg } / \mathrm { m } ^ { 3 }$)
}




\end{enumerate}


\newpage
\item 
\exwhere{$ 2017 $年海南卷}
\begin{enumerate}
\renewcommand{\labelenumi}{\arabic{enumi}.}
% A(\Alph) a(\alph) I(\Roman) i(\roman) 1(\arabic)
%设定全局标号series=example	%引用全局变量resume=example
%[topsep=-0.3em,parsep=-0.3em,itemsep=-0.3em,partopsep=-0.3em]
%可使用leftmargin调整列表环境左边的空白长度 [leftmargin=0em]
\item
关于布朗运动,下列说法正确的是 \xzanswer{ABE} 
。(填入正确答案标号。选对$ 1 $个得$ 2 $分,选对$ 2 $个得$ 3 $分,选对$ 3 $个得$ 4 $分;有选错的得$ 0 $分)
\fivechoices
{布朗运动是液体中悬浮微粒的无规则运动}
{液体温度越高,液体中悬浮微粒的布朗运动越剧烈}
{在液体中的悬浮颗粒只要大于某一尺寸,都会发生布朗运动}
{液体中悬浮微粒的布朗运动使液体分子永不停息地做无规则运动}
{液体中悬浮微粒的布朗运动是液体分子对它的撞击作用不平衡所引起的}

\item 
一粗细均匀的$ U $形管$ ABCD $的$ A $端封闭,$ D $端与大气相通。用水银将一定质量的理想气体封闭在$ U $形管的$ AB $一侧,并将两端向下竖直放置,如图所示。此时$ AB $侧的气体柱长度$ l_1=25 $ $ cm $。管中$ AB $、$ CD $两侧的水银面高度差$ h_1=5 $ $ cm $。现将$ U $形管缓慢旋转$ 180 ^{ \circ } $,使$ A $、$ D $两端在上,在转动过程中没有水银漏出。已知大气压强$ p_0=76 $ $ cmHg $。求旋转后,$ AB $、$ CD $两侧的水银面高度差。
\begin{figure}[h!]
\flushright
\includesvg[width=0.25\linewidth]{picture/svg/296}
\end{figure}

\banswer{
$ 1\ cm $
}





\end{enumerate}


\newpage
\item 
\exwhere{$ 2016 $年新课标 \lmd{1} 卷}
\begin{enumerate}
\renewcommand{\labelenumi}{\arabic{enumi}.}
% A(\Alph) a(\alph) I(\Roman) i(\roman) 1(\arabic)
%设定全局标号series=example	%引用全局变量resume=example
%[topsep=-0.3em,parsep=-0.3em,itemsep=-0.3em,partopsep=-0.3em]
%可使用leftmargin调整列表环境左边的空白长度 [leftmargin=0em]
\item
关于热力学定律,下列说法正确的是 \xzanswer{BDE} 
。(填正确答案标号。选对$ 1 $个得$ 2 $分,选对$ 2 $个得$ 4 $分,选对$ 3 $个得$ 5 $分,每选错$ 1 $个扣$ 3 $分,最低得分为$ 0 $分)
\fivechoices
{气体吸热后温度一定升高}
{对气体做功可以改变其内能}
{理想气体等压膨胀过程一定放热}
{热量不可能自发地从低温物体传到高温物体}
{如果两个系统分别与状态确定的第三个系统达到热平衡,那么这两个系统彼此之间也必定达到热平衡}

\item 
在水下气泡内空气的压强大于气泡表面外侧水的压强,两压强差$ \Delta p $与气泡半径$ r $之间的关系为$\Delta p = \frac { 2 \sigma } { r }$,其中$\sigma = 0.070 \mathrm { N } / \mathrm { m }$。现让水下$ 10\ m $处一半径为$ 0.50\ cm $的气泡缓慢上升。已知大气压强$p _ { 0 } = 1.0 \times 10 ^ { 5 } \mathrm { Pa }$,水的密度$\rho = 1.0 \times 10 ^ { 3 } \mathrm { kg } / \mathrm { m } ^ { 3 }$,重力加速度大小$ g=10\ m/s^{2} $。
\begin{enumerate}
\renewcommand{\labelenumi}{\arabic{enumi}.}
% A(\Alph) a(\alph) I(\Roman) i(\roman) 1(\arabic)
%设定全局标号series=example	%引用全局变量resume=example
%[topsep=-0.3em,parsep=-0.3em,itemsep=-0.3em,partopsep=-0.3em]
%可使用leftmargin调整列表环境左边的空白长度 [leftmargin=0em]
\item
求在水下处气泡内外的压强差;
\item 
忽略水温随水深的变化,在气泡上升到十分接近水面时,求气泡的半径与其原来半径之比的近似值。



\end{enumerate}

\banswer{
\begin{enumerate}
\renewcommand{\labelenumi}{\arabic{enumi}.}
% A(\Alph) a(\alph) I(\Roman) i(\roman) 1(\arabic)
%设定全局标号series=example	%引用全局变量resume=example
%[topsep=-0.3em,parsep=-0.3em,itemsep=-0.3em,partopsep=-0.3em]
%可使用leftmargin调整列表环境左边的空白长度 [leftmargin=0em]
\item
$\Delta p = \frac { 2 \times 0.070 } { 5 \times 10 ^ { - 3 } } \mathrm { Pa } = 28 \mathrm { Pa }$
\item 
$\frac { r _ { 2 } } { r _ { 1 } } = \sqrt [ 3 ] { 2 } \approx 1.3$



\end{enumerate}


}




\end{enumerate}




\newpage	
\item 
\exwhere{$ 2016 $年新课标 \lmd{2} 卷}
\begin{enumerate}
\renewcommand{\labelenumi}{\arabic{enumi}.}
% A(\Alph) a(\alph) I(\Roman) i(\roman) 1(\arabic)
%设定全局标号series=example	%引用全局变量resume=example
%[topsep=-0.3em,parsep=-0.3em,itemsep=-0.3em,partopsep=-0.3em]
%可使用leftmargin调整列表环境左边的空白长度 [leftmargin=0em]
\item
一定量的理想气体从状态$ a $开始,经历等温或等压过程$ ab $、$ bc $、$ cd $、$ da $回到原状态,其$ p -- T $图像如图所示,其中对角线$ ac $的延长线过原点$ O $。下列判断正确的是 \xzanswer{ABE} 
。(填正确答案标号。选对$ 1 $个得$ 2 $分,选对$ 2 $个得$ 4 $分,选对$ 3 $个得$ 5 $分。每选错$ 1 $个扣$ 3 $分,最低得分为$ 0 $分)
\begin{figure}[h!]
\centering
\includesvg[width=0.23\linewidth]{picture/svg/297}
\end{figure}

\fivechoices
{气体在$ a $、$ c $两状态的体积相等}
{气体在状态$ a $时的内能大于它在状态$ c $时的内能}
{在过程$ cd $中气体向外界放出的热量大于外界对气体做的功}
{在过程$ da $中气体从外界吸收的热量小于气体对外界做的功}
{在过程$ bc $中外界对气体做的功等于在过程$ da $中气体对外界做的功}



\item 
一氧气瓶的容积为$ 0.08 $ $ m^{3} $,开始时瓶中氧气的压强为$ 20 $个大气压。某实验室每天消耗$ 1 $个大气压的氧气$ 0.36 $ $ m^{3} $。当氧气瓶中的压强降低到$ 2 $个大气压时,需重新充气。若氧气的温度保持不变,求这瓶氧气重新充气前可供该实验室使用多少天。

\banswer{
$ N=4 $(天)
}





\end{enumerate}


\newpage
\item 
\exwhere{$ 2016 $年新课标 \lmd{3} 卷}
\begin{enumerate}
\renewcommand{\labelenumi}{\arabic{enumi}.}
% A(\Alph) a(\alph) I(\Roman) i(\roman) 1(\arabic)
%设定全局标号series=example	%引用全局变量resume=example
%[topsep=-0.3em,parsep=-0.3em,itemsep=-0.3em,partopsep=-0.3em]
%可使用leftmargin调整列表环境左边的空白长度 [leftmargin=0em]
\item
关于气体的内能,下列说法正确的是 \xzanswer{CDE} 
。(填正确答案标号。选对$ 1 $个得$ 2 $分,选对$ 2 $个得$ 4 $分,选对$ 3 $个得$ 5 $分。每选错$ 1 $个扣$ 3 $分,最低得分为$ 0 $分)
\fivechoices
{质量和温度都相同的气体,内能一定相同}
{气体温度不变,整体运动速度越大,其内能越大}
{气体被压缩时,内能可能不变}
{一定量的某种理想气体的内能只与温度有关}
{一定量的某种理想气体在等压膨胀过程中,内能一定增加}




\item 
一$ U $形玻璃管竖直放置,左端开口,右端封闭,左端上部有一光滑的轻活塞。初始时,管内汞柱及空气柱长度如图所示。用力向下缓慢推活塞,直至管内两边汞柱高度相等时为止。求此时右侧管内气体的压强和活塞向下移动的距离。已知玻璃管的横截面积处处相同;在活塞向下移动的过程中,没有发生气体泄漏;大气压强$ p_0=75.0 $ $ cmHg $。环境温度不变。
\begin{figure}[h!]
\centering
\includesvg[width=0.23\linewidth]{picture/svg/298}
\end{figure}

\banswer{
$ h=9.42\ m $
}




\end{enumerate}

\newpage
\item 
\exwhere{$ 2016 $年江苏卷}
\begin{enumerate}
\renewcommand{\labelenumii}{(\arabic{enumii})}

\item 
在高原地区烧水需要使用高压锅,水烧开后,锅内水面上方充满饱和汽,停止加热,高压锅在密封状态下缓慢冷却,在冷却过程中,锅内水蒸气的变化情况为 \xzanswer{AC} 

\fourchoices
{压强变小}
{压强不变}
{一直是饱和汽}
{变为未饱和汽}


\item 
如图甲所示,在斯特林循环的$ p -- V $图像中,一定质量理想气体从状态$ A $依次经过状态$ B $、$ C $和$ D $后再回到状态$ A $,整个过程由两个等温和两个等容过程组成,$ B \rightarrow C $的过程中,单位体积中的气体分子数目\tk{不变}.(选填“增大”“减小”或“不变”),状态$ A $和状态$ D $的气体分子热运动速率的统计分布图像如图乙所示,则状态$ A $对应的是\tk{①}(选填“①”或“②”);
\begin{figure}[h!]
\centering
\includesvg[width=0.43\linewidth]{picture/svg/299}
\end{figure}

\item 
如图甲所示,在$ A \rightarrow B $和$ D \rightarrow A $的过程中,气体放出的热量分别为$ 4 $ $ J $和$ 20 $ $ J $。在$ B \rightarrow C $和$ C \rightarrow D $的过程中,气体吸收的热量分别为$ 20 $ $ J $和$ 12 $ $ J $;求气体完成一次循环对外界所做的功。

\banswer{
$ 8\ J $
}


\end{enumerate}



\newpage
\item 
\exwhere{$ 2016 $年海南卷}
\begin{enumerate}
\renewcommand{\labelenumi}{\arabic{enumi}.}
% A(\Alph) a(\alph) I(\Roman) i(\roman) 1(\arabic)
%设定全局标号series=example	%引用全局变量resume=example
%[topsep=-0.3em,parsep=-0.3em,itemsep=-0.3em,partopsep=-0.3em]
%可使用leftmargin调整列表环境左边的空白长度 [leftmargin=0em]
\item
一定量的理想气体从状态$ M $可以经历过程$ 1 $或者过程$ 2 $到达状态$ N $,其$ p-V $图像如图所示。在过程$ 1 $中,气体始终与外界无热量交换;在过程$ 2 $中,气体先经历等容变化再经历等压变化。对于这两个过程,下列说法正确的是 \xzanswer{ABE} 
\begin{figure}[h!]
\centering
\includesvg[width=0.23\linewidth]{picture/svg/300}
\end{figure}

\fivechoices
{气体经历过程$ 1 $,其温度降低}
{气体经历过程$ 1 $,其内能减少}
{气体在过程$ 2 $中一直对外放热}
{气体在过程$ 2 $中一直对外做功}
{气体经历过程$ 1 $的内能该变量与经历过程$ 2 $的相同}



\item 
如图,密闭汽缸两侧与一“$ U $”形管的两端相连,汽缸壁导热;“$ U $”形管内盛有密度为$\rho = 7.5 \times 10 ^ { 2 } \mathrm { kg } / \mathrm { m } ^ { 3 }$的液体。一活塞将汽缸分成左、右两个气室,开始时,左气室的体积是右气室的体积的一半,气体的压强均为$P _ { 0 } = 4.5 \times 10 ^ { 3 } \mathrm { Pa }$。外界温度保持不变。缓慢向右拉活塞使$ U $形管两侧液面的高度差$ h=40 $ $ cm $,求此时左、右两气室的体积之比。取重力加速度大小$ g=10\ m/s^{2} $,$ U $形管中气体的体积和活塞拉杆的体积忽略不计。
\begin{figure}[h!]
\flushright
\includesvg[width=0.25\linewidth]{picture/svg/301}
\end{figure}

\banswer{
$V _ { 1 }: V _ { 2 } = 1: 1$
}




\end{enumerate}


\newpage
\item 
\exwhere{$ 2015 $年理综新课标 \lmd{1} 卷}
\begin{enumerate}
\renewcommand{\labelenumi}{\arabic{enumi}.}
% A(\Alph) a(\alph) I(\Roman) i(\roman) 1(\arabic)
%设定全局标号series=example	%引用全局变量resume=example
%[topsep=-0.3em,parsep=-0.3em,itemsep=-0.3em,partopsep=-0.3em]
%可使用leftmargin调整列表环境左边的空白长度 [leftmargin=0em]
\item
下列说法正确的是 \xzanswer{BCD} 
(填正确答案标号,选对一个得$ 2 $分,选对$ 2 $个得$ 4 $分,选对$ 3 $个得$ 5 $分。每选错一个扣$ 3 $分,最低得分为$ 0 $分 )
\fivechoices
{将一块晶体敲碎后,得到的小颗粒是非晶体}
{固体可以分为晶体和非晶体两类,有些晶体在不同的方向上有不同的光学性质}
{由同种元素构成的固体,可能会由于原子的排列方式不同而成为不同的晶体}
{在合适的条件下,某些晶体可以转化为非晶体,某些非晶体也可以转化为晶体}
{在熔化过程中,晶体要吸收热量,但温度保持不变,内能也保持不变}

\item 
如图,一固定的竖直气缸由一大一小两个同轴圆筒组成,两圆筒中各有一个活塞,已知大活塞的质量为$ m_{1}=2.50 \ kg $,横截面积为$ s_1=80.0 \ cm^{2} $,小活塞的质量为$ m_{2}=1.50 \ kg $,横截面积为$ s_2=40.0 \ cm^{2} $;两活塞用刚性轻杆连接,间距保持为$ l=40.0 \ cm $,气缸外大气压强为$ p=1.00 \times 10^5\ Pa $,温度为$ T=303\ K $。初始时大活塞与大圆筒底部相距$ l/2 $,两活塞间封闭气体的温度为$ T_1=495\ K $,现气缸内气体温度缓慢下降,活塞缓慢下移,忽略两活塞与气缸壁之间的摩擦,重力加速度$ g $取$ 10 \ m/s ^{2} $,求:
\begin{enumerate}
\renewcommand{\labelenumi}{\arabic{enumi}.}
% A(\Alph) a(\alph) I(\Roman) i(\roman) 1(\arabic)
%设定全局标号series=example	%引用全局变量resume=example
%[topsep=-0.3em,parsep=-0.3em,itemsep=-0.3em,partopsep=-0.3em]
%可使用leftmargin调整列表环境左边的空白长度 [leftmargin=0em]
\item
在大活塞与大圆筒底部接触前的瞬间,缸内封闭气体的温度;

\item 
缸内封闭的气体与缸外大气达到热平衡时,缸内封闭气体的压强.



\end{enumerate}
\begin{figure}[h!]
\flushright
\includesvg[width=0.15\linewidth]{picture/svg/302}
\end{figure}

\banswer{
\begin{enumerate}
\renewcommand{\labelenumi}{\arabic{enumi}.}
% A(\Alph) a(\alph) I(\Roman) i(\roman) 1(\arabic)
%设定全局标号series=example	%引用全局变量resume=example
%[topsep=-0.3em,parsep=-0.3em,itemsep=-0.3em,partopsep=-0.3em]
%可使用leftmargin调整列表环境左边的空白长度 [leftmargin=0em]
\item
$T _ { 2 } = 330 \mathrm { K }$
\item 
$p _ { 2 } = 1.01 \times 10 ^ { 5 } \mathrm { pa }$


\end{enumerate}


}




\end{enumerate}


\newpage
\item 
\exwhere{$ 2015 $年理综新课标 \lmd{2} 卷}
\begin{enumerate}
\renewcommand{\labelenumi}{\arabic{enumi}.}
% A(\Alph) a(\alph) I(\Roman) i(\roman) 1(\arabic)
%设定全局标号series=example	%引用全局变量resume=example
%[topsep=-0.3em,parsep=-0.3em,itemsep=-0.3em,partopsep=-0.3em]
%可使用leftmargin调整列表环境左边的空白长度 [leftmargin=0em]
\item
关于扩散现象,下列说法正确的是 \xzanswer{ACD} 
。(填正确答案标号。选对$ 1 $个得$ 2 $分,选对$ 2 $个得$ 4 $分,选对$ 3 $个的$ 5 $分。每选错$ 1 $个扣$ 3 $分,最低得分为$ 0 $分。)
\fivechoices
{温度越高,扩散进行得越快}
{扩散现象是不同物质间的一种化学反应}
{扩散现象是由物质分子无规则运动产生的}
{扩散现象在气体、液体和固体中都能发生}
{液体中的扩散现象是由于液体的对流形成的}

\item 
如图,一粗细均匀的$ U $形管竖直放置,$ A $侧上端封闭,$ B $侧上端与大气相通,下端开口处开关$ K $关闭;$ A $侧空气柱的长度为$ l=10.0 \ cm $,$ B $侧水银面比$ A $侧的高$ h=3.0 \ cm $。现将开关$ K $打开,从$ U $形管中放出部分水银,当两侧水银面的高度差$ h_1=10.0 \ cm $时将开关$ K $关闭。已知大气压强$ p_0=75 \ cmHg $。
\begin{enumerate}
\renewcommand{\labelenumi}{\arabic{enumi}.}
% A(\Alph) a(\alph) I(\Roman) i(\roman) 1(\arabic)
%设定全局标号series=example	%引用全局变量resume=example
%[topsep=-0.3em,parsep=-0.3em,itemsep=-0.3em,partopsep=-0.3em]
%可使用leftmargin调整列表环境左边的空白长度 [leftmargin=0em]
\item
求放出部分水银后$ A $侧空气柱的长度;
\item 
此后再向$ B $侧注入水银,使$ A $、$ B $两侧的水银面达到同一高度,求注入的水银在管内的长度。



\end{enumerate}
\begin{figure}[h!]
\flushright
\includesvg[width=0.25\linewidth]{picture/svg/303}
\end{figure}

\banswer{
\begin{enumerate}
\renewcommand{\labelenumi}{\arabic{enumi}.}
% A(\Alph) a(\alph) I(\Roman) i(\roman) 1(\arabic)
%设定全局标号series=example	%引用全局变量resume=example
%[topsep=-0.3em,parsep=-0.3em,itemsep=-0.3em,partopsep=-0.3em]
%可使用leftmargin调整列表环境左边的空白长度 [leftmargin=0em]
\item
$ l_{A}=12.0 \ cm $
\item 
$ \Delta h =13.2\ cm $


\end{enumerate}


}





\end{enumerate}


\newpage
\item 
\exwhere{$ 2015 $年理综福建卷}
(本题共有两小题,每小题$ 6 $分,共$ 12 $分。每小题只有一个选项符合题意)
\begin{enumerate}
\renewcommand{\labelenumii}{(\arabic{enumii})}

\item 
下列有关分子动理论和物质结构的认识,其中正确的是 \xzanswer{B} 
 。(填选项前的字母)
\fourchoices
{分子间距离减小时分子势能一定减小}
{温度越高,物体中分子无规则运动越剧烈}
{物体内热运动速率大的分子数占总分子数比例与温度无关}
{非晶体的物理性质各向同性而晶体的物理性质都是各向异性}



\item 
如图,一定质量的理想气体,由状态$ a $经过$ ab $过程到达状态$ b $或者经过$ ac $过程到达状态$ c $。设气体在状态$ b $和状态$ c $的温度分别为$ T_b $和$ T_c $,在过程$ ab $和$ ac $中吸收的热量分别为$ Q_{ab} $和$ Q_{ac} $,则 \xzanswer{C} 
\begin{figure}[h!]
\centering
\includesvg[width=0.23\linewidth]{picture/svg/304}
\end{figure}

\fourchoices
{$ T_b >T_c $,$ Q_{ab}>Q_{ac} $}
{$ T_b >T_c $,$ Q_{ab}<Q_{ac} $}
{$ T_b =T_c $,$ Q_{ab}>Q_{ac} $}
{$ T_b =T_c $,$ Q_{ab}<Q_{ac} $}

\end{enumerate}	

\item 	
\exwhere{$ 2015 $年理综重庆卷}
\begin{enumerate}
\renewcommand{\labelenumi}{\arabic{enumi}.}
% A(\Alph) a(\alph) I(\Roman) i(\roman) 1(\arabic)
%设定全局标号series=example	%引用全局变量resume=example
%[topsep=-0.3em,parsep=-0.3em,itemsep=-0.3em,partopsep=-0.3em]
%可使用leftmargin调整列表环境左边的空白长度 [leftmargin=0em]
\item
某驾驶员发现中午时车胎内的气压高于清晨时的,且车胎体积增大.若这段时间胎内气体质量不变且可视为理想气体,那么 \xzanswer{D} 

\fourchoices
{外界对胎内气体做功,气体内能减小}
{外界对胎内气体做功,气体内能增大}
{胎内气体对外界做功,内能减小}
{胎内气体对外界做功,内能增大}



\item 
北方某地的冬天室外气温很低,吹出的肥皂泡会很快冻结。若刚吹出时肥皂泡内气体温度为$ T_{1} $,压强为$ P_{1} $,肥皂泡冻结后泡内气体温度降为$ T_{2} $. 整个过程中泡内气体视为理想气体,不计体积和质量变化,大气压强为$ P_{0} $. 求冻结后肥皂膜内外气体的压强差。


\banswer{
$\frac { T _ { 2 } } { T _ { 1 } } P _ { 1 } - P _ { 0 }$
}




\end{enumerate}


\item 
\exwhere{$ 2015 $年江苏卷}
\begin{enumerate}
\renewcommand{\labelenumii}{(\arabic{enumii})}

\item 
对下列几种固体物质的认识,正确的有 \xzanswer{AD} 


\fourchoices
{食盐熔化过程中,温度保持不变,说明食盐是晶体}
{烧热的针尖接触涂有蜂蜡薄层的云母片背面,熔化的蜂蜡呈椭圆形,说明蜂蜡是晶体}
{天然石英表现为各向异性,是由于该物质的微粒在空间的排列不规则}
{石墨和金刚石的物理性质不同,是由于组成它们的物质微粒排列结构不同}

\item 
在装有食品的包装袋中充入氮气,可以起到保质作用。 某厂家为检测包装袋的密封性,在包装袋中充满一定量的氮气,然后密封进行加压测试。 测试时,对包装袋缓慢地施加压力. 将袋内的氮气视为理想气体,则加压测试过程中,包装袋内壁单位面积上所受气体分子撞击的作用力 \tk{增大} (选填“增大”、“减小”或“不变”),包装袋内氮气的内能 \tk{不变}(选填“增大”、“减小”或“不变”).

\item 
给某包装袋充入氮气后密封,在室温下,袋中气体压强为 $ 1 $ 个标准大气压、体积为 $ 1 $ $ L $.将其缓慢压缩到压强为 $ 2 $ 个标准大气压时,气体的体积变为 $ 0.45 $ $ L $. 请通过计算判断该包装袋是否漏气.

\banswer{
包装袋漏气。
}

\end{enumerate}





\newpage	
\item 
\exwhere{$ 2015 $年理综山东卷}
\begin{enumerate}
\renewcommand{\labelenumi}{\arabic{enumi}.}
% A(\Alph) a(\alph) I(\Roman) i(\roman) 1(\arabic)
%设定全局标号series=example	%引用全局变量resume=example
%[topsep=-0.3em,parsep=-0.3em,itemsep=-0.3em,partopsep=-0.3em]
%可使用leftmargin调整列表环境左边的空白长度 [leftmargin=0em]
\item
墨滴入水,扩而散之,徐徐混匀。关于该现象的分析正确的是 \xzanswer{BC} 
。(双选,填正确答案标号)

\fourchoices
{混合均匀主要是由于碳粒受重力作用}
{混合均匀的过程中,水分子和碳粒都做无规则运动}
{使用碳粒更小的墨汁,混合均匀的过程进行得更迅速}
{墨汁的扩散运动是由于碳粒和水分子发生化学反应引起的}


\item 
扣在水平桌面上的热杯盖有时会发生被顶起的现象;如图,截面积为$ S $的热杯盖扣在水平桌面上,开始时内部封闭气体的温度为$ 300\ K $,压强为大气压强$ p_{0} $。当封闭气体温度上升至$ 303\ K $时,杯盖恰好被整体顶起,放出少许气体后又落回桌面,其内部压强立即减为$ p_{0} $,温度仍为$ 303\ K $。再经过一段时间,内部气体温度恢复到$ 300\ K $。整个过程中封闭气体均可视为理想气体。求:
\begin{enumerate}
\renewcommand{\labelenumi}{\arabic{enumi}.}
% A(\Alph) a(\alph) I(\Roman) i(\roman) 1(\arabic)
%设定全局标号series=example	%引用全局变量resume=example
%[topsep=-0.3em,parsep=-0.3em,itemsep=-0.3em,partopsep=-0.3em]
%可使用leftmargin调整列表环境左边的空白长度 [leftmargin=0em]
\item
当温度上升到$ 303\ K $且尚未放气时,封闭气体的压强;
\item 
当温度恢复到$ 300\ K $时,竖直向上提起杯盖所需的最小力.


\end{enumerate}
\begin{figure}[h!]
\flushright
\includesvg[width=0.25\linewidth]{picture/svg/305}
\end{figure}

\banswer{
\begin{enumerate}
\renewcommand{\labelenumi}{\arabic{enumi}.}
% A(\Alph) a(\alph) I(\Roman) i(\roman) 1(\arabic)
%设定全局标号series=example	%引用全局变量resume=example
%[topsep=-0.3em,parsep=-0.3em,itemsep=-0.3em,partopsep=-0.3em]
%可使用leftmargin调整列表环境左边的空白长度 [leftmargin=0em]
\item
$p _ { 1 } = \frac { T _ { 1 } } { T _ { 0 } } p _ { 0 } = 1.01 p _ { 0 }$
\item 
$F _ { \min } + p _ { 2 } S = m g + p _ { 0 } S$,$F _ { \min } \approx 0.02 p _ { 0 } S$.



\end{enumerate}


}




\end{enumerate}



\newpage
\item 
\exwhere{$ 2015 $年海南卷}
\begin{enumerate}
\renewcommand{\labelenumi}{\arabic{enumi}.}
% A(\Alph) a(\alph) I(\Roman) i(\roman) 1(\arabic)
%设定全局标号series=example	%引用全局变量resume=example
%[topsep=-0.3em,parsep=-0.3em,itemsep=-0.3em,partopsep=-0.3em]
%可使用leftmargin调整列表环境左边的空白长度 [leftmargin=0em]
\item
已知地球大气层的厚度$ h $远小于地球半径$ R $,空气平均摩尔质量为$ M $,阿伏伽德罗常数为$ N_{A} $,地面大气压强为$ P_{0} $,重力加速度大小为$ g $。由此可估算得,地球大气层空气分子总数为 \tk{$n = \frac { 4 \pi R ^ { 2 } P _ { 0 } N _ { A } } { M g }$} ,空气分子之间的平均距离
为 \tk{$a = \sqrt [ 3 ] { \frac { M g h } { P _ { 0 } N _ { A } } }$} .
 
\item 
如图,一底面积为$ S $、内壁光滑的圆柱形容器竖直放置在水平地面上,开口向上,内有两个质量均为$ m $的相同活塞$ A $和$ B $ ;在$ A $与$ B $之间、$ B $与容器底面之间分别封有一定量的同样的理想气体,平衡时体积均为$ V $。已知容器内气体温度始终不变,重力加速度大小为$ g $,外界大气压强为$ p_{0} $。现假设活塞$ B $发生缓慢漏气,致使$ B $最终与容器底面接触。求活塞$ A $移动的距离。
\begin{figure}[h!]
\flushright
\includesvg[width=0.2\linewidth]{picture/svg/306}
\end{figure}

\banswer{
$\Delta h = \frac { 2 P _ { 0 } S + 3 m g } { \left( P _ { 0 } S + m g \right) S } - \frac { 2 V } { S }$
}




\end{enumerate}







\end{enumerate}	





