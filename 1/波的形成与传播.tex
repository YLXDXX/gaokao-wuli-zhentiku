\bta{波的形成和传播}


\begin{enumerate}
	%\renewcommand{\labelenumi}{\arabic{enumi}.}
	% A(\Alph) a(\alph) I(\Roman) i(\roman) 1(\arabic)
	%设定全局标号series=example	%引用全局变量resume=example
	%[topsep=-0.3em,parsep=-0.3em,itemsep=-0.3em,partopsep=-0.3em]
	%可使用leftmargin调整列表环境左边的空白长度 [leftmargin=0em]
	\item
\exwhere{$ 2013 $ 年北京卷}
 一列沿 $ x $ 轴正方向传播的简谐机械横波,波速为 $ 4 \ m /s $。某时刻波形如图所示,下列说法正确的
是 \xzanswer{D} 
\begin{figure}[h!]
	\centering
	\includesvg[width=0.23\linewidth]{picture/svg/GZ-3-tiyou-1336}
\end{figure}

\fourchoices
{这列波的振幅为 $ 4 \ cm $}
{这列波的周期为 $ 1 \ s $}
{此时 $ x=4 \ m $ 处质点沿 $ y $ 轴负方向运动}
{此时 $ x=4 \ m $ 处质点的加速度为 $ 0 $}



\item 
\exwhere{$ 2013 $ 年上海卷}
一列横波沿水平绳传播,绳的一端在 $ t=0 $ 时开始做周期为 $ T $ 的简谐运动,经过时间 $ t $($ 3 T /4<t  <T $),绳上某点位于平衡位置上方的最大位移处。则在 $ 2t $ 时,该点位于平衡位置的 \xzanswer{A} 

\fourchoices
{上方,且向上运动}
{上方,且向下运动}
{下方,且向上运动}
{下方,且向下运动}


\item 
\exwhere{$ 2014 $ 年理综浙江卷}
下列说法正确的是 \xzanswer{B} 

\fourchoices
{机械波的振幅与波源无关}
{机械波的传播速度由介质本身的性质决定}
{物体受到的静摩擦力方向与其运动方向相反}
{动摩擦因数的数值跟相互接触的两个物体的材料无关}


\item 
\exwhere{$ 2013 $ 年福建卷}
如图,$ t=0 $ 时刻,波源在坐标原点从平衡位置沿 $ y $ 轴正方向开始振动,振动周期为 $ 0.4 \ s $,在同
一均匀介质中形成沿 $ x $ 轴正、负两方向传播的简谐横波。下图中能够正确表示 $ t=0.6 $ 时波形的图是 \xzanswer{C} 

\pfourchoices
{\includesvg[width=4.3cm]{picture/svg/GZ-3-tiyou-1337}}
{\includesvg[width=4.3cm]{picture/svg/GZ-3-tiyou-1338}}
{\includesvg[width=4.3cm]{picture/svg/GZ-3-tiyou-1339}}
{\includesvg[width=4.3cm]{picture/svg/GZ-3-tiyou-1340}}


\item 
\exwhere{$ 2012 $ 年理综浙江卷}
用手握住较长软绳的一端连续上下抖动,形成一列简谐横波。某一时刻的波形如图所示,绳上
$ a $、$ b $ 两质点均处于波峰位置。下列说法正确的是 \xzanswer{D} 
\begin{figure}[h!]
	\centering
	\includesvg[width=0.23\linewidth]{picture/svg/GZ-3-tiyou-1341}
\end{figure}

\fourchoices
{$ a $、$ b $ 两点之间的距离为半个波长}
{$ a $、$ b $ 两点振动开始时刻相差半个周期}
{$ b $ 点完成全振动次数比 $ a $ 点多一次}
{$ b $ 点完成全振动次数比 $ a $ 点少一次}


\item 
\exwhere{$ 2011 $ 年理综北京卷}
介质中有一列简谐机械波传播,对于其中某个振动质点 \xzanswer{D} 

\fourchoices
{它的振动速度等于波的传播速度}
{它的振动方向一定垂直于波的传播方向}
{它在一个周期内走过的路程等于一个波长}
{它的振动频率等于波源的振动频率}



\item 
\exwhere{$ 2011 $ 年理综四川卷}
如图为一列沿 $ x $ 轴负方向传播的简谐横波在 $ t=0 $ 时的波
形图,当 $ Q $ 点在 $ t=0 $ 时的振动状态传到 $ P $ 点时,则 \xzanswer{B} 
\begin{figure}[h!]
	\centering
	\includesvg[width=0.23\linewidth]{picture/svg/GZ-3-tiyou-1342}
\end{figure}

\fourchoices
{$ 1 \ cm <x<3 \ cm $ 范围内的质点正在向 $ y $ 轴的负方向运动}
{$ Q $ 处的质点此时的加速度沿 $ y $ 轴的正方向}
{$ Q $ 处的质点此时正在波峰位置}
{$ Q $ 处的质点此时运动到 $ P $ 处}



\item 
\exwhere{$ 2011 $ 年理综天津卷}
位于坐标原点处的波源 $ A $ 沿 $ y $ 轴做简谐运动。$ A $ 刚好完成一次全振
动时,在介质中形成简谐横波的波形如图所示。$ B $ 是沿波传播方向上介
质的一个质点,则 \xzanswer{ABD} 
\begin{figure}[h!]
	\centering
	\includesvg[width=0.23\linewidth]{picture/svg/GZ-3-tiyou-1343}
\end{figure}

\fourchoices
{波源 $ A $ 开始振动时的运动方向沿 $ y $ 轴负方向。}
{此后的 $ 1/4 $ 周期内回复力对波源 $ A $ 一直做负功。}
{经半个周期时间质点 $ B $ 将向右迁移半个波长}
{在一个周期时间内 $ A $ 所受回复力的冲量为零}


\item 
\exwhere{$ 2011 $ 年理综全国卷}
一列简谐横波沿 $ x $ 轴传播,波长为 $ 1.2 \ m $,振幅为 $ A $。当坐标为 $ x=0 $ 处质元的位移为$ -\frac{\sqrt{3}}{2}A $  且向 $ y $ 轴负方向运动时.坐标为 $ x=0.4 \ m $ 处质元的位移为
$ -\frac{\sqrt{3}}{2}A $ 。当坐标为 $ x=0.2 \ m $ 处的质元位于平衡位置
且向 $ y $ 轴正方向运动时,$ x=0.4 \ m $ 处质元的位移和运动方向分别为 \xzanswer{C} 
\begin{figure}[h!]
	\centering
	\includesvg[width=0.23\linewidth]{picture/svg/GZ-3-tiyou-1344}
\end{figure}


\fourchoices
{$-\frac{1}{2} A 、$ 沿 $y$ 轴正方向}
{$\quad-\frac{1}{2} A,$ 沿 $y$ 轴负方向}
{$-\frac{\sqrt{3}}{2} A$ 、 沿 $y$ 轴正方向}
{$-\frac{\sqrt{3}}{2} A $、 沿 $y$ 轴负方向}



\item 
\exwhere{$ 2015 $ 年理综北京卷}
周期为 $ 2.0 \ s $ 的简谐横波沿 $ x $ 轴传播,该波在某时刻的图像如图所示,此
时质点 $ P $ 沿 $ y $ 轴负方向运动,则该波 \xzanswer{B} 
\begin{figure}[h!]
	\centering
	\includesvg[width=0.23\linewidth]{picture/svg/GZ-3-tiyou-1345}
\end{figure}

\item 
\exwhere{$ 2015 $ 年理综四川卷}
平静湖面传播着一列水面波(横波),在波的传播方向上有相距 $ 3 \ m $ 的甲、
乙两小木块随波上下运动,测得两小木块每分钟上下 $ 30 $ 次,甲在波谷时,乙在波峰,且两木块之
间有一个波峰。这列水面波 \xzanswer{C} 


\fourchoices
{频率是 $ 30 \ Hz $}
{波长是 $ 3 \ m $}
{波速是 $ 1 \ m /s $}
{周期是 $ 0.1 \ s $}


\item 
\exwhere{$ 2016 $ 年上海卷}
甲、乙两列横波在同一介质中分别从波源 $ M $、$ N $ 两点沿 $ x $ 轴相向传播,波速为
$ 2 \ m /s $,振幅相同;某时刻的图像如图所示。
则 \xzanswer{ABD} 
\begin{figure}[h!]
	\centering
	\includesvg[width=0.23\linewidth]{picture/svg/GZ-3-tiyou-1346}
\end{figure}

\fourchoices
{甲乙两波的起振方向相反}
{甲乙两波的频率之比为 $ 3:2 $}
{再经过 $ 3 \ s $,平衡位置在 $ x=7 \ m $ 处的质点振动方向向下}
{再经过 $ 3 \ s $,两波源间(不含波源)有 $ 5 $ 个质点位移为零}


\item 
\exwhere{$ 2017 $ 年天津卷}
手持较长软绳端点 $ O $ 以周期 $ T $ 在竖直方向上做简谐运动,带动绳上的其他质
点振动形成简谐波沿绳水平传播,示意如图。绳上有另一质点 $ P $,且 $ O $、$ P $ 的平衡位置间距为 $ L $。$ t=0 $
时,$ O $ 位于最高点,$ P $ 的位移恰好为零,速度方向竖直向上,下列判断正确的是 \xzanswer{C} 
\begin{figure}[h!]
	\centering
	\includesvg[width=0.23\linewidth]{picture/svg/GZ-3-tiyou-1347}
\end{figure}

\fourchoices
{该简谐波是纵波}
{该简谐波的最大波长为 $ 2L $}
{$t=\frac{T}{8}$ 时, $P$ 在平衡位置上方}
{$t=\frac{3 T}{8}$ 时, $P$ 的速度方向坚直向上}


	
	
	
\end{enumerate}

