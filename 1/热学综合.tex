\bta{热学综合}


\begin{enumerate}[leftmargin=0em]
\renewcommand{\labelenumi}{\arabic{enumi}.}
% A(\Alph) a(\alph) I(\Roman) i(\roman) 1(\arabic)
%设定全局标号series=example	%引用全局变量resume=example
%[topsep=-0.3em,parsep=-0.3em,itemsep=-0.3em,partopsep=-0.3em]
%可使用leftmargin调整列表环境左边的空白长度 [leftmargin=0em]
\item
\exwhere{$ 2013 $年北京卷}
下列说法正确的是 \xzanswer{A} 

\fourchoices
{液体中悬浮的微粒的无规则运动称为布朗运动}
{液体分子的无规则运动称为布朗运动}
{物体从外界吸收热量,其内能一定增加}
{物体对外界做功,其内能一定减少}


\item 
\exwhere{$ 2013 $年上海卷}
液体与固体具有的相同特点是 \xzanswer{B} 


\fourchoices
{都具有确定的形状}
{体积都不易被压缩}
{物质分子的位置都确定}
{物质分子都在固定位置附近振动}


\item 
\exwhere{$ 2013 $年广东卷}
下图为某同学设计的喷水装置,内部装有$ 2\ L $水,上部密封$ 1\ atm $的空气$ 0.5\ L $,保持阀门关闭,再充入$ 1\ atm $的空气$ 0.1L $,设在所有过程中空气可看作理想气体,且温度不变,下列说法正确的有 \xzanswer{AC} 
\begin{figure}[h!]
\centering
\includesvg[width=0.23\linewidth]{picture/svg/275}
\end{figure}

\fourchoices
{充气后,密封气体压强增加}
{充气后,密封气体的分子平均动能增加}
{打开阀门后,密封气体对外界做正功}
{打开阀门后,不再充气也能把水喷光}



\item 
\exwhere{$ 2012 $年理综四川卷}
物体由大量分子组成,下列说法正确的是 \xzanswer{C} 

\fourchoices
{分子热运动越剧烈,物体内每个分子的动能越大}
{分子间引力总是随着分子间的距离减小而减小}
{物体的内能跟物体的温度和体积有关}
{只有外界对物体做功才能增加物体的内能}


\item 
\exwhere{$ 2014 $年物理上海卷}
如图,在水平放置的刚性气缸内用活塞封闭两部分气体$ A $和$ B $,质量一定的两活塞用杆连接。气缸内两活塞之间保持真空,活塞与气缸璧之间无摩擦,左侧活塞面积较大,$ A $、$ B $的初始温度相同。略抬高气缸左端使之倾斜,再使$ A $、$ B $升高相同温度,气体最终达到稳定状态。若始末状态$ A $、$ B $的压强变化量$ \triangle p_A $、$ \triangle p_B $均大于零,对活塞压力的变化量为$ \triangle F_A $、$ \triangle F_B $,则 \xzanswer{AD} 
\begin{figure}[h!]
\centering
\includesvg[width=0.23\linewidth]{picture/svg/276}
\end{figure}


\fourchoices
{$ A $体积增大 }
{$ A $体积减小}
{$ \triangle F_A> \triangle F_B $ }
{$ \triangle p_A< \triangle p_B $}

\item 
\exwhere{$ 2015 $年广东卷}
下图为某实验器材的结构示意图,金属内筒和隔热外筒间封闭了一定体积的空气,内筒中有水,在水加热升温的过程中,被封闭的空气 \xzanswer{AB} 
\begin{figure}[h!]
\centering
\includesvg[width=0.23\linewidth]{picture/svg/277}
\end{figure}

\fourchoices
{内能增大}
{压强增大}
{分子间引力和斥力都减小}
{所有分子运动速率都增大}






\end{enumerate}


