
\bta{选修模块 $ 3 $—$ 5 $(下)}


\begin{enumerate}
	%\renewcommand{\labelenumi}{\arabic{enumi}.}
	% A(\Alph) a(\alph) I(\Roman) i(\roman) 1(\arabic)
	%设定全局标号series=example	%引用全局变量resume=example
	%[topsep=-0.3em,parsep=-0.3em,itemsep=-0.3em,partopsep=-0.3em]
	%可使用leftmargin调整列表环境左边的空白长度 [leftmargin=0em]
	\item
\exwhere{$ 2014 $ 年物理海南卷}
\begin{enumerate}
	%\renewcommand{\labelenumi}{\arabic{enumi}.}
	% A(\Alph) a(\alph) I(\Roman) i(\roman) 1(\arabic)
	%设定全局标号series=example	%引用全局变量resume=example
	%[topsep=-0.3em,parsep=-0.3em,itemsep=-0.3em,partopsep=-0.3em]
	%可使用leftmargin调整列表环境左边的空白长度 [leftmargin=0em]
	\item
在光电效应实验中,用同一种单色光,先后照射锌和银的表面,都能发生光电效应。
对于这两个过程,下列四个物理过程中,一定不同的是 \xzanswer{ACD} 

\fourchoices
{遏止电压}
{饱和光电流}
{光电子的最大初动能}
{逸出功}


\item 
一静止原子核发生$ \alpha $衰变,生成一$ \alpha $粒子及一新核,$ \alpha $粒子垂直进入磁感应强度大小为
$ B $的匀强磁场,其运动轨迹是半径为 $ R $ 的圆。已知$ \alpha $粒子的质量为 $ m $,电荷量为 $ q $;新核的质量为
$ M $;光在真空中的速度大小为 $ c $。求衰变前原子核的质量。

\banswer{
	$M_{0}=(M+m)\left[1+\frac{(q B R)^{2}}{2 M m c^{2}}\right]$
}


	
\end{enumerate}

\item 
\exwhere{$ 2014 $ 年理综福建卷}
\begin{enumerate}
	%\renewcommand{\labelenumi}{\arabic{enumi}.}
	% A(\Alph) a(\alph) I(\Roman) i(\roman) 1(\arabic)
	%设定全局标号series=example	%引用全局变量resume=example
	%[topsep=-0.3em,parsep=-0.3em,itemsep=-0.3em,partopsep=-0.3em]
	%可使用leftmargin调整列表环境左边的空白长度 [leftmargin=0em]
	\item
如图,放射性元素镭衰变过程中释放出$ \alpha $、$ \beta $、$ \gamma $三种射线,分别进入匀强电场和匀强磁场中,
下列说法正确的是 \underlinegap 。(填选项前的字母)
\begin{figure}[h!]
	\centering
 \includesvg[width=0.23\linewidth]{picture/svg/GZ-3-tiyou-1616} 
 \hfil
 \includesvg[width=0.23\linewidth]{picture/svg/GZ-3-tiyou-1617} 
\end{figure}

\fourchoices
{①表示$ \gamma $射线, ③表示$ \alpha $射线}
{②表示$ \beta $射线,③表示$ \alpha $射线}
{④表示$ \alpha $射线,⑤表示$ \gamma $射线}
{⑤表示$ \beta $射线,⑥表示$ \alpha $射线}


 \tk{C} 
 
\item 
一枚火箭搭载着卫星以速率 $ v_{0} $ 进入太空预定位置,由控制系统使箭体与卫星分离。已知前部
分的卫星质量为 $ m_{1} $,后部分的箭体质量为 $ m_{2} $,分离后箭体以速率 $ v_{2} $ 沿火箭原方向飞行,若忽略空
气阻力及分离前后系统质量的变化,则分离后卫星的速率 $ v_{1} $ 为 \underlinegap 。(填选项前的字母)
\begin{figure}[h!]
	\centering
	\includesvg[width=0.23\linewidth]{picture/svg/GZ-3-tiyou-1618}
\end{figure}


\fourchoices
{$ v_{0}-v_{2}$}
{$ v_{0}+v_{2}$}
{$ v_{0}-\frac{m_{2}}{m_{1}} v_{2} $}
{$ v_{0}+\frac{m_{2}}{m_{1}}\left(v_{0}-v_{2}\right)$}


 \tk{D} 


\end{enumerate}


\item 
\exwhere{$ 2013 $ 年新课标  \lmd{1}  卷}
\begin{enumerate}
	%\renewcommand{\labelenumi}{\arabic{enumi}.}
	% A(\Alph) a(\alph) I(\Roman) i(\roman) 1(\arabic)
	%设定全局标号series=example	%引用全局变量resume=example
	%[topsep=-0.3em,parsep=-0.3em,itemsep=-0.3em,partopsep=-0.3em]
	%可使用leftmargin调整列表环境左边的空白长度 [leftmargin=0em]
	\item
一质子束入射到能止靶核$ ^{27}_{13}Al $上,产生如下核反应:
$ P + ^{27}_{13}Al \rightarrow X + n $
式中 $ P $ 代表质子,$ n $ 代表中子,$ X $ 代表核反应产生的新核.由反应式可知,新核 $ X $ 的质子数
为 \underlinegap ,中子数为 \underlinegap 。


 \tk{$ 14 $ \quad $ 13 $} 

\item 
在粗糙的水平桌面上有两个静止的木块 $ A $ 和 $ B $,两者相距为 $ d $。现给 $ A $ 一初速度,使 $ A $ 与
$ B $ 发生弹性正碰,碰撞时间极短。当两木块都停止运动后,相距仍然为 $ d $。已知两木块与桌面之间
的动摩擦因数均为$ \mu $, $ B $ 的质量为 $ A $ 的 $ 2 $ 倍,重力加速度大小为 $ g $。求 $ A $ 的初速度的大小。

\banswer{
	$v_{0}=\sqrt{\frac{28}{5} \mu g d}$
}



\end{enumerate}


\item 
\exwhere{$ 2013 $ 年新课标  \lmd{2}  卷}
\begin{enumerate}
	%\renewcommand{\labelenumi}{\arabic{enumi}.}
	% A(\Alph) a(\alph) I(\Roman) i(\roman) 1(\arabic)
	%设定全局标号series=example	%引用全局变量resume=example
	%[topsep=-0.3em,parsep=-0.3em,itemsep=-0.3em,partopsep=-0.3em]
	%可使用leftmargin调整列表环境左边的空白长度 [leftmargin=0em]
	\item
关于原子核的结合能,下列说法正确的是 \underlinegap (填正确答案标号。选
对 $ 1 $ 个得 $ 2 $ 分,选对 $ 2 $ 个得 $ 4 $ 分,选对 $ 3 $ 个得 $ 5 $ 分;每选错 $ 1 $ 个扣 $ 3 $ 分,最低得分为 $ 0 $ 分)。

\fivechoices
{原子核的结合能等于使其完全分解成自由核子所需的最小能量}
{一重原子核衰变成$ \alpha $粒子和另一原子核,衰变产物的结合能之和一定大于原来重核的结合能}
{铯原子核$(^{133}_{55}Cs) $的结合能小于铅原子核$ (^{208}_{82}Pb) $的结合能}
{比结合能越大,原子核越不稳定}
{自由核子组成原子核时,其质量亏损所对应的能量大于该原子核的结合能}


 \tk{ABC} 

\item 
如图,光滑水平直轨道上有三个质量均为 $ m $ 的物块 $ A $、$ B $、$ C $。$ B $ 的左侧固定一轻弹簧
(弹簧左侧的挡板质最不计)。设 $ A $ 以速度 $ v_{0} $ 朝 $ B $ 运动,压缩弹簧;当 $ A $、$ B $ 速度相等时,$ B $ 与 $ C $
恰好相碰并粘接在一起,然后继续运动。假设 $ B $
和 $ C $ 碰撞过程时间极短。求从 $ A $ 开始压缩弹簧直
至与弹簧分离的过程中:
\begin{enumerate}
	%\renewcommand{\labelenumi}{\arabic{enumi}.}
	% A(\Alph) a(\alph) I(\Roman) i(\roman) 1(\arabic)
	%设定全局标号series=example	%引用全局变量resume=example
	%[topsep=-0.3em,parsep=-0.3em,itemsep=-0.3em,partopsep=-0.3em]
	%可使用leftmargin调整列表环境左边的空白长度 [leftmargin=0em]
	\item
整个系统损失的机械能;
\item 
弹簧被压缩到最短时的弹性势能。
	
\end{enumerate}
\begin{figure}[h!]
	\flushright
	\includesvg[width=0.25\linewidth]{picture/svg/GZ-3-tiyou-1619}
\end{figure}

\banswer{
	\begin{enumerate}
		%\renewcommand{\labelenumi}{\arabic{enumi}.}
		% A(\Alph) a(\alph) I(\Roman) i(\roman) 1(\arabic)
		%设定全局标号series=example	%引用全局变量resume=example
		%[topsep=-0.3em,parsep=-0.3em,itemsep=-0.3em,partopsep=-0.3em]
		%可使用leftmargin调整列表环境左边的空白长度 [leftmargin=0em]
		\item
		$\Delta E=\frac{1}{16} m v_{0}^{2}$
		\item 
		$E_{P}=\frac{13}{48} m v_{0}^{2}$
	\end{enumerate}
}


\end{enumerate}


\item 
\exwhere{$ 2013 $ 年重庆卷}
\begin{enumerate}
	%\renewcommand{\labelenumi}{\arabic{enumi}.}
	% A(\Alph) a(\alph) I(\Roman) i(\roman) 1(\arabic)
	%设定全局标号series=example	%引用全局变量resume=example
	%[topsep=-0.3em,parsep=-0.3em,itemsep=-0.3em,partopsep=-0.3em]
	%可使用leftmargin调整列表环境左边的空白长度 [leftmargin=0em]
	\item
一列简谐波沿直线传播,某时刻该列波上正好经过平衡位置的两质点相距 $ 6 \ m $,且这
两质点之间的波峰只有一个,则该简谐波可能的波长为 \xzanswer{C} 

\fourchoices
{$ 4 \ m $、$ 6 \ m $ 和 $ 8 \ m $}
{$ 6 \ m $、$ 8 \ m $ 和 $ 12 \ m $}
{$ 4 \ m $、$ 6 \ m $ 和 $ 12 \ m $}
{$ 4 \ m $、$ 8 \ m $ 和 $ 12 \ m $}



\item 
利用半圆柱形玻璃,可减小激光束的发散
程度。在如图所示的光路中,$ A $ 为激光的出射点,$ O $ 为
半圆柱形玻璃横截面的圆心,$ AO $ 过半圆顶点。若某条从
$ A $点发出的与 $ AO $ 成$ \alpha $角的光线,以入射角 $ i $ 入射到半圆
弧上,出射光线平行于 $ AO $,求此玻璃的折射率。
\begin{figure}[h!]
	\flushright
	\includesvg[width=0.25\linewidth]{picture/svg/GZ-3-tiyou-1620}
\end{figure}


\banswer{
	$n=\frac{\sin i}{\sin (i-\alpha)}$
}


\end{enumerate}


\item 
\exwhere{$ 2013 $年江苏卷}
\begin{enumerate}
	%\renewcommand{\labelenumi}{\arabic{enumi}.}
	% A(\Alph) a(\alph) I(\Roman) i(\roman) 1(\arabic)
	%设定全局标号series=example	%引用全局变量resume=example
	%[topsep=-0.3em,parsep=-0.3em,itemsep=-0.3em,partopsep=-0.3em]
	%可使用leftmargin调整列表环境左边的空白长度 [leftmargin=0em]
	\item
如果一个电子的德布罗意波长和一个中子的相等,则它们的 \underlinegap 也相等.

\fourchoices
{速度}
{动能}
{动量}
{总能量}

 \tk{C} 

\item 
根据玻尔原子结构理论,氦离子$ (He^{+}) $的能级图如图所
示. 电子处在$ n=3 $ 轨道上比处在$ n=5 $ 轨道上离氦核的距离 \underlinegap (选填“近”或“远”). 当大量$ He^{+} $处在$ n=4 $ 的激发态时,由于
跃迁所发射的谱线有 \underlinegap 条.
\begin{figure}[h!]
	\flushright
	\includesvg[width=0.25\linewidth]{picture/svg/GZ-3-tiyou-1621}
\end{figure}

 \tk{近 \quad 6} 


\item 
如图所示,进行太空行走的宇
航员$ A $和$ B $的质量分别为$ 80 \ kg $和$ 100 \ kg $,
他们携手远离空间站,相对空间站的速
度为$ 0.1 \ m /s.A $ 将$ B $向空间站方向轻推
后, $ A $ 的速度变为$ 0.2 \ m /s $,求此时$ B $ 的速度大小和方向.
\begin{figure}[h!]
	\flushright
\begin{subfigure}{0.4\linewidth}
	\centering
	\includesvg[width=0.7\linewidth]{picture/svg/GZ-3-tiyou-1622} 
	\caption{}\label{}
\end{subfigure}
\begin{subfigure}{0.4\linewidth}
	\centering
	\includesvg[width=0.7\linewidth]{picture/svg/GZ-3-tiyou-1623} 
	\caption{}\label{}
\end{subfigure}
\end{figure}

 \tk{$ v_{B}=0.02 \ m/s  $;离开空间站方向} 
	

\end{enumerate}




\item 
\exwhere{$ 2013 $ 年山东卷}
恒星向外辐射的能量来自于其内部发生的各种热核反应在,当温度达到 $ 10^{8} \ K $ 时,可以发生
“氦燃烧”。
\begin{enumerate}
	%\renewcommand{\labelenumi}{\arabic{enumi}.}
	% A(\Alph) a(\alph) I(\Roman) i(\roman) 1(\arabic)
	%设定全局标号series=example	%引用全局变量resume=example
	%[topsep=-0.3em,parsep=-0.3em,itemsep=-0.3em,partopsep=-0.3em]
	%可使用leftmargin调整列表环境左边的空白长度 [leftmargin=0em]
	\item
完成“氦燃烧”的核反应方程: $ ^{4}_{2}He+ $  \underlinegap  $ \rightarrow ^{8}_{4}Be+ \gamma $。



\item 
$ ^{8}_{4}Be $ 是一种不稳定的粒子,其半衰期为 $ 2.6 \times 10^{-16} \ s $。一定质量的 $ ^{8}_{4}Be $ 经 $ 7.8 \times 10^{-16} \ s $ 后所剩 $ ^{8}_{4}Be $
占开始时的 \underlinegap 。
\end{enumerate}

 \tk{
\begin{enumerate}
	%\renewcommand{\labelenumi}{\arabic{enumi}.}
	% A(\Alph) a(\alph) I(\Roman) i(\roman) 1(\arabic)
	%设定全局标号series=example	%引用全局变量resume=example
	%[topsep=-0.3em,parsep=-0.3em,itemsep=-0.3em,partopsep=-0.3em]
	%可使用leftmargin调整列表环境左边的空白长度 [leftmargin=0em]
	\item
	$ ^{4}_{2}He $或$ \alpha $
	\item 
	$  \frac{ 1 }{ 8 }  $或$ 12.5 \% $
\end{enumerate}
} 




\item 
\exwhere{$ 2013 $ 年海南卷}
\begin{enumerate}
	%\renewcommand{\labelenumi}{\arabic{enumi}.}
	% A(\Alph) a(\alph) I(\Roman) i(\roman) 1(\arabic)
	%设定全局标号series=example	%引用全局变量resume=example
	%[topsep=-0.3em,parsep=-0.3em,itemsep=-0.3em,partopsep=-0.3em]
	%可使用leftmargin调整列表环境左边的空白长度 [leftmargin=0em]
	\item
原子核 $ ^{232}_{90} Th $ 具有天然放射性,它经过若干次$ \alpha $衰变和$ \beta $衰变后会变成新的原子核。下
列原子核中,有三种是 $ ^{232}_{90} Th $ 衰变过程中可以产生的,它们是
 \underlinegap 
(填正确答案标号。选对 $ 1 $ 个得 $ 2 $
分,选对 $ 2 $ 个得 $ 3 $ 分.选对 $ 3 $ 个得 $ 4 $ 分;每选错 $ 1 $ 个扣 $ 2 $ 分,最低得分为 $ 0 $ 分)




\fivechoices
{${ }_{82}^{204} Pb$}
{${ }_{82}^{203} Pb$}
{${ }_{84}^{210} Po$}
{${ }_{88}^{224} Ra$}
{${ }_{88}^{226} Ra$}


 \tk{ACD} 


\item 
如图,光滑水平面上有三个物块 $ A $、$ B $ 和 $ C $,它们
具有相同的质量,且位于同一直线上。开始时,三个物块均静
止,先让 $ A $ 以一定速度与 $ B $ 碰撞,碰后它们粘在一起,然后又一起与 $ C $ 碰撞并粘在一起,求前后
两次碰撞中损失的动能之比。
\begin{figure}[h!]
	\flushright
	\includesvg[width=0.25\linewidth]{picture/svg/GZ-3-tiyou-1624}
\end{figure}


\banswer{
	$\frac{\Delta E_{1}}{\Delta E_{2}}=3$
}



\item 
如图所示,光滑水平轨道上放置长板 $ A $(上表面粗糙)和滑块 $ C $,滑块 $ B $ 置于 $ A $ 的左端。三
者质量分别为 $ m_{A} =2 \ kg $、$ m_{B} =1 \ kg $、$ m_{C} =2 \ kg $。开始时 $ C $ 静止,$ A $、$ B $ 一起以 $ v_{0} =5 \ m /s $ 的速度匀速向右
运动,$ A $ 与 $ C $ 发生碰撞(时间极短)后 $ C $ 向右
运动,经过一段时间,$ A $、$ B $ 再次达到共同速
度一起向右运动,且恰好不再与 $ C $ 碰撞。求 $ A $
与 $ C $ 发生碰撞后瞬间 $ A $ 的速度大小。
\begin{figure}[h!]
	\flushright
	\includesvg[width=0.25\linewidth]{picture/svg/GZ-3-tiyou-1625}
\end{figure}

 \tk{$ 2 \ m/s  $} 

\end{enumerate}


\item 
\exwhere{$ 2013 $ 年福建卷}
\begin{enumerate}
	%\renewcommand{\labelenumi}{\arabic{enumi}.}
	% A(\Alph) a(\alph) I(\Roman) i(\roman) 1(\arabic)
	%设定全局标号series=example	%引用全局变量resume=example
	%[topsep=-0.3em,parsep=-0.3em,itemsep=-0.3em,partopsep=-0.3em]
	%可使用leftmargin调整列表环境左边的空白长度 [leftmargin=0em]
	\item
在卢瑟福$ \alpha $粒子散射实验中,金箔中的原子核可以看作静止不动,下列各图画出的是其中两个$ \alpha $
粒子经历金箔散射过程的径迹,其中正确的是 \underlinegap 。(填选图下方的字母)


\pfourchoices
{\includesvg[width=4.3cm]{picture/svg/GZ-3-tiyou-1626}}
{\includesvg[width=4.3cm]{picture/svg/GZ-3-tiyou-1627}}
{\includesvg[width=4.3cm]{picture/svg/GZ-3-tiyou-1628}}
{\includesvg[width=4.3cm]{picture/svg/GZ-3-tiyou-1629}}

 \tk{C} 

\item 
将静置在地面上,质量为 $ M $(含燃料)的火箭模型点火升空,在极短时间内以相对地面的速度
$ v_{0} $ 竖直向下喷出质量为 $ m $ 的炽热气体。忽略喷气过程重力和空气阻力的影响,则喷气结束时火箭
模型获得的速度大小是
 \underlinegap 
。(填选项前的字母)

\fourchoices
{$\frac{m}{M} v_{0}$}
{$\frac{M}{m} v_{0}$}
{$\frac{M}{M-m} v_{0}$}
{$\frac{m}{M-m} v_{0}$}

 \tk{D} 
	
\end{enumerate}


\item 
\exwhere{$ 2012 $ 年理综新课标卷}
\begin{enumerate}
	%\renewcommand{\labelenumi}{\arabic{enumi}.}
	% A(\Alph) a(\alph) I(\Roman) i(\roman) 1(\arabic)
	%设定全局标号series=example	%引用全局变量resume=example
	%[topsep=-0.3em,parsep=-0.3em,itemsep=-0.3em,partopsep=-0.3em]
	%可使用leftmargin调整列表环境左边的空白长度 [leftmargin=0em]
	\item
氘核和氚核可发生热核聚变而释放巨大的能量,该反应方程为:${ }_{1}^{2} H+{ }_{1}^{3} H \rightarrow{ }_{2}^{4} He+x$,式
中 $ x $ 是 某 种 粒 子 。 已 知 : ${ }_{1}^{2} H $、$ { }_{1}^{3} H $、$ { }_{2}^{4} He$ 和 粒 子 $ x $ 的 质 量 分 别 为 $ 2.0141 \ u $、 $ 3.0161 \ u $、 $ 4.0026 \ u $ 和
$ 1. 0087 \ u $; $ 1 \ u=931.5 \ MeV/c^{2} $, $ c $ 是 真 空 中 的 光 速 。 由 上 述 反 应 方 程 和 数 据 可 知 , 粒 子 $ x $ 是 \underlinegap 
,该反应释放出的能量为  \underlinegap  $ MeV $(结果保留 $ 3 $ 位有效数字)


 \tk{ $ ^{1}_{0}n $ (或中子) \quad  $ 17.6 $} 

\item 
如图,小球 $ a $、$ b $ 用等长细线悬挂于同一固定点 $ O $。让球 $ a $ 静
止下垂,将球 $ b $ 向右拉起,使细线水平。从静止释放球 $ b $,两球碰后粘在
一起向左摆动,此后细线与竖直方向之间的最大偏角为 $ 60 \degree $。忽略空气阻
力,求:
\begin{enumerate}
	%\renewcommand{\labelenumi}{\arabic{enumi}.}
	% A(\Alph) a(\alph) I(\Roman) i(\roman) 1(\arabic)
	%设定全局标号series=example	%引用全局变量resume=example
	%[topsep=-0.3em,parsep=-0.3em,itemsep=-0.3em,partopsep=-0.3em]
	%可使用leftmargin调整列表环境左边的空白长度 [leftmargin=0em]
	\item
两球 $ a $、$ b $ 的质量之比;
\item 
两球在碰撞过程中损失的机械能与球 $ b $ 在碰前的最大动能之比。
\end{enumerate}
\begin{figure}[h!]
	\flushright
	\includesvg[width=0.25\linewidth]{picture/svg/GZ-3-tiyou-1630}
\end{figure}

\banswer{
	\begin{enumerate}
		%\renewcommand{\labelenumi}{\arabic{enumi}.}
		% A(\Alph) a(\alph) I(\Roman) i(\roman) 1(\arabic)
		%设定全局标号series=example	%引用全局变量resume=example
		%[topsep=-0.3em,parsep=-0.3em,itemsep=-0.3em,partopsep=-0.3em]
		%可使用leftmargin调整列表环境左边的空白长度 [leftmargin=0em]
		\item
		$\frac{M}{m}=\sqrt{2}-1$
		\item 
		$\frac{\Delta E}{E_{k b}}=\frac{2-\sqrt{2}}{2}$
	\end{enumerate}
}

	
\end{enumerate}


\item 
\exwhere{$ 2012 $年理综山东卷}
\begin{enumerate}
	%\renewcommand{\labelenumi}{\arabic{enumi}.}
	% A(\Alph) a(\alph) I(\Roman) i(\roman) 1(\arabic)
	%设定全局标号series=example	%引用全局变量resume=example
	%[topsep=-0.3em,parsep=-0.3em,itemsep=-0.3em,partopsep=-0.3em]
	%可使用leftmargin调整列表环境左边的空白长度 [leftmargin=0em]
	\item
氢原子第$ n $能级的能量为 $E_{n}=\frac{E_{1}}{n^{2}}$,其中$ E_{1} $为基态能量。当氢原子由第$ 4 $能级跃迁到第$ 2 $能级
$ n_{2} $

时,发出光子的频率为$ \nu_{1} $;若氢原子由第$ 2 $能级跃迁到基态,发出光子的频率为$ \nu_{2} $,则
$\frac{\nu_{1}}{\nu_{2}}=$ \underlinegap 。

 \tk{$  \frac{ 1 }{ 4 }  $} 

\item 
光滑水平轨道上有三个木块$ A $、$ B $、$ C $,质量分别为$ m_{A} =3  m $、$ m_{B} = m_{C} =m $,开始时$ B $、$ C $均静止,
$ A $以初速度$ v_{0} $向右运动,$ A $与$ B $相撞后分开,$ B $又与$ C $发生碰
撞并粘在一起,此后$ A $与$ B $间的距离保持不变。求$ B $与$ C $碰
撞前$ B $的速度大小。
\begin{figure}[h!]
	\flushright
	\includesvg[width=0.25\linewidth]{picture/svg/GZ-3-tiyou-1631}
\end{figure}

\banswer{
	$v_{B}=\frac{6}{5} v_{0}$
}

	
\end{enumerate}



\item 
\exwhere{$ 2012 $ 年物理江苏卷}
\begin{enumerate}
	%\renewcommand{\labelenumi}{\arabic{enumi}.}
	% A(\Alph) a(\alph) I(\Roman) i(\roman) 1(\arabic)
	%设定全局标号series=example	%引用全局变量resume=example
	%[topsep=-0.3em,parsep=-0.3em,itemsep=-0.3em,partopsep=-0.3em]
	%可使用leftmargin调整列表环境左边的空白长度 [leftmargin=0em]
	\item
如图所示是某原子的能级图, $ a $、$ b $、$ c $ 为原子跃迁所发出的三种波长
的光. 在下列该原子光谱的各选项中,谱线从左向右的波长依次增大,则
正确的是 \underlinegap 。
\begin{figure}[h!]
	\centering
	\includesvg[width=0.23\linewidth]{picture/svg/GZ-3-tiyou-1632}
\end{figure}


\pfourchoices
{\includesvg[width=4.3cm]{picture/svg/GZ-3-tiyou-1633}}
{\includesvg[width=4.3cm]{picture/svg/GZ-3-tiyou-1634}}
{\includesvg[width=4.3cm]{picture/svg/GZ-3-tiyou-1635}}
{\includesvg[width=4.3cm]{picture/svg/GZ-3-tiyou-1636}}



 \tk{C} 


\item 
一个中子与某原子核发生核反应,生成一个氘核,其核反应方程式为 \underline{\hbox to 30mm{}} . 该反
应放出的能量为 $ Q $,则氘核的比结合能为 \underlinegap .

 \tk{${ }_{0}^{1} n+{ }_{1}^{1} H \rightarrow{ }_{1}^{2} H ; \quad \frac{Q}{2}$} 

\item 
$ A $、$ B $ 两种光子的能量之比为 $ 2:1 $,它们都能使某种金属发生光电效应,且所产生的光电子最大
初动能分别为 $ E_{A} $、$ E_{B} $. 求 $ A $、$ B $ 两种光子的动量之比和该金属的逸出功.

\banswer{
	$p_{A}: \quad p_{B}=2: 1$ \quad $W_{0}=E_{A}-2 E_{B}$
}

	
\end{enumerate}



\item 
\exwhere{$ 2012 $ 年理综福建卷}
\begin{enumerate}
	%\renewcommand{\labelenumi}{\arabic{enumi}.}
	% A(\Alph) a(\alph) I(\Roman) i(\roman) 1(\arabic)
	%设定全局标号series=example	%引用全局变量resume=example
	%[topsep=-0.3em,parsep=-0.3em,itemsep=-0.3em,partopsep=-0.3em]
	%可使用leftmargin调整列表环境左边的空白长度 [leftmargin=0em]
	\item
关于近代物理,下列说法正确的是 \underlinegap 。(填选项前的字母)
\fourchoices
{$ \alpha $ 射线是高速运动的氦原子}
{核聚变反应方程 ${ }_{0}^{1} n{ }_{1}^{2} H+{ }_{1}^{3} H \rightarrow{ }_{2}^{4} H e+{ }_{0}^{1} n$ 中 $,{ }_{0}^{1} n$ 表示质子}
{从金属表面逸出的光电子的最大初动能与照射光的频率成正比}
{玻尔将量子观念引入原子领域,其理论能够解释氢原子光谱的特征}

 \tk{D} 

\item 
如图,质量为 $ M $ 的小船在静止水面上以速率 $ v_{0} $ 向右匀速行驶,一质量为 $ m $ 的救生员站在船
尾,相对小船静止。若救生员以相对水面速率 $ v $ 水平向左跃入水中,则救生员跃出后小船的速率为 \underlinegap 。(填选项前的字母)
\begin{figure}[h!]
	\centering
	\includesvg[width=0.23\linewidth]{picture/svg/GZ-3-tiyou-1637}
\end{figure}
\fourchoices
{$v_{0}+\frac{m}{M} v$}
{$v_{0}-\frac{m}{M} v$}
{$v_{0}+\frac{m}{M}\left(v_{0}+v\right)$}
{$v_{0}+\frac{m}{M}\left(v_{0}+v\right)$}

 \tk{A} 


\end{enumerate}


\item 
\exwhere{$ 2012 $ 年物理海南卷}
\begin{enumerate}
	%\renewcommand{\labelenumi}{\arabic{enumi}.}
	% A(\Alph) a(\alph) I(\Roman) i(\roman) 1(\arabic)
	%设定全局标号series=example	%引用全局变量resume=example
	%[topsep=-0.3em,parsep=-0.3em,itemsep=-0.3em,partopsep=-0.3em]
	%可使用leftmargin调整列表环境左边的空白长度 [leftmargin=0em]
	\item
产生光电效应时,关于逸出光电子的最大初动能 $ E_{k} $,下列说法正确的是 \underlinegap 
(填入正确选项前的字母。选对 $ 1 $ 个给 $ 2 $ 分,选对 $ 2 $ 个给 $ 3 $ 分,选对 $ 3 $ 个给 $ 4 $
分;每选错 $ 1 $ 个扣 $ 2 $ 分,最低得分为 $ 0 $ 分)。
\fivechoices
{对于同种金属,$ E_{k} $ 与照射光的强度无关}
{对于同种金属,$ E_{k} $ 与照射光的波长成正比}
{对于同种金属,$ E_{k} $ 与照射光的时间成正比}
{对于同种金属,$ E_{k} $ 与照射光的频率成线性关系}
{对于不同种金属,若照射光频率不变,$ E_{k} $ 与金属的逸出功成线性关系}



 \tk{ADE} 
 
 
\item 
一静止的 ${ }_{92}^{238} U$ 核经$ \alpha $衰变成为 ${ }_{90}^{234} Th$,释放出的总动能为 $ 4.27 \ MeV $。问此衰变后 ${ }_{90}^{234} Th$
核的动能为多少 $ MeV $(保留 $ 1 $ 位有效数字)?


\banswer{
	$\frac{1}{2} m_{T h} v_{T h}^{2}=0.07 \ MeV$
}




\end{enumerate}


\item 
\exwhere{$ 2011 $ 年新课标卷}
\begin{enumerate}
	%\renewcommand{\labelenumi}{\arabic{enumi}.}
	% A(\Alph) a(\alph) I(\Roman) i(\roman) 1(\arabic)
	%设定全局标号series=example	%引用全局变量resume=example
	%[topsep=-0.3em,parsep=-0.3em,itemsep=-0.3em,partopsep=-0.3em]
	%可使用leftmargin调整列表环境左边的空白长度 [leftmargin=0em]
	\item
在光电效应试验中,某金属的截止频率相应的波长为$ \lambda _{0} $,该金属的逸出功为 \underlinegap 。若
用波长为$ \lambda $($ \lambda < \lambda _{0} $)的单色光做该实验,则其遏止电压为 \underlinegap 。已知电子的电荷量、真空中的光
速和布朗克常量分别为 $ e $、$ c $ 和 $ h $。

 \tk{$W_{\text {逸 }}=h \frac{c}{\lambda_{0}}$ \quad $U_{\text {截止 }}=\frac{h c}{e} \frac{\lambda_{0}-\lambda}{\lambda_{0} \lambda}$} 


\item 
如图,$ ABC $ 三个木块的质量均为 $ m $,置于光滑的水平面上,$ B $、$ C $ 之间有一轻质弹簧,
弹簧的两端与木块接触而不固连,将弹簧压紧到不能再压
缩时用细线把 $ B $ 和 $ C $ 紧连,使弹簧不能伸展,以至于 $ B $、
$ C $可视为一个整体。现 $ A $ 以初速 $ v_{0} $ 沿 $ B $、$ C $ 的连线方向朝 $ B $ 运动,与 $ B $ 相碰并粘合在一起,以后
细线突然断开,弹簧伸展,从而使 $ C $ 与 $ A $,$ B $ 分离,已知 $ C $ 离开弹簧后的速度恰为 $ v_{0} $,求弹簧释
放的势能。
\begin{figure}[h!]
	\flushright
	\includesvg[width=0.25\linewidth]{picture/svg/GZ-3-tiyou-1638}
\end{figure}

\banswer{
	$E_{p}=\frac{1}{3} m v_{0}^{2}$
}


\end{enumerate}


\item 
\exwhere{$ 2011 $ 年理综福建卷}
\begin{enumerate}
	%\renewcommand{\labelenumi}{\arabic{enumi}.}
	% A(\Alph) a(\alph) I(\Roman) i(\roman) 1(\arabic)
	%设定全局标号series=example	%引用全局变量resume=example
	%[topsep=-0.3em,parsep=-0.3em,itemsep=-0.3em,partopsep=-0.3em]
	%可使用leftmargin调整列表环境左边的空白长度 [leftmargin=0em]
	\item
爱因斯坦提出了光量子概念并成功地解释光电效应的规律而获得
$ 1921 $ 年的诺贝尔物理学奖。某种金属逸出光电子的最大初动能 $ E_{km} $ 与
入射光频率$ \nu $的关系如图所示,其中$ \nu _{0} $ 为极限频率。从图中可以确定的
是 \underlinegap 。(填选项前的字母)
\fourchoices
{逸出功与$ \nu $有关}
{$ E_{km} $ 与入射光强度成正比}
{$ \nu < \nu _{0} $ 时,会逸出光电子}
{图中直线的斜率与普朗克常量有关}


 \tk{D} 

\item 
在光滑水平面上,一质量为 $ m $、速度大小为 $ v $ 的 $ A $ 球与质量为 $ 2 \ m $ 静止的 $ B $ 球碰撞后,$ A $ 球的速
度方向与碰撞前相反。则碰撞后 $ B $ 球的速度大小可能是 \underlinegap 。(题选项前的字母)
\begin{figure}[h!]
	\centering
	\includesvg[width=0.23\linewidth]{picture/svg/GZ-3-tiyou-1639}
\end{figure}

\fourchoices
{$ 0.6v $}
{$ 0.4v $}
{$ 0.3v $}
{$ 0.2v $}

 \tk{A} 




	
\end{enumerate}



\item
\exwhere{$ 2011 $ 年海南卷}
\begin{enumerate}
	%\renewcommand{\labelenumi}{\arabic{enumi}.}
	% A(\Alph) a(\alph) I(\Roman) i(\roman) 1(\arabic)
	%设定全局标号series=example	%引用全局变量resume=example
	%[topsep=-0.3em,parsep=-0.3em,itemsep=-0.3em,partopsep=-0.3em]
	%可使用leftmargin调整列表环境左边的空白长度 [leftmargin=0em]
	\item
$ 2011 $ 年 $ 3 $ 月 $ 11 $ 日,日本发生九级大地震,造成福岛核电站的核泄漏事故。在泄露的
污染物中含有 $ ^{131}I $ 和 $ ^{137}Cs $ 两种放射性核素,它们通过一系列衰变产生对人体有危害的辐射。在下
列四个式子中,有两个能分别反映 $ ^{131}I $ 和 $ ^{137}Cs $ 衰变过程,它们分别是 \underlinegap 和 \underlinegap (填入
正确选项前的字母)。$ ^{131}I $ 和 $ ^{137}Cs $ 原子核中的中子数分别是 \underlinegap 和 \underlinegap .


\fourchoices
{$X_{1} \rightarrow{ }_{56}^{137} B a+{ }_{0}^{1} n$}
{$X_{2} \rightarrow_{54}^{131} X e+{ }_{-1}^{0} e$}
{$X_{3} \rightarrow{ }_{56}^{137} B a+{ }_{-1}^{0} e$}
{$X_{4} \rightarrow{ }_{54}^{131} X e+{ }_{1}^{1} P$}


 \tk{B、C;78、82}
 
  
\item 
一质量为 $ 2 \ m $ 的物体 $ P $ 静止于光滑水平地面上,其截面如图所示。图中 $ ab $ 为粗糙的水
平面,长度为 $ L $;$ bc $ 为一光滑斜面,斜面和水平面通过
与 $ ab $ 和 $ bc $ 均相切的长度可忽略的光滑圆弧连接。现有
一质量为 $ m $ 的木块以大小为 $ v_{0} $ 的水平初速度从 $ a $ 点向左
运动,在斜面上上升的最大高度为 $ h $,返回后在到达 $ a $ 点前与物体 $ P $ 相对静止。重力加速度为 $ g $。
求:
\begin{enumerate}
	%\renewcommand{\labelenumi}{\arabic{enumi}.}
	% A(\Alph) a(\alph) I(\Roman) i(\roman) 1(\arabic)
	%设定全局标号series=example	%引用全局变量resume=example
	%[topsep=-0.3em,parsep=-0.3em,itemsep=-0.3em,partopsep=-0.3em]
	%可使用leftmargin调整列表环境左边的空白长度 [leftmargin=0em]
	\item
木块在 $ ab $ 段受到的摩擦力 $ f $;
\item 
木块最后距 $ a $ 点的距离 $ s $。
\end{enumerate}
\begin{figure}[h!]
	\flushright
	\includesvg[width=0.25\linewidth]{picture/svg/GZ-3-tiyou-1640}
\end{figure}

\banswer{
	\begin{enumerate}
		%\renewcommand{\labelenumi}{\arabic{enumi}.}
		% A(\Alph) a(\alph) I(\Roman) i(\roman) 1(\arabic)
		%设定全局标号series=example	%引用全局变量resume=example
		%[topsep=-0.3em,parsep=-0.3em,itemsep=-0.3em,partopsep=-0.3em]
		%可使用leftmargin调整列表环境左边的空白长度 [leftmargin=0em]
		\item
		$f=\frac{m\left(v_{0}^{2}-3 g h\right)}{3 L}$
		\item 
		$s=\frac{v_{0}^{2}-6 g h}{v_{0}^{2}-3 g h} L$
	\end{enumerate}
}


	
\end{enumerate}


\item 
\exwhere{$ 2011 $ 年理综山东卷}
\begin{enumerate}
	%\renewcommand{\labelenumi}{\arabic{enumi}.}
	% A(\Alph) a(\alph) I(\Roman) i(\roman) 1(\arabic)
	%设定全局标号series=example	%引用全局变量resume=example
	%[topsep=-0.3em,parsep=-0.3em,itemsep=-0.3em,partopsep=-0.3em]
	%可使用leftmargin调整列表环境左边的空白长度 [leftmargin=0em]
	\item
碘 $ 131 $ 核不稳定,会发生$ \beta $衰变,其半衰变期为 $ 8 $ 天。
\begin{enumerate}
	%\renewcommand{\labelenumi}{\arabic{enumi}.}
	% A(\Alph) a(\alph) I(\Roman) i(\roman) 1(\arabic)
	%设定全局标号series=example	%引用全局变量resume=example
	%[topsep=-0.3em,parsep=-0.3em,itemsep=-0.3em,partopsep=-0.3em]
	%可使用leftmargin调整列表环境左边的空白长度 [leftmargin=0em]
	\item
碘 $ 131 $ 核的衰变方程: ${ }_{53}^{131} \mathrm{I} \rightarrow$  \underlinegap (衰变后的元素用 $ X $ 表示)。

\item 
经过 \underlinegap 天有 $ 75 \% $的碘 $ 131 $ 核发生了衰变。

\end{enumerate}

 \tk{
\begin{enumerate}
	%\renewcommand{\labelenumi}{\arabic{enumi}.}
	% A(\Alph) a(\alph) I(\Roman) i(\roman) 1(\arabic)
	%设定全局标号series=example	%引用全局变量resume=example
	%[topsep=-0.3em,parsep=-0.3em,itemsep=-0.3em,partopsep=-0.3em]
	%可使用leftmargin调整列表环境左边的空白长度 [leftmargin=0em]
	\item
	${ }_{54}^{131} X+{ }_{-1}^{0} e$
	\item 
	$ 16 $
\end{enumerate}
} 


\item 
如图所示,甲、乙两船的总质量(包括船、人和货物)分
别为 $ 10 m $、$ 12 m $,两船沿同一直线同一方向运动,速度分别
为 $ 2 v_{0} $、$ v_{0} $。为避免两船相撞,乙船上的人将一质量为 $ m $ 的货
物沿水平方向抛向甲船,甲船上的人将货物接住,求抛出货
物的最小速度。(不计水的阻力)
\begin{figure}[h!]
	\flushright
	\includesvg[width=0.25\linewidth]{picture/svg/GZ-3-tiyou-1641}
\end{figure}

\banswer{
	$ v_{min} = 4 v_{0} $
}


	
\end{enumerate}


\item 
\exwhere{$ 2011 $ 年物理江苏卷}
\begin{enumerate}
	%\renewcommand{\labelenumi}{\arabic{enumi}.}
	% A(\Alph) a(\alph) I(\Roman) i(\roman) 1(\arabic)
	%设定全局标号series=example	%引用全局变量resume=example
	%[topsep=-0.3em,parsep=-0.3em,itemsep=-0.3em,partopsep=-0.3em]
	%可使用leftmargin调整列表环境左边的空白长度 [leftmargin=0em]
	\item
下列描绘两种温度下黑体辐射强度与波长关系的图中,符合黑体辐射规律的是 \xzanswer{A} 
\pfourchoices
{\includesvg[width=4.3cm]{picture/svg/GZ-3-tiyou-1642}}
{\includesvg[width=4.3cm]{picture/svg/GZ-3-tiyou-1643}}
{\includesvg[width=4.3cm]{picture/svg/GZ-3-tiyou-1644}}
{\includesvg[width=4.3cm]{picture/svg/GZ-3-tiyou-1645}}

\item 
按照玻尔原子理论,氢原子中的电子离原子核越远,氢原子的能量 \underlinegap (选填“越大”或
“越小”)。已知氢原子的基态能量为 $ E_{1} $($ E_{1} <0 $),电子质量为 $ m $,基态氢原子中的电子吸收一频率
为$ \nu $的光子被电离后,电子速度大小为 \underlinegap (普朗克常量为 $ h $)。

 \tk{越大 \quad $v=\sqrt{\frac{2\left(h v+E_{1}\right)}{m}}$} 
	


\item 
有些核反应过程是吸收能量的。例如在 $X+{ }_{7}^{14} N \rightarrow{ }_{8}^{17} O+{ }_{1}^{1} H$ 中,核反应吸收的能量
$Q=\left[\left(m_{O}+m_{H}\right)-\left(m_{X}+m_{N}\right)\right] c^{2}$
,在该核反应方程中,$ X $ 表示什么粒子?$ X $ 粒子以动能 $ E_{k} $ 轰击
静止的 $ ^{14}_{7} N $,若 $ E_{k} =Q $,则该核反应能否发生?请简要说明理由。

\banswer{
	$ X $为$ ^{4}_{2}He $;不能实现,因为不能同时满足能量守恒和动量守恒的要求
}





\end{enumerate}






	
	
	
\end{enumerate}

