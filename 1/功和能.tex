\bta{功和能}

\begin{enumerate}[leftmargin=0em]
\renewcommand{\labelenumi}{\arabic{enumi}.}
% A(\Alph) a(\alph) I(\Roman) i(\roman) 1(\arabic)
%设定全局标号series=example	%引用全局变量resume=example
%[topsep=-0.3em,parsep=-0.3em,itemsep=-0.3em,partopsep=-0.3em]
%可使用leftmargin调整列表环境左边的空白长度 [leftmargin=0em]
\item
\exwhere{$ 2019 $年物理江苏卷}
如图所示,轻质弹簧的左端固定,并处于自然状态.小物块的质量为$ m $,从$ A $点向左沿水平地面运动,压缩弹簧后被弹回,运动到$ A $点恰好静止.物块向左运动的最大距离为$ s $,与地面间的动摩擦因数为$ \mu $,重力加速度为$ g $,弹簧未超出弹性限度.在上述过程中 \xzanswer{BC} 
\begin{figure}[h!]
\centering
\includesvg[width=0.3\linewidth]{picture/svg/775}
\end{figure}

\fourchoices
{弹簧的最大弹力为$ \mu mg $}
{物块克服摩擦力做的功为$ 2 \mu mgs $}
{弹簧的最大弹性势能为$ \mu mgs $}
{物块在$ A $点的初速度为$\sqrt{2 \mu gs}$}


\item
\exwhere{$ 2012 $年物理海南卷}
下列关于功和机械能的说法,正确的是 \xzanswer{BC} 

\fourchoices
{在有阻力作用的情况下,物体重力势能的减少不等于重力对物体所做的功}
{合力对物体所做的功等于物体动能的改变量}
{物体的重力势能是物体与地球之间的相互作用能,其大小与势能零点的选取有关}
{运动物体动能的减少量一定等于其重力势能的增加量}



\item
\exwhere{$ 2012 $年理综安徽卷}
如图所示,在竖直平面内有一半径为$ R $的圆弧轨道,半径$ OA $水平、$ OB $竖直,一个质量为$ m $的小球自$ A $的正上方$ P $点由静止开始自由下落,小球沿轨道到达最高点$ B $时恰好对轨道没有压力。已知$ AP=2R $,重力加速度为$ g $,则小球从$ P $到$ B $的运动过程中 \xzanswer{D} 
\begin{figure}[h!]
\centering
\includesvg[width=0.17\linewidth]{picture/svg/776}
\end{figure}

\fourchoices
{重力做功$ 2mgR $ }
{机械能减少$ mgR $}
{合外力做功$ mgR $ }
{克服摩擦力做功$ \frac{ 1 }{ 2 } mgR $}


\item
\exwhere{$ 2013 $年新课标 \lmd{2} 卷}
目前,在地球周围有许多人造地球卫星绕着它转,其中一些卫星的轨道可近似为圆,且轨道半径逐渐变小。若卫星在轨道半径逐渐变小的过程中,只受到地球引力和稀薄气体阻力的作用,则下列判断正确的是 \xzanswer{BD} 

\fourchoices
{卫星的动能逐渐减小}
{由于地球引力做正功,引力势能一定减小}
{由于气体阻力做负功,地球引力做正功,机械能保持不变}
{卫星克服气体阻力做的功小于引力势能的减小}



\item
\exwhere{$ 2014 $年物理上海卷}
静止在地面上的物体在竖直向上的恒力作用下上升,在某一高度撤去恒力。不计空气阻力,在整个上升过程中,物体机械能随时间变化关系是 \xzanswer{C} 
\begin{figure}[h!]
\centering
\includesvg[width=0.83\linewidth]{picture/svg/777}
\end{figure}




\item 
\exwhere{$ 2013 $年全国卷大纲卷}
如图,一固定斜面倾角为$ 30 ^{ \circ } $,一质量为$ m $的小物块自斜面底端以一定的初速度,沿斜面向上做匀减速运动,加速度的大小等于重力加速度的大小$ g $。若物块上升的最大高度为$ H $,则此过程中,物块的 \xzanswer{AC} 


\begin{minipage}[h!]{0.7\linewidth}
\vspace{0.3em}
\fourchoices
{动能损失了$ 2mgH $}
{动能损失了$ mgH $}
{机械能损失了$ \frac{ 1 }{ 2 } mgH $}
{机械能损失了}
\vspace{0.3em}
\end{minipage}
\hfill
\begin{minipage}[h!]{0.3\linewidth}
\flushright
\vspace{0.3em}
\includesvg[width=0.8\linewidth]{picture/svg/778}
\vspace{0.3em}
\end{minipage}



\item
\exwhere{$ 2014 $年理综广东卷}
下图是安装在列车车厢之间的摩擦缓冲器结构图,图中①和②为楔块,③和④为垫块,楔块与弹簧盒、垫块间均有摩擦,在车厢相互撞击时弹簧压缩过程中 \xzanswer{B} 
\begin{figure}[h!]
\centering
\includesvg[width=0.3\linewidth]{picture/svg/779}
\end{figure}

\fourchoices
{缓冲器的机械能守恒}
{摩擦力做功消耗机械能}
{垫块的动能全部转化成内能}
{弹簧的弹性势能全部转化为动能}




\item 
\exwhere{$ 2015 $年上海卷}
如图,光滑平行金属导轨固定在水平面上,左端由导线相连,导体棒垂直静置于导轨上构成回路。在外力$ F $作用下,回路上方的条形磁铁竖直向上做匀速运动。在匀速运动过程中外力$ F $做功$ W_F $,磁场力对导体棒做功$ W_{1} $,磁铁克服磁场力做功$ W_{2} $,重力对磁铁做功$ W_G $,回路中产生的焦耳热为$ Q $,导体棒获得的动能为$ E_{k} $。则 \xzanswer{BCD} 


\begin{minipage}[h!]{0.7\linewidth}
\vspace{0.3em}
\fourchoices
{$ W_{1} =Q $}
{$ W_{2} - W_{1} =Q $}
{$ W_{1} = E_{k} $	}
{$ W_F+W_G=Q+ E_{k} $}
\vspace{0.3em}
\end{minipage}
\hfill
\begin{minipage}[h!]{0.3\linewidth}
\flushright
\vspace{0.3em}
\includesvg[width=0.6\linewidth]{picture/svg/781}
\vspace{0.3em}
\end{minipage}





\item
\exwhere{$ 2017 $年新课标$ \lmd{3} $卷}
如图,一质量为$ m $,长度为$ l $的均匀柔软细绳$ PQ $竖直悬挂。用外力将绳的下端$ Q $缓慢地竖直向上拉起至$ M $点,$ M $点与绳的上端$ P $相距$ \frac{ 1 }{ 3 } l $。重力加速度大小为$ g $。在此过程中,外力做的功为 \xzanswer{A} 
\begin{figure}[h!]
\centering
\includesvg[width=0.12\linewidth]{picture/svg/782}
\end{figure}


\fourchoices
{$ \frac { 1 } { 9 } m g l $}
{$ \frac { 1 } { 6 } m g l $}
{$ \frac { 1 } { 3 } m g l $}
{$ \frac { 1 } { 2 } m g l $}



\item
\exwhere{$ 2017 $年浙江选考卷}
火箭发射回收是航天技术的一大进步。如图所示,火箭在返回地面前的某段运动,可看成先匀速后减速的直线运动,最后撞落在地面上。不计火星质量的变化,则 \xzanswer{D} 
\begin{figure}[h!]
\centering
\includesvg[width=0.23\linewidth]{picture/svg/783}
\end{figure}

\fourchoices
{火箭在匀速下降过程中机械能守恒}
{火箭在减速下降过程中携带的检测仪器处于失重状态}
{火箭在减速下降过程中合力做功,等于火箭机械能的变化}
{火箭着地时,火箭对地的作用力大于自身的重力}




\item 
\exwhere{$ 2015 $年理综天津卷}
如图所示,固定的竖直光滑长杆上套有质量为$ m $的小圆环,圆环与水平状态的轻质弹簧一端连接,弹簧的另一端连接在墙上,并且处于原长状态,现让圆环由静止开始下滑,已知弹簧原长为$ L $,圆环下滑到最大距离时弹簧的长度变为$ 2L $(未超过弹性限度),则在圆环下滑到最大距离的过程中 \xzanswer{B} 
\begin{figure}[h!]
\centering
\includesvg[width=0.12\linewidth]{picture/svg/784}
\end{figure}


\fourchoices
{圆环的机械能守恒}
{弹簧弹性势能变化了$\sqrt{3}mgL$}
{圆环下滑到最大距离时,所受合力为零}
{圆环重力势能与弹簧弹性势能之和保持不变}




\item
\exwhere{$ 2015 $年江苏卷}
如图所示,轻质弹簧一端固定,另一端与一质量为 $ m $、套在粗糙竖直固定杆 $ A $ 处的圆环相连,弹簧水平且处于原长。 圆环从 $ A $ 处由静止开始下滑,经过 $ B $ 处的速度最大,到达 $ C $ 处的速度为零,$ AC $ $ = $ $ h $. 圆环在 $ C $ 处获得一竖直向上的速度 $ v $,恰好能回到 A. 弹簧始终在弹性限度内,重力加速度为 $ g $. 则圆环 \xzanswer{BD} 
\begin{figure}[h!]
\centering
\includesvg[width=0.23\linewidth]{picture/svg/785}
\end{figure}


\fourchoices
{下滑过程中,加速度一直减小}
{下滑过程中,克服摩擦力做的功为$ \frac{ 1 }{ 4 } mv^{2} $}
{在 $ C $ 处,弹簧的弹性势能为 $\frac { 1 } { 4 } m v ^ { 2 } - m g h$}
{上滑经过 $ B $ 的速度大于下滑经过 $ B $ 的速度}




\item
\exwhere{$ 2015 $年理综新课标 \lmd{1} 卷}
如图,一半径为$ R $,粗糙程度处处相同的半圆形轨道竖直固定放置,直径$ POQ $水平。一质量为$ m $的质点自$ P $点上方高度$ R $处由静止开始下落,恰好从$ P $点进入轨道。质点滑到轨道最低点$ N $时,对轨道的压力为$ 4mg $,$ g $为重力加速度的大小。用$ W $表示质点从$ P $点运动到$ N $点的过程中克服摩擦力所做的功。则 \xzanswer{C} 
\begin{figure}[h!]
\centering
\includesvg[width=0.23\linewidth]{picture/svg/786}
\end{figure}


\fourchoices
{$W = \frac { 1 } { 2 } m g R$ ,质点恰好可以到达$ Q $点}
{$W > \frac { 1 } { 2 } m g R$,质点不能到达$ Q $点}
{$W = \frac { 1 } { 2 } m g R$,质点到达$ Q $后,继续上升一段距离}
{$W < \frac { 1 } { 2 } m g R$ ,质点到达$ Q $后,继续上升一段距离}





\item
\exwhere{$ 2014 $年物理海南卷}
如图,质量相同的两物体$ a $、$ b $,用不可伸长的轻绳跨接在同一光滑的轻质定滑轮两侧,$ a $在水平桌面的上方,$ b $在水平粗糙桌面上。初始时用力压住$ b $使$ a $、$ b $静止,撤去此压力后,$ a $开始运动,在$ a $下降的过程中,$ b $始终未离开桌面。在此过程中 \xzanswer{AD} 
\begin{figure}[h!]
\centering
\includesvg[width=0.23\linewidth]{picture/svg/787}
\end{figure}

\fourchoices
{$ a $的动能小于$ b $的动能}
{两物体机械能的变化量相等}
{$ a $的重力势能的减小量等于两物体总动能的增加量}
{绳的拉力对$ a $所做的功与对$ b $所做的功的代数和为零}



\item 
\exwhere{$ 2014 $年理综福建卷}
如图,两根相同的轻质弹簧,沿足够长的光滑斜面放置,下端固定在斜面底部挡板上,斜面固定不动。质量不同、形状相同的两物块分别置于两弹簧上端。现用外力作用在物块上,使两弹簧具有相同的压缩量,若撤去外力后,两物块由静止沿斜面向上弹出并离开弹簧,则从撤去外力到物块速度第一次减为零的过程,两物块 \xzanswer{C} 
\begin{figure}[h!]
\centering
\includesvg[width=0.23\linewidth]{picture/svg/788}
\end{figure}

\fourchoices
{最大速度相同 }
{最大加速度相同}
{上升的最大高度不同 }
{重力势能的变化量不同}



\item
\exwhere{$ 2013 $年江苏卷}
如图所示,水平桌面上的轻质弹簧一端固定,另一端与小物块相连. 弹簧处于自然长度时物块位于$ O $点(图中未标出). 物块的质量为$ m $,$ AB $ $ =a $,物块与桌面间的动摩擦因数为$ \mu $. 现用水平向右的力将物块从$ O $ 点拉至$ A $ 点,拉力做的功为$ W $. 撤去拉力后物块由静止向左运动, 经$ O $点到达$ B $点时速度为零. 重力加速度为$ g $. 则上述过程中 \xzanswer{BC} 
\begin{figure}[h!]
\centering
\includesvg[width=0.23\linewidth]{picture/svg/789}
\end{figure}


\fourchoices
{物块在$ A $点时,弹簧的弹性势能等于$W - \frac { 1 } { 2 } \mu m g a$}
{物块在$ B $点时,弹簧的弹性势能小于$W - \frac { 3 } { 2 } \mu m g a$}
{经$ O $点时,物块的动能小于 $W - \mu m g a$}
{物块动能最大时弹簧的弹性势能小于物块在$ B $点时弹簧的弹性势能}




\item
\exwhere{$ 2016 $年新课标$ \lmd{2} $卷}
如图,小球套在光滑的竖直杆上,轻弹簧一端固定于$ O $点,另一端与小球相连。现将小球从$ M $点由静止释放,它在下降的过程中经过了$ N $点。已知$ M $、$ N $两点处,弹簧对小球的弹力大小相等,且$ \angle ONM< \angle OMN< \frac{\pi}{2} $。在小球从$ M $点运动到$ N $点的过程中 \xzanswer{BCD} 


\begin{minipage}[h!]{0.7\linewidth}
\vspace{0.3em}
\fourchoices
{弹力对小球先做正功后做负功}
{有两个时刻小球的加速度等于重力加速度}
{弹簧长度最短时,弹力对小球做功的功率为零}
{小球到达$ N $点时的动能等于其在$ M $、$ N $两点的重力势能差}

\vspace{0.3em}
\end{minipage}
\hfill
\begin{minipage}[h!]{0.3\linewidth}
\flushright
\vspace{0.3em}
\includesvg[width=0.5\linewidth]{picture/svg/790}
\vspace{0.3em}
\end{minipage}



\item 
\exwhere{$ 2014 $年理综重庆卷}
下图为“嫦娥三号”探测器在月球上着陆最后阶段的示意图。首先在发动机作用下,探测器受到推力在距月面高度为$ h_{1} $处悬停(速度为$ 0 $,$ h_{1} $远小于月球半径);接着推力改变,探测器开始竖直下降,到达距月面高度为$ h_{2} $处的速度为$ v $;此后发动机关闭,探测器仅受重力下落到月面。已知探测器总质量为$ m $(不包括燃料),地球和月球的半径比为$ k_{1} $,质量比为$ k_{2} $,地球表面附近的重力加速度为$ g $。求: 
\begin{enumerate}
\renewcommand{\labelenumi}{\arabic{enumi}.}
% A(\Alph) a(\alph) I(\Roman) i(\roman) 1(\arabic)
%设定全局标号series=example	%引用全局变量resume=example
%[topsep=-0.3em,parsep=-0.3em,itemsep=-0.3em,partopsep=-0.3em]
%可使用leftmargin调整列表环境左边的空白长度 [leftmargin=0em]
\item
月球表面附近的重力加速度大小及探测器刚接触月面时的速度大小;
\item 
从开始竖直下降到刚接触月面时,探测器机械能的变化。



\end{enumerate}
\begin{figure}[h!]
\flushright
\includesvg[width=0.25\linewidth]{picture/svg/791}
\end{figure}


\banswer{
\begin{enumerate}
\renewcommand{\labelenumi}{\arabic{enumi}.}
% A(\Alph) a(\alph) I(\Roman) i(\roman) 1(\arabic)
%设定全局标号series=example	%引用全局变量resume=example
%[topsep=-0.3em,parsep=-0.3em,itemsep=-0.3em,partopsep=-0.3em]
%可使用leftmargin调整列表环境左边的空白长度 [leftmargin=0em]
\item
$\frac { k _ { 1 } ^ { 2 } } { k _ { 2 } } g$ \qquad $\sqrt { v ^ { 2 } + \frac { 2 k _ { 1 } ^ { 2 } g h _ { 2 } } { k _ { 2 } } }$
\item 
$\frac { 1 } { 2 } m v ^ { 2 } - \frac { k _ { 1 } ^ { 2 } } { k _ { 2 } } m g \left( h _ { 1 } - h _ { 2 } \right)$


\end{enumerate}


}




\item 
\exwhere{$ 2011 $年理综浙江卷}
节能混合动力车是一种可以利用汽油及所储存电能作为动力来源的汽车。有一质量$ m=1000 \ kg $的混合动力轿车,在平直公路上以$ v_{1} =90 \ km/h $匀速行驶,发动机的输出功率为$ P=50\ kW $。当驾驶员看到前方有$ 80 \ km/h $的限速标志时,保持发动机功率不变,立即启动利用电磁阻尼带动的发电机工作给电池充电,使轿车做减速运动,运动$ L=72\ m $后,速度变为$ v_{2} =72 \ km/h $。此过程中发动机功率的$ 1/5 $用于轿车的牵引,$ 4/5 $用于供给发电机工作,发动机输送给发电机的能量最后有$ 50 \% $转化为电池的电能。假设轿车在上述运动过程中所受阻力保持不变。求:

\begin{enumerate}
\renewcommand{\labelenumi}{\arabic{enumi}.}
% A(\Alph) a(\alph) I(\Roman) i(\roman) 1(\arabic)
%设定全局标号series=example	%引用全局变量resume=example
%[topsep=-0.3em,parsep=-0.3em,itemsep=-0.3em,partopsep=-0.3em]
%可使用leftmargin调整列表环境左边的空白长度 [leftmargin=0em]
\item
轿车以$ 90 \ km/h $在平直公路上匀速行驶时,所受阻力$ F_{ \text{阻} } $的大小;
\item 
轿车从$ 90 \ km/h $减速到$ 72 \ km/h $过程中,获得的电能$ E_{ \text{电} } $;
\item 
轿车仅用其在上述减速过程中获得的电能$ E $电维持$ 72 \ km/h $匀速运动的距离$ L ^{\prime} $。



\end{enumerate}

\banswer{
\begin{enumerate}
\renewcommand{\labelenumi}{\arabic{enumi}.}
% A(\Alph) a(\alph) I(\Roman) i(\roman) 1(\arabic)
%设定全局标号series=example	%引用全局变量resume=example
%[topsep=-0.3em,parsep=-0.3em,itemsep=-0.3em,partopsep=-0.3em]
%可使用leftmargin调整列表环境左边的空白长度 [leftmargin=0em]
\item
$ 2 \times 10^3 \ N $
\item 
$ 6.3 \times 10^4 $ $ J $
\item 
$ 31.5 \ m $
\end{enumerate}


}


\newpage
\item 
\exwhere{$ 2015 $年理综北京卷}
如图所示,弹簧的一端固定,另一端连接一个物块,弹簧质量不计,物块(可视为质点)的质量为$ m $,在水平桌面上沿$ x $轴运动,与桌面间的动摩擦因数为$ \mu $,以弹簧原长时物块的位置为坐标原点$ O $,当弹簧的伸长量为$ x $时,物块所受弹簧弹力大小为$ F=kx $,$ k $为常量。
\begin{enumerate}
\renewcommand{\labelenumi}{\arabic{enumi}.}
% A(\Alph) a(\alph) I(\Roman) i(\roman) 1(\arabic)
%设定全局标号series=example	%引用全局变量resume=example
%[topsep=-0.3em,parsep=-0.3em,itemsep=-0.3em,partopsep=-0.3em]
%可使用leftmargin调整列表环境左边的空白长度 [leftmargin=0em]
\item
请画出$ F $随$ x $变化的示意图;并根据$ F-x $图像求物块沿$ x $轴从$ O $点运动到位置$ x $的过程中弹力所做的功。
\item 
物块由$ x_{1} $向右运动到$ x_{3} $,然后由$ x_{3} $返回到$ x_{2} $,在这个过程中
\begin{enumerate}
\renewcommand{\labelenumiii}{\alph{enumiii}.}
% A(\Alph) a(\alph) I(\Roman) i(\roman) 1(\arabic)
%设定全局标号series=example	%引用全局变量resume=example
%[topsep=-0.3em,parsep=-0.3em,itemsep=-0.3em,partopsep=-0.3em]
%可使用leftmargin调整列表环境左边的空白长度 [leftmargin=0em]
\item
求弹力所做的功,并据此求弹性势能的变化量;
\item 
求滑动摩擦力所做的功;并与弹力做功比较,说明为什么不存在与摩擦力对应的“摩擦力势能”的概念。




\end{enumerate}
\begin{figure}[h!]
\flushright
\includesvg[width=0.4\linewidth]{picture/svg/792}
\end{figure}





\end{enumerate}


\banswer{
\begin{enumerate}
\renewcommand{\labelenumi}{\arabic{enumi}.}
% A(\Alph) a(\alph) I(\Roman) i(\roman) 1(\arabic)
%设定全局标号series=example	%引用全局变量resume=example
%[topsep=-0.3em,parsep=-0.3em,itemsep=-0.3em,partopsep=-0.3em]
%可使用leftmargin调整列表环境左边的空白长度 [leftmargin=0em]
\item
$W = - \frac { 1 } { 2 } k x ^ { 2 }$
\item 
$a. \ \Delta E _ { p } = \frac { 1 } { 2 } k x _ { 2 } ^ { 2 } - \frac { 1 } { 2 } k x _ { 1 } ^ { 2 } \quad ; \quad b . \ W _ { f } = - \mu m g \left( 2 x _ { 3 } - x _ { 2 } - x _ { 1 } \right)$



\end{enumerate}


}


\newpage
\item 
\exwhere{$ 2011 $年上海卷}
如图$ (a) $,磁铁$ A $、$ B $的同名磁极相对放置,置于水平气垫导轨上。$ A $固定于导轨左端,$ B $的质量$ m=0.5 \ kg $,可在导轨上无摩擦滑动。将$ B $在$ A $附近某一位置由静止释放,由于能量守恒,可通过测量$ B $在不同位置处的速度,得到$ B $的总势能随位置$ x $的变化规律,见图$ (c) $中曲线 \lmd{1} 。若将导轨右端抬高,使其与水平面成一定角度(如图$ (b) $所示),则$ B $的总势能曲线如图$ (c) $中 \lmd{2} 所示。将$ B $在$ x=20 \ cm $处由静止释放,求:(解答时必须写出必要的推断说明。取$ g=9.8 \ m/s^{2} $)
\begin{enumerate}
\renewcommand{\labelenumi}{\arabic{enumi}.}
% A(\Alph) a(\alph) I(\Roman) i(\roman) 1(\arabic)
%设定全局标号series=example	%引用全局变量resume=example
%[topsep=-0.3em,parsep=-0.3em,itemsep=-0.3em,partopsep=-0.3em]
%可使用leftmargin调整列表环境左边的空白长度 [leftmargin=0em]
\item
$ B $在运动过程中动能最大的位置;
\item 
运动过程中$ B $的最大速度和最大位移;
\item 
图$ (c) $中直线 \lmd{3} 为曲线 \lmd{2} 的渐近线,求导轨的倾角;
\item 
若$ A $、$ B $异名磁极相对放置,导轨的倾角不变,在图$ (c) $上画出$ B $的总势能随$ x $的变化曲线。



\end{enumerate}
\begin{figure}[h!]
\centering
\includesvg[width=0.83\linewidth]{picture/svg/793}
\end{figure}


\banswer{
\begin{enumerate}
\renewcommand{\labelenumi}{\arabic{enumi}.}
% A(\Alph) a(\alph) I(\Roman) i(\roman) 1(\arabic)
%设定全局标号series=example	%引用全局变量resume=example
%[topsep=-0.3em,parsep=-0.3em,itemsep=-0.3em,partopsep=-0.3em]
%可使用leftmargin调整列表环境左边的空白长度 [leftmargin=0em]
\item
势能最小处动能最大,由图线 \lmd{2} 得$ x=6.1\ cm $。(在$ 5.9 \sim 6.3 \ cm $间均视为正确)
\item 
$v _ { m } = \sqrt { \frac { 2 E _ { k m } } { m } } = \sqrt { \frac { 2 \times 0.43 } { 0.5 } } = 1.31 ( \mathrm { m } / \mathrm { s } )$ , ($ v_{m} $在$ 1.29 \sim 1.33 $ $ m/s $间均视为正确) \\
$\Delta x = 20.0 - 2.0 = 18.0 ( \mathrm { cm } )$ , ($ \Delta x $在$ 17.9 \sim 18.1 \ cm $间均视为正确)
\item 
$\theta = \sin ^ { - 1 } \left( \frac { k \times 10 ^ { 2 } } { m g } \right) = \sin ^ { - 1 } \frac { 4.23 } { 0.5 \times 9.8 } = 59.7 ^ { \circ }$ , ($ \theta $在$59 ^ { \circ } \sim 61 ^ { \circ }$间均视为正确)
\item 
若异名磁极相对放置,$ A $,$ B $间相互作用势能为负值,总势能如图。\\
 \includesvg[width=0.23\linewidth]{picture/svg/794} \\
 得:$\Delta E = \frac { 1 } { 2 } m v ^ { 2 } - \frac { k _ { 1 } ^ { 2 } } { k _ { 2 } } \cdot m g \left( h _ { 1 } - h _ { 2 } \right)$
\end{enumerate}


}








\end{enumerate}

