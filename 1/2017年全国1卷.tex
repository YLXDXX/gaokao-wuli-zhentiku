\bta{$ 2017 $ 年全国普通高等学校招生考试(全国卷 \lmd{1} ) }

\begin{center}
\heiti 
\zihao{4}
物理部分
\end{center}
\vspace{2em}

\begin{enumerate}
\heiti
\renewcommand{\labelenumi}{\arabic{enumi}.}
% A(\Alph) a(\alph) I(\Roman) i(\roman) 1(\arabic)
%设定全局标号series=example	%引用全局变量resume=example
%[topsep=-0.3em,parsep=-0.3em,itemsep=-0.3em,partopsep=-0.3em]
%可使用leftmargin调整列表环境左边的空白长度 [leftmargin=0em]
\item[一、]
选择题:本题共 $ 8 $ 小题,每小题 $ 6 $ 分。在每小题给出的四个选项中,第 $ 1 \sim 5 $ 题只有一项符合题
目要求,第 $ 6 \sim 8 $ 题有多项符合题目要求。全部选对的得 $ 6 $ 分,选对但不全的得 $ 3 $ 分,有选错的
得 $ 0 $ 分。




\end{enumerate}


\begin{enumerate}
\renewcommand{\labelenumi}{\arabic{enumi}.}
% A(\Alph) a(\alph) I(\Roman) i(\roman) 1(\arabic)
%设定全局标号series=example	%引用全局变量resume=example
%[topsep=-0.3em,parsep=-0.3em,itemsep=-0.3em,partopsep=-0.3em]
%可使用leftmargin调整列表环境左边的空白长度 [leftmargin=0em]
\item
将质量为 $ 1.00 \ kg $ 的模型火箭点火升空,$ 50 \ g $ 燃烧的燃气以大小为 $ 600 \ m/s $ 的速度从火箭喷口在
很短时间内喷出。在燃气喷出后的瞬间,火箭的动量大小为(喷出过程中重力和空气阻力可忽
略) \xzanswer{A} 

\fourchoices
{$ 30 \ kg \times m/s $}
{$ 5.7 \times 10^{2} \ kg \times m/s $}
{$ 6.0 \times 10^{2} \ kg \times m/s $}
{$ 6.3 \times 10^{2} \ kg \times m/s $}



\item
发球机从同一高度向正前方依次水平射出两个速度不同的乒乓球(忽略空气的影响)。速度较大
的球越过球网,速度较小的球没有越过球网;其原因是 \xzanswer{C} 

\fourchoices
{速度较小的球下降相同距离所用的时间较多}
{速度较小的球在下降相同距离时在竖直方向上的速度较大}
{速度较大的球通过同一水平距离所用的时间较少}
{速度较大的球在相同时间间隔内下降的距离较大}



\item
如图,空间某区域存在匀强电场和匀强磁场,电场方向竖直向上(与纸面平行)
,磁场方向垂直
于纸面向里,三个带正电的微粒 $ a $、 $ b $、 $ c $ 电荷量相等,质量分别为 $ ma $、 $ mb $、 $ mc $。已知在该
区域内,$ a $ 在纸面内做匀速圆周运动,$ b $ 在纸面内向右做匀速直线运动,$ c $ 在纸面内向左做匀速
直线运动。下列选项正确的是 \xzanswer{B} 
\begin{figure}[h!]
\centering
\includesvg[width=0.23\linewidth]{picture/svg/GZ-3-tiyou-0578}
\end{figure}


\fourchoices
{$m_{a}>m_{b}>m_{c}$}
{$m_{b}>m_{a}>m_{c}$}
{$ m_{c}>m_{a}>m_{b}$}
{$m_{c}>m_{b}>m_{a}$}



\item 
大科学工程“人造太阳”主要是将氘核聚变反应释放的能量用来发电。氘核聚变反应方程是
$ ^2_1H+^2_1H \rightarrow ^3_2He+ ^1_0n $。已知 $ ^2_1H $ 的质量为 $ 2.0136 \ u $, $ ^3_2He $ 的质量为 $ 3.0150 \ u $,$ ^1_0 n $ 的质量为 $ 1.0087 \ u $,$ 1 \ u=931 \ MeV/c^{2} $。氘核聚变反应中释放的核能约为 \xzanswer{B} 

\fourchoices
{$ 3.7 \ MeV $}
{$ 3.3 \ MeV $}
{$ 2.7 \ MeV $}
{$ 0.93 \ MeV $}


\item
扫描隧道显微镜($ STM $)可用来探测样品表面原子尺度上的形貌。为了有效隔离外界振动对 $ STM $
的扰动,在圆底盘周边沿其径向对称地安装若干对紫铜薄板,并施加磁场来快速衰减其微小振
动,如图所示。无扰动时,按下列四种方案对紫铜薄板施加恒磁场;出现扰动后,对于紫铜薄
板上下及左右振动的衰减最有效的方案是 \xzanswer{A} 
\begin{figure}[h!]
\centering
\includesvg[width=0.83\linewidth]{picture/svg/GZ-3-tiyou-0579}
\end{figure}





\item
如图,三根相互平行的固定长直导线 $ L_{1} $、$ L_{2} $ 和 $ L_{3} $ 两两等距,均通有电流 $ I $ ,$ L_{1} $ 中电流方向与 $ L_{2} $
中的相同,与 $ L_{3} $ 中的相反,下列说法正确的是 \xzanswer{BC} 
\begin{figure}[h!]
\centering
\includesvg[width=0.23\linewidth]{picture/svg/GZ-3-tiyou-0580}
\end{figure}


\fourchoices
{$ L_{1} $ 所受磁场作用力的方向与 $ L_{2} $、 $ L_{3} $ 所在平面垂直}
{$ L_{3} $ 所受磁场作用力的方向与 $ L_{1} $、 $ L_{2} $ 所在平面垂直}
{$ L_{1} $、 $ L_{2} $ 和 $ L_{3} $ 单位长度所受的磁场作用力大小之比为 $ 1:1:\sqrt{3} $}
{$ L_{1} $、 $ L_{2} $ 和 $ L_{3} $ 单位长度所受的磁场作用力大小之比为 $ \sqrt{3}:\sqrt{3}:1 $}



\newpage
\item 
在一静止点电荷的电场中,任一点的电势 $ j $ 与该点到点电荷的距离 $ r $ 的关系如图所示。电场中
四个点 $ a $、 $ b $、 $ c $ 和 $ d $ 的电场强度大小分别 $ E_{a} $、 $ E_{b} $、 $ E_{c} $ 和 $ E_{d} $。点 $ a $ 到点电荷的距离 $ r_{a} $ 与点 $ a $
的电势 $\varphi_{a}$ 已在图中用坐标( $ r_{a} $,$ \varphi_a $ )标出,其余类推。现将一带正电的试探电荷由 $ a $ 点依次经

$ b $、 $ c $ 点移动到 $ d $ 点,在相邻两点间移动的过程中,电场力所做的功分别为 $ W_{ab} $、 $ W_{bc} $ 和 $ W_{cd} $。
下列选项正确的是 \xzanswer{AC} 
\begin{figure}[h!]
\centering
\includesvg[width=0.26\linewidth]{picture/svg/GZ-3-tiyou-0581}
\end{figure}


\fourchoices
{$ E_{a}: E_{b}=4: 1 $}
{$E_{c}: E_{d}=2: 1$}
{$W_{a b}: W_{b c}=3: 1$}
{$W_{b c}: W_{c d}=1: 3$}


\banswer{

}



\item
如图,柔软轻绳 $ ON $ 的一端 $ O $ 固定,其中间某点 $ M $ 拴一重物,用手拉住绳的另一端 $ N $。初始
时, $ OM $ 竖直且 $ MN $ 被拉直, $ OM $ 与 $ MN $ 之间的夹角为 $ \alpha $ ( $ \alpha >\frac{\pi}{2} $
)。现将重物向右上方缓
慢拉起,并保持夹角 $ \alpha $ 不变。在 $ OM $ 由竖直被拉到水平的过程中 \xzanswer{AD} 
\begin{figure}[h!]
\centering
\includesvg[width=0.16\linewidth]{picture/svg/GZ-3-tiyou-0582}
\end{figure}


\fourchoices
{$ MN $ 上的张力逐渐增大}
{$ MN $ 上的张力先增大后减小}
{$ OM $ 上的张力逐渐增大}
{$ OM $ 上的张力先增大后减小}


\banswer{

}


\newpage
\begin{enumerate}[leftmargin=0em]
\renewcommand{\labelenumii}{}
% A(\Alph) a(\alph) I(\Roman) i(\roman) 1(\arabic)
%可使用leftmargin调整列表环境左边的空白长度
\item[二、]
\btd{非选择题:共 $ 62 $ 分。第 $ 9 \sim 12 $ 题为必考题,每个试题考生都必须作答。第 $ 13 \sim 14 $ 题为选考题,
考生根据要求作答。}

\end{enumerate}

\begin{enumerate}[leftmargin=-2em]
\renewcommand{\labelenumii}{}
% A(\Alph) a(\alph) I(\Roman) i(\roman) 1(\arabic)
%可使用leftmargin调整列表环境左边的空白长度
\item
\btd{(一)必考题(共 $ 47 $ 分)}
\end{enumerate} 



\item
($ 5 $ 分)某探究小组为了研究小车在桌面上的直线运动,用自制“滴水计时器”计量时间。实验前,
将该计时器固定在小车旁,如图($ a $)所示。实验时,保持桌面水平,用手轻推一下小车。在
小车运动过程中,滴水计时器等时间间隔地滴下小水滴,图($ b $)记录了桌面上连续的 $ 6 $ 个水
滴的位置。(已知滴水计时器每 $ 30 \ s $ 内共滴下 $ 46 $ 个小水滴)
\begin{figure}[h!]
\centering
\includesvg[width=0.83\linewidth]{picture/svg/GZ-3-tiyou-0583}
\end{figure}

\begin{enumerate}
\renewcommand{\labelenumi}{\arabic{enumi}.}
% A(\Alph) a(\alph) I(\Roman) i(\roman) 1(\arabic)
%设定全局标号series=example	%引用全局变量resume=example
%[topsep=-0.3em,parsep=-0.3em,itemsep=-0.3em,partopsep=-0.3em]
%可使用leftmargin调整列表环境左边的空白长度 [leftmargin=0em]
\item

由图($ b $)可知,小车在桌面上是 \tk{从右向左} (填“从右向左”或“从左向右”)运动的。


\item 
该小组同学根据图($ b $)的数据判断出小车做匀变速运动。小车运动到图($ b $)中 $ A $ 点位
置时的速度大小为 \tk{$ 0.19 $} $ m/s $,加速度大小为 \tk{$ 0.037 $} $ m/s $。(结果均保留 $ 2 $ 位有效数字)

\end{enumerate}



\newpage
\item 
($ 10 $ 分)某同学研究小灯泡的伏安特性,所使用的器材有:小灯泡 $ L $ (额定电压 $ 3.8 \ V $,额定
电流 $ 0.32 \ A $);电压表 \voltmetermytikz (量程 $ 3 \ V $,内阻 $ 3 \ k\Omega $);电流表 \ammetermytikz (量程 $ 0.5 \ A $,内阻 $ 0.5 \ \Omega $);固
定电阻 $ R_{0} $ (阻值 $ 1000 \ \Omega $);滑动变阻器 $ R $ (阻值 $ 0 \sim 9.0 \ \Omega $);电源 $ E $ (电动势 $ 5 \ V $,内阻不计);
开关 $ S $;导线若干。
\begin{figure}[h!]
\centering
\includesvg[width=0.83\linewidth]{picture/svg/GZ-3-tiyou-0584}
\end{figure}

\begin{enumerate}
\renewcommand{\labelenumi}{\arabic{enumi}.}
% A(\Alph) a(\alph) I(\Roman) i(\roman) 1(\arabic)
%设定全局标号series=example	%引用全局变量resume=example
%[topsep=-0.3em,parsep=-0.3em,itemsep=-0.3em,partopsep=-0.3em]
%可使用leftmargin调整列表环境左边的空白长度 [leftmargin=0em]
\item
实验要求能够实现在 $ 0 \sim 3.8 \ V $ 的范围内对小灯泡的电压进行测量,画出实验电路原理图。

\banswer{
实验电路原理图如图所示
 \includesvg[width=0.23\linewidth]{picture/svg/GZ-3-tiyou-0585} 
}


\item 
实验测得该小灯泡伏安特性曲线如图($ a $)所示。由实验曲线可知,随着电流的增加小灯泡的电阻 \tk{增大} (填“增大”“不变”或“减
小”),灯丝的电阻率
\tk{增大} 
(填“增大”“不变”或“减小”)。

\item 
用另一电源 $ E_{0} $ (电动势 $ 4 \ V $,内阻 $ 1.00 \ \Omega $)和题给器材连接成图($ b $)所示的电路,调节
滑动变阻器 $ R $ 的阻值,可以改变小灯泡的实际功率。闭合开关 $ S $,在 $ R $ 的变化范围内,
小灯泡的最小功率为 \tk{$ 0.39 $} $ W $,最大功率为 \tk{$ 1.17 $} $ W $。(结果均保留 $ 2 $ 位小数)

\end{enumerate}

\banswer{

}




\newpage
\item 
($ 12 $ 分)一质量为 $8.00 \times 10^{4} \ \mathrm{kg}$ 的太空飞船从其飞行轨道返回地面。飞船在离地面高度
$1.60 \times 10^{5} \ \mathrm{m}$处以 $7.50 \times 10^{3} \ \mathrm{m} / \mathrm{s}$ 的速度进入大气层,逐渐减慢至速度为 $ 100 \ m/s $ 时下落到
地面。取地面为重力势能零点,在飞船下落过程中,重力加速度可视为常量,大小取为 $ 9.8 \ m/s^{2} $。(结果保留 $ 2 $ 位有效数字)
\begin{enumerate}
\renewcommand{\labelenumi}{\arabic{enumi}.}
% A(\Alph) a(\alph) I(\Roman) i(\roman) 1(\arabic)
%设定全局标号series=example	%引用全局变量resume=example
%[topsep=-0.3em,parsep=-0.3em,itemsep=-0.3em,partopsep=-0.3em]
%可使用leftmargin调整列表环境左边的空白长度 [leftmargin=0em]
\item
分别求出该飞船着地前瞬间的机械能和它进入大气层时的机械能;
\item 
求飞船从离地面高度 $ 600 \ m $ 处至着地前瞬间的过程中克服阻力所做的功,已知飞船在该
处的速度大小是其进入大气层时速度大小的 $ 2.0 \% $。




\end{enumerate}
\banswer{
\begin{enumerate}
\renewcommand{\labelenumi}{\arabic{enumi}.}
% A(\Alph) a(\alph) I(\Roman) i(\roman) 1(\arabic)
%设定全局标号series=example	%引用全局变量resume=example
%[topsep=-0.3em,parsep=-0.3em,itemsep=-0.3em,partopsep=-0.3em]
%可使用leftmargin调整列表环境左边的空白长度 [leftmargin=0em]
\item
$E_{k 0}=4.0 \times 10^{8} \mathrm{J}$ \quad $E_{h}=2.4 \times 10^{12} \ \mathrm{J}$
\item 
$W=9.7 \times 10^{8} \ \mathrm{J}$

\end{enumerate}


}




\item 
($ 20 $ 分)真空中存在电场强度大小为 $ E_{1} $ 的匀强电场,一带电油滴在该电场中竖直向上做匀速
直线运动,速度大小为 $ v_{0} $。在油滴处于位置 $ A $ 时,将电场强度的大小突然增大到某值,但保
持其方向不变。持续一段时间 $ t_{1} $ 后,又突然将电场反向,但保持其大小不变;再持续同样一段
时间后,油滴运动到 $ B $ 点。重力加速度大小为 $ g $。
\begin{enumerate}
\renewcommand{\labelenumi}{\arabic{enumi}.}
% A(\Alph) a(\alph) I(\Roman) i(\roman) 1(\arabic)
%设定全局标号series=example	%引用全局变量resume=example
%[topsep=-0.3em,parsep=-0.3em,itemsep=-0.3em,partopsep=-0.3em]
%可使用leftmargin调整列表环境左边的空白长度 [leftmargin=0em]
\item
求油滴运动到 $ B $ 点时的速度。
\item 
求增大后的电场强度的大小;为保证后来的电场强度比原来的大,试给出相应的 $ t_{1} $ 和 $ v_{0} $
应满足的条件。已知不存在电场时,油滴以初速度 $ v_{0} $ 做竖直上抛运动的最大高度恰好等于
$ B $、 $ A $ 两点间距离的两倍。




\end{enumerate}

\banswer{
\begin{enumerate}
\renewcommand{\labelenumi}{\arabic{enumi}.}
% A(\Alph) a(\alph) I(\Roman) i(\roman) 1(\arabic)
%设定全局标号series=example	%引用全局变量resume=example
%[topsep=-0.3em,parsep=-0.3em,itemsep=-0.3em,partopsep=-0.3em]
%可使用leftmargin调整列表环境左边的空白长度 [leftmargin=0em]
\item
$v_{2}=v_{0}-2 g t_{1}$
\item 
若$ B $点在$ A $点之上,$E_{2}=\left[2-2 \frac{v_{0}}{g t_{1}}+\frac{1}{4}\left(\frac{v_{0}}{g t_{1}}\right)^{2}\right] E_{1}$ \quad 为使 $E_{2}>E_{1},$ 应有$2-2 \frac{v_{0}}{g t_{1}}+\frac{1}{4}\left(\frac{v_{0}}{g t_{1}}\right)^{2}>1$。即当 $0<t_{1}<\left(1-\frac{\sqrt{3}}{2}\right) \frac{v_{0}}{g}$或 $t_{1}>\left(1+\frac{\sqrt{3}}{2}\right) \frac{v_{0}}{g}$才是可能的;
\\
若 $B$ 在 $A$ 点之下,$E_{2}=\left[2-2 \frac{v_{0}}{g t_{1}}-\frac{1}{4}\left(\frac{v_{0}}{g t_{1}}\right)^{2}\right] E_{1}$。为使 $E_{2}>E_{1},$ 应有
$2-2 \frac{v_{0}}{g t_{1}}-\frac{1}{4}\left(\frac{v_{0}}{g t_{1}}\right)^{2}>1$
即 $t_{1}>\left(\frac{\sqrt{5}}{2}+1\right) \frac{v_{0}}{g}$
另一解为负,不符合题意


\end{enumerate}


}




\newpage
\begin{enumerate}[leftmargin=-2em]
\renewcommand{\labelenumii}{}
% A(\Alph) a(\alph) I(\Roman) i(\roman) 1(\arabic)
%可使用leftmargin调整列表环境左边的空白长度
\item
\btd{(二)选考题:共 $ 15 $ 分。请考生从 $ 2 $ 道物理题中任选一题作答。如果多做,则按所做的第一题计
分。}
\end{enumerate}


\item


[物理——选修 $ 3 $–$ 3 $]($ 15 $ 分)
\begin{enumerate}
\renewcommand{\labelenumi}{\arabic{enumi}.}
% A(\Alph) a(\alph) I(\Roman) i(\roman) 1(\arabic)
%设定全局标号series=example	%引用全局变量resume=example
%[topsep=-0.3em,parsep=-0.3em,itemsep=-0.3em,partopsep=-0.3em]
%可使用leftmargin调整列表环境左边的空白长度 [leftmargin=0em]
\item
($ 5 $ 分)氧气分子在 $ 0 \ \celsius $和 $ 100 \ \celsius $温度下单位速率间隔的分子数占总分子数的百分比随
气体分子速率的变化分别如图中两条曲线所示。下列说法正确的是 \tk{ABC} 。(填正确答案
标号。选对 $ 1 $ 个得 $ 2 $ 分,选对 $ 2 $ 个得 $ 4 $ 分,选对 $ 3 $ 个得 $ 5 $ 分。每选错 $ 1 $ 个扣 $ 3 $ 分,最低得分
为 $ 0 $ 分)
\begin{figure}[h!]
\centering
\includesvg[width=0.28\linewidth]{picture/svg/GZ-3-tiyou-0586}
\end{figure}

\fivechoices
{图中两条曲线下面积相等}
{图中虚线对应于氧气分子平均动能较小的情形}
{图中实线对应于氧气分子在 $ 100 \ \celsius $时的情形}
{图中曲线给出了任意速率区间的氧气分子数目}
{与 $ 0 \ \celsius $时相比,$ 100 \ \celsius $时氧气分子速率出现在 $ 0 \sim 400 \ m/s $ 区间内的分子数占总分子数的百分比较大}


\item 
($ 10 $ 分)如图,容积均为 $ V $ 的汽缸 $ A $、 $ B $ 下端有细管(容积可忽略)连通,阀门 $ K_{2} $ 位
于细管的中部, $ A $、 $ B $ 的顶部各有一阀门 $ K_{1} $、 $ K_{3} $; $ B $ 中有一可自由滑动的活塞(质量、
体积均可忽略)。初始时,三个阀门均打开,活塞在 $ B $ 的底部;关闭 $ K_{2} $、 $ K_{3} $,通过 $ K_{1} $ 给
汽缸充气,使 $ A $ 中气体的压强达到大气压 $ p_{0} $ 的 $ 3 $ 倍后关闭 $ K_{1} $。已知室温为 $ 27 \ \celsius $,汽缸导
热。
\begin{enumerate}
\renewcommand{\labelenumi}{\arabic{enumi}.}
% A(\Alph) a(\alph) I(\Roman) i(\roman) 1(\arabic)
%设定全局标号series=example	%引用全局变量resume=example
%[topsep=-0.3em,parsep=-0.3em,itemsep=-0.3em,partopsep=-0.3em]
%可使用leftmargin调整列表环境左边的空白长度 [leftmargin=0em]
\item
打开 $ K_{2} $,求稳定时活塞上方气体的体积和压强;
\item 
接着打开 $ K_{3} $,求稳定时活塞的位置;
\item 
再缓慢加热汽缸内气体使其温度升高 $ 20 \ \celsius $,求此时活塞下方气体的压强。




\end{enumerate}
\begin{figure}[h!]
\flushright
\includesvg[width=0.25\linewidth]{picture/svg/GZ-3-tiyou-0587}
\end{figure}


\banswer{
\begin{enumerate}
\renewcommand{\labelenumi}{\arabic{enumi}.}
% A(\Alph) a(\alph) I(\Roman) i(\roman) 1(\arabic)
%设定全局标号series=example	%引用全局变量resume=example
%[topsep=-0.3em,parsep=-0.3em,itemsep=-0.3em,partopsep=-0.3em]
%可使用leftmargin调整列表环境左边的空白长度 [leftmargin=0em]
\item
$p_{1}=2 p_{0}$
\item 
$p_{2}^{\prime}=\frac{3}{2} p_{0}$
\item 
$p_{3}=1.6 p_{0}$

\end{enumerate}


}


\end{enumerate}





\newpage
\item 

[物理——选修 $ 3 $–$ 4 $]($ 15 $ 分)
\begin{enumerate}
\renewcommand{\labelenumi}{\arabic{enumi}.}
% A(\Alph) a(\alph) I(\Roman) i(\roman) 1(\arabic)
%设定全局标号series=example	%引用全局变量resume=example
%[topsep=-0.3em,parsep=-0.3em,itemsep=-0.3em,partopsep=-0.3em]
%可使用leftmargin调整列表环境左边的空白长度 [leftmargin=0em]
\item
($ 5 $ 分)如图($ a $),在 $ xy $ 平面内有两个沿 $ z $ 方向做简谐振动的点波源 $ S_{1}(0,4) $ 和 $ S_{2}(0,-2) $。
两波源的振动图线分别如图($ b $)和图($ c $)所示,两列波的波速均为 $ 1.00 \ m/s $。两列波
从波源传播到点 $ A(8,-2) $ 的路程差为 \tk{2} $ m $,两列波引起的点 $ B(4,1) $ 处质点的振动相
互 \tk{减弱} (填“加强”或“减弱”),点 $ C(0,0.5) $ 处质点的振动相互 \tk{加强} (填“加强”
或“减弱”)。
\begin{figure}[h!]
\centering
\includesvg[width=0.73\linewidth]{picture/svg/GZ-3-tiyou-0588}
\end{figure}

\item 
($ 10 $ 分)如图,一玻璃工件的上半部是半径为 $ R $ 的半球体, $ O $ 点为球心;下半部是半径
为 $ R $、高位 $ 2R $ 的圆柱体,圆柱体底面镀有反射膜。有一平行于中心轴 $ OC $ 的光线从半球面
射入,该光线与 $ OC $ 之间的距离为 $ 0.6R $。已知最后从半球面射出的光线恰好与入射光线平
行(不考虑多次反射)。求该玻璃的折射率。
\begin{figure}[h!]
\flushright
\includesvg[width=0.2\linewidth]{picture/svg/GZ-3-tiyou-0589}
\end{figure}

\banswer{
$n=\sqrt{2.05} \approx 1.43$
}


\end{enumerate}





\end{enumerate}






