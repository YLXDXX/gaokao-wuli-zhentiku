\bta{自由落体运动和竖直上抛运动}

\begin{enumerate}[leftmargin=0em]
\renewcommand{\labelenumi}{\arabic{enumi}.}
% A(\Alph) a(\alph) I(\Roman) i(\roman) 1(\arabic)
%设定全局标号series=example	%引用全局变量resume=example
%[topsep=-0.3em,parsep=-0.3em,itemsep=-0.3em,partopsep=-0.3em]
%可使用leftmargin调整列表环境左边的空白长度 [leftmargin=0em]
\item
\exwhere{$ 2019 $年物理全国\lmd{1}卷}
如图,篮球架下的运动员原地垂直起跳扣篮,离地后重心上升的最大高度为$ \frac{H}{4} $。上升第一个所用的时间为$ t_1 $,第四个$ \frac{H}{4} $所用的时间为$ t_2 $。不计空气阻力,则$\frac { t _ { 2 } } { t _ { 1 } }$满足 \xzanswer{C} 
\begin{figure}[h!]
\centering
\includesvg[width=0.16\linewidth]{picture/svg/385}
\end{figure}



\fourchoices
{$1 < \frac { t _ { 2 } } { t _ { 1 } } < 2 \quad$}
{$2 < \frac { t _ { 2 } } { t _ { 1 } } < 3 \quad$}
{$\quad 3 < \frac { t _ { 2 } } { t _ { 1 } } < 4 \quad$}
{$4 < \frac { t _ { 2 } } { t _ { 1 } } < 5$}




\item
\exwhere{$ 2018 $年海南卷}
一攀岩者以$ 1\ m/s $的速度匀速向上攀登,途中碰落了岩壁上的石块,石块自由下落。$ 3 \ s $后攀岩者听到石块落地的声音,此时他离地面的高度约为 \xzanswer{C} 
\fourchoices
{10 m}
{30 m}
{50 m}
{70 m}





\item
\exwhere{$ 2018 $年江苏卷}
从地面竖直向上抛出一只小球,小球运动一段时间后落回地面.忽略空气阻力,该过程中小球的动能$ E_k $与时间$ t $的关系图象是 \xzanswer{A} 
\begin{figure}[h!]
\centering
\includesvg[width=0.83\linewidth]{picture/svg/386}
\end{figure}

\item 
\exwhere{$ 2017 $年浙江选考卷}
拿一个长约$ 1.5\ m $的玻璃筒,一端封闭,另一端有开关,把金属片和小羽毛放到玻璃筒里。把玻璃筒倒立过来,观察它们下落的情况,然后把玻璃筒里的空气抽出,再把玻璃筒倒立过来,再次观察它们下落的情况,下列说法正确的是 \xzanswer{C} 


\begin{minipage}[h!]{0.7\linewidth}
\vspace{0.3em}
\fourchoices
{玻璃筒充满空气时,金属片和小羽毛下落一样快}
{玻璃筒充满空气时,金属片和小羽毛均做自由落体运动}
{玻璃筒抽出空气后,金属片和小羽毛下落一样快}
{玻璃筒抽出空气后,金属片比小羽毛下落快}


\vspace{0.3em}
\end{minipage}
\hfill
\begin{minipage}[h!]{0.3\linewidth}
\flushright
\vspace{0.3em}
\includesvg[width=0.2\linewidth]{picture/svg/387}
\vspace{0.3em}
\end{minipage}


\item 
\exwhere{$ 2014 $年物理上海卷}
在离地高$ h $处,沿竖直方向同时向上和向下抛出两个小球,它们的初速度大小均为$ v $,不计空气阻力,两球落地的时间差为 \xzanswer{A} 
\fourchoices
{$ \frac{2v}{g} $}
{$ \frac{v}{g} $}
{$ \frac{2h}{v} $}
{$ \frac{h}{v} $}





\item
\exwhere{$ 2014 $年物理海南卷}
将一物体以某一初速度竖直上抛。物体在运动过程中受到一大小不变的空气阻力作用,它从抛出点到最高点的运动时间为$ t_{1} $,再从最高点回到抛出点的运动时间为$ t_{2} $,如果没有空气阻力作用,它从抛出点到最高点所用的时间为$ t_{0} $,则 \xzanswer{B} 

\fourchoices
{$ t _ { 1 } > t _ { 0 } \quad t _ { 2 } < t _ { 1 } $}
{$ t _ { 1 } < t _ { 0 } \quad t _ { 2 } > t _ { 1 } $}
{$ t _ { 2 } > t _ { 0 } \quad t _ { 2 } > t _ { 1 } $}
{$ t _ { 1 } < t _ { 0 } \quad t _ { 2 } < t _ { 1 } $}


\item
\exwhere{$ 2012 $年上海卷}
小球每隔$ 0.2 \ s $从同一高度抛出,做初速为$ 6 \ m/s $的竖直上抛运动,设它们在空中不相碰。第$ 1 $个小球在抛出点以上能遇到的小球个数为,$ g $取$ 10 \ m/s ^{2} $ \xzanswer{C} 
\fourchoices
{三个}
{四个}
{五个}
{六个}




\item 
\exwhere{$ 2013 $年安徽卷}
由消防水龙带的喷嘴喷出水的流量是$ 0.28\ m^{3}/min $,水离开喷口时的速度大小为$ 16\sqrt{3} \ m/s $,方向与水平面夹角为$ 60 ^{ \circ } $,在最高处正好到达着火位置,忽略空气阻力,则空中水柱的高度和水量分别是(重力加速度$ g $取$ 10 \ m/s ^{2} $) \xzanswer{A} 


\fourchoices
{$ 28.8\ m \quad 1.12 \times 10^{-2} \ m^{3} $ }
{$ 28.8\ m \quad 0.672 \ m^{3} $}
{$ 38.4\ m \quad 1.29 \times 10^{-2} \ m^{3} $ }
{$ 38.4\ m \quad 0.776 \ m^{3} $}

\item
\exwhere{$ 2015 $年理综山东卷}
距地面高$ 5 \ m $的水平直轨道上$ A $、$ B $两点相距$ 2 \ m $,在$ B $点用细线悬挂一小球,离地高度为$ h $,如图。小车始终以$ 4 \ m/s $的速度沿轨道匀速运动,经过$ A $点时将随车携带的小球由轨道高度自由卸下,小车运动至$ B $点时细线被轧断,最后两球同时落地。不计空气阻力,取重力加速度的大小$ g=10 \ m/s $²。可求得$ h $等于 \xzanswer{A} 
\begin{figure}[h!]
\centering
\includesvg[width=0.23\linewidth]{picture/svg/388}
\end{figure}


\fourchoices
{$ 1.25\ m $}
{$ 2.25\ m $}
{$ 3.75\ m $}
{$ 4.75\ m $}











\end{enumerate}



